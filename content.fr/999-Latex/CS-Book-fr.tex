\documentclass[letterpaper,french,twocolumn,openany]{book}
\usepackage[T1]{fontenc}    % Encodage T1 (adapté au français)
%French-specific commands
%____________________________________________________________________
% https://en.wikibooks.org/wiki/LaTeX/Special_Characters
\usepackage[french]{babel}
\usepackage[autolanguage]{numprint} % for the \nombre command
\usepackage{hyperref}   % for linking references https://en.wikibooks.org/wiki/LaTeX/Hyperlinks
\usepackage{lmodern}        % Caractères plus lisibles
\usepackage{babel}          % Réglages linguistiques (avec french)
\pagestyle{empty}           % N'affiche pas de numéro de page
\usepackage[explicit]{titlesec}  % pour modifier le style de chapitre
\usepackage{color}        % pour les coleurs de police
\usepackage[table,dvipsnames]{xcolor}  % pour les couleurs des tableaux https://en.wikibooks.org/wiki/LaTeX/Colors
\usepackage{graphicx}     % pour inclure des images
\usepackage{wrapfig}
%\usepackage[dvipsnames]{xcolor} % pour les couleurs
\usepackage{fancyhdr} % pour les entete
\usepackage{ifthen}  % branchements conditionnels
\usepackage{xpatch} 
\usepackage[export]{adjustbox}  % pour positionner des images dans la page
\usepackage{tabularx}  % pour les tableaux
\usepackage{wrapfig}   % pour le que le texte ne recouvre pas les figures/tableaux
\usepackage{stfloats}   % pour le que le texte ne recouvre pas les figures/tableaux
\usepackage{multicol}   % gestion de plusieurs colonnes (dans une cellule de tableau pas ex)
\usepackage{enumitem}   % pour la personnalisation des liste
\usepackage{parskip}     % for paragraph indenttion
\usepackage{titletoc}    % for having ToC under each chapter
%____________________________________________________________________
%%  Default Font
\usepackage[sfdefault]{cabin}
\renewcommand*\familydefault{\sfdefault} %% Only if the base font of the document is to be sans serif
\usepackage[T1]{fontenc}

%____________________________________________________________________
% page margins
% https://www.overleaf.com/learn/latex/Page_size_and_margins
% https://fr.overleaf.com/learn/latex/Single_sided_and_double_sided_documents
% https://distrib-coffee.ipsl.jussieu.fr/pub/mirrors/ctan/macros/latex/contrib/geometry/geometry.pdf
\usepackage[
  top=16mm,
  bottom=18mm,
  marginparwidth=30mm,
  marginparsep=2mm,
  outer=50mm,
  inner=20mm,
  headheight=7mm,
  headsep=7mm
  %textwidth=345pt,
]{geometry}
%____________________________________________________________________
\newcommand*{\definitchapitre}[2]{\chapter{#1 \label{#2} \definitentete{#1}}}
%____________________________________________________________________
% commande pour afficher l'image de tête de chapitre
\newcommand*{\afficheimagechapitre}[1]{
    \hspace*{-76pt}
    \ifthenelse{\equal{#1}{1}}{
        \includegraphics[width=\paperwidth]{img/CS-page4-01.png}
    }
    {
        \ifthenelse{\equal{#1}{2}}{
            \includegraphics[width=\paperwidth]{img/CS-page5-01.png}
        }
        {
            \ifthenelse{\equal{#1}{3}}{
                \includegraphics[width=\paperwidth]{img/CS-page7-01.png}
                }
                {
                    \ifthenelse{\equal{#1}{4}}{
                        \includegraphics[width=\paperwidth]{img/CS-page14-01.png}
                        }
                        {
                            \ifthenelse{\equal{#1}{5}}{
                                \includegraphics[width=\paperwidth]{img/CS-page20-01.png}
                                }
                                {
                                    \ifthenelse{\equal{#1}{6}}{
                                        \includegraphics[width=\paperwidth]{img/CS-page34-01.png}
                                        }
                                        {
                                            \ifthenelse{\equal{#1}{7}}{
                                                \includegraphics[width=\paperwidth]{img/CS-page38-01.png}
                                                }
                                                {
                                                    \ifthenelse{\equal{#1}{8}}{
                                                        \includegraphics[width=\paperwidth]{img/CS-page60-01.png}
                                                        }
                                                        {
                                                            \ifthenelse{\equal{#1}{9}}{
                                                                \includegraphics[width=\paperwidth]{img/CS-page95-01.png}
                                                                }
                                                                {
                                                                    \ifthenelse{\equal{#1}{10}}{
                                                                        \includegraphics[height=8.25cm]{img/CS-page201-01.png}
                                                                    }{
                                                                        IMAGE A DEFINIR
                                                                    }
                                                                }
                                                        }
                                                }
                                        }
                                }
                        }
                }
        }
    }
}
%____________________________________________________________________
%  Style pour les en-têtes
\newcommand*{\definitentete}[1]{
    \fancypagestyle{plain}{ %
        \fancyhf{} % remove everything
        \renewcommand{\headrulewidth}{1pt} % remove lines as well
        \renewcommand{\footrulewidth}{1pt}
        \fancyhead[RO]{#1}
        \fancyhead[LE]{CYPHER SYSTEM}
    }
}
%____________________________________________________________________
%  définition des en-têtes
\setlength{\headheight}{15.2pt}
\pagestyle{fancy}
\fancyhead[LE]{CYPHER SYSTEM}
%____________________________________________________________________-
\newcommand*{\smallcslogo}{\includegraphics[height=3mm]{img/CS-mini-logo.png}}
%  définition du style des listes de nom capacités
\newlist{abnamelist}{itemize}{1}
\setlist[abnamelist]{label={\smallcslogo}}
%____________________________________________________________________
\newcommand*{\abilitylistitem}[2]{#1 (\hyperref[#2]{\pageref{#2}})}
%____________________________________________________________________-
%  définition du style des listes de capacités (chapitre 9)
%\newlist{ablist}{itemize}{1}
%\setlist[ablist]{leftmargin=*,label={\includegraphics[height=3mm]{CS-mini-logo.png} :}}
%\setlist[ablist]{label={\includegraphics[height=3mm]{CS-mini-logo.png}}}
\newcommand*{\abilitydetaillistitem}[2]{\item \textbf{#1 :} #2}
%-------------------------------------------------------------------
%  style de paragraphe 
\newcommand*{\chapterfirstletter}[2]{\begin{wrapfigure}{l}{2mm}
    {\Huge\textcolor{#2}{\textbf{#1}}}
\end{wrapfigure}}
%----------------------------------------------------------------
% commande pour la couleur de chaque partie
\newcommand*{\textcolorpartzero}[1]{\textcolor{RedViolet}{#1}}
\newcommand*{\textcolorpartone}[1]{\textcolor{BlueViolet}{#1}}
%\newcommand*{\getcolorpart}[1]{\ifthenelse{\equal{#1}{0}}{RedViolet}{\ifthenelse{\equal{#1}{1}}{BlueViolet}{\ifthenelse{\equal{#1}{2}}{BlueViolet}{}}}}
%\newcommand*{\textcolorpart}[2]{\textcolor{\getcolorpart{#1}}{#2}}
\newcommand*{\getcolorpartone}{BlueViolet}
%  commande pour définir chaque chapitre
\newcommand*{\startchapter}[3]{\chapter{#1}
    \label{#2}
    \fancypagestyle{plain}{ %
    \fancyhf{} % remove everything
    \renewcommand{\headrulewidth}{1pt} % remove lines as well
    \renewcommand{\footrulewidth}{0pt}
    \xpretocmd\headrule{\color{#3}}{}{\PatchFailed}
    \fancyhead[RO]{\textcolor{#3}{\textsc{#1}}}
    \fancyhead[LE]{\textcolor{#3}{CYPHER SYSTEM}}
    \fancyhf[FLE,FRO]{\thepage}
}
\fancyhf{} % remove everything
\renewcommand{\headrulewidth}{1pt} % remove lines as well
\renewcommand{\footrulewidth}{0pt}
\xpretocmd\headrule{\color{#3}}{}{\PatchFailed}
\fancyhead[RO]{\textcolor{#3}{#1}}
\fancyhead[LE]{\textcolor{#3}{CYPHER SYSTEM}}
\fancyhf[FLE,FRO]{\thepage}
}
%#######################################################################
\title{CYPHER SYSTEM LIVRE DES R\`{E}GLES}
\author{Monte Cook, Bruce R. Cordell, Sean K. Reynolds}
\date{March 2019}
%#######################################################################
%     DOCUMENT START
%#######################################################################
\begin{document}
\frontmatter
%-------------------------
%  cover
\begin{titlepage}
    \centering
    \begin{figure*}
        \hspace*{-75pt}
        \vspace*{-75pt}
        \includegraphics*[height=\paperheight]{img/CS-Cover.png}    
    \end{figure*}
\end{titlepage}
%-------------------------
% blank page
%-------------------------
\mainmatter
\tableofcontents
%____________________________________________________________________
%   reset headers
\fancypagestyle{plain}{ %
    \fancyhf{} % remove everything
}
\fancyhf[FLE,FRO]{\thepage}
%#######################################################################
%   PART 0 : The CYPHER SYSTEM
%#######################################################################
\part*{LE CYPHER SYSTEM}
%____________________________________________________________________
% https://distrib-coffee.ipsl.jussieu.fr/pub/mirrors/ctan/macros/latex/contrib/titlesec/titlesec.pdf
% https://borntocode.fr/latex-personnaliser-les-titres-chapter/
%% define format for chapter
\titleformat{\chapter} % command
    [display] % shape
    {\bfseries\Large\centering} % format
    {\textcolor{RedViolet}{Chapitre \ \thechapter \\ \huge \uppercase{#1}} \\ \afficheimagechapitre{\thechapter}} % label
    {0pt} % sep
    {
        \centering
    } % before-code
    [
        \small
    ] % after-code
\titlespacing{\chapter}
    {0mm} %left
    {-0pt} %before-step
    {-10mm} %after-step
    {} %right
%____________________________________________________________________
%% define format for section
% numberless
\titleformat{name=\section,numberless}[runin] 
    {\normalfont\Large\bfseries}
    {\\ \textcolor{RedViolet} {#1}}
    {20pt}
    {\Large}
    [\\]
% paragraphe indentation
\setlength{\parindent}{3mm}
%#######################################################################
    %#######################################################################
    %            CHAPTER 1
    %#######################################################################
\startchapter{Des mondes d'aventures}{ch:chapter1}{RedViolet}
    \chapterfirstletter{F}{RedViolet}inalement, tout ce que nous voulons, c'est précisément de jouer au jeu auquel nous voulons jouer. Toutes les Meneuses et Meneurs ont un cadre de campagne parfait dans un coin de leur tête. Les joueuses et joueurs ont cette idée de personnage qui serait leur meilleur personnage jamais créé, si seulement ils avaient la chance de le créer et de le jouer. Ces rêves de jouer exactement ce que vous souhaitez jouer sont la raison d'être de ce livre.
    Il s'agit d'une version révisée du livre de règles original du Cypher System. J'ai réuni le contenu de ce livre à partir des jeux du Cypher System existant à l'époque \textendash Numenera et The Strange. C'était essentiellement une compilation de tout ce matériel de jeu, ainsi que beaucoup de suggestions sur la façon de l'utiliser de la manière que vous souhaitez. Les joueurs et les Meneuses nous ont dit que cela répondait bien à ces besoins.
    
    Nous avons beaucoup appris depuis. Pas tellement sur les règles du système elles-mêmes \textendash qui restent essentiellement inchangées \textendash mais sur la manière dont nous voulons utiliser ce type de livre, et donc sur la manière de présenter l'information. Il est vraiment difficile de créer quelque chose qui soit utilisable par n'importe qui pour n'importe quoi et de le présenter de manière réellement conviviale. Mais je pense que c'est exactement ce que nous avons fait avec ce livre. Les innovations que vous trouverez dans ces pages \textendash la façon dont toutes les capacités ont été cataloguées pour que vous puissiez les utiliser comme bon vous semble, l'accent mis sur les cyphers subtils, l'étendue des genres présentés \textendash rendent ce matériel plus facile à utiliser et plus facile à personnaliser.
    
    Le tout nouveau contenu, comme le système d'arc narratif, le système d'artisanat, les informations supplémentaires sur les genres, etc., rendra, je l'espère, vos parties plus amusantes et vos histoires plus riches.
    
    Mais permettez-moi de répéter : nous n'avons pas changé la façon dont le jeu fonctionne. Vous pouvez utiliser ce livre en parallèle avec l'ancien livre de règles du Cypher System sans trop de problèmes.
    
    D'une certaine manière, ce livre est un volume complémentaire à un livre que j'ai écrit intitulé Your Best Game Ever. Ce livre est un guide indépendant du système de jeux pour comprendre et apprécier les jeux de rôle. Le présent ouvrage prend les idées et suggestions présentées dans ce dernier et leur donne un ensemble de règles qui les rend possibles. Mon objectif est de vous donner les outils pour avoir votre meilleur jeu jamais joué. Et cela, je crois, implique de pouvoir jouer dans le cadre et avec les personnages que vous avez toujours voulu.
    
    Maintenant, espérons-le, vous pouvez enfin le faire.

    Amusez-bien
    
    \includegraphics[height=20pt]{img/CS-page4-02.png}

    Monte Cook

    Mars 2019
    %#######################################################################
    %            CHAPTER 2
    %#######################################################################
    \chapter{Tout est permis}
\label{ch:chapter2}
    %---------------------------
    % https://texdoc.org/serve/fancyhdr/0
    \fancypagestyle{plain}{ %
        \fancyhf{} % remove everything
        \renewcommand{\headrulewidth}{1pt} % remove lines as well
        \renewcommand{\footrulewidth}{0pt}
        \xpretocmd\headrule{\color{RedViolet}}{}{\PatchFailed}
        \fancyhead[RO]{\textcolor{RedViolet}{textsc{Tout est permis}}}
        \fancyhead[LE]{\textcolor{RedViolet}{CYPHER SYSTEM}}
    }
    \chapterfirstletter{P}{RedViolet}
    our commencer, nous nous adressons directement aux meneuses (ou MJ). Les joueurs comme les MJ utiliseront ce livre, mais il est fort probable que ce soit d'abord le MJ qui le consulte.

    Ce que vous tenez entre vos mains est un guide. Un mode d'emploi. Vous ne pouvez pas simplement vous asseoir et commencer à jouer, car le manuel du Cypher System n'est pas conçu pour être utilisé de cette façon. Il vous faut d'abord y mettre quelque chose de votre propre invention. Il n'y a pas de cadre ni de monde prédéfinis ici. Le système est conçu pour vous aider à dépeindre n'importe quel monde ou cadre que vous pouvez imaginer.

    Considérez ce livre comme un coffre à jouets. Vous pouvez sortir ce que vous voulez et y jouer comme bon vous semble. Vous n'utiliserez pas tout ce qu'il contient, du moins pas d'un seul coup. Vous utiliserez des parties de ce livre pour construire le jeu que vous souhaitez jouer. Sortez quelques éléments et essayez-les. Remettez en place ceux qui ne vous conviennent pas, et essayez d'autres. Utilisez certains maintenant et gardez-en d'autres pour votre prochaine partie. Vous avez toute la liberté possible (en fait, de nombreux mondes).

    A propos de mondes, vous devez décider quel cadre de campagne utiliser, en fonction du genre que vous avez choisi. Cela peut être n'importe quoi. Prenez votre livre ou film préféré, ou concevoir quelque chose à partir de rien.

    Alors, en pratique, ce que vous choisissez ici, c'est l'expérience que vous voulez vivre\textendash et que vous voulez faire vivre aux joueurs. C'est une décision tellement fondamentale que tout le groupe devrait peut-être y participer. Demandez aux autres joueurs quel genre ils aiment et quels types d'expériences ils souhaitent vivre. C'est essentiel, car cela garantit que tout le monde obtient ce qu'il attend du jeu.

    Bien sûr, tout le contenu de ce livre ne convient pas à tous les genres. Vous, en tant que MJ, devrez le lire une fois que vous aurez choisi un genre et sélectionner les types, les axes et ainsi de suite. Ensuite, informez vos joueurs du matériel que vous avez décidé de rendre disponible afin qu'ils puissent créer des personnages adaptés au genre


    \section*{GENRES}
    Jeter un coup d'œil à la 3ième partie Genre qui contient un nombre de chapitres consacrés aux genres. Ce sont des catégories assez larges, et nous les utiliserons dans ce livre comme point de départ. Ces catégories sont : Fantasy, moderne, science-fiction, hoerreur, romance, super-héros, post-apocalyptique, contes de fées, et historique.

    Avec ces descriptions assez génériques, nous pouvons couvrir la plupart des cadres de jeu (mais probablement pas tous)  que vous pouvez jouer avec le Cypher System. Certains de ces genres nécessite du matériel unique, des artifacts, ou des descripteurs. Certains ont besoin de nouvelles règles pour améliorer l'immersion que vous recherchez.

    Nous parlons d'immersion, parce que sous beaucoup d'aspect, c'est ce qu'un genre est. Si vous voulez faire vivre l'expérience d'être terrifié par des zombies qui rodent autour de la maison de votre personnage, vous voulez de l'horreur. Si vous voulez faire vivre l'expérience d'être extrêmement puissant et utiliser ces pouvoirs pour protéger le monde des extra-terrestres, vous voulez des super-héros (avec peut-être une touche de science-fiction).


    \section*{CADRES DE CAMPAGNE}
    Bien que les genres soient des catégories utiles pour organiser vos idées, ce que vous allez réellement créer, c'est un cadre. Des étiquettes comme « science-fiction » ou « space opera » sont pratiques, mais au final, ce qui compte, c'est le cadre spécifique que vous établissez.

    Votre cadre \textendash qu'il s'agisse d'une création originale ou d'une adaptation\textendash vous appartient entièrement. Ne vous inquiétez pas de ce que d'autres pourraient considérer comme approprié pour un genre donné. Une fois que vous commencez à assembler votre cadre, vous voudrez peut-être parcourir à nouveau les sections sur la création de personnages dans ce livre. Ce qui est habituellement adapté à un genre fantastique, par exemple, peut ne pas convenir à votre propre univers de fantasy.

    Imaginons que, dans votre monde, la magie du feu soit toujours maléfique et uniquement pratiquée par des prêtres possédés par des démons. Dans ce cas, l'axe Porte une auréole de feu ne serait pas approprié pour les personnages des joueurs, même s'il est parfaitement acceptable dans d'autres jeux de fantasy.

    Plus vous définissez précisément les détails de votre cadre, plus il sera facile d'en ajuster les éléments. Et plus votre univers s'éloigne des clichés du genre, plus vous devrez adapter les choix possibles. Mais ce n'est pas un problème : les cadres spécifiques et distincts sont souvent les plus amusants, les plus mémorables et les plus engageants pour vos joueurs. Ils valent largement l'effort supplémentaire.


    \section*{ADAPTER LES RÈGLES}
    Parfois, vous devez modifier certaines choses pour qu'elles correspondent à vos besoins et envies. Prenons par exemple la saveur « magie » que vous pouvez attribuer à n'importe quel type présenté dans le chapitre 5. Elle s'appelle « magie » et possède de nombreux éléments associés à ce concept, mais il serait très simple de changer son nom en « psionique », « pouvoirs mutants » ou tout autre terme adapté à votre univers.

    En d'autres termes, sélectionner du contenu dans ce livre peut ne pas suffire. Vous pourriez avoir besoin d'ajuster certains éléments ici et là. Heureusement, la plupart du matériel est conçu pour être modifié ou adapté. En fait, grâce à la simplicité des mécaniques de base du Cypher System, effectuer des ajustements est un jeu d'enfant. Ce n'est pas un système où un petit changement risque d'entraîner un effet domino aux conséquences imprévues.

    Dans le chapitre 7, vous trouverez des directives pour créer de nouveaux descripteurs. Le chapitre 8 contient une section entière consacrée à la création de nouveaux axes adaptés à votre propre jeu. De plus, les types de personnages du chapitre 5 sont conçus pour être personnalisés et remodelés. 

    Lorsque vous apportez des modifications, concentrez-vous moins sur l'équilibrage du jeu et davantage sur la narration des histoires que vous souhaitez raconter, tout en permettant aux joueurs de créer et d'incarner les personnages qu'ils veulent. Si vous parvenez à faire ces deux choses correctement, tout le monde sera satisfait. Et au final, c'est précisément ce qu'est l'équilibrage du jeu. 

    Vous pouvez également consulter le chapitre 25 pour approfondir la façon de modifier les mécaniques du jeu. Mais dans l'ensemble, ce chapitre vous rappellera ce que vous venez de lire : c'est *votre* jeu, et vous êtes libre d'en faire ce que vous voulez.


\include{./tex/Chapter-03.tex}
%#######################################################################
%    PART 1 : CHARACTERS
%#######################################################################
\part*{\uppercase{Personnages}}
%____________________________________________________________________
% https://distrib-coffee.ipsl.jussieu.fr/pub/mirrors/ctan/macros/latex/contrib/titlesec/titlesec.pdf
% https://borntocode.fr/latex-personnaliser-les-titres-chapter/
%% define format for chapter for part 2
\titleformat{\chapter} % command
    [display] % shape
    {\bfseries\Large\centering} % format
    {\textcolor{BlueViolet}{Chapitre \ \thechapter \\ \huge \uppercase{#1}} \\ \afficheimagechapitre{\thechapter}} % label
    {0.5ex} % sep
    {
        \centering
    } % before-code
    [
    \centering
    ] % after-code
    \titlespacing*{\chapter}
    {0mm} %left
    {-0pt} %before-step
    {0mm} %after-step
    {} %right
%____________________________________________________________________
%% define format for section
% \titleformat{\section} % Command to format section titles
% {\normalfont\Large\bfseries} % Format: normal font, large size, bold
% {\thesection}{1em} % Label format: section number followed by 1em space
% {} % Before the title
% [\titlerule] % After the title: horizontal line

% \titleformat{\section} % command
%     [display] % shape
%     {\bfseries\Large} % format
%     {\textcolor{BlueViolet}{#1}} % label
%     {0.5ex} % sep
%     {
%         \Large
%     } % before-code

%     \titlespacing*{\section}
%     {0mm} %left
%     {-0pt} %before-step
%     {0mm} %after-step
%     {} %right
%____________________________________________________________________
% numberless
\titleformat{name=\section,numberless}[runin] 
    {\normalfont\Large\bfseries}
    {\\ \textcolor{BlueViolet} {#1}}
    {20pt}
    {\Large}
    [\\]
%____________________________________________________________________
%% define format for subsection
% numberless
\titleformat{name=\subsection,numberless}[runin] 
    {\normalfont\large\bfseries}
    {\\ \textcolor{BlueViolet} {#1}}
    {20pt}
    {\large}
    [\\]
%#######################################################################
%\MiniToC % table of content for part 1
\include{./tex/Chapter-04.tex}
%#######################################################################
%            CHAPTER 5
%#######################################################################
\startchapter{Type}{ch:chapter5}{\getcolorpartone}
% \chapter{Type}
% \label{ch:chapter5}
% \fancypagestyle{plain}{ %
% \fancyhf{} % remove everything
% \renewcommand{\headrulewidth}{1pt} % remove lines as well
% \renewcommand{\footrulewidth}{0pt}
% \xpretocmd\headrule{\color{\getcolorpartone}}{}{\PatchFailed}
% \fancyhead[RO]{\textcolor{\getcolorpartone}{Type}}
% \fancyhead[LE]{\textcolor{\getcolorpartone}{CYPHER SYSTEM}}
% }
\chapterfirstletter{L}{\getcolorpartone}e Type de personnage est le coeur de votre personnage. Son type vous aide à déterminer sa place dans le monde et les relations avec les autres dans la campagne. C'est le "nom" dans la phrase "Je suis un/e textit{nom+adjectif} qui textit{proposition}"

Vous pouvez choisir parmi quatre types de personnage: Guerrier, Adepte, Explorateur, et Emissaire. Toutefois, vous pourriez ne pas vouloir utiliser ces termes génériques. Ce chapitre vous propose, pour chaque type, quelques alternatives de noms qui pourraient être plus adapté à un genre particulier. Vous trouverez peut-être que des noms comme "Guerrier" ou "Explorateur" ne sonnent pas juste, en particulier dans des campagnes se déroulant à une époque contemporaine. Comme toujours, vous êtes libre de faire comme vous voulez.

Comme le type est la base sur laquelle votre personnage est bati, il est important de considérer comment le type se tient avec la campagne sélectionnée. Pour se faire, les types sont en pratique des archetypes. Un Guerrier, par exemple, pourrait être n'importe qui du chevalier en armure étincellante au policier dans la rue ou au baroudeur cybernétique vétéran de milliers de guerres futuristes.

Pour faciliter l'usage des quatre types dans les diverses campagnes, différentes méthodes appelées "préférences" sont présentées dans le Chapitre 6: Préférence piur aider à personnaliser les différents types pour de la fantasy, de la science fiction, ou d'autres genres (ou pour s'ajuster à différents concepts de personnage).

Au final, des options plus fondamentales de personnalisation sont fournies à la fin de ce chapitre.

\section*{Guerrier}

\begin{description}
    \item[Fantasy/Contes:] guerrier, combattant, escrimeur, chevalier, barbare, soldat, myrmidon, valkyrie
    \item[Moderne/Horreur/Romance:] officier de police, soldat, gardien, détective, vigile, athlète
    \item[Science fiction:] officier de sécurité, guerrier, homme de troupe, soldat, mercenaire
    \item[Superhéro/Post-Apocalyptique:] héro, brique, cogneur
\end{description}

Vous êtes un bon allié à avoir dans un combat. Vous savez comment utiliser des armes et vous défendre.En fonction du genre et de la campagne, cela pourrait signifier de porter une épée et un bouclier dans une arêne de gladiateurs, un AK-47 et des grenades en bandoulière dans la jungle, ou un fusil blaster et une armure mécanique dans l'exploration d'une planète lointaine.

\begin{description}
    \item[Rôle Individuel:] Les Guerriers sont orienté sur le physique et l'action. ils auront plus l'habitude de surmonter un péril en utilisant la force que d'autre moyen, et ils prennent souvent le chemin le plus court vers leur objectif.
    \item[Rôle dans un Groupe:] Les guerriers subissent et infligent généralement le plus de dégâts dans une situation dangereuse. Il leur incombe souvent de protéger les autres membres du groupe contre les menaces. Cela signifie parfois que les guerriers assument également des rôles de commandement, du moins au combat et dans d'autres moments de danger.
    \item[Rôle en société:] Les Guerriers ne sont pas toujours des soldats ou des mercenaires. N'importe qui est est toujours prêt pour la violence, ou même la violence potentielle, peut être un Guerrier dans un sens général. Cela inclut les gardes, les gardiens, les officiers de police, les marins, ou les personnes dans d'autres rôle ou profession qui savent comment se défendre avec talent.
    \item[Guerier Expérimentés:] Alors que les guerriers progressent, leur compétence en combat—que ce soit en se défendant ou en infligeant des dommages—augmente à un rang impressionant. A un rang supérieur, ils peuvent souvent se prendre un groupe d'aversaires tout seul ou affronter sur son terrain n'importe qui.
\end{description}

\begin{table}[t]
    \begin{tabular}{ | p{70mm} | }
        \hline
        \rowcolor{SkyBlue!50}
        \textcolor{\getcolorpartone}{\Large \uppercase{Intrusion de Joueur}} \\
        \rowcolor{SkyBlue!50}
        Une intrusion de joueur est quand un joueur choisi d'altérer quelque chose dans la campagne, rendant les choses plus facile pour le PJ. Dans l'idée, c'est l'inverse d'un intrusion de MJ qui donne au joueur des points d'expérience (XP) en introduisant une complication inatendue pour le personnage. Dans le cas du joueur, ce dernier dépense un 1 XP et présente une solution à un problème ou une complication. Ce que l'intrusion du joueur peut faire est en général, c'est introduire un changement du monde ou des circonstances, plutôt que de changer directement le personnage. Par exemple, une intrusion qui propose que le cypher qui vient d'être utilisé a une charge supplémentaire, est appropriée, mais une intrusion qui propose que le personnage est soigné ne l'est pas. Si le joueur n'a pas d'XP à dépenser, il ne peut pas utiliser d'intrusion de joueur.

        Quelques exemples d'intrusion de joueur sont proposées pour chaque type. Cela dit, toutes les intrusions de joueur listées ici ne sont pas appropriées à chaque situation. Le MJ peut autoriser les joueurs à proposer d'autres suggestions d'intrusion de joueur, mais la Meneuse a le dernier mot pour savoir si une intrusion est appropriée en fonction du type de personnage et de la situation. Si la Meneuse refuse l'intrusion, le joueur ne dépense pas de XP, et l'intrusion n'a simplement pas lieu.

        Utiliser une intrusion ne requiert pas du personnage d'utiliser une action pour l'activer. Une intrusion de joueur survient tout simplement. \\
        \hline
    \end{tabular}
\end{table}

\section*{Intrusions de Joueur pour un Guerrier}

Vous pouvez dépenser 1 XP pour utiliser une des intrusions de joueur suivantes, à condition que la situation est appropriée et que la Meneuse soit d'accord.
\begin{description}
    \item[Position Parfaite:] Vous combattez au moins trois adversaires et chacun d'eux se trouve exactement à la bonne position pour vous pour faire un mouvement pour lequel vous vous êtes entrainé il y a longtemps, vous permettant de les attaquer tous les trois en une seule action. Faites un jet d'attaque pour chaque adversaire. Vous restez limité par la quantité d'Effort que vous pouvez allouer en une seule action.
    \item[Vieil Ami:] Un ancien companion d'arme se présente de manière spontanée et vous fourni de l'aide dans ce que vous êtes en train de faire. Il doit accomplir sa propre mission et ne peut pas rester plus longtemps que pour vous aider, parler un peu après et peut-être partager un repas rapide.
    \item[Arme cassée:] L'arme de votre adversaire a un point faible. Pendant le combat, l'arme est endommagée et descend de deux rangs sur le suivi des dommages des objets.
\end{description}

\section*{Réserves de Stat pour un Guerrier}
\begin{table}[h]
    \begin{tabular}{ l c }
        \textbf{Stat} & \textbf{Valeur de Réserve au Démarrage} \\
        \rowcolor{SkyBlue!50}
        Puissance & 10 \\
        \rowcolor{SkyBlue!20}
        Célérité & 10 \\
        \rowcolor{SkyBlue!50}
        Intellect & 8 \\
    \end{tabular}
\end{table}

Vous avez 6 points supplémentaires à répartir parmi vos Réserves de stat comme vous le souhaitez.

\section*{Guerrier de Premier Rang}

Les Guerriers de Premier Rang ont les capacités suivantes:
\begin{description}
    \item[Effort:] Votre Effort est de 1.
    \item[Physical Nature:] Vous avez un Avantage de Puissance de 1 et un Avantage de Célérité de 0, ou vous avez un Avantage de Puissance de 0 et un Avantage de Célérité de 1. Dans tous les cas, vous avez un Avantage d'Intellect de 0.
    \item[Cypher Use:] Vous pouvez porter deux cyphers en même temps.
    \item[Armes:] Vous avez la pratique des armes légères, moyennes et lourdes et n'avez aucune pénalité quand vous utilisez une arme quelconque.
    \item[Equipment au départ:] Des vêtements appropriés et deux armes de votre choix, ainsi que un objet cher, deux objets modérement chers, et jusqu'à quatre objets peu chers.
    \item[Capacités Spéciales:] Choisissez quatre capacités listées ci-cessous. Vous ne pouvez pas choisir la même capacité plus d'une fois, à moins que sa description dit le contraire. La description complète de chaque capacité listée se trouve dans le chapitre Capacités, qui dispose aussi des descriptions pour les préférences et les capacités de focus en un seul grand catalogue.
\end{description}

\begin{abnamelist}
    \item Avantage de Stat Amélioré (\pageref{subsec:ab_improved_edge})
    \item Choc (\pageref{subsec:ab_bash})
    \item Claque (\pageref{subsec:ab_swipe})
    \item Compétences physiques (\pageref{subsec:ab_physical_skills})
    \item Contrôler le terrain (\pageref{subsec:ab_control_the_field})
    \item Entraîné sans armure (\pageref{subsec:ab_trained_without_armor})
    \item Lancer rapide (\pageref{subsec:ab_quick_throw})
    \item Pas besoin d'armes (\pageref{subsec:ab_no_need_for_weapons})
    \item Pratique des armures (\pageref{subsec:ab_practiced_in_armor})
    \item Prouesses au combat (\pageref{subsec:ab_combat_prowess})
    \item Tir d'Opportunité \hyperref[subsec:ab_overwatch]{\pageref{subsec:ab_overwatch}}
\end{abnamelist}


\section*{ Guerrier de Second Rang}

Choisissez NNN des capacités ci-dessous (ou du rang inférieur) pour l'ajouter à votre répertoire. Vous pouvez en plus remplacer l'une de vos capacités de rang inférieur par une différente d'un rang inférieur.
\begin{abnamelist}
\item Attaque successive (\pageref{subsec:ab_successive_attack})
\item Compétence avec les attaques (\pageref{subsec:ab_skill_with_attacks})
\item Compétence en défense (\pageref{subsec:ab_skill_with_defense})
\item Coup écrasant (\pageref{subsec:ab_crushing_blow})
\item Hémorragie (\pageref{subsec:ab_hemorrhage})
\item Recharger (\pageref{subsec:ab_reload})
\end{abnamelist}

\section*{ Guerrier de Troisième Rang}

Choisissez NNN des capacités ci-dessous (ou du rang inférieur) pour l'ajouter à votre répertoire. Vous pouvez en plus remplacer l'une de vos capacités de rang inférieur par une différente d'un rang inférieur.
\begin{abnamelist}
\item Découpe (\hyperref[subsec:ab_slice]{\pageref{subsec:ab_slice}})
\item Expérimenté en armure (\pageref{subsec:ab_experienced_in_armor})
\item Fureur (\pageref{subsec:ab_fury})
\item Pulvérisation (\pageref{subsec:ab_spray})
\item Réaction (\pageref{subsec:ab_reaction})
\item Résistance énergétique (\pageref{subsec:ab_energy_resistance})
\item Saisissez l'instant (\pageref{subsec:ab_seize_the_moment})
\item Se Fendre (\pageref{subsec:ab_lunge})
\item Tir Double (\pageref{subsec:ab_trick_shot})
\item Utilisation experte des cyphers (\pageref{subsec:ab_expert_cypher_use})
\item Vigilance (\pageref{subsec:ab_vigilance})
\item Visée mortelle (\pageref{subsec:ab_deadly_aim})
\end{abnamelist}

\begin{table*}[b]
    \hspace*{-40pt}
    \centering
    \begin{tabular}{ c p{18cm} }
        \multicolumn{2}{ l }{\Large \textcolor{\getcolorpartone}{Relation avec l'histoire personnelle d'un Guerrier}} \\
        \multicolumn{2}{ l }{Votre type vous aide à déterminer la relation que vous avez avec le cadre de campagne. Lancez un d20 ou choisissez dans la liste ci-après pour déterminer un fait bien particulier à propos de votre histoire personnel qui fourni une relation avec le reste du monde. Vous pouvez aussi créer votre propre relation.} \\
        \textbf{d20} & \textbf{Histoire personnelle} \\ [0.5ex]
         \rowcolor{SkyBlue!50}
         1 & Vous étiez dans l'armée et vous y avez toujours des amis. Votre ancien commandant se souvient bien de vous. \\
         \rowcolor{SkyBlue!20}
        2 & Vous étiez le garde du corps d'unefemme riche qui vous a accusé de vol. Vous avez quitté son service en disgrâce.  \\
        \rowcolor{SkyBlue!50}
        3 & Vous étiez le videur d'un bar du coin pendant un temps, et les habitués se souviennent de vous. \\
        \rowcolor{SkyBlue!20}
        4 & Vous vous êtes entrainé avec un maître reconnu. Il vous respecte mais il a beaucoup d'ennemis. \\
        \rowcolor{SkyBlue!50}
        5 & Vous vous êtes entrainé dans un monastère isolé. Les moines sont toujours vos frêres mais vous êtes un étranger pourtout les autres. \\
        \rowcolor{SkyBlue!20}
        6 & Vous n'avez pas été vraiment entrainé. Vos compétences viennent naturellement (ou de manière surnatuelle). \\
        \rowcolor{SkyBlue!50}
        7 & Vous avez passé du temps dans les rues et avez été en prison pendant un moment. \\
        \rowcolor{SkyBlue!20}
        8 & Vous avez été réquisitionné dans une armée, mais vous avez déserté rapidement. \\
        \rowcolor{SkyBlue!50}
        9 & Vous avez servi de garde du coprs à un puissant criminel qui vous doit sa vie. \\
        \rowcolor{SkyBlue!20}
        10 & Vous avez travaillé comme officier de police ou comme une sorte de gendarme. Tout le monde vous connait, mais les opinions qu'il ont de vous peuvent varier. \\
        \rowcolor{SkyBlue!50}
        11 & Votre grand frêre ou grande soeur est un personnage tristement célèbre qui a été disgracié. \\
        \rowcolor{SkyBlue!20}
        12 & Vous avez servi comme garde pour quelqu'un qui a beaucoup voyagé. Vous connaissez beaucoup de monde un peu partout. \\
        \rowcolor{SkyBlue!50}
        13 & Votre meilleur ami est un enseignant ou un savant. C'est une bonne source d'information. \\
        \rowcolor{SkyBlue!20}
        14 & Vous et un ami fumez tout les deux la même sorte de tabac rare et cher. Vous vous réunissez au moins une fois par semaine pour parler un peu et fumer. \\
        \rowcolor{SkyBlue!50}
        15 & Votre oncle dirige un théatre en ville. Vous connaissez tous les acteurs et pouvez regarder les spectacles gratuitement. \\
        \rowcolor{SkyBlue!20}
        16 & Votre ami artisan peut quelque fois vous demander de l'aide. Toutefois il vous paie correctement. \\
        \rowcolor{SkyBlue!50}
        17 & Votre maître a écrit un livre sur les arts martiaux. De temps en temps, des personnes vous cherche pour vous demander des éclaircissement sur certains passages un peu étranges. \\
        \rowcolor{SkyBlue!20}
        18 & Une personne avec qui vous avez combattu dans l'armée est maintenant le maire d'une ville voisine. \\
        \rowcolor{SkyBlue!50}
        19 & Vous avez sauvé la vie d'une famille alors que leur maison était en flamme. Elle a une dette envers vous et les voisins voient en vous un héro. \\
        \rowcolor{SkyBlue!20}
        20 & Votre ancien entraineur attend toujours de vous que vous revenniez nettoyer après les cours;quand vous le faites, il partage avec vous de temps en temps des rumeurs interressantes. \\
    \end{tabular}
\end{table*}

\section*{ Guerrier de Quatrième Rang}

Choisissez NNN des capacités ci-dessous (ou du rang inférieur) pour l'ajouter à votre répertoire. Vous pouvez en plus remplacer l'une de vos capacités de rang inférieur par une différente d'un rang inférieur.
\begin{abnamelist}
\item Dur comme du Bois (\pageref{subsec:ab_tough_as_nails})
\item Défenseur expérimenté (\pageref{subsec:ab_experienced_defender})
\item Effets accrus (\pageref{subsec:ab_increased_effects})
\item Effort incroyable (\pageref{subsec:ab_amazing_effort})
\item Feinte (\pageref{subsec:ab_feint})
\item Guerrier Capable (\pageref{subsec:ab_capable_warrior})
\item Momentum (\pageref{subsec:ab_momentum})
\item Percer les Défenses (\pageref{subsec:ab_pry_open})
\item Tir Précis (\pageref{subsec:ab_snipe})
\end{abnamelist}

\section*{ Guerrier de Cinquième Rang}

Choisissez NNN des capacités ci-dessous (ou du rang inférieur) pour l'ajouter à votre répertoire. Vous pouvez en plus remplacer l'une de vos capacités de rang inférieur par une différente d'un rang inférieur.
\begin{abnamelist}
\item Attaque sautée (\pageref{subsec:ab_jump_attack})
\item Maîtrise de la défense (\pageref{subsec:ab_mastery_with_defense})
\item Maîtrise des attaques (\pageref{subsec:ab_mastery_with_attacks})
\item Maîtrise en Armure (\pageref{subsec:ab_mastery_in_armor})
\item Parade (\pageref{subsec:ab_parry})
\item Succès amélioré (\pageref{subsec:ab_improved_success})
\item Tirs en éventail (\pageref{subsec:ab_arc_spray})
\item Utilisation adroite des cyphers (\pageref{subsec:ab_adroit_cypher_use})
\end{abnamelist}

\section*{ Guerrier de Sixième Rang}

Choisissez NNN des capacités ci-dessous (ou du rang inférieur) pour l'ajouter à votre répertoire. Vous pouvez en plus remplacer l'une de vos capacités de rang inférieur par une différente d'un rang inférieur.
\begin{abnamelist}
\item Arme et corps (\pageref{subsec:ab_weapon_and_body})
\item Attaque Tournoyante (\pageref{subsec:ab_spin_attack})
\item Coup final (\pageref{subsec:ab_finishing_blow})
\item Encore et encore (\pageref{subsec:ab_again_and_again})
\item Meurtrier (\pageref{subsec:ab_murderer})
\item Moment magnifique (\pageref{subsec:ab_magnificent_moment})
\end{abnamelist}

\section{ Adepte }
\begin{description}
\item[Fantasy/Contes:]
\item[Moderne/Horreur/Romance:]
\item[Science fiction:]
\item[Superhéro/Post-Apocalyptique:]
\end{description}


\begin{description}
\item[Rôle Individuel:]
\item[Rôle dans un Groupe:]
\item[Rôle en société:]
\item[Adepte Expérimenté:]
\end{description}


\section*{Intrusions de Joueur pour un Adepte }

Vous pouvez dépenser 1 XP pour utiliser un des intrusions de joueur suivantes, à condition que la situation est appropriée et que la Meneuse soit d'accord.
\begin{description}
\item[Intrusion1:]
\item[Intrusion2:]
\item[Intrusion3:]
\end{description}

\section*{Réserves de Stat pour un Adepte }
\begin{table}[h]
    \begin{tabular}{ l c }
        \textbf{stat} & \textbf{Valeurs de Réserve au Démarrage} \\
        \rowcolor{SkyBlue!50}
        Puissance & 7 \\
        \rowcolor{SkyBlue!50}
        Célérité & 9 \\
        \rowcolor{SkyBlue!50}
        Intellect & 12 \\
    \end{tabular}
\end{table}

Vous avez 6 points supplémentaires à répartir parmi vos Réserves de stat comme vous le souhaitez.

\section*{ Adepte de Premier Rang}

Les Adeptes de Premier Rang ont les Capacités suivantes:
\begin{description}
\item[Effort:] Votre Effort est de 1.
\item[Avantage:] Intellect=1
\item[Utilisation de Cypher:] Vous pouvez porter 3 cuphers en même temps.
\item[Equpement au départ:]
\item[Capacités spéciales:]
\end{description}

\begin{abnamelist}
\item Assaut Magique (\pageref{subsec:ab_onslaught})
\item Briser (\pageref{subsec:ab_shatter})
\item Champ de résonance (\pageref{subsec:ab_resonance_field})
\item Coup écrasant (\pageref{subsec:ab_crushing_blow})
\item Distorsion (\pageref{subsec:ab_distortion})
\item Effacer les souvenirs (\pageref{subsec:ab_erase_memories})
\item Formation magique (\pageref{subsec:ab_magic_training})
\item Grand Pas (\pageref{subsec:ab_far_step})
\item Magie Prosaïque (\pageref{subsec:ab_hedge_magic})
\item Poussée (\pageref{subsec:ab_push})
\item Protection (\pageref{subsec:ab_ward})
\end{abnamelist}

\section*{ Adepte de Second Rang}

Choisissez NNN des capacités ci-dessous (ou du rang inférieur) pour l'ajouter à votre répertoire. Vous pouvez en plus remplacer l'une de vos capacités de rang inférieur par une différente d'un rang inférieur.
\begin{abnamelist}
\item Adaptation (\pageref{subsec:ab_adaptation})
\item Lecture mentale (\pageref{subsec:ab_mind_reading})
\item Lumière coupante (\pageref{subsec:ab_cutting_light})
\item Récupérer des souvenirs (\pageref{subsec:ab_retrieve_memories})
\item Révèle (\pageref{subsec:ab_reveal})
\item Stase (\pageref{subsec:ab_stasis})
\item Survol (\pageref{subsec:ab_hover})
\end{abnamelist}

\section*{ Adepte de Troisième Rang}

Choisissez NNN des capacités ci-dessous (ou du rang inférieur) pour l'ajouter à votre répertoire. Vous pouvez en plus remplacer l'une de vos capacités de rang inférieur par une différente d'un rang inférieur.
\begin{abnamelist}
\item Barrière de champ de force (\pageref{subsec:ab_force_field_barrier})
\item Capteur (\pageref{subsec:ab_sensor})
\item Contre-mesures (\pageref{subsec:ab_countermeasures})
\item Feu et Glace (\pageref{subsec:ab_fire_and_ice})
\item Oeil pour Cibler (\pageref{subsec:ab_targeting_eye})
\item Protection énergétique (\pageref{subsec:ab_energy_protection})
\item Utilisation adroite des cyphers (\pageref{subsec:ab_adroit_cypher_use})
\end{abnamelist}

\section*{ Adepte de Quatrième Rang}

Choisissez NNN des capacités ci-dessous (ou du rang inférieur) pour l'ajouter à votre répertoire. Vous pouvez en plus remplacer l'une de vos capacités de rang inférieur par une différente d'un rang inférieur.
\begin{abnamelist}
\item Contrôle mental (\pageref{subsec:ab_mind_control})
\item Exil (\pageref{subsec:ab_exile})
\item Invisibilité (\pageref{subsec:ab_invisibility})
\item Nuage de matière (\pageref{subsec:ab_matter_cloud})
\item Projection (\pageref{subsec:ab_projection})
\item Remodeler (\pageref{subsec:ab_reshape})
\item Régénérer (\pageref{subsec:ab_regenerate})
\item Toucher Mortel (\pageref{subsec:ab_death_touch})
\item Traitement rapide (\pageref{subsec:ab_rapid_processing})
\item Trou de ver (\pageref{subsec:ab_wormhole})
\end{abnamelist}

\section*{ Adepte de Cinquième Rang}

Choisissez NNN des capacités ci-dessous (ou du rang inférieur) pour l'ajouter à votre répertoire. Vous pouvez en plus remplacer l'une de vos capacités de rang inférieur par une différente d'un rang inférieur.
\begin{abnamelist}
\item Absorber l'énergie (\pageref{subsec:ab_absorb_energy})
\item Concussion (\pageref{subsec:ab_concussion})
\item Conjuration (\pageref{subsec:ab_conjuration})
\item Connaître l'inconnu (\pageref{subsec:ab_knowing_the_unknown})
\item Créer (\pageref{subsec:ab_create})
\item Poussière Retourne à la Poussière (\pageref{subsec:ab_dust_to_dust})
\item Téléportation (\pageref{subsec:ab_teleportation})
\item Véritables sens (\pageref{subsec:ab_true_senses})
\end{abnamelist}

\section*{ Adepte de Sixième Rang}

Choisissez NNN des capacités ci-dessous (ou du rang inférieur) pour l'ajouter à votre répertoire. Vous pouvez en plus remplacer l'une de vos capacités de rang inférieur par une différente d'un rang inférieur.
\begin{abnamelist}
\item Contrôle de la météo (\pageref{subsec:ab_control_weather})
\item Déplacer des montagnes (\pageref{subsec:ab_move_mountains})
\item Traversez les mondes (\pageref{subsec:ab_traverse_the_worlds})
\item Tremblement de terre (\pageref{subsec:ab_earthquake})
\item Usurpation de Cypher (\pageref{subsec:ab_usurp_cypher})
\end{abnamelist}

\section{ Explorateur }
\begin{description}
\item[Fantasy/Contes:]
\item[Moderne/Horreur/Romance:]
\item[Science fiction:]
\item[Superhéro/Post-Apocalyptique:]
\end{description}


\begin{description}
\item[Rôle Individuel:]
\item[Rôle dans un Groupe:]
\item[Rôle en société:]
\item[Explorateur Expérimenté:]
\end{description}


\section*{Intrusions de Joueur pour un Explorateur }

Vous pouvez dépenser 1 XP pour utiliser un des intrusions de joueur suivantes, à condition que la situation est appropriée et que la Meneuse soit d'accord.
\begin{description}
\item[Intrusion1:]
\item[Intrusion2:]
\item[Intrusion3:]
\end{description}

\section*{Réserves de Stat pour un Explorateur }
\begin{table}[h]
\begin{tabular}{ l c }
\textbf{stat} & \textbf{Valeurs de Réserve au Démarrage} \\
\rowcolor{SkyBlue!50}
Puissance & 10 \\
\rowcolor{SkyBlue!50}
Célérité & 9 \\
\rowcolor{SkyBlue!50}
Intellect & 9 \\
\end{tabular}
\end{table}

Vous avez 6 points supplémentaires à répartir parmi vos Réserves de stat comme vous le souhaitez.

\section*{ Explorateur de Premier Rang}

Les Explorateurs de Premier Rang ont les Capacités suivantes:
\begin{description}
\item[Effort:] Votre Effort est de 1.
\item[Avantage:] Puissance=1
\item[Utilisation de Cypher:] Vous pouvez porter 2 cuphers en même temps.
\item[Equpement au départ:]
\item[Capacités spéciales:]
\end{description}

\begin{abnamelist}
\item Avantage de Stat Amélioré (\pageref{subsec:ab_improved_edge})
\item Bloquer (\pageref{subsec:ab_block})
\item Compétences en Connaissances (\pageref{subsec:ab_knowledge_skills})
\item Compétences physiques (\pageref{subsec:ab_physical_skills})
\item Déchiffrer (\pageref{subsec:ab_decipher})
\item Endurance (\pageref{subsec:ab_endurance})
\item Entraîné sans armure (\pageref{subsec:ab_trained_without_armor})
\item Muscles de fer (\pageref{subsec:ab_muscles_of_iron})
\item Pas besoin d'armes (\pageref{subsec:ab_no_need_for_weapons})
\item Pied Léger (\pageref{subsec:ab_fleet_of_foot})
\item Pratique de toutes les armes (\pageref{subsec:ab_practiced_with_all_weapons})
\item Pratique des armures (\pageref{subsec:ab_practiced_in_armor})
\item Sens du Danger (\pageref{subsec:ab_danger_sense})
\item Sursaut de confiance (\pageref{subsec:ab_surging_confidence})
\item Trouver le chemin (\pageref{subsec:ab_find_the_way})
\end{abnamelist}

\section*{ Explorateur de Second Rang}

Choisissez NNN des capacités ci-dessous (ou du rang inférieur) pour l'ajouter à votre répertoire. Vous pouvez en plus remplacer l'une de vos capacités de rang inférieur par une différente d'un rang inférieur.
\begin{abnamelist}
\item Activer les autres (\pageref{subsec:ab_enable_others})
\item Augmentation de la portée (\pageref{subsec:ab_range_increase})
\item Compétence en défense (\pageref{subsec:ab_skill_with_defense})
\item Compétences d'enquête (\pageref{subsec:ab_investigative_skills})
\item Compétences de voyage (\pageref{subsec:ab_travel_skills})
\item Coordination Main-Oeil (\pageref{subsec:ab_hand_to_eye})
\item Curieux (\pageref{subsec:ab_curious})
\item Déjouer le Danger (\pageref{subsec:ab_foil_danger})
\item Evasion (\pageref{subsec:ab_escape})
\item Instinct de Danger (\pageref{subsec:ab_danger_instinct})
\item Oeil pour les détails (\pageref{subsec:ab_eye_for_detail})
\item Rester en Alerte (\pageref{subsec:ab_stand_watch})
\item Récupération rapide (\pageref{subsec:ab_rapid_recovery})
\end{abnamelist}

\section*{ Explorateur de Troisième Rang}

Choisissez NNN des capacités ci-dessous (ou du rang inférieur) pour l'ajouter à votre répertoire. Vous pouvez en plus remplacer l'une de vos capacités de rang inférieur par une différente d'un rang inférieur.
\begin{abnamelist}
\item Briseur de Pierre (\pageref{subsec:ab_stone_breaker})
\item Chute contrôlée (\pageref{subsec:ab_controlled_fall})
\item Compétence avec les attaques (\pageref{subsec:ab_skill_with_attacks})
\item Courir et combattre (\pageref{subsec:ab_run_and_fight})
\item Course d'obstacles (\pageref{subsec:ab_obstacle_running})
\item Expérimenté en armure (\pageref{subsec:ab_experienced_in_armor})
\item Extraire du hasard (\pageref{subsec:ab_wrest_from_chance})
\item Ignorez la Douleur (\pageref{subsec:ab_ignore_the_pain})
\item Pensez à votre sortie (\pageref{subsec:ab_think_your_way_out})
\item Résilience (\pageref{subsec:ab_resilience})
\item Saisissez l'instant (\pageref{subsec:ab_seize_the_moment})
\item Trouver les Pièges (\pageref{subsec:ab_trapfinder})
\item Utilisation experte des cyphers (\pageref{subsec:ab_expert_cypher_use})
\end{abnamelist}

\section*{ Explorateur de Quatrième Rang}

Choisissez NNN des capacités ci-dessous (ou du rang inférieur) pour l'ajouter à votre répertoire. Vous pouvez en plus remplacer l'une de vos capacités de rang inférieur par une différente d'un rang inférieur.
\begin{abnamelist}
\item Compétence d'expert (\pageref{subsec:ab_expert_skill})
\item Coureur (\pageref{subsec:ab_runner})
\item Dur comme du Bois (\pageref{subsec:ab_tough_as_nails})
\item Effets accrus (\pageref{subsec:ab_increased_effects})
\item Guerrier Capable (\pageref{subsec:ab_capable_warrior})
\item Lire les signes (\pageref{subsec:ab_read_the_signs})
\item Pas subtiles (\pageref{subsec:ab_subtle_steps})
\end{abnamelist}

\section*{ Explorateur de Cinquième Rang}

Choisissez NNN des capacités ci-dessous (ou du rang inférieur) pour l'ajouter à votre répertoire. Vous pouvez en plus remplacer l'une de vos capacités de rang inférieur par une différente d'un rang inférieur.
\begin{abnamelist}
\item Amitié de groupe (\pageref{subsec:ab_group_friendship})
\item Attaque sautée (\pageref{subsec:ab_jump_attack})
\item Difficile à tuer (\pageref{subsec:ab_hard_to_kill})
\item Libre de se déplacer (\pageref{subsec:ab_free_to_move})
\item Maîtrise de la défense (\pageref{subsec:ab_mastery_with_defense})
\item Parade (\pageref{subsec:ab_parry})
\item Physiquement doué (\pageref{subsec:ab_physically_gifted})
\item Prendre le commandement (\pageref{subsec:ab_take_command})
\item Utilisation adroite des cyphers (\pageref{subsec:ab_adroit_cypher_use})
\item Vigilant (\pageref{subsec:ab_vigilant})
\end{abnamelist}

\section*{ Explorateur de Sixième Rang}

Choisissez NNN des capacités ci-dessous (ou du rang inférieur) pour l'ajouter à votre répertoire. Vous pouvez en plus remplacer l'une de vos capacités de rang inférieur par une différente d'un rang inférieur.
\begin{abnamelist}
\item Annuler le danger (\pageref{subsec:ab_negate_danger})
\item Attaque Tournoyante (\pageref{subsec:ab_spin_attack})
\item Encore et encore (\pageref{subsec:ab_again_and_again})
\item Inspire des actions coordonnées (\pageref{subsec:ab_inspire_coordinated_actions})
\item Maîtrise des attaques (\pageref{subsec:ab_mastery_with_attacks})
\item Maîtrise en Armure (\pageref{subsec:ab_mastery_in_armor})
\item Partager la défense (\pageref{subsec:ab_share_defense})
\item Vitalité sauvage (\pageref{subsec:ab_wild_vitality})
\end{abnamelist}

\section{ Emissaire }
\begin{description}
\item[Fantasy/Contes:]
\item[Moderne/Horreur/Romance:]
\item[Science fiction:]
\item[Superhéro/Post-Apocalyptique:]
\end{description}


\begin{description}
\item[Rôle Individuel:]
\item[Rôle dans un Groupe:]
\item[Rôle en société:]
\item[Emissaire Expérimenté:]
\end{description}


\section*{Intrusions de Joueur pour un Emissaire }

Vous pouvez dépenser 1 XP pour utiliser un des intrusions de joueur suivantes, à condition que la situation est appropriée et que la Meneuse soit d'accord.
\begin{description}
\item[Intrusion1:]
\item[Intrusion2:]
\item[Intrusion3:]
\end{description}

\section*{Réserves de Stat pour un Emissaire }
\begin{table}[h]
\begin{tabular}{ l c }
\textbf{stat} & \textbf{Valeurs de Réserve au Démarrage} \\
\rowcolor{SkyBlue!50}
Puissance & 8 \\
\rowcolor{SkyBlue!50}
Célérité & 9 \\
\rowcolor{SkyBlue!50}
Intellect & 11 \\
\end{tabular}
\end{table}

Vous avez 6 points supplémentaires à répartir parmi vos Réserves de stat comme vous le souhaitez.

\section*{ Emissaire de Premier Rang}

Les Emissaires de Premier Rang ont les Capacités suivantes:
\begin{description}
\item[Effort:] Votre Effort est de 1.
\item[Avantage:] Intellect=1
\item[Utilisation de Cypher:] Vous pouvez porter 2 cuphers en même temps.
\item[Equpement au départ:]
\item[Capacités spéciales:]
\end{description}

\begin{abnamelist}
\item Anecdote (\pageref{subsec:ab_anecdote})
\item Attitude de commandement (\pageref{subsec:ab_demeanor_of_command})
\item Babel (\pageref{subsec:ab_babel})
\item Changement d'Identité (\pageref{subsec:ab_spin_identity})
\item Compréhension (\pageref{subsec:ab_understanding})
\item Compétences d'interaction (\pageref{subsec:ab_interaction_skills})
\item Effacer les souvenirs (\pageref{subsec:ab_erase_memories})
\item Embrouiller (\pageref{subsec:ab_fast_talk})
\item Encouragement (\pageref{subsec:ab_encouragement})
\item Envoûtement (\pageref{subsec:ab_enthrall})
\item Inspire l'Agression (\pageref{subsec:ab_inspire_aggression})
\item Pratique des armes moyennes (\pageref{subsec:ab_practiced_with_medium_weapons})
\item Présence terrifiante (\pageref{subsec:ab_terrifying_presence})
\end{abnamelist}

\section*{ Emissaire de Second Rang}

Choisissez NNN des capacités ci-dessous (ou du rang inférieur) pour l'ajouter à votre répertoire. Vous pouvez en plus remplacer l'une de vos capacités de rang inférieur par une différente d'un rang inférieur.
\begin{abnamelist}
\item Compétence en défense (\pageref{subsec:ab_skill_with_defense})
\item Compétences d'interaction (\pageref{subsec:ab_interaction_skills})
\item Démotiver (\pageref{subsec:ab_disincentivize})
\item Facilité Inspirante (\pageref{subsec:ab_inspiring_ease})
\item Pratique des armures (\pageref{subsec:ab_practiced_in_armor})
\item Recueillir des renseignements (\pageref{subsec:ab_gather_intelligence})
\item Récupération Rapide d'un autre (\pageref{subsec:ab_speedy_recovery})
\item Suivant de base (\pageref{subsec:ab_basic_follower})
\item Trahison inattendue (\pageref{subsec:ab_unexpected_betrayal})
\item Transmettre un idéal (\pageref{subsec:ab_impart_ideal})
\item Calmer un Etranger (\pageref{subsec:ab_calm_stranger})
\end{abnamelist}

\section*{ Emissaire de Troisième Rang}

Choisissez NNN des capacités ci-dessous (ou du rang inférieur) pour l'ajouter à votre répertoire. Vous pouvez en plus remplacer l'une de vos capacités de rang inférieur par une différente d'un rang inférieur.
\begin{abnamelist}
\item Accélérer (\pageref{subsec:ab_up_to_speed})
\item Dire les Choses (\pageref{subsec:ab_telling})
\item Disciple expert (\pageref{subsec:ab_expert_follower})
\item Esprit perspicace (\pageref{subsec:ab_discerning_mind})
\item Grande Déception (\pageref{subsec:ab_grand_deception})
\item Lecture mentale (\pageref{subsec:ab_mind_reading})
\item Mené par l'enquête (\pageref{subsec:ab_lead_by_inquiry})
\item Parfait Inconnu (\pageref{subsec:ab_perfect_stranger})
\item Se fondre dans le décor (\pageref{subsec:ab_blend_in})
\item Talent Oratoire (\pageref{subsec:ab_oratory})
\item Utilisation experte des cyphers (\pageref{subsec:ab_expert_cypher_use})
\item Vif d'esprit (\pageref{subsec:ab_quick_wits})
\end{abnamelist}

\section*{ Emissaire de Quatrième Rang}

Choisissez NNN des capacités ci-dessous (ou du rang inférieur) pour l'ajouter à votre répertoire. Vous pouvez en plus remplacer l'une de vos capacités de rang inférieur par une différente d'un rang inférieur.
\begin{abnamelist}
\item Anticipation de l'attaque (\pageref{subsec:ab_anticipate_attack})
\item Compétences accrues (\pageref{subsec:ab_heightened_skills})
\item Elaborer une stratégie (\pageref{subsec:ab_strategize})
\item Feinte (\pageref{subsec:ab_feint})
\item Lire les signes (\pageref{subsec:ab_read_the_signs})
\item Moqueries confondantes (\pageref{subsec:ab_confounding_banter})
\item Psychose (\pageref{subsec:ab_psychosis})
\item Stimuler l'effort (\pageref{subsec:ab_spur_effort})
\item Suggestion (\pageref{subsec:ab_suggestion})
\end{abnamelist}

\section*{ Emissaire de Cinquième Rang}

Choisissez NNN des capacités ci-dessous (ou du rang inférieur) pour l'ajouter à votre répertoire. Vous pouvez en plus remplacer l'une de vos capacités de rang inférieur par une différente d'un rang inférieur.
\begin{abnamelist}
\item Aura fétide (\pageref{subsec:ab_foul_aura})
\item Compétence avec les attaques (\pageref{subsec:ab_skill_with_attacks})
\item Connaître l'inconnu (\pageref{subsec:ab_knowing_the_unknown})
\item Discipline de vigilance (\pageref{subsec:ab_discipline_of_watchfulness})
\item Expérimenté en armure (\pageref{subsec:ab_experienced_in_armor})
\item Fuir (\pageref{subsec:ab_flee})
\item Régénérer (\pageref{subsec:ab_regenerate})
\item Stimuler (\pageref{subsec:ab_stimulate})
\item Utilisation adroite des cyphers (\pageref{subsec:ab_adroit_cypher_use})
\end{abnamelist}

\section*{ Emissaire de Sixième Rang}

Choisissez NNN des capacités ci-dessous (ou du rang inférieur) pour l'ajouter à votre répertoire. Vous pouvez en plus remplacer l'une de vos capacités de rang inférieur par une différente d'un rang inférieur.
\begin{abnamelist}
\item Assumer le contrôle (\pageref{subsec:ab_assume_control})
\item Brise Esprit (\pageref{subsec:ab_shatter_mind})
\item Contrôle des foules (\pageref{subsec:ab_crowd_control})
\item Gestion de bataille (\pageref{subsec:ab_battle_management})
\item Mot de commandement (\pageref{subsec:ab_word_of_command})
\item Recruter un adjoint (\pageref{subsec:ab_recruit_deputy})
\item Succès inspirant (\pageref{subsec:ab_inspiring_success})
\item Véritables sens (\pageref{subsec:ab_true_senses})
\end{abnamelist}


%#######################################################################
%            CHAPTER 6
%#######################################################################
\startchapter{Préférence}{ch:chapter6}{\getcolorpartone}
\raggedright
\chapterfirstletter{L}{\getcolorpartone}es préférences sont des groupes de capacités spéciales que la Meneuse et les joueurs peuvent utiliser pour modifier le type de personnage pour mieux l'affiner vers l'image qu'ils en ont ou pour qu'il soit mieux adapté au genre ou au cadre de campagne. Par exemple, si une joueuse veut créer un personnage voleur utilisant la magie, elle pourrait jouer une Adepte avec une préférence pour la furtivité. Dans un cadre de campagne de science fiction, un Guerrier pourrait avoir des connaissances en machinerie, donc le personnage pourrait avoir une préférence pour la technologie.

Lors de l'avancement du personnage vers un nouveau rang, les capacités d'une préférence peuvent être échangées avec les capacités de type (une pour une). Ainsi, pour ajouter la capacité de préférence de furtivité Sens du Danger à un Guerrier, quelque-chose d'autre—peut-être Choc—doit être sacrifiée. Maintenant, le personnage peut choisir Sens du Danger comme csi c'était une capacité de Guerrier de premier rang, mais il ne pourrait plus jamais choisir Choc.

La Meneuse doit toujours être impliquée pour ajouter une préférence à un type. Par exemple, elle pourrait savoir que pour son jeu de science fiction, elle souhaite un type appelé "Glam", qui est un Emissaire avec une préférence pour les capacités de technologie—en particulier celles qui font du personnage un flamboyant pilote de vaisseaux spacial. Ainsi, elle échange les capacités de permier rang Changement d'Identité et Inspire l'Agression pour les capacités de préférence de technologie Datajack et Compétences techniques afin que le personnage puisse se connecter directement au vaisseau et puisse prendre les compétences pilotage et ordinateurs.

Au final, la préférence est un outil pour la Meneuse pour créer facilement des types specifiques pour des campagnes en créant de légères modifications des quatre types de base. Bien que les joueurs puisse souhaiter utiliser les préférences pour avoir le personnage qu'ils souhaitent, souvenez-vous qu'ils peuvent aussi très bien ajuster leurs personnages avec les descripteurs et foci.

La description complète de cheque capacité listée ci-dessous peut être trouvée dans le Chapitre 9: Capacités, qui contient aussi les descriptions pour les capacités de type et de focus.

\section*{Préférence de Combat}

La Préférence de combat rend un personnage plus martial. Un Emissaire avec une préférence de combat dans une campagne de fantasy serait un barde de combat. Un Explorateur avec Préférence de combat dans un jeu historique pourrait être un pirate. Un Adepte avec une préférence de combat dans un décor de science-fiction pourrait être un vétéran de mille guerres psychiques.

Capacités de Combat de rang 1


    Pratique des armes moyennes (171)
    Pratique des armures (171)
    Sens du Danger (124)

Capacités de Combat de rang 2

(Cypher System Rulebook, page 36)

    Entraîné sans armure (193)
    Prouesses au combat (120)
    Soif de sang (115)

Capacités de Combat de rang 3

(Cypher System Rulebook, page 36)

    Attaque successive (187)
    Compétence avec les attaques (183)
    Compétence en défense (183)
    Pratique de toutes les armes (171)

Capacités de Combat de rang 4

(Cypher System Rulebook, page 37)

    Attaque successive (187)
    Compétence avec les attaques (183)
    Compétence en défense (183)
    Pratique de toutes les armes (171)

Capacités de Combat de rang 5

(Cypher System Rulebook, page 37)

    Cible difficile (148)
    Défenseur expérimenté (136)
    Parade (168)

Capacités de Combat de rang 6

(Cypher System Rulebook, page 37)

    Compétence en Attaque Supérieure (147)
    Maîtrise de la défense (161)
    Maîtrise en Armure (161)

\section*{Préférence de Magie}

Vous avez quelques connaissances en magie. Vous pourriez e pas être u sorcier, mais vous en connaissez les bases—comment ça marche, et comment accomplir quelques actes extarordinaires. Bien sûr, dans votre cadre de campagne, la "magie" peut en fait être des pouvoirs psychiques, des capacités de mutant, de la technologie extra-terrestre, ou n'importe quoi d'autre qui produit des effets intéressants et utiles.

Un Explorateur avec une préférence pour la magie peut être un chasseur de mage, et un Emissaire avec une préférence pour la magie pourrait être un barde-sorcier. Bien qu'un Adepte avec une préférence pour la magie est toujours un Adepte, mais vous pouvez trouver qu'échanger certaines des capacités de base du type avec celles données ici peuvent mieux orienter le personnage comme le souhaitez.

\section*{Préférence de Compétences et Connaissances}

Cette préférence est destinée aux personnages occupant des rôles qui nécessitent plus de connaissances et une plus grande application des talents dans le monde réel. C'est moins tape-à-l'œil et dramatique que les pouvoirs surnaturels ou la capacité de diviser plusieurs ennemis, mais parfois l'expertise oule savoir-faire est la véritable solution à un problème.
Un guerrier doté de compétences et de connaissances pourrait être un ingénieur militaire. Un explorateur doté de compétences et de connaissances pourrait être un scientifique de terrain. Un orateur avec cette Préférence pourrait être un enseignant.

Un Guerrier doté de compétences et de connaissances pourrait être un ingénieur militaire. Un Explorateur doté de compétences et de connaissances pourrait être un scientifique de terrain. Un Emissaire avec cette Préférence pourrait être un enseignant.

\section*{Préférence de Furtivité}

Les personnages avec la Préférence de Furtivité sont doués pour la discrétion, l'infiltration dans les endroits protégés, et la tromperie. Ils utilisent ces capacités de façon variée, dont le combat. Un Explorateur avec un préférence en furtivité pourrait être un voleur, tandis qu'un Guerrier avec préférence de furtivité serait un assassin. Un Explorateur avec un préférence en furtivité dans une campagne de super-héros pourrait être un justicier qui rôde dans les rues la nuit.

\section*{Préférence de Technologie}

Les personnages ayant une Préférence de technologie sont généralement issus d'un cadre de campagne de science-fiction ou du moins des temps modernes (même si tout est possible). Ils excellent dans l'utilisation, la manipulation et la construction de machines. Un explorateur doté d'une Préférence technologique pourrait être un pilote de vaisseau spatial, et un orateur doté d'une Préférence technologique pourrait être un techno-prêtre. Certaines des capacités les moins orientées vers l'ordinateur pourraient convenir à un personnage steampunk, tandis qu'un personnage moderne pourrait utiliser certaines des capacités qui n'impliquent pas de vaisseaux spatiaux ou d'ultratechnologie.


%#######################################################################
%            CHAPTER 7
%#######################################################################
\startchapter{Descripteur}{ch:chapter7}{CSCOLORPARTONE}
\raggedright
\chapterfirstletter{v}{CSCOLORPARTONE}otre descripteur définit votre personnage, il lui donne son caractère. Les différences entre un Explorateur Charmeur et un Explorateur Vicieux sont considérables. Le descripteur change la façon de chaque pesonnage d'accomplir chaque action. Votre descripteur place votre personnage dans la situation (la première aventure, qui démarre la campagne) et contribue à le motiver. C'est l'adjectif de la phrase « Je suis un *adjectif nom* qui *verbes* ».

Les descripteurs offrent un ensemble unique de capacités, de compétences ou de modifications supplémentaires à vos Réserves de statistiques. Toutes les propositions d'un descripteur ne sont pas des modifications positives du personnage. Par exemple, certains descripteurs ont des incapacités --- des tâches pour lesquelles un personnage n'est pas doué. Vous pouvez considérer les incapacités comme des compétences négatives : au lieu d'être un peu meilleur dans ce genre de tâche, vous êtes un peu pire. Si vous devenez compétent dans une tâche pour laquelle vous êtes incapable, elles s'annulent. N'oubliez pas que les personnages sont définis autant par ce pour quoi ils ne sont *pas* bons que par ce pour quoi ils *sont* bons.

Les descripteurs proposent également quelques brèves suggestions sur la manière dont votre personnage s'est impliqué avec le reste du groupe lors de sa première aventure. Vous pouvez les utiliser ou non, comme vous le souhaitez.

Cette section détaille cinquante descripteurs. Choisissez-en un pour votre personnage. Vous pouvez choisir le descripteur de votre choix, quel que soit votre type. À la fin de ce chapitre, quelques options sont proposées pour personnaliser les descripteurs, notamment faire de l'espèce d'un personnage leur descripteur.

Votre descripteur compte le plus lorsque vous êtes un personnage débutant. Les avantages (et peut-être les inconvénients) découlant de votre descripteur finiront par être éclipsés par l'importance croissante de votre type et de votre orientation. Cependant, l'influence de votre descripteur restera au moins quelque peu importante tout au long de la vie de votre personnage.

%%%%%%%%%%%%%%%%%%%%%%%%%%%%%%%%%%%%%%%%%%%%%%%%%%%%%%%%%%%%%%%%%%%%%
%      Descriptor details
%--------------------------

%--------------------------
\label{sec:descappealing}\section*{Attirant}

\textcolor{gray}{\emph{Appealing}}

Vous êtes attirant aux yeux des autres, mais peut-être plus important encore, vous êtes sympathique et charismatique. Vous avez ce « quelque chose de spécial » qui attire les autres vers vous. Vous savez souvent ce qu'il faut dire pour faire rire quelqu'un, le mettre à l'aise ou l'inciter à agir. Les gens comme vous veulent vous aider et veulent être votre ami.

\textbf{Charismatique:} +2 à votre Réserve d'Intellect.

\textbf{Compétence:} Vous êtes entraîné dans les relations sociales agréables.

\textbf{Resistant aux Charmes:} Vous êtes conscient de la façon dont les autres peuvent manipuler et charmer les gens, et vous remarquez lorsque ces tactiques sont utilisées contre vous. Grâce à cette prise de conscience, vous êtes entraîné à résister à toute forme de persuasion ou de séduction si vous le souhaitez.

\textbf{Lien initial à la Première Aventure:} Choisissez parmi la liste des options ci-dessous comment vous vous êtes retrouvé impliqué dans la première aventure:

1. Vous avez rencontré un parfait inconnu (l'un des autres PJ) et vous l'avez tellement charmé qu'il vous a invité.

2. Les PJ cherchaient quelqu'un d'autre, mais vous les avez convaincus que vous étiez parfait.

3. Un pur hasard --- parce que vous suivez le flux des choses et que tout se passe généralement bien.

4. Votre charisme a permis à l'un des PJ de se sortir d'une situation difficile il y a longtemps, et il vous demande toujours de le rejoindre dans de nouvelles aventures.


%--------------------------
\label{sec:descsharpeyed}\section*{Au Regard-perçant}

\textcolor{gray}{\emph{Sharp-Eyed}}

Vous êtes perspicace et bien conscient de votre environnement. Vous remarquez les petits détails et vous vous en souvenez. Vous pouvez être difficile à surprendre.

\textbf{Compétence:} Vous êtes entraîné aux actions d'Initiative.

\textbf{Compétence:} Vous êtes entraîné aux actions de perception.

\textbf{Trouve la faille:} Si un adversaire a une faiblesse évidente (subit des dégâts supplémentaires dus au feu, ne peut pas voir de son oeil gauche, etc.), le MJ vous dira de quoi il s'agit.

\textbf{Lien initial à la Première Aventure:} Choisissez parmi la liste des options ci-dessous comment vous vous êtes retrouvé impliqué dans la première aventure:

1. Vous avez entendu parler de ce qui se passait, avez remarqué une faille dans le plan des autres PJ et vous vous êtes joint à eux pour les aider.

2. Vous avez remarqué que les PJ ont un ennemi (ou au moins une queue) dont ils n'avaient pas conscience.

3. Vous avez vu que les autres PJ préparaient quelque chose d'intéressant et vous vous êtes impliqués.

4. Vous avez remarqué des choses étranges, et tout cela semble lié.


%--------------------------
\label{sec:descbeneficent}\section*{Bienfaisant}

\textcolor{gray}{\emph{Beneficent}}

Aider les autres est votre vocation. C'est pour ça que vous êtes là. Les autres se réjouissent de votre nature extravertie et charitable, et vous vous réjouissez de leur bonheur. Vous faites de votre mieux lorsque vous aidez les gens, soit en leur expliquant comment ils peuvent surmonter au mieux un défi, *soit en leur démontrant comment le faire par eux-même (translation to be confirmed)*.

\textbf{Généreux:} Les alliés qui ont passé la dernière journée avec vous ajoute +1 à leur jet de Récupération.

\textbf{Altruiste:} Si vous vous tenez à côté d'une créature qui subit des dégâts, vous pouvez intercéder et subir vous-même 1 point de ces dégâts (réduisant les dégâts infligés à la créature de 1 point). Si vous possédez une Armure, cela ne vous apporte aucun avantage lorsque vous utilisez cette capacité.

\textbf{Compétence:} Vous êtes entraîné à toutes les tâches liées aux intéractions sociales agréables, pour mettre les autres à l'aise et gagner leur confiance.

\textbf{Serviable:} Chaque fois que vous aidez un autre personnage, ce personnage en bénéficie comme si vous aviez été entraîné, même si vous n'êtes pas entraîné ou spécialisé dans la tâche tentée.

\textbf{Inaptitude:} Quand vous êtes seul, toutes les tâches d'Intellect et de Célérité sont entravées.

\textbf{Lien initial à la Première Aventure:} Choisissez parmi la liste des options ci-dessous comment vous vous êtes retrouvé impliqué dans la première aventure:

1. Même si vous ne connaissiez pas auparavant la plupart des autres PJ, vous vous êtes invité à leur quête.

2. Vous avez vu les PJ lutter pour surmonter un problème et vous les avez rejoints de manière désintéressée pour vous aider.

3. Vous êtes presque certain que les PJs échoueront sans vous.

4. Le choix était entre votre vie en lambeaux et aider les autres. Depuis, vous n'avez pas regardé en arrière.


%--------------------------
\label{sec:descstrongwilled}\section*{Borné}

\textcolor{gray}{\emph{Strong-Willed}}

Vous êtes un peu borné, volontaire et indépendant. Personne ne peut vous convaincre ou changer d'avis si vous ne souhaitez pas que cela change. Cette qualité ne vous rend pas nécessairement intelligent, mais elle fait de vous un bastion de volonté et de détermination. Vous vous habillez et agissez probablement avec style unique et tendance, sans vous soucier de ce que pensent les autres.

\textbf{Volontaire:} +4 à votre Réserve d'Intellect.

\textbf{Compétence:} Vous êtes entraîné à résister aux effets mentaux.

\textbf{Compétence:} Vous êtes entraîné dans des tâches nécessitant une concentration ou une concentration incroyable.

\textbf{Inaptitude:} Volontaire ne veut pas dire brillant. Toute tâche qui implique de résoudre des énigmes ou des problèmes, de mémoriser des choses ou d'utiliser des connaissances est entravée.

\textbf{Lien initial à la Première Aventure:} Choisissez parmi la liste des options ci-dessous comment vous vous êtes retrouvé impliqué dans la première aventure:

1. Contre votre bon jugement, vous avez rejoint les autres PJ parce que vous avez vu qu'ils étaient en danger.

2. L'un des autres PJ vous a convaincu que rejoindre le groupe serait dans votre intérêt.

3. Vous avez peur de ce qui pourrait arriver si les autres PC tombaient en panne.

4. Il y a une récompense en jeu et vous avez besoin d'argent.


%--------------------------
\label{sec:desccalm}\section*{Calme}

\textcolor{gray}{\emph{Calm}}

Vous avez passé la majeure partie de votre vie à des activités sédentaires (livres, films, passe-temps, etc.) plutôt qu'à des activités actives. Vous connaissez bien toutes sortes de domaines universitaires ou autres activités intellectuelles, mais rien de physique. Vous n'êtes pas nécessairement faible ou affaibli (bien que ce soit un bon descripteur pour les personnages âgés), mais vous n'avez aucune expérience dans les activités plus physiques.Calme est un excellent descripteur pour les personnages qui n'ont jamais eu l'intention de vivre des aventures mais qui y ont été plongés, un trope qui se produit souvent dans les jeux modernes et en particulier dans les jeux d'horreur.

\textbf{Adore les Livres:} +2 à votre Réserve d'Intellect.

\textbf{Compétences:} Vous êtes entraîné à quatre compétences non-physiques de votre choix.

\textbf{Anecdote:} Vous pouvez trouver un fait aléatoire pertinent à la situation actuelle lorsque vous le souhaitez. Il s'agit toujours d'un fait, et non d'une conjecture ou d'une supposition, et doit être quelque chose que vous auriez pu logiquement lire ou voir dans le passé. Vous pouvez le faire une seule fois, bien que la capacité soit renouvelée à chaque fois que vous effectuez un jet de récupération.

\textbf{Inaptitude:} Non seulement vous n'êtes pas un Guerrier, mais toutes les attaques physiques sont désavantagées.

\textbf{Inaptitude:} Vous n'êtes pas vraiment quelqu'un à l'aise dans la nature. Toutes les tâches pour grimper, courir, sauter et nager sont désavantagées.

\textbf{Lien initial à la Première Aventure:} Choisissez parmi la liste des options ci-dessous comment vous vous êtes retrouvé impliqué dans la première aventure:

1. Vous avez lu quelque part la situation actuelle et avez décidé de la vérifier par vous-même.

2. Vous étiez au bon (mauvais ?) endroit au bon (mauvais ?) moment.

3. Tout en évitant une situation totalement différente, vous êtes entré dans votre situation actuelle.

4. L'un des autres PJ vous a entraîné dans cette aventure.


%--------------------------
\label{sec:descrisktaking}\section*{Casse-cou}

\textcolor{gray}{\emph{Risk-Taking}}

Cela fait partie de votre nature de remettre en question ce que les autres pensent ne pas pouvoir ou ne devrait pas être fait. Vous n'êtes pas fou, bien sûr – vous n'essaieriez pas de franchir un gouffre d'un kilomètre de large simplement parce que vous avez le courage de le faire. Il y a l'impossible et puis il y a le tout juste possible. Vous aimez pousser ces derniers plus loin que les autres, car cela vous procure un élan de satisfaction et de plaisir lorsque vous réussissez. Plus vous réussissez, plus vous vous retrouvez à la recherche du prochain défi risqué contre lequel vous essayer.

\textbf{Agile:} +4 à votre Réserve de Célérité.

\textbf{Compétence:} Vous savez exploiter le risque et vous êtes entraîné à des tâches qui impliquent un élément de hasard, comme jouer à des jeux ou choisir entre deux ou trois options apparemment égales.

\textbf{Tenter la Chance:} Vous pouvez choisir de réussir automatiquement une tâche sans lancer de dés, à condition que la difficulté de la tâche ne dépasse pas 6. Cependant, lorsque vous le faites, vous déclenchez également une intrusion de MJ comme si vous aviez obtenu un 1. L'intrusion n'invalide pas la réussite, mais cela la modifie probablement d'une manière ou d'une autre. Vous pouvez le faire une seule fois, bien que cette capacité se renouvelle à chaque fois que vous effectuez un jet de récupération de dix heures.

\textbf{Inaptitude:} Vous êtes peut-être agile, mais vous n'êtes pas sournois. Les tâches liées à la furtivité et au silence sont entravées.

\textbf{Lien initial à la Première Aventure:} Choisissez parmi la liste des options ci-dessous comment vous vous êtes retrouvé impliqué dans la première aventure:

1. Il semblait y avoir des chances égales que les autres PJ ne réussissent pas, ce qui vous semblait bien.

2. Vous pensez que les tâches qui vous attendent vous présenteront des défis uniques et enrichissants.

3. L'un de vos plus grands risques n'a pas été réalisé et vous avez besoin d'argent pour vous aider à payer cette dette.

4. Vous vous êtes vanté de n'avoir jamais vu de risque que vous n'aimiez pas, c'est ainsi que vous êtes arrivé à votre point actuel.


%--------------------------
\label{sec:desclucky}\section*{Chanceux}

\textcolor{gray}{\emph{Lucky}}

Vous comptez sur le hasard et la chance opportune pour vous sortir de nombreuses situations. Quand les gens disent que quelqu'un est né sous une bonne étoile, ils parlent de vous. Lorsque vous vous essayez à quelque chose de nouveau, aussi peu familier que soit la tâche, vous trouvez le plus souvent une mesure de succès. Même lorsqu'une catastrophe survient, elle est rarement aussi grave qu'elle pourrait l'être. Le plus souvent, les petites choses semblent se dérouler comme vous le souhaitez, vous gagnez des concours et vous êtes souvent au bon endroit au bon moment.

\textbf{Réserve de Chance:} Vous disposez d'une Réserve supplémentaire appelé Chance qui commence par 3 points et a une valeur maximale de 3 points. Lorsque vous dépensez des points d'une autre Réserve, vous pouvez d'abord prendre un, quelques-uns ou tous les points de votre Réserve de chance. Lorsque vous effectuez un jet de récupération pour récupérer des points dans n'importe quelle autre Réserve, votre Réserve de chance est également rafraîchie du même nombre de points. Lorsque votre Réserve de chance est à 0 point, elle ne compte pas dans votre jauge de dégâts. Facilitateur.

\textbf{Avantage:} Quand vous utilisez 1 XP pour relancer un d20 pour n'importe quel jet dont l'effet n'affecte que vous, ajoutez 3 au résultat au nouveau jet de dé.

\textbf{Lien initial à la Première Aventure:} Choisissez parmi la liste des options ci-dessous comment vous vous êtes retrouvé impliqué dans la première aventure:

1. Sachant que les gens chanceux remarquent et profitent activement des opportunités, vous vous êtes impliqué dans votre première aventure par choix.

2. Vous êtes vraiment rentré dans quelqu'un d'autre au cours de cette aventure par pure chance.

3. Vous avez trouvé une mallette au bord de la route. Il était en mauvais état, mais à l'intérieur vous avez trouvé de nombreux documents étranges qui vous ont conduit jusqu'ici.

4. Votre chance vous a sauvé lorsque vous avez évité un véhicule roulant à grande vitesse par une chute fortuite à travers une ouverture dans le sol (un trou d'égout, si dans un cadre moderne). Sous terre, vous avez trouvé quelque chose que vous ne pouviez ignorer.


%--------------------------
\label{sec:descchaotic}\section*{Chaotique}

\textcolor{gray}{\emph{Chaotic}}

Le danger ne signifie pas grand-chose pour vous, principalement parce que vous ne pensez pas beaucoup aux répercussions. En fait, vous aimez créer des surprises, rien que pour voir ce qui va se passer. Plus le résultat est inattendu, plus vous êtes heureux. Parfois, vous êtes particulièrement maniaque et, pour le bien de vos compagnons, vous vous empêchez de prendre des mesures dont vous savez qu'elles mèneront au désastre.

\textbf{Tumultueux:} +4 à votre Réserve de Célérité.

\textbf{Compétence:} Vous êtes entraîné aux actions de Défense d'Intellect.

\textbf{Chaotique:} Une fois après chaque jet de récupération de dix heures, si le premier résultat ne vous plaît pas, vous pouvez relancer un jet de dé de votre choix. Si vous le faites, et quel que soit le résultat, le MJ vous présente une intrusion du MJ.

\textbf{Inaptitude:} Votre corps est un peu usé par les excès occasionnels. Les tâches de défense de Puissance sont désavantagées.

\textbf{Lien initial à la Première Aventure:} Choisissez parmi la liste des options ci-dessous comment vous vous êtes retrouvé impliqué dans la première aventure:

1. Un autre PJ vous a recruté alors que vous aviez un bon comportement, avant de réaliser à quel point vous étiez chaotique.

2. Vous avez des raisons de croire qu'être avec les autres PJ vous aidera à prendre le contrôle de votre comportement erratique.

3. Un autre PJ vous a libéré de captivité, et pour le remercier, vous vous êtes porté volontaire pour l'aider.

4. Vous n'avez aucune idée de la façon dont vous avez rejoint les PJ. Vous continuez simplement avec cela pour le moment jusqu'à ce que les réponses se présentent.


%--------------------------
\label{sec:desccharming}\section*{Charmant}

\textcolor{gray}{\emph{Charming}}

Vous êtes un beau parleur et un charmeur. Que ce soit par des moyens apparemment surnaturels ou simplement par des mots, vous pouvez convaincre les autres de faire ce que vous souhaitez. Très probablement, vous êtes physiquement attirant ou du moins très charismatique, et les autres aiment écouter votre voix. Vous faites probablement attention à votre apparence et restez bien soigné. Vous vous faites facilement des amis. *Vous jouez sur la facette de la personnalité de votre statistique Intellect ; l'intelligence n'est pas votre point fort (translation to be confirmed)*. Vous êtes sympathique, mais pas nécessairement studieux ou volontaire.

\textbf{Personable:} +2 à votre Réserve d'Intellect.

\textbf{Compétence:} Vous êtes entraîné dans toutes les tâches d'intéractions sociales positives ou plaisantes.

\textbf{Compétence:} Vous êtes entraîné quand vous uitliser une capacité spéciale qui influence l'esprit des autres.

\textbf{Contact:} Vous avez un contact important qui occupe une position influente, comme un noble mineur, le capitaine de la garde/police de la ville ou le chef d'une grande bande de voleurs. Vous et le MJ devriez régler les détails ensemble.

\textbf{Inaptitude:} Vous n'avez jamais été doué pour étudier ou retenir les faits. Toute tâche impliquant des connaissances, des connaissances ou une compréhension est désavantagée.

\textbf{Inaptitude:} Votre volonté n'est pas votre point fort. Les actions de défense pour résister à des attaques mentales sont désavantagées.

\textbf{Equipement Supplémentaire:} Vous avez réussi à obtenir des réductions et des bonus décents ces dernières semaines. En conséquence, vous disposez de suffisamment d'argent en poche pour acheter un article à prix modéré.

\textbf{Lien initial à la Première Aventure:} Choisissez parmi la liste des options ci-dessous comment vous vous êtes retrouvé impliqué dans la première aventure:

1. Vous avez convaincu l'un des autres PJ de vous dire ce qu'il faisait.

2. Vous avez tout organisé et convaincu les autres de vous rejoindre.

3. L'un des autres PJ vous a rendu service, et maintenant vous remboursez cette obligation en l'aidant dans la tâche à accomplir.

4. Il y a une récompense en jeu et vous avez besoin d'argent.


%--------------------------
\label{sec:desccruel}\section*{Cruel}

\textcolor{gray}{\emph{Cruel}}

Le malheur et la souffrance ne vous touchent pas. Lorsqu'une autre personne endure des difficultés, vous avez du mal à vous en soucier, et vous pouvez même apprécier la douleur et les difficultés que cette personne éprouve si elle vous a fait du mal dans le passé. Votre côté cruel peut provenir de l'amertume provoquée par vos propres luttes et déceptions. Vous pourriez être un pragmatique acharné, faisant ce que vous estimez devoir faire même si les autres sont pires à cause de cela. Ou vous pourriez être un sadique, se réjouissant de la douleur que vous infligez.Être cruel ne fait pas nécessairement de vous un méchant. Votre cruauté peut être réservée à ceux qui vous contrarient ou à d'autres personnes qui vous sont utiles. Vous êtes peut-être devenu cruel à la suite d'une expérience extrêmement horrible. Les abus et la torture, par exemple, peuvent priver la personne de toute compassion pour les autres êtres vivants.De plus, vous n'avez pas besoin d'être cruel dans toutes les situations. En fait, les autres pourraient vous considérer comme aimable, amical et même serviable. Mais lorsque vous êtes en colère ou frustré, votre double nature se révèle, et ceux qui ont mérité votre mépris risquent d'en souffrir.

\textbf{Calculateur:} +2 à votre Réserve d'Intellect.

\textbf{Cruauté:} Lorsque vous utilisez la force, vous pouvez choisir de mutiler ou de lui infliger des blessures douloureuses pour prolonger la souffrance de votre ennemi. Chaque fois que vous infligez des dégâts, vous pouvez choisir d'infliger 2 points de dégâts de moins pour faciliter votre prochaine attaque contre cet ennemi.

\textbf{Compétence:} Vous êtes entraîné dans les tâches en relation avec in tasks lié à la tromperie, à l'intimidation et à la persuasion lorsque vous interagissez avec des personnages éprouvant une douleur physique ou émotionnelle.

\textbf{Inaptitude:} Vous avez du mal à vous connecter avec les autres, à comprendre leurs Focus ou à partager leurs sentiments. Toute tâche visant à déterminer les Focus, les sentiments ou les dispositions d'un autre personnage est désavantagée.

\textbf{Equipement Supplémentaire:} Vous possédez un souvenir précieux de la dernière personne que vous avez détruite. Le prix du souvenir est modéré et vous pouvez le vendre ou l'échanger contre un objet de valeur égale ou moindre.

\textbf{Lien initial à la Première Aventure:} Choisissez parmi la liste des options ci-dessous comment vous vous êtes retrouvé impliqué dans la première aventure:

1. Vous pensez que vous pourriez obtenir un avantage à long terme en aidant les autres PJ et que vous pourrez peut-être utiliser cet avantage contre vos ennemis.

2. En rejoignant les PJ, vous voyez une opportunité d'accroître votre pouvoir et votre statut personnels aux dépens des autres.

3. Vous espérez rendre la vie d'un autre PJ plus difficile en rejoignant le groupe.

4. Rejoindre les PJ vous donne l'opportunité d'échapper à la justice pour un crime que vous avez commis.


%--------------------------
\label{sec:desccreative}\section*{Créatif}

\textcolor{gray}{\emph{Creative}}

Peut-être avez-vous un cahier dans lequel vous notez vos idées afin de pouvoir les développer plus tard. Peut-être vous envoyez-vous par courrier électronique des idées qui vous frappent à l'improviste afin de pouvoir les trier dans un document électronique. Ou peut-être que vous vous asseyez simplement, regardez votre écran et, par une force de volonté incroyable, produisez quelque chose à partir de rien. Quelle que soit la manière dont votre don fonctionne, vous êtes créatif : vous codez, écrivez, composez, sculptez, concevez, dirigez ou créez de toute autre manière des récits qui captivent les autres avec votre vision.

\textbf{Inventive:} +2 à votre Réserve d'Intellect.

\textbf{Original:} Vous proposez toujours quelque chose de nouveau. Vous êtes entraîné dans toute tâche liée à la création d'un récit (comme une histoire, une pièce de théâtre ou un scénario). Cela inclut la tromperie, si la tromperie implique un récit que vous êtes capable de raconter.

\textbf{Compétence:} Vous êtes naturellement inventif. Vous êtes entraîné à une compétence créative spécifique de votre choix : écriture, codage informatique, composition musicale, peinture, dessin, etc.

\textbf{Compétence:} Vous aimez résoudre des énigmes, etc. Vous êtes entraîné aux tâches de résolution d'énigmes.

\textbf{Compétence:} Pour être créatif, il faut toujours apprendre. Vous êtes entraîné à toute tâche qui implique la découverte de quelque chose de nouveau, par exemple lorsque vous fouillez dans une bibliothèque, une banque de données, des archives d'actualités ou une collection de connaissances similaire.

\textbf{Inaptitude:} Vous êtes inventif mais pas charmant. Toutes les tâches liées à une interaction sociale agréable sont désavantagées.

\textbf{Lien initial à la Première Aventure:} Choisissez parmi la liste des options ci-dessous comment vous vous êtes retrouvé impliqué dans la première aventure:

1. Vous faisiez des recherches pour un projet et avez convaincu les PJ de vous accompagner.

2. Vous recherchez de nouveaux marchés pour les résultats de votre production créative.

3. Vous êtes tombé sur la mauvaise clientèle, mais elle a grandi en vous.

4. Une vie créative est souvent confrontée à des obstacles financiers. Vous avez rejoint les PJ parce que vous espériez que cela serait rentable.


%--------------------------
\label{sec:descinquisitive}\section*{Curieux}

\textcolor{gray}{\emph{Inquisitive}}

Le monde est vaste et mystérieux, avec des merveilles et des secrets qui vous surprendront pendant plusieurs vies. Vous ressentez un tiraillement dans votre cœur, un appel à explorer les ruines des civilisations passées, à découvrir de nouveaux peuples, de nouveaux lieux et toutes les merveilles bizarres que vous pourriez trouver en cours de route. Cependant, même si vous ressentez le besoin de parcourir le monde, vous savez qu'il existe de nombreux dangers et vous prenez des précautions pour vous assurer que vous êtes prêt à toute éventualité. La recherche, la préparation et la préparation vous aideront à vivre assez longtemps pour voir tout ce que vous voulez voir et faire tout ce que vous voulez faire.Vous avez probablement à tout moment sur vous une douzaine de livres et de récits de voyage sur le monde. Lorsque vous ne prenez pas la route et ne regardez pas autour de vous, vous passez votre temps le nez dans un livre, apprenant tout ce que vous pouvez sur l'endroit où vous allez afin de savoir à quoi vous attendre une fois sur place.

\textbf{Intelligent:} +4 à votre Réserve d'Intellect.

\textbf{Compétence:} Vous avez envie d'apprendre. Vous êtes entraîné à toute tâche qui implique d'apprendre quelque chose de nouveau, que vous parliez à un local pour obtenir des informations ou que vous fouilliez dans de vieux livres pour découvrir des traditions.

\textbf{Compétence:} Vous avez fait une étude du monde. Vous êtes entraîné à toute tâche impliquant la géographie ou l'histoire.

\textbf{Inaptitude:} Vous avez tendance à vous concentrer sur les détails, ce qui vous rend quelque peu inconscient de ce qui se passe autour de vous. Toute tâche visant à entendre ou à remarquer les dangers autour de vous est entravée.

\textbf{Inaptitude:} Lorsque vous voyez quelque chose d'intéressant, vous hésitez en prenant en compte tous les détails. Les actions d'initiative (pour déterminer qui commence le combat en premier) sont désavantagées.

\textbf{Equipement Supplémentaire:} Vous disposez de trois livres sur les sujets que vous choisissez.

\textbf{Lien initial à la Première Aventure:} Choisissez parmi la liste des options ci-dessous comment vous vous êtes retrouvé impliqué dans la première aventure:

1. L'un des PJ vous a approché pour obtenir des informations relatives à la mission, après avoir entendu dire que vous étiez un expert.

2. Vous avez toujours voulu voir l'endroit où vont les autres PJ.

3. Vous étiez intéressé par ce que faisaient les autres PJ et avez décidé de les suivre.

4. L'un des PJ vous fascine, peut-être en raison d'une capacité spéciale ou étrange dont il dispose.


%--------------------------
\label{sec:descmechanical}\section*{Doué pour la mécanique}

\textcolor{gray}{\emph{Mechanical}}

Vous avez un talent particulier avec les machines de toutes sortes, et vous savez les comprendre et, le cas échéant, les réparer. Peut-être êtes-vous un peu un inventeur, créant de nouvelles machines de temps en temps. Vous êtes appelé "technophile", "tech", "mech", "gear-head", "motor-head" ou l'un des nombreux autres surnoms. Les mécaniciens portent généralement des vêtements de travail pratiques et transportent de nombreux outils.

\textbf{Intelligent:} +2 à votre Réserve d'Intellect.

\textbf{Compétence:} Vous êtes entraîné dans toutes les actions impliquant l'identification ou la compréhension des machines.

\textbf{Compétence:} Vous êtes entraîné dans toutes les actions impliquant l'utilisation, la réparation ou la fabrication de machines.

\textbf{Equipement Supplémentaire:} Vous commencez avec une variété de machines-outils.

\textbf{Lien initial à la Première Aventure:} Choisissez parmi la liste des options ci-dessous comment vous vous êtes retrouvé impliqué dans la première aventure:

1. Alors que vous répariez une machine à proximité, vous avez entendu les autres PJ parler.

2. Vous avez besoin d'argent pour acheter des outils et des pièces.

3. Il était clair que la mission ne pourrait pas réussir sans vos compétences et connaissances.

4. Un autre PJ vous a demandé de les rejoindre.


%--------------------------
\label{sec:desctough}\section*{Dur-à-Cuire}

\textcolor{gray}{\emph{Tough}}

Vous êtes costaud et vous pouvez subir pas mal de chocs physiques. Vous avez des épaules larges et une machoire carrée. Les durs-à-cuir ont souvent des cicatrices visibles.

\textbf{Resilient:} +1 à l'Armure.

\textbf{Bonne santé:} Ajoutre 1 à vos jets de récupération.

\textbf{Compétence:} Vous êtes entraîné dans les actions de défense de Puissance.

\textbf{Equipement Supplémentaire:} Vous avez une arme légère supplémentaire.

\textbf{Lien initial à la Première Aventure:} Choisissez parmi la liste des options ci-dessous comment vous vous êtes retrouvé impliqué dans la première aventure:

1. Vous agissez comme garde du corps pour l'un des autres PJ.

2. L'un des PJ est votre frère et sœur, et vous êtes venu pour veiller sur eux.

3. Vous avez besoin d'argent parce que votre famille est endettée.

4. Vous êtes intervenu pour défendre l'un des PJ lorsque ce personnage a été menacé. En discutant avec eux par la suite, vous avez entendu parler de la tâche du groupe.


%--------------------------
\label{sec:descempathic}\section*{Empathique}

\textcolor{gray}{\emph{Empathic}}

Les autres sont des livres ouverts pour vous. Vous avez peut-être le don de lire les récits d'une personne, ces mouvements subtils qui traduisent l'humeur et la disposition d'un individu. Ou alors, vous pouvez recevoir des informations de manière plus directe, en ressentant les émotions d'une personne comme s'il s'agissait de choses tangibles, des sensations qui effleurent légèrement votre esprit. Votre don pour l'empathie vous aide à naviguer dans les situations sociales et à les contrôler pour éviter les malentendus et empêcher que des conflits inutiles n'éclatent.Le bombardement constant d'émotions de la part de votre entourage a probablement des conséquences néfastes. Vous pourriez évoluer selon l'humeur du moment, passant d'un bonheur vertigineux à un chagrin amer sans aucun avertissement. Ou vous pourriez vous fermer et rester impénétrable aux yeux des autres par sentiment d'auto-préservation ou par peur inconsciente que tout le monde puisse apprendre ce que vous ressentez vraiment.

\textbf{Overt d'esprit:} +4 à votre Réserve d'Intellect.

\textbf{Compétence:} Vous êtes entraîné dans les tâches impliquant de ressentir d'autres émotions, de discerner des dispositions ou d'avoir une idée des gens qui vous entourent.

\textbf{Compétence:} Vous êtes entraîné dans toutes les tâches impliquant une interaction sociale, agréable ou non.

\textbf{Inaptitude:} Être si réceptif aux pensées et aux humeurs des autres vous rend vulnérable à tout ce qui attaque votre esprit. Les jets de défense intellectuelle sont désavantagés.

\textbf{Lien initial à la Première Aventure:} Choisissez parmi la liste des options ci-dessous comment vous vous êtes retrouvé impliqué dans la première aventure:

1. Vous avez senti l'engagement des autres PJ dans la tâche et vous vous êtes senti poussé à les aider.

2. Vous avez établi un lien étroit avec un autre PJ et vous ne supportez pas de vous en séparer.

3. Vous avez senti quelque chose d'étrange chez l'un des PJ et avez décidé de rejoindre le groupe pour voir si vous pouvez le ressentir à nouveau et découvrir la vérité.

4. Vous avez rejoint les PJ pour échapper à une relation désagréable ou à un environnement négatif.


%--------------------------
\label{sec:deschardy}\section*{Endurant}

\textcolor{gray}{\emph{Hardy}}

Votre corps a été construit pour supporter les abus. Que vous buviez des boissons fortes tout en tenant le bar dans votre troquet préféré ou que vous échangez des coups avec un voyou dans une ruelle, vous continuez, ignorant les blessures et les blessures qui pourraient ralentir ou neutraliser une personne inférieure. Ni la faim ni la soif, ni la chair coupée ni les os brisés ne peuvent vous arrêter. Vous continuez simplement à surmonter la douleur et à continuer.Aussi en forme et en bonne santé que vous soyez, les signes d'usure se manifestent dans la myriade de cicatrices qui sillonnent votre corps, votre nez trois fois cassé, vos oreilles en chou-fleur et de nombreuses autres défigurations que vous portez avec fierté.

\textbf{Puissant:} +4 à votre Réserve de Puissance.

\textbf{Guérit rapidement:} Vous divisez par deux le temps nécessaire pour effectuer un jet de récupération (minimum une action).

\textbf{Quasiment Inarrêtable:} Tant que vous êtes diminué sur le suivi des dégâts, vous fonctionnez comme si vous étiez en bonne santé. Pendant que vous êtes handicapé, vous fonctionnez comme si vous étiez diminué. En d'autres termes, vous ne subissez pas les effets d'une déficience jusqu'à ce que vous deveniez handicapé, et vous ne subissez jamais les effets d'une déficience. Vous mourrez quand même si tous vos Réserves de statistiques sont à 0.

\textbf{Compétence:} Vous êtes entraîné auc actions de défense de Puissance.

\textbf{Inaptitude:} Vous êtes grand, fort, et lent à réagir. Toute tâche impliquant de l'initiative est désavantagée.

\textbf{Lourd:} Lorsque vous appliquez un Effort lors d'un jet de Célérité, vous devez dépenser 1 point supplémentaire de votre réserve de Célérité.

\textbf{Lien initial à la Première Aventure:} Choisissez parmi la liste des options ci-dessous comment vous vous êtes retrouvé impliqué dans la première aventure:

1. Les PJ vous ont recruté après avoir pris connaissance de votre réputation de survivant.

2. Vous avez rejoint les PJ parce que vous voulez ou avez besoin d'argent.

3. Les PJ vous ont proposé un défi égal à votre puissance physique.

4. Vous pensez que la seule façon pour les PJ de réussir est que vous soyez là pour les protéger.


%--------------------------
\label{sec:desclearned}\section*{Erudit}

\textcolor{gray}{\emph{Learned}}

Vous avez étudié seul ou avec un moniteur. Vous connaissez beaucoup de choses et êtes expert sur quelques sujets, comme l'histoire, la biologie, la géographie, la mythologie, la nature ou tout autre domaine d'étude. Les personnages érudits transportent généralement quelques livres avec eux et passent leur temps libre à lire.

\textbf{Intelligent:} +2 à votre Réserve d'Intellect.

\textbf{Compétence:} Vous êtes entraîné dans trois domaines de connaissances de votre choix.

\textbf{Inaptitude:} Vous avez peu d'aptitudes sociales. Toute tâche impliquant du charme, de la persuasion ou de l'étiquette est désavantagée.

\textbf{Equipement Supplémentaire:} Vous disposez de deux livres supplémentaires sur des sujets de votre choix.

\textbf{Lien initial à la Première Aventure:} Choisissez parmi la liste des options ci-dessous comment vous vous êtes retrouvé impliqué dans la première aventure:

1. Un des autres PJ vous a demandé de venir grâce à vos connaissances.

2. Vous avez besoin d'argent pour financer vos études.

3. Vous pensiez que cette tâche pourrait mener à des découvertes importantes et intéressantes.

4. Un collègue vous a demandé de participer à la mission en guise de faveur.


%--------------------------
\label{sec:descweird}\section*{Etrange}

\textcolor{gray}{\emph{Weird}}

Vous n'êtes pas comme les autres, et ça vous va. Les gens ne semblent pas vous comprendre – ils semblent même découragés par vous – mais peu importe ? Vous comprenez le monde mieux qu'eux parce que vous êtes bizarre, tout comme le monde dans lequel vous vivez. Le concept de « bizarre » vous est bien connu. Des appareils étranges, des lieux anciens, des créatures bizarres, des tempêtes qui peuvent vous transformer, des champs d'énergie vivants, des conspirations, des extraterrestres et des choses que la plupart des gens ne peuvent même pas nommer peuplent le monde, et vous prospérez grâce à eux. Vous avez un attachement particulier à tout cela, et plus vous en découvrez sur l'étrangeté du monde, plus vous pourriez en découvrir sur vous-même.Les personnages étranges peuvent être des mutants ou des personnes nées avec des qualités étranges, mais parfois ils ont commencé « normaux » et ont adopté l'étrange par choix.

\textbf{Lumière Intérieure:} +2 à votre Réserve d'Intellect.

\textbf{Bizarrerie Physique:} Vous avez un aspect physique unique qui est bizarre. Selon le paramètre, cela peut varier considérablement. Vous pourriez avoir des cheveux violets ou des pointes métalliques sur la tête. Peut-être que vos mains ne se connectent pas à vos bras, même si elles bougent comme si c'était le cas. Peut-être qu'un troisième œil regarde du côté de votre tête, ou que des vrilles superflues poussent dans votre dos. Quoi qu'il en soit, votre bizarrerie peut être une mutation, un trait surnaturel (une bénédiction ou une malédiction), une caractéristique sans explication, ou simplement un tatouage vraiment sauvage qui attire beaucoup d'attention.

\textbf{Un sens pour l'étrange:} Parfois, à la discrétion du MJ, des choses étranges liées au surnaturel ou à ses effets sur le monde semblent vous interpeller. Vous pouvez les sentir de loin, et si vous vous approchez à distance d'une telle chose, vous pouvez sentir si elle est ouvertement dangereuse ou non.

\textbf{Compétence:} Vous êtes entraîné dans les connaissances surnaturelles.

\textbf{Inaptitude:} Les gens vous trouvent énervant. Toutes les tâches liées à une interaction sociale agréable sont entravées.

\textbf{Lien initial à la Première Aventure:} Choisissez parmi la liste des options ci-dessous comment vous vous êtes retrouvé impliqué dans la première aventure:

1. Cela semblait bizarre, alors pourquoi pas ?

2. Que les autres PJ s'en rendent compte ou non, leur mission est liée à quelque chose d'étrange que vous connaissez, alors vous vous êtes impliqué.

3. En tant qu'expert en matière d'étrangeté, vous avez été spécifiquement recruté par les autres PJ.

4. Vous vous êtes senti attiré par l'idée de rejoindre les autres PJ, mais vous ne savez pas pourquoi.


%--------------------------
\label{sec:descexiled}\section*{Exilé}

\textcolor{gray}{\emph{Exiled}}

Vous avez parcouru un chemin long et solitaire, laissant votre maison et votre vie derrière vous. Vous avez peut-être commis un crime odieux, quelque chose de si horrible que votre peuple vous a forcé à partir, et si vous osez revenir, vous risquez la mort. Vous avez peut-être été accusé d'un crime que vous n'avez pas commis et vous devez maintenant payer le prix du mauvais acte de quelqu'un d'autre. Votre exil peut être le résultat d'une gaffe sociale : peut-être avez-vous fait honte à votre famille ou à un ami, ou vous êtes-vous embarrassé devant vos pairs, une autorité ou quelqu'un que vous respectez. Quelle que soit la raison, vous avez laissé votre ancienne vie derrière vous et vous efforcez maintenant d'en refaire une nouvelle.

\textbf{Autonome:} +2 à votre Réserve de Puissance.

\textbf{Solitaire:} Vous ne gagnez aucun avantage lorsque vous recevez de l'aide pour une tâche d'un autre personnage entraîné ou spécialisé dans cette tâche.

\textbf{Compétence:} Vous êtes entraîné dans toutes les tâches impliquant de se faufiler.

\textbf{Compétence:} Vous êtes entraîné dans toutes les tâches impliquant la recherche de nourriture, la chasse et la recherche d'endroits sûrs pour se reposer ou se cacher.

\textbf{Inaptitude:} Vivre seul aussi longtemps que vous avez pu le faire vous rend lent à faire confiance aux autres et vous rend maladroit dans les situations sociales. Toute tâche impliquant une interaction sociale est entravée.

\textbf{Equipement Supplémentaire:} Vous avez un souvenir de votre passé : une vieille photo, un médaillon avec quelques mèches de cheveux à l'intérieur ou un briquet que vous a offert une personne importante. Vous gardez l'objet à portée de main et vous le retirez pour vous aider à vous souvenir de meilleurs moments.

\textbf{Lien initial à la Première Aventure:} Choisissez parmi la liste des options ci-dessous comment vous vous êtes retrouvé impliqué dans la première aventure:

1. Les autres PJ ont gagné votre confiance en vous aidant lorsque vous en aviez besoin. Vous les accompagnez pour les rembourser.

2. En explorant par vous-même, vous avez découvert quelque chose d'étrange. Lorsque vous vous êtes rendu dans un lieu particulier, les PJ étaient les seuls à vous croire, et ils vous ont accompagné pour vous aider à résoudre le problème.

3. L'un des autres PJ vous rappelle quelqu'un que vous avez connu.

4. Vous êtes fatigué de votre isolement. Rejoindre les autres PJ vous donne une chance d'appartenir.


%--------------------------
\label{sec:descstrong}\section*{Fort}

\textcolor{gray}{\emph{Strong}}

Vous êtes extrêmement fort et physiquement puissant, et vous utilisez bien ces qualités, que ce soit par la violence ou par des prouesses. Vous avez probablement une carrure musclée et des muscles impressionnants.

\textbf{Très Puissant:} +4 à votre Réserve de Puissance.

\textbf{Compétence:} Vous êtes entraîné dans toutes les actions impliquant la destruction d'objets inanimés.

\textbf{Compétence:} Vous êtes entraîné dans toutes les actions de saut.

\textbf{Equipement Supplémentaire:} Vous disposez d'une arme moyenne ou lourde supplémentaire.

\textbf{Lien initial à la Première Aventure:} Choisissez parmi la liste des options ci-dessous comment vous vous êtes retrouvé impliqué dans la première aventure:

1. Contre votre bon jugement, vous avez rejoint les autres PJ parce que vous avez vu qu'ils étaient en danger.

2. L'un des autres PJ vous a convaincu que rejoindre le groupe serait dans votre intérêt.

3. Vous avez peur de ce qui pourrait arriver si les autres PC tombaient en panne.

4. Il y a une récompense en jeu et vous avez besoin d'argent.


%--------------------------
\label{sec:descmad}\section*{Fou}

\textcolor{gray}{\emph{Mad}}

Vous avez approfondi des sujets que les gens n'étaient pas censés connaître. Vous possédez des connaissances dans des domaines qui dépassent la portée de la plupart des gens, mais ces connaissances ont un prix terrible. Vous êtes probablement dans une forme physique douteuse et souffrez occasionnellement de tics nerveux. Vous marmonnez parfois sans vous en rendre compte.

\textbf{Bien informé:} +4 à votre Réserve d'Intellect.

\textbf{Eclairs de Génie:} Chaque fois qu'une telle connaissance est appropriée, le MJ vous fournit des informations bien qu'il n'y ait aucune explication claire sur la manière dont vous pourriez savoir une telle chose. Ceci est à la discrétion du MJ, mais cela devrait se produire aussi souvent qu'une fois par session.

\textbf{Comportement ératique:} Vous avez tendance à agir de manière erratique ou irrationnelle. Lorsque vous êtes en présence d'une découverte majeure ou soumis à un stress important (comme une menace physique sérieuse), le MJ peut introduire une intrusion du MJ qui oriente votre prochaine action sans vous attribuer d'XP. Vous pouvez toujours payer 1 XP pour refuser l'intrusion. L'influence du MJ est la manifestation de votre folie et c'est donc toujours quelque chose que vous ne feriez probablement pas autrement, mais elle ne vous est pas directement et évidemment nuisible, sauf circonstances atténuantes. (Par exemple, si un ennemi surgit soudainement de l'obscurité, vous pourriez passer le premier tour à babiller de manière incohérente ou à crier le nom de votre premier véritable amour.)

\textbf{Compétence:} Vous êtes entraîné à un domaine de connaissance (probablement quelque chose de bizarre ou d'ésotérique).

\textbf{Inaptitude:} Votre esprit est assez fragile. Les tâches visant à résister aux attaques mentales sont désavantagées.

\textbf{Lien initial à la Première Aventure:} Choisissez parmi la liste des options ci-dessous comment vous vous êtes retrouvé impliqué dans la première aventure:

1. Des voix dans votre tête vous ont dit de partir.

2. Vous avez tout déclenché et convaincu les autres de vous rejoindre.

3. L'un des autres PJ a obtenu un livre de connaissances pour vous, et maintenant vous lui rendez cette faveur en l'aidant dans la tâche à accomplir.

4. Vous vous sentez contraint par une intuition inexplicable.


%--------------------------
\label{sec:descstealthy}\section*{Furtif}

\textcolor{gray}{\emph{Stealthy}}

Vous êtes sournois, glissant et rapide. Ces talents vous aident à vous cacher, à vous déplacer tranquillement et à réaliser des tours qui nécessitent un tour de passe-passe. Très probablement, vous êtes nerveux et petit. Cependant, vous n'êtes pas vraiment un sprinter --- vous êtes plus adroit que rapide.

\textbf{Rapide:} +2 à votre Réserve de Célérité.

\textbf{Compétence:} Vous êtes entraîné dans toutes les tâches furtives.

\textbf{Compétence:} Vous êtes entraîné dans toutes les interactions impliquant des mensonges ou des supercheries.

\textbf{Compétence:} Vous êtes entraîné dans toutes les capacités spéciales impliquant des illusions ou des supercheries.

\textbf{Inaptitude:} Vous êtes sournois mais pas rapide. Toutes les tâches liées au mouvement sont gênées.

\textbf{Lien initial à la Première Aventure:} Choisissez parmi la liste des options ci-dessous comment vous vous êtes retrouvé impliqué dans la première aventure:

1. Vous avez tenté de voler l'un des autres PC. Ce personnage vous a attrapé et vous a forcé à les accompagner.

2. Vous suiviez l'un des autres PJ pour des raisons qui vous sont propres, ce qui vous a amené à l'action.

3. Un employeur de PNJ vous a secrètement payé pour vous impliquer.

4. Vous avez entendu les autres PJ parler d'un sujet qui vous intéressait, vous avez donc décidé d'approcher le groupe.


%--------------------------
\label{sec:desckind}\section*{Gentil}

\textcolor{gray}{\emph{Kind}}

Il a toujours été facile pour vous de voir les choses du point de vue des autres. Cette capacité vous a rendu sensible à ce qu'ils veulent ou ont réellement besoin. De votre point de vue, vous appliquez simplement le vieux proverbe selon lequel « il est plus facile d'attraper des mouches avec du miel qu'avec du vinaigre », mais d'autres voient simplement votre comportement comme de la gentillesse. Bien sûr, être gentil prend du temps, et le vôtre est limité. Vous avez appris qu'une petite fraction des gens ne mérite pas votre temps ou votre gentillesse : les vrais sadiques, narcissiques et autres personnes similaires ne feront que gaspiller votre énergie. Vous les traitez donc rapidement, réservant votre gentillesse à ceux qui la méritent et peuvent bénéficier de votre attention.

\textbf{Emotionellement Intuitif:} +2 à votre Réserve d'Intellect.

\textbf{Compétence:} Vous savez ce que c'est de se mettre à la place de quelqu'un d'autre. Vous êtes entraîné dans toutes les tâches liées à une interaction sociale agréable et au discernement des dispositions des autres.

\textbf{Karma:} Parfois, des inconnus vous aident simplement. Pour obtenir l'aide d'un étranger, vous devez dépenser un jet de récupération d'une action, de dix minutes ou d'une heure (sans bénéficier de son bénéfice de guérison), en échange de quoi le MJ détermine la nature de l'aide que vous obtenez. Habituellement, un acte de gentillesse ne suffit pas à renverser complètement une mauvaise situation, mais il peut atténuer une mauvaise situation et ouvrir de nouvelles opportunités. Par exemple, si vous êtes capturé, un garde desserre légèrement vos liens, vous apporte de l'eau ou vous délivre un message.

\textbf{Inaptitude:} Être gentil comporte quelques risques. Toutes les tâches liées à la détection des mensonges sont désavantagées.

\textbf{Lien initial à la Première Aventure:} Choisissez parmi la liste des options ci-dessous comment vous vous êtes retrouvé impliqué dans la première aventure:

1. Un PJ avait besoin de votre aide et vous avez accepté de l'accompagner par gentillesse.

2. Vous avez donné à la mauvaise personne accès à votre argent et vous devez maintenant en récupérer une partie.

3. Vous êtes prêt à mettre votre bienveillance en route et à aider plus de personnes que vous ne le pourriez si vous ne rejoigniez pas les PJ.

4. Votre travail, qui semblait personnellement gratifiant, en est l'exact opposé. Vous rejoignez les PJ pour échapper à la corvée.


%--------------------------
\label{sec:descgraceful}\section*{Grâcieux}

\textcolor{gray}{\emph{Graceful}}

Vous avez un parfait sens de l'équilibre, vous bougez et parlez avec grâce et beauté. Vous êtes rapide, souple, flexible et adroit. Votre corps est parfaitement adapté à la danse et vous utilisez cet avantage au combat pour esquiver les coups. Vous pouvez porter des vêtements qui améliorent votre agilité de mouvement et votre sens du style.

\textbf{Agile:} +2 à votre Réserve de Célérité.

\textbf{Compétence:} Vous êtes entraîné dans toutes les tâches impliquant un équilibre et des mouvements prudents.

\textbf{Compétence:} Vous êtes entraîné dans toutes les tâches impliquant les arts du spectacle physique.

\textbf{Compétence:} Vous êtes entraîné dans toutes les tâches de défense de Célérité.

\textbf{Lien initial à la Première Aventure:} Choisissez parmi la liste des options ci-dessous comment vous vous êtes retrouvé impliqué dans la première aventure:

1. Contre votre bon sens, vous avez rejoint les autres PJ parce que vous avez vu qu'ils étaient en danger.

2. L'un des autres PJ vous a convaincu que rejoindre le groupe serait dans votre intérêt.

3. Vous avez peur de ce qui pourrait arriver si les autres PC tombaient en panne.

4. Il y a une récompense en jeu et vous avez besoin d'argent.


%--------------------------
\label{sec:deschideous}\section*{Hideux}

\textcolor{gray}{\emph{Hideous}}

Vous êtes physiquement répugnant selon presque toutes les normes humaines. Vous avez peut-être eu un accident grave, une mutation nuisible ou simplement une mauvaise chance génétique, mais vous êtes incontestablement laid.Cependant, vous avez plus que compensé votre apparence par d'autres moyens. Parce que vous devez cacher votre apparence, vous excellez à vous faufiler inaperçu ou à vous déguiser. Mais peut-être le plus important, étant ostracisé pendant que les autres socialisaient, vous avez pris le temps de grandir pour vous développer comme bon vous semble --- vous êtes devenu fort ou rapide, ou vous avez affiné votre esprit.

\textbf{Versatile:} Vous obtenez 4 points supplémentaires à répartir entre vos Réserves de statistiques.

\textbf{Compétence:} Vous êtes entraîné à l'intimidation et toute autre interaction basée sur la peur, si vous montrez votre vrai visage.

\textbf{Compétence:} Vous êtes entraîné aux tâches de déguisement et de furtivité.

\textbf{Inaptitude:} Toutes les tâches liées à une interaction sociale agréable sont désavantagées.

\textbf{Lien initial à la Première Aventure:} Choisissez parmi la liste des options ci-dessous comment vous vous êtes retrouvé impliqué dans la première aventure:

1. L'un des autres PJ s'est approché de vous alors que vous étiez déguisé, vous recrutant en croyant que vous étiez quelqu'un d'autre.

2. En rôdant aux alentours, vous avez entendu les plans des autres PJs et réalisé que vous souhaitiez participer.

3. Un des autres PJ vous a invité, mais vous vous demandez si c'était par pitié.

4. Vous avez fait preuve d'intimidation et de fanfaronnades pour vous frayer un chemin.


%--------------------------
\label{sec:deschonorable}\section*{Honorable}

\textcolor{gray}{\emph{Honorable}}

Vous êtes digne de confiance, juste et franc. Vous essayez de faire ce qui est juste, d'aider les autres et de bien les traiter. Mentir et tricher ne sont pas des moyens d'avancer : ces choses sont réservées aux faibles, aux paresseux ou aux méprisables. Vous passez probablement beaucoup de temps à réfléchir à votre honneur personnel, à la meilleure façon de le préserver et de le défendre en cas de défi. Au combat, vous êtes franc et offrez du quart à n'importe quel ennemi.Ce sentiment d'honneur vous a probablement été inculqué par un parent ou un mentor. Parfois, la distinction entre ce qui est honorable et ce qui ne l'est pas varie selon les écoles de pensée, mais dans l'ensemble, les personnes honorables peuvent s'entendre sur la plupart des aspects de ce que signifie l'honneur.

\textbf{Vigoureux:} +2 à votre Réserve de Puissance.

\textbf{Compétence:} Vous êtes entraîné aux intéractions plaisantes.

\textbf{Compétence:} Vous êtes entraîné à discerner les véritables Focus des gens ou voir à travers les mensonges.

\textbf{Lien initial à la Première Aventure:} Choisissez parmi la liste des options ci-dessous comment vous vous êtes retrouvé impliqué dans la première aventure:

1. Les objectifs des PJ semblent honorables et louables.

2. Vous voyez que ce que les autres PJ sont sur le point de faire est dangereux et vous aimeriez aider à les protéger.

3. Un des autres PJ vous a invité, ayant entendu parler de votre fiabilité.

4. Vous avez demandé poliment si vous pouviez rejoindre les autres PJ dans leur mission.


%--------------------------
\label{sec:descimpulsive}\section*{Impulsif}

\textcolor{gray}{\emph{Impulsive}}

Vous avez du mal à contenir votre enthousiasme. Pourquoi attendre quand vous pouvez simplement le faire (quoi que ce soit) et le faire ? Vous traitez les problèmes lorsqu'ils surviennent plutôt que de planifier à l'avance. Éteindre les petits incendies évite désormais qu'ils ne se transforment en un grand incendie plus tard. Vous êtes le premier à prendre des risques, à vous lancer et à donner un coup de main, à vous engager dans des passages sombres et à trouver le danger.Votre impulsivité vous cause probablement des ennuis. Alors que d'autres peuvent prendre le temps d'étudier les objets qu'ils découvrent, vous utilisez ces objets sans hésitation. Après tout, la meilleure façon d'apprendre ce qu'une chose peut faire est de l'utiliser. Lorsqu'un explorateur prudent peut regarder autour de lui et vérifier s'il y a un danger à proximité, vous devez vous empêcher physiquement d'avancer. Pourquoi tourner auour du pot alors que des événements passionnants sont à notre porté ?Les personnages impulsifs ont des ennuis. C'est leur truc, et c'est très bien. Mais si vous entraînez constamment vos collègues PJ dans des ennuis (ou pire, si vous les faites gravement blesser ou tuer), ce sera pour le moins ennuyeux. Une bonne règle de base est que l'impulsivité ne signifie pas toujours une prédilection à faire la mauvaise chose. Parfois, c'est l'envie de faire la bonne chose.

\textbf{Casse-cou:} +2 à votre Réserve de Célérité.

\textbf{Compétence:} Vous êtes entraîné dans les actions d'initiative (pour déterminer qui commence le combat en premier).

\textbf{Compétence:} Vous êtes entraîné dans les actions de défense de Célérité.

\textbf{Inaptitude:} Vous essayez n'importe quoi une fois, mais vous vous ennuyez rapidement par la suite. Toute tâche qui implique de la patience, de la volonté ou de la discipline est désavantagée.

\textbf{Lien initial à la Première Aventure:} Choisissez parmi la liste des options ci-dessous comment vous vous êtes retrouvé impliqué dans la première aventure:

1. Vous avez entendu ce que faisaient les autres PJ et avez soudainement décidé de les rejoindre.

2. Vous avez rassemblé tout le monde après avoir entendu des rumeurs concernant quelque chose d'intéressant que vous vouliez voir ou faire.

3. Vous avez dépensé tout votre argent et vous vous retrouvez maintenant à court d'argent.

4. Vous avez des ennuis pour avoir agi de manière imprudente. Vous rejoignez les autres PC car ils offrent une issue à votre problème.


%--------------------------
\label{sec:descbrash}\section*{Impétieux}

\textcolor{gray}{\emph{Brash}}

Vous êtes du genre affirmé, confiant en vos capacités, énergique et peut-être un peu irrévérencieux envers les idées avec lesquelles vous n'êtes pas d'accord. Certaines personnes vous qualifient d'audacieux et de courageux, mais ceux que vous avez mis à leur place pourraient vous qualifier d'enflé et d'arrogant. Peu importe. Ce n'est pas dans votre nature de vous soucier de ce que les autres pensent de vous, à moins que ces personnes ne soient vos amis ou votre famille. Même quelqu'un d'aussi impétueux que vous sait que les amis doivent parfois passer en premier.

\textbf{Energetique:} +2 à votre Réserve de Célérité.

\textbf{Compétence:} Vous êtes entraîné à l'initiative.

\textbf{Audacieux:} Vous êtes entraîné à toutes les actions qui impliquent de dépasser ou d'ignorer les effets de la peur ou de l'intimidation.

\textbf{Lien initial à la Première Aventure:} Choisissez parmi la liste des options ci-dessous comment vous vous êtes retrouvé impliqué dans la première aventure:

1. Vous avez remarqué quelque chose de bizarre et, sans trop réfléchir, vous avez sauté à pieds joints.

2. Vous êtes arrivé là où vous êtes à cause d'un défi parce que, hé, vous ne reculez pas devant les défis.

3. Quelqu'un vous a appelé, mais au lieu de vous lancer dans une bagarre, vous vous êtes retrouvé dans votre situation actuelle.

4. Vous avez dit à votre ami que rien ne pouvait vous effrayer et que rien de ce que vous verriez ne vous ferait changer d'avis. Ils vous ont amené à votre point actuel.


%--------------------------
\label{sec:descintelligent}\section*{Intelligent}

\textcolor{gray}{\emph{Intelligent}}

Vous êtes plutôt intelligent. Votre mémoire est vive et vous comprenez facilement les concepts avec lesquels d'autres pourraient avoir du mal. Cette aptitude ne signifie pas nécessairement que vous avez suivi des années d'éducation formelle, mais que vous avez beaucoup appris dans votre vie, principalement parce que vous maîtrisez rapidement les choses et que vous retenez beaucoup de choses.

\textbf{Intelligent:} +2 à votre Réserve d'Intellect.

\textbf{Compétence:} Vous êtes entraîné dans un domaine de connaissance de votre choix.

\textbf{Compétence:} Vous êtes entraîné dans toutes les actions qui impliquent de se souvenir ou de mémoriser des choses que vous vivez directement. Par exemple, au lieu de bien vous souvenir des détails géographiques que vous avez lus dans un livre, vous pouvez vous souvenir d'un chemin à travers un ensemble de tunnels que vous avez déjà explorés.

\textbf{Lien initial à la Première Aventure:} Choisissez parmi la liste des options ci-dessous comment vous vous êtes retrouvé impliqué dans la première aventure:

1. L'un des autres PJ vous a demandé votre avis sur la mission, sachant que si vous pensiez que c'était une bonne idée, c'était probablement le cas.

2. Vous avez vu la valeur de ce que faisaient les autres PJ.

3. Vous pensiez que cette tâche pourrait mener à des découvertes importantes et intéressantes.

4. Un collègue vous a demandé de participer à la mission en guise de faveur.


%--------------------------
\label{sec:descintuitive}\section*{Intuitif}

\textcolor{gray}{\emph{Intuitive}}

Vous êtes souvent chatouillé par le sentiment de savoir ce que quelqu'un va dire, comment il va réagir ou comment les événements pourraient se dérouler. Peut-être avez-vous un sens mutant, peut-être pouvez-vous voir quelques instants à venir dans le temps, ou peut-être êtes-vous simplement doué pour lire les gens et extrapoler une situation. Quoi qu'il en soit, beaucoup de ceux qui vous regardent dans les yeux détournent immédiatement le regard, comme s'ils avaient peur de ce que vous pourriez voir dans leur expression.

\textbf{Inné:} +2 à votre Réserve d'Intellect.

\textbf{Compétence:} Vous êtes entraîné aux tâches de perception.

\textbf{Sais quoi faire:} Vous pouvez agir immédiatement, même si ce n'est pas votre tour. Ensuite, lors de votre prochain tour normal, toute action que vous entreprenez est désavantagée. Vous pouvez le faire une seule fois, bien que la capacité soit renouvelée à chaque fois que vous effectuez un jet de récupération.

\textbf{Lien initial à la Première Aventure:} Choisissez parmi la liste des options ci-dessous comment vous vous êtes retrouvé impliqué dans la première aventure:

1. Vous saviez simplement que vous deviez venir.

2. Vous avez convaincu l'un des autres PJ que votre intuition est inestimable.

3. Vous pensiez que quelque chose de terrible allait se produire si vous n'y alliez pas.

4. Vous êtes convaincu que la raison pour laquelle vous en êtes arrivé à ce point deviendra bientôt claire.


%--------------------------
\label{sec:descjovial}\section*{Jovial}

\textcolor{gray}{\emph{Jovial}}

Vous êtes joyeux, sympathique et extraverti. Vous mettez les autres à l'aise avec un grand sourire et une blague, éventuellement à vos frais, même si taquiner légèrement vos compagnons qui peuvent le supporter est aussi l'un de vos passe-temps favoris. Parfois, les gens disent qu'on ne prend jamais rien au sérieux. Ce n'est pas vrai, bien sûr, mais vous avez appris que s'attarder trop longtemps sur le mal prive rapidement le monde de sa joie. Vous avez toujours une nouvelle blague dans votre poche parce que vous les collectionnez comme certaines personnes collectionnent les bouteilles de vin.

\textbf{Spirituel:} +2 à votre Réserve d'Intellect.

\textbf{Compétence:} Vous êtes convivial et mettez la plupart des gens à l'aise avec votre attitude. Vous êtes entraîné à toutes les tâches liées à une interaction sociale agréable.

\textbf{Compétence:} Vous avez un avantage pour comprendre les punchlines des blagues que vous n'avez jamais entendues auparavant. Vous êtes entraîné à toutes les tâches liées à la résolution d'énigmes et d'énigmes.

\textbf{Lien initial à la Première Aventure:} Choisissez parmi la liste des options ci-dessous comment vous vous êtes retrouvé impliqué dans la première aventure:

1. Vous avez résolu une énigme avant de réaliser que répondre vous lancerait dans l'aventure.

2. Les autres PJ pensaient que vous apporteriez une légèreté bien nécessaire à l'équipe.

3. Vous avez décidé que se divertir sans travailler n'était pas le meilleur moyen de vivre la vie, alors vous avez rejoint les PJ.

4. Il s'agissait soit d'aller avec les PJ, soit d'affronter une situation qui n'était tout sauf joviale.


%--------------------------
\label{sec:desccraven}\section*{Lâche}

\textcolor{gray}{\emph{Craven}}

Le courage vous fait défaut à chaque instant. Vous manquez de volonté et de détermination pour tenir bon face au danger. La peur vous ronge le cœur, ronge votre esprit, vous conduit à la distraction jusqu'à ce que vous ne puissiez plus la supporter. La plupart du temps, vous reculez devant les confrontations. Vous fuyez les menaces et hésitez face à des décisions difficiles.Pourtant, malgré toute la peur qui vous hante et peut-être vous fait honte, votre nature lâche se révèle être un allié utile de temps en temps. Écouter vos peurs vous a aidé à échapper au danger et à éviter de prendre des risques inutiles. D'autres ont peut-être souffert à votre place, et vous êtes peut-être le premier à l'admettre, mais secrètement, vous ressentez un intense soulagement d'avoir évité un destin impensable et terrible.Des descripteurs comme Lâche, Cruel et Déshonorant pourraient ne pas convenir à tous les groupes. Ce sont des traits de caractère attribués plutôt pour les opposants des PJs et certaines personnes souhaitent que leurs PJ soient entièrement héroïques. Mais d'autres ne voient pas d'inconvénient à ce qu'un peu de grisaille morale vienne s'ajouter à ce mélange. D'autres encore voient des choses comme Lâche et Cruel comme des traits à surmonter à mesure que leurs personnages se développent (ce qui leur vaut probablement des descripteurs différents).

\textbf{Furtive:} +2 à votre Réserve de Célérité.

\textbf{Compétence:} Vous êtes entraîné dans les tâches de furtivité.

\textbf{Compétence:} Vous êtes entraîné dans les actions pour courir.

\textbf{Compétence:} Vous êtes entraîné pour toutes les actions prises pour échapper au danger, fuir devant une situation dangeureuse ou se sortir du pétrin.

\textbf{Inaptitude:} Vous n'entrez pas volontairement dans des situations dangereuses. Toute action d'initiative (pour déterminer qui commence le combat en premier) est désavantagée.

\textbf{Inaptitude:} Vous vous décomposez lorsque vous devez entreprendre seul une tâche potentiellement dangereuse. Toute tâche de ce type (comme attaquer une créature par vous-même) est désavantagée.

\textbf{Equipement Supplémentaire:} Vous disposez d'un porte-bonheur ou d'un dispositif de protection pour vous protéger du danger.

\textbf{Lien initial à la Première Aventure:} Choisissez parmi la liste des options ci-dessous comment vous vous êtes retrouvé impliqué dans la première aventure:

1. Vous pensez être pourchassé et vous avez engagé l'un des autres PJ comme votre protecteur.

2. Vous cherchez à échapper à votre honte et à vous rapprocher de personnes compétentes dans l'espoir de réparer votre réputation.

3. L'un des autres PJ vous a intimidé pour que vous veniez.

4. Le groupe a répondu à vos appels à l'aide lorsque vous étiez en difficulté.


%--------------------------
\label{sec:descclumsy}\section*{Maladroit}

\textcolor{gray}{\emph{Clumsy}}

Sans grâce et maladroit, on vous a dit que vous en sortiriez, mais vous ne l'avez jamais fait. Vous faites souvent tomber des objets, trébuchez sur vos propres pieds ou renversez des objets (ou des personnes). Certaines personnes sont frustrées par cette qualité, mais la plupart la trouvent drôle et même un peu charmante.Certains joueurs ne veulent peut-être pas être définis par une qualité « négative » comme Maladroit, mais en vérité, même ce type de descripteur présente suffisamment d'avantages pour en faire des personnages capables et talentueux. Ce que font réellement les descripteurs négatifs, c'est créer des personnages plus intéressants et complexes qui sont souvent très amusants à jouer.

\textbf{Empôté:} -2 à votre Réserve de Célérité.

\textbf{Musclé:} +2 à votre Réserve de Puissance.

\textbf{Inélegant:} Vous avez un certain charme adorable. Vous êtes entraîné à toutes les interactions sociales agréables lorsque vous vous exprimez une manière légère et avec autodérision.

\textbf{Chance Bête:} Le MJ peut introduire une intrusion du MJ sur vous, en fonction de votre maladresse, sans vous attribuer d'XP (comme si vous aviez obtenu un 1 sur un jet de d20). Cependant, si cela se produit, 50 % du temps, votre maladresse joue à votre avantage. Plutôt que de vous faire (beaucoup) du mal, cela vous aide, ou cela fait du mal à vos ennemis. Vous glissez, mais c'est juste à temps pour esquiver une attaque. Vous tombez, mais vous faites trébucher vos ennemis lorsque vous vous écrasez dans leurs jambes. Vous vous retournez trop rapidement, mais vous finissez par faire tomber l'arme des mains de votre ennemi. Vous et le MJ devez travailler ensemble pour déterminer les détails. Si le MJ le souhaite, il peut normalement utiliser les intrusions du MJ en fonction de votre maladresse (en attribuant de l'XP).

\textbf{Compétence:} Vous êtes un gars costaud. Vous êtes entraîné aux tâches impliquant de casser des objets.

\textbf{Inaptitude:} Toute tâche qui implique l'équilibre, la grâce ou la coordination main-oeil est désavantagée.

\textbf{Lien initial à la Première Aventure:} Choisissez parmi la liste des options ci-dessous comment vous vous êtes retrouvé impliqué dans la première aventure:

1. Vous étiez au bon endroit au bon moment.

2. Vous disposiez d'une information dont les autres PJ avaient besoin pour élaborer leurs plans.

3. Un frère ou une sœur vous a recommandé aux autres PC.

4. Vous êtes tombé sur les PJ alors qu'ils discutaient de leur mission, et ils se sont pris d'affection pour vous.


%--------------------------
\label{sec:descclever}\section*{Malin}

\textcolor{gray}{\emph{Clever}}

Vous avez l'esprit vif et vous réfléchissez bien. Vous comprenez les gens, vous pouvez donc les tromper, mais vous êtes rarement dupe. Parce que vous voyez facilement les choses telles qu'elles sont, vous obtenez rapidement un aperçu du terrain, évaluez les menaces et les alliés et évaluez les situations avec précision. Peut-être êtes-vous physiquement attirant ou utilisez-vous votre esprit pour surmonter vos imperfections physiques ou mentales.

\textbf{Malin:} +2 à votre Réserve d'Intellect.

\textbf{Compétence:} Vous êtes entraîné dans toutes les interactions impliquant des mensonges et de la tricherie.

\textbf{Compétence:} Vous êtes entraîné dans les jets de défense pour résister aux attaques mentales.

\textbf{Compétence:} Vous êtes entraîné dans toutes les tâches pour identifier ou évaluer un danger, des mensonges, ou la qualité, l'importance ou la fonction de quelque chose.

\textbf{Inaptitude:} YVous n'avez jamais été doué pour étudier ou retenir des connaissances triviales. Toute tâche impliquant des connaissances, des connaissances ou une compréhension est désavantagée.

\textbf{Equipement Supplémentaire:} Vous voyez à travers les plans des autres et les convainquez parfois de vous croire, même si, peut-être, ils ne devraient pas le faire. Grâce à votre comportement intelligent, vous disposez d'un article coûteux supplémentaire.

\textbf{Lien initial à la Première Aventure:} Choisissez parmi la liste des options ci-dessous comment vous vous êtes retrouvé impliqué dans la première aventure:

1. Vous avez convaincu l'un des autres PJ de vous dire ce qu'il faisait.

2. De loin, tu as observé qu'il se passait quelque chose d'intéressant.

3. Vous avez parlé de cette situation parce que vous pensiez que cela pourrait rapporter de l'argent.

4. Vous soupçonnez que les autres PJ ne réussiront pas sans vous.


%--------------------------
\label{sec:descdoomed}\section*{Maudit}

\textcolor{gray}{\emph{Doomed}}

Vous êtes bien certain que votre destin vous mène, inextricablement, vers une fin terrible. Ce destin pourrait être le vôtre seul, ou vous pourriez entraîner les personnes les plus proches de vous.

\textbf{Jumpy:} +2 à votre Réserve de Célérité.

\textbf{Compétence:} Toujours à l'affût du danger, vous êtes entraîné aux tâches liées à la perception.

\textbf{Compétence:} Vous avez l'esprit défensif, vous êtes donc entraîné aux tâches de défense de Célérité.

\textbf{Compétence:} Vous êtes cynique et vous attendez au pire. Ainsi, vous résistez aux chocs mentaux. Vous êtes entraîné à des tâches de défense d'Intellect liées à la perte de votre santé mentale ou de votre équanimité.

\textbf{Maudit:} Une fois sur deux, le MJ utilise l'intrusion du MJ sur votre personnage, vous ne pouvez pas le refuser et ne recevez pas d'XP pour cela (vous obtenez toujours un XP à attribuer à un autre joueur). C'est parce que vous êtes condamné. L'univers est un endroit froid et indifférent, et vos efforts sont pour le moins vains.

\textbf{Lien initial à la Première Aventure:} Choisissez parmi la liste des options ci-dessous comment vous vous êtes retrouvé impliqué dans la première aventure:

1. Vous avez tenté de l'éviter, mais les événements semblent conspirer pour vous attirer là où vous êtes.

2. Pourquoi pas ? Cela n'a pas d'importance. Vous êtes maudit, quoi que vous fassiez.

3. L'un des autres PJ vous a sauvé la vie, et maintenant vous remboursez cette obligation en l'aidant dans la tâche à accomplir.

4. Vous pensez que votre seul espoir d'éviter votre sort réside peut-être sur cette voie.


%--------------------------
\label{sec:desctonguetied}\section*{Mutique}

\textcolor{gray}{\emph{Tongue-Tied}}

Vous n'avez jamais été très bavard. Lorsque vous êtes obligé d'interagir avec les autres, vous ne pensez jamais à la bonne chose à dire : les mots vous manquent complètement, ou ils sont complètement faux. Vous finissez souvent par dire précisément la mauvaise chose et par insulter quelqu'un sans le vouloir. La plupart du temps, vous restez silencieux. Cela fait de vous un auditeur --- un observateur attentif. Cela signifie également que vous êtes plus doué pour faire les choses que pour en parler. Vous êtes prompt à agir.

\textbf{Des actions, pas des mots:} +2 à votre Réserve de Puissance, et +2 à votre Réserve de Célérité.

\textbf{Compétence:} Vous êtes entraîné à la compétence perception.

\textbf{Compétence:} Vous êtes entraîné à l'initiative( sauf s'il s'agit d'une situation sociale).

\textbf{Inaptitude:} Toutes les tâches liées aux interactions sociales sont désavantagées.

\textbf{Inaptitude:} Toutes les tâches impliquant une communication verbale ou un relais d'information sont désavantagées.

\textbf{Lien initial à la Première Aventure:} Choisissez parmi la liste des options ci-dessous comment vous vous êtes retrouvé impliqué dans la première aventure:

1. Vous venez de suivre et personne ne vous a dit de partir.

2. Vous avez vu quelque chose d'important que les autres PJ n'ont pas vu et (avec quelques efforts) avez réussi à le leur faire comprendre.

3. Vous êtes intervenu pour sauver l'un des autres PJ alors qu'il était en danger.

4. L'un des autres PJ vous a recruté pour vos talents.


%--------------------------
\label{sec:descmystical}\section*{Mystique}

\textcolor{gray}{\emph{Mystical}}

Vous vous considérez comme mystique, en harmonie avec le mystérieux et le paranormal. Vos vrais talents résident dans le surnaturel. Vous avez probablement de l'expérience avec les traditions anciennes et vous pouvez ressentir et manier le surnaturel --- même si cela signifie « magie », « phénomènes psychiques » ou autre chose, cela dépend de vous (et probablement aussi de ceux qui vous entourent). Les personnages mystiques portent souvent des bijoux, comme une bague ou une amulette, ou ont des tatouages ou d'autres marques qui montrent leurs intérêts.

\textbf{Intelligent:} +2 à votre Réserve d'Intellect.

\textbf{Compétence:} Vous êtes entraîné dans toutes les actions impliquant l'identification ou la compréhension du surnaturel.

\textbf{Sentir la Magie:} Vous pouvez sentir si le surnaturel est actif dans des situations où sa présence n'est pas évidente. Vous devez étudier attentivement un objet ou un lieu pendant une minute pour savoir si une touche mystique est à l'œuvre.

\textbf{Sort:} Vous pouvez exécuter la capacité Magie Prosaïque comme un sort lorsque vous avez une main libre et que vous pouvez payer le coût en points d'Intellect.

\textbf{Inaptitude:} Vous avez des manières ou une aura que les autres trouvent un peu déconcertantes. Toute tâche impliquant le charme, la persuasion ou la tromperie est désavantagée.

\textbf{Lien initial à la Première Aventure:} Choisissez parmi la liste des options ci-dessous comment vous vous êtes retrouvé impliqué dans la première aventure:

1. Un rêve vous a guidé jusqu'à ce point.

2. Vous avez besoin d'argent pour financer vos études.

3. Vous pensiez que la mission serait un excellent moyen d'en apprendre davantage sur le surnaturel.

4. Divers signes et présages vous ont conduit ici.


%--------------------------
\label{sec:descmysterious}\section*{Mystérieux}

\textcolor{gray}{\emph{Mysterious}}

La silhouette sombre qui se cache silencieusement dans un coin ? C'est vous. Personne ne sait vraiment d'où vous venez ni quelles sont vos Focus : vous cachez bien votre jeu. A plupart des gens sont perplexes, mais cela ne fait pas de vous un mauvais ami ou un mauvais allié. Vous êtes simplement doué pour garder les choses pour vous, vous déplacer sans être vu et dissimuler votre présence et votre identité.

\textbf{Compétence:} Vous êtes entraîné aux tâches de furtivité.

\textbf{Compétence:} Vous êtes entraîné à résister à un interrogatoire ou à des astuces pour vous faire parler.

\textbf{Touche-à-tout:} Vous tirez des talents et des capacités apparemment de nulle part. Vous pouvez tenter une tâche pour laquelle vous n'avez aucune formation comme si vous étiez entraîné, tenter une tâche pour laquelle vous êtes entraîné comme si vous étiez spécialisé ou gagner un niveau d'effort gratuit avec une tâche pour laquelle vous êtes spécialisé. Cette capacité est actualisée à chaque fois. vous faites un jet de récupération, mais les utilisations ne s'accumulent jamais.

\textbf{Inaptitude:} Les gens ne savent jamais où ils en sont avec vous. Toute tâche consistant à amener les gens à vous croire ou à vous faire confiance est désavantagée.

\textbf{Lien initial à la Première Aventure:} Choisissez parmi la liste des options ci-dessous comment vous vous êtes retrouvé impliqué dans la première aventure:

1. Vous vous présentez soudainement un jour.

2. Vous avez convaincu l'un des autres PJ que vous aviez des compétences inestimables.

3. Un personnage tout aussi mystérieux vous a indiqué où être et quand (mais pas pourquoi) rejoindre le groupe.

4. Quelque chose --- un sentiment, un rêve --- vous a dit où être et quand rejoindre le groupe.


%--------------------------
\label{sec:descnaive}\section*{Naif}

\textcolor{gray}{\emph{Naive}}

Vous avez vécu une vie protégée. Votre enfance a été sûre et sécurisée, vous n'avez donc pas eu l'occasion d'en apprendre beaucoup sur le monde --- et encore moins d'en faire l'expérience. Que vous vous entraîniez pour quelque chose, que vous ayez le nez plongé dans un livre ou que vous soyez simplement séquestré dans un endroit isolé, vous n'avez pas fait grand-chose, rencontré beaucoup de gens ou vu beaucoup de choses intéressantes jusqu'à présent. Cela va probablement changer bientôt, mais à mesure que vous avancez dans un monde plus vaste, vous le faites sans une partie de la compréhension que les autres possèdent sur la façon dont tout cela fonctionne.

\textbf{Frais et Dispo:} Vous ajoutez +1 à vos vos jets de Récupération.

\textbf{Incorruptible:} Vous êtes entraîné à aux tâches de défense d'Intellect et toutes tâches impliquant de résister à la tentation.

\textbf{Compétence:} You're wide-eyed. Vous êtes entraîné aux tâches de perception.

\textbf{Inaptitude:} Toute tâche qui implique de démystifier des tromperies ou de déterminer les Focus secrètes de quelqu'un est désavantagée.

\textbf{Lien initial à la Première Aventure:} Choisissez parmi la liste des options ci-dessous comment vous vous êtes retrouvé impliqué dans la première aventure:

1. Quelqu'un vous a dit que vous devriez vous impliquer.

2. Vous aviez besoin d'argent, et cela vous semblait être un bon moyen d'en gagner.

3. Vous pensiez que vous pourriez apprendre beaucoup en rejoignant les autres PJ.

4. Cela avait l'air amusant.


%--------------------------
\label{sec:descfoolish}\section*{Pas très brillant}

\textcolor{gray}{\emph{Foolish}}

Tout le monde ne peut pas être brillant. Oh, vous ne vous considérez pas comme stupide, et vous ne l'êtes pas. C'est juste que d'autres pourraient avoir un peu plus de...sagesse. Vous préférez avancer tête première dans la vie et laisser les autres s'inquiéter des choses. S'inquiéter ne vous a jamais aidé, alors pourquoi s'embêter ? Vous prenez les choses au pied de la lettre et ne vous inquiétez pas de ce que demain pourrait vous apporter.Les gens vous traitent d'« idiot » ou de « crétin », mais cela ne vous dérange pas beaucoup.Cela peut être libérateur et vraiment amusant de jouer un personnage un peu idiot. D'une certaine manière, la pression de toujours faire ce qui est juste et intelligent a disparu. D'un autre côté, si vous incarnez un personnage comme un crétin maladroit dans toutes les situations, cela peut devenir ennuyeux pour tout le monde autour de la table. Comme pour tout, la modération est la clé.

\textbf{Imprudent:} –4 à votre Réserve d'Intellect.

\textbf{Insouciant:} Vous réussissez plus par chance qu'autre chose. Chaque fois que vous lancez un dé pour une tâche, lancez deux fois et obtenez le résultat le plus élevé.

\textbf{Faiblesse intellectuelle :} Chaque fois que vous dépensez des points de votre réserve d'Intellect, cela vous coûte 1 point de plus que d'habitude.

\textbf{Inaptitude:} Toute tâche de défense intellectuelle est désavantagée.

\textbf{Inaptitude:} Toute tâche qui implique de voir à travers une tromperie, une illusion ou un piège est désavantagée.

\textbf{Lien initial à la Première Aventure:} Choisissez parmi la liste des options ci-dessous comment vous vous êtes retrouvé impliqué dans la première aventure:

1. Qui sait ? Sur le moment cela semblait être une bonne idée.

2. Quelqu'un vous a demandé de rejoindre les autres PJ. Ils vous ont dit de ne pas poser trop de questions, et cela vous a semblé bien.

3. Votre parent (ou une figure parentale/mentor) vous a impliqué pour vous donner quelque chose à faire et peut-être « vous apprendre un peu de bon sens ».

4. Les autres PJ avaient besoin de muscles pour ne pas trop réfléchir.


%--------------------------
\label{sec:descperceptive}\section*{Perspicace}

\textcolor{gray}{\emph{Perceptive}}

Peu vous échappe. Vous repérez les petits détails du monde qui vous entoure et êtes habile à faire des déductions à partir des informations que vous trouvez. Vos talents font de vous un détective exceptionnel, un scientifique redoutable ou un éclaireur talentueux.Même si vous êtes habile à trouver des indices, vous n'avez aucune compétence pour détecter les signaux sociaux. Vous négligez une infraction causée par vos déductions ou à quel point votre examen minutieux peut rendre les gens autour de vous inconfortables. Vous avez tendance à considérer les autres comme des nains intellectuels par rapport à vous, ce qui ne vous sert pas à grand-chose lorsque vous avez besoin d'une faveur.

\textbf{Malin:} +2 à votre Réserve d'Intellect.

\textbf{Compétence:} Vous avez le sens du détail. Vous êtes entraîné à toute tâche qui implique de trouver ou de remarquer de petits détails.

\textbf{Compétence:} Vous savez un peu tout. Vous êtes entraîné à toute tâche qui implique d'identifier des objets ou de rappeler un détail mineur ou une anecdote.

\textbf{Compétence:} Votre capacité à faire des déductions peut être imposante. Vous êtes entraîné à toute tâche qui implique d'intimider une autre créature.

\textbf{Inaptitude:} Votre confiance en vous apparaît comme de l'arrogance aux yeux des personnes qui ne vous connaissent pas. Toute tâche impliquant des interactions sociales positives est désavantagée.

\textbf{Equipement Supplémentaire:} Vous disposez d'un sac d'outils légers.

\textbf{Lien initial à la Première Aventure:} Choisissez parmi la liste des options ci-dessous comment vous vous êtes retrouvé impliqué dans la première aventure:

1. Vous avez entendu les autres PJ discuter de leur mission et avez proposé vos services.

2. L'un des PJ vous a demandé de venir, pensant que vos talents seraient inestimables pour la mission.

3. Vous pensez que la mission des PJ est liée d'une manière ou d'une autre à l'une de vos enquêtes.

4. Un tiers vous a recruté pour suivre les PJ et voir ce qu'ils faisaient.


%--------------------------
\label{sec:descfast}\section*{Prompt}

\textcolor{gray}{\emph{Fast}}

Vous vous déplacez rapidement, êtes capable de sprinter par courtes rafales et de travailler avec vos mains avec dextérité. Vous êtes doué pour franchir des distances rapidement, mais pas toujours en douceur. Vous êtes probablement mince et musclé.

\textbf{Rapide:} +4 à votre Réserve de Célérité.

\textbf{Compétence:} Vous êtes entraîné dans les actions d'initiative (pour déterminer qui commence le combat en premier).

\textbf{Compétence:} Vous êtes entraîné dans les actions en course.

\textbf{Inaptitude:} Vous êtes rapide mais pas nécessairement gracieux. Toute tâche impliquant l'équilibre est désavantagé.

\textbf{Lien initial à la Première Aventure:} Choisissez parmi la liste des options ci-dessous comment vous vous êtes retrouvé impliqué dans la première aventure:

1. Contre votre bon jugement, vous avez rejoint les autres PJ parce que vous avez vu qu'ils étaient en danger.

2. L'un des autres PJ vous a convaincu que rejoindre le groupe serait dans votre intérêt.

3. Vous avez peur de ce qui pourrait arriver si les autres PC tombaient en panne.

4. Il y a une récompense en jeu et vous avez besoin d'argent.


%--------------------------
\label{sec:descswift}\section*{Rapide}

\textcolor{gray}{\emph{Swift}}

Vous êtes vif. Parce que vous êtes rapide, vous pouvez accomplir des tâches plus rapidement que les autres. Cependant, vous n'êtes pas seulement rapide avec vos pieds : vous êtes rapide avec vos mains, et vous réfléchissez et réagissez rapidement. Vous parlez même vite.

\textbf{Energetique:} +2 à votre Réserve de Célérité.

\textbf{Compétence:} Vous êtes entraîné à courir.

\textbf{Rapide:} Vous pouvez vous déplacer sur une courte distance tout en effectuant une autre action au cours du même tour, ou vous pouvez vous déplacer sur une longue distance au cours de votre action sans avoir besoin d'effectuer un quelconque jet de dés.

\textbf{Inaptitude:} Vous êtes un sprinter, pas un coureur de fond. Vous n'avez pas beaucoup d'endurance. Les jets de défense pourraient être désavantagés.

\textbf{Lien initial à la Première Aventure:} Choisissez parmi la liste des options ci-dessous comment vous vous êtes retrouvé impliqué dans la première aventure:

1. Vous êtes intervenu pour sauver l'un des autres PJ qui en avait cruellement besoin.

2. L'un des autres PJ vous a recruté pour vos talents uniques.

3. Vous êtes impulsif et cela semblait être une bonne idée à l'époque.

4. Cette mission est liée à un objectif personnel qui vous est propre.


%--------------------------
\label{sec:descrugged}\section*{Rugueux}

\textcolor{gray}{\emph{Rugged}}

Vous êtes un amoureux de la nature, habitué à vivre dans la dure et à affronter les éléments. Très probablement, vous êtes un chasseur, un cueilleur ou un naturaliste expérimenté. Des années de vie dans la nature ont laissé des traces avec un visage usé, des cheveux sauvages ou des cicatrices. Vos vêtements sont probablement beaucoup moins raffinés que ceux portés par les citadins.

\textbf{Compétence:} Vous êtes entraîné dans toutes les tâches impliquant l'escalade, le saut, la course et la natation.

\textbf{Compétence:} Vous êtes entraîné dans toutes les tâches impliquant le dressage, l'équitation ou l'apaisement d'animaux naturels.

\textbf{Compétence:} Vous êtes entraîné dans toutes les tâches impliquant l'identification ou l'utilisation de plantes naturelles.

\textbf{Inaptitude:} Vous n'avez aucune grâce sociale et préférez les animaux aux humains. Toute tâche impliquant le charme, la persuasion, l'étiquette ou la tromperie est entravée.

\textbf{Equipement Supplémentaire:} Vous transportez un sac d'explorateur avec une corde, des rations pour deux jours, un sac de couchage et d'autres outils nécessaires à la survie en plein air.

\textbf{Lien initial à la Première Aventure:} Choisissez parmi la liste des options ci-dessous comment vous vous êtes retrouvé impliqué dans la première aventure:

1. Contre votre bon jugement, vous avez rejoint les autres PJ parce que vous avez vu qu'ils étaient en danger.

2. L'un des autres PJ vous a convaincu que rejoindre le groupe serait dans votre intérêt.

3. Vous avez peur de ce qui pourrait arriver si les autres PC tombaient en panne.

4. Il y a une récompense en jeu et vous avez besoin d'argent.


%--------------------------
\label{sec:descresilient}\section*{Résistant}

\textcolor{gray}{\emph{Resilient}}

Vous pouvez subir de nombreuses épreuves, tant physiques que mentales, tout en en redemandant. Il en faut beaucoup pour vous rabaisser. Ni les chocs ni les dommages physiques ou mentaux n'ont d'effet durable. Vous êtes difficile à dissuader. Imperturbable. Inarrêtable.

\textbf{Résistant:} +2 à votre Réserve de Puissance, et +2 à votre Réserve d'Intellect.

\textbf{Récupération:} Vous pouvez effectuer un jet de récupération supplémentaire chaque jour. Ce jet n'est qu'une action. Vous pouvez donc faire deux jets de récupération qui demandent chacun une action, un jet qui prend dix minutes, un quatrième lancer qui prend une heure et un cinquième lancer qui nécessite dix heures de repos.

\textbf{Compétence:} Vous êtes entraîné aux tâches de défense de Puissance.

\textbf{Compétence:} Vous êtes entraîné aux tâches de défense d'Intellect.

\textbf{Inaptitude:} Vous êtes robuste mais pas nécessairement fort. Toute tâche impliquant de déplacer, plier ou casser des objets est désavantagée.

\textbf{Inaptitude:} Vous avez beaucoup de volonté et de force mentale, mais vous n'êtes pas nécessairement intelligent. Toute tâche impliquant des connaissances ou la résolution de problèmes ou d'énigmes est désavantagée.

\textbf{Lien initial à la Première Aventure:} Choisissez parmi la liste des options ci-dessous comment vous vous êtes retrouvé impliqué dans la première aventure:

1. Vous avez vu que les PJ ont clairement besoin de quelqu'un comme vous pour les aider.

2. Quelqu'un vous a demandé de surveiller un des PJ en particulier, et vous avez accepté.

3. Vous vous ennuyez et avez désespérément besoin de relever un défi.

4. Vous avez perdu un pari – injustement, pensez-vous – et avez dû prendre la place de quelqu'un dans cette mission.


%--------------------------
\label{sec:descdishonorable}\section*{Sans Honneur}

\textcolor{gray}{\emph{Dishonorable}}

Il n'y a pas d'honneur parmi les voleurs, ni les traîtres, les traîtres, les menteurs ou les tricheurs. Vous êtes toutes ces choses, et soit vous n'en perdez pas le sommeil, soit vous niez la vérité aux autres ou à vous-même. Quoi qu'il en soit, vous êtes prêt à faire tout ce qu'il faut pour parvenir à vos fins. L'honneur, l'éthique et les principes ne sont que des mots. À votre avis, ils n'ont pas leur place dans le monde réel.

\textbf{Sneaky:} +4 à votre Réserve de Célérité.

\textbf{Bien mérité:} Lorsque le MJ donne à un autre joueur un point d'expérience à attribuer à quelqu'un pour une intrusion du MJ, ce joueur ne peut pas vous le donner.

\textbf{Compétence:} Vous êtes entraîné à la tromperie.

\textbf{Compétence:} Vous êtes entraîné à la furtivité.

\textbf{Compétence:} Vous êtes entraîné à l'intimidation.

\textbf{Inaptitude:} Les gens ne vous aiment pas ou ne vous font pas confiance. Les interactions sociales agréables sont désavantagées.

\textbf{Lien initial à la Première Aventure:} Choisissez parmi la liste des options ci-dessous comment vous vous êtes retrouvé impliqué dans la première aventure:

1. Vous êtes intéressé par ce que font les PJ, alors vous leur avez menti pour entrer dans leur groupe.

2. En vous rôdant, vous avez entendu les plans des PJ et réalisé que vous souhaitiez entrer.

3. L'un des autres PJ vous a invité, n'ayant aucune idée de ce que vous êtes vraiment.

4. Vous avez fait preuve d'intimidation et de fanfaronnades pour vous frayer un chemin.


%--------------------------
\label{sec:descskeptical}\section*{Sceptique}

\textcolor{gray}{\emph{Skeptical}}

Vous possédez une attitude interrogative face à des affirmations qui sont souvent tenues pour acquises par les autres. Vous n'êtes pas nécessairement un « Thomas qui doute » (un sceptique qui refuse de croire quoi que ce soit sans expérience personnelle directe), mais vous avez souvent tiré profit de la remise en question des déclarations, des opinions et des connaissances reçues qui vous ont été présentées par d'autres.

\textbf{Insightful:} +2 à votre Réserve d'Intellect.

\textbf{Compétence:} Vous êtes entraîné dans l'identification.

\textbf{Compétence:} Vous êtes entraîné dans toutes les actions qui consistent à décrypter une ruse, une illusion, une ruse rhétorique destinée à éluder le problème, ou un mensonge. Par exemple, vous êtes plus doué pour garder un oeil sur la tasse contenant la balle cachée, ressentir une illusion ou réaliser si quelqu'un vous ment (mais seulement si vous vous concentrez spécifiquement et utilisez cette compétence).

\textbf{Lien initial à la Première Aventure:} Choisissez parmi la liste des options ci-dessous comment vous vous êtes retrouvé impliqué dans la première aventure:

1. Vous avez entendu d'autres PJ s'exprimer sur un sujet avec une opinion sur laquelle vous étiez plutôt sceptique, vous avez donc décidé d'approcher le groupe et de demander des preuves.

2. Vous suiviez l'un des autres PJ parce que vous vous méfiiez de lui, ce qui vous a amené à l'action.

3. Votre théorie sur la non-existence du surnaturel ne peut être invalidée que par vos propres sens, alors vous êtes arrivés.

4. Vous avez besoin d'argent pour financer vos recherches.


%--------------------------
\label{sec:descguarded}\section*{Suspicieux}

\textcolor{gray}{\emph{Guarded}}

Vous cachez votre vraie nature derrière un masque et répugnez à laisser quiconque voir qui vous êtes vraiment. Vous protéger, physiquement et émotionnellement, est ce qui vous tient le plus à coeur et vous préférez garder tout le monde à une distance de sécurité. Vous vous méfiez peut-être de tous ceux que vous rencontrez, vous attendant au pire de la part des gens pour ne pas être surpris lorsqu'ils vous donnent raison. Ou vous pourriez simplement être un peu réservé, en faisant attention à ne pas laisser les gens montrer votre extérieur bourru à la personne que vous êtes vraiment.Personne ne peut être aussi réservé que vous et se faire de nombreux amis. Très probablement, vous avez une personnalité abrasive et avez tendance à être pessimiste dans vos perspectives. Vous soignez probablement une vieille blessure et découvrez que la seule façon de la gérer est de la garder, ainsi que votre personnalité, verrouillées.

\textbf{Suspicieux:} +2 à votre Réserve d'Intellect.

\textbf{Compétence:} Vous êtes entraîné aux tâches de défense d'Intellect.

\textbf{Compétence:} Vous êtes entraîné à toutes les tâches impliquant de discerner la vérité, de percer les déguisements et de reconnaître les mensonges et autres tromperies.

\textbf{Inaptitude:} Votre nature méfiante vous rend désagréable. Toute tâche impliquant la tromperie ou la persuasion est désavantagée.

\textbf{Lien initial à la Première Aventure:} Choisissez parmi la liste des options ci-dessous comment vous vous êtes retrouvé impliqué dans la première aventure:

1. L'un des PJ a réussi à surmonter vos défenses et à se lier d'amitié avec vous.

2. Vous voulez voir ce que font les PJ, alors vous les accompagnez pour les surprendre en train de commettre des actes répréhensibles.

3. Vous vous êtes fait quelques ennemis et vous vous associez aux PJ pour vous protéger.

4. Les PJ sont les seuls à vous supporter.


%--------------------------
\label{sec:descvirtuous}\section*{Vertueux}

\textcolor{gray}{\emph{Virtuous}}

Faire ce qu'il faut est un mode de vie. Vous vivez selon un code, et ce code est quelque chose auquel vous êtes attentif chaque jour. A chaque fois que vous glissez, vous vous reprochez votre faiblesse et vous vous remettez aussitôt sur les rails. Votre code inclut probablement la modération, le respect des autres, la propreté et d'autres caractéristiques que la plupart des gens considéreraient comme des vertus, tandis que vous évitez leurs contraires : la paresse, l'avidité, la gourmandise, etc.

\textbf{Intrépide:} +2 à votre Réserve de Puissance.

\textbf{Compétence:} Vous êtes entraîné à discerner les véritables Focus des gens ou voir à travers les mensonges.

\textbf{Compétence:} Votre adhésion à un code moral strict a endurci votre esprit contre la peur, le doute et les influences extérieures. Vous êtes entraîné à des tâches de défense intellectuelle.

\textbf{Lien initial à la Première Aventure:} Choisissez parmi la liste des options ci-dessous comment vous vous êtes retrouvé impliqué dans la première aventure:

1. Les PJ font quelque chose de vertueux, et vous êtes tout à fait dans ce sens.

2. Les PJ sont sur le chemin de la perdition, et vous considérez que votre tâche est de les mettre sur la bonne voie morale.

3. Un des autres PJ vous a invité, ayant entendu parler de vos voies vertueuses.

4. Vous faites passer la vertu avant le sens et défendez l'honneur de quelqu'un face à une organisation ou un pouvoir bien plus grand que vous. Vous avez rejoint les PJ parce qu'ils vous offraient aide et amitié alors que, par peur des représailles, personne d'autre ne le faisait.


%--------------------------
\label{sec:descvicious}\section*{Vicieux}

\textcolor{gray}{\emph{Vicious}}

Vous essayez de cacher ce qu'il y a à l'intérieur, de le replier sur vous-même quand tout en vous crie de lâcher prise, de les faire payer, de les faire souffrir et de les faire saigner. Parfois, vous réussissez pour vos amis : souriez comme ils sourient, riez quand ils rient et parfois même ressentez vos propres émotions. Mais il est toujours là, ce sentiment de joie frénétique mêlé de haine qui jaillit parfois de vous lorsque vous affrontez un ennemi. La violence que vos amis peuvent tolérer, mais vous craignez parfois qu'ils apprennent également que vous êtes cruel.

\textbf{Compétence:} Vous êtes entraîné à traquer les créatures. Si une créature vous a fait du tort, la tâche de suivi est facilitée.

\textbf{Sanguinaire:} Une fois que vous commencez à vous battre, vous ne voyez que du rouge. Vous infligez 2 points de dégâts supplémentaires à chaque attaque.

\textbf{Berserk:} Une fois que vous commencez à vous battre, il est difficile pour vous de vous arrêter. En fait, c'est une tâche de difficulté 2 en Intellect de le faire, même si votre ennemi se rend ou si vous n'avez plus d'ennemis. Si ce dernier cas se produit et que vous ne parvenez pas à vous arrêter, vous attaquez l'allié le plus proche à courte portée.

\textbf{Equipement Supplémentaire:} Vous disposez d'un carnet que vous utilisez pour répertorier ceux qui vous ont fait du tort.

\textbf{Lien initial à la Première Aventure:} Choisissez parmi la liste des options ci-dessous comment vous vous êtes retrouvé impliqué dans la première aventure:

1. Un autre PJ vous a vu abattre un méchant ivrogne dans une taverne, sans se rendre compte que c'était vous qui aviez commencé le combat.

2. Vous vouliez vous éloigner d'une mauvaise situation, alors vous êtes allé avec les autres PJ.

3. Vous voulez changer et vous espérez qu'être avec les autres PJ vous aidera à vous calmer.

4. L'un des autres PJ vous a demandé de venir, pensant que votre méchanceté pourrait être exploitée au profit de la mission.



%#######################################################################
%            CHAPTER 8
%#######################################################################
\startchapter{Focus}{ch:chapter8}{CSCOLORPARTONE}
\raggedright
Le Focus est ce qui rend un personnage unique. Il n'y a pas deux PJ dans un groupe qui devraient avoir le même focus. Un Focus donne des avantages à un personnage lorsqu'il crée son personnage et à chaque fois qu'il passe au rang suivant. C'est le groupe verbal de la phrase "Je suis un *adjectif nom* qui *groupe verbal*".

Ce chapitre contient près d'une centaine d'exemples de Focus, tels que Se Revêt d'un Halo de Feu, Serait plutôt en train de lire et Pilote un Vaisseau Spatial. Ces Focus peuvent être choisies et utilisées telles que présentées par un joueur, ou par le MJ qui les ajoute à une liste de Focus disponibles pour ses joueurs lors de leur prochaine campagne.

De plus, la seconde moitié de ce chapitre fournit des outils permettant au MJ ou à un joueur entreprenant de créer ses propres Focus personnalisés qui correspondent parfaitement aux besoins d'un jeu ou d'une campagne donnée, comme présenté dans Création de nouveaux Focus.

\section*{Choisir un Focus}
Tous les Foci ne conviennent pas à tous les genres. Le chapitre Genre fournit des conseils, mais cette section propose quelques grandes généralisations. Évidemment, le MJ peut inclure toutes les Focus disponibles dans son environnement. Les Focus finissent par être une distinction importante dans ce cas, car Dispose de Pouvoirs Mentaux, par exemple, indique clairement que des capacités psychiques existent dans la campagne, tout comme Hurle à la Lune implique l'existence de lycanthropes comme les loups-garous, et Pilote des Vaisseaux Spaciaux, nécessite bien sûr des vaisseaux spatiaux disponibles pour piloter.

Lorsqu'un Focuss est choisi pour un personnage, celui-ci obtient une connexion spéciale avec un ou plusieurs de ses camarades PJ, une capacité de premier rang et peut-être un équipement de départ supplémentaire : une ou deux pièces d'équipement qui pourraient être nécessaires pour que le personnage puisse utiliser leur capacité, ou cela pourrait bien se marier avec le Focus. Par exemple, un personnage capable de construire des choses a besoin d'un ensemble d'outils. Un personnage constamment en feu a besoin d'un ensemble de vêtements insensibles aux flammes. Un personnage qui dessine des runes pour lancer des sorts a besoin d'outils d'écriture. Un personnage qui tue des monstres avec une épée a besoin d'une épée. Et ainsi de suite. Cela dit, de nombreuses Focus ne nécessitent pas d'équipement supplémentaire.

Chaque focus propose également une ou plusieurs suggestions --- intrusions GM --- pour les effets ou conséquences possibles de très bons ou de très mauvais lancers de dés.

Quelques Focus présentés dans ce chapitre fournissent une « Option d'échange de type » qui permet à un joueur d'échanger une capacité qui serait autrement acquise grâce à son type contre la capacité indiquée. Un joueur n'est pas obligé de procéder à l'échange; ils ont simplement le choix. Par exemple, le Focus Aime le Aime le Vide offre la possibilité d'acquérir la capacité Ayez une Combinaison Spatiale, Vous Voyagerez au lieu d'une capacité de type.

Au fur et à mesure qu'un personnage progresse vers un nouveau rang, un Focus confère plus de capacités. L'avantage de chaque rang est généralement appelé Action ou Enabler. Si une capacité est étiquetée Action, un personnage doit effectuer une action pour l'utiliser. Si une capacité est étiquetée Facilitateur, elle améliore d'autres actions ou donne un autre avantage, mais ce n'est pas une action. Une capacité qui permet à un personnage de faire exploser ses ennemis avec des lasers est une action. Une capacité qui accorde des dégâts supplémentaires lorsqu'une attaque est effectuée est un Facilitateur. Un Facilitateur est utilisé au même tour qu'une autre action, et souvent dans le cadre d'une autre action.

Les avantages de chaque rang sont indépendants et cumulatifs avec les avantages des autres rangs (sauf indication contraire). Ainsi, si une capacité de premier rang confère +1 à l'armure et qu'une capacité de quatrième rang confère également +1 à l'armure, lorsque le personnage atteint le quatrième rang, un total de +2 à l'armure est accordé.

Aux rangs 3 et 6, le personnage est invité à choisir une capacité parmi les deux options proposées.

Enfin, vous pouvez choisir si vous souhaitez développer l'histoire derrière le Focus (bien que ce ne soit pas obligatoire).


\section*{Connexions de Focus}
Choisissez une connexion qui va bien avec le Focus. Si vous êtes un MJ en train de choisir (ou de créer) un ou plusieurs Focus pour vos joueurs, choisissez jusqu'à quatre des connexions suivantes.
\begin{enumerate}
\item Choisissez un autre PJ. Pour des raisons que vous ne connaissez pas, ce personnage est totalement immunisé contre vos capacités de Focus, que vous les utilisiez pour vous aider ou pour vous nuire.
\item Choisissez un autre PJ. Vous connaissiez ce personnage il y a des années, mais vous ne pensez pas qu'il vous connaissait.
\item Choisissez un autre PJ. Vous essayez toujours de l'impressionner, mais vous ne savez pas pourquoi.
\item Choisissez un autre PJ. Ce personnage a une habitude qui vous ennuie, mais vous êtes par ailleurs assez impressionné par ses capacités.
\item Choisissez un autre PJ. Ce personnage montre du potentiel pour apprécier votre paradigme particulier, votre style de combat ou tout autre attribut fourni par le Focus. Vous aimeriez le former, mais vous n'êtes pas forcément qualifié pour enseigner (c'est à vous de décider), et il pourrait ne pas être intéressé (c'est à lui de décider).
\item Choisissez un autre PJ. S'il est à portée immédiate lorsque vous êtes en combat, il constitue parfois un atout, et parfois il gêne accidentellement vos jets d'attaque (50\% de chances dans tous les cas, déterminés par combat).
\item Choisissez un autre PJ. Vous lui avez déjà sauvé la vie et il se sente clairement redevable envers vous. Vous souhaiteriez qu'il ne le fasse pas; cela fait juste partie du travail.
\item Choisissez un autre PJ. Ce personnage s'est récemment moqué de vous d'une manière qui vous a vraiment blessé. La façon dont vous gérez cela (le cas échéant) dépend de vous.
\item Choisissez un autre PJ. Ce personnage sait que vous avez souffert aux mains d'entités robotiques dans le passé. C'est à vous de décider si vous détestez les robots maintenant, ce qui peut affecter votre relation avec le personnage s'il est amical avec les robots ou s'il possède des pièces robotiques.
\item Choisissez un autre PJ. Ce personnage vient du même endroit que vous et vous vous connaissez étant enfants.
\item Choisissez un autre PJ. Dans le passé, il vous enseignait quelques astuces à utiliser lors d'un combat.
\item Choisissez un autre PJ. Ce personnage ne semble pas approuver vos méthodes.
\item Choisissez un autre PJ. Il y a bien longtemps, vous étiez tous les deux dans des camps opposés dans un combat. Vous avez gagné, même si vous avez « triché » à ses yeux (mais de votre point de vue, tout est juste dans un combat). Il est peut-être prêt pour une revanche, mais cela dépend de lui.
\item Choisissez un autre PJ. Vous essayez toujours d'impressionner ce personnage avec vos compétences, votre esprit, votre apparence ou votre bravade. Peut-être qu'il est un rival, peut-être que vous avez besoin de son respect, ou peut-être que vous êtes intéressé de manière romantique par lui.
\item Choisissez un autre PJ. Vous craignez que ce personnage soit jaloux de vos capacités et craignez que cela puisse entraîner des problèmes.
\item Choisissez un autre PJ. Vous l'avez accidentellement attrapé dans un piège que vous aviez tendu et il a dû se libérer tout seul.
\item Choisissez un autre PJ. Vous avez déjà été embauché pour retrouver quelqu'un qui était proche de ce personnage.
\item Choisissez deux PJ (de préférence ceux qui sont susceptibles de gêner vos attaques). Lorsque vous ratez une attaque et que les règles de la MJ vous imposent de frapper quelqu'un d'autre que votre cible, vous touchez l'un de ces deux personnages.
\item Choisissez un autre PJ. Vous ne savez pas comment ni d'où, mais ce personnage a un plan pour les bouteilles d'alcool rare et peut vous les procurer à moitié prix.
\item Choisissez un autre PJ. Vous avez récemment perdu un bien et vous êtes convaincu qu'il l'a pris. Qu'il l'ait fait ou non, cela dépend de lui.
\item Choisissez un autre PJ. Il semble toujours savoir où vous êtes, ou du moins dans quelle direction vous vous situez par rapport à lui.
\item Choisissez un autre PJ. Vous voir utiliser vos capacités de concentration semble déclencher un souvenir désagréable chez ce personnage. Cette mémoire appartient à l'autre PJ, même s'il ne peut pas être en mesure de s'en souvenir consciemment.
\item Choisissez un autre PJ. Quelque chose chez lui interfère avec vos capacités. Lorsqu'il se trouve à côté de vous, vos capacités de concentration coûtent 1 point supplémentaire.
\item Choisissez un autre PJ. Quelque chose en eux complète vos capacités. Lorsqu'il se tient à côté de vous, la première capacité de concentration que vous utilisez sur une période de 24 heures coûte 2 points de moins.
\item Choisissez un autre PJ. Vous connaissez ce personnage depuis un certain temps et il vous a aidé à prendre le contrôle de vos capacités de concentration.
\item Choisissez un autre PJ. Dans le passé de ce personnage, il a vécu une expérience dévastatrice en tentant quelque chose que vous faites naturellement grâce à votre concentration. C'est à lui de décider s'il choisit de vous en parler.
\item Choisissez un autre PJ. Sa maladresse occasionnelle et son comportement bruyant vous irritent.
\item Choisissez un autre PJ. Dans un passé récent, alors que vous vous entraîniez, vous l'avez accidentellement frappé lors d'une attaque, le blessant grièvement. C'est à lui de décider s'il vous en veu ou s'il vous pardonne.
\item Choisissez un autre PJ. Il vous doit une somme d'argent importante.
\item Choisissez un autre PJ. Dans un passé récent, alors que vous échappiez à une menace, vous avez accidentellement laissé ce personnage se débrouiller tout seul. Il  a survécu, mais de justesse. C'est au joueur de ce personnage de décider s'il vous en veut ou s'il a décidé de vous pardonner.
\item Choisissez un autre PJ. Récemment, il vous a mis accidentellement (ou peut-être intentionnellement) dans une position de danger. Vous allez bien maintenant, mais vous vous méfiez de lui.
\item Choisissez un autre PJ. De votre point de vue, il semble nerveux face à une idée, une personne ou une situation spécifique. Vous aimeriez lui apprendre à être plus à l'aise avec ses peurs (s'il vous le permette).
\item Choisissez un autre PJ. Il vous a traité de lâche une fois.
\item Choisissez un autre PJ. Ce personnage vous reconnaît toujours, vous ou ce que vous produisez, que vous soyez déguisé ou disparu depuis longtemps lorsqu'il arrive sur les lieux.
\item Choisissez un autre PJ. Vous avez provoqué par inadvertance un accident qui l'a plongés dans un sommeil si profond qu'il ne se sont pas réveillé pendant trois jours. Qu'il vous pardonne ou non, c'est à lui de décider.
\item Choisissez un autre PJ. Vous êtes presque sûr d'avoir un lien de parenté d'une manière ou d'une autre.
\item Choisissez un autre PJ. Vous avez accidentellement appris quelque chose qu'il essayait de garder secret.
\item Choisissez un autre PJ. Il est particulièrement sensible à l'utilisation de vos capacités de Focus plus flashy et deviennent parfois éblouis pendant quelques rounds, ce qui gêne leurs actions.
\item Choisissez un autre PJ. Il semble posséder un objet précieux qui vous appartenait autrefois, mais que vous avez perdu à un jeu de hasard il y a des années.
\item Choisissez un autre PJ. Sans vous, ce personnage aurait échoué à un test de réussite mentale.
\item Choisissez un autre PJ. Sur la base de quelques commentaires que vous avez entendus, vous soupçonnez qu'il n'accorde pas la plus haute estime à votre domaine de formation ou à votre passe-temps favori.
\item Choisissez un autre PJ dont l'intérêt se confond avec le vôtre. Cette étrange connexion l'affecte d'une manière ou d'une autre. Par exemple, si le personnage utilise une arme, votre capacité de concentration améliore parfois son attaque d'une manière ou d'une autre.
\item Choisissez un autre PJ. Il souffre d'un vertige terrible. Vous aimeriez lui apprendre à être plus à l'aise en hauteur. Il doit décider d'accepter ou non votre offre.
\item Choisissez un autre PJ. Il est sceptique quant à vos affirmations sur quelque chose d'important qui s'est produit dans votre passé. Il pourrait même tenter de vous discréditer ou de découvrir le « secret » de votre histoire, mais cela dépend de lui.
\item Choisissez un autre PJ. Il a le don de reconnaître les points faibles de vos plans ou de vos projets.
\item Choisissez un autre PJ. Le visage de ce personnage vous intrigue tellement d'une manière que vous ne comprenez pas que vous vous retrouvez parfois à dessiner son image dans la terre ou sur un autre support auquel vous avez accès.
\item Choisissez un autre PJ. Ce personnage possède un élément supplémentaire de l'équipement régulier que vous lui avez donné, soit quelque chose que vous avez fabriqué, soit un objet que vous vouliez simplement lui donner. (Il choisit l'article.)
\item Choisissez un autre PJ. Il vous a chargé de faire un travail pour lui. Vous avez déjà été payé mais n'avez pas encore terminé le travail.
\item Choisissez un autre PJ. Vous avez travaillé ensemble dans le passé et le travail s'est mal terminé.
\item Choisissez un autre PJ. Pendant qu'il se tient à côté de vous et utilise leur action pour se concentrer sur votre aide, la portée de l'une de vos capacités de concentration est doublée.
\end{enumerate}

\section*{L'Histoire derrière les Focus}
Les Focus de ce livre ont été volontairement réduites à l'essentiel afin d'avoir l'application la plus large possible dans plusieurs genres. Une ou deux phrases descriptives résument chacune d'entre elles. Après avoir choisi un Focus, vous avez la possibilité d'élargir sa présentation en ajoutant plus d'histoire et de description pertinentes pour le monde ou pour le personnage.

Par exemple, si vous choisissez Opère sous Couverture, la description récapitulative est "Sous l'apparence de quelqu'un d'autre, vous cherchez à trouver des réponses que les puissants ne veulent pas divulguer". Si vous choisissez Poursuit des Sciences Etranges, le résumé est le suivant: "Votre perspicacité et vos capacités surnaturelles font de vous un scientifique capable de prouesses incroyables." Ces descriptions fournissent ce que vous devez savoir pour utiliser le focus.

Cependant, si vous le souhaitez (et *uniquement* si vous le souhaitez ; rien n'est obligatoire), vous pouvez ajouter davantage à ces descriptions d'une manière pertinente pour votre jeu. Par exemple, si vous choisissez Opère sous Couverture et Poursuit des Sciences Etranges pour une utilisation dans un genre moderne tel que l'horreur, la fantasy urbaine, l'espionnage ou quelque chose de similaire, vous pouvez développer les descriptions comme indiqué dans les exemples suivants.

**Opère sous Couverture:** L'espionnage n'est pas quelque chose dont vous savez rien. Du moins, c'est ce que vous voulez faire croire à tout le monde, car en vérité, vous avez été formé comme espion ou agent secret. Vous pourriez travailler pour un gouvernement ou pour vous-même. Vous pourriez être un détective de police ou un criminel. Vous pourriez même devenir journaliste d'investigation.

Quoi qu'il en soit, vous apprenez des informations que d'autres tentent de garder secrètes. Vous collectez des rumeurs et des chuchotements, des histoires et des preuves durement acquises, et vous utilisez ces connaissances pour vous aider dans vos propres efforts et, le cas échéant, pour fournir à vos employeurs les informations qu'ils souhaitent. Alternativement, vous pouvez vendre ce que vous avez appris à ceux qui sont prêts à payer plus cher.

Vous portez probablement des couleurs sombres (noir, gris anthracite ou bleu nuit) pour vous fondre dans l'ombre, à moins que la couverture que vous avez choisie ne vous oblige à ressembler à quelqu'un d'autre.

**Poursuit des Sciences Etranges:** Vous pourriez être un scientifique respecté, ayant publié dans plusieurs revues à comité de lecture. Ou bien vous pourriez être considéré comme un excentrique par vos contemporains, poursuivant des théories marginales sur ce que d'autres considèrent comme peu de preuves. La vérité est que vous avez un don particulier pour passer au crible ce qui est possible. Vous pouvez trouver de nouvelles informations et débloquer des phénomènes étranges grâce à vos expériences. Là où d'autres voient une corne d'abondance farfelue, vous passez au crible les théories du complot à la recherche de révélations. Que vous meniez vos enquêtes en tant qu'entrepreneur gouvernemental, chercheur universitaire, scientifique d'entreprise ou curieux dans votre propre laboratoire en suivant votre muse, vous repoussez les limites de ce qui est possible.

Vous vous souciez probablement plus de votre travail que de trivialités telles que votre apparence, votre comportement poli ou approprié, ou les normes sociales, mais là encore, un excentrique comme vous pourrait également renverser la situation sur ce stéréotype.

Si vous souhaitez aller encore plus loin, vous pouvez déterminer d'où viennent les capacités de concentration d'un personnage. Selon le genre, ils pourraient tirer ces capacités d'un entraînement avancé et persistant, via des runes magiques, des éléments cybernétiques, de leur héritage génétique ou de leur accès à une technologie avancée. Par exemple, un personnage pourrait être capable de faire exploser des cibles avec des éclairs parce qu'elles ont été zappées par un rayonnement étrange ou parce qu'elles ont ramassé un pistolet éclair. D'un autre côté, c'est peut-être parce que leur entraînement intense leur a permis d'apprendre la magie de la foudre. Les possibilités sont presque infinies et c'est à vous de les inclure ou de les renoncer. Parce que quelle que soit la manière dont les capacités d'un focus ont été acquises, il suffit également qu'elles fonctionnent.

\section*{Focus}
La description complète de chaque capacité de concentration répertoriée dans cette section se trouve dans le chapitre Capacités, qui contient des descriptions du type, de la saveur et des capacités de concentration dans un seul et vaste catalogue.
%____________________________________________________________________---------------
%%%%%%%%%%%%%%%%%%%%%%%%%%%%%%%%%%%%%%%%%%%%%%%%%%%%%%%%%%%%%%%%%%%%%%%
%_____________________________________________________%
\phantomsection\label{sec:focusplaystoomanygames}\section*{ A Joué à Trop de Jeux}

\textcolor{gray}{\emph{ Plays Too Many Games}}

Les leçons, les réflexes et les stratégies que vous avez appris en jouant à trop de jeux ont des applications dans le monde réel, où les gens qui ne jouent pas assez travaillent dur et vivent leur vie morne.

\begin{abnamelist}
\item Rang 1: Joueur (\pageref{subsec:ab_gamer}) Leçons de jeu (\pageref{subsec:ab_game_lessons})
\item Rang 2:
Résoudre des Enigmes (\pageref{subsec:ab_resist_tricks}) Voir dans le Noir (\pageref{subsec:ab_zero_dark_eyes})
\item Rang 3:
Choisissez en une: Avantage de Célérité Amélioré (\pageref{subsec:ab_enhanced_speed_edge}) ou Objectif du tireur d'élite (\pageref{subsec:ab_snipers_aim})
\item Rang 4:
Intellect amélioré (\pageref{subsec:ab_enhanced_intellect}) Jeux d'esprit (\pageref{subsec:ab_mind_games})
\item Rang 5:
Endurance du joueur (\pageref{subsec:ab_gamers_fortitude})
\item Rang 6:
Choisissez en une: Dieu du jeu (\pageref{subsec:ab_gaming_god}) ou Sursaut mental (\pageref{subsec:ab_mind_surge})
\end{abnamelist}
\textbf{Intrusion de la Meneuse:}
%_____________________________________________________%
\phantomsection\label{sec:focussailedbeneaththejollyroger}\section*{ A Navigué sous Pavillon Pirate}

\textcolor{gray}{\emph{ Sailed Beneath the Jolly Roger}}

Vous avez navigué avec un équipage de redoutables pirates, mais vous avez décidé de mettre fin à vos jours de pirate et de rejoindre une autre cause. La question est : votre passé vous laissera-t-il partir si facilement ?

\begin{abnamelist}
\item Rang 1: Ignorez la Douleur (\pageref{subsec:ab_ignore_the_pain}) Marin (\pageref{subsec:ab_sailor})
\item Rang 2:
Prendre l'avantage (\pageref{subsec:ab_taking_advantage}) Réputation redoutable (\pageref{subsec:ab_fearsome_reputation})
\item Rang 3:
Choisissez en une: Compétence avec les attaques (\pageref{subsec:ab_skill_with_attacks}) ou Compétence en Défense Supérieure (\pageref{subsec:ab_greater_skill_with_defense})
\item Rang 4:
Habiletés motrices (\pageref{subsec:ab_movement_skills}) Le pied marin (\pageref{subsec:ab_sea_legs})
\item Rang 5:
Perdu dans le chaos (\pageref{subsec:ab_lost_in_the_chaos})
\item Rang 6:
Choisissez en une: Attaque successive (\pageref{subsec:ab_successive_attack}) ou Duel à mort (\pageref{subsec:ab_duel_to_the_death})
\end{abnamelist}
\textbf{Intrusion de la Meneuse:}
%_____________________________________________________%
\phantomsection\label{sec:focusdescendsfromnobility}\section*{ A des Ascendants Nobles}

\textcolor{gray}{\emph{ Descends From Nobility}}

Descendant de la richesse et du pouvoir, vous portez un titre noble et les capacités conférées par une éducation privilégiée. \newline\textbf{Option d'échange de type:} Suite de Servants

\begin{abnamelist}
\item Rang 1: Noblesse privilégiée (\pageref{subsec:ab_privileged_nobility})
\item Rang 2:
Interlocuteur qualifié (\pageref{subsec:ab_trained_interlocutor})
\item Rang 3:
Choisissez en une: Commandement avancé (\pageref{subsec:ab_advanced_command}) ou Courage du noble (\pageref{subsec:ab_nobles_courage})
\item Rang 4:
Disciple expert (\pageref{subsec:ab_expert_follower})
\item Rang 5:
Affirmer votre privilège (\pageref{subsec:ab_asserting_your_privilege})
\item Rang 6:
Choisissez en une: Assistance compétente (\pageref{subsec:ab_able_assistance}) ou Esprit de leader (\pageref{subsec:ab_mind_of_a_leader})
\end{abnamelist}
\textbf{Intrusion de la Meneuse:}
%_____________________________________________________%
\phantomsection\label{sec:focusislicensedtocarry}\section*{ A le Droit de Porter une Arme à Feu}

\textcolor{gray}{\emph{ Is Licensed to Carry}}

Vous portez une arme à feu et vous savez comment l'utiliser lors d'un combat.  Bien que Is Licensed to Carry soit conçu pour les armes à feu modernes, il pourrait s'appliquer aux armes à silex, aux blasters laser futuristes ou à d'autres armes à distance. 

\begin{abnamelist}
\item Rang 1: Pratique des armes à feu (\pageref{subsec:ab_practiced_with_guns}) Tireur (\pageref{subsec:ab_gunner})
\item Rang 2:
Tir prudent (\pageref{subsec:ab_careful_shot})
\item Rang 3:
Choisissez en une: Augmente les dommages (\pageref{subsec:ab_damage_dealer}) ou Tireur entraîné (\pageref{subsec:ab_trained_gunner})
\item Rang 4:
Tir Rapide (\pageref{subsec:ab_snap_shot})
\item Rang 5:
Tirs en éventail (\pageref{subsec:ab_arc_spray})
\item Rang 6:
Choisissez en une: Dégâts mortels (\pageref{subsec:ab_lethal_damage}) ou Tir Spécial (\pageref{subsec:ab_special_shot})
\end{abnamelist}
\textbf{Intrusion de la Meneuse:}
%_____________________________________________________%
\phantomsection\label{sec:focushasathousandfaces}\section*{ A un Millier de Visages}

\textcolor{gray}{\emph{ Has A Thousand Faces}}

Vous pouvez changer votre apparence pour ressembler à n’importe qui d’autre.

\begin{abnamelist}
\item Rang 1: Changement de Visage (\pageref{subsec:ab_face_morph}) Compétences d'interaction (\pageref{subsec:ab_interaction_skills})
\item Rang 2:
Altération corporelle (\pageref{subsec:ab_body_morph}) Chair de guerre (\pageref{subsec:ab_war_flesh})
\item Rang 3:
Choisissez en une: Déguiser un autre (\pageref{subsec:ab_disguise_other}) ou Résilience (\pageref{subsec:ab_resilience})
\item Rang 4:
Pensez à votre sortie (\pageref{subsec:ab_think_your_way_out}) Sans âge (\pageref{subsec:ab_ageless})
\item Rang 5:
La mémoire devient une action (\pageref{subsec:ab_memory_becomes_action})
\item Rang 6:
Choisissez en une: Divisez votre esprit (\pageref{subsec:ab_divide_your_mind}) ou Déduire des pensées (\pageref{subsec:ab_infer_thoughts})
\end{abnamelist}
\textbf{Intrusion de la Meneuse:}
%_____________________________________________________%
\phantomsection\label{sec:focuswasforetold}\section*{ A été Choisi(e)}

\textcolor{gray}{\emph{ Was Foretold}}

Vous êtes « l’élu » et la prophétie, la prédiction, le pronostic ou toute autre méthode de détermination attend de vous de grandes choses un jour.

\begin{abnamelist}
\item Rang 1: Compétences d'interaction (\pageref{subsec:ab_interaction_skills}) Connaître (\pageref{subsec:ab_knowing})
\item Rang 2:
Destiné à la grandeur (\pageref{subsec:ab_destined_for_greatness})
\item Rang 3:
Choisissez en une: Résilience durement gagnée (\pageref{subsec:ab_hard_won_resilience}) ou Surmontez tous les obstacles (\pageref{subsec:ab_overcome_all_obstacles})
\item Rang 4:
Centre d'attention (\pageref{subsec:ab_center_of_attention})
\item Rang 5:
Montrez-leur le chemin (\pageref{subsec:ab_show_them_the_way})
\item Rang 6:
Choisissez en une: Comme le prédit la prophétie (\pageref{subsec:ab_as_foretold_in_prophecy}) ou Potentiel amélioré plus important (\pageref{subsec:ab_greater_enhanced_potential})
\end{abnamelist}
\textbf{Intrusion de la Meneuse:}
%_____________________________________________________%
\phantomsection\label{sec:focusabsorbsenergy}\section*{ Absorbe l'Energie}

\textcolor{gray}{\emph{ Absorbs Energy}}

Vous pouvez exploiter l’énergie cinétique et la transformer en d’autres types d’énergie.

\begin{abnamelist}
\item Rang 1: Absorber l'énergie cinétique (\pageref{subsec:ab_absorb_kinetic_energy}) Libération d'énergie (\pageref{subsec:ab_release_energy})
\item Rang 2:
Dynamiser un objet (\pageref{subsec:ab_energize_object})
\item Rang 3:
Choisissez en une: Absorber l'énergie pure (\pageref{subsec:ab_absorb_pure_energy}) ou Absorption d'énergie cinétique améliorée (\pageref{subsec:ab_improved_absorb_kinetic_energy})
\item Rang 4:
Surcharge d'énergie (\pageref{subsec:ab_overcharge_energy})
\item Rang 5:
Energiser la créature (\pageref{subsec:ab_energize_creature})
\item Rang 6:
Choisissez en une: Energiser la foule (\pageref{subsec:ab_energize_crowd}) ou Surcharge d'Appareil (\pageref{subsec:ab_overcharge_device})
\end{abnamelist}
\textbf{Intrusion de la Meneuse:}
%_____________________________________________________%
\phantomsection\label{sec:focusperformsfeatsofstrength}\section*{ Accompli des Prouesses de Force}

\textcolor{gray}{\emph{ Performs Feats of Strength}}

Prodige en musculature, vous pouvez transporter un poids incroyable, projeter votre corps dans les airs et percer des portes.

\begin{abnamelist}
\item Rang 1: Athlète (\pageref{subsec:ab_athlete}) Avantage de Puissance Amélioré (\pageref{subsec:ab_enhanced_might_edge})
\item Rang 2:
Tour de force (\pageref{subsec:ab_feat_of_strength})
\item Rang 3:
Choisissez en une: Lancer (\pageref{subsec:ab_throw}) ou Poing de Fer (\pageref{subsec:ab_iron_fist})
\item Rang 4:
Puissance améliorée supérieure (\pageref{subsec:ab_greater_enhanced_might})
\item Rang 5:
Coup brutal (\pageref{subsec:ab_brute_strike})
\item Rang 6:
Choisissez en une: Attaque sautée (\pageref{subsec:ab_jump_attack}) ou Puissance améliorée supérieure (\pageref{subsec:ab_greater_enhanced_might})
\end{abnamelist}
\textbf{Intrusion de la Meneuse:}
%_____________________________________________________%
\phantomsection\label{sec:focushelpstheirfriends}\section*{ Aide ses Amis}

\textcolor{gray}{\emph{ Helps Their Friends}}

Vous aimez vos amis et les aidez à sortir de toute difficulté, quoi qu'il arrive. \newline\textbf{Option d'échange de type:} Conseils d'un ami

\begin{abnamelist}
\item Rang 1: Aide amicale (\pageref{subsec:ab_friendly_help}) Courageux (\pageref{subsec:ab_courageous})
\item Rang 2:
Faites face aux vicissitudes (\pageref{subsec:ab_weather_the_vicissitudes})
\item Rang 3:
Choisissez en une: Compétence avec les attaques (\pageref{subsec:ab_skill_with_attacks}) ou Copain (\pageref{subsec:ab_buddy_system})
\item Rang 4:
En Danger (\pageref{subsec:ab_in_harms_way}) Physique amélioré (\pageref{subsec:ab_enhanced_physique})
\item Rang 5:
Inspirer l'action (\pageref{subsec:ab_inspire_action})
\item Rang 6:
Choisissez en une: Compétence en défense (\pageref{subsec:ab_skill_with_defense}) ou Considération approfondie (\pageref{subsec:ab_deep_consideration})
\end{abnamelist}
\textbf{Intrusion de la Meneuse:}
%_____________________________________________________%
\phantomsection\label{sec:focuslovesthevoid}\section*{ Aime le Vide}

\textcolor{gray}{\emph{ Loves the Void}}

Lorsqu'il n'y a que vous, votre combinaison spatiale et le panorama d'étoiles qui défilent pour toujours et à jamais, vous êtes en paix. \newline\textbf{Option d'échange de type:} Ayez une Combinaison Spatiale, Vous Voyagerez

\begin{abnamelist}
\item Rang 1: Adepte de la microgravité (\pageref{subsec:ab_microgravity_adept}) Compétences sous Vide Spatial (\pageref{subsec:ab_vacuum_skilled})
\item Rang 2:
Avantage de Célérité Amélioré (\pageref{subsec:ab_enhanced_speed_edge}) Physique amélioré (\pageref{subsec:ab_enhanced_physique})
\item Rang 3:
Choisissez en une: Armure Corporelle (\pageref{subsec:ab_fusion_armor}) ou Combat spatial (\pageref{subsec:ab_space_fighting})
\item Rang 4:
Sauter en Microgravité (\pageref{subsec:ab_push_off_and_throw}) Silencieux comme l'espace (\pageref{subsec:ab_silent_as_space})
\item Rang 5:
Evitement par microgravité (\pageref{subsec:ab_microgravity_avoidance})
\item Rang 6:
Choisissez en une: Champ réactif (\pageref{subsec:ab_reactive_field}) ou Tir en apesanteur (\pageref{subsec:ab_weightless_shot})
\end{abnamelist}
\textbf{Intrusion de la Meneuse:}
%_____________________________________________________%
\phantomsection\label{sec:focuslearnsquickly}\section*{ Apprend Rapidement}

\textcolor{gray}{\emph{ Learns Quickly}}

Vous faites face aux mauvaises situations au fur et à mesure qu’elles surviennent, apprenant à chaque fois de nouvelles leçons.

\begin{abnamelist}
\item Rang 1: Intellect amélioré (\pageref{subsec:ab_enhanced_intellect}) Voilà votre problème (\pageref{subsec:ab_theres_your_problem})
\item Rang 2:
Etude rapide (\pageref{subsec:ab_quick_study})
\item Rang 3:
Choisissez en une: Difficile à distraire (\pageref{subsec:ab_hard_to_distract}) Avantage d'Intellect Amélioré (\pageref{subsec:ab_enhanced_intellect_edge}) ou Compétences en Gage (\pageref{subsec:ab_flex_skill})
\item Rang 4:
Passer l'information au suivant (\pageref{subsec:ab_pay_it_forward})
\item Rang 5:
Appris des trucs (\pageref{subsec:ab_learned_a_few_things}) Intellect amélioré (\pageref{subsec:ab_enhanced_intellect})
\item Rang 6:
Choisissez en une: Compétence en Défense Supérieure (\pageref{subsec:ab_greater_skill_with_defense}) ou Deux choses à la fois (\pageref{subsec:ab_two_things_at_once})
\end{abnamelist}
\textbf{Intrusion de la Meneuse:}
%_____________________________________________________%
\phantomsection\label{sec:focusmurders}\section*{ Assassine}

\textcolor{gray}{\emph{ Murders}}

Vous êtes un assassin, que ce soit par métier, par inclination, ou parce que vous vouliez vous faire tuer. (Quelqu'un qui Assassine peut disposer d'un équipement supplémentaire, notamment trois doses d'un poison de lame de niveau 2 qui inflige 5 points de dégâts.)

\begin{abnamelist}
\item Rang 1: Attaque surprise (\pageref{subsec:ab_surprise_attack}) Compétences d'assassin (\pageref{subsec:ab_assassin_skills})
\item Rang 2:
Infiltrateur (\pageref{subsec:ab_infiltrator}) Mort rapide (\pageref{subsec:ab_quick_death})
\item Rang 3:
Choisissez en une: Artisan de Poisons (\pageref{subsec:ab_poison_crafter}) ou Conscience (\pageref{subsec:ab_awareness})
\item Rang 4:
Attaque Surprise Améliorée (\pageref{subsec:ab_better_surprise_attack})
\item Rang 5:
Augmente les dommages (\pageref{subsec:ab_damage_dealer})
\item Rang 6:
Choisissez en une: Meurtrier (\pageref{subsec:ab_murderer}) ou Plan d'évasion (\pageref{subsec:ab_escape_plan})
\end{abnamelist}
\textbf{Intrusion de la Meneuse:}
%_____________________________________________________%
\phantomsection\label{sec:focusmoveslikeacat}\section*{ Bouge comme un Chat}

\textcolor{gray}{\emph{ Moves Like a Cat}}

Souple, flexible et gracieux, vous vous déplacez rapidement et en douceur, et ne semblez jamais être là où se trouve le danger.

\begin{abnamelist}
\item Rang 1: Célérité améliorée supérieure (\pageref{subsec:ab_greater_enhanced_speed}) Equilibre (\pageref{subsec:ab_balance})
\item Rang 2:
Chute en toute sécurité (\pageref{subsec:ab_safe_fall}) Habiletés motrices (\pageref{subsec:ab_movement_skills})
\item Rang 3:
Choisissez en une: Difficile à toucher (\pageref{subsec:ab_hard_to_hit}) Avantage de Célérité Amélioré (\pageref{subsec:ab_enhanced_speed_edge}) ou Célérité améliorée supérieure (\pageref{subsec:ab_greater_enhanced_speed})
\item Rang 4:
Frappe rapide (\pageref{subsec:ab_quick_strike})
\item Rang 5:
Glissant (\pageref{subsec:ab_slippery})
\item Rang 6:
Choisissez en une: Célérité améliorée supérieure (\pageref{subsec:ab_greater_enhanced_speed}) ou Sursaut de Célérité Parfait (\pageref{subsec:ab_perfect_speed_burst})
\end{abnamelist}
\textbf{Intrusion de la Meneuse:}
%_____________________________________________________%
\phantomsection\label{sec:focusbrandishesanexoticshield}\section*{ Brandit un Bouclier Exotique}

\textcolor{gray}{\emph{ Brandishes an Exotic Shield}}

Vous déployez un incroyable bouclier de force pure qui offre une protection et des options offensives.

\begin{abnamelist}
\item Rang 1: Bouclier de Champ de Force (\pageref{subsec:ab_force_field_shield}) Frappe de Force (\pageref{subsec:ab_force_bash})
\item Rang 2:
Bouclier enveloppant (\pageref{subsec:ab_enveloping_shield})
\item Rang 3:
Choisissez en une: Lancer un bouclier de force (\pageref{subsec:ab_throw_force_shield}) ou Pulsation de Guérison (\pageref{subsec:ab_healing_pulse})
\item Rang 4:
Bouclier énergisé (\pageref{subsec:ab_energized_shield})
\item Rang 5:
Mur de Force (\pageref{subsec:ab_force_wall})
\item Rang 6:
Choisissez en une: Bouclier Explosif (\pageref{subsec:ab_shield_burst}) ou Bouclier rebondissant (\pageref{subsec:ab_bouncing_shield})
\end{abnamelist}
\textbf{Intrusion de la Meneuse:}
%_____________________________________________________%
\phantomsection\label{sec:focuscalculatestheincalculable}\section*{ Calcule l'Incalculable}

\textcolor{gray}{\emph{ Calculates the Incalculable}}

Des capacités mathématiques impressionnantes vous permettent de modéliser le monde en temps réel, vous donnant ainsi un avantage sur tout le monde.

\begin{abnamelist}
\item Rang 1: Equation prédictive (\pageref{subsec:ab_predictive_equation}) Mathématiques supérieures (\pageref{subsec:ab_higher_mathematics})
\item Rang 2:
Modèle prédictif (\pageref{subsec:ab_predictive_model})
\item Rang 3:
Choisissez en une: Défense subconsciente (\pageref{subsec:ab_subconscious_defense}) ou Intellect amélioré (\pageref{subsec:ab_enhanced_intellect})
\item Rang 4:
Diagramme de combat (\pageref{subsec:ab_cognizant_offense})
\item Rang 5:
Intellect Amélioré Supérieur (\pageref{subsec:ab_greater_enhanced_intellect}) Mathématiques Complémentaires (\pageref{subsec:ab_further_mathematics})
\item Rang 6:
Choisissez en une: Connaître l'inconnu (\pageref{subsec:ab_knowing_the_unknown}) ou Intellect Amélioré Supérieur (\pageref{subsec:ab_greater_enhanced_intellect})
\end{abnamelist}
\textbf{Intrusion de la Meneuse:}
%_____________________________________________________%
\phantomsection\label{sec:focuschannelsdivineblessings}\section*{ Canalise les Bénédictions Divines}

\textcolor{gray}{\emph{ Channels Divine Blessings}}

Fervent disciple d’un être divin, vous canalisez une partie du pouvoir de votre divinité pour réaliser des merveilles.

\begin{abnamelist}
\item Rang 1: Bénédiction des Dieux (\pageref{subsec:ab_blessing_of_the_gods})
\item Rang 2:
Intellect amélioré (\pageref{subsec:ab_enhanced_intellect})
\item Rang 3:
Choisissez en une: Cube de flammes (\pageref{subsec:ab_fire_bloom}) ou Radiance divine (\pageref{subsec:ab_divine_radiance})
\item Rang 4:
Explosion Divine (\pageref{subsec:ab_overawe})
\item Rang 5:
Intervention divine (\pageref{subsec:ab_divine_intervention})
\item Rang 6:
Choisissez en une: Invoquer un démon (\pageref{subsec:ab_summon_demon}) ou Symbole divin (\pageref{subsec:ab_divine_symbol})
\end{abnamelist}
\textbf{Intrusion de la Meneuse:}
%_____________________________________________________%
\phantomsection\label{sec:focus}\section*{ Chasse}

\textcolor{gray}{\emph{ }}

Vous êtes un chasseur traquant qui excelle à abattre la proie que vous avez choisie.

\begin{abnamelist}
\item Rang 1: Attaque avec style (\pageref{subsec:ab_attack_flourish}) Pisteur (\pageref{subsec:ab_tracker})
\item Rang 2:
Furtif (\pageref{subsec:ab_sneak}) Proie (\pageref{subsec:ab_quarry})
\item Rang 3:
Choisissez en une: Combats de Horde (\pageref{subsec:ab_horde_fighting}) ou Court et Attrape (\pageref{subsec:ab_sprint_and_grab})
\item Rang 4:
Attaque surprise (\pageref{subsec:ab_surprise_attack})
\item Rang 5:
Volonté du Chasseur (\pageref{subsec:ab_hunters_drive})
\item Rang 6:
Choisissez en une: Compétence en Attaque Supérieure (\pageref{subsec:ab_greater_skill_with_attacks}) ou Multiples Proies (\pageref{subsec:ab_multiple_quarry})
\end{abnamelist}
\textbf{Intrusion de la Meneuse:}
%_____________________________________________________%
\phantomsection\label{sec:focuslooksfortrouble}\section*{ Cherche les Ennuis}

\textcolor{gray}{\emph{ Looks for Trouble}}

Vous êtes un bagarreur et aimez les bons combats.

\begin{abnamelist}
\item Rang 1: Poings de fureur (\pageref{subsec:ab_fists_of_fury}) Premier Soins (\pageref{subsec:ab_wound_tender})
\item Rang 2:
Protecteur (\pageref{subsec:ab_protector}) Simple et Direct (\pageref{subsec:ab_straightforward})
\item Rang 3:
Choisissez en une: Compétence avec les attaques (\pageref{subsec:ab_skill_with_attacks}) ou Potentiel amélioré plus important (\pageref{subsec:ab_greater_enhanced_potential})
\item Rang 4:
Knock Out (\pageref{subsec:ab_knock_out})
\item Rang 5:
Maîtrise des attaques (\pageref{subsec:ab_mastery_with_attacks})
\item Rang 6:
Choisissez en une: Dégâts mortels (\pageref{subsec:ab_lethal_damage}) ou Puissance améliorée supérieure (\pageref{subsec:ab_greater_enhanced_might})
\end{abnamelist}
\textbf{Intrusion de la Meneuse:}
%_____________________________________________________%
\phantomsection\label{sec:focusfightswithpanache}\section*{ Combat avec Panache}

\textcolor{gray}{\emph{ Fights With Panache}}

Vous êtes un casse-cou audacieux qui se bat avec un style flamboyant et amusant à regarder.

\begin{abnamelist}
\item Rang 1: Attaque avec style (\pageref{subsec:ab_attack_flourish})
\item Rang 2:
Blocage rapide (\pageref{subsec:ab_quick_block})
\item Rang 3:
Choisissez en une: Attaque acrobatique (\pageref{subsec:ab_acrobatic_attack}) ou Vantardise flamboyante (\pageref{subsec:ab_flamboyant_boast})
\item Rang 4:
Bloquer pour un autre (\pageref{subsec:ab_block_for_another}) Meurtre Rapide (\pageref{subsec:ab_fast_kill})
\item Rang 5:
Utilisation de l'environnement (\pageref{subsec:ab_using_the_environment})
\item Rang 6:
Choisissez en une: Esprit Agile (\pageref{subsec:ab_agile_wit}) ou Retour à l'expéditeur (\pageref{subsec:ab_return_to_sender})
\end{abnamelist}
\textbf{Intrusion de la Meneuse:}
%_____________________________________________________%
\phantomsection\label{sec:focusbattlesrobots}\section*{ Combat les Robots}

\textcolor{gray}{\emph{ Battles Robots}}

Vous excellez dans la lutte contre les robots, les automates et les entités machines.

\begin{abnamelist}
\item Rang 1: Compétences techniques (\pageref{subsec:ab_tech_skills}) Vulnérabilités des machines (\pageref{subsec:ab_machine_vulnerabilities})
\item Rang 2:
Chasse aux machines (\pageref{subsec:ab_machine_hunting}) Défense contre les robots (\pageref{subsec:ab_defense_against_robots})
\item Rang 3:
Choisissez en une: Attaque surprise (\pageref{subsec:ab_surprise_attack}) ou Mécanismes de désactivation (\pageref{subsec:ab_disable_mechanisms})
\item Rang 4:
Combattant de Robot (\pageref{subsec:ab_robot_fighter})
\item Rang 5:
Drain de Puissance (\pageref{subsec:ab_drain_power/})
\item Rang 6:
Choisissez en une: Dégâts mortels (\pageref{subsec:ab_lethal_damage}) ou Désactiver les mécanismes (\pageref{subsec:ab_deactivate_mechanisms})
\end{abnamelist}
\textbf{Intrusion de la Meneuse:}
%_____________________________________________________%
\phantomsection\label{sec:focuscommandsmentalpowers}\section*{ Commande aux pouvoirs Mentaux}

\textcolor{gray}{\emph{ Commands Mental Powers}}

Vous avez perfectionné le pouvoir de votre esprit pour accomplir des actes psychiques incroyables.

\begin{abnamelist}
\item Rang 1: Télépathique (\pageref{subsec:ab_telepathic})
\item Rang 2:
Lecture mentale (\pageref{subsec:ab_mind_reading})
\item Rang 3:
Choisissez en une: Projection Psychique (\pageref{subsec:ab_psychic_burst}) ou Suggestion Psychique (\pageref{subsec:ab_psychic_suggestion})
\item Rang 4:
Utiliser les sens des autres (\pageref{subsec:ab_use_senses_of_others})
\item Rang 5:
Précognition (\pageref{subsec:ab_precognition})
\item Rang 6:
Choisissez en une: Contrôle mental (\pageref{subsec:ab_mind_control}) ou Réseau télépathique (\pageref{subsec:ab_telepathic_network})
\end{abnamelist}
\textbf{Intrusion de la Meneuse:}
%_____________________________________________________%
\phantomsection\label{sec:focusfocusesmindovermatter}\section*{ Concentre l'Esprit sur la Matière}

\textcolor{gray}{\emph{ Focuses Mind Over Matter}}

Vous pouvez déplacer des objets par télékinésie avec votre esprit sans les toucher physiquement.

\begin{abnamelist}
\item Rang 1: Détourner les attaques (\pageref{subsec:ab_divert_attacks})
\item Rang 2:
Télékinésie (\pageref{subsec:ab_telekinesis})
\item Rang 3:
Choisissez en une: Améliorer la force (\pageref{subsec:ab_enhance_strength}) ou Manteau d'opportunité (\pageref{subsec:ab_cloak_of_opportunity})
\item Rang 4:
Ramener (\pageref{subsec:ab_apportation})
\item Rang 5:
Attaque psychokinétique (\pageref{subsec:ab_psychokinetic_attack})
\item Rang 6:
Choisissez en une: Ramener amélioré (\pageref{subsec:ab_improved_apportation}) ou Remodeler (\pageref{subsec:ab_reshape})
\end{abnamelist}
\textbf{Intrusion de la Meneuse:}
%_____________________________________________________%
\phantomsection\label{sec:focusdriveslikeamaniac}\section*{ Conduit comme un Dingue}

\textcolor{gray}{\emph{ Drives Like A Maniac}}

Qu'il s'agisse d'être en équilibre sur deux roues, de sauter un autre véhicule ou de conduire de front vers une voiture ennemie venant en sens inverse, vous ne pensez pas aux risques lorsque vous êtes au volant. (Quelqu'un qui Conduit comme un Dingue doit avoir accès à un véhicule.)

\begin{abnamelist}
\item Rang 1: Chauffeur (\pageref{subsec:ab_driver}) Conduite Dangereuse (\pageref{subsec:ab_driving_on_the_edge})
\item Rang 2:
Regarde les en Face (\pageref{subsec:ab_stare_them_down}) Surfeur de voiture (\pageref{subsec:ab_car_surfer})
\item Rang 3:
Choisissez en une: Avantage de Célérité Amélioré (\pageref{subsec:ab_enhanced_speed_edge}) ou Chauffeur expert (\pageref{subsec:ab_expert_driver})
\item Rang 4:
Célérité améliorée (\pageref{subsec:ab_enhanced_speed}) Oeil perçant (\pageref{subsec:ab_sharp_eyed})
\item Rang 5:
Quelque chose sur la route (\pageref{subsec:ab_something_in_the_road})
\item Rang 6:
Choisissez en une: As du Volant (\pageref{subsec:ab_trick_driver}) ou Dégâts mortels (\pageref{subsec:ab_lethal_damage})
\end{abnamelist}
\textbf{Intrusion de la Meneuse:}
%_____________________________________________________%
\phantomsection\label{sec:focusbuildsrobots}\section*{ Construit des Robots}

\textcolor{gray}{\emph{ Builds Robots}}

Vos créations robotiques font ce qu'on leur commande.  Le mot « robot » est utilisé dans ce Focus, bien que le robot que vous créez puisse être très différent de celui créé par quelqu'un d'autre, selon le genre. Les robots Steampunk, les robots organiques ou même les golems magiques sont tous des « robots » réalisables. 

\begin{abnamelist}
\item Rang 1: Assistant Robot (\pageref{subsec:ab_robot_assistant}) Constructeur de robots (\pageref{subsec:ab_robot_builder})
\item Rang 2:
Contrôle du robot (\pageref{subsec:ab_robot_control})
\item Rang 3:
Choisissez en une: Compétence en Défense Supérieure (\pageref{subsec:ab_greater_skill_with_defense}) ou Disciple expert (\pageref{subsec:ab_expert_follower})
\item Rang 4:
Mise à niveau du robot (\pageref{subsec:ab_robot_upgrade})
\item Rang 5:
Flotte de robots (\pageref{subsec:ab_robot_fleet})
\item Rang 6:
Choisissez en une: Evolution du robot (\pageref{subsec:ab_robot_evolution}) ou Mise à niveau du robot (\pageref{subsec:ab_robot_upgrade})
\end{abnamelist}
\textbf{Intrusion de la Meneuse:}
%_____________________________________________________%
\phantomsection\label{sec:focusworksforaliving}\section*{ Construit et Répare}

\textcolor{gray}{\emph{ Works for a Living}}

Vous tirez une grande satisfaction d'un travail bien fait, qu'il s'agisse de coder, de construire des maisons ou d'exploiter des astéroïdes.

\begin{abnamelist}
\item Rang 1: Travaux Manuels (\pageref{subsec:ab_handy})
\item Rang 2:
Muscles de fer (\pageref{subsec:ab_muscles_of_iron})
\item Rang 3:
Choisissez en une: Improviser (\pageref{subsec:ab_improvise}) ou Oeil pour les détails (\pageref{subsec:ab_eye_for_detail})
\item Rang 4:
Puissance Améliorée (\pageref{subsec:ab_enhanced_might}) Tenez bon (\pageref{subsec:ab_tough_it_out})
\item Rang 5:
Compétence d'expert (\pageref{subsec:ab_expert_skill})
\item Rang 6:
Choisissez en une: Potentiel amélioré plus important (\pageref{subsec:ab_greater_enhanced_potential}) ou Résilience durement gagnée (\pageref{subsec:ab_hard_won_resilience})
\end{abnamelist}
\textbf{Intrusion de la Meneuse:}
%_____________________________________________________%
\phantomsection\label{sec:focusworksthesystem}\section*{ Contourne le Système}

\textcolor{gray}{\emph{ Works the System}}

Vous pouvez exploiter les failles des systèmes artificiels, y compris, mais sans s'y limiter, le code informatique.

\begin{abnamelist}
\item Rang 1: Hackez l'impossible (\pageref{subsec:ab_hack_the_impossible}) Programmation informatique (\pageref{subsec:ab_computer_programming})
\item Rang 2:
Connecté (\pageref{subsec:ab_connected})
\item Rang 3:
Choisissez en une: Artiste de la confiance (\pageref{subsec:ab_confidence_artist}) ou Compétence avec les attaques (\pageref{subsec:ab_skill_with_attacks})
\item Rang 4:
Confondre l'ennemi (\pageref{subsec:ab_confuse_enemy})
\item Rang 5:
Entretenir l'amitié (\pageref{subsec:ab_work_the_friendship})
\item Rang 6:
Choisissez en une: Demande d'une faveur (\pageref{subsec:ab_call_in_favor}) ou Potentiel amélioré plus important (\pageref{subsec:ab_greater_enhanced_potential})
\end{abnamelist}
\textbf{Intrusion de la Meneuse:}
%_____________________________________________________%
\phantomsection\label{sec:focuscontrolsgravity}\section*{ Contrôle la Gravité}

\textcolor{gray}{\emph{ Controls Gravity}}

Vous pouvez influencer l’attraction de la gravité elle-même. \newline\textbf{Option d'échange de type:} Lourd

\begin{abnamelist}
\item Rang 1: Survol (\pageref{subsec:ab_hover})
\item Rang 2:
Avantage de Célérité Amélioré (\pageref{subsec:ab_enhanced_speed_edge})
\item Rang 3:
Choisissez en une: Définir le bas (\pageref{subsec:ab_define_down}) ou Gravité Coupante (\pageref{subsec:ab_gravity_cleave})
\item Rang 4:
Champ de gravité (\pageref{subsec:ab_field_of_gravity})
\item Rang 5:
Vol (\pageref{subsec:ab_flight})
\item Rang 6:
Choisissez en une: Gravité Coupante Améliorée (\pageref{subsec:ab_improved_gravity_cleave}) ou Poids du monde (\pageref{subsec:ab_weight_of_the_world})
\end{abnamelist}
\textbf{Intrusion de la Meneuse:}
%_____________________________________________________%
\phantomsection\label{sec:focusemploysmagnetism}\section*{ Contrôle le Magnétisme}

\textcolor{gray}{\emph{ Employs Magnetism}}

Vous maîtrisez le métal et le pouvoir du magnétisme.

\begin{abnamelist}
\item Rang 1: Déplacer le métal (\pageref{subsec:ab_move_metal})
\item Rang 2:
Repousser le métal (\pageref{subsec:ab_repel_metal})
\item Rang 3:
Choisissez en une: Détruire le métal (\pageref{subsec:ab_destroy_metal}) ou Tir Guidé (\pageref{subsec:ab_guide_bolt})
\item Rang 4:
Champ magnétique (\pageref{subsec:ab_magnetic_field})
\item Rang 5:
Commander le Métal (\pageref{subsec:ab_command_metal})
\item Rang 6:
Choisissez en une: Diamagnétisme (\pageref{subsec:ab_diamagnetism}) ou Lancer Objet en Fer (\pageref{subsec:ab_iron_punch})
\end{abnamelist}
\textbf{Intrusion de la Meneuse:}
%_____________________________________________________%
\phantomsection\label{sec:focuscontrolsbeasts}\section*{ Contrôle les Bêtes Sauvages}

\textcolor{gray}{\emph{ Controls Beasts}}

Votre capacité à communiquer et à diriger des bêtes est surnaturelle.

\begin{abnamelist}
\item Rang 1: Une Bête comme Compagnon (\pageref{subsec:ab_beast_companion})
\item Rang 2:
Apaiser la Bête Sauvage (\pageref{subsec:ab_soothe_the_savage}) Communication (\pageref{subsec:ab_communication})
\item Rang 3:
Choisissez en une: Monture (\pageref{subsec:ab_mount}) ou Plus forts ensemble (\pageref{subsec:ab_stronger_together})
\item Rang 4:
Compagnon amélioré (\pageref{subsec:ab_improved_companion}) Yeux de bête (\pageref{subsec:ab_beast_eyes})
\item Rang 5:
Appel de Bête (\pageref{subsec:ab_beast_call})
\item Rang 6:
Choisissez en une: Comme s'il n'y a qu'une seule créature (\pageref{subsec:ab_as_if_one_creature}) ou Contrôler le sauvage (\pageref{subsec:ab_control_the_savage})
\end{abnamelist}
\textbf{Intrusion de la Meneuse:}
%_____________________________________________________%
\phantomsection\label{sec:focuswieldsinvisibleforce}\section*{ Contrôle une Force Invisible}

\textcolor{gray}{\emph{ Wields Invisible Force}}

Vous pliez la lumière et manipulez des faisceaux de force pour l’attaque et la défense.

\begin{abnamelist}
\item Rang 1: Disparaître (\pageref{subsec:ab_vanish})
\item Rang 2:
Force enchevêtrante (\pageref{subsec:ab_entangling_force}) Sens aiguisés (\pageref{subsec:ab_sharp_senses})
\item Rang 3:
Choisissez en une: Barrière de champ de force (\pageref{subsec:ab_force_field_barrier}) ou Invisibilité Multiple (\pageref{subsec:ab_multi_vanish})
\item Rang 4:
Invisibilité (\pageref{subsec:ab_invisibility})
\item Rang 5:
Champ défensif (\pageref{subsec:ab_defensive_field})
\item Rang 6:
Choisissez en une: Concussion (\pageref{subsec:ab_concussion}) ou Générer un champ de force (\pageref{subsec:ab_generate_force_field})
\end{abnamelist}
\textbf{Intrusion de la Meneuse:}
%_____________________________________________________%
\phantomsection\label{sec:focuscopiessuperpowers}\section*{ Copie des Superpouvoirs}

\textcolor{gray}{\emph{ Copies Superpowers}}

Vous pouvez copier les compétences, les capacités et les super pouvoirs des autres.

\begin{abnamelist}
\item Rang 1: Compétences en Gage (\pageref{subsec:ab_flex_skill})
\item Rang 2:
Pouvoir de copie (\pageref{subsec:ab_copy_power})
\item Rang 3:
Choisissez en une: Pouvoirs génériques (\pageref{subsec:ab_wildcard_powers}) ou Voler le pouvoir (\pageref{subsec:ab_steal_power})
\item Rang 4:
Copie Améliorée (\pageref{subsec:ab_improved_copying})
\item Rang 5:
Mémoire de Pouvoir Copié (\pageref{subsec:ab_power_memory})
\item Rang 6:
Choisissez en une: Copie multiple (\pageref{subsec:ab_multiple_copying}) ou Copie étonnante (\pageref{subsec:ab_amazing_copying})
\end{abnamelist}
\textbf{Intrusion de la Meneuse:}
%_____________________________________________________%
\phantomsection\label{sec:focusabidesinstone}\section*{ Demeure dans la pierre}

\textcolor{gray}{\emph{ Abides in Stone}}

Votre chair est constituée de minéraux durs, ce qui fait de vous un humanoïde imposant et difficile à blesser.

\begin{abnamelist}
\item Rang 1: Corps de Golem (\pageref{subsec:ab_golem_body}) Guérison du Golem (\pageref{subsec:ab_golem_healing})
\item Rang 2:
Prise de Golem (\pageref{subsec:ab_golem_grip})
\item Rang 3:
Choisissez en une: Cogneur Entraîné (\pageref{subsec:ab_trained_basher}) Armement (\pageref{subsec:ab_weaponization}) ou Piétinement de Golem (\pageref{subsec:ab_golem_stomp})
\item Rang 4:
Réserves Partagées (\pageref{subsec:ab_deep_reserves})
\item Rang 5:
Cogneur Spécialisé (\pageref{subsec:ab_specialized_basher}) Toujours comme une statue (\pageref{subsec:ab_still_as_a_statue})
\item Rang 6:
Choisissez en une: Sursaut mental (\pageref{subsec:ab_mind_surge}) ou Ultra amélioration (\pageref{subsec:ab_ultra_enhancement})
\end{abnamelist}
\textbf{Intrusion de la Meneuse:}
%_____________________________________________________%
\phantomsection\label{sec:focusleads}\section*{ Dirige}

\textcolor{gray}{\emph{ Leads}}

Votre capacité naturelle de leadership vous permet de commander aux autres, y compris à un groupe de fidèles.

\begin{abnamelist}
\item Rang 1: Bon conseil (\pageref{subsec:ab_good_advice}) Charisme naturel (\pageref{subsec:ab_natural_charisma})
\item Rang 2:
Potentiel amélioré (\pageref{subsec:ab_enhanced_potential}) Suivant de base (\pageref{subsec:ab_basic_follower})
\item Rang 3:
Choisissez en une: Commandement avancé (\pageref{subsec:ab_advanced_command}) ou Disciple expert (\pageref{subsec:ab_expert_follower})
\item Rang 4:
Captiver ou Inspirer (\pageref{subsec:ab_captivate_or_inspire})
\item Rang 5:
Potentiel amélioré plus important (\pageref{subsec:ab_greater_enhanced_potential})
\item Rang 6:
Choisissez en une: Bande de suivants (\pageref{subsec:ab_band_of_followers}) ou Esprit de leader (\pageref{subsec:ab_mind_of_a_leader})
\end{abnamelist}
\textbf{Intrusion de la Meneuse:}
%_____________________________________________________%
\phantomsection\label{sec:focuskeepsamagically}\section*{ Dispose d'un Allié Magique}

\textcolor{gray}{\emph{ Keeps a Magic Ally}}

Une créature magique alliée liée à un objet (comme un djinn mineur dans une lampe, ou un fantôme dans une pipe) est votre ami, votre Protecteur et votre arme.

\begin{abnamelist}
\item Rang 1: Créature magique liée (\pageref{subsec:ab_bound_magic_creature})
\item Rang 2:
Lien d'objet (\pageref{subsec:ab_object_bond}) Placard caché (\pageref{subsec:ab_hidden_closet})
\item Rang 3:
Choisissez en une: Monture (\pageref{subsec:ab_mount}) ou Souhait mineur (\pageref{subsec:ab_minor_wish})
\item Rang 4:
Lien d'Objet Amélioré (\pageref{subsec:ab_improved_object_bond})
\item Rang 5:
Souhait modéré (\pageref{subsec:ab_moderate_wish})
\item Rang 6:
Choisissez en une: Faites confiance à la chance (\pageref{subsec:ab_trust_to_luck}) ou Maîtrise des liens d'objet (\pageref{subsec:ab_object_bond_mastery})
\end{abnamelist}
\textbf{Intrusion de la Meneuse:}
%_____________________________________________________%
\phantomsection\label{sec:focusentertains}\section*{ Divertit}

\textcolor{gray}{\emph{ Entertains}}

Vous vous donnez en spectacle principalement pour divertir les autres.

\begin{abnamelist}
\item Rang 1: Légèreté (\pageref{subsec:ab_levity})
\item Rang 2:
Facilité Inspirante (\pageref{subsec:ab_inspiring_ease})
\item Rang 3:
Choisissez en une: Compétences en Connaissances (\pageref{subsec:ab_knowledge_skills}) ou Potentiel amélioré plus important (\pageref{subsec:ab_greater_enhanced_potential})
\item Rang 4:
Calme (\pageref{subsec:ab_calm})
\item Rang 5:
Assistance compétente (\pageref{subsec:ab_able_assistance})
\item Rang 6:
Choisissez en une: Maître du spectacle (\pageref{subsec:ab_master_entertainer}) ou Performance vindicative (\pageref{subsec:ab_vindictive_performance})
\end{abnamelist}
\textbf{Intrusion de la Meneuse:}
%_____________________________________________________%
\phantomsection\label{sec:focusshredsthewallsoftheworld}\section*{ Déchire les Murs du Monde}

\textcolor{gray}{\emph{ Shreds the Walls of the World}}

La vitesse et le phasage vous donnent une capacité unique à échapper au danger et à infliger simultanément des dégâts.

\begin{abnamelist}
\item Rang 1: Sprint de Phase (\pageref{subsec:ab_phase_sprint}) Toucher perturbateur (\pageref{subsec:ab_disrupting_touch})
\item Rang 2:
Rayer l'Existence (\pageref{subsec:ab_scratch_existence})
\item Rang 3:
Choisissez en une: Invisibilité de Phase (\pageref{subsec:ab_invisible_phasing}) ou Traverser les murs (\pageref{subsec:ab_walk_through_walls})
\item Rang 4:
Détonation de phase (\pageref{subsec:ab_phase_detonation})
\item Rang 5:
Très long sprint (\pageref{subsec:ab_very_long_sprinting})
\item Rang 6:
Choisissez en une: Déchirer L'Existence (\pageref{subsec:ab_shred_existence}) ou Intouchable en mouvement (\pageref{subsec:ab_untouchable_while_moving})
\end{abnamelist}
\textbf{Intrusion de la Meneuse:}
%_____________________________________________________%
\phantomsection\label{sec:focusdefendstheweak}\section*{ Défend les Faibles}

\textcolor{gray}{\emph{ Defends the Weak}}

Vous défendez les impuissants, les faibles et ceux qui ne sont pas protégés.

\begin{abnamelist}
\item Rang 1: Bouclier de protection (\pageref{subsec:ab_warding_shield}) Courageux (\pageref{subsec:ab_courageous})
\item Rang 2:
Perspicacité (\pageref{subsec:ab_insight}) Véritable défenseur (\pageref{subsec:ab_true_defender})
\item Rang 3:
Choisissez en une: Double protection (\pageref{subsec:ab_dual_wards}) ou Véritable Gardien (\pageref{subsec:ab_true_guardian})
\item Rang 4:
Défi de combat (\pageref{subsec:ab_combat_challenge})
\item Rang 5:
Sacrifice volontaire (\pageref{subsec:ab_willing_sacrifice})
\item Rang 6:
Choisissez en une: Réanimer (\pageref{subsec:ab_resuscitate}) ou Véritable défenseur (\pageref{subsec:ab_true_defender})
\end{abnamelist}
\textbf{Intrusion de la Meneuse:}
%_____________________________________________________%
\phantomsection\label{sec:focusisidolizedbymillions}\section*{ Est Idolatré par Millions}

\textcolor{gray}{\emph{ Is Idolized by Millions}}

Vous êtes une célébrité et la plupart des gens vous adorent.

\begin{abnamelist}
\item Rang 1: Entourage (\pageref{subsec:ab_entourage}) Talent de célébrité (\pageref{subsec:ab_celebrity_talent})
\item Rang 2:
Avantages de la célébrité (\pageref{subsec:ab_perks_of_stardom})
\item Rang 3:
Choisissez en une: Compétence avec les attaques (\pageref{subsec:ab_skill_with_attacks}) ou Santé incroyable (\pageref{subsec:ab_incredible_health})
\item Rang 4:
Captiver avec Eclat (\pageref{subsec:ab_captivate_with_starshine}) Disciple expert (\pageref{subsec:ab_expert_follower})
\item Rang 5:
Est-ce que tu sais qui je suis? (\pageref{subsec:ab_do_you_know_who_i_am?})
\item Rang 6:
Choisissez en une: Compagnon amélioré (\pageref{subsec:ab_improved_companion}) ou Transcendez le scénario (\pageref{subsec:ab_transcend_the_script})
\end{abnamelist}
\textbf{Intrusion de la Meneuse:}
%_____________________________________________________%
\phantomsection\label{sec:focusiswantedbythelaw}\section*{ Est Recherché par la Loi}

\textcolor{gray}{\emph{ Is Wanted by the Law}}

Des affiches "WANTED, DEAD or ALIVE" (ou leur équivalent) sont apparues avec votre visage. C'est à vous de décider si c'est une erreur qui est devenue incontrôlable ou si vous tueriez quelqu'un juste pour un regard.

\begin{abnamelist}
\item Rang 1: Célérité améliorée (\pageref{subsec:ab_enhanced_speed}) Sens du Danger (\pageref{subsec:ab_danger_sense})
\item Rang 2:
Attaque surprise (\pageref{subsec:ab_surprise_attack})
\item Rang 3:
Choisissez en une: Attaque successive (\pageref{subsec:ab_successive_attack}) ou Réputation de hors-la-loi (\pageref{subsec:ab_outlaw_reputation})
\item Rang 4:
Meurtre Rapide (\pageref{subsec:ab_fast_kill})
\item Rang 5:
Bande de Desperados (\pageref{subsec:ab_band_of_desperados})
\item Rang 6:
Choisissez en une: Dégâts mortels (\pageref{subsec:ab_lethal_damage}) ou Pas encore mort (\pageref{subsec:ab_not_dead_yet})
\end{abnamelist}
\textbf{Intrusion de la Meneuse:}
%_____________________________________________________%
\phantomsection\label{sec:focusemergedfromtheobelisk}\section*{ Est sorti de l'Obélisque}

\textcolor{gray}{\emph{ Emerged From the Obelisk}}

Votre corps, dur comme du cristal, vous confère une suite de capacités uniques, acquises après une interaction avec un obélisque cristallin flottant.

\begin{abnamelist}
\item Rang 1: Corps de Cristal (\pageref{subsec:ab_crystalline_body})
\item Rang 2:
Survol (\pageref{subsec:ab_hover})
\item Rang 3:
Choisissez en une: Habiter le cristal (\pageref{subsec:ab_inhabit_crystal}) ou Immobile (\pageref{subsec:ab_immovable})
\item Rang 4:
Lentille Cristalline (\pageref{subsec:ab_crystal_lens})
\item Rang 5:
Fréquence de résonance (\pageref{subsec:ab_resonant_frequency})
\item Rang 6:
Choisissez en une: Retour à l'Obélisque (\pageref{subsec:ab_return_to_the_obelisk}) ou Tremblement de résonance (\pageref{subsec:ab_resonant_quake})
\end{abnamelist}
\textbf{Intrusion de la Meneuse:}
%_____________________________________________________%
\phantomsection\label{sec:focusscavenges}\section*{ Est un Récupérateur}

\textcolor{gray}{\emph{ Scavenges}}

Lorsque vous ne courez pas et ne vous cachez pas, vous fouillez les ruines de la civilisation à la recherche de vestiges utiles pour assurer votre survie.

\begin{abnamelist}
\item Rang 1: Connaissance des ruines (\pageref{subsec:ab_ruin_lore}) Survivant post-apocalyptique (\pageref{subsec:ab_post_apocalyptic_survivor})
\item Rang 2:
De Déchet à Objet (\pageref{subsec:ab_junkmonger})
\item Rang 3:
Choisissez en une: Prendre l'avantage (\pageref{subsec:ab_taking_advantage}) ou Santé incroyable (\pageref{subsec:ab_incredible_health})
\item Rang 4:
Sachez où chercher (\pageref{subsec:ab_know_where_to_look})
\item Rang 5:
Cyphers recyclés (\pageref{subsec:ab_recycled_cyphers})
\item Rang 6:
Choisissez en une: Champ réactif (\pageref{subsec:ab_reactive_field}) ou Pilleur d'artifact (\pageref{subsec:ab_artifact_scavenger})
\end{abnamelist}
\textbf{Intrusion de la Meneuse:}
%_____________________________________________________%
\phantomsection\label{sec:focusexistspartiallyoutofphase}\section*{ Existe Partiellement Hors de Phase}

\textcolor{gray}{\emph{ Exists Partially Out of Phase}}

Un peu translucide, vous êtes légèrement déphasé et pouvez vous déplacer à travers des objets solides.

\begin{abnamelist}
\item Rang 1: Traverser les murs (\pageref{subsec:ab_walk_through_walls})
\item Rang 2:
Phase défensive (\pageref{subsec:ab_defensive_phasing})
\item Rang 3:
Choisissez en une: Attaque de Phase (\pageref{subsec:ab_phased_attack}) ou Porte de phase (\pageref{subsec:ab_phase_door})
\item Rang 4:
Fantôme (\pageref{subsec:ab_ghost})
\item Rang 5:
Intouchable (\pageref{subsec:ab_untouchable})
\item Rang 6:
Choisissez en une: Attaque de Phase Améliorée (\pageref{subsec:ab_enhanced_phased_attack}) ou Mettre un Ennemi en Phase (\pageref{subsec:ab_phase_foe})
\end{abnamelist}
\textbf{Intrusion de la Meneuse:}
%_____________________________________________________%
\phantomsection\label{sec:focusexistsintwoplacesatonce}\section*{ Existe en Deux Endroits en Même Temps}

\textcolor{gray}{\emph{ Exists in Two Places at Once}}

Vous existez en deux endroits en même Ttemps.

\begin{abnamelist}
\item Rang 1: Duplicata (\pageref{subsec:ab_duplicate})
\item Rang 2:
Partager les sens (\pageref{subsec:ab_share_senses})
\item Rang 3:
Choisissez en une: Duplicata résilient (\pageref{subsec:ab_resilient_duplicate}) ou Duplication Supérieure (\pageref{subsec:ab_superior_duplicate})
\item Rang 4:
Transfert de dégâts (\pageref{subsec:ab_damage_transference})
\item Rang 5:
Effort coordonné (\pageref{subsec:ab_coordinated_effort})
\item Rang 6:
Choisissez en une: Duplicata résilient (\pageref{subsec:ab_resilient_duplicate}) ou Multiplicité (\pageref{subsec:ab_multiplicity})
\end{abnamelist}
\textbf{Intrusion de la Meneuse:}
%_____________________________________________________%
\phantomsection\label{sec:focusexploresdarkplaces}\section*{ Explore des Endroits Sombres}

\textcolor{gray}{\emph{ Explores Dark Places}}

Vous êtes l'archétype du chasseur de trésors, du charognard et du chercheur d'objets perdus.

\begin{abnamelist}
\item Rang 1: Explorateur Confirmé (\pageref{subsec:ab_superb_explorer})
\item Rang 2:
Infiltrateur Confirmé (\pageref{subsec:ab_superb_infiltrator}) Yeux ajustés (\pageref{subsec:ab_eyes_adjusted})
\item Rang 3:
Choisissez en une: Client Fuyant (\pageref{subsec:ab_slippery_customer}) ou Frappe nocturne (\pageref{subsec:ab_nightstrike})
\item Rang 4:
Résilience durement gagnée (\pageref{subsec:ab_hard_won_resilience})
\item Rang 5:
Explorateur des ténèbres (\pageref{subsec:ab_dark_explorer})
\item Rang 6:
Choisissez en une: Attaque aveuglante (\pageref{subsec:ab_blinding_attack}) ou Fusionne avec les Ténèbres (\pageref{subsec:ab_embraced_by_darkness})
\end{abnamelist}
\textbf{Intrusion de la Meneuse:}
%_____________________________________________________%
\phantomsection\label{sec:focusthunders}\section*{ Fait Résonner le Tonnerre}

\textcolor{gray}{\emph{ Thunders}}

Vous émettez un son destructeur et manipulez les paysages sonores.

\begin{abnamelist}
\item Rang 1: Faisceau de Tonnerre (\pageref{subsec:ab_thunder_beam})
\item Rang 2:
Barrière de conversion sonore (\pageref{subsec:ab_sound_conversion_barrier})
\item Rang 3:
Choisissez en une: Annuler le son (\pageref{subsec:ab_nullify_sound}) ou Echolocation (\pageref{subsec:ab_echolocation})
\item Rang 4:
Cri fracassant (\pageref{subsec:ab_shattering_shout})
\item Rang 5:
Amplifier les sons (\pageref{subsec:ab_amplify_sounds}) Grondement subsonique (\pageref{subsec:ab_subsonic_rumble})
\item Rang 6:
Choisissez en une: Tremblement de terre (\pageref{subsec:ab_earthquake}) ou Vibration mortelle (\pageref{subsec:ab_lethal_vibration})
\end{abnamelist}
\textbf{Intrusion de la Meneuse:}
%_____________________________________________________%
\phantomsection\label{sec:focusworksmiracles}\section*{ Fait des Miracles}

\textcolor{gray}{\emph{ Works Miracles}}

Vous pouvez guérir les autres d’un simple toucher, modifier le temps pour aider les autres et êtes généralement aimé de tous.

\begin{abnamelist}
\item Rang 1: Main Guérisseuse (\pageref{subsec:ab_healing_touch})
\item Rang 2:
Soulager (\pageref{subsec:ab_alleviate})
\item Rang 3:
Choisissez en une: Santé miraculeuse (\pageref{subsec:ab_miraculous_health}) ou Source de guérison (\pageref{subsec:ab_font_of_healing})
\item Rang 4:
Inspirer l'action (\pageref{subsec:ab_inspire_action})
\item Rang 5:
Annuler (\pageref{subsec:ab_undo})
\item Rang 6:
Choisissez en une: Main Guérisseuse Supérieure (\pageref{subsec:ab_greater_healing_touch}) ou Restaurer la vie (\pageref{subsec:ab_restore_life})
\end{abnamelist}
\textbf{Intrusion de la Meneuse:}
%_____________________________________________________%
\phantomsection\label{sec:focuscraftsillusions}\section*{ Façonne des Illusions}

\textcolor{gray}{\emph{ Crafts Illusions}}

Vous façonnez des images à partir de la lumière qui sont si parfaites qu’elles semblent réelles.

\begin{abnamelist}
\item Rang 1: Illusion mineure (\pageref{subsec:ab_minor_illusion})
\item Rang 2:
Déguisement illusoire (\pageref{subsec:ab_illusory_disguise})
\item Rang 3:
Choisissez en une: Illusion majeure (\pageref{subsec:ab_major_illusion}) ou Lancer Illusion (\pageref{subsec:ab_cast_illusion})
\item Rang 4:
Image Miroir (\pageref{subsec:ab_illusory_selves})
\item Rang 5:
Image terrifiante (\pageref{subsec:ab_terrifying_image})
\item Rang 6:
Choisissez en une: Illusion grandiose (\pageref{subsec:ab_grandiose_illusion}) ou Illusion permanente (\pageref{subsec:ab_permanent_illusion})
\end{abnamelist}
\textbf{Intrusion de la Meneuse:}
%_____________________________________________________%
\phantomsection\label{sec:focuscraftsuniqueobjects}\section*{ Façonne des Objets Uniques}

\textcolor{gray}{\emph{ Crafts Unique Objects}}

Vous êtes un inventeur d'objets étranges et utiles.

\begin{abnamelist}
\item Rang 1: Artisan (\pageref{subsec:ab_crafter}) Maîtrise de l'Identification des appareils (\pageref{subsec:ab_master_identifier})
\item Rang 2:
Bricoleur d'artefacts (\pageref{subsec:ab_artifact_tinkerer}) Travail rapide (\pageref{subsec:ab_quick_work})
\item Rang 3:
Choisissez en une: Armes intégrées (\pageref{subsec:ab_built_in_weaponry}) ou Maître artisan (\pageref{subsec:ab_master_crafter})
\item Rang 4:
Maîtrise des Cyphers (\pageref{subsec:ab_cyphersmith})
\item Rang 5:
Innovateur (\pageref{subsec:ab_innovator})
\item Rang 6:
Choisissez en une: Armure Corporelle (\pageref{subsec:ab_fusion_armor}) ou Inventeur (\pageref{subsec:ab_inventor})
\end{abnamelist}
\textbf{Intrusion de la Meneuse:}
%_____________________________________________________%
\phantomsection\label{sec:focusridesthelightning}\section*{ Façonne la Foudre}

\textcolor{gray}{\emph{ Rides the Lightning}}

Vous créez et déchargez de l’énergie électrique.

\begin{abnamelist}
\item Rang 1: Charge (\pageref{subsec:ab_charge}) Choc Electrique (\pageref{subsec:ab_shock})
\item Rang 2:
Comme l'éclair (\pageref{subsec:ab_bolt_rider})
\item Rang 3:
Choisissez en une: Armure électrique (\pageref{subsec:ab_electric_armor}) ou Charge Drainante (\pageref{subsec:ab_drain_charge})
\item Rang 4:
Eclairs de Puissance (\pageref{subsec:ab_bolts_of_power})
\item Rang 5:
Vol électrique (\pageref{subsec:ab_electrical_flight})
\item Rang 6:
Choisissez en une: Aussi Rapide que l'Eclair (\pageref{subsec:ab_flash_across_the_miles}) ou Mur de foudre (\pageref{subsec:ab_wall_of_lightning})
\end{abnamelist}
\textbf{Intrusion de la Meneuse:}
%_____________________________________________________%
\phantomsection\label{sec:focusfusesfleshandsteel}\section*{ Fusionne la Chair et l'Acier}

\textcolor{gray}{\emph{ Fuses Flesh and Steel}}

Votre corps est en partie une machine.

\begin{abnamelist}
\item Rang 1: Corps amélioré (\pageref{subsec:ab_enhanced_body})
\item Rang 2:
Interface (\pageref{subsec:ab_interface})
\item Rang 3:
Choisissez en une: Armement (\pageref{subsec:ab_weaponization}) ou Ensemble de détection (\pageref{subsec:ab_sensing_package})
\item Rang 4:
Fusion (\pageref{subsec:ab_fusion})
\item Rang 5:
Réserves Partagées (\pageref{subsec:ab_deep_reserves})
\item Rang 6:
Choisissez en une: Sursaut mental (\pageref{subsec:ab_mind_surge}) ou Ultra amélioration (\pageref{subsec:ab_ultra_enhancement})
\end{abnamelist}
\textbf{Intrusion de la Meneuse:}
%_____________________________________________________%
\phantomsection\label{sec:focusfusesmindandmachine}\section*{ Fussionne l'Esprit et la Machine}

\textcolor{gray}{\emph{ Fuses Mind and Machine}}

Les aides électroniques implantées dans votre cerveau font de vous une surpuissance cérébrale.

\begin{abnamelist}
\item Rang 1: Compétences en Connaissances (\pageref{subsec:ab_knowledge_skills}) Intellect amélioré (\pageref{subsec:ab_enhanced_intellect})
\item Rang 2:
Une information dans le réseau (\pageref{subsec:ab_network_tap})
\item Rang 3:
Choisissez en une: Processeur d'action (\pageref{subsec:ab_action_processor}) ou Télépathie machine (\pageref{subsec:ab_machine_telepathy})
\item Rang 4:
Compétences en Connaissances (\pageref{subsec:ab_knowledge_skills}) Intellect Amélioré Supérieur (\pageref{subsec:ab_greater_enhanced_intellect})
\item Rang 5:
Voir l'avenir (\pageref{subsec:ab_see_the_future})
\item Rang 6:
Choisissez en une: Amélioration de la machine (\pageref{subsec:ab_machine_enhancement}) ou Sursaut mental (\pageref{subsec:ab_mind_surge})
\end{abnamelist}
\textbf{Intrusion de la Meneuse:}
%_____________________________________________________%
\phantomsection\label{sec:focusdefendsthegate}\section*{ Garde le Passage}

\textcolor{gray}{\emph{ Defends the Gate}}

Tout le monde veut que vous soyez à ses côtés lorsqu’il s’agit d’un combat, car rien ne vous échappe.

\begin{abnamelist}
\item Rang 1: Position fortifiée (\pageref{subsec:ab_fortified_position}) Ralliez-vous à moi (\pageref{subsec:ab_rally_to_me})
\item Rang 2:
Esprit de Puissance (\pageref{subsec:ab_mind_for_might})
\item Rang 3:
Choisissez en une: Constructeur de fortifications (\pageref{subsec:ab_fortification_builder}) ou Détourner les attaques (\pageref{subsec:ab_divert_attacks})
\item Rang 4:
Puissance améliorée supérieure (\pageref{subsec:ab_greater_enhanced_might})
\item Rang 5:
Champ de renforcement (\pageref{subsec:ab_reinforcing_field})
\item Rang 6:
Choisissez en une: Attaque Etourdissante (\pageref{subsec:ab_stun_attack}) ou Générer un champ de force (\pageref{subsec:ab_generate_force_field})
\end{abnamelist}
\textbf{Intrusion de la Meneuse:}
%_____________________________________________________%
\phantomsection\label{sec:focusgrowstotoweringheights}\section*{ Grandit Jusqu'au Ciel}

\textcolor{gray}{\emph{ Grows to Towering Heights}}

Pendant de brèves périodes, vous pouvez grandir et, avec suffisamment d'expérience, atteindre des hauteurs imposantes.

\begin{abnamelist}
\item Rang 1: Agrandir (\pageref{subsec:ab_enlarge}) Agrandissement Efrayant (\pageref{subsec:ab_freakishly_large})
\item Rang 2:
Avantages d'être grand (\pageref{subsec:ab_advantages_of_being_big}) Plus grand (\pageref{subsec:ab_bigger})
\item Rang 3:
Choisissez en une: Enorme (\pageref{subsec:ab_huge}) ou Lancer (\pageref{subsec:ab_throw})
\item Rang 4:
Saisir (\pageref{subsec:ab_grab})
\item Rang 5:
Gargantuesque (\pageref{subsec:ab_gargantuan})
\item Rang 6:
Choisissez en une: Colossal (\pageref{subsec:ab_colossal}) ou Dégâts mortels (\pageref{subsec:ab_lethal_damage})
\end{abnamelist}
\textbf{Intrusion de la Meneuse:}
%_____________________________________________________%
\phantomsection\label{sec:focusshepherdsthecommunity}\section*{ Guide la Communauté}

\textcolor{gray}{\emph{ Shepherds the Community}}

Vous gardez le lieu où vous vivez à l'abri de tout danger.

\begin{abnamelist}
\item Rang 1: Activiste communautaire (\pageref{subsec:ab_community_activist}) Connaissance de la communauté (\pageref{subsec:ab_community_knowledge})
\item Rang 2:
Compétence avec les attaques (\pageref{subsec:ab_skill_with_attacks})
\item Rang 3:
Choisissez en une: Compétence en Défense Supérieure (\pageref{subsec:ab_greater_skill_with_defense}) ou Fureur du berger (\pageref{subsec:ab_shepherds_fury})
\item Rang 4:
Potentiel amélioré plus important (\pageref{subsec:ab_greater_enhanced_potential})
\item Rang 5:
Evasion (\pageref{subsec:ab_escape})
\item Rang 6:
Choisissez en une: Compétence en Attaque Supérieure (\pageref{subsec:ab_greater_skill_with_attacks}) ou Mur de protection (\pageref{subsec:ab_protective_wall})
\end{abnamelist}
\textbf{Intrusion de la Meneuse:}
%_____________________________________________________%
\phantomsection\label{sec:focusshepherdsspirits}\section*{ Guide les Esprits}

\textcolor{gray}{\emph{ Shepherds Spirits}}

Les âmes errantes, les esprits de la nature et les êtres élémentaires vous aident et vous soutiennent. (Dans certains contextes, le Focus Guide les Esprits s’applique à un seul type d’esprit, comme les esprits des défunts, les esprits de la nature, etc.)

\begin{abnamelist}
\item Rang 1: Interrogez les esprits (\pageref{subsec:ab_question_the_spirits})
\item Rang 2:
Esprit Complice (\pageref{subsec:ab_spirit_accomplice})
\item Rang 3:
Choisissez en une: Commander un Esprit (\pageref{subsec:ab_command_spirit}) ou Sens surnaturels (\pageref{subsec:ab_preternatural_senses})
\item Rang 4:
Esprit Protecteur (\pageref{subsec:ab_wraith_cloak})
\item Rang 5:
Appeler l'esprit d'un mort (\pageref{subsec:ab_call_dead_spirit})
\item Rang 6:
Choisissez en une: Absorber l'Esprit (\pageref{subsec:ab_infuse_spirit}) ou Appeler un esprit d'un autre monde (\pageref{subsec:ab_call_otherworldly_spirit})
\end{abnamelist}
\textbf{Intrusion de la Meneuse:}
%_____________________________________________________%
\phantomsection\label{sec:focushowlsatthemoon}\section*{ Hurle à la Lune}

\textcolor{gray}{\emph{ Howls at the Moon}}

Pendant de brèves périodes, vous devenez une créature redoutable et puissante avec des problèmes de contrôle.

\begin{abnamelist}
\item Rang 1: Forme animale (\pageref{subsec:ab_animal_shape})
\item Rang 2:
Contrôle du Changement de Forme (\pageref{subsec:ab_controlled_change})
\item Rang 3:
Choisissez en une: Forme de bête Supérieure (\pageref{subsec:ab_greater_beast_form}) ou Forme de bête plus grande (\pageref{subsec:ab_bigger_beast_form})
\item Rang 4:
Changement contrôlé Supérieur (\pageref{subsec:ab_greater_controlled_change})
\item Rang 5:
Forme de bête améliorée (\pageref{subsec:ab_enhanced_beast_form})
\item Rang 6:
Choisissez en une: Contrôle parfait (\pageref{subsec:ab_perfect_control}) ou Dégâts mortels (\pageref{subsec:ab_lethal_damage})
\end{abnamelist}
\textbf{Intrusion de la Meneuse:}
%_____________________________________________________%
\phantomsection\label{sec:focusignoresphysicaldistance}\section*{ Ignore les Distances Physiques}

\textcolor{gray}{\emph{ Ignores Physical Distance}}

Vous pouvez vous téléporter d'un endroit à un autre en traversant brièvement une dimension parallèle.

\begin{abnamelist}
\item Rang 1: Compression dimensionnelle (\pageref{subsec:ab_dimensional_squeeze})
\item Rang 2:
Opportuniste (\pageref{subsec:ab_opportunist})
\item Rang 3:
Choisissez en une: Clignotement défensif (\pageref{subsec:ab_defensive_blinking}) ou Sauts de Téléportation (\pageref{subsec:ab_teleportation_burst})
\item Rang 4:
Téléportation courte (\pageref{subsec:ab_short_teleportation})
\item Rang 5:
Téléportation moyenne (\pageref{subsec:ab_medium_teleportation})
\item Rang 6:
Choisissez en une: Blessure de téléportation (\pageref{subsec:ab_teleportive_wound}) ou Téléportation (\pageref{subsec:ab_teleportation})
\end{abnamelist}
\textbf{Intrusion de la Meneuse:}
%_____________________________________________________%
\phantomsection\label{sec:focusblazeswithradiance}\section*{ Illumine avec Eclat}

\textcolor{gray}{\emph{ Blazes With Radiance}}

Vous pouvez créer de la lumière, la sculpter, la détourner de vous ou la rassembler pour l'utiliser comme une arme.

\begin{abnamelist}
\item Rang 1: Eclairé (\pageref{subsec:ab_enlightened}) Toucher lumineux (\pageref{subsec:ab_illuminating_touch})
\item Rang 2:
Couleurs Eblouissantes (\pageref{subsec:ab_dazzling_sunburst})
\item Rang 3:
Choisissez en une: Compétence en Défense Supérieure (\pageref{subsec:ab_greater_skill_with_defense}) ou Lumière brûlante (\pageref{subsec:ab_burning_light})
\item Rang 4:
Lumière du soleil (\pageref{subsec:ab_sunlight})
\item Rang 5:
Disparaître (\pageref{subsec:ab_vanish})
\item Rang 6:
Choisissez en une: Champ défensif (\pageref{subsec:ab_defensive_field}) ou Lumière vivante (\pageref{subsec:ab_living_light})
\end{abnamelist}
\textbf{Intrusion de la Meneuse:}
%_____________________________________________________%
\phantomsection\label{sec:focusinterpretsthelaw}\section*{ Interprète la Loi}

\textcolor{gray}{\emph{ Interprets the Law}}

Vous excellez à convaincre les autres de partager vos opinions.

\begin{abnamelist}
\item Rang 1: Connaissance de la loi (\pageref{subsec:ab_knowledge_of_the_law}) Persuasion et Tromperie (\pageref{subsec:ab_opening_statement})
\item Rang 2:
Débat (\pageref{subsec:ab_debate})
\item Rang 3:
Choisissez en une: Assistance compétente (\pageref{subsec:ab_able_assistance}) ou Avantage d'Intellect Amélioré (\pageref{subsec:ab_enhanced_intellect_edge})
\item Rang 4:
Fustiger (\pageref{subsec:ab_castigate})
\item Rang 5:
Personne ne sait mieux (\pageref{subsec:ab_no_one_knows_better})
\item Rang 6:
Choisissez en une: Potentiel amélioré plus important (\pageref{subsec:ab_greater_enhanced_potential}) ou Stagiaire juridique (\pageref{subsec:ab_legal_intern})
\end{abnamelist}
\textbf{Intrusion de la Meneuse:}
%_____________________________________________________%
\phantomsection\label{sec:focustouchesthesky}\section*{ Invoque la Tempète}

\textcolor{gray}{\emph{ Touches The Sky}}

Vous pouvez invoquer des tempêtes ou les briser.

\begin{abnamelist}
\item Rang 1: Survol (\pageref{subsec:ab_hover})
\item Rang 2:
Armure de vent (\pageref{subsec:ab_wind_armor})
\item Rang 3:
Choisissez en une: Eclairs de Puissance (\pageref{subsec:ab_bolts_of_power}) ou Graine de Tempête (\pageref{subsec:ab_storm_seed})
\item Rang 4:
Surfeur des Vents (\pageref{subsec:ab_windrider})
\item Rang 5:
Explosion de froid (\pageref{subsec:ab_cold_burst})
\item Rang 6:
Choisissez en une: Chariot à vent (\pageref{subsec:ab_wind_chariot}) ou Contrôle de la météo (\pageref{subsec:ab_control_weather})
\end{abnamelist}
\textbf{Intrusion de la Meneuse:}
%_____________________________________________________%
\phantomsection\label{sec:focusthrowswithdeadlyaccuracy}\section*{ Lance avec une Précision Mortelle}

\textcolor{gray}{\emph{ Throws With Deadly Accuracy}}

Tout ce qui quitte votre main va exactement là où vous souhaitez qu'il aille et à la portée et à la vitesse nécessaires pour produire l'impact parfait.

\begin{abnamelist}
\item Rang 1: Précision (\pageref{subsec:ab_precision})
\item Rang 2:
Visée prudente (\pageref{subsec:ab_careful_aim})
\item Rang 3:
Choisissez en une: Compétence en Défense Supérieure (\pageref{subsec:ab_greater_skill_with_defense}) ou Lancer rapide (\pageref{subsec:ab_quick_throw})
\item Rang 4:
Lanceur spécialisé (\pageref{subsec:ab_specialized_throwing}) Tout est une arme (\pageref{subsec:ab_everything_is_a_weapon})
\item Rang 5:
Tourbillon de lancers (\pageref{subsec:ab_whirlwind_of_throws})
\item Rang 6:
Choisissez en une: Dégâts mortels (\pageref{subsec:ab_lethal_damage}) ou Maîtrise de la défense (\pageref{subsec:ab_mastery_with_defense})
\end{abnamelist}
\textbf{Intrusion de la Meneuse:}
%_____________________________________________________%
\phantomsection\label{sec:focusdanceswithdarkmatter}\section*{ Manipule la Matière Noire}

\textcolor{gray}{\emph{ Dances With Dark Matter}}

Vous pouvez manipuler l'ombre et la matière « noire ».

\begin{abnamelist}
\item Rang 1: Rubans de matière noire (\pageref{subsec:ab_ribbons_of_dark_matter})
\item Rang 2:
Ailes du Vide (\pageref{subsec:ab_void_wings})
\item Rang 3:
Choisissez en une: Frappe de matière noire (\pageref{subsec:ab_dark_matter_strike}) ou Manteau de matière noire (\pageref{subsec:ab_dark_matter_shroud})
\item Rang 4:
Coquille de matière noire (\pageref{subsec:ab_dark_matter_shell})
\item Rang 5:
Surfeur de Matière Noire (\pageref{subsec:ab_windwracked_traveler})
\item Rang 6:
Choisissez en une: Embrassez la nuit (\pageref{subsec:ab_embrace_the_night}) ou Structure de matière noire (\pageref{subsec:ab_dark_matter_structure})
\end{abnamelist}
\textbf{Intrusion de la Meneuse:}
%_____________________________________________________%
\phantomsection\label{sec:focuswalksthewildwoods}\section*{ Marche dans Les Forêts Primaires}

\textcolor{gray}{\emph{ Walks The Wild Woods}}

Un adepte de la magie de la nature qui s'appuie sur le pouvoir et la force des arbres.

\begin{abnamelist}
\item Rang 1: Récupération du patient (\pageref{subsec:ab_patient_recovery}) Vie en pleine nature (\pageref{subsec:ab_wilderness_life})
\item Rang 2:
Corps en bois (\pageref{subsec:ab_wooden_body})
\item Rang 3:
Choisissez en une: Compagnon Arbre (\pageref{subsec:ab_tree_companion}) ou Sensibilisation à la nature sauvage (\pageref{subsec:ab_wilderness_awareness})
\item Rang 4:
Voyage dans les arbres (\pageref{subsec:ab_tree_travel})
\item Rang 5:
Grand arbre (\pageref{subsec:ab_great_tree})
\item Rang 6:
Choisissez en une: Floraison réparatrice (\pageref{subsec:ab_restorative_bloom}) ou Forêt Terrifiante (\pageref{subsec:ab_dreadwood})
\end{abnamelist}
\textbf{Intrusion de la Meneuse:}
%_____________________________________________________%
\phantomsection\label{sec:focusmastersweaponry}\section*{ Maîtrise l'Armement}

\textcolor{gray}{\emph{ Masters Weaponry}}

Vous êtes un maître d'arme d'un type particulier d'arme, qu'il s'agisse d'une épée, d'un fouet, d'un poignard, d'un pistolet ou autre. (Quelqu’un qui Maîtrise l'Armement peut disposer d’un équipement supplémentaire, notamment une arme de haute qualité.)

\begin{abnamelist}
\item Rang 1: Fabricant d'armes (\pageref{subsec:ab_weapon_crafter}) Maître d'Arme (\pageref{subsec:ab_weapon_master})
\item Rang 2:
Défense avec Arme (\pageref{subsec:ab_weapon_defense})
\item Rang 3:
Choisissez en une: Attaque rapide (\pageref{subsec:ab_rapid_attack}) ou Frappe désarmante (\pageref{subsec:ab_disarming_strike})
\item Rang 4:
Ne jamais échouer (\pageref{subsec:ab_never_fumble})
\item Rang 5:
Maîtrise extrême (\pageref{subsec:ab_extreme_mastery})
\item Rang 6:
Choisissez en une: Frappe mortelle (\pageref{subsec:ab_deadly_strike}) ou Meurtrier (\pageref{subsec:ab_murderer})
\end{abnamelist}
\textbf{Intrusion de la Meneuse:}
%_____________________________________________________%
\phantomsection\label{sec:focusmasterstheswarm}\section*{ Maîtrise l'Essaim}

\textcolor{gray}{\emph{ Masters the Swarm}}

Insectes. Rats. Chauves-souris. Même les oiseaux. Vous maîtrisez un type de petite créature qui vous obéit.

\begin{abnamelist}
\item Rang 1: Influence d'Essaim (\pageref{subsec:ab_influence_swarm})
\item Rang 2:
Contrôle de l'Essaim (\pageref{subsec:ab_control_swarm})
\item Rang 3:
Choisissez en une: Armure vivante (\pageref{subsec:ab_living_armor}) ou Compétence avec les attaques (\pageref{subsec:ab_skill_with_attacks})
\item Rang 4:
Appeller un essaim (\pageref{subsec:ab_call_swarm})
\item Rang 5:
Gagner un compagnon inhabituel (\pageref{subsec:ab_gain_unusual_companion})
\item Rang 6:
Choisissez en une: Compétence en Défense Supérieure (\pageref{subsec:ab_greater_skill_with_defense}) ou Essaim mortel (\pageref{subsec:ab_deadly_swarm})
\end{abnamelist}
\textbf{Intrusion de la Meneuse:}
%_____________________________________________________%
\phantomsection\label{sec:focusmastersdefense}\section*{ Maîtrise la Défense}

\textcolor{gray}{\emph{ Masters Defense}}

Vous utilisez un équipement de protection et des techniques pratiquées pour éviter de vous blesser lors d'un combat.

\begin{abnamelist}
\item Rang 1: Maîtrise du Bouclier (\pageref{subsec:ab_shield_master})
\item Rang 2:
Pratique des armures (\pageref{subsec:ab_practiced_in_armor}) Robuste (\pageref{subsec:ab_sturdy})
\item Rang 3:
Choisissez en une: Esquive et résistance (\pageref{subsec:ab_dodge_and_resist}) ou Esquiver et répondre (\pageref{subsec:ab_dodge_and_respond})
\item Rang 4:
Expérimenté en armure (\pageref{subsec:ab_experienced_in_armor}) Tour de Volonté (\pageref{subsec:ab_tower_of_will})
\item Rang 5:
Rien que Défendre (\pageref{subsec:ab_nothing_but_defend})
\item Rang 6:
Choisissez en une: Maître de la défense (\pageref{subsec:ab_defense_master}) ou Portez-la bien (\pageref{subsec:ab_wear_it_well})
\end{abnamelist}
\textbf{Intrusion de la Meneuse:}
%_____________________________________________________%
\phantomsection\label{sec:focusmastersspells}\section*{ Maîtrise les Sortilèges}

\textcolor{gray}{\emph{ Masters Spells}}

En vous spécialisant dans le lancement de sortilèges et en tenant un livre de sorts, vous pouvez rapidement lancer des sorts d'arc de foudre, de feu roulant, d'ombre rampante et d'invocation.

\begin{abnamelist}
\item Rang 1: Tir Arcanique (\pageref{subsec:ab_arcane_flare})
\item Rang 2:
Rayon de confusion (\pageref{subsec:ab_ray_of_confusion})
\item Rang 3:
Choisissez en une: Cube de flammes (\pageref{subsec:ab_fire_bloom}) ou Invoquer une araignée géante (\pageref{subsec:ab_summon_giant_spider})
\item Rang 4:
Interrogation de l'âme (\pageref{subsec:ab_soul_interrogation})
\item Rang 5:
Mur de Granit (\pageref{subsec:ab_granite_wall})
\item Rang 6:
Choisissez en une: Invoquer un démon (\pageref{subsec:ab_summon_demon}) ou Mot de mort (\pageref{subsec:ab_word_of_death})
\end{abnamelist}
\textbf{Intrusion de la Meneuse:}
%_____________________________________________________%
\phantomsection\label{sec:focusneedsnoweapon}\section*{ N'a pas Besoin d'Arme}

\textcolor{gray}{\emph{ Needs No Weapon}}

Des coups de poing, de pied, de coude, des genoux et des mouvements complets du corps sont toutes les armes dont vous avez besoin.

\begin{abnamelist}
\item Rang 1: Chair de Pierre (\pageref{subsec:ab_flesh_of_stone}) Poings de fureur (\pageref{subsec:ab_fists_of_fury})
\item Rang 2:
Avantage par Désavantage (\pageref{subsec:ab_advantage_to_disadvantage}) Style de combat à mains nues (\pageref{subsec:ab_unarmed_fighting_style})
\item Rang 3:
Choisissez en une: Potentiel amélioré plus important (\pageref{subsec:ab_greater_enhanced_potential}) ou Se déplacer comme l'eau (\pageref{subsec:ab_moving_like_water})
\item Rang 4:
Détourner les attaques (\pageref{subsec:ab_divert_attacks})
\item Rang 5:
Attaque Etourdissante (\pageref{subsec:ab_stun_attack})
\item Rang 6:
Choisissez en une: Dégâts mortels (\pageref{subsec:ab_lethal_damage}) ou Maître du style de combat à mains nues (\pageref{subsec:ab_master_of_unarmed_fighting_style})
\end{abnamelist}
\textbf{Intrusion de la Meneuse:}
%_____________________________________________________%
\phantomsection\label{sec:focusdoesn'tdomuch}\section*{ Ne Fait pas Grand Chose}

\textcolor{gray}{\emph{ Doesn't Do Much}}

Vous êtes un fainéant, mais vous en savez un peu sur beaucoup de choses.

\begin{abnamelist}
\item Rang 1: Leçons de vie (\pageref{subsec:ab_life_lessons})
\item Rang 2:
Totalement Chill (\pageref{subsec:ab_totally_chill})
\item Rang 3:
Choisissez en une: Compétence avec les attaques (\pageref{subsec:ab_skill_with_attacks}) ou Improviser (\pageref{subsec:ab_improvise})
\item Rang 4:
Compétence en Défense Supérieure (\pageref{subsec:ab_greater_skill_with_defense}) Leçons de vie (\pageref{subsec:ab_life_lessons})
\item Rang 5:
Potentiel amélioré plus important (\pageref{subsec:ab_greater_enhanced_potential})
\item Rang 6:
Choisissez en une: S'appuyer sur les expériences de la vie (\pageref{subsec:ab_drawing_on_lifes_experiences}) ou Vif d'esprit (\pageref{subsec:ab_quick_wits})
\end{abnamelist}
\textbf{Intrusion de la Meneuse:}
%_____________________________________________________%
\phantomsection\label{sec:focusneversaysdie}\section*{ Ne S'Avoue Jamais Vaincu}

\textcolor{gray}{\emph{ Never Says Die}}

Vous n’abandonnez jamais, vous pouvez ignorer les coups et revenir toujours pour vous battre.

\begin{abnamelist}
\item Rang 1: Récupération améliorée (\pageref{subsec:ab_improved_recovery}) Passer à travers (\pageref{subsec:ab_push_on_through})
\item Rang 2:
Ignorez la Douleur (\pageref{subsec:ab_ignore_the_pain})
\item Rang 3:
Choisissez en une: Fièvre sanguinolente (\pageref{subsec:ab_blood_fever}) ou Réserves cachées (\pageref{subsec:ab_hidden_reserves})
\item Rang 4:
Choisissez en une: Détermination croissante (\pageref{subsec:ab_increasing_determination}) ou Survivre à l'ennemi (\pageref{subsec:ab_outlast_the_foe})
\item Rang 5:
Pas encore mort (\pageref{subsec:ab_not_dead_yet})
\item Rang 6:
Choisissez en une: Défi final (\pageref{subsec:ab_final_defiance}) ou Ignorer l'Affliction (\pageref{subsec:ab_ignore_affliction})
\end{abnamelist}
\textbf{Intrusion de la Meneuse:}
%_____________________________________________________%
\phantomsection\label{sec:focusoperatesundercover}\section*{ Opère sous Couverture}

\textcolor{gray}{\emph{ Operates Undercover}}

Sous l’apparence de quelqu’un d’autre, vous cherchez à trouver des réponses que les puissants ne veulent pas divulguer. (Quelqu'un qui Opère sous Couverture pourrait avoir un équipement supplémentaire comprenant un kit de déguisement.)

\begin{abnamelist}
\item Rang 1: Enquêter (\pageref{subsec:ab_investigate})
\item Rang 2:
Déguisement (\pageref{subsec:ab_disguise})
\item Rang 3:
Choisissez en une: Agent Provocateur (\pageref{subsec:ab_agent_provocateur}) ou Courir et combattre (\pageref{subsec:ab_run_and_fight})
\item Rang 4:
Rapide Tromperie (\pageref{subsec:ab_pull_a_fast_one})
\item Rang 5:
Utiliser ce qui est disponible (\pageref{subsec:ab_using_whats_available})
\item Rang 6:
Choisissez en une: Faites confiance à la chance (\pageref{subsec:ab_trust_to_luck}) ou Frappe mortelle (\pageref{subsec:ab_deadly_strike})
\end{abnamelist}
\textbf{Intrusion de la Meneuse:}
%_____________________________________________________%
\phantomsection\label{sec:focusspeaksfortheland}\section*{ Parle au Nom de la Terre}

\textcolor{gray}{\emph{ Speaks for the Land}}

Votre connexion spirituelle avec la nature et l’environnement vous confère des capacités mystiques.

\begin{abnamelist}
\item Rang 1: Connaissances en milieu sauvage (\pageref{subsec:ab_wilderness_lore}) Graines de fureur (\pageref{subsec:ab_seeds_of_fury})
\item Rang 2:
Feuillage agrippant (\pageref{subsec:ab_grasping_foliage})
\item Rang 3:
Choisissez en une: Apaiser la Bête Sauvage (\pageref{subsec:ab_soothe_the_savage}) ou Communication (\pageref{subsec:ab_communication})
\item Rang 4:
Carnivore sous la Lune (\pageref{subsec:ab_moon_shape})
\item Rang 5:
Eruption d'insectes (\pageref{subsec:ab_insect_eruption})
\item Rang 6:
Choisissez en une: Appeler la tempête (\pageref{subsec:ab_call_the_storm}) ou Tremblement de terre (\pageref{subsec:ab_earthquake})
\end{abnamelist}
\textbf{Intrusion de la Meneuse:}
%_____________________________________________________%
\phantomsection\label{sec:focustalkstomachines}\section*{ Parle aux Machines}

\textcolor{gray}{\emph{ Talks to Machines}}

Vous utilisez votre cerveau organique comme un ordinateur, en interface « sans fil » avec n’importe quel appareil électronique. Vous pouvez les contrôler et les influencer d’une manière que d’autres ne peuvent pas.

\begin{abnamelist}
\item Rang 1: Affinité machine (\pageref{subsec:ab_machine_affinity}) Interface distante (\pageref{subsec:ab_distant_interface})
\item Rang 2:
Charmer une Machine (\pageref{subsec:ab_charm_machine}) Puissance d'attraction (\pageref{subsec:ab_coaxing_power})
\item Rang 3:
Choisissez en une: Commander une Machine (\pageref{subsec:ab_command_machine}) ou Interface intelligente (\pageref{subsec:ab_intelligent_interface})
\item Rang 4:
Combattant de Robot (\pageref{subsec:ab_robot_fighter}) Compagnon machine (\pageref{subsec:ab_machine_companion})
\item Rang 5:
Collecte d'informations (\pageref{subsec:ab_information_gathering})
\item Rang 6:
Choisissez en une: Compagnon Machine Amélioré (\pageref{subsec:ab_improved_machine_companion}) ou Contrôle de Machine (\pageref{subsec:ab_control_machine})
\end{abnamelist}
\textbf{Intrusion de la Meneuse:}
%_____________________________________________________%
\phantomsection\label{sec:focusseparatesmindfrombody}\section*{ Peut Séparer son Esprit de son Corps}

\textcolor{gray}{\emph{ Separates Mind From Body}}

Vous pouvez projeter votre esprit hors de votre corps pour voir des endroits lointains et découvrir des secrets qui autrement resteraient cachés.

\begin{abnamelist}
\item Rang 1: Troisième œil (\pageref{subsec:ab_third_eye})
\item Rang 2:
Esprit ouvert (\pageref{subsec:ab_open_mind}) Sens aiguisés (\pageref{subsec:ab_sharp_senses})
\item Rang 3:
Choisissez en une: Troisième œil itinérant (\pageref{subsec:ab_roaming_third_eye}) ou Trouver ce qui est caché (\pageref{subsec:ab_find_the_hidden})
\item Rang 4:
Capteur (\pageref{subsec:ab_sensor})
\item Rang 5:
Passager psychique (\pageref{subsec:ab_psychic_passenger})
\item Rang 6:
Choisissez en une: Capteur amélioré (\pageref{subsec:ab_improved_sensor}) ou Projection mentale (\pageref{subsec:ab_mental_projection})
\end{abnamelist}
\textbf{Intrusion de la Meneuse:}
%_____________________________________________________%
\phantomsection\label{sec:focuspilotsstarcraft}\section*{ Pilote un Vaisseau Spatial}

\textcolor{gray}{\emph{ Pilots Starcraft}}

Vous êtes un excellent pilote de vaisseau.

\begin{abnamelist}
\item Rang 1: Connaissances en Prêt (\pageref{subsec:ab_flex_lore}) Pilote (\pageref{subsec:ab_pilot})
\item Rang 2:
Mentalement résistant (\pageref{subsec:ab_mentally_tough}) Récupération et confort (\pageref{subsec:ab_salvage_and_comfort})
\item Rang 3:
Choisissez en une: Pilote Expert (\pageref{subsec:ab_expert_pilot}) A l'Aise à Bord (\pageref{subsec:ab_ship_footing}) ou Compagnon machine (\pageref{subsec:ab_machine_companion})
\item Rang 4:
Célérité améliorée (\pageref{subsec:ab_enhanced_speed}) Réseau de capteurs (\pageref{subsec:ab_sensor_array})
\item Rang 5:
Comme le dos de votre main (\pageref{subsec:ab_like_the_back_of_your_hand})
\item Rang 6:
Choisissez en une: Pilote Incomparable (\pageref{subsec:ab_incomparable_pilot}) Compétence avec les attaques (\pageref{subsec:ab_skill_with_attacks}) ou Télécommande (\pageref{subsec:ab_remote_control})
\end{abnamelist}
\textbf{Intrusion de la Meneuse:}
%_____________________________________________________%
\phantomsection\label{sec:focuswearsasheenofice}\section*{ Porte un Eclat de Glace}

\textcolor{gray}{\emph{ Wears a Sheen of Ice}}

Vous maîtrisez la puissance hivernale du froid et de la glace.

\begin{abnamelist}
\item Rang 1: Armure de Glace (\pageref{subsec:ab_ice_armor})
\item Rang 2:
Toucher Glacial (\pageref{subsec:ab_frost_touch})
\item Rang 3:
Choisissez en une: Création de Glace (\pageref{subsec:ab_ice_creation}) ou Toucher de Froid Paralysant (\pageref{subsec:ab_freezing_touch})
\item Rang 4:
Armure de glace résiliente (\pageref{subsec:ab_resilient_ice_armor})
\item Rang 5:
Explosion de froid (\pageref{subsec:ab_cold_burst})
\item Rang 6:
Choisissez en une: Gantelets d'hiver (\pageref{subsec:ab_winter_gauntlets}) ou Tempête de Glace (\pageref{subsec:ab_ice_storm})
\end{abnamelist}
\textbf{Intrusion de la Meneuse:}
%_____________________________________________________%
\phantomsection\label{sec:focuswieldsanenchantedweapon}\section*{ Porte une Arme Enchantée}

\textcolor{gray}{\emph{ Wields An Enchanted Weapon}}

Vous possédez une arme aux capacités étranges, et votre connaissance de ses pouvoirs vous a permis de créer avec elle un style de combat unique.

\begin{abnamelist}
\item Rang 1: Arme Electrique (\pageref{subsec:ab_charge_weapon}) Arme enchantée (\pageref{subsec:ab_enchanted_weapon}) Pouvoir inné (\pageref{subsec:ab_innate_power})
\item Rang 2:
Frappe Explosive (\pageref{subsec:ab_power_crash})
\item Rang 3:
Choisissez en une: Attaque rapide (\pageref{subsec:ab_rapid_attack}) ou Lancer une arme enchantée (\pageref{subsec:ab_throw_enchanted_weapon})
\item Rang 4:
Arme de défense (\pageref{subsec:ab_defending_weapon})
\item Rang 5:
Mouvement enchanté (\pageref{subsec:ab_enchanted_movement})
\item Rang 6:
Choisissez en une: Attaque Tournoyante (\pageref{subsec:ab_spin_attack}) ou Frappe mortelle (\pageref{subsec:ab_deadly_strike})
\end{abnamelist}
\textbf{Intrusion de la Meneuse:}
%_____________________________________________________%
\phantomsection\label{sec:focuswearspowerarmor}\section*{ Porte une Armure Mécanique}

\textcolor{gray}{\emph{ Wears Power Armor}}

Vous portez une armure fantastique.

\begin{abnamelist}
\item Rang 1: Armure motorisée (\pageref{subsec:ab_powered_armor}) Puissance Améliorée (\pageref{subsec:ab_enhanced_might})
\item Rang 2:
Affichage Tête Haute (\pageref{subsec:ab_heads_up_display})
\item Rang 3:
Choisissez en une: Armure Corporelle (\pageref{subsec:ab_fusion_armor}) ou Santé incroyable (\pageref{subsec:ab_incredible_health})
\item Rang 4:
Explosion de Force (\pageref{subsec:ab_force_blast})
\item Rang 5:
Armure Renforcée par Champs de Force (\pageref{subsec:ab_field_reinforced_armor})
\item Rang 6:

\end{abnamelist}
\textbf{Intrusion de la Meneuse:}
%_____________________________________________________%
\phantomsection\label{sec:focusconductsweirdscience}\section*{ Poursuit des Sciences Etranges}

\textcolor{gray}{\emph{ Conducts Weird Science}}

Votre perspicacité et vos capacités surnaturelles font de vous un scientifique capable de prouesses incroyables.

\begin{abnamelist}
\item Rang 1: Analyse en laboratoire (\pageref{subsec:ab_lab_analysis}) Compétences en Connaissances (\pageref{subsec:ab_knowledge_skills})
\item Rang 2:
Modifier l'appareil (\pageref{subsec:ab_modify_device})
\item Rang 3:
Choisissez en une: Mieux vivre grâce à la chimie (\pageref{subsec:ab_better_living_through_chemistry}) ou Santé incroyable (\pageref{subsec:ab_incredible_health})
\item Rang 4:
Compétences en Connaissances (\pageref{subsec:ab_knowledge_skills}) Juste un peu fou (\pageref{subsec:ab_just_a_bit_mad})
\item Rang 5:
Percée scientifique étrange (\pageref{subsec:ab_weird_science_breakthrough})
\item Rang 6:
Choisissez en une: Incroyable exploit scientifique (\pageref{subsec:ab_incredible_feat_of_science}) Champ défensif (\pageref{subsec:ab_defensive_field}) ou Inventeur (\pageref{subsec:ab_inventor})
\end{abnamelist}
\textbf{Intrusion de la Meneuse:}
%_____________________________________________________%
\phantomsection\label{sec:focustakesanimalshape}\section*{ Prend une Forme Animale}

\textcolor{gray}{\emph{ Takes Animal Shape}}

Vous pouvez vous transformer en animal.

\begin{abnamelist}
\item Rang 1: Forme animale (\pageref{subsec:ab_animal_shape})
\item Rang 2:
Apaiser la Bête Sauvage (\pageref{subsec:ab_soothe_the_savage}) Communication (\pageref{subsec:ab_communication})
\item Rang 3:
Choisissez en une: Forme animale plus grande (\pageref{subsec:ab_bigger_animal_shape}) ou Forme de bête Supérieure (\pageref{subsec:ab_greater_beast_form})
\item Rang 4:
Analyse d'animal (\pageref{subsec:ab_animal_scrying})
\item Rang 5:
Difficile à tuer (\pageref{subsec:ab_hard_to_kill})
\item Rang 6:
Choisissez en une: Prêter une forme animale (\pageref{subsec:ab_lend_animal_shape}) ou Vitesse floue (\pageref{subsec:ab_blurring_speed})
\end{abnamelist}
\textbf{Intrusion de la Meneuse:}
%_____________________________________________________%
\phantomsection\label{sec:focuswouldratherbereading}\section*{ Préfèrerait Lire}

\textcolor{gray}{\emph{ Would Rather Be Reading}}

Les livres sont vos amis. Qu'y a-t-il de plus important que la connaissance ? Rien.

\begin{abnamelist}
\item Rang 1: La connaissance, c'est le pouvoir (\pageref{subsec:ab_knowledge_is_power})
\item Rang 2:
Intellect Amélioré Supérieur (\pageref{subsec:ab_greater_enhanced_intellect})
\item Rang 3:
Choisissez en une: Appliquer vos connaissances (\pageref{subsec:ab_applying_your_knowledge}) ou Compétences en Gage (\pageref{subsec:ab_flex_skill})
\item Rang 4:
Connaître l'inconnu (\pageref{subsec:ab_knowing_the_unknown}) La connaissance, c'est le pouvoir (\pageref{subsec:ab_knowledge_is_power})
\item Rang 5:
Intellect Amélioré Supérieur (\pageref{subsec:ab_greater_enhanced_intellect})
\item Rang 6:
Choisissez en une: La connaissance, c'est le pouvoir (\pageref{subsec:ab_knowledge_is_power}) Lire les signes (\pageref{subsec:ab_read_the_signs}) ou Tour de l'Intellect (\pageref{subsec:ab_tower_of_intellect})
\end{abnamelist}
\textbf{Intrusion de la Meneuse:}
%_____________________________________________________%
\phantomsection\label{sec:focusmetesoutjustice}\section*{ Rend la Justice}

\textcolor{gray}{\emph{ Metes Out Justice}}

Vous redressez les torts, protégez les innocents et punissez les coupables.

\begin{abnamelist}
\item Rang 1: Désignation (\pageref{subsec:ab_designation}) Porter un jugement (\pageref{subsec:ab_make_judgment})
\item Rang 2:
Défendre les innocents (\pageref{subsec:ab_defend_the_innocent}) Désignation Améliorée (\pageref{subsec:ab_improved_designation})
\item Rang 3:
Choisissez en une: Défendre tous les innocents (\pageref{subsec:ab_defend_all_the_innocent}) ou Punir le coupable (\pageref{subsec:ab_punish_the_guilty})
\item Rang 4:
Désignation supérieure (\pageref{subsec:ab_greater_designation}) Trouver le coupable (\pageref{subsec:ab_find_the_guilty})
\item Rang 5:
Punir tous les coupables (\pageref{subsec:ab_punish_all_the_guilty})
\item Rang 6:
Choisissez en une: Au diable les coupables (\pageref{subsec:ab_damn_the_guilty}) ou Inspirez les innocents (\pageref{subsec:ab_inspire_the_innocent})
\end{abnamelist}
\textbf{Intrusion de la Meneuse:}
%_____________________________________________________%
\phantomsection\label{sec:focusstandslikeabastion}\section*{ Résiste Comme une Citadelle}

\textcolor{gray}{\emph{ Stands Like a Bastion}}

Votre armure, ainsi que votre taille, votre force, votre entraînement incroyable ou l'amélioration de votre machine, vous rendent difficile à déplacer ou à blesser.  Certains personnages qui Résiste Comme une Citadelle sont peut-être déjà des experts en armure. Ils peuvent choisir une capacité de Rang 1 différente au lieu de Pratique des armures. 

\begin{abnamelist}
\item Rang 1: Défenseur expérimenté (\pageref{subsec:ab_experienced_defender}) Pratique des armures (\pageref{subsec:ab_practiced_in_armor})
\item Rang 2:
Résistez aux éléments (\pageref{subsec:ab_resist_the_elements})
\item Rang 3:
Choisissez en une: Inamovible (\pageref{subsec:ab_unmovable}) Pratique de toutes les armes (\pageref{subsec:ab_practiced_with_all_weapons}) ou Puissance améliorée supérieure (\pageref{subsec:ab_greater_enhanced_might})
\item Rang 4:
Mur vivant (\pageref{subsec:ab_living_wall})
\item Rang 5:
Maîtrise en Armure (\pageref{subsec:ab_mastery_in_armor}) Robustesse (\pageref{subsec:ab_hardiness})
\item Rang 6:
Choisissez en une: Dégâts mortels (\pageref{subsec:ab_lethal_damage}) ou Entraînement au bouclier (\pageref{subsec:ab_shield_training})
\end{abnamelist}
\textbf{Intrusion de la Meneuse:}
%_____________________________________________________%
\phantomsection\label{sec:focussolvesmysteries}\section*{ Résout des Mystères}

\textcolor{gray}{\emph{ Solves Mysteries}}

Vous maîtrisez la déduction et utilisez des faits et des indices pour trouver la réponse.

\begin{abnamelist}
\item Rang 1: Détective (\pageref{subsec:ab_sleuth}) Enquêteur (\pageref{subsec:ab_investigator})
\item Rang 2:
Hors de danger (\pageref{subsec:ab_out_of_harms_way})
\item Rang 3:
Choisissez en une: Compétence avec les attaques (\pageref{subsec:ab_skill_with_attacks}) ou Vous avez étudié (\pageref{subsec:ab_you_studied})
\item Rang 4:
Tirer une conclusion (\pageref{subsec:ab_draw_conclusion})
\item Rang 5:
Désamorcer la situation (\pageref{subsec:ab_defuse_situation})
\item Rang 6:
Choisissez en une: Compétence en Défense Supérieure (\pageref{subsec:ab_greater_skill_with_defense}) ou Prendre l'initiative (\pageref{subsec:ab_seize_the_initiative})
\end{abnamelist}
\textbf{Intrusion de la Meneuse:}
%_____________________________________________________%
\phantomsection\label{sec:focusawakensdreams}\section*{ Réveille les rêves}

\textcolor{gray}{\emph{ Awakens Dreams}}

Vous pouvez extraire des images de rêves et leur donner vie dans le monde éveillé.

\begin{abnamelist}
\item Rang 1: Artisanat des Rêves (\pageref{subsec:ab_dreamcraft}) Science du Sommeil (\pageref{subsec:ab_oneirochemy})
\item Rang 2:
Voleur de rêves (\pageref{subsec:ab_dream_thief})
\item Rang 3:
Choisissez en une: Intellect amélioré (\pageref{subsec:ab_enhanced_intellect}) ou Le rêve devient réalité (\pageref{subsec:ab_dream_becomes_reality})
\item Rang 4:
Rêverie (\pageref{subsec:ab_daydream})
\item Rang 5:
Cauchemar (\pageref{subsec:ab_nightmare})
\item Rang 6:
Choisissez en une: Chambre des rêves (\pageref{subsec:ab_chamber_of_dreams}) ou Champ réactif (\pageref{subsec:ab_reactive_field})
\end{abnamelist}
\textbf{Intrusion de la Meneuse:}
%_____________________________________________________%
\phantomsection\label{sec:focusworksthebackalleys}\section*{ Rôde dans les Bas Quartiers}

\textcolor{gray}{\emph{ Works the Back Alleys}}

Vous avancez sans être vu, volant les riches pour parvenir à vos fins.

\begin{abnamelist}
\item Rang 1: Compétences furtives (\pageref{subsec:ab_stealth_skills})
\item Rang 2:
Contacts avec la pègre (\pageref{subsec:ab_underworld_contacts})
\item Rang 3:
Choisissez en une: Entraînement de guilde (\pageref{subsec:ab_guild_training}) ou Rapide Tromperie (\pageref{subsec:ab_pull_a_fast_one})
\item Rang 4:
Maître voleur (\pageref{subsec:ab_master_thief})
\item Rang 5:
Combattant Hors-la-loi (\pageref{subsec:ab_dirty_fighter})
\item Rang 6:
Choisissez en une: Mettre le paquet (\pageref{subsec:ab_all_out_con}) ou Rat des Allées (\pageref{subsec:ab_alley_rat})
\end{abnamelist}
\textbf{Intrusion de la Meneuse:}
%_____________________________________________________%
\phantomsection\label{sec:focusconsortswiththedead}\section*{ S'Associe avec les Morts}

\textcolor{gray}{\emph{ Consorts With the Dead}}

Les morts répondent à vos questions, et leurs cadavres réanimés vous servent.

\begin{abnamelist}
\item Rang 1: Orateur pour les morts (\pageref{subsec:ab_speaker_for_the_dead})
\item Rang 2:
Nécromancie (\pageref{subsec:ab_necromancy})
\item Rang 3:
Choisissez en une: Esprit parle moi d'ici (\pageref{subsec:ab_reading_the_room}) ou Réparer la chair (\pageref{subsec:ab_repair_flesh})
\item Rang 4:
Réparer la chair (\pageref{subsec:ab_repair_flesh})
\item Rang 5:
Regard terrifiant (\pageref{subsec:ab_terrifying_gaze})
\item Rang 6:
Choisissez en une: Mot de mort (\pageref{subsec:ab_word_of_death}) ou Véritable Nécromancie (\pageref{subsec:ab_true_necromancy})
\end{abnamelist}
\textbf{Intrusion de la Meneuse:}
%_____________________________________________________%
\phantomsection\label{sec:focussoarsonamazingwings}\section*{ S'Envole Grâce à ses Ailes}

\textcolor{gray}{\emph{ Soars On Amazing Wings}}

De nombreux super-héros peuvent voler et certains ont même des ailes. Vous pouvez utiliser vos ailes pour vous déplacer, attaquer et vous défendre.

\begin{abnamelist}
\item Rang 1: Survol (\pageref{subsec:ab_hover}) Vol Court (\pageref{subsec:ab_flight_exertion})
\item Rang 2:
Ailes comme Arme (\pageref{subsec:ab_wing_weapons})
\item Rang 3:
Choisissez en une: Attaque acrobatique (\pageref{subsec:ab_acrobatic_attack}) ou Compagnon volant (\pageref{subsec:ab_flying_companion})
\item Rang 4:
Difficile à toucher (\pageref{subsec:ab_hard_to_hit})
\item Rang 5:
Accélérer (\pageref{subsec:ab_up_to_speed})
\item Rang 6:
Choisissez en une: Cible difficile (\pageref{subsec:ab_hard_target}) ou Maître de la défense (\pageref{subsec:ab_defense_master})
\end{abnamelist}
\textbf{Intrusion de la Meneuse:}
%_____________________________________________________%
\phantomsection\label{sec:focusstretches}\section*{ S'Etire}

\textcolor{gray}{\emph{ Stretches}}

Votre corps est élastique et caoutchouteux, capable de s’étirer sur de grandes longueurs et de se comprimer lorsqu’il est frappé.

\begin{abnamelist}
\item Rang 1: Contorsionniste (\pageref{subsec:ab_contortionist}) Grand Pas (\pageref{subsec:ab_far_step})
\item Rang 2:
Chute en toute sécurité (\pageref{subsec:ab_safe_fall}) Poignée élastique (\pageref{subsec:ab_elastic_grip})
\item Rang 3:
Choisissez en une: Contourner la barrière (\pageref{subsec:ab_bypass_barrier}) ou Détournement (\pageref{subsec:ab_misdirect})
\item Rang 4:
Résilience (\pageref{subsec:ab_resilience})
\item Rang 5:
Libre de se déplacer (\pageref{subsec:ab_free_to_move})
\item Rang 6:
Choisissez en une: Briser les rangs (\pageref{subsec:ab_break_the_ranks}) ou Pas encore mort (\pageref{subsec:ab_not_dead_yet})
\end{abnamelist}
\textbf{Intrusion de la Meneuse:}
%_____________________________________________________%
\phantomsection\label{sec:focusrunsaway}\section*{ S'enfuit}

\textcolor{gray}{\emph{ Runs Away}}

Votre premier réflexe est de fuir le danger, et vous y êtes devenu très fort.

\begin{abnamelist}
\item Rang 1: Devenez défensif (\pageref{subsec:ab_go_defensive})
\item Rang 2:
Célérité améliorée (\pageref{subsec:ab_enhanced_speed}) Rapide à fuir (\pageref{subsec:ab_quick_to_flee})
\item Rang 3:
Choisissez en une: Célérité améliorée supérieure (\pageref{subsec:ab_greater_enhanced_speed}) ou Vitesse de course incroyable (\pageref{subsec:ab_incredible_running_speed})
\item Rang 4:
Détermination croissante (\pageref{subsec:ab_increasing_determination}) Vif d'esprit (\pageref{subsec:ab_quick_wits})
\item Rang 5:
Aller au sol (\pageref{subsec:ab_go_to_ground})
\item Rang 6:
Choisissez en une: Bouquet d'évasion (\pageref{subsec:ab_burst_of_escape}) ou Compétence en Défense Supérieure (\pageref{subsec:ab_greater_skill_with_defense})
\end{abnamelist}
\textbf{Intrusion de la Meneuse:}
%_____________________________________________________%
\phantomsection\label{sec:focussculptshardlight}\section*{ Sculpte la Lumière Solide}

\textcolor{gray}{\emph{ Sculpts Hard Light}}

Vous créez des objets physiques à partir d’une lumière solide que vous pouvez utiliser à des fins offensives et défensives.

\begin{abnamelist}
\item Rang 1: Lueur automatique (\pageref{subsec:ab_automatic_glow}) Lumière temporaire (\pageref{subsec:ab_temporary_light})
\item Rang 2:
Force enchevêtrante (\pageref{subsec:ab_entangling_force})
\item Rang 3:
Choisissez en une: Lumière plus Forte (\pageref{subsec:ab_harder_light}) ou Sculpter la lumière (\pageref{subsec:ab_sculpt_light})
\item Rang 4:
Intellect Amélioré Supérieur (\pageref{subsec:ab_greater_enhanced_intellect})
\item Rang 5:
Lumière sculptée améliorée (\pageref{subsec:ab_improved_sculpt_light})
\item Rang 6:
Choisissez en une: Champ défensif (\pageref{subsec:ab_defensive_field}) ou Vol (\pageref{subsec:ab_flight})
\end{abnamelist}
\textbf{Intrusion de la Meneuse:}
%_____________________________________________________%
\phantomsection\label{sec:focusfightsdirty}\section*{ Se Bat Sans Respecter de Règle}

\textcolor{gray}{\emph{ Fights Dirty}}

Vous ferez n'importe quoi pour gagner un combat : mordre, gratter, donner un coup de pied, tromper et pire encore.

\begin{abnamelist}
\item Rang 1: Pisteur (\pageref{subsec:ab_tracker}) Traqueur (\pageref{subsec:ab_stalker})
\item Rang 2:
Furtif (\pageref{subsec:ab_sneak}) Proie (\pageref{subsec:ab_quarry})
\item Rang 3:
Choisissez en une: Attaque surprise (\pageref{subsec:ab_surprise_attack}) ou Trahison (\pageref{subsec:ab_betrayal})
\item Rang 4:
Guerrier Capable (\pageref{subsec:ab_capable_warrior}) Jeux d'esprit (\pageref{subsec:ab_mind_games})
\item Rang 5:
Utilisation de l'environnement (\pageref{subsec:ab_using_the_environment})
\item Rang 6:
Choisissez en une: Meurtrier (\pageref{subsec:ab_murderer}) ou Torsion du couteau (\pageref{subsec:ab_twisting_the_knife})
\end{abnamelist}
\textbf{Intrusion de la Meneuse:}
%_____________________________________________________%
\phantomsection\label{sec:focuswieldstwoweaponsatonce}\section*{ Se Bat avec Deux Armes à la fois}

\textcolor{gray}{\emph{ Wields Two Weapons at Once}}

Vous portez de l'acier dans chaque main, prêt à affronter n'importe quel ennemi.

\begin{abnamelist}
\item Rang 1: Deux Armes Légères (\pageref{subsec:ab_dual_light_wield})
\item Rang 2:
Double frappe (\pageref{subsec:ab_double_strike}) Infiltrateur (\pageref{subsec:ab_infiltrator})
\item Rang 3:
Choisissez en une: Coupe Précise (\pageref{subsec:ab_precise_cut}) ou Deux Armes Moyennes (\pageref{subsec:ab_dual_medium_wield})
\item Rang 4:
Double défense (\pageref{subsec:ab_dual_defense})
\item Rang 5:
Double Distraction (\pageref{subsec:ab_dual_distraction})
\item Rang 6:
Choisissez en une: Attaque Tournoyante (\pageref{subsec:ab_spin_attack}) ou Attaque de désarmement (\pageref{subsec:ab_disarming_attack})
\end{abnamelist}
\textbf{Intrusion de la Meneuse:}
%_____________________________________________________%
\phantomsection\label{sec:focusinfiltrates}\section*{ Se Cache dans les Ombres}

\textcolor{gray}{\emph{ Infiltrates}}

La subtilité, la ruse et la furtivité vous permettent d'accéder là où les autres ne peuvent pas aller.

\begin{abnamelist}
\item Rang 1: Compétences furtives (\pageref{subsec:ab_stealth_skills}) Sentir les Attitudes (\pageref{subsec:ab_sense_attitudes})
\item Rang 2:
Evitement (\pageref{subsec:ab_flight_not_fight}) Usurper l'identité (\pageref{subsec:ab_impersonate})
\item Rang 3:
Choisissez en une: Compétence avec les attaques (\pageref{subsec:ab_skill_with_attacks}) ou Conscience (\pageref{subsec:ab_awareness})
\item Rang 4:
Invisibilité (\pageref{subsec:ab_invisibility})
\item Rang 5:
Esquive (\pageref{subsec:ab_evasion})
\item Rang 6:
Choisissez en une: Lavage de cerveau (\pageref{subsec:ab_brainwashing}) ou Saut de Côté (\pageref{subsec:ab_spring_away})
\end{abnamelist}
\textbf{Intrusion de la Meneuse:}
%_____________________________________________________%
\phantomsection\label{sec:focusrages}\section*{ Se Met en Rage}

\textcolor{gray}{\emph{ Rages}}

Quand vous devenez fou, tout le monde vous craint.

\begin{abnamelist}
\item Rang 1: Frénésie (\pageref{subsec:ab_frenzy})
\item Rang 2:
Habiletés motrices (\pageref{subsec:ab_movement_skills}) Puissance améliorée supérieure (\pageref{subsec:ab_greater_enhanced_might})
\item Rang 3:
Choisissez en une: Combattant sans armure (\pageref{subsec:ab_unarmored_fighter}) ou Frappe Renversante (\pageref{subsec:ab_power_strike})
\item Rang 4:
Frénésie supérieure (\pageref{subsec:ab_greater_frenzy})
\item Rang 5:
Attaquez et attaquez encore (\pageref{subsec:ab_attack_and_attack_again})
\item Rang 6:
Choisissez en une: Dégâts mortels (\pageref{subsec:ab_lethal_damage}) ou Potentiel amélioré plus important (\pageref{subsec:ab_greater_enhanced_potential})
\end{abnamelist}
\textbf{Intrusion de la Meneuse:}
%_____________________________________________________%
\phantomsection\label{sec:focusbearsahalooffire}\section*{ Se Revêt d'un Halo de Feu}

\textcolor{gray}{\emph{ Bears a Halo of Fire}}

Vous pouvez envelopper votre corps de flammes, ce qui vous protège et nuit à vos ennemis.

\begin{abnamelist}
\item Rang 1: Manteau de flammes (\pageref{subsec:ab_shroud_of_flame})
\item Rang 2:
Lancement de flammes (\pageref{subsec:ab_hurl_flame})
\item Rang 3:
Choisissez en une: Ailes de Feu (\pageref{subsec:ab_wings_of_fire}) ou Main ardente du destin (\pageref{subsec:ab_fiery_hand_of_doom})
\item Rang 4:
Lame de Feu (\pageref{subsec:ab_flameblade})
\item Rang 5:
Vrilles de feu (\pageref{subsec:ab_fire_tendrils})
\item Rang 6:
Choisissez en une: Piste Infernale (\pageref{subsec:ab_inferno_trail}) ou Serviteur du Feu (\pageref{subsec:ab_fire_servant})
\end{abnamelist}
\textbf{Intrusion de la Meneuse:}
%_____________________________________________________%
\phantomsection\label{sec:focusshrinkstominutesize}\section*{ Se Réduit à une Taille Infime}

\textcolor{gray}{\emph{ Shrinks To Minute Size}}

Vous pouvez réduire à la taille d'un bug et, avec suffisamment d'expérience, encore plus petit.

\begin{abnamelist}
\item Rang 1: Rétrécir (\pageref{subsec:ab_shrink}) Sans être remarqué (\pageref{subsec:ab_beneath_notice})
\item Rang 2:
Avantages d'être petit (\pageref{subsec:ab_advantages_of_being_small}) Plus petit (\pageref{subsec:ab_smaller})
\item Rang 3:
Choisissez en une: Agrandir (\pageref{subsec:ab_enlarge}) ou Changements Rapides (\pageref{subsec:ab_quick_switch})
\item Rang 4:
Petit vol (\pageref{subsec:ab_small_flight})
\item Rang 5:
Rétrécir les autres (\pageref{subsec:ab_shrink_others})
\item Rang 6:
Choisissez en une: Minuscule (\pageref{subsec:ab_tiny}) ou Plus grand (\pageref{subsec:ab_bigger})
\end{abnamelist}
\textbf{Intrusion de la Meneuse:}
%_____________________________________________________%
\phantomsection\label{sec:focussiphonspower}\section*{ Siphonne les Pouvoirs}

\textcolor{gray}{\emph{ Siphons Power}}

Vous aspirez le pouvoir des machines et des créatures afin de vous donner du pouvoir.  Les robots et autres machines vivantes doivent être traités comme des créatures, et non comme des machines, dans le but d’en siphonner l’énergie. 

\begin{abnamelist}
\item Rang 1: Drain de Machine (\pageref{subsec:ab_drain_machine})
\item Rang 2:
Drain de Créature (\pageref{subsec:ab_drain_creature})
\item Rang 3:
Choisissez en une: Drain à distance (\pageref{subsec:ab_drain_at_a_distance}) ou Draîner la vie (\pageref{subsec:ab_unraveling_consumption})
\item Rang 4:
Stocker l'énergie (\pageref{subsec:ab_store_energy})
\item Rang 5:
Partagez le pouvoir (\pageref{subsec:ab_share_the_power})
\item Rang 6:
Choisissez en une: Libération explosive (\pageref{subsec:ab_explosive_release}) ou Siphon solaire (\pageref{subsec:ab_sun_siphon})
\end{abnamelist}
\textbf{Intrusion de la Meneuse:}
%_____________________________________________________%
\phantomsection\label{sec:focusslaysmonsters}\section*{ Tue les Monstres}

\textcolor{gray}{\emph{ Slays Monsters}}

Vous tuez des monstres.  Bien que manier une épée dans un environnement où les gens ne portent généralement pas de telles armes soit acceptable, vous pouvez modifier les capacités liées à l'épée de Tue les Monstres pour utiliser une arme différente, comme un pistolet à balles d'argent. 

\begin{abnamelist}
\item Rang 1: Connaissance des monstres (\pageref{subsec:ab_monster_lore}) Fléau des Monstres (\pageref{subsec:ab_monster_bane}) Pratique des épées (\pageref{subsec:ab_practiced_with_swords})
\item Rang 2:
Volonté de Légende (\pageref{subsec:ab_will_of_legend})
\item Rang 3:
Choisissez en une: Epéiste entraîné (\pageref{subsec:ab_trained_slayer}) Détournement (\pageref{subsec:ab_misdirect}) ou Fléau des Monstres Amélioré (\pageref{subsec:ab_improved_monster_bane})
\item Rang 4:
Continuez le combat (\pageref{subsec:ab_fight_on})
\item Rang 5:
Compétence en Attaque Supérieure (\pageref{subsec:ab_greater_skill_with_attacks})
\item Rang 6:
Choisissez en une: Fléau des Monstres Géants (\pageref{subsec:ab_heroic_monster_bane}) ou Meurtrier (\pageref{subsec:ab_murderer})
\end{abnamelist}
\textbf{Intrusion de la Meneuse:}
%_____________________________________________________%
\phantomsection\label{sec:focususeswildmagic}\section*{ Utilise la Magie Sauvage}

\textcolor{gray}{\emph{ Uses Wild Magic}}

Lanceur de sorts qui apprend une variété de sorts au lieu de se concentrer sur un seul type de magie.

\begin{abnamelist}
\item Rang 1: Lancement de Cypher (\pageref{subsec:ab_cypher_casting}) Répertoire magique (\pageref{subsec:ab_magical_repertoire})
\item Rang 2:
Répertoire étendu (\pageref{subsec:ab_expanded_repertoire})
\item Rang 3:
Choisissez en une: Magie Sauvage plus rapide (\pageref{subsec:ab_faster_wild_magic}) ou Sursaut de Cypher (\pageref{subsec:ab_cypher_surge})
\item Rang 4:
Répertoire étendu (\pageref{subsec:ab_expanded_repertoire})
\item Rang 5:
Entraînement magique (\pageref{subsec:ab_magical_training})
\item Rang 6:
Choisissez en une: Instinct de Magie Sauvage (\pageref{subsec:ab_wild_insight}) ou Maximiser le Cypher (\pageref{subsec:ab_maximize_cypher})
\end{abnamelist}
\textbf{Intrusion de la Meneuse:}
%_____________________________________________________%
\phantomsection\label{sec:focususeswildmagic}\section*{ Utilise la Magie Sauvage}

\textcolor{gray}{\emph{ Uses Wild Magic}}

Lanceur de sorts qui apprend une variété de sorts au lieu de se concentrer sur un seul type de magie.

\begin{abnamelist}
\item Rang 1: Lancement de Cypher (\pageref{subsec:ab_cypher_casting}) Répertoire magique (\pageref{subsec:ab_magical_repertoire})
\item Rang 2:
Répertoire étendu (\pageref{subsec:ab_expanded_repertoire})
\item Rang 3:
Choisissez en une: Magie Sauvage plus rapide (\pageref{subsec:ab_faster_wild_magic}) ou Sursaut de Cypher (\pageref{subsec:ab_cypher_surge})
\item Rang 4:
Répertoire étendu (\pageref{subsec:ab_expanded_repertoire})
\item Rang 5:
Entraînement magique (\pageref{subsec:ab_magical_training})
\item Rang 6:
Choisissez en une: Instinct de Magie Sauvage (\pageref{subsec:ab_wild_insight}) ou Maximiser le Cypher (\pageref{subsec:ab_maximize_cypher})
\end{abnamelist}
\textbf{Intrusion de la Meneuse:}
%_____________________________________________________%
\phantomsection\label{sec:focusmoveslikethewind}\section*{ Va Comme le Vent}

\textcolor{gray}{\emph{ Moves Like the Wind}}

Vous pouvez vous déplacer si vite que vous devenez flou.

\begin{abnamelist}
\item Rang 1: Célérité améliorée supérieure (\pageref{subsec:ab_greater_enhanced_speed}) Pied Léger (\pageref{subsec:ab_fleet_of_foot})
\item Rang 2:
Difficile à toucher (\pageref{subsec:ab_hard_to_hit})
\item Rang 3:
Choisissez en une: Célérité améliorée supérieure (\pageref{subsec:ab_greater_enhanced_speed}) ou Sursaut de Célérité (\pageref{subsec:ab_speed_burst})
\item Rang 4:
En un Clin d'oeil (\pageref{subsec:ab_blink_of_an_eye})
\item Rang 5:
Difficile à voir (\pageref{subsec:ab_hard_to_see})
\item Rang 6:
Choisissez en une: Sursaut de Célérité Parfait (\pageref{subsec:ab_perfect_speed_burst}) ou Vitesse de course incroyable (\pageref{subsec:ab_incredible_running_speed})
\end{abnamelist}
\textbf{Intrusion de la Meneuse:}
%_____________________________________________________%
\phantomsection\label{sec:focuslivesinthewilderness}\section*{ Vit dans la Nature Sauvage}

\textcolor{gray}{\emph{ Lives in the Wilderness}}

Vous pouvez survivre dans des étendues sauvages où d'autres périssent.

\begin{abnamelist}
\item Rang 1: Puissance Améliorée (\pageref{subsec:ab_enhanced_might}) Vie en pleine nature (\pageref{subsec:ab_wilderness_life})
\item Rang 2:
Explorateur de la Nature (\pageref{subsec:ab_wilderness_explorer}) Vivre de la terre (\pageref{subsec:ab_living_off_the_land})
\item Rang 3:
Choisissez en une: Encouragement de la Nature (\pageref{subsec:ab_wilderness_encouragement}) ou Sens et sensibilités animales (\pageref{subsec:ab_animal_senses_and_sensibilities})
\item Rang 4:
Sensibilisation à la nature sauvage (\pageref{subsec:ab_wilderness_awareness})
\item Rang 5:
La nature est de votre côté (\pageref{subsec:ab_the_wild_is_on_your_side})
\item Rang 6:
Choisissez en une: Camouflage sauvage (\pageref{subsec:ab_wild_camouflage}) ou Faire Corps avec la Nature (\pageref{subsec:ab_one_with_the_wild})
\end{abnamelist}
\textbf{Intrusion de la Meneuse:}
%_____________________________________________________%
\phantomsection\label{sec:focusseesbeyond}\section*{ Voit Au-Delà}

\textcolor{gray}{\emph{ Sees Beyond}}

Vous avez un sens psychique qui vous permet de voir ce que les autres ne peuvent pas voir.

\begin{abnamelist}
\item Rang 1: Voir l'invisible (\pageref{subsec:ab_see_the_unseen})
\item Rang 2:
Voir à travers la matière (\pageref{subsec:ab_see_through_matter})
\item Rang 3:
Choisissez en une: Capteur (\pageref{subsec:ab_sensor}) ou Trouver ce qui est caché (\pageref{subsec:ab_find_the_hidden})
\item Rang 4:
Visualisation à distance (\pageref{subsec:ab_remote_viewing})
\item Rang 5:
Voir à travers le temps (\pageref{subsec:ab_see_through_time})
\item Rang 6:
Choisissez en une: Conscience totale (\pageref{subsec:ab_total_awareness}) ou Projection mentale (\pageref{subsec:ab_mental_projection})
\end{abnamelist}
\textbf{Intrusion de la Meneuse:}
%_____________________________________________________%
\phantomsection\label{sec:focusfliesfasterthanabullet}\section*{ Vole Plus Vite qu'une Balle}

\textcolor{gray}{\emph{ Flies Faster Than a Bullet}}

Vous pouvez voler et vous êtes super fort, difficile à blesser et rapide aussi. Y a-t-il quelque chose que vous ne pouvez pas faire ?

\begin{abnamelist}
\item Rang 1: Survol (\pageref{subsec:ab_hover})
\item Rang 2:
Potentiel amélioré plus important (\pageref{subsec:ab_greater_enhanced_potential})
\item Rang 3:
Choisissez en une: Réserves cachées (\pageref{subsec:ab_hidden_reserves}) ou Voir à travers la matière (\pageref{subsec:ab_see_through_matter})
\item Rang 4:
Accélérer (\pageref{subsec:ab_up_to_speed}) En un Clin d'oeil (\pageref{subsec:ab_blink_of_an_eye})
\item Rang 5:
Pas encore mort (\pageref{subsec:ab_not_dead_yet})
\item Rang 6:
Choisissez en une: Ignorer l'Affliction (\pageref{subsec:ab_ignore_affliction}) ou Lumière brûlante (\pageref{subsec:ab_burning_light})
\end{abnamelist}
\textbf{Intrusion de la Meneuse:}
%_____________________________________________________%
\phantomsection\label{sec:focustravelsthroughtime}\section*{ Voyage à Travers le Temps}

\textcolor{gray}{\emph{ Travels Through Time}}

Vous pouvez voir à travers le temps, essayer de le traverser et éventuellement même le parcourir.  Bien que tous les choix de personnages soient soumis à l'approbation du MJ, Voyage à Travers le Temps est un sujet sur lequel le MJ et le joueur devraient probablement avoir une longue conversation à l'avance, afin que le joueur connaisse les règles du voyage dans le temps (le cas échéant) qui existent dans le réglage du MJ. Un personnage avec cette concentration peut modifier radicalement un décor, si les règles du voyage dans le temps le permettent. 

\begin{abnamelist}
\item Rang 1: Anticipation (\pageref{subsec:ab_anticipation})
\item Rang 2:
Voir Historique (\pageref{subsec:ab_see_history})
\item Rang 3:
Choisissez en une: Accélération temporelle (\pageref{subsec:ab_temporal_acceleration}) ou Boucle temporelle (\pageref{subsec:ab_time_loop})
\item Rang 4:
Dislocation temporelle (\pageref{subsec:ab_temporal_dislocation})
\item Rang 5:
Doppelganger Temporel (\pageref{subsec:ab_time_doppelganger})
\item Rang 6:
Choisissez en une: Appel à travers le temps (\pageref{subsec:ab_call_through_time}) ou Voyage dans le temps (\pageref{subsec:ab_time_travel})
\end{abnamelist}
\textbf{Intrusion de la Meneuse:}
%%%%%%%%%%%%%%%%%%%%%%%%%%%%%%%%%%%%%%%%%%%%%%%%%%%%%%%%%%%%%%%%%%%%%%%
\section*{Créer un nouveau Focus}
Cette section fournit tout ce dont vous avez besoin pour créer vos propres Focus.

Chaque Focus a un thème principal, qu'il s'agisse d'exploration, de manipulation d'énergie ou simplement d'infliger beaucoup de dégâts au combat. Ces grandes classifications sont appelées catégories de Focus.

Chaque catégorie de Focus a un thème principal, suivi de directives de sélection qui décrivent comment choisir les capacités pour chaque rang du chapitre Capacités, du rang 1 au rang 6.

Le Focus nouvellement créée doit être nommée sous la forme d'un verbe, comme Contrôle les Bêtes Sauvages ou Demeure dans la pierre. Par exemple, un Focus utilisant le feu créée en suivant les directives de la catégorie des Focus de manipulation d'énergie pourrait être appelé Se Revêt d' un Halo de Feu (l'un des exemples de Focus de ce chapitre). Alternativement, un nouveau Focus utilisant le feu devrait recevoir un tout nouveau nom comme Attiser les flammes de l'Apocalypse ou Allumer les feux avec une pensée.

\section*{Catégories de Focus}
\begin{itemize}
    \item Basic (\pageref{subsec:basic})
    \item Combat défensif  (\pageref{subsec:combat_defensif})
    \item Combat offensif  (\pageref{subsec:combat_defensif})
    \item Expertise des mouvements  (\pageref{subsec:expertise_des_mouvements})
    \item Exploration (\pageref{subsec:exploration})
    \item Influence  (\pageref{subsec:influence})
    \item Irrégulier  (\pageref{subsec:irregulier})
    \item Manipulation d'énergie (\pageref{subsec:manipulation_denergie})
    \item Manipulation de l'environement (\pageref{subsec:manipulation_de_lenvironement})
    \item Soutien (\pageref{subsec:soutien})
    \item Utilisation d'Alliés (\pageref{subsec:utilisation_dallies})
\end{itemize}

\section*{Choisir une Capacité en fonction de la Puissance Relative}

Les indications de sélection des capacités vous invitent à choisir une capacité parmi l'une des trois gammes : Rang bas, Rang moyen et Rang élevé. Ces plages correspondent aux «grades» de puissance donnés pour chaque capacité. Ces capacités sont ensuite classées en catégories de capacités en fonction du type de choses qu'ils font : les capacités qui améliorent les attaques physiques sont dans la catégorie des compétences d'attaque, les capacités qui aident les alliés sont dans la catégorie de soutien, et ainsi de suite. Recherchez les notes et les catégories dans la section Catégories de capacités et puissance relative du chapitre Capacités.

Les capacités de rang bas sont mieux adaptées aux options de Focus aux rangs 1 et 2. Les capacités de rang intermédiaire sont mieux adaptées aux options de Focus aux rangs 3 et 4. Les capacités de rang élevé sont mieux adaptées aux options de Focus aux rangs 5 et 6.

Cela dit, vous trouverez parfois approprié d'attribuer une capacité de bas rang au rang 3 ou 4, ou peut-être une capacité de milieu de gamme au rang 1 ou 2. Faites-le avec parcimonie, mais ne l'excluez pas. C'est peut-être le seul moyen d'obtenir toutes les capacités souhaitées pour le Focus que vous développez. Les capacités de rang supérieur coûtent généralement plus de points de réserve à utiliser. Ainsi, si une capacité de milieu de gamme est disponible au rang 1 ou 2, ou qu'une capacité de rang supérieur est disponible au rang 3 ou 4, le coût plus élevé sera un facteur d'équilibrage.

\section*{Equilibrer les Capacités}
Les indications au sein de chaque catégorie contribuent grandement à garantir que le Focus que vous développez sera équilibrée. Parfois, il peut être approprié d'accorder une capacité de faible puissance avec une capacité normale à un rang donné, en fonction des besoins du focus. Une "capacité de faible puissance" est délibérément ouverte à l'interprétation du MJ, mais d'une manière générale, elle ne devrait pas être plus puissante qu'une capacité de bas rang (c'est-à-dire une capacité qui est normalement disponible au rang 1 ou 2).

Par exemple, quelqu'un qui utilise le froid pourrait être capable de créer de petites sculptures de neige en plus d'émettre un rayon froid. Quelqu'un qui utilise de l'électricité pourrait être en mesure de recharger un artefact épuisé ou disposer d'un atout pour gérer les systèmes électriques. Et ainsi de suite.

Souvent, les indications de Focus mentionnent cela comme une possibilité. Cependant, vous disposez d'une grande latitude pour décider si un Focus nécessite une capacité supplémentaire, même si les indications de ce rang n'en indiquent pas. Si vous ajoutez une capacité, ou s'il existe une capacité de puissance plus élevée dans un rang qui ne devrait normalement pas l'avoir, cela peut signifier que le choix donné au rang suivant, ou au rang précédent, n'est pas aussi bon. Équilibrer un Focus est un peu un art. Résistez à l'envie de donner trop de puissance à le Focus, mais ne la sous-estimez pas non plus.

\section*{Les Indications de Capacité ne sont pas Prescriptives}
Chaque catégorie de Focus fournit une ligne directrice sur le type de capacité que vous devez sélectionner à chaque rang. Mais ne considérez pas les indications comme quelque chose que vous ne pouvez pas modifier. Elles ne sont pas prescriptives; elles ne sont qu'un point de départ. Vous souhaiterez peut-être varier le type de capacité d'un rang particulier qui n'est pas indiqué dans les indications. Tant que la capacité choisie se situe dans la courbe de puissance attendue pour ce Rang, tout va bien. L'indication' n'est pas censée être une camisole de force.

Par exemple, si vous construisez un Focus d'utilisation du froid pour un jeu se déroulant dans un genre fantastique, vous pouvez décider qu'une capacité qui invoque un démon est un meilleur choix à un rang particulier qu'une capacité qui inflige des dégâts dans une zone, ce que demande l'indication Rang 5 pour la manipulation de l'énergie. Faire le changement est probablement particulièrement valable si vous appelez votre nouveau focus quelque chose comme Cannalise le Neuvième Cercle.

\section*{Echange de Capacité}
Si vous créez un Focus et que vous pensez qu'elle devrait fournir une suite de capacités au premier rang qui la surchargeraient mécaniquement, vous avez la possibilité d'en ajouter une en tant que capacité « d'échange ». Pour ce faire, il suffit de permettre à un personnage d'échanger l'une de ses capacités de type contre une capacité de Focus de moindre rang. La capacité est acquise à la place d'une des capacités normalement accordées par le type du personnage.

\section*{Concept et Categorie}
Choisir de créer un Focus qui utilise un concept particulier --- par exemple, créer des illusions --- ne vous oblige pas à créer un Focus dans une catégorie particulière --- dans ce cas, la manipulation de l'environnement. Un Focus peut être construite de différentes manières en utilisant une énergie, un outil ou un concept particulier, chacun conduisant finalement à un Focus qui fournit des résultats différents. Tout dépend de vos objectifs. Dans ce cas, la création d'illusions pourrait être utilisée pour influencer les autres, ce qui plaide en faveur d'un Focus basée sur les indications de la catégorie d'influence.

De la même manière, si un Focus accorde à un personnage la possibilité d'invoquer une sorte de force ou d'énergie, cela ne signifie pas que le Focus doit automatiquement être construite en utilisant les indications de la catégorie de manipulation d'énergie (même si bien sûr cela serait possible si attaquer et vous protéger avec cette énergie est le but). Mais un Focus pourrait être construite pour accorder des capacités de création d'énergie ou de force principalement axées sur la résistance, cela suggère une orientation vers le combat défensif (quelqu'un qui peut encaisser beaucoup de dégats dans un combat); ou des capacités se concentrant sur tirer avec le souci principal de maximiser les dégâts, suggérant ainsi un Focus de catégorie combat offensif ; ou alors vous créez un suivant composé de cette énergie ou force, suggérant ainsi un Focus de catégorie Utilisation d'Alliés (c'est-à-dire quelqu'un qui utilise des créatures aidantes, des PNJ, ou même qui duplique des versions d'eux-mêmes pour vous donner un coup de pouce).

Voici un autre exemple : la Motivaton Contrôle la Gravité pourrait éventuellement être un Focus de catégorie manipulation de l'environnement ou de catégorie manipulation d'énergie. Cela dépend si l'accent est davantage mis sur l'écrasement et le maintien des objets en place (manipulation de l'environnement) ou sur le fait de faire exploser des objets et de se protéger grâce à la gravité (manipulation d'énergie).

La même souplesse du concept s'applique à d'autres domaines. Par exemple, si quelqu'un est capable d'invoquer et de modeler de la terre brute, il peut l'utiliser pour se transformer en un être de pierre (combat défensif), pour battre des ennemis (combat offensif), ou pour créer des murs, des barricades et des boucliers pour protéger leurs alliés (soutien).

Si vous recherchez une capacité et que vous n'arrivez pas à trouver celle qui vous convient dans le vaste catalogue du chapitre Capacités, envisagez d'en modifier une afin d'en créer une nouvelle (et pour accomplir ce dont vous avez besoin). Cette modification consiste à utiliser les mécanismes sous-jacents d'une capacité tels qu'ils sont écrits, mais que vous en modifiez les effets visibles d'une manière ou d'une autre. Par exemple, vous êtes peut-être en train de créer un nouvel objectif de déplacement de terre, mais vous ne parvenez pas à trouver suffisamment de capacités liées à la terre pour répondre à vos besoins. Il est assez facile de modifier d'autres capacités pour qu'ils utilisent la terre au lieu du feu, du froid ou du magnétisme. Par exemple, Ailes de Feu pourrait devenir Ailes de Terre, Armure de Glace pourrait devenir Armure de Terre, et ainsi de suite. Ces altérations ne changent rien sauf le type de dégâts et les éventuelles répercussions (par exemple, Ailes de Terre pourrait générer des nuages de poussière dans leur sillage).

\section*{Des Capacités qui font référence à d'autres Capacités}
Certaines capacités du chapitre Capacités font référence à d'autres capacités. Si,pour votre Focus ou votre type, vous sélectionnez une capacité qui fait référence ou modifie une capacité de rang inférieur, incluez également cette capacité de rang inférieur dans votre type ou focus en tant que sélection qu'un PJ peut faire à un rang inférieur.

\section*{Création d'une toute nouvelle Capacité}
Vous pouvez aller plus loin que la modification superficielle et créer une ou plusieurs nouvelles capacités. Ce faisant, essayez de trouver quelque chose d'aussi proche que possible de l'effet souhaité, puis utilisez-le comme modèle. Dans tous les cas, décider du coût d'une capacité en ce qui concerne la réserve d'un personnage est l'un des aspects les plus importants pour obtenir une bonne capacité.

Vous avez pu remarqué que les capacités de haut-rang sont les plus coûteuses. C'est en parti parcequ'elles permettent plus de choses, mais aussi parceque les personnages de haut-rang ont plus d'Avantage que les personnages de rang inférieur, ce qui signifie qu'ils dépensent moins de points dans leurs Réserves. Un personnage de troisème rang avec un Avantage de 3 dans la Réserve appropriée ne dépensera aucun point pour une capacité qui coûte 3 points ou moins. C'est parfait pour les capacités de rang inférieur, mais vous devez plutôt vouloir qu'un personnage réfléchisse un petit peu avant d'utiliserleurs capacités les plus puissantes. Cela veut dire que les capacités devraient coûter au moins 1 point de plus que l'Avantage que le personnage est supposé avoir à ce rang. (Souvent un personnage aura un Avantage dans les Réserve principale égal à son rang.)

Une bonne règle approximative est qu'une capacité typique devrait coûter autant de point que son rang.

\section*{Choisir des intrusions de MJ}
Pensez aux genres de choses qui pourraient surprendre, alarmer, ou tourner à la catastrophe pour quelqu'un avec le Focus qui vient d'être créée, et assignez les en tant qu'intrusion de MJ pour ce Focus. En général c'est souvent fait de manière improvisée en cours de partie. Mais en leur accordant un peu de réflexion pendant l'élaboration du focus, quand les idées sont toutes fraîches dans votre tête, cela a de bonnes chances de fournir des options diaboliques.
%%%%%%%%%%%%%%%%%%%%%%%%%%%%%%%%%%%%%%%%%%%%%%%%%%%%%%%%%%%%%%%%%%%
\section*{Catégories de Focus}
%%%%%%%%%%%%%%%%%%%%%%%%%%%%%%%%%%%%%%%%%%%%%%%%%%
\section*{Basic}
\label{subsec:basic}
Les Focus qui reposent principalement sur la fourniture d'un entrainement à des compétences, d'atouts pour les tâches et d'améliorations des Réserves de statistiques et des Avantages afin d'améliorer un personnage entrent dans la catégorie Basic. Un thème général est également inclus, comme pour la plupart des autres catégories, qui donne un sens aux différentes capacités de base fournies.

De plus, comme les avantages apportés par de tels Focus sont pour la plupart simples (généralement à quelques exceptions près), la plupart des Focus Basic seraient également appropriés pour des campagnes non fantastiques où la magie, la superscience ou les capacités psychiques n'entrent normalement pas en jeu. Cela dit, ce n'est pas parce que les capacités accordées par les Focus Basic sont simples qu'elles ne sont pas puissantes lorsqu'elles sont combinées avec les capacités accordées par le type, le descripteur, les cyphers et d'autres aspects du personnage.

\textbf{Connexion} Choisissez quatre connexions pertinentes dans la liste des Connections de Focus.
\begin{description}
    \item[Equipement Supplémentaire] N'importe quel objet nécessaire pour accomplir le thème général du focus. Par exemple, un Focus appelée Préfèrerait Lire devrait fournir quelques livres au personnage. Un Focus appélée Construit et Répar" devrait fournir un ensemble d'outils.

    \item[Suggestions d'Effet Mineur] La prochaine action est facilité.

    \item[Suggestions d'Effet Majeur] Faites un jet de Récupération gratuit, qui ne prend pas une action et qui ne compte pas dans le total des jets de récupération de la journée.
\end{description}

La liste ci-après ne sont que des exemples et n'est pas une liste complète de toutes les Focus possibles pour cette catégorie.

* Ne Fait pas Grand Chose
* Interprète la Loi
* Apprend Rapidement
* Construit et Répare
* Préfèrerait Lire

\textbf{Indications pour la Sélection de Capacités}
\begin{description}
\item[Rang1] Choisissez une Capacité qui donne un entrainement ou un atout aux compétences associées au thème du focus, ou qui donne 5 ou 6 points à une Réserve particulière.
Autrement, choisissez une Capacité qui donne seulement 2 ou 3 points à une Réserve spécifique et une Capacité qui donne un entrainement ou un atout dans seulement une tâche.

\item[Rang 2] Choisissez n'importe quelle sorte de Capacité qui n'a pas été sélectionnée au rang 1.

\item[Rang 3] Choisissez deux Capacités de rang intermédiare. Donnez-les toutes deux en tant qu'options pour le Focus; un PJ choisira l'une ou l'autre.
Une option pourrait être une Capacité non-fantastique qui améliore les capacités du personnage dans le thème du focus. Par exemple, si le thème implique de faire attention d'une certaine manière, une capacité de récupération d'information serait appropriée.

L'autre option peut être de, soit améliorer l'Avantage du personnage dans une statistique, soit fournir au personnage une forme de défence.

\item[Rang 4] Choisissez une autre Capacité qui donne un entrainement supplémentaire ou un atout à des compétences associées avec le thème du focus, ou qui donne 5 ou 6 points à la Réserve la plus appropriée pour le Focus, ou choisissez deux Capacités qui fournissent seulement 2 ou 3 points en plus d'une autre capacité de rang 4 qui améliore une seule tâche ou compétence.
Une alternative peut être de fournir une Capacité qui dévie un peu du thème du focus, comme suggéré au rang 5 ci-après.

Au final, si la motivation n'a pas encore fournie une forme de protection, une Capacité défensive peut être inscrite au rang 4.

\item[Rang 5] Choisissez une Capacité qui permet au pesonnage de dévier légèrement---peut-être comme la Capacité Compétence d'expert qui lui donne un succès automatique ans une tâche pour laquelle il est entrainé.

Ou alors, si un bénéfice non-standard a été fourni au rang4, accordez les bénéfices suggérés au rang 4 ici.

\item[Rang 6] choisissez deux Capacités de rang supérieur. Donnez les en tant qu'options pour le Focus; un PJ choisira l'une ou l'autre.
Une option peut être une Capacité qui donné à nouveau 5 ou 6 points  à la Réserve la plus appropriée pour le Focus, ou bien que le joueur peut répartir comme il le souhaite. Ou alors, un entraînement dans une compétence offensive, ou bien défensive, conviendrait.

L'autre option du rang 6 peut être de donner au personnage une Capacité toute nouvelle dans le thème, mais par contre, en-dehors du champs du fantastique. Par exemple, une Capacité qiu permet au personnage de faire deux actions au lieu d'une seule est raisonable. Donner un nouvel entrainement , un nouvel atout ou un Avantage fera l'affaire.
\end{description}
%%%%%%%%%%%%%%%%%%%%%%%%%%%%%%%%%%%%%%%%%%%%%%%%%%
\section*{Exploration}
\label{subsec:exploration}
Les Focus qui permettent à un personnage de récupérer des informations, de survivre dans des environnements non-familiers, et de trouver leur chemin vers de nouveaux emplacements ou de traquer des créatures particulières ou des adversaires, sont des Focus d'exploration. Survivre dans des environnements non-familiers requiert une sélection raisonnable d'options défensives; toutefois, les Capacités qui permettent à un personnage de trouver et d'apprendre sont à donenr en priorité.

Les Focus d'exploration sont basées sur une variété de méthodes, bien que les piliers sont l'entrainement et l'expertise. Certaines méthodes requierent des outils spécifiques (comme unvéhicule) pour accorder les bénéfices fournis, tandis que d'autres pourraient se baser sur le supernaturel ou la superscience  pour apprendre de nouvelles choses et explorer de nouveaux endroits étranges et lointains.

\textbf{Connexion:} Choisissez quatre connexions pertinentes dans la liste des Connections de Focus.
\begin{description}
    \item[Equipement Supplémentaire] Tout objet nécessaire à l'exploration. Par exemple, des cartes et/ou un compas pourraient faire partie du matériel de base, tandis qu'un personnage qui utilise des pouvoirs psychiques pourrait avoir besoin d'un miroir ou d'une sphère de cristal pour regarder dedans. L'équipement pourrait aussi inclure l'accès à un véhicule nécessaire pour l'exploration.

    \item[Suggestions d'Effet Mineur] Vous avez un Atout pour n'importe quelle action qui implique vos sens, pour percevoir ou pour attaquer, jusqu'à la fin du prochain round.

    \item[Suggestions d'Effet Majeur] Votre Avantage d'Intellect augmente de 1 jusqu'à la fun du prochain round.
\end{description}

La liste ci-après ne sont que des exemples et n'est pas une liste complète de toutes les Focus possibles pour cette catégorie.

* Explore des Endroits Sombres
* Infiltrates
* Opère sous Couverture
* Peut Séparer son Esprit de son Corps
* Pilote un Vaisseau Spatial
* Voit Au-Delà

\textbf{Indications pour la Sélection de Capacités}
\begin{description}
\item[Rang 1] Choisissez une Capacité de rang inférieur qui permet au personnage des moyens pour de l'exploration de base, de la survie ou pour de la récupération d'information, dans le thème du focus.
Quelque fois, en fonction du focus, une Capacité de rang inférieur supplémenaire est appropriée. Souvent, c'est une Capacité qui donne un entrainement dans une compétence dans un type de connaissance associée ou une compétence associée (bien que cela puisse être couvert par la Capacité principale). D'une autre manière, elle pourrait donner un simple bonus de 2 ou 3 points à la Réserve de Puissance.

\item[Rang 2] Choisissez une autre Capacité de rang inférieur qui donne un moyen supplémentaire lié à l'exploration, la survie ou la récupértion d'information.
Par exemple, un Focus  dédiée à survivre dans des conditions sauvages pourrait donner une Capacité (ou deux) qui rend plus facile à éviter les catastrophes naturelles, les poisons, les terrains difficiles, et ainsi de suite. Un Focus dédiée à l'exploration d'un endroit en particulier pourrait donner des moyens pour avoir accès à cet endroit, ou un moyen que les autres n'ont pas habituellement (comme un moyen de voir dans le noir).

\item[Rang 3] Choisissez deux capacités de rang intermédiaire. Donnez-les tous les deux comme options pour le Focus; le PJ choisira l'un ou l'autre.
Une des options devrait améliorer un peu plus le moyen d'exploration de base déjà acquis, ou alors qui devrait donner un nouveau moyen d'exploration, de survie ou de récupération d'information.
L'autre option devrait être quelque chose qui bénéficie au personnage, soit de manière offensive, soit de manière défensive (en particulier si le Focus ne l'a pas déjà accordée) ou quelque chose qui étend un peu plus le moyen pour le personnage pour explorer dans le thème du focus.

\item[Rang 4] Choisissez une capacité de rang intermédiaire offensive ou défensive (quoique ce soit qui n'ait pas été proposé au rang 3) qui bénéficie au personnage. Ou alors, si les moyens offensifs et défensif sont déjà bien representésn choisissez une Capacité de rang intermédiaire différente qui étend la capacité du personnagé à explorer, survivre ou récupérer des informations.

\item[Rang 5] Choisissez une capacité de rang supérieur qui atténue certaines pénalités pour l'exploration, la survie ou la récupération d'information dans un endroit normalement inhospitaliers.

\item[Rang 6] Choisissez deux capacités de rang supérieur. Donnez-les tous les deux comme options pour le Focus; le PJ choisira l'un ou l'autre.
Une des options devrait améliorer encore plus le moyen d'exploration de base déjà accordé, ou elle devrait donner un tout nouveau moyen d'exploration, de survie ou de récupération d'information.
L'autre option devrait être quelque chose qui bénéficie au personnage, soit de manière offensive, soit de manière défensive, ou encore un tout moyen qui étend encore plus sa possibilité d'explorer dans le thème du focus.
\end{description}
%%%%%%%%%%%%%%%%%%%%%%%%%%%%%%%%%%%%%%%%%%%%%%%%%%
\section*{Influence}
\label{subsec:influence}
Un Focus qui donne la priorité sur l'autorité et l'influence (que ce soit pour commander des personnes ou des machines), pour aider les autres, ou pour atteindre un autre position prestigieuse et importante, est un Focus d'Influence.

Ces Focus donne de l'influence au travers de l'entrainement et la persuasion, par une manipulation mentale directe, par l'utilisation de la célébrité pour attirer l'attention des personnes et influencer leurs actions, ou simplement en connaissant et en apprennant des choses pour peut affecter les décisions utltérieures. Ainsi le concept d'influence est assez large.

\textbf{Connexion:} Choisissez quatre connexions pertinentes dans la liste des Connections de Focus.

\begin{description}
    \item[Equipement Supplémentaire] Toute objet nécessaire pour permettre l'influence suggérée devrait être accordé en tant qu'équipement suppllémentaire. Certaines Focus d'influence ne nécessite rien de spécial pour obtenir ou conserver leurs bénéfices.

\item[Suggestions d'Effet Mineur] La portée ou la durée de la Capacité d'influence est doublée.

\item[Suggestions d'Effet Majeur] Un allié ou une cible indirecte peut effectuer une action supplémentaire.

La liste ci-après ne sont que des exemples et n'est pas une liste complète de toutes les Focus possibles pour cette catégorie.

* Commande aux pouvoirs Mentaux
* Contourne le Système
* Est Idolatré par Millions
* Fussionne l'Esprit et la Machine
* Parle aux Machines
* Poursuit des Sciences Etranges
* Résout des Mystères

\textbf{Indications pour la Sélection de Capacités}

\item[Rang 1] Choississez une Capacité de rang inférieur qui permet au personnage d'apprendre quelque chose assez significatif pour qu'il puisse choisir une meilleure suite d'actions (ou d'utiliser cette connaissance pour persuader ou intimider). Comment le personnage apprend l'information varie selon les spécificités du focus. Un personnage pourrait avoir besoin de faire des expériences pour obtenir des réponses, un autre pourrait ouvrir un lien télépathique avec d'autres pour échanger de l'information secrètement et rapidement, tandis qu'un autre pourrait simplement avoir été formé dans les tâches d'intéractions.
Quelque fois une Capacité supplémentaire de rang inférieur peut être appropriée en fonction du focus. Souvent c'est une Capacité qui donne un entrainement dans une compétence dans un champs de connaissance.

\item[Rang 2] Choississez une Capacité de rang inférieur qui améliore le moyen par lequel le personnage peut exercer son influence. Cela pourrait ouvrir de nouvelles possibilités pour le thème du focus, ou bien simplement augmenter le moyen de base déjà fourni. Par exemple, cette Capacité de rang 2 peut faciliter un peu plus les tâches liées à l'influence, comme en autorisant un télépathe à lire l'esprit de personnes qui ont des secrets qu'ils ne dévoileraient pas autrement, ou en donnant de l'influence sur des objets physiques (soit en les améliorant, soit en en apprennant plussur eux).

\item[Rang 3] Choisissez deux capacités de rang intermédiaire. Donnez-les tous les deux comme options pour le Focus; le PJ choisira l'un ou l'autre.
Une des options devrait fournir un moyen offensif ou défensif dans le cadre spécifique deu type d'influence du focus. Par exemple, un inventeur peut créer un sérum qui lui donne une capacité améliorée (qui pourrait être utilisée pour l'ataque ou la défense), un télépathe peut avoir une méthode pour blesser des adversaires ave de l'énergié mentale, et une personne avec seulement les compétences de base dans les débats et l'influence au travers de la célébrité pourrait avoir besoin d'entrainement aux armes ou bien de son entourage.
L'autre option de rang intermédiaire devrait fournir une Capacité supplémentaire pour influencer, toujours dans le thème du focus, ou améliorer un peu plus la Capacité de base d'influence déjà choisie. Cette option n'est pas directement offensive ou défensice, mais fournit soit une toute nouvelle Capacité liée à la Capacité de base, ou améliore la force, la portée ou débloque une extension de la Capacité déjà retenue. Par exemple, un Télépaths pourrait avoir une Capacité de suggestion psychique.

\item[Rang 4] Choisissez une capacité de rang intermédiaire qui est une utilisation, soit offensive soit défensive, de la Capacité d'influence, en tout cas une qui n'ait pas été séletionnée au rang précédent.
Ou alors, cette Capacité peut permettre un nouveau moyen lié au genre d'influence donné par le Focus.

\item[Rang 5] Choisissez une avant-dernière Capacité de rang supérieur qui utilise la Capacité d'influence spécifique donnée aux rangs inférieurs.
Ou alors, choisissez une Capacité qui n'a pas été sélectionnée à un rang précédent, qui ouvre une nouvelle possibilité de moyen d'influence. Par exemple, si l'influence sur laquelle se base le Focus est télépathique, la Capacité de rang 5 pourrait autoriser le personnage à voir dans le future pour obtenir des atouts pour s'occuper des adverversaires (et des alliés).

\item[Rang 6] Choisissez deux capacités de rang supérieur. Donnez-les tous les deux comme options pour le Focus; le PJ choisira l'un ou l'autre.
Une des options devrait fournir un moyen soit offensif, soit défensif, à l'opposé du moyen fourni au rang 4 (bien que ce soit de rang supérieur au lieu de rang intermédiaire).
L'autre option devrait être quelquechose qui explore un peut plus l'usage de l'influence de base permise par le Focus. Si le choix du rang 5 était offensif ou défensif, cela pourrait être une Capacité encore meilleure liée au genre d'influence exercé, ou une manière différente d'utiliser cette Capacité pour débloquer une facette encore incore inconnue de la Capacité.
%%%%%%%%%%%%%%%%%%%%%%%%%%%%%%%%%%%%%%%%%%%%%%%%%%
\section*{Manipulation d'énergie}
\label{subsec:manipulation_denergie}
Un Focus de Manipulation d'énergie offre des Capacités pour invoquer le feu, l'électricité, la force, le magnétisme ou des formes d'énergies non-standard comme le froid, la pierre, ou quelque chose d'étrange comme le vide ou l'ombre. Ces Capacités donne dh'abitude au personnage une maière d'atteindre quelque chose comme un équilibre entre l'attaque et se donner à eux-mêmes ou à ses alliés une protection supplémentaire. Le Focus donne aussi habituellement des Capacités qui permettent d'autres façon d'utiliser une énergie spécifique pour des choses comme le transport, créer une grande concentration d'énergie qui affecte plusieurs cibles, ou créer un objet temporairement ou une barrière d'énergie.

\textbf{Connexion:} Choisissez quatre connexions pertinentes dans la liste des Connections de Focus.

\begin{description}
    \item[Equipement Supplémentaire] Une ou plusieurs pièces d'équipement immunisées contre l'énergie manipulée, cela peut être un ensemble de vêtements. Ou alors, quelque chose qui est lié à l'énergie générée. Certaines Focus dans cette catégorie ne requiert pas d'équipement supplémentaire.
    \item[Capacités énergétiques] Si le Type de pesonnage fournit des Capacités spéciales qui normalement utilisent une autre forme d'énergie, ces capacités produisent alors la sorte d'énergie du focus. Par exemple, si un personnage utilise cette Focus pour manipuler l'électricité, les éclairs de force deviennent des éclairs d'électricité. Ces altérations ne changent rien à part le type de dommage et tout effet secondaire (par exemple, l'électricité peut provoquer des courts-circuits dans les systèmes électroniques).
    \item[Suggestions d'Effet Mineur] La cible ou quelque chose près de la cible est désavantagée à cause de l'énergie résiduelle.
    \item[Suggestions d'Effet Majeur] Un objet important porté par la cible est détruit.
\end{description}
La liste ci-après ne sont que des exemples et n'est pas une liste complète de toutes les Focus possibles pour cette catégorie.

* Absorbe l'Energie
* Façonne la Foudre
* Fait Résonner le Tonnerre
* Manipule la Matière Noire
* Porte un Eclat de Glace
* Se Revêt d'un Halo de Feu

\textbf{Indications pour la Sélection de Capacités}

    \item[Rang 1] Choisissez une Capacité de rang inférieur qui soit inflige des dommages, soit fournit une protection, par l'utilisation de l'énergie d'une manière ou d'une autre.
    Quelque fois, une capacité supplémentaire de moindre puissance est appropriée en fonction du type d'énergie. Par exemple, un Focus qui contrôle le froid peut accorder une Capacité pour créer des sculptures de neige. Un Focus qui contrôle l'électricité pour fournir une Capacité pour charger un artifact épuisé, ou avoir un atout pour manipuler des systèmes électriques. Un Focus qui absorbe l'énergie peut donner une Capacité pour la libérer en tant qu'attaque de base. Et ainsi de suite.
    \item[Rang 2] Choisissez n'importe quel sorte de Capacité qui n'a pa sété choisie au rang 1.

    \item[Rang 3] Choisissez desu Capacités de rang intermédiaire. Proposez les en tant qu'options pour le Focus. Le PJ choisira l'une ou l'autre.
    Une option peut être une Capacité qui inflige des dommages par l'utilisation du type d'énergie sélectionnée (avec peut-être un effet secondaire).

    L'autre option peut accorder, par l'utilisation du type d'énergie, un mouvement amélioré, une protection supplémentaire, ou quelque chose de complètement nouveau, comme de drainer l'énergie d'une machine (si elle utilise de l'électricité), d'enfermer une victime dans des couches de glace (si elle utilise le froid), de créer un silence parfait (si elle utilise le son), de créer un flash de lumière éblouissant (si elle utilise de la lumière), etc.

    \item[Rang 4] Choisissez n'impote quelle genre de Capacité qui n'a pas été choisie au rang 3.

    \item[Rang 5] Choisissez une Capacité de rang supérieur (avec si possible un effet secondaire) qui affecte plus d'une cible en utilisant l'énergie sélectionnée, ou une Capacité qui utilise l'énergie d'une façon qui n'a pas été précédement décrite, comme indiqué aux rangs 3 et 6.

    \item[Rang 6] Choisissez deux Capacités de rang supérieur. Proposez les en tant qu'options pour le Focus; Le PJ choisira l'une ou l'autre.
    Une de ces Capacités de rang supérieur devrait utiliser l'énergie choisie pour infliger beaucoup de dommages à une ou plusieurs cibles.

    L'autre option devrait utiliser l'énergie sélectionnée pour accomplir une tâche qui n'est pas proposée par une capacité de rang inférieur. Par exemple, façonner un suivant enflammé (si c'est le feu), se téléporter sur une grande distance en tant qu'éclair (si c'est l'électricité), créer un objet solide à partir de lénergie, etc.
\end{description}
%%%%%%%%%%%%%%%%%%%%%%%%%%%%%%%%%%%%%%%%%%%%%%%%%%
\section*{Manipulation de l'Environnement}
\label{subsec:manipulation_de_lenvironement}
Les Focus qui permettent à un personnage de déplacer des objets, d'affecter la gravité, de créer des objets (ou des illusions d'objets), et ainsi de suite, sont des Focus de Manipulation de l'environnement. Ceci dit, dans beaucoup de cas, comme de l'énergie est utilisée pour toutes ces actions, les catégories de l'énergie et de l'environnement se recoupent par endroit. Un Focus de Manipulation de l'environnement donne la priorité aux Capacités qui affectent indirectement les adversaires et les alliés au travers d'objets, de forces et de modification de l'environnement; un Focus de Manipulation de l'énergie  donne la priorité à provoquer des dommages directement sur les cibles avec l'énergie ou la force choisie.

Par exemple, plutôt que de foudroyer un adversaire avec une pulsation de gravité qui fait des dommages, un personnag, utilisant un Focus de Manipulation de l'environnement basée sur la gravité, a plus de chance d'avoir des Capacités qui maintiennent la cible sur place, ou qui utilisent la gravité pour lancer des objets lourds pour attaquer, ou qui diminuent la gravité dans une zone définie ou sur un objet particulier.

\textbf{Connexion:} Choisissez quatre connexions pertinentes dans la liste des Connections de Focus.

\begin{description}
    \item[Equipement Supplémentaire] Tout objet nécessaire pour manipuler l'environnement autour du personnage. Par exemple, une personne avec un Focus qui donne une Capacité de fabrication d'objets pourrait nécessiter des outils. Certaines Focus dans cette catégorie ne nécessitent rien de particulier pour acquérir ou conserver leurs bénéfices.
    \item[Capacités de Manipulation de l'environnement] Les thèmes des Focus qui implique des énergies visibles ou non peuvent avoir un impact sur l'apparence des Capacités de votre type. De telles changements, s'il y en a, ne font rien d'autre que changer l'apparence des effets. Si la gravité est manipulée, peut-être qu'une pâle lueur bleutée apparait à chaque utilisation des Capacités, même les Capacités du Type. Si une illusion est générée, peut-être que des effets visuels et auditifs impressionants accompagnent ce genre de Capacités, comme l'apparence d'un tentacule bestial qui enserre la cible quand la Capacité Stase est utilisée.
    \item[Suggestions d'Effet Mineur] La cible pivote et sa prochaine attaque est atténuée.
    \item[Suggestions d'Effet Majeur] Le personnage est régénéré et récupère 4 points dans une Réserve.
\end{description}
La liste ci-après ne sont que des exemples et n'est pas une liste complète de toutes les Focus possibles pour cette catégorie.

* Calcule l'Incalculable
* Concentre l'Esprit sur la Matière
* Contrôle la Gravité
* Contrôle le Magnétisme
* Façonne des Illusions
* Façonne des Objets Uniques
* Illumine avec Eclat
* Réveille les rêves

\textbf{Indications pour la Sélection de Capacités}
\begin{description}
\item[Rang 1] Choisissez une Capacité de rang inférieur qui confère une utilisation de base d'une Capacité qui modifie l'environnement en utilisant le thème du focus. Par exemple, un Focus qui affecte la gravité peut fournir une Capacité qui peut rendre une cible plus légère ou plus lourde. Un Focus de création d'illusion peut fournir une Capacité qui permet la création d'une image. Un Focus de fabrication d'objet peut fournir une compétence de base pour la création d'un type particulier d'objet. Un Focus de prédiction, peut calculer des probabilités de résultats et fournir au personnage des bénéfices liés à ces informations.

Quelque fois, une Capacité supplémentaire de faible puissance, dépendant du focus. Souvent, c'est une Capacité qui donne un entrainement dans une compétence dans une catégorie de connaisance.

\item[Rang 2] Choisissez une Capacité de rang inférieur qui confère un moyen offensif ou défensif lié au thème du focus.
Ou alors, cette Capacité peut fournir un moyen supplémentaire ou tout nouveau pour manipuler l'environnement lié au thème du focus.

\item[Rang 3] Choisissez deux capacités de rang intermédiaire. Donnez-les tous les deux comme options pour le Focus; le PJ choisira l'un ou l'autre.
Une des options devrait être une Capacité de rang intermédiaire lié au thème du focus qui fournit une Capacité de manipulation de l'environnement supplémentaire, ou qui améliore une Capacité de manipulation de l'environnement qui a été choisie précédemment. Cette Capacité n'est pas directement offensive ou défensive, mais fournit soit une toute nouvelle Capacité liée à la Capacité de base, soit une Capacité qui augmente la puissance, la portée, ou une autre extension d'une Capacité précédemment choisie.

L'autre option de rang intermédiaire devrait fournir une Capacité offensive ou défensive liée à la forme spécifique du mouvement que le Focus permet.

\item[Rang 4] Choisissez une Capacité de rang intermédiaire dont l'usage est soit offensive soit défensive, dans tous les cas il ne faut pas que ce soit une option déjà proposée à un rang précédent.

\item[Rang 5] Choisissez une Capacité de rang supérieur de manipulation de l'environnement. Par exemple, si le Focus de manipulation est sur l'illusion, cette Capacité pourrait hanter une cible avec des images terrifiantes. Si le Focus est basée sur la gravité, elle pourrait débloquer le vol. Si c'est magnétique, la Capacité peut permettre au personnage de changer la forme du métal. Si ce sont des pouvoirs télékinétiques, la Capacité pourrait autoriser le personnage à balancer des objets volumineux sur des cibles.

\item[Rang 6] Choisissez deux capacités de rang supérieur. Donnez-les tous les deux comme options pour le Focus; le PJ choisira l'un ou l'autre.
Une des Capacités devrait fournir un moyen soit offensif, soit défensif, à l'opposé de la Capacité proposée au rang 4 (bien que que de rang supérieur).

L'autre option devrait être quelqu chose qui explorer un peu plus loin l'usage de la manipulation de l'environnement. Cette Capacité devrait être plus puissante que celle de rang 5, ou une autre façon d'utiliser cette Capacité pour débloquer une facette non encore explorée.
\end{description}
%%%%%%%%%%%%%%%%%%%%%%%%%%%%%%%%%%%%%%%%%%%%%%%%%%
\section*{Utilisation d'Alliés}
\label{subsec:utilisation_dallies}
Les Focus qui donne la priorité à fournir des suivants PNJ à un personnage sont des Focus d'Utilisation d'Alliés. Les suivants fournissent une aide au PJ de façon très variée, mais surtout ils donnent un Atout pour les actions du personnage.

Il y a plusieurs thèmes potentiels dans la catégorie Utilisation d'Alliés, depuis les capacités qui permettent au pesonnage d'invoquer ou de fabriquer des alliés, à celles qui leur donne la possibilité d'attirer des suivants par la célébrité, la magie, l'autorité ou le charisme.

\textbf{Connexion:} Choisissez quatre connexions pertinentes dans la liste des Connections de Focus.

\begin{description}
    \item[Equipement Supplémentaire] N'importe quel objet qui peut être nécessaire au personnage pour conserver un allié. Par exemple, quelqu'un avec un Focus qui utilise de la super-science pour créer des alliés robots pourrait avoir besoin d'outils pour construire et réparer ces alliés. Certaines Focus dans cette catégorie n'ont besoin de rien pour acquérir oou conserver leurs bénéfices.

    \item[Suggestions d'Effet Mineurs] Les tâches de l'allié PNJ sont facilitées à son prohain tour..

    \item[Suggestions d'Effets Majeurs] L'allié PNJ gagne une action supplémentaire immédiatement.
\end{description}

La liste ci-après ne sont que des exemples et n'est pas une liste complète de toutes les Focus possibles pour cette catégorie.

* Construit des Robots
* S'Associe avec les Morts
* Contrôle les Bêtes Sauvages
* Existe en Deux Endroits en Même Temps
* Dirige
* Maîtrise l'Essaim
* Guide les Esprits

\textbf{Indications pour la Sélection de Capacités}
\begin{description}
\item[Rang 1] Choisissez une capacité de rang inférieur qui donne un PNJ de niveau 2 au personnage, ou qui donne un bénéfice similaire fournit par un PNJ. Vous pouvez également fournir les bases pour gagner de tels alliés PNJ à des rangs plus élevés en choisissant une capacité qui donne au personnage une influence sur les autres.
Parfois, une capacité supplémentaire de rang inférieur est appropriée, en fonction du focus. Il s'agit souvent d'une capacité qui confère un entrainement dans une compétence dans un domaine de connaissances ou une compétence connexe. Par exemple, un entrainement dans une compétence liée au type d'allié PNJ que le personnage gagne serait appropriée.

\item[Rang 2] Choisissez une capacité de rang inférieur qui confère une influence sur des types de PNJ similaires à ceux obtenus par l'allié au rang précédent. Si aucun allié n'a été gagné au Rang précédent, cette capacité devrait offrir cet avantage maintenant.
Parfois, une capacité secondaire peut être appropriée en plus de la capacité fournie ci-dessus, par exemple une capacité de faible puissance qui accorde 2 ou 3 points à une Réserve.

\item[Rang 3] Choisissez deux capacités de rang intermédiaire. Donnez-les tous les deux comme options pour le Focus; le PJ choisira l'un ou l'autre.
Une option devrait être une capacité de rang intermédiaire qui améliore l'allié PNJ précédemment fourni (généralement du niveau 2 au niveau 3) ou accorde un allié supplémentaire.

L'autre option devrait être quelque chose qui profite au personnage --- peut-être une capacité offensive ou défensive, ou quelque chose qui élargit son influence sur ses alliés (ou alliés potentiels).

\item[Rang 4] Choisissez une capacité de rang intermédiaire qui donne au personnage une capacité offensive ou défensive s'il n'en a pas acquis auparavant, de préférence dans le thème du focus. Par exemple, si le personnage gagne des alliés en raison de son charisme, cette capacité peut lui permettre de commander ses ennemis pendant de brèves périodes. Si le personnage gagne des alliés en les construisant ou en les appelant, cette capacité peut lui permettre d'affecter des entités du même type qui ne sont pas déjà ses alliés.
Ou alors, cette capacité pourrait améliorer davantage un allié précédemment gagné du niveau 3 au niveau 4, ou accorder un allié supplémentaire.

\item[Rang 5] Choisissez une capacité qui améliore le personnage en lui fournissant une défense, une Réserve améliorée de statistiques ou un autre type de protection. Alternativement, cette capacité pourrait ouvrir une nouvelle possibilité pour influencer et appeler des alliés PNJ liés au thème du focus. Par exemple, quelqu'un qui garde des bêtes alliées pourrait acquérir la capacité d'invoquer une horde de bêtes inférieures. Quelqu'un qui construit des robots pourrait acquérir la capacité de construire plusieurs robots assistants de moindre importance. Et ainsi de suite.

Enfin, cette capacité pourrait améliorer un allié précédemment gagné au niveau 5.

\item[Rang 6] Choisissez deux capacités de rang supérieur. Donnez-les tous les deux comme options pour le Focus; le PJ choisira l'un ou l'autre.
L'une des capacités devrait améliorer un allié précédemment acquis au niveau 5, si cela n'était pas déjà fourni au rang 5. Si tel est le cas, cette capacité pourrait être fournie *en plus* de deux autres capacités associées.

L'autre option de rang supérieur devrait fournir au personnage une poignée d'alliés de niveau 3.

La dernière capacité de rang supérieur pourrait être une nouvelle possibilité pour influencer et appeler des alliés PNJ liés au thème du focus. Par exemple, quelqu'un qui gagne des alliés grâce à un charisme et un entrainement pourrait acquérir la capacité d'apprendre des informations autrement impossibles à glaner.
\end{description}
%%%%%%%%%%%%%%%%%%%%%%%%%%%%%%%%%%%%%%%%%%%%%%%%%%
\section*{Irrégulier}
\label{subsec:irregulier}
La plupart des foci ont un thème principal, un "histoire pour le personnage" qui implique logiquement une série de capacités associées. Toutefois, certains thèmes de foci sont si vagues qu'ils ne correspondent à aucune catégorie, à part une catégorie irrégulière spécifique.

Les foci irréguliers offrent un ensemble de capacités disparates. Cela est généralement dû au fait que le thème global exige de la variabilité et l'accès à plusieurs types de capacités différents. Souvent, ces foci se retrouvent dans des genres qui suggèrent des ajustements de règles supplémentaires pour exploiter encore davantage leur utilisation, comme les changements de puissance dans le genre des super-héros ou la magie dans le genre de la fantasy. Cependant, d'autres foci irréguliers sont possibles.

\textbf{Connexion:} Choisissez quatre connexions pertinentes dans la liste des Connections de Focus.

\begin{description}
    \item[Equipement Supplémentaire] Tout objet nécessaire au thème du focus. Par exemple, un focus sur un thème de super-héro peut fournir un costume de super-héro.

    \item[Suggestions d'Effet Mineur] La cible est aussi étourdie pendant un round, durant lequel toutes ses tâches sont entravées.

    \item[Suggestions d'Effet Majeur] La cible est sonnée et perd son prochain tour.
\end{description}

La liste ci-après ne sont que des exemples et n'est pas une liste complète de toutes les Focus possibles pour cette catégorie.

* Canalise les Bénédictions Divines
* A des Ascendants Nobles
* Est sorti de l'Obélisque
* Vole Plus Vite qu'une Balle
* Maîtrise les Sortilèges
* Parle au Nom de la Terre

\textbf{Indications pour la Sélection de Capacités}
\begin{description}
\item[Rang 1] Choisissez une capacité de rang inférieur qui donne un des bénéfices promis par le thème du focus et qu'un personnage de rang 1 devrait avoir.
Quelque fois, pour certain focus, une capacité supplémentaire de faible puissance est appropriée. Souvent c'est une capacité qui confère un entrainement dans une compétence dans un domaine de connaissance associé au focus, ou une autre compétence qui peut être associée au focus. D'une autre manière, le focus peut accorder un simple bonus de 2 ou 3 points à une Réserve.

\item[Rang 2] Choisissez une capacité de rang inférieur qui donne un des bénéfices promis par le thème du focus, et qui n'est pas liée directement à la capacité fournie au rang 1. Ceci dit, si une défense n'a pas été donnée au rang 1, le rang 2 est un bon endroit pour la placer.

\item[Rang 3] Choisissez deux capacités de rang intermédiaire. Donnez-les tous les deux comme options pour le Focus; le PJ choisira l'un ou l'autre.
Une option devrait fournir un des bénéfices promis par le thème du focus, une qui ne serait pas liée à ceux fournit aux rangs précédents.
L'autre option devrait inclure une méthode d'attaque si aucune n'a été déjà accordée. D'une autre manière, si les capacités de rang inférieur ne fournit pas exactement ce qu'il faut au personnage à ce stade, cette option peut augmenter un peu plus une capacité débloquée à un rang inférieur.

\item[Rang 4] Choisissez une capacité de rang intermédiaire qui fournit un des bénéfices promis par le thème du focus, et qui n'est pas liée aux capacités fournies précédement.

\item[Rang 5] Choisissez une capacité de rang supérieur qui fournit un des bénéfices promis par le thème du focus, et qui n'est pas liée aux capacités fournies aux rangs précédents.

\item[Rang 6] Choisissez deux capacités de rang supérieur. Donnez-les tous les deux comme options pour le Focus; le PJ choisira l'un ou l'autre.
Une option devrait fournir un des bénéfices promis par le thème du focus, une qui ne serait pas liée à ceux fournit aux rangs précédents. Toutefois, cette capacité pourrait aussi fournir un version supérieure d'une capacité de rang inférieur si une option de rang intermédiaire ou de rang inférieur n'était pas suffisante.

L'autre option devrait offrir une méthode alternative pour compléter le personnage d'une manière qui ne reproduit pas la première option de Rang 6. Par exemple, si la première option fournissait une sorte d'attaque, celle-ci pourrait être une interaction, une collecte d'informations ou une capacité de guérison, en fonction du thème général du focus.

\end{description}
%%%%%%%%%%%%%%%%%%%%%%%%%%%%%%%%%%%%%%%%%%%%%%%%%%
\section*{Expertise des Mouvement}
\label{expertise_des_mouvements}
Les foci qui privilégient des formes de mouvement novatrices — pour exceller en combat, échapper à des situations quand la plupart ne le peuvent pas, se déplacer furtivement dans un but de vol ou d'évasion, ou accéder à des lieux normalement inaccessibles — appartiennent à la catégorie de l'expertise en mouvement. Ces foci offrent généralement des moyens d'accorder soit de l'offensive, soit de la défense par le mouvement, bien qu'ils puissent parfois permettre les deux.

Le focus classique d'expertise en mouvement repose sur la vitesse pour effectuer plus d'attaques et éviter d'être touché, bien que l'agilité générale puisse également offrir le même avantage. D'autres focus de cette catégorie peuvent s'inscrire dans ce thème en permettant à un personnage de devenir immatériel, de changer de forme pour devenir quelque chose comme de l'eau ou de l'air, ou de se déplacer instantanément par téléportation.

\textbf{Connexion:} Choisissez quatre connexions pertinentes dans la liste des Connections de Focus.

\begin{description}
    \item[Equipement Supplémentaire] Tout objet nécessaire pour atteindre de grandes vitesses, changer d'état ou obtenir autrement les avantages du focus devrait être fourni en tant qu'équipement supplémentaire. Certains foci dans cette catégorie ne nécessitent rien pour obtenir ou conserver leurs avantages.

\item[Suggestions d'Effet Mineur] La cible est aussi étourdie pendant un round, durant lequel toutes ses tâches sont entravées.

\item[Suggestions d'Effet Majeur] La cible est sonnée et perd sa prochaine action.
\end{description}

La liste ci-après ne sont que des exemples et n'est pas une liste complète de toutes les Focus possibles pour cette catégorie.

* Existe Partiellement Hors de Phase
* Bouge comme un Chat
* Va Comme le Vent
* S'enfuit
* Déchire les Murs du Monde
* Voyage à Travers le Temps
* Rôde dans les Bas Quartiers

\textbf{Indications pour la Sélection de Capacités}
\begin{description}
\item[Rang 1] Choisissez une capacité de rang inférieur qui accorde le bénéfice de base du style de mouvement défini pour le focus, que ce soit une vitesse accrue, de l'agilité, de l'immatérialité, etc. Parfois, une capacité de faible puissance supplémentaire est appropriée, selon le focus. Si le bénéfice de base du mouvement exige une certaine compréhension ou un entraînement supplémentaire, cette capacité pourrait fournir ce genre de compétence. Alternativement, si le mouvement fourni semble également devoir débloquer un avantage offensif ou défensif de base (reposant sur l'utilisation de la capacité de base initiale), ajoutez-le également.

\item[Rang 2] Choisissez une capacité de rang inférieur qui procure une faculté offensive ou défensive liée au thème du focus. Alternativement, cette capacité peut fournir une faculté supplémentaire liée au type de mouvement et qui donne des informations utiles au personnage qui ne seraient pas accessible par ailleurs sans le focus.

\item[Rang 3] Choisissez deux capacités de rang intermédiaire. Donnez-les tous les deux comme options pour le Focus; le PJ choisira l'un ou l'autre.
Une option devrait fournir une faculté de mouvement supplémentaire ou qui améliore un peu plus la capacité de mouvement de base, en relation avec les thème du focus. Ce n'est pas directement offensif ou défensif, mais donne au personnage un nouveau niveau de capacity ou un capacité complètement nouvelle liée à la capacité de base du mouvement choisi.

L'autre option devrait fournir une faculté soit offensive, soit défensive, liée à la forme spécifique du mouvement fournie par le focus.

\item[Rang 4] Choisissez une capacité de rang intermédiaire qui améliore les avantages donnés par le paradigme du focus d'amélioration du mouvement. Cela peut fournir une forme de défense nouvelle ou meilleure (diretement ou indirectement si le déplacement se fait dans un endroit ou une temporalité où le danger ne menace plus), ou une forme offensive nouvelle ou améliorée.

\item[Rang 5] Choisissez une avant-dernière capacité de rang supérieur liée au mouvement. Par exemple, si le mouvement fourni par le focus est temporel, cette capacité peut autoriser un saut temporel. Si le focus améliore la rapidité, cette capacité peut autoriser le personnage à se déplacer sur une très longue distance en une action. Et ainsi de suite.
alternativement, débloquez un capacité qui n'est pas encore découverte et qui découle du pouvoir de mouvement fournit par le focus.

\item[Rang 6] Choisissez deux capacités de rang supérieur. Donnez-les tous les deux comme options pour le Focus; le PJ choisira l'un ou l'autre.
Une des options devrait fournir un moyen soit offensif, soit défensif, à l'opposé de la capacité du rang 4 (bien qu'elle soit de rang supérieur au lieu de rang intermédiaire).

L'autre option devrait être quelque chose qui étend un peu plus l'utilisation de la capacité de mouvement. Si le choix du rang 5 était l'avant-dernière capacité, cela peut être une dernière capacité encore meilleure liée au mouvement.
\end{description}
%%%%%%%%%%%%%%%%%%%%%%%%%%%%%%%%%%%%%%%%%%%%%%%%%%
\section*{Combat Offensif}
\label{subsec:combat_offensif}
Les foci de combat offensif donnent le plus d'importance à infliger des dommages lors d'une bataille. Les foci dans cette catégorie donne quelques capacités défensives, mais la priorité est donnée à augmenter les dommages au maximum.

Pour y arriver, un focus de combat offensif accorde la maitrise d'un style particulier d'art martial, comme par exemple un entrainement avec une arme en particulier ou sans arme, ou alors l'utilisation d'un outil spécial (ou même une forme d'énergie). Un style peut être quelque chose d'aussi singulier que d'être le meilleur à combattre un type spécifique d'ennemi, ou quelque chose de plus générique, comme d'adopter un style de combat particulièrement viceux ou qui ne respecte pas les règles. Un combattant offensif peut utiliser le feu, une force, ou du magnétisme comme méthode préférée pour infliger des dommages.

\textbf{Connexion:} Choisissez quatre connexions pertinentes dans la liste des Connections de Focus.

\begin{description}
    \item[Equipement Supplémentaire] L'arme, l'outil, l'objet spécial ou substance (s'il y en a) nécessaire pour pratiquer le style de combat particulier. Par exemple, une dose de poison de niveau 5 pour Se Bat Sans Respecter de Règle ou Assassine, un trophée d'un ennemi abbatu pour Combat les Robots, ou des vêtements remarquables pour Combat avec Panache.

\item[Suggestions d'Effet Mineur] La cible est tellement impressionnée par votre style qu'elle est étourdie pour un round, entravant ses actions pendant ce temps.

\item[Suggestions d'Effet Majeur] Faites une attaque supplémentaire immediatement en utilisant la méthode d'attaque du focus pendant votre tour.
\end{description}

La liste ci-après ne sont que des exemples et n'est pas une liste complète de toutes les Focus possibles pour cette catégorie.

* Combat les Robots
* Se Bat Sans Respecter de Règle
* Combat avec Panache
* Chasse
* A le Droit de Porter une Arme à Feu
* Cherche les Ennuis
* Maîtrise l'Armement
* Assassine
* N'a pas Besoin d'Arme
* Accompli des Prouesses de Force
* Se Met en Rage
* Tue les Monstres
* Lance avec une Précision Mortelle
* Se Bat avec Deux Armes à la fois

\textbf{Indications pour la Sélection de Capacités}
\begin{description}
\item[Rang 1] Choisissez une capacité de rang inférieur qui inflige un dommage supplémentaire quand le pesonnage attaque en utilisant le style, l'énergie ou l'attitude spécifique du focus, ou quand elle est dirigée contre l'ennemi choisi pour le focus.
Quelque fois, une capacité de rang inférieur supplémentaire est appropriée, en fonction du focus. Par exemple, un focus qui donne une abilité avec une arme spéciale peut accorder un entrainement pour les tâches d'artisanat associées avec cette arme. Un focus qui permet d'augmenter les domages contre une sorte particulière d'ennemi peut accorder un entrainement dans des copétences pour reconnaitre, localiser, ou juste avoir des connaissances générales à propos de cet ennenmi. Un style de combat qui implique de combattre de manière vicieuse ou sale peut accorder un entrainement en intimidation. Et ainsi de suite.

Si le focus consiste à combattre un ennemi particulier, des pouvoirs secondaires supplémentaires (plus que ce qui pourrait être proposés) peuvent être appropriés. Ceux-ci peuvent soit améliorer l'efficacité contre l'ennemi choisi, soit offir des capacités plus génériques, mais liées au focus, qui permettent au personnage de les utiliser même quand ils ne combattent pas cet ennemi particulier.

\item[Rang 2] Choisissez une capacité de rang inférieur qui accorde une forme de défense en utilisant cette arme, ce style d'arme, ou l'énergie sélectionnée. Si le style d'arme est particulièrement efficace pour combattre un certain type d'adversaire, cette capacité devrait être une déense contre ce type d'ennemi. D'une autre manière, le focus peut accorder une autre méthode pour augmenter les dommages ans le contexte du focus.
Quelque fois une capacité de rang inférieur supplémentaire est aappropriée au rang 2. Dans ce cas, choisissez une capacité de rang inférieur qui n'a pas été acquise au rang 1.

\item[Rang 3] Choisissez deux capacités de rang intermédiaire. Donnez-les tous les deux comme options pour le Focus; le PJ choisira l'un ou l'autre.
Une option devrait être d'infliger des dommages supplémentaires quand on utilise le style de combat, l'énergie, ou l'attitude, ou l'ennemi sélectionné du focus. Cela pourrait être aussi simple qu'une capacité qui accorde une attaque supplémentaire contre l'ennemi sélectionné.

L'autre option devrait fournir une méthode pour neutraliser temporairement un adversaire en le désarmant, l'étourdissant ou l'assomant, le ralentissant, le restreignant, ou le déconcertant en utilisant le style de combat, l'énergie, ou l'attitude, ou l'ennemi sélectionné du focus.

\item[Rang 4] Choisissez une capacité de ang intermédiaire qui améliore un peu plus les avantages accordés par le paradigme du focus. Souvent cela inclut un entrainement dans une attaque particulière. D'une autre manière, la capacité peut augmenter les avantages acquis en gagnant un certain statut lors du combat, comme de gagner la surprise.

\item[Rang 5] Choisissez une capacité de rang supérieur qui inflige des dommages. D'une autre manière, si le focus est de se concentrer sur un type particulier d'ennemi, cette capacité peut accorder au personnage une chance de complètement neutraliser, détruire, aveugler ou tuer une cible spécifique d'au plus de niveau 3 (ou plus élevé, si le focus se concentre sur un seul ennemi).

\item[Rang 6] Choisissez deux capacités de rang supérieur. Donnez-les tous les deux comme options pour le Focus; le PJ choisira l'un ou l'autre.
Une des options devrait utiliser le paradigme du focus pour infliger des dommages exceptionnels.

L'autre option peut être une différente manière d'infliger des dommages, soit en utilisant le paradigme du focus, soit juste beaucoup de dommages en général (et en s'appuyant sur les capacités des rangs précédents pour améliorer les possibilités de toucher). Cela peut être contre des cibles multiples si la première option est pour une seule cible, ou pour tuer directement ou neutraliser une cible (à partir du niveau 4, mais avec des indications pour utiliser l'Effort pour augmenter le niveau de la cible), ou pour sélectionner un autre adversaire, faire une autre attaque ou se retirer pour combattre une autre fois.

\end{description}
%%%%%%%%%%%%%%%%%%%%%%%%%%%%%%%%%%%%%%%%%%%%%%%%%%
\section*{Soutien}
\label{subsec:soutien}
Les foci qui permettent au pesonnage d'aider les autres, de les défendre, de les soigner, et ainsi de suite, sont des foci de support. Bien sur, la plupat des capacités des foci sont souvent utilisés pour aider les autres, mais les foci de support (comme Siphonne les Pouvoirs)

Foci that allow a character to help others succeed, defend others, heal others who are hurt, and so on are support foci. Of course, most foci abilities are often used in aid of others, but support foci (such as Siphonne les Pouvoirs) prioritize aiding, healing, and improving the character who takes the focus.

Support foci rely on a variety of methods to provide their help, including martial training used in defense, supernatural ou sci-fi means of providing healing, ou simply easing the cares of others through entertainment.

\textbf{Connexion:} Choisissez quatre connexions pertinentes dans la liste des Connections de Focus.

\begin{description}
    \item[Equipement Supplémentaire] Any object necessary to provide support. For instance, someone with a focus that uses entertainment to help others would require an instrument ou similar object in aid of their craft. Some foci in this category don't require anything to gain ou retain their benefits.

\item[Suggestions d'Effet Mineur] You can draw an attack without having to use an action at any point before the end of the next round.

\item[Suggestions d'Effet Majeur] You can take an extra action in aid of an ally.
\end{description}

La liste ci-après ne sont que des exemples et n'est pas une liste complète de toutes les Focus possibles pour cette catégorie.

* Défend les Faibles
* Divertit
* Aide ses Amis
* Rend la Justice
* Guide la Communauté
* Siphonne les Pouvoirs
* Fait des Miracles

\textbf{Indications pour la Sélection de Capacités}
\begin{description}
\item[Rang 1] Choose a low-Rang ability that provides some form of defense, aid ou entertainment, benefit to recovery ou healing, ou protection. That defense ou protection could be to the PJ and not to an ally, as one cannot protect another without first being able to protect themselves (and sometimes protecting themselves is the entire point).
Sometimes an additional low-power ability is appropriate, depending on the focus. Often, this is an ability that grants skill training in a related area of knowledge ou a related skill, but it might be something that works with the initial ability that, by itself, wouldn't do much.

\item[Rang 2] Choose a low-Rang ability that follows up on the support style opened in the previous Rang. If the previous Rang's ability provided a means of protection only for the focus taker, this Rang 2 ability should specifically provide aid to another. If the previous Rang specifically provided aid to another, this Rang 2 ability could defend the focus taker ou provide an offensive capability grounded, if possible, in the focus's theme.

\item[Rang 3] Choisissez deux capacités de rang intermédiaire. Donnez-les tous les deux comme options pour le Focus; le PJ choisira l'un ou l'autre.
One option should work within the focus's theme to aid, heal, protect, ou otherwise help another.

The other option should be something that benefits the character, either an offensive ou defensive ability, ou something that broadens their expertise in some fashion. Alternatively, it could be another, different method of helping someone else.

\item[Rang 4] Choose a mid-Rang ability that gives an ally a direct boon ou provides the character with a way to help another. It could also be an ability that harms ou nullifies a foe, as removing foes certainly helps allies.

\item[Rang 5] Choose a high-Rang ability that provides an offensive ou defensive option for the character, if none have been provided yet. If this need has been previously addressed ou is deemed unnecessary, choose a high-Rang ability that provides some form of defense, aid ou entertainment, benefit to recovery ou healing, ou protection to another. For example, a Rang 5 ability might grant an ally an additional free action ou allow them to repeat a failed action.

\item[Rang 6] Choisissez deux capacités de rang supérieur. Donnez-les tous les deux comme options pour le Focus; le PJ choisira l'un ou l'autre.
One of the options should provide an ultimate method of helping another in the theme of the focus.

The other option could provide an alternative ultimate method of helping another; many foci in this category do. However, an option that provides high-Rang offense ou defense is also completely reasonable.
\end{description}
%%%%%%%%%%%%%%%%%%%%%%%%%%%%%%%%%%%%%%%%%%%%%%%%%%
\section*{Combat défensif}
\label{subsec:combat_defensif}
Les Motivations qui donnent la priorité pour être capable de subir des assauts et d'absorber les dommages d'adversaires font parti de la catégorie Combat défensif. Ces Motivations fournissent aussi des Capacités offensives liées à la méthode particulière par laquelle la protection améliorée est donnée, mais les Capacités défensives sont plus accentueés.

Certaines Motivations de Combat défensif impliquent une transformation physique qui donne une protection supplémentaire, et d'autres se basent sur de l'entrainement spécialisé, utilisent des outils comme des boucliers ou une armure lourde, ou fournissent le moyen de se soigner très vite. Les sortes de tranformation physique qu'une Motivation de Combat défensif peuvent fournir varient beaucoup. Une Motivation peut transformer la peau d'un personnage en pierre, renforcer son corps avec du métal, le transformer en créature monstrueuse, les faire grandir tellement que cela devient difficile de les blesser, et ainsi de suite.

\textbf{Connexion:} Choisissez quatre connexions pertinentes dans la liste des Connections de Focus.

\begin{description}
    \item[Equipement Supplémentaire] Tout objet nécessaire pour maintenir une transformation physique (comme un outil pour des réparation pour une transformation partiellement robotique, un bouclier ou un autre outil défensif pour lequel le personnage est entrainé, ou un genre d'amulette ou de sérum). Certaines Motivations de combat défensif n'ont besoin de rien pour obtenir ou maintenir leurs bénéfices.

\item[Suggestions d'Effet Mineur] +2 à l'Armure pour quelques rounds.

\item[Suggestions d'Effet Majeur] Regagner 2 points à la Réserve de Puissance.
\end{description}

La liste ci-après ne sont que des exemples et n'est pas une liste complète de toutes les Focus possibles pour cette catégorie.

* Brandit un Bouclier Exotique
* Demeure dans la pierre
* Fusionne la Chair et l'Acier
* Garde le Passage
* Grandit Jusqu'au Ciel
* Hurle à la Lune
* Maîtrise la Défense
* Ne S'Avoue Jamais Vaincu
* Résiste Comme une Citadelle
* Vit dans la Nature Sauvage

\textbf{Indications pour la Sélection de Capacités}
\begin{description}
\item[Rang 1] Choisissez une Capacité de rang inférieur qui permet une défense dans le thème de la Motivation. Si le thème est simplement de l'entrainement intensif ou l'utilisation d'un outil défensif, la Capacité peut être aussi simple qu'un bonus à l'Armure. Si la protection vient d'une transformation physique cette Capacité fournit des effets de la forme de base, des nénéfices et dans certains cas des inconvénients pour la réalisation de la transformation. Une Capacité de soin de rang inférieur pourrait être appropriée au premier rang.
Quelque fois une Capacité supplémentaire de faible puissance convient en fonction de la Motivation. Si le personnage se transforme, cette Capacité peut fournir un effet secondaire, bien que dans le cas de certaines transformations, cela pourrait être la description de comment quelqu'un avec une physionomie anormale peut complètement guérir. Pour d'autres Capacités, le pouvoir secondaire peut être simplement un entrainement dans une compétence liée, ou cela peut débloquer la possibilité d'utiliser une armure particulière ou un bouclier sans pénalité.

\item[Rang 2] Si le thème de la Motivation n'est pas une transformation physique, choisissez une Capacité de rang inférieur qui offre une méthode supplémentaire pour défendre, soigner des dommages ou éviter des attaques.
Si le thème de la Motivation est une transformation physique, coisissez une Capacité de rang inférieur qui débloque une nouvelle possibilité liée à la forme que prend le personnage.Cela peut être de gagner un meilleur contrôle de la transformation, débloquer une interface robotique, ou bien débloquer encore plus cette forme. Cette Capacité n'est pas nécessaire défensive, bien qu'elle pourrait l'être.

\item[Rang 3] Choisissez deux capacités de rang intermédiaire. Donnez-les tous les deux comme options pour la Motivation; le PJ choisira l'un ou l'autre.
Une des options devrait fournir une forme supplémentaire de protection dans le thème de la Motivation, telle que plus de possibilités défensives débloquées de la transformation (qui peut aussi venir avec de nouvelles posisbilités offensives) ou un simple entrainement physique si la défense est acquise par les compétences ou les soins.
L'autre option devrait fournir une possibilité offensive , particulièrement si vous créez une Motivation qui n'est pas une transformation et qui n'a pas encore de bénéfices offensifs. Cette possibilité pourrait être une attaque améliorée ou qui permet un autre néfice au combat, comme s'échapper rapidement ou (de l'autre côté du spectre) devenir inébranlable.

\item[Rang 4] Choisissez une capacité de rang intermédiaire qui améliore les avantages donnés par le paradigme de focus "j'inflige des dommages". Cela inclut souvent un entrainement dans un type particulier de défense. Ou alors, cela peut augmenter les avantages fournis par des capacités de défense précédentes, que ce soit par un meilleur contrôle sur une transformation, gagner des chances supplémentaires pour éviter des dommages ou des tâches pour ré-essayer des tâches associées avec une détermination renforcée, et ainsi de suite. Si le focus manque d'options offensives, c'est une bonne place pour en ajouter.

\item[Rang 5] Choisisssez une capacité de rang supérieur qui accorde une protection, sous la forme par exemple de se débarrasser d'un état affaibli (incluant la mort). Si le focus accorde une transformation physique, cette capacité peut débloquer ou améliorer un peu plus une capacité déjà acquise, que ce soit de manière offensive, défensive, ou quelque chose lié à de l'exploration ou des intéractions (comme de voler si la forme est ailée, ou d'intimider si la forme est effrayante, et ainsi de suite).

\item[Rang 6] Choisissez deux capacités de rang supérieur. Donnez-les tous les deux comme options pour le Focus; le PJ choisira l'un ou l'autre.
Une des options devrait utiliser le paradigme du focus pour améliorer la défense, la protection ou une capacité pour se débarraser des dommages.

L'autre option peut être une manière différente de défendre. Dans certains cas, la meilleure des défenses étant une bonne attaque, cette option peut fournir une capacité offensive de rang supérieur  pour rester dans le thème du focus, que ce soit une augmentation des dommages pendant l'attaque, ou un meilleur contrôle d'une transformation physique instable.
\end{description}
%%%%%%%%%%%%%%%%%%%%%%%%%%%%%%%%%%%%%%%%%%%%%%%%%%
\section*{Foci de Superhéro}
%%%%%%%%%%%%%%%%%%%%%%%%%%%%%%%%%%%%%%%%%%%%%%%%%%
\section*{Personnaliser des Focus}
De temps en temps, tout le contenu d'un Focus n'est pas adapté à l'idée d'un personnage, ou peut-être que le MJ a besoin d'indications supplémentaires pour créer un nouveau Focus. Dans tous les cas, la solution réside dans les Capacités des Focus à leurs niveaux de base.

Pour chaque rang, un joueur peut sélectionner une des Capacités ci-dessous à la place de la Capacité fournie par le rang. Plusieurs de ces Capacités de remplacement, particulièrement aux rangs les plus élevés, peuvent impliquer une modification corporelle, de l'intégration de systèmes high-tech, d'apprendre des sorts puissants, de découvrir des secrets interdits, ou quelque chose similaire approprié au genre.

\textbf{Rang 1}

* Potentiel amélioré
* Prouesses au combat

\textbf{Rang 2}

* Capacité de rang inférieur: choisissez une Capacité de remplacement de rang 1 ci-dessus.
* Compétence avec les attaques
* Compétence en Défense Supérieure
* Pratique de toutes les armes

\textbf{Rang 3}

* Capacité de rang inférieur: choisissez n'importe quelle Capacité de remplacement de rang 1 ou 2 ci-dessus.
* Armure Corporelle
* Santé incroyable

\textbf{Rang 4}

* Capacité de rang inférieur: choisissez n'importe quelle Capacité de remplacement de Rang 1, 2, ou 3, ci-dessus.
* Armes intégrées
* Résistance au poison

\textbf{Rang 5}

* Capacité de rang inférieur: choisissez n'importe quelle Capacité de remplacement de Rang 1, 2, 3, ou 4 ci-dessus.
* Adaptation
* Champ défensif

\textbf{Rang 6}

* Capacité de rang inférieur: choisissez n'importe quelle Capacité de remplacement de Rang 1, 2, 3, 4, ou 5 ci-dessus.
* Abilities
* Champ réactif

\include{./tex/Chapter-09.tex}
%#######################################################################
%            CHAPTER 10
%#######################################################################
\startchapter{équipement}{ch:chapter10}{\getcolorpartone}
% \chapter{équipement}
% \label{ch:chapter10}
% \fancypagestyle{plain}{ %
% \fancyhf{} % remove everything
% \renewcommand{\headrulewidth}{1pt} % remove lines as well
% \renewcommand{\footrulewidth}{0pt}
% \xpretocmd\headrule{\color{BlueViolet}}{}{\PatchFailed}
% \fancyhead[RO]{\textcolor{BlueViolet}{\uppercase{é}quipement}}
% \fancyhead[LE]{\textcolor{BlueViolet}{CYPHER SYSTEM}}
% }


%#######################################################################
%   PART 2 Les règles
%#######################################################################
\part*{\uppercase{Les Règles}}
%____________________________________________________________________
% https://distrib-coffee.ipsl.jussieu.fr/pub/mirrors/ctan/macros/latex/contrib/titlesec/titlesec.pdf
% https://borntocode.fr/latex-personnaliser-les-titres-chapter/
%% define format for chapter for part 2
\titleformat{\chapter} % command
    [display] % shape
    {\bfseries\Large\centering} % format
    {\textcolor{RawSienna}{Chapitre \ \thechapter \\ \huge \uppercase{#1}} \\ \afficheimagechapitre{\thechapter}} % label
    {0.5ex} % sep
    {
        \centering
    } % before-code
    [
    \centering
    ] % after-code
    \titlespacing*{\chapter}
    {0mm} %left
    {-0pt} %before-step
    {0mm} %after-step
    {} %right
%____________________________________________________________________
%% define format for section
% numberless
\titleformat{name=\section,numberless}[runin] 
    {\normalfont\Large\bfseries}
    {\\ \textcolor{BlueVRawSiennaiolet} {#1}}
    {20pt}
    {\Large}
    [\\]
%____________________________________________________________________
%% define format for subsection
    \titleformat{\subsection} % command
    [display] % shape
    {\bfseries\large} % format
    {\textcolor{RawSienna}{#1}} % label
    {0.5ex} % sep
    {
        \large
    } % before-code

    \titlespacing*{\subsection}
    {0mm} %left
    {-0pt} %before-step
    {0mm} %after-step
    {} %right
%____________________________________________________________________
% numberless
\titleformat{name=\subsection,numberless}[runin] 
    {\normalfont\large\bfseries}
    {\\ \textcolor{RawSienna} {#1}}
    {20pt}
    {\large}
    [\\]
%#######################################################################

%#######################################################################
%   Part 3 Les Genres
%#######################################################################
\part*{\uppercase{Les Genres}}
%____________________________________________________________________
% https://distrib-coffee.ipsl.jussieu.fr/pub/mirrors/ctan/macros/latex/contrib/titlesec/titlesec.pdf
% https://borntocode.fr/latex-personnaliser-les-titres-chapter/
%% define format for chapter for part 3
\titleformat{\chapter} % command
    [display] % shape
    {\bfseries\Large\centering} % format
    {\textcolor{OliveGreen}{Chapitre \ \thechapter \\ \huge \uppercase{#1}} \\ \afficheimagechapitre{\thechapter}} % label
    {0.5ex} % sep
    {
        \centering
    } % before-code
    [
    \centering
    ] % after-code
    \titlespacing*{\chapter}
    {0mm} %left
    {-0pt} %before-step
    {0mm} %after-step
    {} %right
%____________________________________________________________________
%% define format for section
% \titleformat{\section} % Command to format section titles
% {\normalfont\Large\bfseries} % Format: normal font, large size, bold
% {\thesection}{1em} % Label format: section number followed by 1em space
% {} % Before the title
% [\titlerule] % After the title: horizontal line

% \titleformat{\section} % command
%     [display] % shape
%     {\bfseries\Large} % format
%     {\textcolor{BlueViolet}{#1}} % label
%     {0.5ex} % sep
%     {
%         \Large
%     } % before-code

%     \titlespacing*{\section}
%     {0mm} %left
%     {-0pt} %before-step
%     {0mm} %after-step
%     {} %right
%____________________________________________________________________
% numberless
\titleformat{name=\section,numberless}[runin] 
    {\normalfont\Large\bfseries}
    {\\ \textcolor{OliveGreen} {#1}}
    {20pt}
    {\Large}
    [\\]
%____________________________________________________________________
%% define format for subsection
    % \titleformat{\subsection} % command
    % [display] % shape
    % {\bfseries\large} % format
    % {\textcolor{OliveGreen}{#1}} % label
    % {0.5ex} % sep
    % {
    %     \large
    % } % before-code

    % \titlespacing*{\subsection}
    % {0mm} %left
    % {-0pt} %before-step
    % {0mm} %after-step
    % {} %right
%____________________________________________________________________
% numberless
\titleformat{name=\subsection,numberless}[runin] 
    {\normalfont\large\bfseries}
    {\\ \textcolor{OliveGreen} {#1}}
    {20pt}
    {\large}
    [\\]
%#######################################################################
%#######################################################################
%   Part 4 Maîtriser Le Jeu
%#######################################################################
\part*{\uppercase{Maîtriser le Jeu}}
%____________________________________________________________________
% https://distrib-coffee.ipsl.jussieu.fr/pub/mirrors/ctan/macros/latex/contrib/titlesec/titlesec.pdf
% https://borntocode.fr/latex-personnaliser-les-titres-chapter/
%% define format for chapter for part 4
\titleformat{\chapter} % command
    [display] % shape
    {\bfseries\Large\centering} % format
    {\textcolor{ForestGreen}{Chapitre \ \thechapter \\ \huge \uppercase{#1}} \\ \afficheimagechapitre{\thechapter}} % label
    {0.5ex} % sep
    {
        \centering
    } % before-code
    [
    \centering
    ] % after-code
    \titlespacing*{\chapter}
    {0mm} %left
    {-0pt} %before-step
    {0mm} %after-step
    {} %right
%____________________________________________________________________
%% define format for section
% \titleformat{\section} % Command to format section titles
% {\normalfont\Large\bfseries} % Format: normal font, large size, bold
% {\thesection}{1em} % Label format: section number followed by 1em space
% {} % Before the title
% [\titlerule] % After the title: horizontal line

% \titleformat{\section} % command
%     [display] % shape
%     {\bfseries\Large} % format
%     {\textcolor{BlueViolet}{#1}} % label
%     {0.5ex} % sep
%     {
%         \Large
%     } % before-code

%     \titlespacing*{\section}
%     {0mm} %left
%     {-0pt} %before-step
%     {0mm} %after-step
%     {} %right
%____________________________________________________________________
% numberless
\titleformat{name=\section,numberless}[runin] 
    {\normalfont\Large\bfseries}
    {\\ \textcolor{ForestGreen} {#1}}
    {20pt}
    {\Large}
    [\\]
%____________________________________________________________________
%% define format for subsection
    % \titleformat{\subsection} % command
    % [display] % shape
    % {\bfseries\large} % format
    % {\textcolor{ForestGreen}{#1}} % label
    % {0.5ex} % sep
    % {
    %     \large
    % } % before-code

    % \titlespacing*{\subsection}
    % {0mm} %left
    % {-0pt} %before-step
    % {0mm} %after-step
    % {} %right
%____________________________________________________________________
% numberless
\titleformat{name=\subsection,numberless}[runin] 
    {\normalfont\large\bfseries}
    {\\ \textcolor{ForestGreen} {#1}}
    {20pt}
    {\large}
    [\\]
%#######################################################################

\backmatter
\end{document}

