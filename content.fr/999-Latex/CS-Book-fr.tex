\documentclass[a4paper,french,twocolumn]{book}
\usepackage[T1]{fontenc}    % Encodage T1 (adapté au français)
%French-specific commands
%____________________________________________________________________
% https://en.wikibooks.org/wiki/LaTeX/Special_Characters
\usepackage[french]{babel}
\usepackage[autolanguage]{numprint} % for the \nombre command

\usepackage{lmodern}        % Caractères plus lisibles
\usepackage{babel}          % Réglages linguistiques (avec french)
\pagestyle{empty}           % N'affiche pas de numéro de page
\usepackage[explicit]{titlesec}  % pour modifier le style de chapitre
\usepackage{color}        % pour les coleurs de police
\usepackage[table,dvipsnames]{xcolor}  % pour les couleurs des tableaux https://en.wikibooks.org/wiki/LaTeX/Colors
\usepackage{graphicx}     % pour inclure des images
\usepackage{wrapfig}
%\usepackage[dvipsnames]{xcolor} % pour les couleurs
\usepackage{fancyhdr} % pour les entete
\usepackage{ifthen}  % branchements conditionnels
\usepackage{xpatch} 
\usepackage[export]{adjustbox}  % pour positionner des images dans la page
\usepackage{tabularx}  % pour les tableaux
\usepackage{wrapfig}   % pour le que le texte ne recouvre pas les figures/tableaux
\usepackage{stfloats}   % pour le que le texte ne recouvre pas les figures/tableaux
\usepackage{multicol}   % gestion de plusieurs colonnes (dans une cellule de tableau pas ex)
\usepackage{enumitem}   % pour la personnalisation des liste
\usepackage{hyperref}   % for linking references https://en.wikibooks.org/wiki/LaTeX/Hyperlinks
%____________________________________________________________________
%%  Default Font
\usepackage[sfdefault]{cabin}
\renewcommand*\familydefault{\sfdefault} %% Only if the base font of the document is to be sans serif
\usepackage[T1]{fontenc}

%____________________________________________________________________
% page margins
% https://www.overleaf.com/learn/latex/Page_size_and_margins
% https://fr.overleaf.com/learn/latex/Single_sided_and_double_sided_documents
\usepackage[
  top=20mm,
  bottom=20mm,
  marginparwidth=111pt
  %textwidth=345pt,
]{geometry}
%____________________________________________________________________
\newcommand*{\definitchapitre}[2]{\chapter{#1 \label{#2} \definitentete{#1}}}
%____________________________________________________________________
% commande pour afficher l'image de tête de chapitre
\newcommand*{\afficheimagechapitre}[1]{
    \hspace*{-72pt}
    \ifthenelse{\equal{#1}{1}}{
        \includegraphics[height=8.25cm]{CS-page4-01.png}
    }
    {
        \ifthenelse{\equal{#1}{2}}{
            \includegraphics[height=8.25cm]{CS-page5-01.png}
        }
        {
            \ifthenelse{\equal{#1}{3}}{
                \includegraphics[height=8.25cm]{CS-page7-01.png}
                }
                {
                    \ifthenelse{\equal{#1}{4}}{
                        \includegraphics[height=8.25cm]{CS-page14-01.png}
                        }
                        {
                            \ifthenelse{\equal{#1}{5}}{
                                \includegraphics[height=8.25cm]{CS-page20-01.png}
                                }
                                {
                                    \ifthenelse{\equal{#1}{6}}{
                                        \includegraphics[height=8.25cm]{CS-page34-01.png}
                                        }
                                        {
                                            \ifthenelse{\equal{#1}{7}}{
                                                \includegraphics[height=8.25cm]{CS-page38-01.png}
                                                }
                                                {
                                                    \ifthenelse{\equal{#1}{8}}{
                                                        \includegraphics[height=8.25cm]{CS-page60-01.png}
                                                        }
                                                        {
                                                            \ifthenelse{\equal{#1}{9}}{
                                                                \includegraphics[height=8.25cm]{CS-page95-01.png}
                                                                }
                                                                {
                                                                    \ifthenelse{\equal{#1}{10}}{
                                                                        \includegraphics[height=8.25cm]{CS-page201-01.png}
                                                                    }{
                                                                        IMAGE A DEFINIR
                                                                    }
                                                                }
                                                        }
                                                }
                                        }
                                }
                        }
                }
        }
    }
}
%____________________________________________________________________---------------
%  Style pour les en-têtes
\newcommand*{\definitentete}[1]{\fancypagestyle{plain}{ %
    \fancyhf{} % remove everything
    \renewcommand{\headrulewidth}{1pt} % remove lines as well
    \renewcommand{\footrulewidth}{1pt}
    \fancyhead[RO]{#1}
    \fancyhead[LE]{CYPHER SYSTEM}
}
}
%____________________________________________________________________
%  définition des en-têtes
\setlength{\headheight}{15.2pt}
\pagestyle{fancy}
\fancyhead[LE]{CYPHER SYSTEM}
%____________________________________________________________________-
\newcommand*{\smallcslogo}{\includegraphics[height=3mm]{CS-mini-logo.png}}
%  définition du style des listes de nom capacités
\newlist{abnamelist}{itemize}{1}
\setlist[abnamelist]{label={\smallcslogo}}
%____________________________________________________________________
\newcommand*{\abilitylistitem}[2]{#1 (\hyperref[#2]{\pageref{#2}})}
%____________________________________________________________________-
%  définition du style des listes de capacités (chapitre 9)
%\newlist{ablist}{itemize}{1}
%\setlist[ablist]{leftmargin=*,label={\includegraphics[height=3mm]{CS-mini-logo.png} :}}
%\setlist[ablist]{label={\includegraphics[height=3mm]{CS-mini-logo.png}}}
\newcommand*{\abilitydetaillistitem}[2]{\item \textbf{#1 :} #2}
%#######################################################################
\title{CYPHER SYSTEM LIVRE DES R\`{E}GLES}
\author{Monte Cook, Bruce R. Cordell, Sean K. Reynolds}
%#######################################################################
\begin{document}
\frontmatter
\maketitle
\tableofcontents
\mainmatter
%____________________________________________________________________
%   reset headers
\fancypagestyle{plain}{ %
    \fancyhf{} % remove everything
}
%#######################################################################
    \part*{LE CYPHER SYSTEM}
    %____________________________________________________________________---------------
    % https://distrib-coffee.ipsl.jussieu.fr/pub/mirrors/ctan/macros/latex/contrib/titlesec/titlesec.pdf
    % https://borntocode.fr/latex-personnaliser-les-titres-chapter/
    %% define format for chapter
    \titleformat{\chapter} % command
        [display] % shape
        {\bfseries\Large\centering} % format
        {\textcolor{RedViolet}{Chapitre \ \thechapter \\ \huge \uppercase{#1}} \\ \afficheimagechapitre{\thechapter}} % label
        {0.5ex} % sep
        {
            \centering
        } % before-code
        [
        \centering
        ] % after-code
    \titlespacing*{\chapter}
        {0mm} %left
        {-0pt} %before-step
        {0mm} %after-step
        {} %right
    %____________________________________________________________________---------------
    %% define format for section
    \titleformat{\section} % Command to format section titles
    {\normalfont\Large\bfseries} % Format: normal font, large size, bold
    {\thesection}{1em} % Label format: section number followed by 1em space
    {} % Before the title
    [\titlerule] % After the title: horizontal line

    \titleformat{\section} % command
        [display] % shape
        {\bfseries\Large} % format
        {\textcolor{RedViolet}{#1}} % label
        {0.5ex} % sep
        {
            \Large
        } % before-code

        \titlespacing*{\section}
        {0mm} %left
        {-0pt} %before-step
        {0mm} %after-step
        {} %right
    %____________________________________________________________________---------------
    % numberless
    \titleformat{name=\section,numberless}[runin] 
        {\normalfont\Large\bfseries}
        {\\ \textcolor{RedViolet} {#1}}
        {20pt}
        {\Large}
        [\\]
    
    %#######################################################################
    %            CHAPTER 1
    %#######################################################################
    \chapter{Des mondes d'aventures}\label{ch:chapter1}
    %---------------------------
    % https://texdoc.org/serve/fancyhdr/0
    \fancypagestyle{plain}{ %
        \fancyhf{} % remove everything
        \renewcommand{\headrulewidth}{1pt} % remove lines as well
        \renewcommand{\footrulewidth}{0pt}
        \xpretocmd\headrule{\color{RedViolet}}{}{\PatchFailed}
        \fancyhead[RO]{\textcolor{RedViolet}{\textsc{Des mondes d'aventures}}}
        \fancyhead[LE]{\textcolor{RedViolet}{CYPHER SYSTEM}}
    }
    Finalement, tout ce que nous voulons, c'est précisément de jouer au jeu auquel nous voulons jouer. Toutes les Meneuses et Meneurs ont un cadre de campagne parfait dans un coin de leur tête. Les joueuses et joueurs ont cette idée de personnage qui serait leur meilleur personnage jamais créé, si seulement ils avaient la chance de le créer et de le jouer. Ces rêves de jouer exactement ce que vous souhaitez jouer sont la raison d'être de ce livre.

    Il s'agit d'une version révisée du livre de règles original du Cypher System. J'ai réuni le contenu de ce livre à partir des jeux du Cypher System existant à l'époque \textendash Numenera et The Strange. C'était essentiellement une compilation de tout ce matériel de jeu, ainsi que beaucoup de suggestions sur la façon de l'utiliser de la manière que vous souhaitez. Les joueurs et les Meneuses nous ont dit que cela répondait bien à ces besoins.

    Nous avons beaucoup appris depuis. Pas tellement sur les règles du système elles-mêmes \textendash qui restent essentiellement inchangées \textendash mais sur la manière dont nous voulons utiliser ce type de livre, et donc sur la manière de présenter l'information. Il est vraiment difficile de créer quelque chose qui soit utilisable par n'importe qui pour n'importe quoi et de le présenter de manière réellement conviviale. Mais je pense que c'est exactement ce que nous avons fait avec ce livre. Les innovations que vous trouverez dans ces pages \textendash la façon dont toutes les capacités ont été cataloguées pour que vous puissiez les utiliser comme bon vous semble, l'accent mis sur les cyphers subtils, l'étendue des genres présentés \textendash rendent ce matériel plus facile à utiliser et plus facile à personnaliser.

    Le tout nouveau contenu, comme le système d'arc narratif, le système d'artisanat, les informations supplémentaires sur les genres, etc., rendra, je l'espère, vos parties plus amusantes et vos histoires plus riches.

    Mais permettez-moi de répéter : nous n'avons pas changé la façon dont le jeu fonctionne. Vous pouvez utiliser ce livre en parallèle avec l'ancien livre de règles du Cypher System sans trop de problèmes.

    D'une certaine manière, ce livre est un volume complémentaire à un livre que j'ai écrit intitulé Your Best Game Ever. Ce livre est un guide indépendant du système de jeux pour comprendre et apprécier les jeux de rôle. Le présent ouvrage prend les idées et suggestions présentées dans ce dernier et leur donne un ensemble de règles qui les rend possibles. Mon objectif est de vous donner les outils pour avoir votre meilleur jeu jamais joué. Et cela, je crois, implique de pouvoir jouer dans le cadre et avec les personnages que vous avez toujours voulu.

    Maintenant, espérons-le, vous pouvez enfin le faire.


    Amusez-bien
    
    \includegraphics[height=20pt]{CS-page4-02.png}

    
    Monte Cook

    Mars 2019
    %#######################################################################
    %            CHAPTER 2
    %#######################################################################
    \chapter{uppercase{Tout est permis}\label{ch:chapter2}}
    %---------------------------
    % https://texdoc.org/serve/fancyhdr/0
    \fancypagestyle{plain}{ %
        \fancyhf{} % remove everything
        \renewcommand{\headrulewidth}{1pt} % remove lines as well
        \renewcommand{\footrulewidth}{0pt}
        \xpretocmd\headrule{\color{RedViolet}}{}{\PatchFailed}
        \fancyhead[RO]{\textcolor{RedViolet}{textsc{Tout est permis}}}
        \fancyhead[LE]{\textcolor{RedViolet}{CYPHER SYSTEM}}
    }
    Pour commencer, nous nous adressons directement aux meneuses (ou MJ). Les joueurs comme les MJ utiliseront ce livre, mais il est fort probable que ce soit d'abord le MJ qui le consulte.

    Ce que vous tenez entre vos mains est un guide. Un mode d'emploi. Vous ne pouvez pas simplement vous asseoir et commencer à jouer, car le manuel du Cypher System n'est pas conçu pour être utilisé de cette façon. Il vous faut d'abord y mettre quelque chose de votre propre invention. Il n'y a pas de cadre ni de monde prédéfinis ici. Le système est conçu pour vous aider à dépeindre n'importe quel monde ou cadre que vous pouvez imaginer.

    Considérez ce livre comme un coffre à jouets. Vous pouvez sortir ce que vous voulez et y jouer comme bon vous semble. Vous n'utiliserez pas tout ce qu'il contient, du moins pas d'un seul coup. Vous utiliserez des parties de ce livre pour construire le jeu que vous souhaitez jouer. Sortez quelques éléments et essayez-les. Remettez en place ceux qui ne vous conviennent pas, et essayez d'autres. Utilisez certains maintenant et gardez-en d'autres pour votre prochaine partie. Vous avez toute la liberté possible (en fait, de nombreux mondes).

    A propos de mondes, vous devez décider quel cadre de campagne utiliser, en fonction du genre que vous avez choisi. Cela peut être n'importe quoi. Prenez votre livre ou film préféré, ou concevoir quelque chose à partir de rien.

    Alors, en pratique, ce que vous choisissez ici, c'est l'expérience que vous voulez vivre\textendash et que vous voulez faire vivre aux joueurs. C'est une décision tellement fondamentale que tout le groupe devrait peut-être y participer. Demandez aux autres joueurs quel genre ils aiment et quels types d'expériences ils souhaitent vivre. C'est essentiel, car cela garantit que tout le monde obtient ce qu'il attend du jeu.

    Bien sûr, tout le contenu de ce livre ne convient pas à tous les genres. Vous, en tant que MJ, devrez le lire une fois que vous aurez choisi un genre et sélectionner les types, les axes et ainsi de suite. Ensuite, informez vos joueurs du matériel que vous avez décidé de rendre disponible afin qu'ils puissent créer des personnages adaptés au genre


    \section*{GENRES}
    Jeter un coup d'œil à la 3ième partie Genre qui contient un nombre de chapitres consacrés aux genres. Ce sont des catégories assez larges, et nous les utiliserons dans ce livre comme point de départ. Ces catégories sont : Fantasy, moderne, science-fiction, hoerreur, romance, super-héros, post-apocalyptique, contes de fées, et historique.

    Avec ces descriptions assez génériques, nous pouvons couvrir la plupart des cadres de jeu (mais probablement pas tous)  que vous pouvez jouer avec le Cypher System. Certains de ces genres nécessite du matériel unique, des artifacts, ou des descripteurs. Certains ont besoin de nouvelles règles pour améliorer l'immersion que vous recherchez.

    Nous parlons d'immersion, parce que sous beaucoup d'aspect, c'est ce qu'un genre est. Si vous voulez faire vivre l'expérience d'être terrifié par des zombies qui rodent autour de la maison de votre personnage, vous voulez de l'horreur. Si vous voulez faire vivre l'expérience d'être extrêmement puissant et utiliser ces pouvoirs pour protéger le monde des extra-terrestres, vous voulez des super-héros (avec peut-être une touche de science-fiction).


    \section*{CADRES DE CAMPAGNE}
    Bien que les genres soient des catégories utiles pour organiser vos idées, ce que vous allez réellement créer, c'est un cadre. Des étiquettes comme « science-fiction » ou « space opera » sont pratiques, mais au final, ce qui compte, c'est le cadre spécifique que vous établissez.

    Votre cadre \textendash qu'il s'agisse d'une création originale ou d'une adaptation\textendash vous appartient entièrement. Ne vous inquiétez pas de ce que d'autres pourraient considérer comme approprié pour un genre donné. Une fois que vous commencez à assembler votre cadre, vous voudrez peut-être parcourir à nouveau les sections sur la création de personnages dans ce livre. Ce qui est habituellement adapté à un genre fantastique, par exemple, peut ne pas convenir à votre propre univers de fantasy.

    Imaginons que, dans votre monde, la magie du feu soit toujours maléfique et uniquement pratiquée par des prêtres possédés par des démons. Dans ce cas, l'axe Porte une auréole de feu ne serait pas approprié pour les personnages des joueurs, même s'il est parfaitement acceptable dans d'autres jeux de fantasy.

    Plus vous définissez précisément les détails de votre cadre, plus il sera facile d'en ajuster les éléments. Et plus votre univers s'éloigne des clichés du genre, plus vous devrez adapter les choix possibles. Mais ce n'est pas un problème : les cadres spécifiques et distincts sont souvent les plus amusants, les plus mémorables et les plus engageants pour vos joueurs. Ils valent largement l'effort supplémentaire.


    \section*{ADAPTER LES RÈGLES}
    Parfois, vous devez modifier certaines choses pour qu'elles correspondent à vos besoins et envies. Prenons par exemple la saveur « magie » que vous pouvez attribuer à n'importe quel type présenté dans le chapitre 5. Elle s'appelle « magie » et possède de nombreux éléments associés à ce concept, mais il serait très simple de changer son nom en « psionique », « pouvoirs mutants » ou tout autre terme adapté à votre univers.

    En d'autres termes, sélectionner du contenu dans ce livre peut ne pas suffire. Vous pourriez avoir besoin d'ajuster certains éléments ici et là. Heureusement, la plupart du matériel est conçu pour être modifié ou adapté. En fait, grâce à la simplicité des mécaniques de base du Cypher System, effectuer des ajustements est un jeu d'enfant. Ce n'est pas un système où un petit changement risque d'entraîner un effet domino aux conséquences imprévues.

    Dans le chapitre 7, vous trouverez des directives pour créer de nouveaux descripteurs. Le chapitre 8 contient une section entière consacrée à la création de nouveaux axes adaptés à votre propre jeu. De plus, les types de personnages du chapitre 5 sont conçus pour être personnalisés et remodelés. 

    Lorsque vous apportez des modifications, concentrez-vous moins sur l'équilibrage du jeu et davantage sur la narration des histoires que vous souhaitez raconter, tout en permettant aux joueurs de créer et d'incarner les personnages qu'ils veulent. Si vous parvenez à faire ces deux choses correctement, tout le monde sera satisfait. Et au final, c'est précisément ce qu'est l'équilibrage du jeu. 

    Vous pouvez également consulter le chapitre 25 pour approfondir la façon de modifier les mécaniques du jeu. Mais dans l'ensemble, ce chapitre vous rappellera ce que vous venez de lire : c'est *votre* jeu, et vous êtes libre d'en faire ce que vous voulez.

    %#######################################################################
    %            CHAPTER 3
    %#######################################################################
    \chapter{\uppercase{Comment jouer au Cypher System}}\label{ch:chapter3}
    %---------------------------
    % https://texdoc.org/serve/fancyhdr/0
    \fancypagestyle{plain}{ %
        \fancyhf{} % remove everything
        \renewcommand{\headrulewidth}{1pt} % remove lines as well
        \renewcommand{\footrulewidth}{0pt}
        \xpretocmd\headrule{\color{RedViolet}}{}{\PatchFailed}
        \fancyhead[RO]{\textcolor{RedViolet}{\textsc{Comment jouer au Cypher System}}}
        \fancyhead[LE]{\textcolor{RedViolet}{CYPHER SYSTEM}}
    }
    Les règles du Cypher System sont assez simples à la base, car toute la mécanique de jeu repose sur quelques concepts fondamentaux.

    Ce chapitre donne une brève explication de comment jouer, c'est une bonne entrée en matière pour comprendre ces mécanismes. Une fois que vous avez compris les concepts de base, vous pourrez aller consulter Chapitre 11: Les Règles du Jeux pour des informations plus détaillées.
    
    Le Cypher System utilise un dé à 20 faces (1d20) pour déterminer le résultat de la plupart des actions. A chaque fois qu'un jet de dé de n'importe quel type est demandé et que le type de dé n'est pas spécifié, jeter un d20.

    La Meneuse définit une difficulté pour une tâche donnée. Il y a 10 degrés de difficulté. Ainsi, la difficulté d'une tâche peut être évaluée sur une échelle de 1 à 10.

    Chaque difficulté a un nombre seuil associé. Ce nombre seuil est toujours trois fois la difficulté de la tâche, ainsi une tâche de difficulté de 1 a un nombre seuil de 3, mais une tâche de difficulté de 4 a une nombre de seuil de 12. Pour réussir la tâche, vous devez obtenir un jet de dé égal ou supérieur. Consulter la table des Difficulté d'une Tâche pour voir comment cela fonctionne.

    Les compétences des personnages, des circonstances favorables ou un bon équipement peuvent diminuer la difficulté d'une tâche. Par exemple, si un personnage est entraîné en escalade, il transforme une tache d'escalade de difficulté 6 en une tache d'escalade de difficulté 5. Ceci est appelé faciliter la difficulté d'un cran (ou juste faciliter la difficulté, étant assumé qu'elle est facilitée d'un cran). Si il est spécialisé dans l'escalade, il transforme une tâche d'escalade de difficulté 6 en une tâche d'escalade de difficulté 4. Ceci est appelé *faciliter la difficulté de deux crans*. Diminuer la difficulté d'une tâche est aussi appelée *faciliter une tâche*. Certaines situations peuvent augmenter la difficulté d'une tâche. Augmenter la difficulté d'un cran est aussi appelé « entraver une tâche ».

    % https://en.wikibooks.org/wiki/LaTeX/Tables
    % https://www.overleaf.com/learn/latex/Tables
    \begin{table*}[t!]
        \hspace*{-90pt}
        \centering
         \begin{tabular}{ c c c p{10cm} } 
        \multicolumn{4}{ l }{\Large \textcolor{RedViolet}{Difficulté de la Tâche}} \\
        \textbf{Difficulté de la Tâche} & \textbf{Description} & \textbf{Nombre Cible} & \textbf{Aide} \\ [0.5ex]
         \rowcolor{RedViolet!50}
         0 & Routine & 0 & Tâche typique qui demande de l'attention, mais la plupart des gens y arrivent habituellement.  \\ 
         \rowcolor{RedViolet!20}
         1 & Simple & 3 & La plupart des gens arrivent à le faire la plupart du temps. \\ 
         \rowcolor{RedViolet!50}
         2 & Standard & 6 & Tâche typique qui demande de l'attention, mais la plupart des gens y arrivent habituellement.  \\
         \rowcolor{RedViolet!20}
         3 & Exigeant & 9 & Demande une bonne concentration; la plupart des gens ont une chance sur deux de réussir. \\
         \rowcolor{RedViolet!50}
         4 & Difficile & 12 & les personnes entraînées ont 50\% de chance de réussir. \\
         \rowcolor{RedViolet!20}
         5 & Gageure & 15 & Même les personnes entraînés échouent souvent.  \\
         \rowcolor{RedViolet!50}
         6 & Intimidant & 18 & Les personnes normales ne réussissent quasiment jamais. \\
         \rowcolor{RedViolet!20}
         7 & Formidable & 21 & Impossible sans de bonnes compétences ou beaucoup d'effort. \\
         \rowcolor{RedViolet!50}
         8 & Héroïque & 24 & Une tâche digne d'être racontée durant des années après. \\
         \rowcolor{RedViolet!20}
         9 & Immortel & 27 & Une tâche digne de légendes qui durent quelques générations. \\
         \rowcolor{RedViolet!50}
         10 & Impossible & 30 & Une tâche que des humains normaux n'envisageraient pas (mais qui respecte les lois de la physique). \\
         \end{tabular}
         \label{table:taskdifficulty}
    \end{table*}


    Une compétence est une catégorie de connaissance, de capacité ou d'activité relative à une tâche, telle que l'escalade, la géographie, ou la persuasion. Un personnage qui possède une compétence est meilleur pour accomplir une tâche associée qu'un personnage qui ne l'a pas. Le niveau de compétence d'un personnage est soit entraîné (raisonnablement compétent) ou spécialisé (très compétent).

    Si vous êtes entraîné dans une compétence relative à une tâche, vous facilitez la difficulté de cette tâche d'un cran. Si vous êtes spécialisé, vous la facilitez de deux crans. Une compétence ne peut jamais diminuer la difficulté d'une tâche de plus de deux crans.

    Tout autre élément qui diminue la difficulté (de l'aide d'un allié, une pièce spécifique d'un équipement, ou d'autres avantages) est appelé un atout. Les atouts ne peuvent jamais diminuer la difficulté de plus de deux crans.

    Vous pouvez aussi diminuer la difficulté d'une tâche en appliquant de l'Effort.

    Pour résumer, trois choses peuvent diminuer la difficulté d'une tâche: les compétences, les atouts, et l'Effort.

    Si vous pouvez faciliter une tâche jusqu'à ce sa difficulté soir réduite à zéro, vous réussissez automatiquement et vous n'avez pas besoin de jeter un dé.


    \section*{QUAND JETER UN DÉ ?}
    A chaque fois qu'un personnage tente d'accomplir une tâche, la Meneuse assigne une difficulté à cette tâche, et vous lancez un d20 contre le nombre seuil associé.

    Quand vous sautez d'un véhicule en flamme, balancez une hache sur une bête mutante, nagez dans une rivière en crue, identifiez un objet étrange, convainquez un marchand de vous faire un prix, fabriquez un objet, utilisez un pouvoir pour contrôler l'esprit d'un adversaire, ou utilisez un fusil blaster pour faire un trou dans un mur, vous jetez un d20.

    Toutefois, si vous tentez quelque chose dont la difficulté est de 0, alors aucun jet n'est nécessaire\textendash vous réussissez automatiquement. Beaucoup d'actions ont une difficulté de 0. Par exemple, marcher dans une pièce et ouvrir une porte, utiliser une capacité spéciale pour annuler la gravité pour voler, utiliser une capacité pour protéger votre ami de la radiation, ou activer un appareil (dont vous comprenez le fonctionnement) pour ériger un champs de force. Toutes ces tâches sont des actions de routine et ne nécessitent aucun jet de dé.

    En utilisant des compétences, des atouts, et de l'Effort, vous pouvez faciliter la difficulté d'une tâche potentielle jusqu'à zéro et ainsi supprimer la nécessité de jeter un dé. Marcher sur une poutre est compliquer pour la plupart des personnes, mais pour un gymnaste expérimenté c'est la routine. Vous ne pouvez jamais faciliter la difficulté d'une attaque sur un adversaire jusqu'à 0 et réussir sans jeter un dé.

    Si il n'y pas besoin de jeter un dé, il n'y a aucune chance d'échouer. Toutefois, s'il n'y a aucune chance pour un succès remarquable (dans le Cypher System, cela signifie d'habitude de faire un 19 ou 20, ce qui est appelé un jet spécial).

    \begin{table*}[t!]
        \hspace*{-50pt}
        \centering
         \begin{tabular}{ | p{17cm} | } 
            \hline
            \rowcolor{RedViolet!20} {\Large \textcolor{RedViolet}{GLOSSAIRE}} \\
            \rowcolor{RedViolet!20} \textbf{Meneuse:} La joueuse ou le joueur qui n'interprète pas un personnage mais qui plutôt indique la direction que prend l'histoire et qui interprète tous les PNJs. \\
            \rowcolor{RedViolet!20} \textbf{Personnage non-joueur (PNJ):} Personnages interprèté par la Meneuse. Pensez à eux comme des personnages mineurs dans l'histoire, ou les adversaires. Cela inclut toute sorte de créature aussi bien que les êtres humains. \\
            \rowcolor{RedViolet!20} \textbf{Groupe:} Un groupe de personnages joueur (et peut être quelques alliés PNJs). Personnage joueur (PJ): un personnage interprèté par une joueuse ou un joueur plutôt que par la Meneuse. Penser aux PJs comme les principaux personnages de l'histoire. \\
            \rowcolor{RedViolet!20} \textbf{Joueur:} Les joueurs qui interprète des personnages dans le jeu. \\
            \rowcolor{RedViolet!20} \textbf{Session:} Une simple expérience de jeu. Cela dure habituellement quelques heures. Quelque fois, une aventure peut être terminée en une session. La plupart du temps, une aventure est étalée sur plusieurs sessions. \\
            \rowcolor{RedViolet!20} \textbf{Aventure:} Une partie de la campagne avec un commencement et une fin. C'est en général définit au départ par un objectif pour les PJs et à la fin par si ou non ils parviennent à atteindre l'objectif. \\
            \rowcolor{RedViolet!20} \textbf{Campagne:} Une série de sessions liées entre elles par une histoire principale avec les mêmes PJs. Souvent, mais pas toujours, une campagne implique un certain nombre d'aventures. \\
            \hline
         \end{tabular}
         \label{table:basicglossary}
    \end{table*}

    
    \begin{table*}[b]
        \hspace*{0pt}
        \centering
         \begin{tabular}{ | m{14cm} | } 
            \hline
            \rowcolor{RedViolet!20} Avec le Cypher System, ce sont les joueurs qui font tous les jets de dés. Si un personnage attaque une créature, le joueur lance le dé pour l'attaque. Si une créature attaque un personnage, le joueur fait un jet de défense.. \\
            \hline
         \end{tabular}
         \label{table:hint3_1}
    \end{table*}

    \section*{COMBAT}

    Faire une attaque dans un combat fonctionne comme tout autre tâche: la Meneuse assigne une difficulté à la tâche, et vous jetez un d20 contre le nombre seuil.

    La difficulté de votre jet d'attaque dépend d'à quel point votre adversaire est puissant. Comme toutes les tâches ont une difficulté qui va de 1 à 10, les créatures ont un niveau qui va de 1 à 10. LA plupart du temps, la difficulté de votre jet d'attaque est la même que le niveau de la créature. Par exemple, si vous attaquez un bandit de niveau 2, c'est une tâche de difficulté 2, donc le nombre seuil est de 6.

    Il est important de noter que ce sont uniquement les joueurs qui lancent les dés.. Si un personnage attaque une créature, le joueur fait le jet d'attaque. Si une créature attaque un personnage, le joueur fait un jet de défense.Les dommages infligés par une attaque ne sont pas déterminés par un jet de dé\textendash c'est un nombre fixe basé sur l'arme ou l'attaque utilisée. Par exemple, une lance inflige toujours 4 points de dommages.

    Votre caractéristique d'Armure réduit les dommages que vous subissez directement. Vous obtenez de l'Armure en portant une armure physique (un blouson de cuir dans l'époque contemporaine ou une côte de maille dans une campagne médiévale) ou de capacités spéciales.. Comme les dommages infligés par les armes, l'Armure est un nombre fixe, par un jet de dé. Si vous êtes attaqué, soustrayez votre Armure des dommages que vous subissez. Par exemple, un blouson de cuir vous donne +1 à votre Armure, cela signifie que vous diminuez d'un point les dommages subis. Si l'Armure réduit les dommages à 0, vous ne subissez aucun dommage de l'attaque.

    Quand vous verrez le mot "Armure" avec une majuscule dans les règles du jeu (en dehors du nom d'une capacité spéciale), cela réfère à votre caractéristique d'Armure\textendash le nombre que vous soustrayez des dommages subis. Quand vous voyez le mot "armure" avec une minuscule, cela réfère à toute armure physique que vous pourriez porter.

    Les armes physiques typiques sont classées en trois catégories: légères, moyennes et lourdes.

    \textbf{Les armes légères} n'infligent que 2 points de dommages, mais elles facilitent les jets d'attaque car elles sont rapides et simples à utiliser. Les armes légères sont les poings, les pieds, les massues, les couteaux, les hachettes, les rapières, les petits pistolets, et ainsi de suite. Les armes qui sont particulièrement petites sont des armes légères.

    \textbf{Les armes moyennes} infligent 4 point de dommage. Les armes moyennes incluent les épées, les haches de bataille, les maces, les arbalètes, les lances, les pistolets, les blasters, et ainsi de suite. La plupart des armes sont moyennes. tout ce qui peut être utilisé à une main (même si c'est souvent utilisé à deux mains comme les bâtons ou les lances) est une arme moyenne.

    \textbf{Les armes lourdes} infligent 6 points de dommage et elles doivent être utilisées à deux mains pour attaquer. Les armes lourde sont les grandes épées, les grandes haches, les marteaux de guerre, les hallebardes, les arbalètes lourdes, les fusils blaser, et ainsi de suite. Tout ce qui doit être utilisé avec deux mains est une arme lourde.


    \section*{LES JETS SPÉCIAUX}
    Quand le résultat d'un jet de dé est un 19 naturel (le d20 indique 19) et que c'est un succès, alors vous obtenez un effet mineur. En combat, un effet mineur inflige 3 points de dommages supplémentaires à votre attaque, ou, si vous préférez un résultat spécial, vous pouvez décidez à la place de repousser votre adversaire, de le distraire ou quelque chose de similaire. Quand ce n'est pas en combat, un effet mineur peut signifier que vous réussissez l'action avec une grâce particulière. Par exemple, si vous sautez d'un rebord, vous atterrissez légèrement sur vos pieds, ou si vous essayez de persuader quelqu'un, vous le convainquez que vous êtes plus malin que vous ne l'êtes. En d'autres mots, non seulement vous réussissez mais vous allez un peu plus loin.

    Quand le résultat d'un jet de dé est un 20 naturel (le d20 indique 20) et que c'est un succès, alors vous obtenez un effet majeur C'est similaire à un effet mineur, mais le résultat est plus impressionnant. En combat, un effet majeur inflige 4 points de dommage supplémentaires à votre attaque, mais comme pour un effet mineur, vous pouvez choisir à la place un effet plus dramatique, comme de mettre à terre votre adversaire, l'étourdir, ou prendre une action supplémentaire. En dehors du combat, un effet majeur signifie qu'un bénéfice survient, en fonction des circonstances. Par exemple, quand vous grimpez sur un mur, vous faites l'ascension deux fois plus vite. Quand un jet de dé vous confère un effet majeur, vous pouvez choisir un effet mineur à la place si vous préférez.

    En combat (et seulement en combat), quand vous faites un 17 ou un 18 naturel à votre jet d'attaque, vous ajoutez respectivement 1 ou 2 points aux dommages. Cela ne donne pas d'effet spécial, seulement des dommages supplémentaires.

    Faire un 1 au jet de dé est toujours mauvais. Cela signifie que la Meneuse peut introduire une nouvelle complication dans la rencontre.

    \begin{figure*}[b]
        \includegraphics[height=10cm,center]{CS-page10-01.png}
    \end{figure*}

    \section*{PORTÉE ET RAPIDITÉ}
    Les distances (ou portées) sont simplifiées en quatre catégories de distance: immédiates, courtes, longues et très longues.

    \textbf{Une distance immédiate} à partir d'un personnage se situe dans les quelques pas autour. Si un personnage se tient dans une petite pièce, tout se qui se trouve dans la pièce est à portée immédiate. Au plus, la portée immédiate est de 3m (10 pieds).

    \textbf{Une distance courte} est ce qui est au-delà de la portée immédiate mais moins que 15m (50 pieds) environ.

    \textbf{Une longue distance} est ce qui est au-delà de la courte portée mais moins que 30m (100 pieds) environ.

    \textbf{Une très longue distance} est ce qui est au-delà de la longue portée mais moins que 150m (500 pieds) environ. Au-delà de cette portée les distances sont toujours spécifique\textendash300m (1000 pieds), 1,5km (1 mille), et ainsi de suite.

    L'idée est qu'il n'est pas nécessaire de mesurer les distances de manière précise. La portée immédiate est là immédiatement, pratiquement à côté du personnage. Une courte portée est proche. Une longue portée est loin et une très longue portée est vraiment loin.

    Toutes les armes et les capacités spéciales utilisent ces termes pour les portées. par exemple, toutes les armées de mêlée ont une portée immédiate\textendash ce sont des armes de close-combat, et vous pouvez les utiliser pour attaque n'importe qui dans une portée immédiate. Un couteau de lancer (et la plupart des armes lancées) ont une portée courte. Un arc a une portée longue. La capacité d'Adepte Assaut Magique a aussi une courte portée.

    Un personnage peut se déplacer dans une portée immédiate dans le cadre d'une autre action. En d'autres mots, il peut faire quelques pas jusqu'au panneau de contrôle et activer un levier. Il peut se fendre dans une petite pièce pour attaquer un ennemi. Il peut ouvrir une porte et traverser.

    Un personnage peut se déplacer sur une courte portée en tant qu'action pendant un tour. Il peut aussi essayer de se déplacer sur une longue portée pendant toute leur action, mais le joueur devra faire un jet de dé pour voir si le personnage glisse ou trébuche par conséquence de déplacer aussi vite.

    Par exemple, si les PJs combattent un groupe de cultistes, chaque personnage peut attaquer n'importe quel cultiste dans une mêlée classique\textendash ils sont tous dans une portée immédiate. Les positions exactes ne sont pas importantes. Les créatures dans un combat sont toujours en train de se déplacer, changer de position ou de se bousculer. Toutefois, si un cultiste se met en retrait pour tirer au pistolet un personnage devra peut-être utiliser toute une action pour se déplacer sur une courte portée pour l'attaquer. Cela n'a pas d'importance que le culstiste soir à 6m ou 12m\textendash cela est simplement considéré comme une courte portée. Cela aura de l'importance si le cultiste se trouve à plus de 15m car à cette distance cela nécessitera un long ou un très long déplacement.

    \begin{figure*}[b]
        \centering
        \includegraphics[height=10cm,center]{CS-page11-01.png}
    \end{figure*}

    \section*{POINTS D'EXPÉRIENCE}
    Les Points d'Expérience (XP) sont des récompenses données aux joueurs quand la Meneuse s'immisce dans l'histoire (ceci est appelé une intrusion de la Meneuse) avec un défi nouveau et inattendu. Par exemple, au milieu d'un combat, la Meneuse peut informer le joueur que son arme lui échappe des mains. Toutefois, pour s'immiscer de cette façon, la Meneuse doit récompenser le joueur avec 2 points d'XP. Le joueur qui obtient cette récompense, doit, à son tour, donner immédiatement un de ces deux points de XP à un autre joueur et justifier ce don (peut-être que le joueur a eu une bonne idée, a fait une bonne blague, a fait une action qui a sauvé une vie, etc).

    Toutefois, le joueur peut refuser l'intrusion de MJ. Dans ce cas, il n'obtient pas les 2 points de XP de la Meneuse, et il doit dépenser un point de XP de leur propre réserve de XP. Si le joueur n'a pas de XP à dépenser, il ne peut pas refuser l'intrusion.

    La Meneuse peut aussi donner aux joueurs des XP entre les sessions en tant que récompenses pour avoir fait des découvertes pendant l'aventure. Les découvertes sont des évènements intéressants, des secrets merveilleux, des artefacts puissants, des réponses à des mystères, ou des solutions à des problèmes (tels que où les kidnappeurs détiennent leur victime ou comment les PJs ont réparé leur vaisseau). Vous ne méritez pas de XP pour avoir tuer des adversaires ou l'accomplissement de défis standards au court du jeu. La découverte est l'âme du Cypher System.

    Les points d'expérience sont utilisés principalement pour l'avancement du personnage (pour des détails, consultez le chapitre Chapitre 4: Créer votre Personnage), mais un joueur peut aussi dépenser 1 XP pour relancer n'importe quel jet de dé et choisir le meilleur d'entre les deux.

    \begin{table*}[t]
        \hspace*{0pt}
        \centering
         \begin{tabular}{ | m{14cm} | } 
            \hline
            \rowcolor{RedViolet!20} You don't earn XP for killing foes or overcoming standard challenges in the
            course of play. Discovery is the soul of the Cypher System. \\
            \hline
         \end{tabular}
         \label{table:hint3_2}
    \end{table*}


    \section*{CYPHERS}
    Les Cyphers sont des capacités à usage unique. Dans beaucoup de campagnes, les cyphers ne sont pas des objets physiques--ils peuvent être un sort lancé sur un personnage, une bénédiction d'un dieu, ou simplement un coup de pouce du destin qui lui donne un avantage. Dans certaines campagnes, les cyphers sont des objets physiques que les personnages peuvent transporter. Que les cyphers sont des objets physiques ou non, il font parti du personnage (comme son équipement ou une capacité spéciale) et sont des éléments que les personnages peuvent utiliser pendant le jeu. La forme que prend les cyphers physiques dépend du type de campagne. Dans un monde médiéval fantastique cela peut être des potions ou des baguettes magiques, mais pour de la science fiction cela peut être des cristaux extra-terrestres ou des prototypes.

    Les personnages vont trouver de nouveaux Cyphers fréquemment au cours du jeu, les joueurs ne devraient donc pas hésiter à utiliser les capacités de leur cyphers. Comme les cyphers sont toujours différents, les personnages auront toujours des nouveaux pouvoirs spéciaux à essayer. 


    \section*{AUTRES DÉS}
    En plus du d20, vous aurez besoin d'un d6 (un dé à six faces). Vous aurez rarement besoin de jeter un d100, que vous pourrez obetnir en jetant deux fois un d20 et en retenant le dernier chiffre du premier dé pour les dizaines, et le dernier chiffre du secod jet pour les unités. Par exemple, jeter les dés et obtenir un 17 et un 9 vous donne 79, obtenir un 3 et un 18 vous donne un 38, et obtenir un 20 et un 10 vous donne 00 (ou 100). Si vous avez un d10 (un dé à dix faces), vous pourez l'utiliser à la place du d20 pour avoir un nombre entre 1 et 100.

    \begin{figure*}[h]
        \centering
        \includegraphics[height=10cm,center]{CS-page12-01.png}
    \end{figure*}

%#######################################################################
\part*{\uppercase{Personnages}}
%____________________________________________________________________---------------
% https://distrib-coffee.ipsl.jussieu.fr/pub/mirrors/ctan/macros/latex/contrib/titlesec/titlesec.pdf
% https://borntocode.fr/latex-personnaliser-les-titres-chapter/
%% define format for chapter for part 2
\titleformat{\chapter} % command
    [display] % shape
    {\bfseries\Large\centering} % format
    {\textcolor{BlueViolet}{Chapitre \ \thechapter \\ \huge #1} \\ \afficheimagechapitre{\thechapter}} % label
    {0.5ex} % sep
    {
        \centering
    } % before-code
    [
    \centering
    ] % after-code
    \titlespacing*{\chapter}
    {0mm} %left
    {-0pt} %before-step
    {0mm} %after-step
    {} %right
%____________________________________________________________________---------------
%% define format for section
\titleformat{\section} % Command to format section titles
{\normalfont\Large\bfseries} % Format: normal font, large size, bold
{\thesection}{1em} % Label format: section number followed by 1em space
{} % Before the title
[\titlerule] % After the title: horizontal line

\titleformat{\section} % command
    [display] % shape
    {\bfseries\Large} % format
    {\textcolor{BlueViolet}{#1}} % label
    {0.5ex} % sep
    {
        \Large
    } % before-code

    \titlespacing*{\section}
    {0mm} %left
    {-0pt} %before-step
    {0mm} %after-step
    {} %right
%____________________________________________________________________---------------
% numberless
\titleformat{name=\section,numberless}[runin] 
    {\normalfont\Large\bfseries}
    {\\ \textcolor{BlueViolet} {#1}}
    {20pt}
    {\Large}
    [\\]
%____________________________________________________________________---------------
%% define format for subsection
    \titleformat{\subsection} % command
    [display] % shape
    {\bfseries\large} % format
    {\textcolor{BlueViolet}{#1}} % label
    {0.5ex} % sep
    {
        \large
    } % before-code

    \titlespacing*{\subsection}
    {0mm} %left
    {-0pt} %before-step
    {0mm} %after-step
    {} %right
%____________________________________________________________________---------------
% numberless
\titleformat{name=\subsection,numberless}[runin] 
    {\normalfont\large\bfseries}
    {\\ \textcolor{BlueViolet} {#1}}
    {20pt}
    {\large}
    [\\]
%#######################################################################
%            CHAPTER 4
%#######################################################################
\chapter{\uppercase{Créer Votre Personnage}}\label{ch:chapter4}
\fancypagestyle{plain}{ %
\fancyhf{} % remove everything
\renewcommand{\headrulewidth}{1pt} % remove lines as well
\renewcommand{\footrulewidth}{0pt}
\xpretocmd\headrule{\color{BlueViolet}}{}{\PatchFailed}
\fancyhead[RO]{\textcolor{BlueViolet}{\textsc{Créer Votre Personnage}}}
\fancyhead[LE]{\textcolor{BlueViolet}{CYPHER SYSTEM}}
}

Ce chapitre vous détaille comment créer des personnages pour jouer à une partie de jeux de rôle basée sur le Cypher System. Cela nécessite un ensemble de décisions qui vont donner corps à votre personnage, de façon à ce que plus vous comprennez le type de personnage vous avez envie de jouer et plus la création du personnage sera simple. Le processus implique de bien comprendre les principes des trois statistiques (Puissance, Célérité et Intellect) et de choisir trois aspects (Réserve, Avantage et Effort) qui vont déterminer les possibilités de votre personnage.

\section*{Les Statistiques du Personnage}

Chaque personnage joueur a trois caractéristiques bien définies qui sont appelées "statistiques" ou "stats". Ces stats sont Puissance, Célérité et Intellect. Ce sont des catégories assez larges qui couvrent des aspects différents d'un personnage.

\subsection*{Puissance}

La puissance définit à quel point votre personnage est fort et résistant. Les concepts de force, endurance, constitution, résistance et les prouesses physiques sont englobées dans cette stat. La Puissance n'est pas relative à la taille; c'est plutôt une mesure absolue. Un éléphant a plus de Puissance que le plus puissant des tigres, qui a plus de Puissance que le plus puissant des rats, qui a plus de Puissance que la plus puissante des araignées.

La Puissance gouverne les actions comme de force une porte ou de marcher pendant des jours sans manger ou de résister à une maladie. C'est aussi le moyen principal pour déterminer combien de dommages votre personnage peut supporter dans une situation dangereuse. Les personnages physiques, endurants ou plutôt orientés dans le combat devraient se concentrer sur la Puissance.

La Puissance peut être pensée comme Puissance/Santé car elle contrôle à quel point vous être fort et à quel niveau de chocs physiques vous pouvez supporter.

\subsection*{Célérité}

La Célérité décrit à quel point votre personnage est rapide et bien coordonné. La stat inclut l'agilité, la vitesse, le mouvement, la dextérité et les reflexes. La Célérité contrôle les actions telles qu'éviter les attaques, se dissimuler et s'infiltrer, lancer une balle avec précision. Cette stat permet de déterminer si vous pouvez vous déplacer rapidement dans votre tour. Les personnages agiles, rapides, ou les as de la dissimulation, tout comme ceux qui souhaitent exceller en tir à distance, devraient avoir une bonne Célérité.

La Célérité peut être pensée comme de la Vitesse ou de l'Agilité car elle contrôle votre célérité et vos reflexes.

\subsection*{Intellect}

Cette stat détermine à quel point votre personnage peut être malin, éduqué et apprécié. Cela inclut l'intelligence, la sagesse, le charisme, l'éducation, le raisonnement, l'esprit, la volonté et le charme. L'intelligence contrôle la résolution d'énigmes, se souvenir de faits, énoncer des mensonges convaincants, et utiliser des pouvoirs mentaux. Les personnages interressés dans la communication, dans l'érudition, ou dans le contrôle de pouvoirs sunaturels devraient avoir une bonne stat d'Intellect.

\section*{Réserve, Avantage, et Effort}

Chacune des trois stats a deux composantes: la Réserve et l'Avantage. Votre Réserve représente votre faculté pure, innée, et votre Avantage représente à quel point vous savez utiliser ce que vous avez. Un troisième composant est lié à ce concept: l'Effort. Quand votre personnage a vraiment besoin d'accomplir une tâche, vous pouvez appliquer de l'Effort.

\subsection*{Réserve}

Votre Réserve est la mesure de base d'une stat. Comparer les Réserves de deux créatures vous donnera un bon apperçu de quelle créature est supérieure dans cette stat. Par exemple, un personnage qui a une Réserve de Puissance de 16 est plus costaud qu'un personnage qui a une Réserve de Puissance de 12. La plupart des personnages commencent avec une Réserve entre 9 et 12 dans la plupart des stats—c'est la fourchette moyenne.

Quand votre personnage est blessé, malade ou attaqué, vous perdez temporairement des points de l'une de vos Réserve de stat. La nature de l'attaque détermine Réserve perd des points. Par exemple, un dommage physique par une épée réduit votre Réserve de Puissance, un poison qui vous rend maladroit réduit votre Réserve de Célérité, et une attaque psionique réduit votre Réserve d'Intellect . Vous pouvez aussi dépensser des points de l'une de vos Réserve pour diminuer la difficulté d'une tâche (voir Effort ci-dessous). Vous pouvez vous reposer pour récupérer des points perdus dans une Réserve de stat, et certaines capacités spéciales ou cyphers peuvent vous permettre de récupérer des points perdus rapidement.

\subsection*{Avantage}

Bien que votre Réserve est la mesure de base d'une stat, votre Avantage est aussi important. Quand quelque chose vous demande de dépenser des points d'une de vos Réserves, votre Avantage associé à la stat réduit ce coût. Cela réduit aussi le coût pour appliquer de l'Effort à un jet de dé.

Par exemple, disons que vous la capacité d'attaque mentale et que son activation coûte 1 point de votre Réserve d'Intellect. Soustrayez votre Avantage d'Intellect du coût d'activation et le résultat représente le nombre de points que vous devez dépenser pour utiliser d'attaque mentale. Si l'utilisation de votre Avantage réduit le coût à zéro, vous pouvez utiliser la capacité gratuitement.

Votre Avantage peut être différent pour chaque stat. Par exemple, vous pourriez avoir un Avantage de Puissance de 1, un Avantage de Célérité de 1, et un Avantage d'Intellect de 0. Vous aurez toujours un avantage d'au moins 1 dans une des stats. Votre Avantage dans une stat réduit le coût de la dépense de points dans cette Réserve, mais pas des autres Réserves. Votre Avantage de Puissance réduit les coûts dépensés dans votre Réserve de Puissance, mais cela n'affecte pas votre Réserve de Célérité ou d'Intellect. Une fois que l'Avantage pour une stat atteint 3, vous pouvez appliquer un niveau d'Effort gratuitement.

Un personnage qui a une Réserve de Puissance faible mais un fort Avantage de Puissance a le potentiel d'accomplir des actions de Puissance de manière plus régulière qu'un personnage qui a une Réserve de Puissance de 0. Un Avantage élevé permet de réduire le coût des dépenses de points de la Réserve associée, ce qui implique qu'il y a plus de points disponibles pour appliquer de l'Effort.

\subsection*{Effort}

Quand votre personnage a vraiment besoin d'accomplir une tâche, vous pouvez appliquer de l'Effort. Pour un personnage débutant, appliquer de l'Effort nécessite de dépenser 3 points de la Réserve de la stat appropriée pour l'action. Ainsi, si votre personnage essaie d'éviter une attaque (un jet de Célérité) et veut augmenter ses chances de succès, vous pouvez appliquer de l'Effort en dépensant 3 points de votre Réserve de Célérité. L'Effort facilite une tâche d'un cran. On appelle cela appliquer un niveau d'Effort.

Vous n'êtes pas obligé d'appliquer de l'Effort si vous ne voulez pas. Si vous choisissez d'appliquer de l'Effort pour une tâche, vous devez le faire avant de jeter le dé—vous ne pouvez pas jeter le dé et décider d'appliquer de l'Effort si votre jet de dé n'est pas bon.

Appliquer de l'Effort peut diminuer la difficulté d'une tâche encore plus: chaque niveau d'Effort facilite la tâche d'un cran supplémentaire. Appliquer un niveau d'Effort facilite la tâche d'un cran, appliquer deux niveaux facilite la tâche de deux crans, et ainsi de suite. Toutefois, chaque niveau d'Effort après le premier ne coûte que 2 points de la Réserve de Stat au lieu de 3. Ainsi, appliquer deux niveaux d'Effort coûte 5 points (3 pour le premier niveau plus 2 pour le second niveau), appliquer trois niveaux coûte 7 points (3 plus 2 plus 2), et ainsi de suite.

Chaque personnage a un score d'Effort, qui indique le nombrede niveau d'Effort maximum qui peut appliquer sur un jet de dé. Un personnage débutant (premier rang) a un Effort de 1, ce qui signifie que vous ne pouvez qu'un seul niveau d'Effort à un jet de dé. Un personnage plus expérimenté a un score d'Effort plus grand et peut appliquer plus de niveaux d'Effort à un jet de dé. Par exemple, un personnage qui a un score d'Effort de 3 peut appliquer jusqu'à 3 niveaux d'Effort pour réduire la difficulté d'une tâche.

Quand vous appliquez de l'Effort, soustrayez votre Avantage associé au total du coût. Par exemple, disons que vous avez besoin de faire un jet de RApidité. Pour augmenter votre chance de succès, vous décidez d'appliquer un niveau d'Effort, ce qui va faciliter la tâche. Normalement, cela vous coûterait 3 points de votre R2serve de Célérité. Toutefois vous avez un Avatange de Célérité de 2, que vous soustrayez au coût de l'Effort. Ainsi, l'application de l'Effort pour le jet de dé ne coûte qu'un point de votre Réserve de Célérité.

Que se passe-t-il si vous appliquez deux niveaux d'Effort à un jet de Célérité au lieu d'un seul ? Cela faciliterait la tâche de deux crans. Normalement, cela couterait 5 point de votre Réserve de Célérité, mais après soustraction de votre Avantage de Célérité de 2, cela ne coute finalement que 3 points.

Une fois que l'Avantage d'une stat atteint 3, vous pouvez appliquer un niveau d'Effort gratuitement. Par exemple, si cous avez un Avantage de RApidité de 3 et que vous souhaitez appliquer un niveau d'Effort à un jet de Célérité, cela vous coute 0 point de votre Réserve de Célérité. (En situation normale, appliquer un niveau d'Effort vous coute 3 points, mais comme vous soustrayez votre Avantage de Célérité, cela le réduit à 0.)

Les compétences et autres avantages permettent aussi de faciliter une tâche, et vous pouvez les utiliser en conjonction de l'Effort. De plus, votre personnage peut avoir des capacités spéciales ou de l'équipement qui vous permet d'appliquer de l'Effort pour accomplir un effet spécial, tel que mettre un adversaire à terre ou affecter plusieurs cibles avec un pouvoir qui n'en affecte qu'un seul normalement.

\section*{Effort et Dommages}

Au lieu d'appliquer de l'Effort pour faciliter votre attaque, vous pouvez appliquer de l'Effort pour augmenter les dommages infligés par l'attaque. Pour chaque niveau d'Effort que vous appliquez ainsi, vous infligez 3 points de dommages supplémentaires. Cela fonctionne pour tout type d'attaque qui inflige des dommages, que ce soit une épée, une arbalète, une attaque psi, ou autre.

Quand vous utilisez de l'Effort pour augmenter des dommages d'attaque de zone,, comme par l'explosion créée par la Capacité d'Adepte Concussion, vous infligez 2 points de dommages supplémentaires au lieu de 3 points. Toutefois, ces points supplémentaires sont infligés à toutes les cibles de la zone. De plus, même si une des cibles résiste à l'attaque, elle subit quand même 1 point de dommage.

\section*{Utilisation multiple de l'Effort et Avantage}

Si votre Effort est de 2 ou plus, vous pouvez appliquer de l'Effort à plusieurs aspects d'une simple action. Par exemple, si vous faites une attaque, vous pouvez appliquer de l'Effort à votre jet d'attaque et appliquer de l'Effort pour augmenter les dommages.

La somme totale de l'Effort que vous appliquez ne peut pas être supérieure à votre score d'Effort. Par exemple, si votre effort est de 2, vous pouvez appliquer jusqu'à 2 niveaux d'Effort. Vous pourriez appliquer un niveau d'Effort au jet d'attaque et un niveau d'Effort pour les dommages, deux niveaux d'Effort pour l'attaque et aucun Effort pour les dommages, ou aucun Effort pour l'attaque et deux niveaux pour les dommages.

Vous pouvez utiliser l'Avantage pour une stat particulière uniquement une fois par action. Par exemple, si vous appliquez, sur la stat de Puissance, de l'Effort pour le jet d'attaque et pour les dommages, vous pouvez utiliser votre Avantage de Puissance pour réduire le coût d'un des usage de l'Effort (jet d'attaue ou dommages), mais pas les deux. Si vous dépensez un point d'Intellect pour activer votre attaque psi et un niveau d'Effort pour faciliter le jet d'attaque, vous pouvez utiliser votre Avantage d'Intellect pour réduire l'un des deux (activation ou attaque), mais pas les deux.

\section*{Exemples de l'utilisation de Stat}

Un personnage débutant est en train de combattre un rat géant. Avec sa lance, le PJ essaie de transpercer le rat, qui est une créature de niveau 2 et qui a donc un nombre seuil de 6. Le personnage se tient sur un rocher et frappe vers le bas sur la bête, et la Meneuse décide que cette tactique est un atout qui facilité d'un cran (vers une difficulté de 1 au lieu de 2 initialement). Cela diminue le nombre seuil à 3. Attaquer avec une lance est une action de Puissance; le personnage a une Réserve de Puissance de 11 et un avantage de Puissance de 0. Avant de faire le jet de dé, le PJ décide d'appliquer un niveau d'Effort pour faciliter l'attaque. Cela coûte 3 points de sa Réserve de Puissance, la réduisant à 8. Mais les points sont bien dépensés. Appliquer l'Effort diminue la difficulté de 1 à 0, et ainsi aucun jet de dé n'est nécessaire---l'attaque réussit automatiquement.

Un autre personnage essaie de convaincre un garde de le laisser entrer dans un bureau pour parler avec un noble influent. La Meneuse décide que c'est une action d'Intellect. Le personnage est de rang 3 et a un Effort de 3, une Réserve d'Intellect de 13 et un Avantage d'Intellect de 1. Avant de jeter le dé, il décide si il applique de l'Effort. Il peut choisir d'appliquer un, deux, ou trois niveaux d'Effort, ou de ne pas en appliquer du tout. Cette action est importante pour lui, et il décide donc d'appliquer deux niveaux d'Effort, facilitant la tâche de deux crans. Grâce à son Avantage d'Intellect, appliquer l'Effort ne coûte que 4 points de sa Réserve d'Intellect (3 points pour le premier niveau d'Effort, plus 2 points pour le second niveau moins 1 point de l'Avantage). Dépenser ces pointsréduit la Réserve d'Intellect à 9. La Meneuse décide que convaincre le garde est une tâche de difficulté 3 (Exigeant) avec un nombre seuil de 9; appliquer deux niveaux d'Effort permet de réduire la difficulté à 1 (simple) et un nombre cible de 3. Le Joueur lance un d20 et obtient un 8. Comme ce résultat est supérieur ou égal au nombre seuil de la tâche, le personnage réussit. Toutefois, si le PJ n'avait pas appliquer d'Effort, il aurait échoué car le résultat du dé (8) aurait été inférieur au nombre seuil initial (9) de la tâche.

\begin{figure*}[t]
    \includegraphics[height=10cm,center]{CS-page17-01.png}
\end{figure*}


\section*{Rangs de Personnage}

Chaque personnage commence le jeu au premier rang. Le rang est une mesure du pouvoir, de résistance et de capacité. Les personnages peuvent avancer jusqu'au rang six. Alors que votre personnage passe au rang supérieur, il gagne de nouvelles capacités, augmente son Effort et peut améliorer un Avantage de stat ou augmenter une stat. En général, même un personnage de rang 1 est déjà assez compétent. Il est facile d'imaginer qu'il a déjà vécu pas mal d'expériences. Ce n'est pas une progression "de zéro à héro", mais plutôt une personne compétente qui affine et polit ses capacités et ses connaissances. Progresser vers un rang supérieur n'est pas vraiment l'objectif des personnages du Cypher System, mais plutôt une représentation de comment les personnages progressent dans l'histoire.

Pour avancer au rang suivant, les personnages gagne des points d'expérience (XP) en poursuivant les arcs de personnage, allant en aventure et en découvrant de nouvelles choses---le système est vraiment basé sur la découverte et l'exploration, tout autant que l'accomplissement d'objectifs personnels. Les points d'expérience ont plusieurs usages, dont un est d'acquérir des bénéfices au personnage. Une fois que votre personnage a acquis quatre bénéfices de personnage, il avance au rang suivant. Chaque bénéfice coûte 4 XP, et vous pouvez les acquérir ans n'importe quel ordre, mais vous devez acquérir un de chaque type de bénéfice (et ainsi avancer vers le rang suivant) avant de pouvoir acquérir le même bénéfice à nouveau. Les quatre bénéfices sont les suivants.


\textbf{Augmenter vos Possibilités:} Vous gagnez 4 points à ajouter à vos Réserves de stat. Vous pouvez allouer ces points parmi les Réserves comme vous l'entendez.

\textbf{Avancer vers la Perfection:} Vous ajoutez 1 à votre Avantage, votre Avantage de Célérité ou à votre Avantage d'Intellect (à vous de choisir).

\textbf{Extra Effort:} Votre score d' Effort augmente de 1.

\textbf{Compétences:} Vous devenez entraîné dans une compétence de votre choix, autre que pour l'attaque ou la défense. Comme décrit dans le Chapitre 11: Règles du Jeu, un personnage entraîné dans une compétence traite la difficulté d'une tâche avec un cran de moins que la normale. La compétence que vous choisissez pour ce bénéfice peut être ce que vous voulez, comme escalade, saut, persuasion, ou dissimulation. Vous pouvez aussi choisir d'avoir plus de savoir dans un type de connaissance, telle que l'histoire ou la géologie. Vous pouvez même choisir une compétence basée sur des capacités spéciales de votre personnage. Par exemple, si votre personnage peut faire un jet d'Intellect pour attaquer un ennemi avec sa force mentale, vous pouvez devenir entraîné à utiliser cette capacité, facilitant la tâche de son utilisation. Si vous choisissez une compétence pour laquelle vous êtes déjà entraîné, vous devenez spécialisé dans cette compétence, facilitant les tâches de deux crans au lieu d'un seul.

\textbf{Autres Options:} Les joueurs peuvent aussi dépenser 4 XP pour acheter d'autres options spéciales à la place d'acquérir une nouvelle compétence. Sélectionner une de ces options compte comme un bénéfice de compétence pour avancer vers le rang suivant. Les options spéciales sont les suivantes:
• Réduire le coût de porter une armure. Cette option diminue le coût de Célérité pour porter une armure de 1.
• Ajouter 2 à votre jets de récupération.
• Sélectionner une nouvelle capacité de votre type de votre rang ou d'un rang inférieur.

\begin{figure*}[b]
    \includegraphics[height=10cm]{CS-page18-01.png}
\end{figure*}

\section*{DESCRIPTEUR DE PERSONNAGE, TYPE ET FOCUS}
Pour créer votre personnage, vous élaborez une phrase simple qui le décrit. Cette phrase, ou proposition, prend la forme suivante:
"Je suis un [mettre un nom ici] [mettre un adjectif ici] qui [mettre un groupe verbal ici]."
Ainsi: "Je suis un adjectif nom qui verbe-complément". Par exemple, vous pourriez énoncer, "Je suis un Robuste Guerrier qui Contrôle les Bêtes Sauvages" ou "Je suis un Séduisant Explorateur qui Concentre l'Esprit sur la Matière".
Dans cette phrase, l'adjectif est appelé votre descripteur.

Le nom est le type de votre personnage.

Le groupe verbal est appelé votre focus.

Même si le type du personnage est au milieu de la phrase, c'est par lui que nous commencerons cette discussion.

Le type de votre personnage est son essence. Dans certains jeux de rôle il peut être appelé classe de personnage. Votre type aide à déterminer la place de personnage dans le monde et la relation avec les autres dans la campagne. C'est le nom dans la phrase "Je suis un adjectif nom qui groupe verbal".

Vous pouvez choisir parmi quatre types de personnage: Guerriers, Adeptes, Explorateurs, Émissaires.

Votre descripteur définit votre personnage---il donne une coloration à tout ce qu'il fait. Votre descripteur met votre personnage en situation (la première aventure qui démarre la campagne) et aide à fournir une motivation. C'est l'adjectif dans la phrase "Je suis un adjectif nom qui groupe verbal".

A moins que la Meneuse ne dise le contraire, vous pouvez choisir n'importe quel descripteur de personnage.

Le Focus est ce que votre personnage fait de mieux. Le Focus fournit la spécificité de votre personnage et donne de nouvelles capacités qui pourraient s'avérer utiles. Votre Focus vous aide aussi à comprendre comment vous associer avec d'autres PJ de votre groupe. C'est le groupe verbal dans la phrase "Je suis un adjectif nom qui groupe verbal".

Il y a plusieurs focus de personnage. Celui que vous choisirez dépend probablement de la campagne et du genre de la partie.

\section*{\uppercase{Capacités Spéciales}}
Les types et focus des personnages donnent aux PJs des capacités spéciales à chaque nouveau rang. L'utilisation de ces capacités coûte en général des points de vos Réserves de stat; ce coût est indiqué entre parenthèses après le nom de la capacité. Votre Avantage dans la stat associée peut réduire ce coût de la capacité, mais souvenez-vous que vous ne pouvez appliquer l'Avantage qu'une fois par action. Par exemple, disons qu'un Adepte avec un Avantage d'Intellect de 2 souhaite  utiliser sa capacité d'Assaut Magique pour créer un éclair de force, ce qui coûte 1 point d'Intellect. Il voudrait aussi augmenter les dommages de l'attaque en utilisant un niveau d'Effort, ce qui coûte 3 points d'Intellect. Le coût total de l'action est de 2 points de Réserve d'Intellect (1 point pour l'éclair de force, plus 3 points pour utiliser l'Effort, moins 2 points de l'Avantage).

Dans certains cas, le coût pour une capacité a un signe + après le nombre. Par exemple, le coût peut être donné comme "2+ Points d'Intellect". Cela signifie que vous pouvez dépenser plus de points ou plus de niveaux d'Effort pour améliorer la capacité, tel que détaillé dans la description de la capacité.
Plusieurs capacités spéciales confèrent au personnage l'option d'accomplir une action qu'il ne pourrait pas faire normalement, telle que projeter des éclairs de givre ou attaquer plusieurs adversaires à la fois. Utiliser une de ces capacités est une action à part entière, et à la fin de la description il est indiqué "Action" pour vous le rappeler. Cette description peut aussi vous fournir plus d'information sur quand et comment vous pouvez accomplir cette action.
Certaines capacités spéciales vous permettent d'accomplir une action familière---une action que vous pourriez déjà faire---sous une forme différente. Par exemple, une capacité peut vous laisser porter une armure lourde, réduire la difficulté d'un jet de défense de Célérité, ou ajouter 2 points aux dommages de feu aux dommages de votre arme. Ces capacités sont de la catégorie des Facilitateurs. Utiliser l'une de ces capacités n'est pas considéré comme une action. Les facilitateurs fonctionnent, soit de manière constante (comme de porter une armure lourde, qui n'est pas une action), soit se produisent au cours d'une autre action (telle que ajouter des dommages de feu aux dommages de votre arme, qui va se produire pendant votre action d'attaque). Si une capacité spéciale est un facilitateur, il est indiqué "Facilitateur" à la fin de sa description.

Certaines capacités spécifient une durée, mais vous pouvez interrompre l'une de vos propre capacité quand vous le souhaitez.

\begin{table*}[b]
    \hspace*{-20mm}
    \centering
     \begin{tabular}{ | m{17cm} | } 
        \hline
        \rowcolor{SkyBlue!50}
        \Large SKILLS \\
        \rowcolor{SkyBlue!50}
        \begin{multicols}{2}
            Quelque fois votre personnage va acquérir un entraînement dans une compétence ou une tâche spécifique. Par exemple, votre focus peut signifier que vous avez été entraîné dans la dissimulation, dans l'escalade et le saut, ou dans les interactions sociales. A d'autres moments, votre personnage pourra choisir une compétence pour en devenir entraîné, et vous pourrez sélectionner une compétence qui s'associe à toute tâche à laquelle vous pensez que vous aurez à accomplir.

            Le Cypher System n'a pas de liste définitive de compétences. Toutefois, la liste ci-après donne des idées:
            \begin{tabular}{ l | l | l } 
                \rowcolor{SkyBlue!50}
                Acrobatie & Géographie & Physique \\
                Astronomie & Géologie & Pickpocket \\
                Biologie & Histoire & Pilotage \\
                Botanique & Identifier & Porter \\
                Chevaucher & Initiative & Réparer \\
                Conduire un véhicule & Intimidation & S'infiltrer \\
                Crochetage & Machiniste & Sauter \\
                Déguisement & Nager & Soigner \\
                Discrétion & Ordinateur & Travail du bois \\
                Dissimulation & Perception & Travail du cuir \\
                Escalade & Persuasion & Travail du métal \\
                Fracasser & Philosophie & Tromper \\
            \end{tabular}
            Vous pouvez choisir une compétence qui incorpore plus d'une de ces définitions ci-dessus (interagir peut inclure tromper, intimidation et persuasion) ou cela peut être une version plus spécifique (se cacher peut être dissimulation quand vous ne bougez pas). Vous pouvez aussi avoir des compétences plus générales liées à une profession, telle que boulanger, marin, ou bûcheron. Si vous voulez sélectionner une compétence qui n'est pas dans cette liste, il est probablement sage de consulter d'abord la Meneuse, mais en général, la chose la plus importante est de choisir des compétences qui soient appropriées à votre personnage.
    
            Souvenez-vous que si vous gagnez une compétence pour laquelle vous êtes déjà entraîné, vous devenez spécialisé dans cette compétence. Comme les descriptions des compétences sont assez vagues, déterminer si vous êtes entraîné ou spécialisé peut demander un peu de réflexion. Par exemple, si vous êtes entraîné à mentir et que plus tard vous gagnez une capacité qui vous donne un entraînement de compétence dans toutes les interactions sociales, alors vous devenez spécialisé dans les mensonges et entraîné dans les autres formes d'interaction. Être entraîné trois fois dans une compétence n'apporte rien de plus que de l'être deux fois (en d'autres termes, spécialisé est le maximum que l'on puisse atteindre dans une compétence).
            Seules les compétences acquises à partir des capacités du type de personnage, ou quelques rares autres cas, vous permettent de devenir entraîné dans les tâches d'attaque ou de défense.
            
            Si vous gagnez une capacité spéciale à partir de votre type, focus ou un autre aspect de votre personnage, vous pouvez la sélectionner à la place d'une compétence et devenir entraîné ou spécialisé dans cette capacité. Par exemple, si vous avez une attaque psi, quand vient le moment de choisir une compétence dans laquelle vous seriez entraîné, vous pouvez sélectionner vote attaque psi en tant que compétence. Cela facilitera l'attaque à chaque fois que vous l'utiliserez. Chaque capacité que vous avez compte comme une compétence unique dans ce cas d'usage. Vous ne pouvez pas sélectionner "tous les pouvoirs mentaux" en tant que seule compétence et devenir entraîné ou spécialisé pour une catégorie aussi large.
            Dans la plupart des campagnes, parler couramment une langue est considéré comme une compétence. Donc si vous voulez parler Espagnol, c'est la même chose qu'être entraîné en biologie ou à nager.
        \end{multicols}
  \\
        \hline
     \end{tabular}
     \label{table:skills}
\end{table*}



%#######################################################################
%            CHAPTER 5
%#######################################################################
\chapter{\uppercase{Type}\label{ch:chapter5}}
\fancypagestyle{plain}{ %
\fancyhf{} % remove everything
\renewcommand{\headrulewidth}{1pt} % remove lines as well
\renewcommand{\footrulewidth}{0pt}
\xpretocmd\headrule{\color{BlueViolet}}{}{\PatchFailed}
\fancyhead[RO]{\textcolor{BlueViolet}{Type}}
\fancyhead[LE]{\textcolor{BlueViolet}{CYPHER SYSTEM}}
}

Le Type de personnage est le coeur de votre personnage. Son type vous aide à déterminer sa place dans le monde et les relations avec les autres dans la campagne. C'est le "nom" dans la phrase "Je suis un/e textit{nom+adjectif} qui textit{proposition}"

Vous pouvez choisir parmi quatre types de personnage: Guerrier, Adepte, Explorateur, et Emissaire. Toutefois, vous pourriez ne pas vouloir utiliser ces termes génériques. Ce chapitre vous propose, pour chaque type, quelques alternatives de noms qui pourraient être plus adapté à un genre particulier. Vous trouverez peut-être que des noms comme "Guerrier" ou "Explorateur" ne sonnent pas juste, en particulier dans des campagnes se déroulant à une époque contemporaine. Comme toujours, vous êtes libre de faire comme vous voulez.

Comme le type est la base sur laquelle votre personnage est bati, il est important de considérer comment le type se tient avec la campagne sélectionnée. Pour se faire, les types sont en pratique des archetypes. Un Guerrier, par exemple, pourrait être n'importe qui du chevalier en armure étincellante au policier dans la rue ou au baroudeur cybernétique vétéran de milliers de guerres futuristes.

Pour faciliter l'usage des quatre types dans les diverses campagnes, différentes méthodes appelées "préférences" sont présentées dans le Chapitre 6: Préférence piur aider à personnaliser les différents types pour de la fantasy, de la science fiction, ou d'autres genres (ou pour s'ajuster à différents concepts de personnage).

Au final, des options plus fondamentales de personnalisation sont fournies à la fin de ce chapitre.

\section*{Guerrier}

\begin{description}
    \item[Fantasy/Contes:] guerrier, combattant, escrimeur, chevalier, barbare, soldat, myrmidon, valkyrie
    \item[Moderne/Horreur/Romance:] officier de police, soldat, gardien, détective, vigile, athlète
    \item[Science fiction:] officier de sécurité, guerrier, homme de troupe, soldat, mercenaire
    \item[Superhéro/Post-Apocalyptique:] héro, brique, cogneur
\end{description}

Vous êtes un bon allié à avoir dans un combat. Vous savez comment utiliser des armes et vous défendre.En fonction du genre et de la campagne, cela pourrait signifier de porter une épée et un bouclier dans une arêne de gladiateurs, un AK-47 et des grenades en bandoulière dans la jungle, ou un fusil blaster et une armure mécanique dans l'exploration d'une planète lointaine.

\begin{description}
    \item[Rôle Individuel:] Les Guerriers sont orienté sur le physique et l'action. ils auront plus l'habitude de surmonter un péril en utilisant la force que d'autre moyen, et ils prennent souvent le chemin le plus court vers leur objectif.
    \item[Rôle dans un Groupe:] Les guerriers subissent et infligent généralement le plus de dégâts dans une situation dangereuse. Il leur incombe souvent de protéger les autres membres du groupe contre les menaces. Cela signifie parfois que les guerriers assument également des rôles de commandement, du moins au combat et dans d'autres moments de danger.
    \item[Rôle en société:] Les Guerriers ne sont pas toujours des soldats ou des mercenaires. N'importe qui est est toujours prêt pour la violence, ou même la violence potentielle, peut être un Guerrier dans un sens général. Cela inclut les gardes, les gardiens, les officiers de police, les marins, ou les personnes dans d'autres rôle ou profession qui savent comment se défendre avec talent.
    \item[Guerier Expérimentés:] Alors que les guerriers progressent, leur compétence en combat—que ce soit en se défendant ou en infligeant des dommages—augmente à un rang impressionant. A un rang supérieur, ils peuvent souvent se prendre un groupe d'aversaires tout seul ou affronter sur son terrain n'importe qui.
\end{description}

\begin{table}[]
    \begin{tabular}{ | l | }
        \rowcolor{SkyBlue!50}
        \textcolor{BlueViolet}{\Large \uppercase{Intrusion de Joueur}} \\
        \rowcolor{SkyBlue!50}
        Une intrusion de joueur est quand un joueur choisi d'altérer quelque chose dans la campagne, rendant les choses plus facile pour le PJ. De manièr conceptuelle, c'est l'inverse d'un intrusion de MJ qui donne au joueur des points d'expérience (XP) en introduisant une complication inatendue pour le personnage. Dans le cas du joueur, ce dernier dépense un 1 XP et présente une solution à un problème ou une complication. Ce que l'intrusion du joueur peut faire est en général introduire un changement du monde ou des circonstances en cours, plutôt que de changer directement le personnage. Par exemple, une intrusion qui propose que le cypher qui vient d'être utilisé a une charge supplémentaire, est appropriée, mais une intrusion qui propose que le personnage est soigné ne l'est pas. Si le joueur n'a pas d'XP à dépenser, il ne peut pas utiliser d'intrusion de joueur.

        Quelques exemples d'intrusion de joueur sont proposées pour chaque type. Cela dit, toutes les intrusions de joueur listées ici ne sont pas appropriées à chaque situation. Le MJ peut autoriser les joueurs à proposer d'autres suggestions d'intrusion de joueur, mais la Meneuse a le dernier mot pour savoir si une intrusion est appropriée en fonction du type de personnage et de la situation. Si la Meneuse refuse l'intrusion, le joueur ne dépense pas de XP, et l'intrusion n'a simplement pas lieu.

        Utiliser une intrusion ne requiert pas du personnage d'utiliser une action pour l'activer. Une intrusion de joueur survient tout simplement. \\
        \hline
    \end{tabular}
\end{table}

\section*{Intrusions de Joueur pour un Guerrier}

Vous pouvez dépenser 1 XP pour utiliser une des intrusions de joueur suivantes, à condition que la situation est appropriée et que la Meneuse soit d'accord.
\begin{description}
    \item[Position Parfaite:] Vous combattez au moins trois adversaires et chacun d'eux se trouve exactement à la bonne position pour vous pour faire un mouvement pour lequel vous vous êtes entrainé il y a longtemps, vous permettant de les attaquer tous les trois en une seule action. Faites un jet d'attaque pour chaque adversaire. Vous restez limité par la quantité d'Effort que vous pouvez allouer en une seule action.
    \item[Vieil Ami:] Un ancien companion d'arme se présente de manière spontanée et vous fourni de l'aide dans ce que vous êtes en train de faire. Il doit accomplir sa propre mission et ne peut pas rester plus longtemps que pour vous aider, parler un peu après et peut-être partager un repas rapide.
    \item[Arme cassée:] L'arme de votre adversaire a un point faible. Pendant le combat, l'arme est endommagée et descend de deux rangs sur le suivi des dommages des objets.
\end{description}

\section*{Réserves de Stat pour un Guerrier}
\begin{table}[h]
    \begin{tabular}{ l c }
        \textbf{Stat} & \textbf{Valeur de Réserve au Démarrage} \\
        \rowcolor{SkyBlue!50}
        Puissance & 10 \\
        \rowcolor{SkyBlue!20}
        Célérité & 10 \\
        \rowcolor{SkyBlue!50}
        Intellect & 8 \\
    \end{tabular}
\end{table}

Vous avez 6 points supplémentaires à répartir parmi vos Réserves de stat comme vous le souhaitez.

\section*{Guerrier de Premier Rang}

Les Guerriers de Premier Rang ont les capacités suivantes:
\begin{description}
    \item[Effort:] Votre Effort est de 1.
    \item[Physical Nature:] Vous avez un Avantage de Puissance de 1 et un Avantage de Célérité de 0, ou vous avez un Avantage de Puissance de 0 et un Avantage de Célérité de 1. Dans tous les cas, vous avez un Avantage d'Intellect de 0.
    \item[Cypher Use:] Vous pouvez porter deux cyphers en même temps.
    \item[Armes:] Vous avez la pratique des armes légères, moyennes et lourdes et n'avez aucune pénalité quand vous utilisez une arme quelconque.
    \item[Equipment au départ:] Des vêtements appropriés et deux armes de votre choix, ainsi que un objet cher, deux objets modérement chers, et jusqu'à quatre objets peu chers.
    \item[Capacités Spéciales:] Choisissez quatre capacités listées ci-cessous. Vous ne pouvez pas choisir la même capacité plus d'une fois, à moins que sa description dit le contraire. La description complète de chaque capacité listée se trouve dans le chapitre Capacités, qui dispose aussi des descriptions pour les préférences et les capacités de focus en un seul grand catalogue.
\end{description}

\begin{abnamelist}
    \item Avantage de Stat Amélioré (\pageref{subsec:ab_improved_edge})
    \item Choc (\pageref{subsec:ab_bash})
    \item Claque (\pageref{subsec:ab_swipe})
    \item Compétences physiques (\pageref{subsec:ab_physical_skills})
    \item Contrôler le terrain (\pageref{subsec:ab_control_the_field})
    \item Entraîné sans armure (\pageref{subsec:ab_trained_without_armor})
    \item Lancer rapide (\pageref{subsec:ab_quick_throw})
    \item Pas besoin d'armes (\pageref{subsec:ab_no_need_for_weapons})
    \item Pratique des armures (\pageref{subsec:ab_practiced_in_armor})
    \item Prouesses au combat (\pageref{subsec:ab_combat_prowess})
    \item Tir d'Opportunité \hyperref[subsec:ab_overwatch]{\pageref{subsec:ab_overwatch}}
\end{abnamelist}

%#######################################################################
%            CHAPTER 6
%#######################################################################
\chapter{\uppercase{Préférence}\label{ch:chapter6}}
\fancypagestyle{plain}{ %
\fancyhf{} % remove everything
\renewcommand{\headrulewidth}{1pt} % remove lines as well
\renewcommand{\footrulewidth}{0pt}
\xpretocmd\headrule{\color{BlueViolet}}{}{\PatchFailed}
\fancyhead[RO]{\textcolor{BlueViolet}{Préférence}}
\fancyhead[LE]{\textcolor{BlueViolet}{CYPHER SYSTEM}}
}
%#######################################################################
%            CHAPTER 7
%#######################################################################
\chapter{\uppercase{Descripteur}\label{ch:chapter7}}
\fancypagestyle{plain}{ %
\fancyhf{} % remove everything
\renewcommand{\headrulewidth}{1pt} % remove lines as well
\renewcommand{\footrulewidth}{0pt}
\xpretocmd\headrule{\color{BlueViolet}}{}{\PatchFailed}
\fancyhead[RO]{\textcolor{BlueViolet}{Descripteur}}
\fancyhead[LE]{\textcolor{BlueViolet}{CYPHER SYSTEM}}
}
%#######################################################################
%            CHAPTER 8
%#######################################################################
\chapter{\uppercase{Focus}\label{ch:chapter8}}
\fancypagestyle{plain}{ %
\fancyhf{} % remove everything
\renewcommand{\headrulewidth}{1pt} % remove lines as well
\renewcommand{\footrulewidth}{0pt}
\xpretocmd\headrule{\color{BlueViolet}}{}{\PatchFailed}
\fancyhead[RO]{\textcolor{BlueViolet}{Focus}}
\fancyhead[LE]{\textcolor{BlueViolet}{CYPHER SYSTEM}}
}

%#######################################################################
%            CHAPTER 9
%#######################################################################
\chapter{\uppercase{Capacités}\label{ch:chapter9}}
\fancypagestyle{plain}{ %
\fancyhf{} % remove everything
\renewcommand{\headrulewidth}{1pt} % remove lines as well
\renewcommand{\footrulewidth}{0pt}
\xpretocmd\headrule{\color{BlueViolet}}{}{\PatchFailed}
\fancyhead[RO]{\textcolor{BlueViolet}{Capacités}}
\fancyhead[LE]{\textcolor{BlueViolet}{CYPHER SYSTEM}}
}

Ce chapitre présente un grand catalogue de plus d'un millier de capacités qu'un personnage peut obtenir par son type, ses préférences et son focus. Elles sont triées par ordre alphabétique.

Le type de personnage, ses préférences et son focus donne un rang particulier à chaque capacité. Toutefois, si vous créez un nouveau focus ou un nouveau type, nous vous fournissons quelques outils supplémentaires.

Le premier outil est un niveau de pouvoir pour chaque capacité. Ce niveau vous indiquera à quel point il est puissant par rapport aux autres capacités. Les capacités qui sont appropriées pour les personnages de rang 1 et 2 sont appelées "capacités de rang inférieur", celles pour les personnages de rang 3 et 4 sont appelées "capacités de rang intermédiaire", et celles pour les personnages de rang 5 et 6 sont appelées "capacités de rang supérieures".

Les capacités sont triées ci-après en catégories basées sur les sortes de choses qu'elles font. Par exemple, les capacités qui améliorent les attaques physiques sont dans la catégorie des compétences en "attaque", les capacités qui soutiennent les alliés sont dans la catégorie "support", et ainsi de suite.

\section*{Catégories de Capacités et Pouvoir Relatif}

Les capacités peuvent être divisées en plusieurs catégories basées sur les sortes de choses qu'elles font. Par exemple, les capacités qui améliorent les attaques physiques sont dans la catégorie des compétences en "attaque", les capacités qui soutiennent les alliés sont dans la catégorie "support", et ainsi de suite. Après chaque description de catégorie vous trouverez une liste de capacités triées par rang (inférieur/intermédiaire/supérieur).

Les catégories sont principalement utilisées par les MJs quand ils définissent les nouveaux focus pour une campagne pour leur permettre de ne chercher qu'une liste simple de capacités au lieu d'essayer de trouver quelque chose d'approprié parmi le miller de capacité dans ce chapitre. Par exemple, la Meneuse pourrait avoir un focus, personnalisée pour sa campagne, et appelée "Est Née dans le Marais" et voudrait une capacité défensive de rang 5. Dans ce cas, la Meneuse peut jeter un coup d'oeil aux capacités de rang supérieur dans la catégorie "protection" et trouver rapidement quelles options sont disponibles.

%____________________________________________________________________---------------
% numberless
\titleformat{name=\section,numberless}[runin] 
    {\normalfont\Large\bfseries}
    {\\ \textcolor{BlueViolet} {\uppercase{Capacités}\ \textendash\ #1}}
    {0.5em}
    {\Large}
    []
%____________________________________________________________________---------------
%% define format for subsection for abilities
\titleformat{name=\subsection,numberless}[runin]
    {\normalfont\scshape}
    {{\includegraphics[height=3mm]{CS-mini-logo.png}}\textbf{#1:}\ } % label
    {0pt}{}[]

\titlespacing*{\subsection}
        {0mm} %left
        {10pt} %before-step
        {0mm} %after-step
        {} %right


%--------------------------
\section*{A}

\subsection*{A l'Aise à Bord}\label{subsec:ab_ship_footing}

Pendant dix minutes, toutes les tâches que vous effectuez sur un vaisseau spatial sont facilitées. Action à initier. (Ship Footing \textendash (182))

\subsection*{Absorber l'Esprit}\label{subsec:ab_infuse_spirit}

lorsque vous tuez une créature ou détruisez un esprit avec une attaque, si vous le souhaitez, son esprit (s'il n'est pas protégé) vous infuse immédiatement et vous regagnez 1d6 points dans l'une de vos Réserves (votre choix). L'esprit est stocké en vous, ce qui signifie qu'il ne peut être questionné, ressuscité ou ramené à la vie par quelque moyen que ce soit, sauf si vous le permettez. Facilitateur. (Infuse Spirit \textendash (153))

\subsection*{Absorber l'énergie}\label{subsec:ab_absorb_energy}

Vous touchez un objet et absorbez son énergie. Si vous touchez un chiffre manifeste, vous le rendez inutile. Si vous touchez un artefact, lancez un jet pour son épuisement. Si vous touchez un autre type de machine ou d'appareil alimenté, le MJ détermine si son énergie est complètement épuisée. Dans tous les cas, vous absorbez l'énergie de l'objet touché et récupérez 1d10 points d'Intellect. Si cela vous donne plus d'Intellect que le maximum de votre Pool, les points supplémentaires sont perdus et vous devez effectuer un jet de défense de Puissance. La difficulté du lancer est égale au nombre de points sur votre maximum que vous avez absorbés. Si vous échouez au jet, vous subissez 5 points de dégâts et êtes incapable d'agir pendant un round. Vous pouvez utiliser cette capacité comme action de défense lorsque vous êtes la cible d'une capacité entrante. Cela annule la capacité entrante et vous absorbez l'énergie comme s'il s'agissait d'un appareil. Action. (Absorb Energy \textendash (108))

\subsection*{Absorber l'énergie cinétique}\label{subsec:ab_absorb_kinetic_energy}

Vous absorbez une partie de l'énergie d'une attaque physique ou d'un impact. Vous annulez 1 point de dommage que vous auriez subi et stockez ce point sous forme d'énergie. Une fois que vous avez absorbé 1 point d'énergie, vous continuez à annuler 1 point de dégâts de tout coup ou impact entrant, mais l'énergie résiduelle s'évapore avec une lueur de lumière inoffensive (vous ne pouvez pas stocker plus de 1 point à la fois). Facilitateur. (Absorb Kinetic Energy \textendash (108))

\subsection*{Absorber l'énergie pure}\label{subsec:ab_absorb_pure_energy}

lorsque vous utilisez Absorber l'énergie cinétique, vous pouvez également absorber et stocker l'énergie provenant d'attaques effectuées avec de l'énergie pure (lumière focalisée, rayonnement, transdimensionnel, psychique, etc.) ou de conduits qui dirigent l'énergie, si vous pouvez établir un contact direct. . Cette capacité ne change pas le nombre de points d'énergie que vous pouvez stocker. Si vous disposez également d'une absorption améliorée de l'énergie cinétique, vous pouvez également absorber 2 points de dégâts provenant d'autres sources d'énergie. Facilitateur. (Absorb Pure Energy \textendash (108))

\subsection*{Absorption d'énergie cinétique améliorée}\label{subsec:ab_improved_absorb_kinetic_energy}

lorsque vous utilisez Absorb Kinetic Energy, au lieu de pouvoir absorber 1 point de dégâts d'une attaque physique ou d'un impact, vous pouvez absorber 2 points. Vous pouvez également stocker jusqu'à 2 points d'énergie provenant de n'importe quelle source. Cependant, vous ne pouvez toujours libérer de l'énergie qu'un seul point à la fois. Facilitateur. (Improved Absorb Kinetic Energy \textendash (151))

\subsection*{Accélération temporelle}\label{subsec:ab_temporal_acceleration}

vous ou une créature volontaire que vous touchez vous déplacez plus rapidement dans le temps. L'effet dure une minute. Tout se déplace plus lentement pour le personnage concerné, tandis que pour tous les autres, le personnage semble se déplacer à une Célérité surnaturelle. Le personnage dispose d'un atout sur toutes les tâches jusqu'à la fin de l'effet. Une fois l'effet terminé, la cible est épuisée et désorientée par l'expérience, ce qui gêne toutes les tâches pendant une heure. Action. (Temporal Acceleration \textendash (190))

\subsection*{Accélérer}\label{subsec:ab_up_to_speed}

si vous ne faites que bouger pendant trois actions consécutives, vous accélérez considérablement et pouvez vous déplacer jusqu'à 200 mph (environ 2000 pieds à chaque round) pendant dix minutes maximum (environ 35 miles), après quoi vous devez vous arrêter. et effectuez un jet de récupération. (Déplacez-vous jusqu'à 322 km/h [environ 600 m à chaque round] pendant dix minutes maximum [environ 56 km].) Facilitateur. (Up to Speed \textendash (195))

\subsection*{Activer les autres}\label{subsec:ab_enable_others}

vous pouvez utiliser les règles de coopération de type "aide" pour offrir un avantage à un autre personnage tentant une tâche physique. Contrairement aux règles d'aide normales, cela ne vous oblige pas à utiliser votre action pour aider l'autre personnage dans la tâche. Cela ne nécessite aucune action de votre part. Facilitateur. (Enable Others \textendash (133))

\subsection*{Activiste communautaire}\label{subsec:ab_community_activist}

lorsque vous parlez à d'autres membres d'une communauté avec laquelle vous avez un lien étroit, vous êtes entraîné aux tâches de persuasion et d'intimidation sur des sujets directement liés à la communauté. Facilitateur. (Community Activist \textendash (121))

\subsection*{Adaptation}\label{subsec:ab_adaptation}

Grâce à une mutation latente, un appareil implanté dans votre colonne vertébrale, un rituel réalisé avec du sang de dragon ou un autre don, vous restez désormais à une température confortable ; vous n'aurez jamais à vous soucier des radiations dangereuses, des maladies ou des gaz ; et peut toujours respirer dans n'importe quel environnement (même dans le vide de l'espace). Facilitateur. (Adaptation \textendash (108))

\subsection*{Adaptation à l'eau}\label{subsec:ab_water_adaptation}

Vous pouvez respirer de l'eau aussi facilement que de l'air. Facilitateur. (Water Adaptation \textendash (196))

\subsection*{Adepte de la microgravité}\label{subsec:ab_microgravity_adept}

Vous ignorez tous les effets néfastes de la faible gravité et de l'absence de gravité sur le mouvement ; vous êtes entraîné aux manœuvres en basse gravité et en apesanteur. (Vous pourriez toujours être soumis aux effets biologiques négatifs d'une exposition à long terme, le cas échéant.) Facilitateur. (Microgravity Adept \textendash (162))

\subsection*{Adepte des manœuvres}\label{subsec:ab_maneuvering_adept}

si vous appliquez au moins un niveau d'effort à une tâche impliquant de grimper, de sauter, d'équilibrer ou tout autre type de manœuvre, vous obtenez un niveau d'effort gratuit. Facilitateur. (Maneuvering Adept \textendash (160))

\subsection*{Affichage Tête Haute}\label{subsec:ab_heads_up_display}

Votre capacité Armure motorisée est livrée avec des systèmes qui vous aident à donner un sens, à analyser et à utiliser vos armes dans votre environnement. Lorsque vous déclenchez cette capacité, vous gagnez un atout sur un jet d'attaque car la combinaison décrit parfaitement les ennemis et stabilise votre visée, que vous effectuiez une attaque au corps à corps ou à distance. Vous pouvez également utiliser l'affichage tête haute pour agrandir votre vision, augmentant ainsi votre portée de vision à 8 km pendant deux tours. Si vous appliquez un niveau d'effort, vous pouvez également voir à travers des matériaux banals (tels que le bois, le béton, le plastique et la pierre) sur une courte distance dans des images en fausses couleurs. Si vous appliquez deux niveaux d'effort, vous pouvez voir à travers des matériaux spéciaux (tels que du plomb solide ou d'autres substances) à une distance immédiate dans des images en fausses couleurs ; cependant, le MJ peut vous demander de réussir d'abord une tâche basée sur l'Intellect, en fonction du matériau bloquant les capteurs de votre armure. Facilitateur. (Heads-Up Display \textendash (148))

\subsection*{Affinité machine}\label{subsec:ab_machine_affinity}

Vous êtes entraîné aux tâches impliquant des machines électriques. Facilitateur. (Machine Affinity \textendash (159))

\subsection*{Affirmer votre privilège}\label{subsec:ab_asserting_your_privilege}

Agissant comme seule une personne privilégiée peut le faire, vous haranguez verbalement un ennemi qui peut vous entendre et vous comprendre avec une telle force qu'il est incapable d'entreprendre la moindre action, y compris d'attaquer, pendant un round. Que vous réussissiez ou échouiez, la prochaine action entreprise par la cible est entravée. Action. (Asserting Your Privilege \textendash (110))

\subsection*{Agent Provocateur}\label{subsec:ab_agent_provocateur}

Choisissez l'un des domaines suivants pour vous former : attaquer avec une arme de votre choix, démolitions, ou furtivité et crochetage (si vous choisissez cette dernière option, vous êtes entraîné aux deux). Facilitateur. (Agent Provocateur \textendash (109))

\subsection*{Agrandir}\label{subsec:ab_enlarge}

vous déclenchez une réaction enzymatique qui attire une masse supplémentaire d'une autre dimension, et vous (et vos vêtements ou votre costume) grandissez. Vous atteignez une hauteur de 9 pieds (3 m) et restez ainsi pendant environ une minute. Pendant ce temps, vous ajoutez 4 points à votre réserve de Puissance, ajoutez +1 à votre Armure et ajoutez +2 à votre Avantage de Puissance. Tant que vous êtes plus grand que la normale, vos jets de défense de Célérité sont gênés et vous êtes habitué à utiliser vos poings comme des armes lourdes.Lorsque les effets d'Agrandir prennent fin, votre armure et votre Avantage de Puissance reviennent à la normale et vous soustrayez de votre réserve de Puissance un nombre de points égal au nombre que vous avez gagné (si cela ramène la réserve à 0, soustrayez d'abord le débordement de votre Réserve de Célérité. puis, si nécessaire, depuis votre Réserve d'Intellect). Chaque fois que vous utilisez Agrandir avant votre prochain jet de récupération de dix heures, vous devez appliquer un niveau d'effort supplémentaire. Ainsi, la deuxième fois que vous utilisez Agrandir, vous devez appliquer un niveau d'Effort ; la troisième fois que vous utilisez Agrandir, deux niveaux d'effort ; et ainsi de suite. Action à initier. (Enlarge \textendash (135))

\subsection*{Agrandissement Efrayant}\label{subsec:ab_freakishly_large}

Votre taille accrue intimide la plupart des gens. Pendant que vous profitez des effets d'Agrandissement, toutes les tâches d'intimidation que vous tentez sont facilitées. Facilitateur. (Freakishly Large \textendash (143))

\subsection*{Agression}\label{subsec:ab_agression}

Vous vous concentrez sur les attaques à un point tel que vous vous rendez vulnérable face à vos adversaires. Tant que cette capacité est active, vous gagnez un atout sur vos attaques de mêlée, et vos jets de défense de Célérité contre les attaques de mêlée et à distance sont gênés. Cet effet dure aussi longtemps que vous le souhaitez, mais il prend fin si aucun combat n'a lieu à portée de vos sens. Facilitateur. (Agression \textendash (109))

\subsection*{Aide amicale}\label{subsec:ab_friendly_help}

si votre ami tente une tâche et échoue, il peut réessayer sans dépenser d'effort si vous l'aidez. Vous offrez cet avantage à votre ami même si vous n'êtes pas entraîné à la tâche qu'il réessaye. Facilitateur. (Friendly Help \textendash (143))

\subsection*{Aiguillon}\label{subsec:ab_goad}

vous pouvez tenter d'inciter une cible à adopter une réaction belliqueuse - et probablement stupide - qui oblige la cible à essayer de réduire la distance entre vous et à tenter de vous frapper physiquement lors de son prochain tour. Ils tentent cette action même si cela les amènerait à rompre la formation ou à abandonner leur couverture ou une position tactiquement supérieure. Que la cible vous frappe ou ne parvienne pas à le faire, elle reprend ses esprits immédiatement après, après quoi toute autre tâche tentant d'aiguillonner à nouveau la cible est entravée. Action à initier. (Goad \textendash (145))

\subsection*{Ailes comme Arme}\label{subsec:ab_wing_weapons}

vous pouvez utiliser vos ailes pour effectuer des attaques au corps à corps (même en vol), en laissant vos mains et vos pieds libres. Vos ailes sont des armes de taille moyenne ou blanches (au choix). Vous êtes habitué à cette attaque. Facilitateur. (Wing Weapons \textendash (199))

\subsection*{Ailes de Feu}\label{subsec:ab_wings_of_fire}

Lorsque votre Manteau de flammes est actif, vous pouvez déployer des ailes de feu et léviter, vous déplaçant à une Célérité allant jusqu'à 20 pieds (6 m) par tour dans n'importe quelle direction pendant une minute. Vous pouvez aussi en prendre un une autre action, hors mouvement, pendant votre tour. Action. (Wings of Fire \textendash (199))

\subsection*{Ailes du Vide}\label{subsec:ab_void_wings}

Des rubans tourbillonnants de matière étrange vous saisissent et vous soulèvent, vous permettant de voler pendant un tour aussi vite que vous pouvez vous déplacer. Facilitateur. (Void Wings \textendash (196))

\subsection*{Aller au sol}\label{subsec:ab_go_to_ground}

vous vous déplacez sur une longue distance et tentez de vous cacher. Lorsque vous le faites, vous gagnez un atout dans la tâche furtive pour vous fondre, disparaître ou échapper aux sens de toutes les personnes auparavant conscientes de votre présence. Action. (Go to Ground \textendash (145))

\subsection*{Allumage}\label{subsec:ab_ignition}

Vous désignez une créature ou un objet inflammable que vous pouvez voir à courte portée pour prendre feu. Il s'agit d'une attaque intellectuelle. La cible subit 6 points de dégâts ambiants par round jusqu'à ce que les flammes s'éteignent, ce qu'une créature peut faire en s'aspergeant d'eau, en roulant sur le sol ou en étouffant les flammes. Habituellement, éteindre les flammes nécessite une action. Action à initier. (Ignition \textendash (150))

\subsection*{Altération corporelle}\label{subsec:ab_body_morph}

Vous modifiez les traits et la coloration de votre visage et de votre corps pendant une heure, cachant votre identité ou vous faisant passer pour quelqu'un. Si vous appliquez un niveau d'effort, vous pouvez imiter une personne spécifique avec suffisamment de précision pour tromper quelqu'un qui la connaît bien ou l'a observée de près (y compris ses empreintes digitales et ses empreintes vocales, mais pas son empreinte rétinienne ou son ADN). Vous disposez d'un atout dans toutes les tâches impliquant un déguisement (ceci s'ajoute à l'atout de Changement de Visage). Vous devez appliquer un niveau d'effort distinct pour pouvoir usurper l'identité d'une espèce différente (comme un humain se transformant en extraterrestre humanoïde). Action. (Body Morph \textendash (115))

\subsection*{Amitié de groupe}\label{subsec:ab_group_friendship}

Vous convainquez une créature sensible de vous considérer (et jusqu'à dix créatures que vous désignez à distance immédiate de vous) de manière positive, comme elle le ferait pour un ami potentiel. Action. (Group Friendship \textendash (147))

\subsection*{Amplifier les sons}\label{subsec:ab_amplify_sounds}

Pendant une minute, vous pouvez amplifier les sons lointains ou petits afin de pouvoir les entendre clairement, même s'il s'agit d'une conversation ou du bruit d'un petit animal se déplaçant dans un terrier souterrain jusqu'à une très longue distance. loin. Vous pouvez tenter de percevoir le son même si des barrières intercédantes le bloquent ou si le son est très léger, bien que cela nécessite quelques cycles de concentration supplémentaires. Discriminer le son souhaité dans un environnement bruyant peut également nécessiter quelques cycles de concentration supplémentaires lorsque vous explorez de manière audible le paysage sonore environnant. Avec suffisamment de temps, vous pourriez identifier chaque conversation, chaque créature respirante et chaque appareil créant du bruit à portée. Action à initier, pouvant aller jusqu'à plusieurs tours à réaliser, selon la difficulté de la tâche. (Amplify Sounds \textendash (109))

\subsection*{Amélioration de la machine}\label{subsec:ab_machine_enhancement}

chaque fois que vous utilisez Effort sur une action Intellect, ajoutez l'une des améliorations suivantes à l'action (votre choix) :- Niveau d'effort gratuit - Effet mineur automatique Facilitateur. (Machine Enhancement \textendash (159))

\subsection*{Amélioration du robot}\label{subsec:ab_robot_improvement}

votre assistant artificiel grâce à la capacité Assistant Robot passe au niveau 4. Enabler. (Robot Improvement \textendash (179))

\subsection*{Améliorer la force}\label{subsec:ab_enhance_strength}

Pendant les dix minutes suivantes, vous gagnez un atout sur les tâches qui dépendent de la force brute, comme déplacer un objet lourd, défoncer une porte ou frapper quelqu'un avec une arme de mêlée. Action à initier. (Enhance Strength \textendash (134))

\subsection*{Analyse d'animal}\label{subsec:ab_animal_scrying}

Si vous connaissez l'emplacement général d'un animal qui est amical envers vous et à moins de 1,5 km de votre emplacement, vous pouvez ressentir grâce à ses sens pendant dix minutes maximum. Si vous n'êtes pas sous forme animale ou si vous n'êtes pas sous une forme similaire à cet animal, vous devez appliquer un niveau d'Effort pour utiliser cette capacité. Action à établir. (Animal Scrying \textendash (109))

\subsection*{Analyse en laboratoire}\label{subsec:ab_lab_analysis}

vous analysez la scène d'un crime, le site d'un incident mystérieux ou une série de phénomènes inexpliqués, et apprenez peut-être une quantité surprenante d'informations sur les auteurs, les participants ou les forces. ) responsable. Pour ce faire, vous devez prélever des échantillons sur les lieux. Les échantillons sont des restes de peinture ou de bois, de la saleté, des photographies de la zone, des cheveux, un cadavre entier, etc. Avec des échantillons en main, vous pouvez découvrir jusqu'à trois informations pertinentes sur la scène, éclaircissant éventuellement un moindre mystère et ouvrant la voie à la résolution d'un plus grand. Le MJ décidera de ce que vous apprendrez et du niveau de difficulté nécessaire pour l'apprendre. (A titre de comparaison, découvrir qu'une victime a été tuée non pas par une chute, comme cela semble immédiatement évident, mais plutôt par électrocution, est pour vous une tâche de difficulté 3.) La tâche est facilitée si vous prenez le temps de transporter les échantillons vers un lieu permanent. laboratoire (si vous y avez accès), au lieu d'effectuer l'analyse avec votre kit scientifique de terrain. Action à lancer, 2d20 minutes à réaliser. (Lab Analysis \textendash (157))

\subsection*{Anecdote}\label{subsec:ab_anecdote}

Vous pouvez remonter le moral d'un groupe de créatures et les aider à se lier en les divertissant avec une anecdote édifiante ou pointue. Pendant l'heure suivante, ceux qui prêtent attention à votre histoire sont entraînés à une tâche que vous choisissez et qui est liée à l'anecdote, à condition qu'il ne s'agisse pas d'une tâche d'attaque ou de défense. Action à lancer, une minute à terminer. (Anecdote \textendash (109))

\subsection*{Annuler}\label{subsec:ab_undo}

vous remontez le temps de quelques secondes, annulant ainsi l'action la plus récente d'une créature. Cette créature peut alors immédiatement répéter la même action ou essayer quelque chose de différent. Action. (Undo \textendash (195))

\subsection*{Annuler le danger}\label{subsec:ab_negate_danger}

Vous annulez définitivement une source de danger potentiel liée à une créature ou à un objet à distance immédiate. Il peut s'agir d'une arme ou d'un appareil tenu par quelqu'un, de la capacité naturelle d'une créature ou d'un piège déclenché par une plaque de pression. Action. (Negate Danger \textendash (165))

\subsection*{Annuler le son}\label{subsec:ab_nullify_sound}

vous émettez des sons parfaitement mal alignés à courte portée pour créer une zone de calme absolu jusqu'à une distance immédiate pendant une minute. Tous les sons sont annulés dans la zone. Action à initier. (Nullify Sound \textendash (166))

\subsection*{Anticipation}\label{subsec:ab_anticipation}

Vous regardez vers l'avenir pour voir comment vos actions pourraient se dérouler. La première tâche que vous effectuez avant la fin du tour suivant gagne un atout. Action. (Anticipation \textendash (110))

\subsection*{Anticipation de l'attaque}\label{subsec:ab_anticipate_attack}

Vous pouvez sentir quand et comment les créatures qui vous attaquent effectueront leurs attaques. Les jets de défense contre la Célérité sont allégés pendant une minute. Action. (Anticipate Attack \textendash (110))

\subsection*{Apaiser l'esprit et le corps}\label{subsec:ab_soothe_mind_and_body}

Le corps et l'esprit sont connectés. Toutes les tâches de guérison que vous tentez sont facilitées par deux étapes. Facilitateur. (Soothe Mind and Body \textendash (184))

\subsection*{Apaiser la Bête Sauvage}\label{subsec:ab_soothe_the_savage}

Vous calmez une bête non humaine à moins de 9 m. Vous devez lui parler (même s'il n'a pas besoin de comprendre vos mots) et il doit vous voir. Il reste calme pendant une minute ou aussi longtemps que vous concentrez toute votre attention sur lui. Le MJ a le dernier mot sur ce qui compte comme une bête non humaine, mais à moins qu'une sorte de tromperie ne soit à l'œuvre, vous devez savoir si vous pouvez affecter une créature avant d'essayer d'utiliser cette capacité sur elle. Les extraterrestres, les entités extradimensionnelles, les créatures très intelligentes et les robots ne comptent jamais. Action. (Soothe the Savage \textendash (184))

\subsection*{Aperçu de l'appareil}\label{subsec:ab_device_insight}

lorsque vous examinez un appareil inconnu, extraterrestre ou de haute technologie, vous pouvez poser une question au MJ pour avoir une idée de ses capacités, de son fonctionnement, de la manière dont il peut être activé ou désactivé, de quoi il s'agit. les faiblesses sont (le cas échéant), comment elles peuvent être réparées ou toute autre question similaire. Il s'agit de choses difficiles ou étranges au-delà de celles facilement identifiées en utilisant les connaissances ou les compétences techniques appropriées. Action. (Device Insight \textendash (128))

\subsection*{Appel de Bête}\label{subsec:ab_beast_call}

Vous invoquez une horde de petits animaux ou une seule bête de niveau 4 pour vous aider temporairement. Ces créatures exécutent vos ordres aussi longtemps que vous concentrez votre attention, mais vous devez utiliser votre action à chaque tour pour les diriger. Les créatures sont originaires de la région et arrivent par leurs propres moyens, donc si vous vous trouvez dans un endroit inaccessible, cette capacité ne fonctionnera pas. Action. (Beast Call \textendash (112))

\subsection*{Appel à travers le temps}\label{subsec:ab_call_through_time}

Vous appelez une créature ou une personne jusqu'au niveau 3 du passé récent, et elle apparaît à côté de vous. Vous pouvez choisir une créature que vous avez déjà rencontrée (même si elle est maintenant morte), ou (pas plus d'une fois par jour) vous pouvez permettre au MJ de déterminer la créature au hasard. Si vous appelez une créature aléatoire, elle a 10 pour cent de chances d'être une créature jusqu'au niveau 5. La créature n'a aucun souvenir de quoi que ce soit avant d'être appelée par vous, bien qu'elle puisse parler et ait les connaissances générales d'une créature de ce type. devrait posséder. La créature décalée dans le temps exécute vos ordres aussi longtemps que vous vous concentrez sur elle, mais vous devez utiliser votre action à chaque tour pour la diriger ; sinon, cela revient au passé.En plus des options normales d'utilisation de l'Effort, vous pouvez choisir d'utiliser l'Effort pour appeler une créature plus puissante ; chaque niveau d'Effort utilisé de cette manière augmente le niveau de la créature de 1. Par exemple, appliquer un niveau d'Effort appelle une créature spécifique jusqu'au niveau 4 ou une créature aléatoire avec 10 pour cent de chances d'atteindre le niveau 6. Action. (Call Through Time \textendash (118))

\subsection*{Appeler l'esprit d'un mort}\label{subsec:ab_call_dead_spirit}

À votre contact, les restes d'une créature morte depuis moins de sept jours apparaissent comme un esprit manifeste (et apparemment physique), dont le niveau est le même que celui de son vivant. L'esprit ressuscité persiste jusqu'à un jour (ou moins, s'il accomplit quelque chose d'important pour lui avant cette date), après quoi il disparaît et ne peut plus revenir.L'esprit élevé se souvient de tout ce qu'il a connu dans la vie et possède la plupart de ses capacités antérieures (mais pas nécessairement son équipement). De plus, il acquiert la capacité de devenir insubstantiel en tant qu'action pendant une minute à la fois. L'esprit élevé ne vous est pas redevable et il n'a pas besoin de rester près de vous pour rester manifesté. Action à initier. (Call dead spirit \textendash (117))

\subsection*{Appeler la tempête}\label{subsec:ab_call_the_storm}

Si vous êtes à l'extérieur ou dans un endroit dont le plafond est à au moins 90 m au-dessus du sol, vous invoquez une couche bouillante de nuages grondants et éclairés par des éclairs jusqu'à 1 500 pieds. (460 m) de diamètre pendant dix minutes. Pendant la journée, l'éclairage naturel sous la tempête est réduit à une lumière tamisée. Pendant que la tempête fait rage, vous pouvez utiliser une action pour envoyer un éclair depuis le nuage pour attaquer une cible que vous pouvez voir directement, lui infligeant 4 points de dégâts (vous pouvez dépenser de l'Effort normalement sur chaque attaque d'éclair individuelle). Trois actions à initier ; action pour déclencher un coup de foudre. (Call the Storm \textendash (117))

\subsection*{Appeler un esprit d'un autre monde}\label{subsec:ab_call_otherworldly_spirit}

vous invoquez une créature spirituelle qui se manifeste pendant un jour maximum (ou moins, si elle accomplit quelque chose d'important pour elle avant cette date), après quoi elle disparaît et ne peut plus être invoquée. L'esprit est une créature de niveau 6 ou inférieur, et il peut être substantiel ou insubstantiel à sa guise (en utilisant une action pour changer). L'esprit ne vous est pas redevable et il n'a pas besoin de rester près de vous pour rester manifesté. Action à initier. (Call Otherworldly Spirit \textendash (117))

\subsection*{Appeller un essaim}\label{subsec:ab_call_swarm}

Si vous vous trouvez dans un endroit où il est possible que les créatures de votre capacité Contrôle d'Essaim arrivent, vous en appelez un essaim pendant une heure. Pendant cette heure, ils font ce que vous commandez par télépathie tant qu'ils sont à longue portée. Ils peuvent envahir et gêner n'importe quelle tâche de l'adversaire. Lorsque les créatures se trouvent à longue portée, vous pouvez leur parler par télépathie et percevoir à travers leurs sens. Action à initier. (Call Swarm \textendash (118))

\subsection*{Appliquer vos connaissances}\label{subsec:ab_applying_your_knowledge}

lorsque vous aidez un autre personnage à entreprendre une action pour laquelle vous n'êtes pas entraîné, vous êtes traité comme si vous y étiez entraîné. Action. (Applying your Knowledge \textendash (110))

\subsection*{Apprendre le chemin}\label{subsec:ab_learning_the_path}

Vous observez ou étudiez une créature, un objet ou un lieu pendant au moins un round. La prochaine fois que vous interagissez avec elle (éventuellement au tour suivant), une tâche associée (comme persuader la créature, l'attaquer ou se défendre contre son attaque) est facilitée. Action. (Learning the Path \textendash (157))

\subsection*{Appris des trucs}\label{subsec:ab_learned_a_few_things}

Vous êtes entraîné dans deux domaines de connaissances de votre choix, ou spécialisé dans un domaine de connaissances de votre choix. Facilitateur. (Learned a Few Things \textendash (157))

\subsection*{Arme Electrique}\label{subsec:ab_charge_weapon}

Dans le cadre d'une attaque avec votre arme enchantée, vous la chargez de pouvoir magique, lui conférant 2 points de dégâts énergétiques supplémentaires. Si vous effectuez plus d'une attaque pendant votre tour, vous choisissez de dépenser ou non le coût de cette capacité avant d'effectuer chaque attaque. Facilitateur. (Charge Weapon \textendash (119))

\subsection*{Arme de défense}\label{subsec:ab_defending_weapon}

lorsque vous utilisez votre arme enchantée, vous êtes entraîné aux tâches de défense rapide. Facilitateur. (Defending Weapon \textendash (126))

\subsection*{Arme enchantée}\label{subsec:ab_enchanted_weapon}

Vous vous harmonisez avec une arme physique, comme une épée, un marteau ou un arc. Vous savez exactement où il se trouve s'il se trouve à une courte distance de vous, et vous connaissez sa direction générale et sa distance s'il est plus éloigné. Toutes vos autres capacités de concentration nécessitent que vous teniez ou maniiez cette arme. Vous ne pouvez être harmonisé qu'à une seule arme à la fois ; vous harmoniser avec une deuxième arme perd l'harmonisation avec la première. Action à initier, dix minutes à réaliser. Facilitateur.Si vous vous harmonisez à une arme différente, inventez une histoire expliquant pourquoi vous êtes capable de le faire et pourquoi vous avez choisi cette nouvelle arme. (Enchanted Weapon \textendash (GF, 31)(CTS, 52))

\subsection*{Arme et corps}\label{subsec:ab_weapon_and_body}

Après avoir effectué une attaque avec une arme de mêlée ou une arme à distance, vous enchaînez avec un coup de poing ou un coup de pied comme attaque supplémentaire, le tout dans le cadre de la même action en un seul round. Les deux attaques peuvent être dirigées vers des ennemis différents. Effectuez un jet d'attaque distinct pour chaque attaque. Vous restez limité par la quantité d'Effort que vous pouvez appliquer sur une action. Tout ce qui modifie votre attaque ou vos dégâts s'applique aux deux attaques, sauf si cela est spécifiquement lié à votre arme. Action. (Weapon and Body \textendash (196))

\subsection*{Armement}\label{subsec:ab_weaponization}

une arme de mêlée légère ou moyenne de votre choix est intégrée à votre corps et vous êtes entraîné à l'utiliser. L'arme est cachée jusqu'à ce que vous souhaitiez l'utiliser. Facilitateur. (Weaponization \textendash (197))

\subsection*{Armes intégrées}\label{subsec:ab_built_in_weaponry}

des implants biomécaniques, un joyau magique fusionné sur votre front ou quelque chose d'aussi sauvage vous fournissent désormais des armes inhérentes. Cela vous permet de tirer une explosion d'énergie à longue portée qui inflige 5 points de dégâts. L'utilisation de cette capacité est gratuite. Action. (Built-In Weaponry \textendash (116))

\subsection*{Armure Corporelle}\label{subsec:ab_fusion_armor}

Une procédure vous donne des implants biométalliques dans des parties importantes de votre corps, vous développez une peau dure comme du métal, ou les bénédictions d'un ange vous protègent ou quelque chose de similaire se produit. Ces changements vous donnent +1 en armure même lorsque vous ne portez pas d'armure physique. Facilitateur. (Fusion Armor \textendash (144))

\subsection*{Armure Renforcée par Champs de Force}\label{subsec:ab_field_reinforced_armor}

vous gagnez +1 en armure lorsque vous portez l'armure assistée de votre capacité Armure motorisée. Facilitateur. (Field-Reinforced Armor \textendash (139))

\subsection*{Armure de Glace}\label{subsec:ab_ice_armor}

Lorsque vous le souhaitez, votre corps est recouvert d'un éclat de glace pendant dix minutes qui vous donne +1 en Armure. Tant que l'éclat est actif, vous ne ressentez aucune gêne due aux températures froides normales et bénéficiez d'un +2 supplémentaire en armure contre les dégâts du froid en particulier. Facilitateur. (Ice Armor \textendash (150))

\subsection*{Armure de glace résiliente}\label{subsec:ab_resilient_ice_armor}

L'éclat de glace que vous générez à l'aide de votre capacité Armure de glace vous donne un +1 supplémentaire à l'armure. Facilitateur. (Resilient Ice Armor \textendash (176))

\subsection*{Armure de vent}\label{subsec:ab_wind_armor}

Lorsque vous le souhaitez, un cyclone de vent entoure votre corps pendant dix minutes, vous donnant +1 à l'Armure et un +2 supplémentaire à l'Armure contre les armes à projectiles physiques spécifiquement. Pendant que le cyclone est actif, vous ne ressentez aucune gêne due au vent et vous pouvez interagir normalement avec d'autres créatures et objets car le flux de vent se détourne automatiquement pour permettre une telle interaction. Facilitateur. (Wind Armor \textendash (199))

\subsection*{Armure motorisée}\label{subsec:ab_powered_armor}

Vous disposez d'une armure motorisée. Il s'agit en fait d'une armure moyenne (+2 à l'Armure) ; cependant, vous ne subissez aucune pénalité de Célérité pour le porter. En outre, votre combinaison offre d'autres avantages : elle fournit de l'air respirable jusqu'à huit heures et un environnement confortable même dans une chaleur âpre, un froid glacial, sous le vide ou sous l'eau jusqu'à une profondeur de 6 km ; et il vous permet de voir dans l'obscurité jusqu'à une courte distance. Enfiler la combinaison nécessite une action (et, bien sûr, l'accès à votre combinaison). Facilitateur. (Powered Armor \textendash (171))

\subsection*{Armure vivante}\label{subsec:ab_living_armor}

Si vous vous trouvez dans un endroit où il est possible que vos créatures d'Contrôle d'Essaim viennent, vous appelez un essaim autour de vous pendant une heure. Ils rampent sur votre corps ou volent autour de vous dans un nuage. Pendant ce temps, vos tâches de défense de Célérité sont facilitées et vous gagnez +1 en armure. Action à initier. (Living Armor \textendash (158))

\subsection*{Armure électrique}\label{subsec:ab_electric_armor}

Lorsque vous le souhaitez, l'électricité crépite sur votre corps pendant dix minutes, vous accordant +1 d'armure. Lorsque vous êtes électrifié, vous disposez d'une armure supplémentaire de +2 contre les dégâts électriques spécifiquement, et vous infligez 2 points de dégâts à toute créature qui vous touche ou vous attaque avec une arme de mêlée conductrice d'électricité. Facilitateur. (Electric Armor \textendash (133))

\subsection*{Artisan}\label{subsec:ab_crafter}

Vous êtes entraîné à la fabrication de deux types d'objets. Facilitateur. (Crafter \textendash (122))

\subsection*{Artisan de Poisons}\label{subsec:ab_poison_crafter}

Vous êtes entraîné à la fabrication, à la détection, à l'identification et à la résistance aux poisons. Votre fabrication de poison vous a conféré une certaine immunité contre les poisons ; vous avez +5 d'armure qui s'applique spécifiquement aux dégâts de poison. Facilitateur. (Poison Crafter \textendash (170))

\subsection*{Artisan expert}\label{subsec:ab_expert_crafter}

au lieu de lancer un jet, vous pouvez choisir de réussir automatiquement une tâche d'artisanat pour laquelle vous êtes entraîné. La tâche doit être de difficulté 4 ou inférieure. Si vous parvenez à réduire la difficulté évaluée d'une tâche de fabrication à 4 ou moins, cette capacité s'applique également à chaque sous-tâche, en supposant que quelque chose ne vous interrompe pas pendant le temps de construction qui s'ensuit. Facilitateur. (Expert Crafter \textendash (137))

\subsection*{Artisan naturel}\label{subsec:ab_natural_crafter}

tous les objets ou structures courants que vous fabriquez sont effectivement d'un niveau supérieur à un exemple moyen de cet objet ou de cette structure. Par exemple, si vous construisez un mur défensif qui serait normalement de niveau 4, son niveau effectif est de 5. Facilitateur. (Natural Crafter \textendash (165))

\subsection*{Artisanat des Rêves}\label{subsec:ab_dreamcraft}

vous extrayez une image d'un rêve dans le monde éveillé et la placez quelque part à longue portée. Le rêve dure jusqu'à une minute et peut être minuscule ou remplir une zone d'un diamètre immédiat. Même s'il paraît solide, le rêve est intangible. Le rêve (une scène, une créature ou un objet) est statique à moins que vous n'utilisiez votre action à chaque tour pour l'animer. Dans le cadre de cette animation, vous pouvez déplacer le rêve sur une courte distance à chaque tour, à condition qu'il reste à longue portée. Si vous animez le rêve, il peut émettre du son mais ne produit pas d'odeur. Une interaction physique directe ou une interaction soutenue avec le rêve le brise en une brume dispersante. Par exemple, attaquer le rêve le brise, tout comme la difficulté de préserver les apparences lorsqu'un PNJ se déplace dans une scène de rêve ou engage une conversation avec une créature de rêve pendant plus de deux rounds. Action à initier ; action à animer. (Dreamcraft \textendash (132))

\subsection*{Artiste de la confiance}\label{subsec:ab_confidence_artist}

lorsque vous piratez un système informatique, organisez une escroquerie, faites les poches, trompez ou trompez un dupe, faufilez quelque chose à un garde, etc., vous gagnez un atout dans cette tâche. Facilitateur. (Confidence Artist \textendash (121))

\subsection*{As du Volant}\label{subsec:ab_trick_driver}

Lorsque vous conduisez une voiture, un camion ou une moto, vos Avantages de Puissance, Célérité et Intellect augmentent de 1. Lorsque vous effectuez un jet de récupération en conduisant, vous récupérez 5 points supplémentaires. Lorsque vous tentez une tâche de conduite ou une astuce extrême, comme sauter un ravin ou un autre véhicule, tourner dans les airs, atterrir en toute sécurité sur un autre véhicule, etc., la tâche est facilitée. Facilitateur. (Trick Driver \textendash (194))

\subsection*{Assaut Magique}\label{subsec:ab_onslaught}

Vous attaquez un ennemi en utilisant des énergies qui attaquent soit sa forme physique, soit son esprit. Dans les deux cas, vous devez pouvoir voir votre cible. Si l'attaque est physique, vous émettez un rayon de force à courte portée qui inflige 4 points de dégâts. Si l'attaque est mentale, vous concentrez votre énergie mentale pour détruire les processus de pensée d'une autre créature à courte portée. Cette tranche mentale inflige 2 points de dégâts d'Intellect (ignore l'Armure). Certaines créatures sans esprit (comme les robots) peuvent être immunisées contre votre tranche mentale. Action. (Onslaught \textendash (167))

\subsection*{Assistance compétente}\label{subsec:ab_able_assistance}

lorsque vous aidez quelqu'un dans une tâche et qu'il applique un niveau d'effort, il obtient un niveau d'effort gratuit sur cette tâche. Facilitateur. (Able Assistance \textendash (108))

\subsection*{Assistant Robot}\label{subsec:ab_robot_assistant}

Un robot de niveau 2 de votre taille ou plus petit (construit par vos soins) vous accompagne et suit vos instructions. Vous et le MJ devez régler les détails de votre robot. Vous ferez probablement des jets pour lui lorsqu'il entreprendra des actions. Un assistant robot au combat ne réalise généralement pas d'attaques séparées mais vous aide dans les vôtres. Sur votre action, si l'assistant artificiel est à côté de vous, il vous sert d'atout pour une attaque que vous effectuez à votre tour. Si le robot est détruit, vous pouvez réparer l'original en quelques jours de bricolage, ou en construire un nouveau avec une semaine de travail à mi-temps. Facilitateur. (Robot Assistant \textendash (178))

\subsection*{Assumer le contrôle}\label{subsec:ab_assume_control}

vous contrôlez les actions d'une autre créature avec laquelle vous avez interagi ou étudié pendant au moins un tour. Cet effet dure dix minutes. La cible doit être de niveau 2 ou inférieur. Une fois que vous avez pris le contrôle, la cible agit comme si elle voulait réaliser votre désir au mieux de ses capacités, en utilisant librement son propre jugement, à moins que vous n'utilisiez une action pour lui donner une instruction spécifique, problème par problème. En plus des options normales d'utilisation de l'Effort, vous pouvez choisir d'utiliser l'Effort pour augmenter le niveau maximum de la cible. Ainsi, pour tenter de commander une cible de niveau 5 (trois niveaux au-dessus de la limite normale), vous devez appliquer trois niveaux d'Effort. Lorsque l'effet prend fin, la cible se souvient de tout ce qui s'est passé et réagit en fonction de sa nature et de votre relation avec elle ; supposer que le contrôle aurait pu détériorer cette relation si elle était auparavant positive. Action à initier. (Assume Control \textendash (111))

\subsection*{Astuces subtiles}\label{subsec:ab_subtle_tricks}

vous pouvez utiliser vos compétences et capacités spéciales d'une manière qui ne donne pas l'impression que vous faites quoi que ce soit. Si la compétence ou la capacité nécessite normalement un mouvement, une phrase ou une autre action évidente de votre part, elle semble se produire d'elle-même. Au lieu d'utiliser vos outils pour crocheter une serrure, la serrure s'ouvre lorsque vous vous tenez à proximité. Au lieu de manipuler un écran d'ordinateur, les informations souhaitées apparaissent sur l'écran lorsque vous le regardez. Au lieu de bluffer devant certains gardes, ils s'écartent à votre approche et vous laissent passer. Cette capacité ne fonctionne généralement que jusqu'à une distance immédiate. Vous devez toujours dépenser des points et effectuer des jets pour utiliser vos compétences et capacités avec Astuces subtiles. Utiliser une compétence ou une capacité de manière subtile entrave la tâche. Cette capacité ne peut pas être utilisée pour dissimuler vos jets d'attaque ou de défense. Facilitateur. (Subtle Tricks \textendash (187))

\subsection*{Athlète}\label{subsec:ab_athlete}

Vous êtes entraîné à porter, grimper, sauter et écraser. Facilitateur. (Athlete \textendash (111))

\subsection*{Attaque Etourdissante}\label{subsec:ab_stun_attack}

vous tentez une tâche de difficulté de Célérité 5 pour étourdir une créature dans le cadre de votre attaque au corps à corps ou à distance. Si vous réussissez, votre attaque inflige ses dégâts normaux et étourdit la créature pendant un round, lui faisant perdre son prochain tour. Si vous échouez, vous effectuez toujours votre jet d'attaque normal, mais vous n'étourdissez pas l'adversaire si vous touchez. Si vous possédez également cette capacité provenant d'une autre source (par exemple en l'ayant comme capacité de type et comme capacité de concentration), son utilisation ne vous coûte que 3 points au lieu de 6 points. Action. (Stun Attack \textendash (187))

\subsection*{Attaque Surprise Améliorée}\label{subsec:ab_better_surprise_attack}

si vous attaquez depuis un point de vue caché, avec surprise ou avant qu'un adversaire n'ait agi, vous obtenez un atout lors de l'attaque (si vous avez une attaque surprise, cela s'ajoute à l'atout de cette capacité). En cas de réussite de cette attaque surprise, vous infligez 2 points de dégâts supplémentaires (pour un total de 4 points de dégâts supplémentaires si vous avez Attaque Surprise). Facilitateur. (Better Surprise Attack \textendash (113))

\subsection*{Attaque Tournoyante}\label{subsec:ab_spin_attack}

vous restez immobile et effectuez des attaques contre jusqu'à cinq ennemis, le tout dans le cadre de la même action en un seul tour. Toutes les attaques doivent être du même type (mêlée ou à distance). Effectuez un jet d'attaque séparé pour chaque ennemi. Vous restez limité par la quantité d'Effort que vous pouvez appliquer sur une action. Tout ce qui modifie votre attaque ou vos dégâts s'applique à toutes ces attaques. En plus des options normales d'utilisation de l'Effort, vous pouvez choisir d'utiliser l'Effort pour augmenter le nombre d'ennemis que vous pouvez attaquer avec cette capacité (un ennemi supplémentaire par niveau d'Effort utilisé de cette manière). Action. (Spin Attack \textendash (185))

\subsection*{Attaque acrobatique}\label{subsec:ab_acrobatic_attack}

vous vous lancez dans l'attaque, en vous tournant ou en vous retournant dans les airs. Si vous obtenez un 17 ou un 18 naturel, vous pouvez choisir d'avoir un effet mineur plutôt que d'infliger des dégâts supplémentaires. Si vous appliquez un effort à l'attaque, vous obtenez un niveau d'effort gratuit sur la tâche. Vous ne pouvez pas utiliser cette capacité si vos coûts d'effort de Célérité sont réduits par le port d'une armure. Facilitateur. (Acrobatic Attack \textendash (108))

\subsection*{Attaque avec style}\label{subsec:ab_attack_flourish}

Avec votre attaque, vous ajoutez des mouvements élégants, des plaisanteries divertissantes ou quelque chose qui divertit ou impressionne les autres. Une créature que vous choisissez à courte portée et qui peut vous voir gagne un atout pour sa prochaine tâche si elle est prise dans un tour ou deux. Facilitateur. (Attack Flourish \textendash (111))

\subsection*{Attaque aveuglante}\label{subsec:ab_blinding_attack}

Si vous disposez d'une source de lumière, vous pouvez l'utiliser pour effectuer une attaque au corps à corps contre une cible. En cas de réussite, l'attaque n'inflige aucun dégât, mais la cible est aveuglée pendant une minute. Action. (Blinding Attack \textendash (115))

\subsection*{Attaque de Phase}\label{subsec:ab_phased_attack}

L'attaque que vous effectuez ce tour-ci ignore l'armure de votre ennemi. Cette capacité fonctionne quel que soit le type d'attaque que vous utilisez (mêlée, à distance, énergétique, etc.). Facilitateur. (Phased Attack \textendash (170))

\subsection*{Attaque de Phase Améliorée}\label{subsec:ab_enhanced_phased_attack}

Cette capacité fonctionne comme la capacité Attaque de Phase, sauf que votre attaque perturbe également les éléments vitaux de l'ennemi, infligeant 5 points de dégâts supplémentaires. Facilitateur. (Enhanced Phased Attack \textendash (135))

\subsection*{Attaque de désarmement}\label{subsec:ab_disarming_attack}

vous tentez une tâche de Célérité pour désarmer un ennemi dans le cadre de votre attaque de mêlée. Si vous réussissez, votre attaque inflige 3 points de dégâts supplémentaires et l'arme de la cible est arrachée de son emprise et atterrit jusqu'à 20 pieds (6 m) de distance. Si vous échouez, vous tentez toujours votre attaque normale, mais vous n'infligez pas de dégâts supplémentaires ni ne désarmez l'adversaire si vous frappez. Action. (Disarming Attack \textendash (129))

\subsection*{Attaque en réponse}\label{subsec:ab_answering_attack}

Si vous êtes touché au corps à corps, vous pouvez effectuer une attaque de mêlée immédiate contre cet attaquant une fois par tour. L'attaque est gênée et vous pouvez toujours effectuer votre action normale pendant le tour. Facilitateur. (Answering Attack \textendash (110))

\subsection*{Attaque psychokinétique}\label{subsec:ab_psychokinetic_attack}

vous pouvez utiliser cette attaque de deux manières. La première consiste à ramasser un objet lourd et à le lancer sur quelqu'un à courte portée. Cette attaque est une action d'Intellect, et si elle réussit, elle inflige 6 points de dégâts à la cible et à l'objet lancé (qui pourrait être un autre ennemi, bien que cela nécessiterait deux jets : un pour attraper le premier ennemi et un autre pour l'attraper). frapper le deuxième ennemi avec le premier). La deuxième façon consiste à déclencher une explosion de Puissance fracassante qui ne fonctionne que contre un objet inanimé ne dépassant pas la moitié de votre taille. Effectuez un jet d'Intellect pour détruire instantanément l'objet ; la tâche est facilitée par trois étapes par rapport à la rupture avec la force brute. Action. (Psychokinetic Attack \textendash (172))

\subsection*{Attaque rapide}\label{subsec:ab_rapid_attack}

Une fois par tour, vous pouvez effectuer une attaque supplémentaire avec l'arme de votre choix. Facilitateur. (Rapid Attack \textendash (174))

\subsection*{Attaque sautée}\label{subsec:ab_jump_attack}

vous tentez un jet de Puissance de difficulté 4 pour sauter haut dans les airs dans le cadre de votre action d'attaque au corps à corps. Si vous réussissez le saut et que votre attaque touche, vous infligez 3 points de dégâts supplémentaires et mettez l'ennemi à terre. Si vous échouez au saut, vous effectuez toujours votre jet d'attaque normal, mais vous n'infligez pas de dégâts supplémentaires ni ne renversez l'adversaire si vous frappez. En plus des options normales d'utilisation de l'Effort, vous pouvez choisir d'utiliser l'Effort pour améliorer votre saut ; chaque niveau d'Effort utilisé de cette manière ajoute +2 pieds à la hauteur et +1 dégâts à l'attaque. Action. (Jump Attack \textendash (156))

\subsection*{Attaque successive}\label{subsec:ab_successive_attack}

Si vous éliminez un ennemi, vous pouvez immédiatement effectuer une autre attaque au cours du même tour contre un nouvel ennemi à votre portée. La deuxième attaque fait partie de la même action. Vous pouvez utiliser cette capacité avec des attaques de mêlée et des attaques à distance. Facilitateur. (Successive Attack \textendash (187))

\subsection*{Attaque surprise}\label{subsec:ab_surprise_attack}

si vous attaquez depuis un point de vue caché, avec surprise ou avant que votre adversaire n'agisse, vous obtenez un atout pour l'attaque. En cas de réussite, vous infligez 2 points de dégâts supplémentaires. Facilitateur. (Surprise Attack \textendash (188))

\subsection*{Attaque étourdissante}\label{subsec:ab_dazing_attack}

vous frappez votre ennemi juste au bon endroit, l'étourdissant de sorte que les tâches qu'il tentera lors de son prochain tour seront gênées. Cette attaque inflige des dégâts normaux. Action. (Dazing Attack \textendash (125))

\subsection*{Attaquer les yeux}\label{subsec:ab_eye_gouge}

Vous effectuez une attaque contre une créature avec un œil. L'attaque est gênée, mais si vous touchez, la créature a du mal à voir pendant l'heure suivante. Pendant ce temps, les tâches de la créature qui dépendent de la vue (qui constituent la plupart des tâches) sont entravées. Action. (Eye Gouge \textendash (138))

\subsection*{Attaquez et attaquez encore}\label{subsec:ab_attack_and_attack_again}

Plutôt que d'accorder des dégâts supplémentaires ou un effet mineur ou majeur, un 17 naturel ou plus sur votre jet d'attaque vous permet de lancer immédiatement une autre attaque. Facilitateur. (Attack and Attack Again \textendash (111))

\subsection*{Attitude de commandement}\label{subsec:ab_demeanor_of_command}

Vous projetez confiance, connaissances et charisme à tous ceux qui vous voient pendant l'heure suivante. Votre comportement est tel que ceux qui vous voient comprennent automatiquement que vous êtes quelqu'un d'important, accompli et doté d'autorité. Lorsque vous parlez, des étrangers qui n'attaquent pas déjà vous donnent au moins un round pour avoir votre mot à dire. Si vous parlez à un groupe qui peut vous comprendre, vous pouvez essayer de lui demander de présenter son chef ou de lui demander de vous emmener vers son chef. Vous gagnez un niveau d'Effort gratuit qui peut être appliqué à une tâche de persuasion que vous tentez pendant cette période. Action à initier. (Demeanor of Command \textendash (127))

\subsection*{Au diable les coupables}\label{subsec:ab_damn_the_guilty}

Vous prononcez des paroles de révélation et de jugement à tous ceux qui se trouvent à proximité. Ceux que vous avez désignés comme coupables grâce à votre capacité de Désignation subissent 3 points de dégâts supplémentaires pour toute attaque qu'ils reçoivent de la part de toute personne ayant entendu votre jugement. Ce jugement dure jusqu'à une minute ou jusqu'à ce qu'ils s'éloignent d'au moins une longue distance de vous. Action. (Damn the Guilty \textendash (124))

\subsection*{Augmentation Cypher}\label{subsec:ab_augment_cypher}

Lorsque vous activez un cypher, ajoutez +1 à son niveau. En plus des options normales d'utilisation de l'Effort, vous pouvez choisir d'utiliser l'Effort pour augmenter le niveau du cypher d'un +1 supplémentaire (par niveau d'Effort appliqué). Vous ne pouvez pas augmenter le niveau du cypher au-dessus de 10. Enabler. (Augment Cypher \textendash (111))

\subsection*{Augmentation de la portée}\label{subsec:ab_range_increase}

les portées augmentent d'un cran. L'immédiat devient court, le court devient long, le long devient très long et le très long devient 1 000 pieds (300 m). Facilitateur. (Range Increase \textendash (174))

\subsection*{Augmentation défensive}\label{subsec:ab_defensive_augmentation}

en améliorant vos systèmes nerveux et immunitaire, vous êtes entraîné aux tâches de défense de Puissance et de défense de Célérité. Facilitateur. (Defensive Augmentation \textendash (127))

\subsection*{Augmente les dommages}\label{subsec:ab_damage_dealer}

Vous infligez 3 points de dégâts supplémentaires avec l'arme de votre choix. Facilitateur. (Damage Dealer \textendash (124))

\subsection*{Aura fétide}\label{subsec:ab_foul_aura}

Vos mots, vos gestes et votre toucher investissent un objet pas plus grand que vous d'une aura de malheur, de peur et de doute pendant une journée. Les créatures capables de vous entendre et de vous comprendre ressentent le besoin de s'éloigner d'au moins une courte distance de l'objet. Si une créature ne s'éloigne pas, toutes les tâches, attaques et défenses qu'elle tente lorsqu'elle se trouve dans l'aura sont entravées. La durée de l'aura est prolongée d'un jour par niveau d'Effort appliqué. L'aura est temporairement bloquée pendant que l'objet est couvert ou confiné. Action à initier. (Foul Aura \textendash (143))

\subsection*{Aussi Rapide que l'Eclair}\label{subsec:ab_flash_across_the_miles}

Vous pouvez vous déplacer presque instantanément vers un endroit ouvert de la planète que vous connaissez, en étant transentraîné en éclair. Si vous appliquez un certain niveau d'effort, vous pouvez tenter de pénétrer dans des endroits couverts dont vous avez connaissance tant qu'il existe un itinéraire depuis l'air libre jusqu'à la zone que vous souhaitez atteindre et que l'électricité peut facilement suivre. Action. (Flash Across the Miles \textendash (141))

\subsection*{Autodocteur}\label{subsec:ab_autodoctor}

Vous êtes entraîné à la guérison, à la réalisation d'interventions chirurgicales et à la résistance à la douleur. Vous pouvez effectuer des opérations chirurgicales sur vous-même, en restant conscient pendant que vous le faites. Facilitateur. (Autodoctor \textendash (111))

\subsection*{Avantage d'Intellect Amélioré}\label{subsec:ab_enhanced_intellect_edge}

Vous gagnez +1 à votre Avantage d'Intellect. Facilitateur. (Enhanced Intellect Edge \textendash (135))

\subsection*{Avantage de Célérité Amélioré}\label{subsec:ab_enhanced_speed_edge}

Vous gagnez +1 à votre Avantage de Célérité. Facilitateur. (Enhanced Speed Edge \textendash (135))

\subsection*{Avantage de Puissance Amélioré}\label{subsec:ab_enhanced_might_edge}

vous gagnez +1 à votre Avantage de Puissance. Facilitateur. (Enhanced Might Edge \textendash (135))

\subsection*{Avantage de Stat Amélioré}\label{subsec:ab_improved_edge}

choisissez un de vos Avantage de statistique qui est de 0. Il passe à 1. Enabler. (Improved Edge \textendash (151))

\subsection*{Avantage par Désavantage}\label{subsec:ab_advantage_to_disadvantage}

Avec un certain nombre de mouvements rapides, vous lancez une attaque contre un ennemi armé, lui infligeant des dégâts et le désarmant de sorte que son arme soit maintenant entre vos mains ou à 10 pieds (3 m) au sol. - à vous de choisir. Cette attaque désarmante est entravée. Action. (Advantage to Disadvantage \textendash (109))

\subsection*{Avantages d'être grand}\label{subsec:ab_advantages_of_being_big}

lorsque vous utilisez Agrandir, vous êtes si grand que vous pouvez déplacer des objets massifs plus facilement, escalader des bâtiments en utilisant des poignées et des pieds inaccessibles aux personnes de taille normale et sauter beaucoup plus loin. Pendant que vous profitez des effets d'Agrandir, toutes les tâches d'escalade, de levage et de saut sont facilitées. Facilitateur. (Advantages of Being Big \textendash (109))

\subsection*{Avantages d'être petit}\label{subsec:ab_advantages_of_being_small}

Vous avez appris à exploiter votre force et votre précision proportionnellement à votre taille. Vos dégâts ne sont plus réduits de moitié lorsque vous utilisez Rétrécir, et les tâches d'escalade et de saut sont facilitées. Facilitateur. (Advantages of Being Small \textendash (109))

\subsection*{Avantages de la célébrité}\label{subsec:ab_perks_of_stardom}

Vous êtes habile à réclamer les récompenses que la renommée peut générer. Lorsque vous êtes reconnu, vous pouvez vous asseoir dans n'importe quel restaurant, être admis dans n'importe quel bâtiment gouvernemental, être invité à n'importe quel spectacle ou événement sportif (même s'ils sont complets), obtenir une place à une réception privée de toute sorte, ou entrez dans n'importe quel club, aussi exclusif soit-il. Face à quelqu'un qui ne peut ou ne veut pas céder immédiatement à votre désir, vous gagnez un atout sur toutes les tâches liées à la persuasion si cette personne vous reconnaît ou est convaincue que vous êtes une célébrité même si elle ne vous reconnaît pas. toi. Facilitateur. (Perks of Stardom \textendash (169))

\subsection*{Ayez une Combinaison Spatiale, Vous Voyagerez}\label{subsec:ab_have_spacesuit_will_travel}

D'une manière ou d'une autre, vous êtes devenu le propriétaire légal d'une combinaison spatiale entièrement fonctionnelle et avancée. La combinaison spatiale fournit +1 d'armure et, plus important encore, vous permet de survivre dans le vide de l'espace en utilisant les réserves de la combinaison jusqu'à douze heures à la fois avec suffisamment de masse de réaction pour vous déplacer en apesanteur sur des jets de gaz ionisé pendant la même période. Après chaque utilisation, la combinaison doit être rechargée, soit avec des cartouches d'air et de masse réactionnelle déjà chargées, soit en laissant la combinaison reposer inutilisée dans une zone à atmosphère respirable pendant au moins deux heures, période pendant laquelle elle rechargera à la fois l'air et masse de réaction utilisant des mécanismes à semi-conducteurs intégrés. L'alimentation électrique de la combinaison est un générateur thermoélectrique à radio-isotopes, ce qui signifie qu'elle fonctionnera pendant quelques décennies avant de devoir être remplacée. Facilitateur. (Have Spacesuit, Will Travel \textendash (148))

\
%--------------------------
\section*{B}

\subsection*{Babel}\label{subsec:ab_babel}

Après avoir entendu parler une langue pendant quelques minutes, on peut la parler et se faire comprendre. Si vous continuez à utiliser la langue pour interagir avec des locuteurs natifs, vos compétences s'améliorent rapidement, au point où vous pourriez être pris pour un locuteur natif après seulement quelques heures passées à parler la nouvelle langue. Facilitateur. (Babel \textendash (112))

\subsection*{Bande de Desperados}\label{subsec:ab_band_of_desperados}

Votre réputation attire une bande de six adeptes PNJ desperados de niveau 2 qui vous sont entièrement dévoués. Vous et le MJ devez régler les détails de ces adeptes. Si un adepte meurt, vous en gagnez un nouveau après au moins deux semaines et un recrutement approprié. Facilitateur. (Band of Desperados \textendash (112))

\subsection*{Bande de suivants}\label{subsec:ab_band_of_followers}

vous gagnez quatre suiveurs de niveau 3. Ils ne sont pas limités sur leurs modifications. Facilitateur. (Band of Followers \textendash (112))

\subsection*{Barrière de champ de force}\label{subsec:ab_force_field_barrier}

Vous créez une barrière opaque et stationnaire d'énergie solide (un champ de force) à portée immédiate. La barrière mesure 10 pieds sur 10 pieds (3 m sur 3 m) et a une épaisseur négligeable. C'est une barrière de niveau 2 et dure dix minutes. Il peut être placé n'importe où, que ce soit contre un objet solide (y compris le sol) ou flottant dans les airs. Chaque niveau d'effort que vous appliquez renforce la barrière d'un niveau. Par exemple, appliquer deux niveaux d'Effort crée une barrière de niveau 4. Action. (Force Field Barrier \textendash (143))

\subsection*{Barrière de conversion sonore}\label{subsec:ab_sound_conversion_barrier}

les attaques qui vous frappent, en particulier les attaques énergétiques telles que la lumière focalisée, la chaleur, les radiations et l'énergie transdimensionnelle, sont partiellement converties en vagues de bruit inoffensif semblable au bruit d'une vague s'écrasant sur le rivage. Cette capacité vous accorde +1 d'armure contre toutes les attaques et +2 d'armure supplémentaire contre les attaques énergétiques. Facilitateur. (Sound Conversion Barrier \textendash (184))

\subsection*{Blessure de téléportation}\label{subsec:ab_teleportive_wound}

Vous touchez une créature et, si votre attaque réussit, vous vous téléportez (jusqu'à votre distance de téléportation maximale normale) avec une partie importante de son corps. Si la cible est de niveau 2 ou inférieur, elle meurt. Si la cible est de niveau 3 ou supérieur, elle subit 6 points de dégâts et est étourdie lors de sa prochaine action. Si la cible est un PJ de n'importe quel niveau, il descend d'un cran sur la piste des dégâts. En plus des options normales d'utilisation de l'Effort, vous pouvez choisir d'utiliser l'Effort pour affecter une cible plus puissante (un niveau d'Effort signifie qu'une cible jusqu'au niveau 3 meurt ou qu'une cible de niveau 4 ou plus subit des dégâts et est étourdie, et ainsi de suite). Action. (Teleportive Wound \textendash (190))

\subsection*{Blocage rapide}\label{subsec:ab_quick_block}

si vous utilisez une arme légère ou moyenne, vous êtes entraîné aux tâches de défense rapide. Facilitateur. (Quick Block \textendash (173))

\subsection*{Bloquer}\label{subsec:ab_block}

Vous bloquez automatiquement la prochaine attaque de mêlée lancée contre vous dans la minute suivante. Action à initier. (Block \textendash (115))

\subsection*{Bloquer pour un autre}\label{subsec:ab_block_for_another}

si vous utilisez une arme légère ou moyenne, vous pouvez bloquer les attaques lancées contre un allié proche de vous. Choisissez une créature à portée immédiate. Vous fournissez un atout aux tâches de défense de Célérité de cette créature. Vous ne pouvez pas utiliser Quick Bloquer lorsque vous utilisez Bloquer for Another. Facilitateur. (Block for Another \textendash (115))

\subsection*{Blâme détourné}\label{subsec:ab_misdirect_blame}

En utilisant des mots biens choisis et votre connaissance des autres, vous pouvez tenter de modifier le récit de sorte qu'une cible jusqu'au niveau 3 à courte portée devienne incertaine de sa conviction dans un domaine simple, comme leur la conviction que vous venez de voler un fruit sur leur stand ou la conviction qu'ils ne vous ont jamais rencontré auparavant. Cet effet ne dure généralement que le temps que vous passez à parler, et peut-être jusqu'à une minute de plus, avant que la cible ne se rende compte de son erreur. En plus des options normales d'utilisation de l'Effort, vous pouvez choisir d'utiliser l'Effort pour augmenter le niveau cible qui peut être affecté. Par la suite, toutes vos tâches visant à persuader ou à interagir socialement avec la cible sont entravées. Action. (Misdirect Blame \textendash (163))

\subsection*{Bon conseil}\label{subsec:ab_good_advice}

n'importe qui peut aider un allié, en lui facilitant la tâche qu'il entreprend. Cependant, vous bénéficiez de clarté et de sagesse. Lorsque vous aidez un autre personnage, il gagne un atout supplémentaire. Facilitateur. (Good Advice \textendash (145))

\subsection*{Booster la fonction de cypher manifeste}\label{subsec:ab_boost_manifest_cypher_function}

ajoutez 3 au niveau de fonctionnement d'un cypher manifeste que vous activez lors de votre prochaine action, ou modifiez un aspect de ses paramètres (portée, durée, zone, etc.) jusqu'à doubler ou jusqu'à un dixième. Action. (Boost Manifest Cypher Function \textendash (CTS, 51))

\subsection*{Booster un Cypher Manifeste}\label{subsec:ab_boost_manifest_cypher}

Le cypher manifeste que vous activez avec votre prochaine action fonctionne comme s'il était 2 niveaux plus haut. Action. (Boost Manifest Cypher \textendash (CTS, 51))

\subsection*{Boucle temporelle}\label{subsec:ab_time_loop}

Vous vous appelez vous-même à partir de quelques instants dans le futur pour vous aider dans le présent. Lors du round où vous utilisez cette capacité, votre futur moi apparaît n'importe où, à portée immédiate, et effectue une action. Au deuxième tour, vous et votre futur moi agissez tous les deux, et l'action de votre futur moi est facilitée. Au troisième tour, vous et votre futur moi disparaissez tous les deux. Au quatrième tour, vous rattrapez votre futur moi, réapparaissez là où votre futur moi est apparu initialement au premier tour et pouvez entreprendre vos actions normalement. Votre futur moi partage vos statistiques, donc tous les dégâts que l'un de vous subit s'appliquent aux mêmes pools de statistiques. Si votre futur moi est tué, vous et votre futur moi disparaissez au troisième tour (comme d'habitude) et vous réapparaissez, mort, au quatrième tour. Ni vous ni votre futur moi ne pouvez à nouveau utiliser Boucle temporelle jusqu'à ce que vous réapparaissiez en tant que futur moi au quatrième tour. Action. En effet, Boucle temporelle vous permet d'effectuer une action lors du tour où vous l'utilisez, deux actions lors du deuxième tour et zéro action lors du troisième tour, puis vous revenez à la normale après cela. (Time Loop \textendash (192))

\subsection*{Bouclier Explosif}\label{subsec:ab_shield_burst}

Lorsque vous effectuez une attaque au corps à corps ou à distance et que vous frappez avec votre Champ de force Shield, celui-ci libère une explosion d'énergie, infligeant 2 points de dégâts supplémentaires à la cible et à tout ce qui se trouve à portée immédiate de la cible. Si vous avez appliqué Effort pour infliger des dégâts supplémentaires dans le cadre de l'attaque, chaque niveau d'Effort n'inflige que 2 points supplémentaires à toutes les cibles au lieu de 3 points. Si vous utilisez Bouclier Explosif avec une attaque de mêlée, vous et les créatures derrière vous n'êtes pas affectés par cette explosion. Si vous utilisez Bouclier Explosif avec une attaque à distance, le bouclier se dissipe après l'attaque puis se reforme entre vos mains. Facilitateur. (Shield Burst \textendash (182))

\subsection*{Bouclier Magique}\label{subsec:ab_magic_shield}

Vous gagnez +1 en Armure pendant une heure. Action à initier. (Magic Shield \textendash (159))

\subsection*{Bouclier de Champ de Force}\label{subsec:ab_force_field_shield}

vous manifestez un petit plan de force pure, qui prend la forme d'un bouclier avec le moindre scintillement d'une pensée. Vous pouvez le rejeter tout aussi facilement. Pour utiliser le bouclier de force, vous devez le tenir dans une de vos mains. Vous êtes habitué à utiliser votre bouclier exotique dans une main comme arme légère de mêlée ; cependant, si vous attaquez à la fois avec votre bouclier et une arme tenue dans l'autre main, les deux attaques sont gênées. Lorsque vous êtes inconscient ou endormi, le champ de force se dissipe. Facilitateur. (Un bouclier, y compris un bouclier produit par un champ de force, constitue un atout pour la tâche de défense rapide d'un personnage lorsqu'il est tenu dans une main.) (Force Field Shield \textendash (143))

\subsection*{Bouclier de protection}\label{subsec:ab_warding_shield}

vous avez +1 à l'armure lorsque vous utilisez un bouclier. Facilitateur. (Warding Shield \textendash (196))

\subsection*{Bouclier enveloppant}\label{subsec:ab_enveloping_shield}

votre capacité Champ de force Shield produit une enveloppe de force qui vous enveloppe pendant que vous tenez le bouclier, vous accordant +1 en armure. Facilitateur. (Enveloping Shield \textendash (136))

\subsection*{Bouclier rebondissant}\label{subsec:ab_bouncing_shield}

lorsque vous utilisez Lancer Force Shield, au lieu de se dissiper après une attaque (qu'elle touche ou rate), il attaquera jusqu'à deux cibles supplémentaires à courte portée. L'effort ou d'autres modificateurs appliqués à la première attaque affectent également toutes les autres cibles. Que vous touchiez toutes, certaines ou aucune de vos cibles, le bouclier se dissipe puis se reforme à votre portée. (Si vous choisissez Bouclier rebondissant et que vous avez déjà utilisé la capacité Lancer Force Shield, vous avez la possibilité d'échanger cette capacité contre Healing Pulse.) Enabler. (Bouncing Shield \textendash (115))

\subsection*{Bouclier énergisé}\label{subsec:ab_energized_shield}

votre bouclier de force issu de votre capacité Champ de force Shield émet désormais une énergie dangereuse chaque fois que vous la manifestez. Chaque fois que vous utilisez votre bouclier comme arme de mêlée ou à distance, il inflige 3 points de dégâts supplémentaires. Facilitateur. (Energized Shield \textendash (134))

\subsection*{Bouquet d'évasion}\label{subsec:ab_burst_of_escape}

vous pouvez effectuer deux actions distinctes ce tour-ci, à condition que l'une d'elles consiste à vous cacher ou à vous déplacer dans une direction qui n'est pas vers un ennemi. Facilitateur. (Burst of Escape \textendash (116))

\subsection*{Bricoleur}\label{subsec:ab_tinker}

Vous faites faire à un appareil quelque chose de différent de son objectif initial. Par exemple, un blaster devient une bombe. Un scanner devient un amplificateur de signal pour un émetteur radio. Un lecteur de musique devient une batterie pour un autre appareil. Le niveau effectif de l'appareil modifié est inférieur de 1 à la normale et l'appareil devient inutilisable (pour son objectif initial) jusqu'à ce qu'il soit réparé. Action à initier. (Tinker \textendash (192))

\subsection*{Bricoleur d'artefacts}\label{subsec:ab_artifact_tinkerer}

si vous passez au moins une journée à bricoler un artefact en votre possession, celui-ci fonctionne à un niveau supérieur à la normale. Cela s'applique à tous les artefacts en votre possession, mais ils ne conservent ce bonus que pour vous. Facilitateur. (Artifact Tinkerer \textendash (110))

\subsection*{Brise Esprit}\label{subsec:ab_shatter_mind}

Vos mots résonnent de manière destructrice dans le cerveau d'une cible intelligente de niveau 1 à courte portée qui peut vous entendre et vous comprendre. Ils détruisent les tissus, les souvenirs et la personnalité, déclenchant un état végétatif. En plus des options normales d'utilisation de l'Effort, vous pouvez choisir d'utiliser l'Effort pour augmenter le niveau maximum de la cible. Ainsi, pour briser l'esprit d'une cible de niveau 5 (quatre niveaux au-dessus de la limite normale), vous devez appliquer quatre niveaux d'Effort. Action. (L'état végétatif créé par Brise Esprit peut être guéri par la magie ou la science avancée, ou par un chiffre de suppression de conditions qui guérit la psychose.) (Shatter Mind \textendash (182))

\subsection*{Briser}\label{subsec:ab_shatter}

Vous interrompez la force fondamentale qui maintient la matière normale ensemble pendant un moment, provoquant la détonation d'un objet que vous choisissez à longue portée. L'objet doit être un petit objet banal composé de matière homogène (comme une coupe d'argile, un lingot de fer, une pierre, etc.). L'objet explose dans un rayon immédiat, infligeant 1 point de dégâts à toutes les créatures et objets dans la zone. Si vous appliquez Effort pour augmenter les dégâts, vous infligez 2 points de dégâts supplémentaires par niveau d'Effort (au lieu de 3 points) ; les cibles dans la zone subissent 1 point de dégâts même si vous échouez au jet d'attaque. Action. (Shatter \textendash (182))

\subsection*{Briser la ligne}\label{subsec:ab_break_the_line}

Vous avez le sens de la discipline de groupe et de la hiérarchie, même parmi vos ennemis. Si un groupe d'ennemis tire un quelconque avantage de sa collaboration, vous pouvez tenter de perturber cet avantage en soulignant le point faible de la ligne, de la formation ou de l'attaque en masse de l'e nnemi. Cet effet dure jusqu'à une minute ou jusqu'à ce que tous les ennemis affectés passent un tour à s'évaluer et à se réinitialiser pour retrouver leur avantage normal. Action à initier. (Break the line \textendash (116))

\subsection*{Briser les rangs}\label{subsec:ab_break_the_ranks}

vous vous déplacez sur une courte distance et attaquez jusqu'à quatre ennemis différents en une seule action tant qu'ils se trouvent tout au long de votre chemin. Tous les modificateurs qui s'appliquent à une attaque s'appliquent à toutes les attaques que vous effectuez. Si vous disposez d'une autre capacité spéciale qui vous permet de vous déplacer et d'effectuer une action, lorsque vous utilisez Break the Ranks, vous gagnez un atout pour attaquer ces ennemis. Action. (Break the ranks \textendash (116))

\subsection*{Briser leur esprit}\label{subsec:ab_break_their_mind}

en utilisant vos mots intelligents et votre connaissance des autres, et en vous donnant quelques tours de conversation pour obtenir quelques éléments de contexte spécifiques concernant votre cible, vous pouvez prononcer une phrase conçue pour provoquer immédiatement votre cible. la détresse psychologique. Si la cible peut vous entendre et vous comprendre, elle subit 6 points de dégâts d'Intellect (ignore l'armure) et oublie le dernier jour de sa vie, ce qui pourrait signifier qu'elle vous oublie et comment elle en est arrivée là où elle se trouve actuellement. En plus des options normales d'utilisation de l'Effort, vous pouvez choisir d'utiliser l'Effort pour tenter de briser l'esprit d'une cible supplémentaire qui peut vous entendre et vous comprendre. Action à initier, action à terminer. (Break their Mind \textendash (116))

\subsection*{Briseur de Pierre}\label{subsec:ab_stone_breaker}

Vos attaques contre des objets infligent 4 points de dégâts supplémentaires lorsque vous utilisez une arme de mêlée que vous maniez à deux mains. Facilitateur. (Stone Breaker \textendash (186))

\subsection*{Brouiller une Machine}\label{subsec:ab_scramble_machine}

Vous rendez une machine à courte portée incapable de fonctionner pendant un tour. Alternativement, vous pouvez empêcher toute action de la machine (ou de quelqu'un essayant d'utiliser la machine) pendant une minute. Action. (Scramble Machine \textendash (179))

\subsection*{Bénédiction des Dieux}\label{subsec:ab_blessing_of_the_gods}

En tant que serviteur des dieux, vous pouvez invoquer des bénédictions en leur nom. Cette bénédiction dépend du comportement général du dieu et de sa zone d'influence. Choisissez deux des capacités décrites ci-dessous.
\begin{description}
\item[Autorité/Loi/Paix] (3 points d'Intellect). Vous empêchez un ennemi capable de vous entendre et de vous comprendre d'attaquer qui que ce soit ou quoi que ce soit pendant un round. Action.
\item[Bienveillance/Justice/Esprit] (2+ points d'Intellect). Un démon, un esprit ou une créature similaire de niveau 1 à courte portée est détruit ou banni. En plus des options normales d'utilisation de l'Effort, vous pouvez choisir d'utiliser l'Effort pour augmenter le niveau maximum de la cible. Ainsi, pour détruire ou bannir une cible de niveau 5 (quatre niveaux au-dessus de la limite normale), vous devez appliquer quatre niveaux d'Effort. Action.
\item[Mort/Ténèbres] (2 points d'Intellect). Une cible que vous choisissez à courte portée se flétrit et subit 3 points de dégâts. Action.
%\item[Désir\/Amour\/Santé] (3 points d'Intellect). D'un simple contact, vous restaurez 1d6 points à une réserve de statistiques de n'importe quelle créature, y compris vous-même. Cette capacité est une tâche de difficulté Intellect 2. Chaque fois que vous tentez de soigner la même créature, la tâche est gênée par une étape supplémentaire. La difficulté revient à 2 après que cette créature se repose pendant dix heures. Action.
\item[Terre/Pierre]. Vous êtes entraîné à l'escalade, à la fabrication de pierres et à la spéléologie. Facilitateur.
%\item[Connaissance\/Sagesse] (3 points d'Intellect). Choisissez jusqu'à trois créatures (y compris potentiellement vous-même). Pendant une minute, un type particulier de tâche (mais pas un jet d'attaque ou de défense) est facilité pour ces créatures, mais uniquement tant qu'elles restent à portée immédiate de vous. Action.
\item[Nature/Animaux/Plantes]. Vous êtes entraîné à la botanique et à la manipulation des animaux naturels. Facilitateur.
%\item[Protection\/Silence] (3 points d'Intellect). Vous créez une bulle de protection silencieuse autour de vous dans un rayon immédiat pendant une minute. La bulle bouge avec vous. Tous les jets de défense pour vous et pour toutes les créatures que vous désignez dans la bulle sont facilités, et aucun bruit, quelle que soit son origine, n'est plus fort qu'une voix normale. Action à initier.
%\item[Ciel\/Air] (2 points d'Intellect). Une créature que vous touchez est immunisée contre les toxines ou les contaminants en suspension dans l'air pendant dix minutes. Action.
%\item[Soleil\/Lumière\/Feu] (2 points d'Intellect). Vous faites prendre feu à une créature ou à un objet à courte portée, infligeant 1 point de dégâts ambiants à chaque tour jusqu'à ce que le feu soit éteint (nécessitant une action). Action.
%\item[Supercherie\/Cupidité\/Commerce]. Vous êtes entraîné à détecter les tromperies d'autres créatures. Facilitateur.
\item[Guerre] (1 point d'Intellect). Une cible que vous choisissez à courte portée (potentiellement vous-même) inflige 2 points de dégâts supplémentaires lors de sa prochaine attaque d'arme réussie. Action.
\item[Eau/Mer] (2 points d'Intellect). Une cible que vous touchez peut respirer de l'eau pendant dix minutes. Action. (Blessing of the Gods \textendash (114))
\end{description}

\
%--------------------------
\section*{C}

\subsection*{Calme}\label{subsec:ab_calm}

Grâce à des blagues, des chansons ou d'autres arts, vous empêchez un ennemi vivant d'attaquer qui que ce soit ou quoi que ce soit pendant un round. Action. (Calm \textendash (118))

\subsection*{Calmer un Etranger}\label{subsec:ab_calm_stranger}

Vous pouvez faire en sorte qu'une créature intelligente reste calme pendant que vous parlez. La créature n'a pas besoin de parler votre langue, mais elle doit pouvoir vous voir. Il reste calme tant que vous concentrez toute votre attention sur lui et qu'il n'est pas attaqué ou menacé. En plus des options normales d'utilisation de l'Effort, vous pouvez choisir d'utiliser l'Effort pour calmer des créatures supplémentaires alliées à votre cible initiale, une créature supplémentaire par niveau d'Effort appliqué. Action. (Calm Stranger \textendash (118))

\subsection*{Camouflage sauvage}\label{subsec:ab_wild_camouflage}

En dessinant vos vêtements autour de vous et en utilisant diverses astuces et votre connaissance approfondie de votre environnement, vous devenez invisible dans la nature pendant dix minutes. Pendant que vous êtes invisible, cet atout facilite vos tâches de furtivité et de défense rapide de deux étapes. Cet effet prend fin si vous faites quelque chose pour révéler votre présence ou votre position : attaquer, utiliser une capacité, déplacer un objet volumineux, etc. Si cela se produit, vous pouvez retrouver l'effet d'invisibilité restant en prenant une mesure pour vous concentrer sur la dissimulation de votre position. Action à initier ou à relancer. (Wild Camouflage \textendash (198))

\subsection*{Capteur}\label{subsec:ab_sensor}

Vous créez un capteur immobile et invisible à portée immédiate qui dure 24 heures. À tout moment pendant cette durée, vous pouvez vous concentrer pour voir, entendre et sentir à travers le capteur, quelle que soit la distance à laquelle vous vous éloignez. Le capteur ne vous accorde pas de capacités sensorielles au-delà de la norme. Si vous disposez également de cette capacité provenant d'une autre source, elle dure deux fois plus longtemps. Action de créer ; action à vérifier. (Sensor \textendash (181))

\subsection*{Capteur amélioré}\label{subsec:ab_improved_sensor}

lorsque vous utilisez le capteur, vous pouvez placer le capteur n'importe où à longue portée. Facilitateur. (Improved Sensor \textendash (152))

\subsection*{Captiver avec Eclat}\label{subsec:ab_captivate_with_starshine}

Tant que vous parlez, vous gardez l'attention de tous les PNJ de niveau 2 ou inférieur qui peuvent vous entendre. Si vous possédez également la capacité Envoûtement, vous pouvez de la même manière captiver tous les PNJ de niveau 3. Action à initier. (Captivate With Starshine \textendash (118))

\subsection*{Captiver ou Inspirer}\label{subsec:ab_captivate_or_inspire}

vous pouvez utiliser cette capacité de deux manières. Soit vos paroles retiennent l'attention de tous les PNJ qui les entendent aussi longtemps que vous parlez, soit vos paroles incitent tous les PNJ qui les entendent à fonctionner comme s'ils étaient d'un niveau supérieur pendant l'heure suivante. Dans les deux cas, vous choisissez quels PNJ sont affectés. Si quelqu'un dans la foule est attaqué pendant que vous essayez de lui parler, vous perdez l'attention de la foule. Action à initier. (Captivate or Inspire \textendash (118))

\subsection*{Carnivore sous la Lune}\label{subsec:ab_moon_shape}

Vous vous transformez en un animal carnivore, comme un loup, un ours ou une autre créature terrestre, pendant une heure maximum. Si vous essayez de changer pendant la journée lorsque vous n'êtes pas profondément sous terre (ou loin de la lumière du jour), vous devez appliquer un niveau d'effort. Dans votre nouvelle forme, vous ajoutez 8 points à votre Réserve de Puissance, gagnez +2 à votre Avantage de Puissance, ajoutez 2 points à votre Speed Pool et gagnez +2 à votre Speed Edge. Revenir à votre forme normale est une tâche de difficulté 2. Sous forme de bête, vous êtes sujet à des crises de rage (déclenchées par l'intrusion du MJ), au cours desquelles vous attaquez toutes les créatures vivantes à courte portée, et la seule façon de mettre fin à la rage est de revenir à votre forme normale. Quoi qu'il en soit, une fois revenu à votre forme normale, vous subissez une pénalité de -1 à tous les jets pendant une heure. Si vous n'avez pas tué et mangé au moins une créature substantielle alors que vous étiez sous forme de bête, la pénalité passe à -2 et affecte tous vos jets du jour suivant. Action pour changer ; action pour revenir en arrière. (Moon Shape \textendash (164))

\subsection*{Casseur}\label{subsec:ab_breaker}

Vous êtes entraîné aux tâches liées à l'endommagement d'objets dans le but de les casser, de les percer ou de les démolir. C'est une action de Puissance pour endommager un objet, et en cas de succès, l'objet descend d'un cran sur la piste de dégâts de l'objet. Si le jet de Puissance dépasse la difficulté de deux crans, l'objet se déplace de deux crans vers le bas de la piste de dégâts de l'objet. Si le jet de Puissance dépasse la difficulté de quatre niveaux, l'objet descend de trois niveaux sur la piste de dégâts de l'objet et est immédiatement détruit. Un matériau fragile réduit le niveau effectif de l'objet, tandis qu'un matériau dur comme le bois ou la pierre ajoute 1 au niveau effectif ou 2 pour les objets très durs comme ceux en métal. Facilitateur. (Breaker \textendash (116))

\subsection*{Cauchemar}\label{subsec:ab_nightmare}

vous attirez une créature horrible de votre pire cauchemar dans le monde éveillé et la envoyez à vos ennemis. Le cauchemar (niveau 5) persiste à chaque round pendant que vous utilisez votre action en vous concentrant sur lui (ou jusqu'à ce que vous le dispersiez ou qu'il soit détruit). Il possède l'une des capacités suivantes, que vous choisissez lorsque vous l'appelez.- Confusion. Au lieu d'effectuer une attaque normale, l'attaque du cauchemar confond la cible pendant un round. Lors de sa prochaine action, la cible attaque un allié. - Horrifier. Au lieu de faire une attaque normale, l'attaque du cauchemar horrifie la cible, qui tombe à genoux (ou des appendices similaires). La cible subit 3 points de dégâts qui ignorent l'Armure et reste étourdie pendant un round, durant lequel toutes ses tâches sont gênées. - Eruption de pustules. Au lieu de lancer une attaque normale, l'attaque du cauchemar provoque l'apparition de pustules rances et douloureuses sur toute la peau de la cible pendant une minute. Si la cible effectue une action violente (comme attaquer une autre créature ou se déplacer au-delà d'une distance immédiate), les pustules éclatent, infligeant 5 points de dégâts qui ignorent l'armure. Action pour initier, action à chaque tour pour se concentrer. (Nightmare \textendash (165))

\subsection*{Cavalier}\label{subsec:ab_rider}

Vous êtes entraîné à monter tout type de créature servant de monture, comme un noble cheval de guerre. Facilitateur. (Rider \textendash (178))

\subsection*{Centre d'attention}\label{subsec:ab_center_of_attention}

Un faisceau littéral (ou métaphorique, selon le genre) de pur rayonnement descend d'en haut et vous met en lumière. Toutes les créatures que vous choisissez à portée immédiate tombent à genoux et perdent leur prochaine action. Les cibles affectées ne peuvent pas se défendre et sont traitées comme impuissantes. Action. (Center of Attention \textendash (119))

\subsection*{Chair de Pierre}\label{subsec:ab_flesh_of_stone}

Vous avez +1 à l'Armure si vous ne portez pas d'armure physique. Facilitateur. (Flesh of Stone \textendash (141))

\subsection*{Chair de guerre}\label{subsec:ab_war_flesh}

vous pouvez instantanément transformer vos mains et vos pieds en griffes et vos dents humaines en crocs, ou retrouver votre apparence humaine normale. Lorsque vous effectuez des attaques avec vos griffes ou vos crocs, elles comptent comme des armes moyennes au lieu d'armes légères. Facilitateur. (War Flesh \textendash (196))

\subsection*{Chambre des rêves}\label{subsec:ab_chamber_of_dreams}

Vous et vos alliés pouvez entrer dans une chambre des rêves, décorée à votre guise, qui contient un certain nombre de portes. Les portes correspondent à d'autres endroits que vous avez visités ou que vous connaissez assez bien. Franchir l'une des portes vous amène à l'endroit souhaité. Il s'agit d'une tâche de difficulté 2 basée sur l'Intellect (qui pourrait être modifiée à la hausse par le MJ si l'emplacement est protégé). Action d'entrer dans la chambre des rêves ; action de franchir une porte dans la chambre. (Chamber of Dreams \textendash (119))

\subsection*{Champ de destruction}\label{subsec:ab_field_of_destruction}

Lorsque vous faites descendre un objet d'un ou plusieurs niveaux sur le suivi des dégâts d'objet, vous gagnez 1 point d'Armure supplémentaire pendant une minute. Facilitateur. (Field of Destruction \textendash (139))

\subsection*{Champ de force}\label{subsec:ab_force_field}

Vous créez une barrière énergétique invisible autour d'une créature ou d'un objet que vous choisissez à courte portée. Le champ de force se déplace avec la créature ou l'objet et dure dix minutes. Si la cible est une créature, elle gagne +1 en Armure ; si la cible est un objet, les attaques contre celui-ci sont gênées. Action. (Force Field \textendash (143))

\subsection*{Champ de gravité}\label{subsec:ab_field_of_gravity}

Lorsque vous le souhaitez, un champ de gravité manipulé autour de vous attire les attaques de projectiles à distance vers le sol. Vous êtes immunisé contre de telles attaques jusqu'à votre tour au tour suivant. Vous devez être conscient d'une attaque pour la déjouer. Cette capacité ne fonctionne pas sur les attaques énergétiques. Facilitateur. (Field of Gravity \textendash (139))

\subsection*{Champ de renforcement}\label{subsec:ab_reinforcing_field}

Vous pouvez renforcer n'importe quel objet ou structure en lui infusant un champ de force pendant une heure. Le champ de force augmente le niveau de l'objet ou de la structure de 2 pour les tâches liées à la durabilité et à la résistance aux dommages et à la destruction. Action à initier. (Reinforcing Field \textendash (175))

\subsection*{Champ de résonance}\label{subsec:ab_resonance_field}

des lignes pâles d'une couleur que vous choisissez forment un entrelacs sur tout votre corps et émettent une faible lumière. L'effet dure une minute. Chaque fois qu'une créature à portée immédiate effectue une attaque contre vous, le motif se dynamise pour bloquer l'attaque. Vous pouvez effectuer un jet de défense Intellect à la place du jet de défense que vous feriez normalement. Si vous le faites et que vous obtenez un effet mineur, la créature qui vous attaque subit 1 point de dégâts. Si vous obtenez un effet majeur, la créature qui vous attaque subit 4 points de dégâts. Action à initier. (Resonance Field \textendash (176))

\subsection*{Champ défensif}\label{subsec:ab_defensive_field}

Grâce à des implants sous-cutanés, un sort permanent, des modifications extraterrestres ou quelque chose de similaire, vous disposez désormais d'un champ de force qui rayonne à 1 pouce (2,5 cm) de votre corps et vous fournit +2 d'armure. Facilitateur. (Defensive Field \textendash (127))

\subsection*{Champ magnétique}\label{subsec:ab_magnetic_field}

Lorsque vous le souhaitez, un champ de magnétisme autour de vous attire les attaques de projectiles métalliques à distance (telles que des flèches, des balles, un couteau en métal lancé, etc.) vers le sol. Vous êtes immunisé contre de telles attaques pendant un round. Vous devez être conscient d'une attaque pour la déjouer. Facilitateur. (Magnetic Field \textendash (159))

\subsection*{Champ réactif}\label{subsec:ab_reactive_field}

Grâce à une amélioration remarquable de la science, de la magie, des psioniques ou de quelque chose d'encore plus étrange, vous disposez désormais d'un champ de force qui rayonne à 1 pouce (2,5 cm) de votre corps et vous fournit +2 à l'armure. De plus, s'il est touché par une attaque au corps à corps, le champ crée un contrecoup qui inflige 4 points de dégâts électriques à l'attaquant. Facilitateur. (Reactive Field \textendash (174))

\subsection*{Chance Etonnante}\label{subsec:ab_uncanny_luck}

lorsque vous lancez un dé pour une tâche et que vous réussissez, lancez à nouveau. Si le deuxième chiffre obtenu est supérieur au premier, vous obtenez un effet mineur. Si vous lancez à nouveau le même chiffre, vous obtenez un effet majeur. Si vous disposez d'une chance étrange provenant d'une autre source ou d'une capacité similaire, c'est à vous de choisir (aucun jet requis) si vous obtenez un effet mineur, un effet majeur ou une activation gratuite de l'une de vos capacités de concentration de niveau 1 à 3. Facilitateur. (Uncanny Luck \textendash (194))

\subsection*{Chanceux-Malchanceux}\label{subsec:ab_gambler}

Chaque jour, choisissez deux numéros différents entre 2 et 16. Un numéro est votre numéro porte-bonheur et l'autre est votre numéro malchanceux. Chaque fois que vous lancez un jet ce jour-là et obtenez un numéro correspondant à votre numéro porte-bonheur, votre prochaine tâche est facilitée. Chaque fois que vous lancez un jet ce jour-là et obtenez un numéro correspondant à votre numéro malchanceux, votre prochaine tâche est entravée. Facilitateur. (Gambler \textendash (144))

\subsection*{Changement contrôlé Supérieur}\label{subsec:ab_greater_controlled_change}

il est plus facile d'entrer et de sortir de la forme accordée par votre capacité Beast Form. Se transformer dans un sens ou dans l'autre est désormais une tâche de difficulté intellectuelle 2. Facilitateur. (Greater Controlled Change \textendash (146))

\subsection*{Changement d'Identité}\label{subsec:ab_spin_identity}

Vous convainquez toutes les créatures intelligentes qui peuvent vous voir, vous entendre et vous comprendre que vous êtes quelqu'un ou quelque chose d'autre que ce que vous êtes réellement. Vous ne vous faites pas passer pour une personne spécifique connue de la victime. Au lieu de cela, vous convainquez la victime que vous êtes quelqu'un qu'elle ne connaît pas et qui appartient à une certaine catégorie de personnes. "Nous sommes du gouvernement." "Je ne suis qu'un simple agriculteur de la ville voisine." "Votre commandant m'a envoyé." Un déguisement n'est pas nécessaire, mais un bon déguisement sera presque certainement un atout pour le rôle impliqué. Si vous tentez de convaincre plusieurs créatures, le coût en Intellect augmente de 1 point par victime supplémentaire. Les créatures trompées le restent jusqu'à une heure, à moins que vos actions ou d'autres circonstances ne révèlent votre véritable identité plus tôt. Action. (Spin Identity \textendash (185))

\subsection*{Changement de Visage}\label{subsec:ab_face_morph}

Vous modifiez vos traits et votre coloration pendant une heure, cachant votre identité ou vous faisant passer pour quelqu'un. Cela n'affecte que votre visage, pas le reste de votre corps. Vous ne pouvez pas reproduire parfaitement le visage de quelqu'un d'autre, mais vous pouvez être suffisamment précis pour tromper quelqu'un qui connaît un peu cette personne. Vous disposez d'un atout dans toutes les tâches impliquant un déguisement. Vous devez appliquer un niveau d'effort pour pouvoir usurper l'identité d'une espèce différente (comme un humain se transformant en extraterrestre humanoïde). Action. (Face Morph \textendash (138))

\subsection*{Changements Rapides}\label{subsec:ab_quick_switch}

Vous pouvez activer Rétrécir dans le cadre d'une autre action (la capacité est désormais un Facilitateur pour vous au lieu d'une action). Pendant que la durée d'une minute de Rétrécissement est active, pendant votre tour, vous pouvez changer de taille une fois avant d'effectuer une action et une fois après avoir effectué une action. Par exemple, pendant votre tour, vous pouvez passer à une taille petite, effectuer une attaque, puis revenir à votre taille normale, ou vous pouvez revenir à votre taille normale, utiliser votre action pour vous déplacer sur une courte distance, puis revenir à une taille petite. Facilitateur. (Quick Switch \textendash (174))

\subsection*{Changer le paradigme}\label{subsec:ab_change_the_paradygm}

Vous influencez la vision du monde d'une créature à qui vous passez au moins un tour à parler, à condition qu'elle puisse vous comprendre. La créature change d'avis sur une croyance importante, qui pourrait inclure quelque chose d'aussi simple que de vous aider au lieu d'essayer de vous tuer, ou cela pourrait être quelque chose de plus ésotérique. Cet effet dure au moins dix minutes, mais il peut durer plus longtemps si la créature n'était pas auparavant votre ennemi. Pendant ce temps, la créature agit conentraînément à la sagesse que vous lui avez transmise. La cible doit être de niveau 2 ou inférieur. En plus des options normales d'utilisation de l'Effort, vous pouvez choisir d'utiliser l'Effort pour augmenter le niveau maximum de la cible de un pour chaque niveau d'Effort appliqué. Action à initier. (Change the Paradygm \textendash (119))

\subsection*{Charge}\label{subsec:ab_charge}

Vous pouvez charger un artefact ou un autre appareil (à l'exception d'un chiffre) afin qu'il puisse être utilisé une seule fois. Le coût est de 1 point d'Intellect plus 1 point par niveau de l'appareil. Action. (Charge \textendash (119))

\subsection*{Charge Drainante}\label{subsec:ab_drain_charge}

Vous pouvez drainer l'énergie d'un artefact ou d'un appareil, vous permettant de regagner 1 point d'Intellect par niveau drainé. Vous regagnez des points à raison de 1 point par tour et devez vous concentrer pleinement sur le processus à chaque tour. Le MJ détermine si l'appareil est complètement vidé (probablement vrai pour la plupart des appareils portables ou plus petits) ou s'il conserve une certaine énergie (probablement vrai pour les grandes machines). Action à initier ; agir à chaque tour pour égoutter. (Drain Charge \textendash (131))

\subsection*{Charge en Horde}\label{subsec:ab_charging_horde}

vous et deux ou plusieurs de vos partisans à côté de vous pouvez agir comme une seule créature pour effectuer une attaque de charge. Lorsque vous le faites, vous vous déplacez tous sur une courte distance, pendant laquelle vous pouvez attaquer tout ce qui se trouve à portée immédiate sur votre chemin avec un atout pour l'attaque. Les cibles qui subissent des dégâts prennent 3 points supplémentaires et sont mises à terre. Action. (Charging Horde \textendash (119))

\subsection*{Chariot à vent}\label{subsec:ab_wind_chariot}

Vous invoquez des vents qui vous soulèvent et vous permettent de voler sur une longue distance à chaque round en combat ou avec une Célérité terrestre allant jusqu'à 200 miles par heure (320 km/h) jusqu'à dix heures. Pour chaque niveau d'Effort que vous appliquez, vous pouvez amener avec vous un allié d'environ votre taille dans les airs ou augmenter la durée de l'effet d'une heure. Action à initier. (Wind Chariot \textendash (199))

\subsection*{Charisme naturel}\label{subsec:ab_natural_charisma}

Vous êtes entraîné à toutes les interactions sociales, qu'elles impliquent le charme, l'apprentissage des secrets d'une personne ou l'intimidation des autres. Facilitateur. (Natural Charisma \textendash (165))

\subsection*{Charmer une Machine}\label{subsec:ab_charm_machine}

Vous convainquez une machine inivntelligente de vous « aimer ». Une machine qui vous aime a 50 % moins de chances de fonctionner si cette fonction risque de vous nuire. Ainsi, si un ennemi tente de faire exploser près de vous une bombe contrôlée par un détonateur qui vous aime, il y a 50 % de chances qu'elle n'explose pas. Action à initier. (Charm Machine \textendash (119))

\subsection*{Chasse aux machines}\label{subsec:ab_machine_hunting}

vous êtes entraîné aux tâches associées au suivi, au repérage ou à la recherche de robots et de machines animées. Vous êtes également entraîné à toutes les tâches furtives. Facilitateur. (Machine Hunting \textendash (159))

\subsection*{Chauffeur}\label{subsec:ab_driver}

Vous êtes entraîné à toutes les tâches liées à la conduite d'une voiture, d'un camion ou d'une moto, y compris les tâches de réparation mécanique. Facilitateur. (Driver \textendash (132))

\subsection*{Chauffeur expert}\label{subsec:ab_expert_driver}

Vous êtes spécialisé dans toutes les tâches liées à la conduite d'une voiture, d'un camion ou d'une moto, y compris les tâches de réparation mécanique. Facilitateur. (Expert Driver \textendash (137))

\subsection*{Chercheur expérimenté}\label{subsec:ab_experienced_finder}

lorsque vous recherchez quelque chose de spécifique, comme un composant rare particulier, un produit chimique nécessaire pour compléter un vaccin contre une maladie, une pièce de rechange nécessaire pour réparer un appareil endommagé, les traces d'un objet spécifique. bête, ou l'épée qu'un voleur vous a volée, cette capacité est d'une grande utilité. Pendant les prochaines 24 heures, si vous vous approchez à proximité de la chose et que les circonstances sont telles qu'il vous est possible d'apercevoir la chose (par exemple, elle n'est pas dans une chambre fermée à clé dont vous n'avez pas la clé), vous trouve le. Cette capacité suppose que vous êtes constamment à l'affût, que vous regardez toujours partout où cela est possible, que vous regardez derrière les obstacles, etc. Si vous courez pour sauver votre vie, si vous dormez ou êtes occupé d'une autre manière, cette capacité ne vous aide pas. Vous utilisez cette capacité au lieu de lancer un jet pour trouver la chose, mais seulement si la difficulté de trouver l'objet est de niveau 6 ou inférieur. Vous pouvez appliquer un Effort pour augmenter le niveau maximum de l'objet que vous essayez de trouver (chaque niveau d'Effort utilisé de cette manière augmente le niveau maximum de 1). Action à initier. (Experienced Finder \textendash (136))

\subsection*{Choc}\label{subsec:ab_bash}

Il s'agit d'une attaque de mêlée percutante. Votre attaque inflige 1 point de dégâts de moins que la normale, mais étourdit votre cible pendant un round, durant lequel toutes les tâches qu'elle effectue sont gênées. Action. (Bash \textendash (112))

\subsection*{Choc Electrique}\label{subsec:ab_shock}

Vos mains crépitent d'électricité, et la prochaine fois que vous touchez une créature, vous infligez 3 points de dégâts. Alternativement, si vous maniez une arme, celle-ci crépite d'électricité pendant dix minutes et inflige 1 point de dégâts supplémentaire par attaque. Action pour le toucher ; facilitateur pour arme. (Shock \textendash (183))

\subsection*{Chute contrôlée}\label{subsec:ab_controlled_fall}

lorsque vous tombez alors que vous êtes capable d'effectuer des actions et à portée d'une surface verticale, vous pouvez tenter de ralentir votre chute. Effectuez un jet de Célérité avec une difficulté de 1 tous les 20 pieds (6 m) de chute. En cas de réussite, vous subissez la moitié des dégâts de la chute. Si vous réduisez la difficulté à 0, vous ne subissez aucun dégât. Facilitateur. (Controlled Fall \textendash (122))

\subsection*{Chute en toute sécurité}\label{subsec:ab_safe_fall}

Vous réduisez les dégâts d'une chute de 5 points. Facilitateur. (Safe Fall \textendash (179))

\subsection*{Cible difficile}\label{subsec:ab_hard_target}

si vous vous déplacez sur une courte distance ou plus loin pendant votre round, tous les jets de défense de Célérité sont facilités. Facilitateur. (Hard Target \textendash (148))

\subsection*{Claque}\label{subsec:ab_swipe}

Il s'agit d'une attaque de mêlée rapide et agile. Votre attaque inflige 1 point de dégâts de moins que la normale mais étourdit votre cible pendant un round, durant lequel toutes les tâches qu'elle effectue sont gênées. Action. (Swipe \textendash (188))

\subsection*{Client Fuyant}\label{subsec:ab_slippery_customer}

lorsque vous appliquez un effort à des tâches impliquant d'échapper aux liens, de s'adapter à des espaces restreints et d'autres tâches contorsionnistes, vous obtenez un niveau d'effort gratuit sur la tâche. Grâce à votre expérience, vous êtes également entraîné aux tâches de défense de Célérité en portant une armure légère ou sans armure. Facilitateur. (Slippery Customer \textendash (183))

\subsection*{Clignotement défensif}\label{subsec:ab_defensive_blinking}

Vous entrez dans un état réactif accru de sorte que lorsque vous êtes frappé suffisamment fort pour subir des dégâts, vous vous téléportez à une distance immédiate dans une direction aléatoire (ni vers le haut ni vers le bas) pour aider à échapper au plus gros de l'attaque. Vos jets de défense Célérité sont allégés pendant une minute. Action. (Defensive Blinking \textendash (127))

\subsection*{Cogneur Entraîné}\label{subsec:ab_trained_basher}

vous êtes entraîné à utiliser les poings de pierre de votre corps de Golem comme arme moyenne. Facilitateur. (Trained Basher \textendash (193))

\subsection*{Cogneur Spécialisé}\label{subsec:ab_specialized_basher}

Vous êtes spécialisé dans l'utilisation des poings de pierre de votre capacité Corps de Golem comme arme moyenne. Facilitateur. (Specialized Basher \textendash (185))

\subsection*{Collecte d'informations}\label{subsec:ab_information_gathering}

Vous parlez par télépathie avec une ou toutes les machines dans un rayon de 1,5 km. Vous pouvez poser une question de base sur eux-mêmes ou sur tout ce qui se passe à proximité et recevoir une réponse simple. Par exemple, lorsque vous vous trouvez dans une zone comportant de nombreuses machines, vous pouvez demander l'emplacement d'une créature ou d'un individu spécifique, et s'ils se trouvent à moins d'un kilomètre et demi de vous, une ou plusieurs machines vous fourniront probablement la réponse. Action. (Information Gathering \textendash (153))

\subsection*{Colossal}\label{subsec:ab_colossal}

Lorsque vous utilisez Agrandir, vous pouvez choisir de grandir jusqu'à une hauteur de base de 60 pieds (18 m). Lorsque vous le faites, vous ajoutez 5 points temporaires supplémentaires à votre réserve de Puissance (plus ceux de Gargantuesque et Plus grand), et vous infligez 2 points de dégâts supplémentaires avec des attaques de mêlée (plus ceux de votre capacité Enorme). Pour chaque niveau d'effort que vous appliquez pour augmenter davantage votre taille, votre taille totale augmente de 10 pieds 3 m) et vous ajoutez 1 point supplémentaire à votre réserve de Puissance. Ainsi, la première fois que vous utilisez Agrandir après un jet de récupération de dix heures, si vous appliquez deux niveaux d'Effort, votre hauteur de base est de 80 pieds (24 m) et vous ajoutez un total de 17 points temporaires à votre Réserve de Puissance. Facilitateur. (Colossal \textendash (120))

\subsection*{Combat spatial}\label{subsec:ab_space_fighting}

en profitant des conditions de microgravité, vous pouvez utiliser l'inertie et la masse à votre avantage. Si vous passez un tour à préparer une attaque de mêlée (ou une attaque à partir d'un objet lancé ou lancé) en apesanteur ou en apesanteur, l'attaque inflige 6 points de dégâts supplémentaires. Facilitateur. (Space Fighting \textendash (184))

\subsection*{Combats de Horde}\label{subsec:ab_horde_fighting}

lorsque deux ennemis ou plus vous attaquent simultanément au corps à corps, vous pouvez les utiliser les uns contre les autres. Vous gagnez un atout pour les jets de défense de Célérité ou les jets d'attaque (votre choix à chaque tour) contre eux. Facilitateur. (Horde Fighting \textendash (149))

\subsection*{Combattant Hors-la-loi}\label{subsec:ab_dirty_fighter}

Vous distrayez, aveuglez, ennuyez, gênez ou interférez de toute autre manière avec un ennemi, entravant ses attaques et ses défenses pendant une minute. Action. (Dirty Fighter \textendash (128))

\subsection*{Combattant aquatique}\label{subsec:ab_aquatic_combattant}

vous ignorez les pénalités pour toute action (y compris les combats) dans des environnements sous-marins. Facilitateur. (Aquatic Combattant \textendash (110))

\subsection*{Combattant de Robot}\label{subsec:ab_robot_fighter}

Lorsque vous combattez un robot ou une machine intelligente, vous êtes entraîné aux attaques et à la défense. Facilitateur. (Robot Fighter \textendash (178))

\subsection*{Combattant mobile}\label{subsec:ab_mobile_fighter}

dans le cadre de votre attaque, vous pouvez sauter sur ou par-dessus des obstacles, vous balancer avec des cordes, courir sur des surfaces étroites ou vous déplacer sur le champ de bataille à votre Célérité normale comme si de telles tâches étaient de routine (difficulté 0 ). Vous ne pouvez pas utiliser cette capacité si vos coûts d'effort de Célérité sont réduits par le port d'une armure. Facilitateur. (Mobile Fighter \textendash (163))

\subsection*{Combattant sans armure}\label{subsec:ab_unarmored_fighter}

sans armure, vous êtes entraîné aux tâches de défense rapide. Facilitateur. (Unarmored Fighter \textendash (194))

\subsection*{Commandement}\label{subsec:ab_command}

Grâce à la seule force de votre volonté, vous pouvez donner un simple ordre impératif à une seule créature vivante, qui tente ensuite d'exécuter votre ordre lors de sa prochaine action. La créature doit être à courte portée et capable de vous comprendre. L'ordre ne peut pas infliger de dégâts directs à la créature ou à ses alliés, donc « Se suicider » ne fonctionnera pas, mais « Fuir » pourrait le faire. De plus, la commande peut exiger que la créature n'effectue qu'une seule action, donc « Déverrouiller la porte » pourrait fonctionner, mais « Déverrouiller la porte et passer à travers elle » ne fonctionnera pas. Une créature commandée peut toujours se défendre normalement et renvoyer une attaque si elle est lancée. Si vous possédez une autre capacité qui vous permet de commander une créature, vous pouvez cibler deux créatures à la fois comme effet de base si vous utilisez l'une ou l'autre capacité. Action. (Command \textendash (120))

\subsection*{Commandement avancé}\label{subsec:ab_advanced_command}

Une cible à courte portée obéit à n'importe quel ordre que vous donnez tant qu'elle peut vous entendre et vous comprendre. De plus, tant que vous continuez à ne rien faire d'autre qu'émettre des commandes (ne prendre aucune autre action), vous pouvez donner une nouvelle commande à cette même cible. Cet effet prend fin lorsque vous arrêtez d'émettre des commandes ou qu'elles sont hors de portée. Action à initier. (Advanced Command \textendash (108))

\subsection*{Commander aux bêtes}\label{subsec:ab_command_beast}

Vous pouvez commander une bête non hostile et non humaine (comme celle que vous avez calmée avec Apaiser la Bête Sauvage) jusqu'au niveau 3 à courte portée. Si vous réussissez, pendant la minute suivante, la bête suit vos ordres verbaux au mieux de sa compréhension et de ses capacités. Le MJ a le dernier mot sur ce qui compte comme une bête non humaine, mais à moins qu'une sorte de tromperie ne soit à l'œuvre, vous devez savoir si vous pouvez affecter une créature avant d'essayer d'utiliser cette capacité sur elle. Les extraterrestres, les entités extradimensionnelles, les créatures très intelligentes et les robots ne comptent jamais.En plus des options normales d'utilisation de l'Effort, vous pouvez choisir d'utiliser l'Effort pour augmenter le niveau maximum de la cible. Ainsi, pour commander une bête de niveau 5 (deux niveaux au-dessus de la limite normale), vous devez appliquer deux niveaux d'Effort. Action à initier. (Command Beast \textendash (120))

\subsection*{Commander le Métal}\label{subsec:ab_command_metal}

Vous remodelez un objet métallique à votre guise. L'objet doit être à portée de vue et à courte portée, et sa masse ne peut pas être supérieure à la vôtre. Vous pouvez affecter plusieurs éléments à la fois tant que leur masse combinée se situe dans ces limites. Vous pouvez fusionner plusieurs éléments ensemble. Vous pouvez utiliser ce pouvoir pour détruire un objet métallique (comme la capacité Détruire le métal), ou vous pouvez le façonner dans une autre forme souhaitée (en gros, à moins que vous n'ayez les compétences de fabrication appropriées). Vous pouvez ensuite déplacer le nouvel objet n'importe où à portée. Par exemple, vous pouvez prendre quelques boucliers métalliques, les fusionner et utiliser la forme obtenue pour bloquer une porte. Vous pouvez utiliser cette capacité pour lancer une attaque (contraindre l'armure d'un ennemi, transformer un objet métallique en éclats que vous lancez sur le champ de bataille, etc.) contre une cible à courte portée. Quelle que soit la forme de l'attaque, il s'agit d'une action Intellect qui inflige 7 points de dégâts. Action. (Command Metal \textendash (120))

\subsection*{Commander un Esprit}\label{subsec:ab_command_spirit}

Vous pouvez commander un esprit ou une créature morte animée jusqu'au niveau 5 à courte portée. Si vous réussissez, la cible ne peut pas vous attaquer pendant une minute, pendant laquelle elle suit vos ordres verbaux si elle peut vous entendre et vous comprendre. Action à initier. (Command Spirit \textendash (121))

\subsection*{Commander une Machine}\label{subsec:ab_command_machine}

Si vous avez charmé une machine non-intelligente ou si vous avez parlé par télépathie avec une machine intelligente, vous pouvez tenter de lui ordonner d'effectuer une action dans la limite de ses capacités lors de son prochain tour. (Si vous utilisez cette capacité pour commander une machine intelligente, elle vous deviendra probablement hostile par la suite.) Action. (Command Machine \textendash (120))

\subsection*{Comme McGyver}\label{subsec:ab_jury_rig}

Vous créez rapidement un objet en utilisant des matériaux qui semblent totalement inappropriés. Vous pouvez fabriquer une bombe avec une boîte de conserve et des produits de nettoyage ménagers, un crochet avec du papier d'aluminium ou une épée avec des meubles cassés. Le niveau de l'objet détermine la difficulté de la tâche, mais la pertinence du matériel la facilite ou la gêne également. Généralement, l'objet ne peut pas être plus grand que quelque chose que vous pouvez tenir dans une main, et il ne fonctionne qu'une seule fois (ou, dans le cas d'une arme ou d'un objet similaire, il est essentiellement utile pour une seule rencontre). Si vous consacrez au moins dix minutes à la tâche, vous pouvez créer un objet de niveau 5 ou inférieur. Vous ne pouvez pas changer la nature des matériaux impliqués. Par exemple, vous ne pouvez pas prendre des barres de fer et faire un tas de pièces d'or ou un panier en osier. Action. (Jury-Rig \textendash (156))

\subsection*{Comme l'éclair}\label{subsec:ab_bolt_rider}

Vous pouvez vous déplacer sur une longue distance d'un endroit à un autre presque instantanément, porté par un éclair. Vous devez être en mesure de voir le nouvel emplacement et il ne doit y avoir aucune barrière intermédiaire. Action. (Bolt Rider \textendash (115))

\subsection*{Comme le dos de votre main}\label{subsec:ab_like_the_back_of_your_hand}

toutes les tâches directement liées à un vaisseau spatial que vous possédez ou avec lequel vous avez un lien direct sont facilitées. Les tâches comprennent la réparation, le ravitaillement, la recherche d'une brèche dans la coque, la recherche d'un passager clandestin, etc. Il en va de même pour tous les jets d'attaque ou de défense que vous effectuez à bord du vaisseau contre les envahisseurs ennemis, ainsi que pour tous les jets d'attaque ou de défense que vous effectuez avec le vaisseau contre des navires ennemis. Facilitateur. (Like the Back of Your Hand \textendash (158))

\subsection*{Comme le prédit la prophétie}\label{subsec:ab_as_foretold_in_prophecy}

vous accomplissez quelque chose qui prouve que vous êtes véritablement l'élu. La prochaine tâche que vous tenterez est facilitée par trois étapes. Vous ne pouvez plus utiliser cette capacité avant d'avoir effectué une action de récupération d'une heure ou de dix heures. Action. (As Foretold in Prophecy \textendash (110))

\subsection*{Comme s'il n'y a qu'une seule créature}\label{subsec:ab_as_if_one_creature}

lorsque vous et votre bête (grâce à votre capacité Une Bête comme Compagnon) êtes à distance immédiate l'un de l'autre, vous pouvez partager les dégâts infligés à l'un ou l'autre de vous. Par exemple, si l'un de vous est touché par une arme et lui inflige 4 points de dégâts, répartissez les dégâts entre vous deux comme bon vous semble. Seules l'Armure et les résistances de la cible initialement endommagée entrent en jeu. Ainsi, si vous avez 2 armures et êtes frappé par une explosion de force pour 4 points de dégâts, votre bête peut subir les 2 points de dégâts que vous subiriez, mais son armure n'entre pas en jeu, ni son immunité aux explosions de force, si n'importe lequel. Facilitateur. (As If One Creature \textendash (110))

\subsection*{Comment les autres pensent}\label{subsec:ab_how_others_think}

Vous avez une idée de la façon dont les gens pensent. Vous êtes entraîné à l'une des tâches suivantes : persuasion, tromperie ou détection de mensonges. Facilitateur. (How Others Think \textendash (149))

\subsection*{Communication}\label{subsec:ab_communication}

Vous pouvez transmettre un concept de base à une créature qui ne peut normalement pas parler ou comprendre la parole. La créature peut également vous donner une réponse très basique à une question simple. Action. (Communication \textendash (121))

\subsection*{Compagnon Arbre}\label{subsec:ab_tree_companion}

Vous animez un arbre d'environ votre taille ou moins, créant une créature de niveau 3 avec 1 armure. L'arbre suit vos commandes verbales pendant une heure, après quoi il redevient un arbre normal (et s'enracine là où il se trouve). À moins que l'arbre ne soit tué par des dégâts, vous pouvez l'animer à nouveau lorsque la durée de la capacité expire, mais tous les dégâts qu'il subit sont reportés dans son nouvel état animé. En plus des options normales d'utilisation de l'Effort, vous pouvez choisir d'utiliser l'Effort pour affecter plus d'arbres ; chaque niveau d'Effort utilisé affecte un arbre supplémentaire. Action. (Tree Companion \textendash (194))

\subsection*{Compagnon Machine Amélioré}\label{subsec:ab_improved_machine_companion}

La machine issue de votre capacité Compagnon Machine s'améliore, devenant une créature de niveau 5 avec la capacité soit de voler sur une longue distance à chaque tour (et de vous transporter) pendant un maximum de dix minutes à la fois, soit de transporter un chiffre supplémentaire pendant vous qui ne compte pas dans votre limite de cypher. Facilitateur. (Improved Machine Companion \textendash (152))

\subsection*{Compagnon amélioré}\label{subsec:ab_improved_companion}

votre compagnon (comme une bête contrôlée) ou votre suiveur passe au niveau 4. En tant que créature de niveau 4, il a un nombre cible de 12 et 12 points de vie, et il inflige 4 points de dégâts (bien que dans la plupart des cas, à la place d'attaquer, il apporte un atout à vos attaques). Vous pouvez acquérir cette capacité une fois par niveau. Chaque fois que vous le sélectionnez, il augmente le niveau de votre compagnon ou suiveur de 1. Enabler. (Improved Companion \textendash (151))

\subsection*{Compagnon explorateur}\label{subsec:ab_fellow_explorer}

vous gagnez un suivant de niveau 2. L'une de leurs modifications doit concerner les tâches liées à la perception. Facilitateur. (Fellow Explorer \textendash (139))

\subsection*{Compagnon machine}\label{subsec:ab_machine_companion}

vous créez une machine animée et intelligente de niveau 3 qui vous accompagne et agit selon vos directives. En tant que compagnon machine de niveau 3, il a un nombre cible de 9 et 9 points de vie, et il inflige 3 points de dégâts. S'il est détruit, il vous faut un mois pour en créer un nouveau. Facilitateur. (Machine Companion \textendash (159))

\subsection*{Compagnon volant}\label{subsec:ab_flying_companion}

vous gagnez une créature compagnon de niveau 3 qui peut voler à la même Célérité que vous ; selon d'autres aspects de votre personnage, il peut s'agir d'un oiseau dressé, d'un drone machine ou d'une créature étrange et utile comme un familier. Cette créature vous accompagne et agit selon vos directives. En tant que compagnon de niveau 3, il a un nombre cible de 9 et 9 points de vie, et il inflige 3 points de dégâts. S'il est tué ou détruit, il vous faut un mois pour trouver ou créer un remplaçant approprié. Facilitateur. (Flying Companion \textendash (CTS, 53))

\subsection*{Compression dimensionnelle}\label{subsec:ab_dimensional_squeeze}

Vous vous tassez dans une dimension de transition, vous permettant d'apparaître instantanément n'importe où vous le souhaitez à courte portée si vous avez un chemin clair et dégagé vers cet emplacement. Vous pouvez franchir une barrière intermédiaire si elle comporte un espace ouvert dans lequel vous pouvez facilement passer votre tête – environ 1 pied carré (30 cm sur 30 cm carrés). En plus des options normales d'utilisation de l'Effort, vous pouvez choisir d'utiliser l'Effort pour passer à travers une ouverture plus petite dans une barrière ; chaque niveau d'effort utilisé de cette manière réduit la taille minimale de l'ouverture d'un quart. Vous atterrissez en toute sécurité lorsque vous utilisez cette capacité. Action.Dimension de transition : une dimension dans laquelle les distances sont plus courtes que celles des autres dimensions, donc le voyage à travers elle est plus rapide que le mouvement normal. (Dimensional Squeeze \textendash (128))

\subsection*{Compréhension}\label{subsec:ab_understanding}

Vous observez ou étudiez une créature ou un objet. Votre prochaine interaction avec cette créature ou cet objet gagne un atout. Action. (Understanding \textendash (194))

\subsection*{Compétence avec les attaques}\label{subsec:ab_skill_with_attacks}

Choisissez un type d'attaque dans lequel vous n'êtes pas déjà entraîné : frappe légère, lame légère, distance légère, frappe moyenne, lame moyenne, portée moyenne, frappe lourde, lame lourde ou distance lourde. Vous êtes entraîné aux attaques utilisant ce type d'arme. Vous pouvez sélectionner cette capacité plusieurs fois. Chaque fois que vous le sélectionnez, vous devez choisir un type d'attaque différent. Facilitateur. (Skill With Attacks \textendash (183))

\subsection*{Compétence d'arme en Prêt}\label{subsec:ab_flex_weapon_skill}

au début de chaque journée, choisissez un type d'attaque : détonation légère, lame légère, à distance légère, dénigrement moyen, lame moyenne, portée moyenne, détonation lourde, lame lourde ou à distance lourde. Pour le reste de la journée, vous êtes entraîné aux attaques utilisant ce type d'arme. Vous ne pouvez pas utiliser cette capacité avec une compétence d'attaque dans laquelle vous êtes déjà entraîné pour vous spécialiser. Facilitateur. (Flex Weapon Skill \textendash (141))

\subsection*{Compétence d'expert}\label{subsec:ab_expert_skill}

au lieu de lancer un d20, vous pouvez choisir de réussir automatiquement une tâche pour laquelle vous êtes entraîné. La tâche doit être de difficulté 4 ou inférieure, et il ne peut pas s'agir d'un jet d'attaque ou d'un jet de défense. Facilitateur. (Un personnage ne peut pas appliquer d'effort ou d'autres capacités à une tâche qu'il accomplit en utilisant la compétence Expert.) (Expert Skill \textendash (137))

\subsection*{Compétence efficace}\label{subsec:ab_effective_skill}

choisissez une compétence non liée au combat lorsque vous obtenez cette capacité. Vous obtenez un effet mineur avec cette compétence lorsque vous obtenez un 14 naturel ou plus (le d20 indique « 14 » ou plus). Vous obtenez un effet majeur avec cette compétence lorsque vous obtenez un 19 naturel ou plus (le d20 indique « 19 » ou plus). Vous pouvez sélectionner cette capacité plusieurs fois. Chaque fois que vous le sélectionnez, vous devez choisir une compétence non-combat différente. Facilitateur. (Effective Skill \textendash (133))

\subsection*{Compétence en Attaque Supérieure}\label{subsec:ab_greater_skill_with_attacks}

choisissez un type d'attaque, même celui dans lequel vous êtes déjà entraîné : frappe légère, lame légère, à distance légère, frappe moyen, lame moyenne, à distance moyenne, frappe lourde, lame lourde ou à distance lourde. Vous êtes entraîné aux attaques utilisant ce type d'arme. Si vous êtes déjà entraîné à ce type d'attaque, vous êtes alors spécialisé dans ce type d'attaque. Facilitateur. (Greater Skill With Attacks \textendash (147))

\subsection*{Compétence en Défense Supérieure}\label{subsec:ab_greater_skill_with_defense}

choisissez un type de tâche de défense, même celle pour laquelle vous êtes déjà entraîné : Puissance, Célérité ou Intellect. Vous êtes entraîné à des tâches de défense de ce type, ou spécialisé si vous êtes déjà entraîné. Vous pouvez sélectionner cette capacité jusqu'à trois fois. Chaque fois que vous le sélectionnez, vous devez choisir un type de tâche de défense différent. Facilitateur. (Greater Skill With Defense \textendash (147))

\subsection*{Compétence en défense}\label{subsec:ab_skill_with_defense}

choisissez un type de tâche de défense pour laquelle vous n'êtes pas déjà entraîné: Puissance, Célérité ou intelligence. Vous êtes entraîné à des tâches de défense de ce type. Facilitateur. (Skill With Defense \textendash (183))

\subsection*{Compétence supplémentaire}\label{subsec:ab_extra_skill}

vous êtes entraîné à une compétence de votre choix (autre que les attaques ou la défense) dans laquelle vous n'êtes pas déjà entraîné. Vous pouvez sélectionner cette capacité plusieurs fois. Chaque fois que vous le sélectionnez, vous devez choisir une compétence différente. Facilitateur. (Extra Skill \textendash (138))

\subsection*{Compétences accrues}\label{subsec:ab_heightened_skills}

vous êtes entraîné à deux tâches de votre choix (autres que les attaques ou la défense). Si vous choisissez une tâche pour laquelle vous êtes déjà entraîné, vous vous spécialisez plutôt dans cette tâche. Vous ne pouvez pas choisir une tâche dans laquelle vous êtes déjà spécialisé. (Heightened Skills \textendash (149))

\subsection*{Compétences d'assassin}\label{subsec:ab_assassin_skills}

Vous êtes entraîné aux tâches de furtivité et de déguisement. Facilitateur. (Assassin Skills \textendash (110))

\subsection*{Compétences d'enquête}\label{subsec:ab_investigative_skills}

Vous êtes entraîné à deux compétences pour lesquelles vous n'êtes pas déjà entraîné. Choisissez deux des éléments suivants : perception, identification, crochetage, évaluation du danger ou bricolage d'appareils. Vous pouvez sélectionner cette capacité plusieurs fois. Chaque fois que vous le sélectionnez, vous devez choisir deux compétences différentes. Facilitateur. (Investigative Skills \textendash (155))

\subsection*{Compétences d'interaction}\label{subsec:ab_interaction_skills}

vous êtes entraîné à deux compétences pour lesquelles vous n'êtes pas déjà entraîné. Choisissez deux des éléments suivants : tromper, persuader, parler en public, voir au-delà de la tromperie ou de l'intimidation. Vous pouvez sélectionner cette capacité plusieurs fois. Chaque fois que vous le sélectionnez, vous devez choisir deux compétences différentes. Facilitateur. (Interaction Skills \textendash (155))

\subsection*{Compétences de voyage}\label{subsec:ab_travel_skills}

Vous êtes entraîné à deux compétences pour lesquelles vous n'êtes pas déjà entraîné. Choisissez deux des éléments suivants : navigation, conduite, course, pilotage ou conduite de véhicule. Vous pouvez sélectionner cette capacité plusieurs fois. Chaque fois que vous le sélectionnez, vous devez choisir deux compétences différentes. Facilitateur. (Travel Skills \textendash (193))

\subsection*{Compétences en Connaissances}\label{subsec:ab_knowledge_skills}

Vous êtes entraîné à deux compétences pour lesquelles vous n'êtes pas déjà entraîné. Choisissez deux domaines de connaissances tels que l'histoire, la géographie, l'archéologie, etc. Vous pouvez sélectionner cette capacité plusieurs fois. Chaque fois que vous le sélectionnez, vous devez choisir deux compétences différentes. Facilitateur. (Knowledge Skills \textendash (157))

\subsection*{Compétences en Gage}\label{subsec:ab_flex_skill}

Au début de chaque journée, choisissez une tâche (autre que les attaques ou la défense) sur laquelle vous vous concentrerez. Pour le reste de la journée, vous êtes entraîné à cette tâche. Vous ne pouvez pas utiliser cette capacité avec une compétence dans laquelle vous êtes déjà entraîné pour vous spécialiser. Facilitateur. (Flex Skill \textendash (141))

\subsection*{Compétences furtives}\label{subsec:ab_stealth_skills}

vous êtes entraîné dans deux des compétences suivantes: déguisement, tromperie, crochetage, vol à la tire, voir à travers la tromperie, tour de passe-passe ou furtivité. Vous pouvez choisir cette capacité plusieurs fois, mais vous devez sélectionner des compétences différentes à chaque fois. Facilitateur. (Stealth Skills \textendash (186))

\subsection*{Compétences multiples}\label{subsec:ab_multiple_skills}

Vous êtes entraîné à trois compétences de votre choix dans lesquelles vous n'êtes pas déjà entraîné. Vous pouvez sélectionner cette capacité plusieurs fois. Chaque fois que vous le sélectionnez, vous devez choisir trois compétences différentes. Facilitateur. (Multiple Skills \textendash (165))

\subsection*{Compétences physiques}\label{subsec:ab_physical_skills}

Vous êtes entraîné à deux compétences pour lesquelles vous n'êtes pas déjà entraîné. Choisissez deux des activités suivantes : équilibrer, grimper, sauter, courir ou nager. Vous pouvez sélectionner cette capacité plusieurs fois. Chaque fois que vous le sélectionnez, vous devez choisir deux compétences différentes. Facilitateur. (Physical Skills \textendash (170))

\subsection*{Compétences sous Vide Spatial}\label{subsec:ab_vacuum_skilled}

vous êtes entraîné à deux des compétences suivantes : soudage sous vide, culture d'algues, conception d'écosystèmes, conception de circuits, maintenance et réparation d'engins spatiaux, ou une compétence similaire liée au voyage et à la colonisation de planètes, de lunes et de stations situées dans le système solaire. . Facilitateur. (Vacuum Skilled \textendash (196))

\subsection*{Compétences techniques}\label{subsec:ab_tech_skills}

vous êtes entraîné à deux compétences pour lesquelles vous n'êtes pas déjà entraîné. Choisissez deux des éléments suivants : artisanat, ordinateurs, identification, machines, pilotage, réparation ou conduite de véhicule. Vous pouvez sélectionner cette capacité plusieurs fois. Chaque fois que vous le sélectionnez, vous devez choisir deux compétences différentes. Facilitateur. (Tech Skills \textendash (189))

\subsection*{Concussion}\label{subsec:ab_concussion}

Vous faites exploser une impulsion de force de concussion à partir d'un point que vous choisissez à longue portée. L'impulsion s'étend jusqu'à une courte portée dans toutes les directions, infligeant 5 points de dégâts à tout ce qui se trouve dans la zone. Même si vous échouez au jet d'attaque, les cibles dans la zone subissent 1 point de dégâts. Action. (Concussion \textendash (121))

\subsection*{Conduite Dangereuse}\label{subsec:ab_driving_on_the_edge}

vous pouvez lancer une attaque avec une arme à portée légère ou moyenne et tenter une tâche de conduite en une seule action. Facilitateur. (Driving on the Edge \textendash (132))

\subsection*{Confondre l'ennemi}\label{subsec:ab_confuse_enemy}

Grâce à une mauvaise direction intelligente impliquant un mouvement de votre manteau, un esquive juste au bon moment, ou un stratagème similaire, vous pouvez tenter de rediriger une attaque physique de mêlée qui autrement vous frapperait. Lorsque vous faites cela, l'attaque mal dirigée touche une autre créature que vous choisissez à portée immédiate de vous et de l'ennemi attaquant. Cette capacité est une tâche de difficulté Intellect 2. Facilitateur. (Confuse Enemy \textendash (121))

\subsection*{Conjuration}\label{subsec:ab_conjuration}

Vous produisez, comme si elle sortait de nulle part, une créature de niveau 5 d'un type que vous avez déjà rencontré. La créature reste une minute puis rentre chez elle. Lorsqu'elle est présente, la créature agit selon vos instructions, mais cela ne nécessite aucune action de votre part. Action. (Conjuration \textendash (121))

\subsection*{Connaissance de la communauté}\label{subsec:ab_community_knowledge}

Si vous vous êtes investi dans une communauté et y avez passé au moins quelques mois, vous pouvez en apprendre davantage à son sujet grâce à diverses méthodes. Parfois, des contacts vous glissent des informations. D'autres fois, vous pouvez tirer des conclusions simplement à partir de ce que vous pouvez voir et entendre. Lorsque vous utilisez cette capacité, vous pouvez poser une question au MJ sur la communauté et obtenir une réponse très courte. Action. (Community Knowledge \textendash (121))

\subsection*{Connaissance de la loi}\label{subsec:ab_knowledge_of_the_law}

Vous êtes entraîné à la loi du pays. Si vous ne connaissez pas la réponse à une question de droit, vous savez où et comment la rechercher (la bibliothèque de droit d'une université est un bon point de départ, mais vous disposez également de sources en ligne). Facilitateur. (Knowledge of the Law \textendash (157))

\subsection*{Connaissance des monstres}\label{subsec:ab_monster_lore}

vous êtes entraîné aux noms, aux habitudes, aux repaires suspectés et aux sujets connexes concernant les monstres de votre monde. Vous pouvez vous faire comprendre dans leur langue (s'ils en ont une). Facilitateur. (Monster Lore \textendash (164))

\subsection*{Connaissance des ruines}\label{subsec:ab_ruin_lore}

vous êtes entraîné au nettoyage, ce qui signifie que vous êtes plus susceptible de trouver des objets utiles et des déchets qui peuvent potentiellement être transentraînés en objets utiles dans les ruines de ce qui a précédé. Facilitateur. (Ruin Lore \textendash (179))

\subsection*{Connaissance divine}\label{subsec:ab_divine_knowledge}

Vous êtes entraîné à toutes les tâches liées à la connaissance des êtres divins. Facilitateur. (Divine Knowledge \textendash (130))

\subsection*{Connaissance du bestiaire}\label{subsec:ab_bestiary_knowledge}

vous êtes entraîné à l'histoire des créatures carnivores non humanoïdes: vous savez les reconnaître, connaître leurs faiblesses et connaître leurs habitudes et leurs comportements. Facilitateur. (Bestiary Knowledge \textendash (113))

\subsection*{Connaissances en Prêt}\label{subsec:ab_flex_lore}

après chaque jet de récupération de dix heures lorsque vous avez accès à une bibliothèque de référence numérique de haute technologie (comme celle que l'on peut trouver dans un vaisseau spatial ou dans un centre d'apprentissage), choisissez un domaine de connaissances lié à une planète spécifique. ou un autre endroit. Le domaine peut concerner les habitations, les coutumes, les gouvernements, les caractéristiques des espèces principales, les personnages importants, etc. Vous êtes entraîné dans ce domaine jusqu'à ce que vous utilisiez à nouveau cette capacité. Vous pouvez utiliser cette capacité avec un domaine de connaissances dans lequel vous êtes déjà entraîné pour vous spécialiser. Facilitateur. (Flex Lore \textendash (141))

\subsection*{Connaissances en milieu sauvage}\label{subsec:ab_wilderness_lore}

Vous êtes entraîné à la navigation en milieu sauvage et à l'identification des plantes et des créatures. Facilitateur. (Wilderness Lore \textendash (199))

\subsection*{Connaissez leurs défauts}\label{subsec:ab_know_their_faults}

Si une créature que vous pouvez voir a une faiblesse particulière, comme une vulnérabilité aux bruits forts, une modification négative de la perception, etc., vous savez de quoi il s'agit. Demandez et le MJ vous le dira ; généralement, cela n'est pas associé à un jet de dés, mais dans certains cas, le MJ peut décider qu'il y a une chance que vous ne le sachiez pas. Dans ces cas-là, vous êtes spécialisé dans la connaissance des faiblesses des créatures. Facilitateur. (Know Their Faults \textendash (156))

\subsection*{Connaître}\label{subsec:ab_knowing}

Vous êtes entraîné dans un domaine de connaissances de votre choix. Facilitateur. (Knowing \textendash (156))

\subsection*{Connaître l'inconnu}\label{subsec:ab_knowing_the_unknown}

En accédant aux ressources appropriées à votre personnage, vous pouvez poser une question au MJ et obtenir une réponse générale. Le MJ attribue un niveau à la question, donc plus la réponse est obscure, plus la tâche est difficile. Généralement, les connaissances que vous pourriez trouver en cherchant ailleurs que votre emplacement actuel sont de niveau 1, et les connaissances obscures du passé sont de niveau 7. Acquérir des connaissances sur le futur est impossible. Action. (Knowing the Unknown \textendash (156))

\subsection*{Connecté}\label{subsec:ab_connected}

Vous connaissez des gens qui font avancer les choses: pas seulement des personnes respectées en position d'autorité, mais aussi une variété de pirates informatiques en ligne et de criminels de rue ordinaires. Ces personnes ne sont pas nécessairement vos amis et ne sont peut-être pas dignes de confiance, mais elles vous doivent une faveur. Vous et le MJdevriez déterminer les détails de vos contacts. Facilitateur. (Connected \textendash (121))

\subsection*{Conscience}\label{subsec:ab_awareness}

Vous devenez hyper-conscient de votre environnement afin de mieux localiser votre cible. Pendant dix minutes, vous êtes conscient de tous les êtres vivants à longue portée (y compris leur position générale), et en vous concentrant (une autre action), vous pouvez tenter de connaître la santé générale et le niveau de Puissance de chacun d'entre eux. Action. (Awareness \textendash (111))

\subsection*{Conscience totale}\label{subsec:ab_total_awareness}

Vous possédez un niveau de conscience si élevé qu'il est très difficile de vous surprendre, de vous cacher ou de vous surprendre. Lorsque vous appliquez un niveau d'Effort à des tâches d'initiative et de perception, vous gagnez deux niveaux d'Effort gratuits. Facilitateur. (Total Awareness \textendash (192))

\subsection*{Conseils d'un ami}\label{subsec:ab_advice_from_a_friend}

Vous connaissez les forces et les faiblesses de votre ami et comment le motiver à réussir. Lorsque vous donnez à un allié une suggestion concernant sa prochaine action, le personnage est entraîné à cette action pendant un round. Action. (Advice from a Friend \textendash (109))

\subsection*{Considération approfondie}\label{subsec:ab_deep_consideration}

lorsque vous élaborez un plan qui implique que vous et vos amis travailliez ensemble pour atteindre un objectif, vous pouvez poser au MJ une question très générale sur ce qui est susceptible de se produire si vous exécutez le plan, et vous obtiendra une réponse simple et brève. De plus, vous gagnez tous un atout pour un jet lié à la mise en œuvre du plan que vous avez élaboré ensemble, à condition que vous mettiez le plan en action dans les quelques jours suivant sa création. Action. (Deep Consideration \textendash (126))

\subsection*{Constructeur de fortifications}\label{subsec:ab_fortification_builder}

chaque fois que vous tentez une tâche de fabrication (ou aidez à la tâche de fabrication) pour construire un mur ou une autre fortification, vous réduisez la difficulté de fabrication de deux étapes, jusqu'à un minimum de difficulté 1. Enabler. (Fortification Builder \textendash (143))

\subsection*{Constructeur de robots}\label{subsec:ab_robot_builder}

vous êtes entraîné aux tâches liées à la construction et à la réparation de robots. À des fins de réparation, vous pouvez utiliser cette compétence pour soigner des robots utilisant une technologie similaire. Facilitateur. (Robot Builder \textendash (178))

\subsection*{Contacts avec la pègre}\label{subsec:ab_underworld_contacts}

Vous connaissez de nombreuses personnes appartenant à diverses communautés qui se livrent à des activités illégales. Ces personnes ne sont pas nécessairement vos amis et ne sont peut-être pas dignes de confiance, mais elles vous reconnaissent comme vos pairs. Vous et le MJ devriez régler les détails de vos contacts dans la pègre. Facilitateur. (Underworld Contacts \textendash (195))

\subsection*{Continuez le combat}\label{subsec:ab_fight_on}

vous ne subissez pas les pénalités normales en cas d'altération sur la piste des dégâts. Si vous êtes affaibli, au lieu de subir la pénalité normale d'être incapable d'entreprendre la plupart des actions, vous pouvez continuer à agir ; cependant, toutes les tâches sont entravées. Facilitateur. (Fight On \textendash (139))

\subsection*{Contorsionniste}\label{subsec:ab_contortionist}

Vous pouvez vous libérer des fixations ou vous faufiler dans un endroit restreint. Vous êtes entraîné à vous évader. Lorsque vous utilisez une action pour vous échapper ou vous déplacer dans une zone restreinte, vous pouvez immédiatement utiliser une autre action. Vous ne pouvez utiliser cette action que pour vous déplacer. Facilitateur. (Contortionist \textendash (121))

\subsection*{Contourner la barrière}\label{subsec:ab_bypass_barrier}

Vous franchissez une porte, un champ de force ou une autre barrière jusqu'à 3 pieds (1 m) d'épaisseur qui bloque votre chemin. En fonction de la barrière, cela peut impliquer de trouver un point faible que vous pouvez franchir, d'appuyer sur le bouton droit par chance, de simplement franchir, ou même des explications plus étranges comme toucher un espace mince entre les dimensions ou une interaction inattendue avec votre équipement. La difficulté de la tâche réside dans le niveau de la barrière. Cette capacité vous permet de passer seul, pas à quelqu'un d'autre, et le passage se ferme à la fin de votre tour (ce qui peut signifier que vous êtes piégé de l'autre côté). Vous disposez d'un atout dans toute tentative de vous en sortir à nouveau. En plus des options normales d'utilisation d'Effort, vous pouvez choisir d'utiliser Effort pour augmenter l'épaisseur maximale de la barrière, chaque niveau ajoutant 3 pieds (1 m). Action. (Bypass Barrier \textendash (116))

\subsection*{Contre-Danger}\label{subsec:ab_counter_danger}

Vous annulez une source de danger potentiel liée à une créature ou un objet à distance immédiate pendant une minute (au lieu d'un round, comme avec Déjouer le Danger). Il peut s'agir d'une arme ou d'un appareil tenu par quelqu'un, de la capacité naturelle d'une créature ou d'un piège déclenché par une plaque de pression. Vous pouvez également tenter de contrer une action (comme bouger ou effectuer une attaque banale conventionnelle avec une arme, une griffe, etc.). Action. (Pour utiliser Contre-Danger, le personnage doit généralement réfléchir rapidement face à un danger immédiat. Cette capacité ne repose pas sur des moyens surnaturels, mais plutôt sur un acte pratique.) (Counter Danger \textendash (122))

\subsection*{Contre-mesures}\label{subsec:ab_countermeasures}

Vous mettez immédiatement fin à un effet en cours (comme un effet créé par une capacité de personnage) à portée immédiate. Alternativement, vous pouvez l'utiliser comme une action de défense pour annuler toute capacité entrante qui vous cible, ou vous pouvez annuler n'importe quel appareil ou l'effet de n'importe quel appareil pendant 1d6 rounds. Vous devez toucher l'effet ou l'appareil pour l'annuler. Action. (Countermeasures \textendash (122))

\subsection*{Contrôle de Machine}\label{subsec:ab_control_machine}

Vous pouvez tenter de contrôler les fonctions de n'importe quelle machine, intelligente ou non, à courte portée pendant dix minutes. Action. (Control Machine \textendash (121))

\subsection*{Contrôle de l'Essaim}\label{subsec:ab_control_swarm}

vos créatures essaim issues de votre capacité Essaim d'influence à courte portée font ce que vous commandez par télépathie pendant dix minutes. Même les insectes communs (niveau 0) en nombre suffisant peuvent envahir une seule créature et gêner ses tâches. Action à initier (Control Swarm \textendash (122))

\subsection*{Contrôle de la météo}\label{subsec:ab_control_weather}

Vous modifiez la météo dans votre région générale. Si cela est effectué à l'intérieur, cela crée des effets mineurs, tels que de la brume, de légers changements de température, etc. Si vous l'effectuez à l'extérieur, vous pouvez créer de la pluie, du brouillard, de la neige, du vent ou tout autre type de temps normal (pas trop violent). Le changement dure un temps naturel, de sorte qu'une tempête peut durer une heure, du brouillard deux ou trois heures et de la neige quelques heures (ou dix minutes si ce n'est pas la saison). Pendant les dix premières minutes suivant l'activation de cette capacité, vous pouvez créer des effets plus dramatiques et spécifiques, tels que des éclairs, des grêlons géants, des tornades, des vents de force ouragan, etc. Ces effets doivent se produire dans un rayon de 1 000 pieds (300 m) de votre emplacement. Vous devez passer votre tour à vous concentrer pour créer un effet ou le maintenir lors d'un nouveau tour. Ces effets infligent 6 points de dégâts à chaque tour. Si vous possédez cette capacité provenant d'une autre source, le coût de la capacité est de 7 points d'Intellect au lieu de 10. Si vous possédez déjà la capacité Graine de Tempête, vous pouvez immédiatement la remplacer par une nouvelle capacité du même niveau. Action à initier. (Control Weather \textendash (122))

\subsection*{Contrôle des foules}\label{subsec:ab_crowd_control}

Vous contrôlez les actions d'un maximum de cinq créatures à courte portée. Cet effet dure une minute. Toutes les cibles doivent être de niveau 2 ou inférieur. Votre contrôle se limite à de simples commandes verbales telles que « Stop », « Fuyez », « Suivez ce garde », « Regardez là-bas » ou « Ecartez-vous de mon chemin ». Toutes les créatures affectées répondent à l'ordre, sauf si vous leur demandez spécifiquement le contraire. En plus des options normales d'utilisation de l'Effort, vous pouvez choisir d'utiliser l'Effort pour augmenter le niveau maximum des cibles ou affecter cinq personnes supplémentaires. Ainsi, pour contrôler un groupe qui a une cible de niveau 4 (deux niveaux au-dessus de la limite normale) ou un groupe de quinze personnes, vous devez appliquer deux niveaux d'Effort. Lorsque la capacité de contrôle des foules prend fin, les créatures se souviennent de vos ordres mais ne se souviennent pas d'avoir été contrôlées : vos ordres semblaient raisonnables à ce moment-là. Action à initier. (Crowd Control \textendash (123))

\subsection*{Contrôle du Changement de Forme}\label{subsec:ab_controlled_change}

vous pouvez essayer d'utiliser votre capacité **Forme de bête** pour prendre votre forme de bête n'importe quelle nuit vous le souhaitez (une tâche d'Intellect de difficulté 3). Toutes les transformations que vous effectuez en utilisant ce pouvoir s'ajoutent aux cinq nuits par mois que vous modifiez involontairement. Action pour changer. (Controlled Change \textendash (122))

\subsection*{Contrôle du robot}\label{subsec:ab_robot_control}

Vous utilisez votre connaissance du commandement et du contrôle des robots (et éventuellement des appareils qui transmettent sur la fréquence appropriée) pour affecter tout système mécanisé ou robot de niveau 2 ou inférieur à courte portée. Vous pouvez rendre plusieurs cibles inactives tant que vous concentrez toute votre attention sur elles. Si vous vous concentrez sur une seule cible, vous pouvez tenter d'en prendre le contrôle actif pendant une minute, en lui ordonnant d'effectuer des tâches simples en votre nom pendant que vous vous concentrez. En plus des options normales d'utilisation d'Effort, vous pouvez choisir d'utiliser Effort pour augmenter le niveau maximum du système mécanisé ou du robot. Ainsi, pour affecter une cible de niveau 4 (deux niveaux au-dessus de la limite normale), vous devez appliquer deux niveaux d'Effort. Action à initier. (Robot Control \textendash (178))

\subsection*{Contrôle mental}\label{subsec:ab_mind_control}

Vous contrôlez les actions d'une autre créature que vous touchez. Cet effet dure une minute. La cible doit être de niveau 2 ou inférieur. Une fois que vous avez établi le contrôle, vous maintenez un contact mental avec la cible et ressentez ce qu'elle ressent. Vous pouvez lui permettre d'agir librement ou annuler son contrôle au cas par cas. En plus des options normales d'utilisation de l'Effort, vous pouvez choisir d'utiliser l'Effort pour augmenter le niveau maximum de la cible ou augmenter la durée d'une minute. Ainsi, pour contrôler l'esprit d'une cible de niveau 5 (trois niveaux au-dessus de la limite normale) ou contrôler une cible pendant quatre minutes (trois minutes au-dessus de la durée normale), vous devez appliquer trois niveaux d'Effort. Lorsque la durée se termine, la créature ne se souvient pas d'avoir été contrôlée ou de quoi que ce soit qu'elle ait fait pendant qu'elle était sous votre commandement. Action à initier. (Mind Control \textendash (162))

\subsection*{Contrôle parfait}\label{subsec:ab_perfect_control}

vous n'avez plus besoin de faire un jet pour utiliser la forme de bête ou reprendre votre forme normale. Vous pouvez changer d'avant en arrière au fur et à mesure de votre action. Lorsque vous revenez à votre forme normale, vous ne subissez plus de pénalité à vos jets. Facilitateur. (Perfect Control \textendash (169))

\subsection*{Contrôler le sauvage}\label{subsec:ab_control_the_savage}

Vous pouvez contrôler une bête non humaine calme dans un rayon de 9 m. Vous le contrôlez aussi longtemps que vous concentrez toute votre attention dessus, en utilisant votre tour à chaque tour. Le MJ a le dernier mot sur ce qui compte comme une bête non humaine, mais à moins qu'une sorte de tromperie ne soit à l'œuvre, vous devez savoir si vous pouvez affecter une créature avant d'essayer d'utiliser cette capacité sur elle. Les extraterrestres, les entités extradimensionnelles, les créatures très intelligentes et les robots ne comptent jamais. Action. (Control the Savage \textendash (122))

\subsection*{Contrôler le terrain}\label{subsec:ab_control_the_field}

Cette attaque de mêlée inflige 1 point de dégâts de moins que la normale, et que vous touchiez ou non la cible, vous la manœuvrez dans une position que vous désirez à portée immédiate. Action. (Control the Field \textendash (121))

\subsection*{Coordination Main-Oeil}\label{subsec:ab_hand_to_eye}

Cette capacité constitue un atout pour toutes les tâches impliquant de la dextérité manuelle, telles que le vol à la tire, le crochetage, les jeux d'agilité, etc. Chaque utilisation dure jusqu'à une minute ; une nouvelle utilisation (pour changer de tâche) remplace l'utilisation précédente. Action à initier. (Hand to Eye \textendash (148))

\subsection*{Copain}\label{subsec:ab_buddy_system}

Choisissez un personnage à côté de vous. Ce personnage devient votre copain pendant dix minutes. Vous êtes entraîné à toutes les tâches impliquant la recherche, la guérison, l'interaction avec et la protection de votre copain. De plus, lorsque vous vous tenez à côté de votre copain, vous disposez tous les deux d'un atout dans les tâches de défense rapide. Vous ne pouvez avoir qu'un seul copain à la fois. Action à initier. (Buddy System \textendash (116))

\subsection*{Copie Améliorée}\label{subsec:ab_improved_copying}

vous pouvez utiliser Pouvoir de copie pour copier des capacités plus puissantes. En plus des options normales d'utilisation de l'Effort avec Pouvoir de Copie, si vous appliquez un niveau d'Effort, le MJ choisit une capacité de niveau intermédiaire qui ressemble le plus à ce pouvoir (au lieu d'une capacité de niveau inférieur). Facilitateur. Lorsque vous utilisez la Copie Améliorée, une capacité copiée doit être de niveau faible, moyen ou élevé selon la façon dont elle est répertoriée dans les catégories de capacités. Peu importe si un type ou un objectif le rend disponible à un niveau inférieur ou supérieur. (Improved Copying \textendash (151))

\subsection*{Copie multiple}\label{subsec:ab_multiple_copying}

lorsque vous utilisez Pouvoir de copie, vous pouvez copier deux capacités de la créature en même temps. En plus des options normales d'utilisation de l'Effort avec Pouvoir de copie, vous pouvez appliquer des niveaux d'Effort pour copier des capacités supplémentaires, chaque niveau d'Effort copiant une capacité supplémentaire au-delà des deux initiaux (trois pour un niveau d'Effort, quatre pour deux niveaux et bientôt). Facilitateur. (Multiple Copying \textendash (164))

\subsection*{Copie étonnante}\label{subsec:ab_amazing_copying}

vous pouvez utiliser Pouvoir de copie pour copier des capacités plus puissantes. En plus des options normales d'utilisation de l'Effort avec Pouvoir de Copie, si vous appliquez deux niveaux d'Effort, le MJ choisit une capacité de haut niveau qui ressemble le plus à ce pouvoir (au lieu d'une capacité de bas niveau). Facilitateur. (Amazing Copying \textendash (109))

\subsection*{Coquille de matière noire}\label{subsec:ab_dark_matter_shell}

Pendant la minute suivante, vous vous couvrez d'une coquille de matière noire. Votre apparence devient une silhouette sombre, et vous gagnez un atout pour les tâches de furtivité et gagnez +1 à votre armure. La coque de matière noire fonctionne parfaitement avec vos désirs, et si vous appliquez un niveau d'effort à n'importe quelle tâche physique pendant que la coque persiste, vous pouvez appliquer un niveau d'effort gratuit supplémentaire à cette même tâche. Action à initier. (Dark Matter Shell \textendash (124))

\subsection*{Corps amélioré}\label{subsec:ab_enhanced_body}

les pièces de votre machine vous accordent +1 à l'Armure, +3 à votre réserve de Puissance et +3 à votre Réserve de Célérité. Les compétences, médicaments et techniques de guérison traditionnels ne fonctionnent qu'à moitié aussi bien pour vous. Chaque fois que vous démarrez en pleine santé, les 5 premiers points de dégâts que vous subissez ne pourront jamais être soignés de cette manière ou récupérés normalement. Au lieu de cela, vous devez utiliser des compétences et des capacités de réparation pour restaurer ces points. Par exemple, si vous commencez avec une réserve de Puissance complète de 10 et subissez 8 points de dégâts, vous pouvez utiliser des jets de récupération pour restaurer 3 points, mais les 5 points restants doivent être restaurés avec des tâches de réparation. Facilitateur. (Enhanced Body \textendash (134))

\subsection*{Corps de Cristal}\label{subsec:ab_crystalline_body}

Vous êtes composé de cristal animé et translucide de couleur ambre. Travaillez avec votre MJ pour décider de votre forme exacte, même s'il s'agit probablement de la forme et de la taille d'un humanoïde. Votre corps de cristal vous accorde +2 à votre armure et +4 à votre réserve de Puissance. Cependant, vous n'êtes pas rapide et vos tâches de défense de Célérité sont gênées. Certaines conditions, comme les maladies banales et les poisons, ne vous affectent pas. Votre corps cristallin se répare plus lentement qu'un corps de chair vivante. Vous ne disposez que des jets de récupération d'un tour, d'une heure et de dix heures disponibles chaque jour ; vous ne disposez pas d'un jet de récupération de dix minutes. Toute capacité dont vous disposez qui nécessite un jet de récupération de dix minutes nécessite à la place un jet de récupération d'une heure. Facilitateur. (Crystalline Body \textendash (123))

\subsection*{Corps de Golem}\label{subsec:ab_golem_body}

vous gagnez +1 en armure, +1 en Puissance et 5 points supplémentaires en Puissance. Vous n'avez pas besoin de manger, de boire ou de respirer (même si vous avez besoin de repos et de sommeil). Vous vous déplacez avec plus de rigidité qu'une créature de chair, ce qui signifie que vous ne pouvez jamais être entraîné ou spécialisé dans les jets de défense de Célérité. De plus, vous êtes habitué à utiliser vos poings de pierre comme arme moyenne. Facilitateur. (Golem Body \textendash (145))

\subsection*{Corps en bois}\label{subsec:ab_wooden_body}

Vous transformez votre corps en bois vivant pendant dix minutes, ce qui vous confère plusieurs avantages. Vous gagnez +1 en Armure et vous êtes entraîné à utiliser vos membres comme armes moyennes. Vous avez besoin d'environ un dixième de moins d'air qu'un humain. Se cacher parmi les arbres ou sur un arbre est facilité. Cependant, sous votre forme en bois, vous vous déplacez plus rigidement qu'une créature de chair, ce qui gêne vos jets de défense de Célérité. Action de modifier ou de revenir en arrière. (Wooden Body \textendash (199))

\subsection*{Couleurs Eblouissantes}\label{subsec:ab_dazzling_sunburst}

Vous envoyez un barrage de couleurs éblouissantes sur une créature à courte portée et, en cas de réussite, vous infligez 2 points de dégâts à la cible. De plus, les attaques de la créature sont gênées lors de son prochain tour, sauf si la cible s'appuie principalement sur ses sens autres que la vue. Action. (Dazzling Sunburst \textendash (125))

\subsection*{Coup brutal}\label{subsec:ab_brute_strike}

vous infligez 4 points de dégâts supplémentaires avec toutes les attaques de mêlée jusqu'à la fin du tour suivant. Facilitateur. (Brute Strike \textendash (116))

\subsection*{Coup final}\label{subsec:ab_finishing_blow}

Si votre ennemi est à terre, étourdi ou, d'une manière ou d'une autre, impuissant ou incapable d'agir lorsque vous frappez, vous infligez 7 points de dégâts supplémentaires en cas de coup réussi. Facilitateur. (Finishing Blow \textendash (140))

\subsection*{Coup écrasant}\label{subsec:ab_crushing_blow}

lorsque vous utilisez une arme de contusion ou une arme blanche à deux mains et que vous appliquez un effort sur l'attaque, vous obtenez un niveau d'effort gratuit sur les dégâts. (Si vous combattez à mains nues, cette attaque est effectuée avec les deux poings ou les deux pieds joints.) Action. (Crushing Blow \textendash (123))

\subsection*{Coupe Précise}\label{subsec:ab_precise_cut}

Vous infligez 1 point de dégâts supplémentaire avec des armes légères. Facilitateur. (Precise Cut \textendash (171))

\subsection*{Courage du noble}\label{subsec:ab_noble's_courage}

Votre noble lignée enseigne que le courage est nécessaire pour accomplir des choses difficiles, fastidieuses ou dangereuses. Lorsque votre esprit serait affecté négativement par un effet allant jusqu'au niveau 4, qu'il soit quelque chose d'aussi manifeste qu'un ordre psychique ou une maladie ou quelque chose d'aussi subtil que la peur ou même l'ennui, votre courage neutralise l'effet pendant une minute ou, si vous sont activement attaqués, jusqu'à la prochaine attaque. Pour chaque niveau d'Effort appliqué, vous pouvez augmenter le niveau de l'effet que vous pouvez neutraliser de 1. Facilitateur. (Noble's Courage \textendash (166))

\subsection*{Courageux}\label{subsec:ab_courageous}

Vous êtes entraîné aux tâches de défense intellectuelle et aux tâches d'initiative. Facilitateur. (Courageous \textendash (122))

\subsection*{Coureur}\label{subsec:ab_runner}

votre mouvement standard passe de court à long. Facilitateur. (Runner \textendash (179))

\subsection*{Courir et combattre}\label{subsec:ab_run_and_fight}

Vous pouvez vous déplacer sur une courte distance et effectuer une attaque au corps à corps qui inflige 2 points de dégâts supplémentaires. Action. (Run and Fight \textendash (179))

\subsection*{Course d'obstacles}\label{subsec:ab_obstacle_running}

pendant la minute suivante, vous pouvez ignorer les obstacles qui ralentissent votre mouvement, vous permettant de voyager à Célérité normale à travers des zones avec des décombres, des clôtures, des tables et des objets similaires sur lesquels vous devrez grimper ou déplacer. autour. Ce mouvement peut inclure glisser sur une balustrade, courir brièvement le long d'un mur ou même marcher sur une créature pour vous améliorer sur quelque chose. Si un obstacle nécessite normalement une tâche de Puissance ou de Célérité pour être surmonté, comme se balancer sur une corde, rester en équilibre sur une corde ou sauter par-dessus un trou, vous êtes entraîné à cette tâche. Facilitateur. (Obstacle Running \textendash (167))

\subsection*{Court et Attrape}\label{subsec:ab_sprint_and_grab}

Vous pouvez courir sur une courte distance et effectuer une attaque au corps à corps pour attraper un ennemi de votre taille ou plus petit. Une attaque réussie signifie que vous saisissez l'ennemi et l'arrêtez s'il bougeait (cela peut être traité comme un plaquage, le cas échéant). Action. (Sprint and Grab \textendash (186))

\subsection*{Cri fracassant}\label{subsec:ab_shattering_shout}

Votre cri concentré crée une résonance destructrice chez une créature ou un objet à longue portée. Rien ne se passe pendant le round où vous frappez votre cible, à part un bourdonnement ou un bourdonnement inquiétant émis par la cible. Mais lors de votre prochain tour, la résonance brise des objets inanimés discrets, inflige des dégâts majeurs aux structures ou inflige 4 points de dégâts à une créature (ignore l'armure). Si vous brisez un objet, il se brise de manière explosive, infligeant 1 point de dégâts à toutes les créatures et objets à portée immédiate de lui. Si vous appliquez Effort pour augmenter les dégâts plutôt que de faciliter la tâche, vous infligez 2 points de dégâts supplémentaires par niveau d'Effort (au lieu de 3 points) ; les cibles dans la zone subissent 1 point de dégâts même si vous échouez au jet d'attaque. Action à initier. (Shattering Shout \textendash (182))

\subsection*{Création de Glace}\label{subsec:ab_ice_creation}

Vous créez un objet solide de glace de votre taille ou moins. L'objet est brut et ne peut comporter aucune pièce mobile, vous pouvez donc fabriquer une épée, un bouclier, une échelle courte, etc. Vos objets de glace sont aussi résistants que le fer, mais si vous n'êtes pas en contact permanent avec eux, ils ne fonctionnent que pendant 1d6 + 6 rounds avant de se briser ou de fondre. Par exemple, vous pouvez fabriquer et manier une épée de glace, mais si vous la donnez à un autre PJ, l'épée ne durera pas aussi longtemps pour ce personnage. En plus des options normales d'utilisation d'Effort, vous pouvez choisir d'utiliser Effort pour créer des objets plus grands que vous. Pour chaque niveau d'Effort utilisé de cette manière, vous pouvez créer un objet jusqu'à deux fois plus grand que vous. Action. (Ice Creation \textendash (150))

\subsection*{Créature magique liée}\label{subsec:ab_bound_magic_creature}

vous avez un allié magique de niveau 3 lié à un objet physique (peut-être un djinn mineur lié à une lampe, un démon mineur lié à une pièce de monnaie ou un esprit lié à un miroir). L'allié magique n'a pas encore toute la Puissance qu'un allié magique pourrait posséder à maturité. Normalement, l'allié reste au repos dans son objet lié. Lorsque vous utilisez une action pour la manifester, elle apparaît à côté de vous comme une créature pouvant converser avec vous. La créature a sa propre personnalité déterminée par le MJ et est d'un niveau supérieur à son niveau de base pour un domaine de connaissance (comme l'histoire locale). Le MJ détermine si l'allié magique a son propre objectif à long terme.Chaque fois que l'allié magique se manifeste physiquement, il le reste pendant une heure maximum. Durant cette période, il vous accompagne et suit vos instructions. L'allié magique doit rester à une distance immédiate de vous ; s'il s'éloigne plus, il est ramené dans son objet à la fin de votre tour suivant et ne peut revenir qu'après votre prochain jet de récupération de dix heures. Il n'attaque pas les créatures, mais il peut utiliser son action comme atout pour toute attaque que vous effectuez pendant votre tour. Sinon, il peut entreprendre des actions de lui-même (même si vous lancerez probablement un jet pour cela). Si la créature est réduite à 0 point de vie, elle se dissipe. Il se reforme dans son objet en 1d6 + 2 jours. Si vous perdez l'objet lié, vous conservez une idée de la direction dans laquelle il se trouve. Action pour manifester la créature magique. (Bound Magic Creature \textendash (115))

\subsection*{Créer}\label{subsec:ab_create}

Vous créez quelque chose à partir de rien. Vous pouvez créer n'importe quel objet de votre choix qui aurait normalement une difficulté de 5 ou moins (en utilisant les règles de fabrication). Une fois créé, l'objet dure un nombre d'heures égal à 6 moins la difficulté de sa création. Ainsi, si vous créez un ensemble de menottes solides (difficulté 5), cela durera une heure. Action. (Create \textendash (122))

\subsection*{Créer de l'eau}\label{subsec:ab_create_water}

Vous faites bouillonner de l'eau à partir d'un endroit du sol que vous pouvez voir. L'eau s'écoule de cet endroit pendant une minute, créant environ 1 gallon (4 litres) au moment où elle s'arrête. Action à initier. (Create Water \textendash (123))

\subsection*{Créer un poison mortel}\label{subsec:ab_create_deadly_poison}

vous créez une dose d'un poison de niveau 2 qui inflige 5 points de dégâts ou gêne les actions de la créature empoisonnée pendant dix minutes (votre choix à chaque fois que vous créez le poison). Vous pouvez appliquer ce poison sur une arme, de la nourriture ou une boisson dans le cadre de l'action de sa création. En plus des options normales d'utilisation de l'Effort, vous pouvez choisir d'utiliser l'Effort pour augmenter le niveau du poison ; chaque niveau d'Effort utilisé de cette manière augmente le niveau de poison de 1. S'il n'est pas utilisé, le poison perd sa Puissance après une heure. Action. (Create Deadly Poison \textendash (123))

\subsection*{Cube de flammes}\label{subsec:ab_fire_bloom}

Le feu apparaît à longue portée, remplissant une zone de 10 pieds (3 m) de rayon et infligeant 3 points de dégâts à toutes les cibles affectées. L'effort appliqué à une attaque compte pour toutes les attaques contre les cibles dans la zone du bloom. Même en cas d'attaque ratée, une cible dans la zone subit toujours 1 point de dégâts. Les objets inflammables présents dans la zone peuvent prendre feu. Action. (Fire Bloom \textendash (140))

\subsection*{Curieux}\label{subsec:ab_curious}

Vous êtes toujours curieux de connaître votre environnement, même à un niveau subconscient. Chaque fois que vous utilisez Effort pour tenter des tâches de navigation, de perception ou d'initiative dans une zone que vous n'avez que rarement ou jamais visitée auparavant, vous pouvez appliquer un niveau d'Effort gratuit supplémentaire. Facilitateur. (Curious \textendash (123))

\subsection*{Cyphers recyclés}\label{subsec:ab_recycled_cyphers}

tous les cyphers manifestes que vous utilisez fonctionnent à un niveau supérieur à la normale. De plus, si vous disposez d'une semaine et d'au moins dix éléments indésirables de la table Junk, vous pouvez bricoler l'un de vos cyphers manifestes, le transformant en un autre cypher du même type que celui que vous aviez dans le passé. Le MJ et le joueur doivent collaborer pour s'assurer que la transformation est logique – par exemple, vous ne pourrez probablement pas transformer une pilule en casque. Facilitateur. (Recycled Cyphers \textendash (175))

\subsection*{Célérité améliorée}\label{subsec:ab_enhanced_speed}

Vous gagnez 3 points dans votre Réserve de Célérité. Facilitateur. (Enhanced Speed \textendash (135))

\subsection*{Célérité améliorée supérieure}\label{subsec:ab_greater_enhanced_speed}

vous gagnez 6 points dans votre Réserve de Célérité. Facilitateur. (Greater Enhanced Speed \textendash (146))

\
%--------------------------
\section*{D}

\subsection*{Dans ma Main}\label{subsec:ab_impetus}

Un objet non-attaché à courte portée que vous pourriez transporter dans une main est attiré vers votre main libre. Si l'objet est coincé ou tenu par une autre créature, vous devez réussir un jet de Puissance pour le libérer, sinon l'objet reste là où il se trouve. Action. (Impetus \textendash (151))

\subsection*{Datajack}\label{subsec:ab_datajack}

Avec l'accès à l'ordinateur, vous vous connectez instantanément et en apprenez un peu plus sur quelque chose que vous pouvez voir. Vous obtenez un atout sur une tâche impliquant cette personne ou cet objet. Action. (Datajack \textendash (124))

\subsection*{De Déchet à Objet}\label{subsec:ab_junkmonger}

Vous êtes entraîné à la fabrication de deux types d'objets à partir de déchets récupérés. Si vous avez récupéré (ou obtenu d'une autre manière) au moins deux déchets de catégories différentes (électronique, plastique, dangereux, métallique, verre ou textile), vous disposez des matériaux dont vous avez besoin pour fabriquer un nouvel objet dans l'un de vos domaines de compétence. formation (sauf si le directeur général en décide autrement). Facilitateur. (Junkmonger \textendash (156))

\subsection*{Demande d'une faveur}\label{subsec:ab_call_in_favor}

Un garde, un médecin, un technicien ou un voyou employé ou allié avec un ennemi est secrètement votre allié ou vous doit une faveur. Lorsque vous demandez la faveur, la cible fait ce qu'elle peut pour vous aider à sortir d'une situation spécifique (vous détache, vous glisse un couteau, laisse une porte de cellule ouverte) de manière à minimiser son risque de révéler sa loyauté partagée à son égard. employeur ou d'autres alliés. Cette capacité est une tâche de difficulté Intellect 3. Chaque fois que vous utilisez cette capacité, la tâche est entravée par une étape supplémentaire. La difficulté revient à 3 après dix heures de repos. Action. (Call In Favor \textendash (117))

\subsection*{Depuis les Ombres}\label{subsec:ab_from_the_shadows}

Si vous réussissez à attaquer une créature qui ignorait auparavant votre présence, vous infligez 3 points de dégâts supplémentaires. Facilitateur. (From the Shadows \textendash (144))

\subsection*{Des choix difficiles}\label{subsec:ab_hard_choices}

Parfois, vous pensez que vous devez mentir à ceux qui vous font confiance pour leur propre bien. Vous êtes spécialisé dans les tâches de tromperie. Facilitateur. (Hard Choices \textendash (148))

\subsection*{Destiné à la grandeur}\label{subsec:ab_destined_for_greatness}

Vous profitez d'une chance surnaturelle comme si quelque chose veillait sur vous et vous protégeait du mal. Lorsque vous descendriez autrement d'un échelon sur la piste des dégâts, effectuez un jet de défense d'Intellect en fonction de la difficulté définie par le niveau de l'ennemi ou de l'effet. Si vous réussissez, vous ne descendez pas cette marche. Si le pas est dû au fait que vous êtes tombé à 0 point dans une réserve, vous êtes toujours à 0 point ; vous ne subissez tout simplement pas les effets négatifs d'une déficience ou d'un handicap. Si autrement vous descendiez la dernière étape du suivi des dégâts jusqu'à la mort, un jet de défense réussi vous maintient à 1 point dans une réserve et vous restez affaibli. Facilitateur. (Destined for Greatness \textendash (127))

\subsection*{Destructeur}\label{subsec:ab_destroyer}

Si vous réussissez une tâche de Puissance à endommager un objet, au lieu de descendre d'un échelon sur la piste de dégâts de l'objet, l'objet descend les trois échelons et est détruit. Action. (Destroyer \textendash (127))

\subsection*{Deux Armes Légères}\label{subsec:ab_dual_light_wield}

vous pouvez utiliser deux armes légères en même temps, effectuant deux attaques distinctes pendant votre tour en une seule action. Vous restez limité par la quantité d'effort que vous pouvez appliquer sur une action, mais comme vous effectuez des attaques séparées, l'armure de votre adversaire s'applique aux deux. Tout ce qui modifie votre attaque ou vos dégâts s'applique aux deux attaques, sauf si cela est spécifiquement lié à l'une des armes. Facilitateur. (Dual Light Wield \textendash (132))

\subsection*{Deux Armes Moyennes}\label{subsec:ab_dual_medium_wield}

vous pouvez utiliser deux armes légères ou des armes moyennes en même temps (ou une arme légère et une arme moyenne), en effectuant deux attaques distinctes pendant votre tour en une seule action. Cette capacité fonctionne par ailleurs comme la capacité Deux Armes Légères. Facilitateur. (Dual Medium Wield \textendash (132))

\subsection*{Deux choses à la fois}\label{subsec:ab_two_things_at_once}

Le test ultime : vous divisez votre attention et effectuez deux actions distinctes ce tour-ci. Facilitateur. (Two Things at Once \textendash (194))

\subsection*{Devenez défensif}\label{subsec:ab_go_defensive}

Quand vous le souhaitez, pendant le combat, vous pouvez entrer dans un état de conscience accrue de la menace. Dans cet état, vous ne pouvez pas utiliser les points de votre Réserve d'Intellect, mais vous gagnez +1 à votre Avantage de Vistesse et gagnez deux atouts pour les tâches de défense rapide. Cet effet dure aussi longtemps que vous le souhaitez ou jusqu'à ce que vous attaquiez un ennemi ou qu'aucun combat n'ait lieu à portée de vos sens. Une fois l'effet de cette capacité terminé, vous ne pouvez plus y accéder pendant une minute. Facilitateur. (Go Defensive \textendash (145))

\subsection*{Diagramme de combat}\label{subsec:ab_cognizant_offense}

pendant le combat, votre cerveau passe dans une sorte de mode de combat où toutes les attaques potentielles que vous pourriez effectuer sont tracées sur des graphiques vectoriels dans votre esprit, ce qui constitue toujours la meilleure option. Vos attaques sont atténuées. Facilitateur. (Cognizant Offense \textendash (119))

\subsection*{Diamagnétisme}\label{subsec:ab_diamagnetism}

vous magnétisez tout objet non métallique à courte portée afin qu'il puisse être affecté par vos autres pouvoirs magnétiques. Ainsi, avec Déplacer le métal, vous pouvez déplacer n'importe quel objet. Avec Repousser le métal, vous êtes entraîné à toutes les tâches de défense rapide, que l'attaque entrante utilise ou non du métal. Et ainsi de suite. Facilitateur. (Diamagnetism \textendash (128))

\subsection*{Dieu du jeu}\label{subsec:ab_gaming_god}

chaque fois que vous utilisez l'Effort sur une action Intellect, ajoutez l'une des améliorations suivantes à l'action (votre choix) :- Niveau d'effort gratuit - Effet mineur automatique Facilitateur. (Gaming God \textendash (144))

\subsection*{Difficile à distraire}\label{subsec:ab_hard_to_distract}

vous êtes entraîné aux tâches de défense intellectuelle. Facilitateur. (Hard to Distract \textendash (148))

\subsection*{Difficile à toucher}\label{subsec:ab_hard_to_hit}

vous êtes entraîné aux tâches de défense rapide. Facilitateur. (Hard to Hit \textendash (148))

\subsection*{Difficile à tuer}\label{subsec:ab_hard_to_kill}

vous pouvez choisir de relancer n'importe quelle tâche de défense que vous effectuez, mais jamais plus d'une fois par round. Facilitateur. (Hard to Kill \textendash (148))

\subsection*{Difficile à voir}\label{subsec:ab_hard_to_see}

lorsque vous bougez, vous êtes flou. Il est impossible de déterminer votre identité lorsque vous passez devant, et dans un round où vous ne faites que bouger, les tâches furtives et les tâches de Défense de Célérité sont facilitées. Facilitateur. (Hard to See \textendash (148))

\subsection*{Dire les Choses}\label{subsec:ab_telling}

Cette capacité fournit un atout pour toute tâche visant à tenter de tromper, persuader ou intimider. Chaque utilisation dure jusqu'à une minute ; une nouvelle utilisation (pour changer de tâche) remplace l'utilisation précédente. Action à initier. (Telling \textendash (190))

\subsection*{Disciple expert}\label{subsec:ab_expert_follower}

Vous gagnez un suivant de niveau 3. Ils ne sont pas limités sur leurs modifications. Vous pouvez utiliser cette capacité plusieurs fois, gagnant à chaque fois un autre adepte de niveau 3. Alternativement, vous pouvez choisir de faire progresser un adepte de niveau 2 que vous possédez déjà au niveau 3, puis de gagner un nouveau adepte de niveau 2. Facilitateur. (Expert Follower \textendash (137))

\subsection*{Discipline de vigilance}\label{subsec:ab_discipline_of_watchfulness}

vous gardez vos alliés sur leurs gardes avec des questions occasionnelles, des blagues et même des simulations d'exercices pour ceux qui souhaitent y participer. Après avoir passé 24 heures avec vous, vos alliés peuvent appliquer un niveau d'effort gratuit. à toutes les tâches d'initiative qu'ils tentent. Cet avantage perdure tant que vous restez en compagnie des alliés. Il se termine si vous partez, mais il reprend si vous revenez en compagnie des alliés dans les 24 heures. Si vous quittez la compagnie des alliés pendant plus de 24 heures, vous devez passer encore 24 heures ensemble pour réactiver l'avantage. Vous devez dépenser le coût en points d'Intellect toutes les 24 heures pendant lesquelles vous souhaitez conserver l'avantage actif. Facilitateur. (Discipline of Watchfulness \textendash (129))

\subsection*{Dislocation temporelle}\label{subsec:ab_temporal_dislocation}

Vous disparaissez et voyagez jusqu'à une heure dans le futur ou le passé. Bien que vous soyez disloqué dans le temps, vous percevez les événements lorsqu'ils se produisent depuis votre position en utilisant vos sens normaux, mais vous ne pouvez ni interagir ni changer quoi que ce soit. Si vous vous projetez dans le passé, vous y restez une heure, après quoi vous avez rattrapé le présent (pour quiconque vous accompagne dans le présent, vous semblez disparaître de l'existence pendant un instant). Si vous vous projetez dans le futur, vous y restez jusqu'à ce que le présent vous rattrape (pour toute personne qui vous accompagne dans le présent, vous disparaissez pendant une heure et réapparaissez à l'endroit que vous avez quitté). Action. (Temporal Dislocation \textendash (190))

\subsection*{Disparaître}\label{subsec:ab_vanish}

Vous devenez invisible pendant une courte période. Bien qu'invisible, vous disposez d'un atout pour les tâches de furtivité et de défense rapide. L'invisibilité prend fin à la fin de votre prochain tour, ou si vous faites quelque chose pour révéler votre présence ou votre position : attaquer, utiliser une capacité, déplacer un gros objet, etc. Action. (Vanish \textendash (196))

\subsection*{Distorsion}\label{subsec:ab_distortion}

Vous modifiez la façon dont une créature volontaire à courte portée réfléchit la lumière pendant une minute. La cible passe rapidement de son apparence normale à une tache sombre. La cible dispose d'un atout aux jets de défense Célérité jusqu'à ce que l'effet disparaisse. Action à initier. (Distortion \textendash (130))

\subsection*{Distraction verbale}\label{subsec:ab_verbal_misdirection}

avec une conversation rapide et des mots déroutants, vous pouvez confondre et distraire toute personne avec qui vous parlez, vous donnant un atout sur les interactions sociales avec cette personne pendant dix minutes. En plus des options normales d'utilisation de l'Effort, vous pouvez choisir d'utiliser l'Effort pour affecter des créatures supplémentaires (une par niveau d'Effort). Facilitateur. (Verbal Misdirection \textendash (196))

\subsection*{Divisez votre esprit}\label{subsec:ab_divide_your_mind}

Vous divisez votre conscience en deux parties. Pendant une minute, vous pouvez effectuer deux actions à chacun de vos tours, mais une seule d'entre elles peut consister à utiliser une capacité spéciale. Action. (Divide Your Mind \textendash (130))

\subsection*{Doppelganger Temporel}\label{subsec:ab_time_doppelganger}

Une copie parfaite de vous apparaît à une distance immédiate. Ce sosie est probablement une version de vous provenant d'une autre chronologie ou du passé. Le sosie est un PNJ de niveau 5 avec 15 points de vie. Il possède votre esprit et vos souvenirs, et vous le contrôlez comme si c'était vous dans un autre corps. En effet, tant que cette capacité est active, vous disposez de deux corps. Si le sosie utilise l'une de vos capacités qui coûte des points, ces points proviennent de vos réserves (y compris la dépense d'effort). Contrôler deux corps à la fois est difficile et distrayant ; tant que cette capacité est active, toutes les tâches effectuées par vous ou par le sosie sont entravées. Le sosie n'a d'autre équipement que de simples vêtements. Il reste jusqu'à une minute, mais disparaît s'il est tué ou si vous utilisez une action pour le renvoyer. Si le sosie est tué, vous subissez 5 points de dégâts qui ignorent l'Armure et vous perdez votre prochaine action. Si vous êtes tué alors que le sosie est présent, vous continuez à vivre en tant que sosie (il devient votre personnage au lieu d'être un PNJ qui disparaît). En plus des options normales d'utilisation de l'Effort, vous pouvez choisir d'utiliser l'Effort pour augmenter la durée de cette capacité ; chaque niveau d'Effort utilisé de cette manière ajoute une minute à l'existence du sosie. Si vous disposez également de cette capacité provenant d'une autre source, vous pouvez utiliser l'une ou l'autre capacité, le sosie est 1 niveau plus haut et il a 3 points de vie supplémentaires. Action. (Time Doppelganger \textendash (191))

\subsection*{Double Distraction}\label{subsec:ab_dual_distraction}

lorsque vous utilisez deux armes, la prochaine attaque de votre adversaire est gênée, et si vous appliquez l'effort lors de votre prochaine attaque contre ce même ennemi, vous obtenez un niveau d'effort gratuit sur la tâche. Facilitateur. (Dual Distraction \textendash (132))

\subsection*{Double défense}\label{subsec:ab_dual_defense}

lorsque vous utilisez deux armes, vous êtes entraîné aux tâches de défense de Célérité. Facilitateur. (Dual Defense \textendash (132))

\subsection*{Double frappe}\label{subsec:ab_double_strike}

lorsque vous utilisez deux armes, vous pouvez choisir d'effectuer un jet d'attaque contre un ennemi. Si vous touchez, vous infligez des dégâts avec les deux armes plus 2 points de dégâts supplémentaires, et comme vous avez effectué une seule attaque, l'Armure de la cible n'est soustraite qu'une seule fois. Action. (Double Strike \textendash (131))

\subsection*{Double protection}\label{subsec:ab_dual_wards}

vous pouvez avoir deux protections de Défenseur dévoué à la fois. Choisir une deuxième balise peut être une action en soi, ou vous pouvez choisir deux balises avec une seule action (et ne payer le coût qu'une seule fois pour cela). Les services doivent rester à distance immédiate les uns des autres. Les avantages fournis par Défenseur dévoué s'appliquent à vos deux protections. Si vos pupilles se séparent, vous choisissez laquelle conserve le bénéfice. S'ils se remettent ensemble, tous deux en retrouvent immédiatement le bénéfice. Facilitateur. (Dual Wards \textendash (132))

\subsection*{Drain de Créature}\label{subsec:ab_drain_creature}

Vous pouvez drainer l'énergie d'une créature vivante que vous touchez, infligeant 3 points de dégâts et restaurant 3 points à votre réserve de Puissance ou de Célérité. Action. (Drain Creature \textendash (131))

\subsection*{Drain de Machine}\label{subsec:ab_drain_machine}

Vous pouvez drainer l'énergie d'un artefact ou d'un appareil alimenté que vous touchez. Si la cible est un robot, vous infligez 3 points de dégâts et restaurez 3 points à votre réserve de Puissance ou de Célérité. Si la cible est un objet, vous restaurez des points dans votre réserve de Puissance ou de Célérité égaux au niveau de la cible. Si la cible est un chiffre manifeste, il est complètement vidé et inutile. Les artefacts et appareils similaires doivent immédiatement vérifier leur épuisement (les objets avec un épuisement de « - » sont soit immunisés contre cette capacité, soit ont un épuisement de 1 sur 1d10 lorsqu'ils sont attaqués avec cette capacité). Action. (Drain Machine \textendash (131))

\subsection*{Drain de Puissance}\label{subsec:ab_drain_power/}

Vous affectez la source d'énergie principale d'un robot ou d'une machine, lui infligeant simultanément les quatre conditions de désactivation des mécanismes. Pour cela, vous devez toucher le robot (si vous effectuez une attaque, cela n'inflige aucun dégât). Action. (Drain Power/ \textendash (131))

\subsection*{Drain à distance}\label{subsec:ab_drain_at_a_distance}

vos capacités Drain de Machine et Drain de Créature fonctionnent sur une cible à courte portée. Facilitateur. (Drain at a Distance \textendash (131))

\subsection*{Draîner la vie}\label{subsec:ab_unraveling_consumption}

Vous pouvez drainer l'énergie d'une créature vivante en la touchant et en vous concentrant pendant une minute ou plus. Chaque minute que vous passez en contact avec la créature et en vous concentrant sur elle, lui inflige 1 point de dégâts (ignore l'armure) et restaure 1 point à votre réserve de Puissance ou de Célérité. En raison du contact prolongé requis pour cette capacité, vous ne pouvez normalement l'utiliser que sur une créature volontaire ou impuissante. Si la créature subit suffisamment de dégâts pour la faire perdre connaissance ou la tuer, elle s'effondre en cendres, en poussière ou en un autre matériau inerte. Action à initier. (Unraveling Consumption \textendash (195))

\subsection*{Duel à mort}\label{subsec:ab_duel_to_the_death}

Choisissez une cible (une seule créature individuelle que vous pouvez voir). Vous êtes entraîné à toutes les tâches impliquant le combat contre cette créature. Lorsque vous attaquez avec succès cette cible, vous infligez +5 dégâts, ou +7 dégâts si la créature engage quelqu'un d'autre à votre place. Vous ne pouvez affronter qu'une seule créature à la fois. Un duel dure jusqu'à une minute, ou jusqu'à ce que vous l'interrompiez. Action à initier. (Duel to the Death \textendash (132))

\subsection*{Duplicata}\label{subsec:ab_duplicate}

Vous faites apparaître une copie de vous-même à tout moment que vous pouvez voir à courte portée. Le double n'a ni vêtements ni biens lorsqu'il apparaît. Le double est un PNJ de niveau 2 avec 6 points de vie. Le duplicata obéit à vos commandes et fait ce que vous lui demandez. Le doublon reste jusqu'à ce que vous le rejetiez à l'aide d'une action ou jusqu'à ce qu'il soit tué. Lorsque le double disparaît, il laisse derrière lui tout ce qu'il portait ou transportait. Si le doublon disparaît parce qu'il a été tué, vous subissez 4 points de dégâts qui ignorent l'Armure, et vous perdez votre prochaine action. Action à initier. (Duplicate \textendash (132))

\subsection*{Duplicata illusoire}\label{subsec:ab_illusory_duplicate}

Vous créez une seule image de vous-même à portée immédiate. L'image vous ressemble tel que vous êtes actuellement (y compris la façon dont vous êtes habillé). L'image peut se déplacer (par exemple, vous pouvez la faire marcher ou attaquer), mais elle ne peut pas se déplacer à plus d'une distance immédiate de l'endroit où vous l'avez créée. L'illusion inclut le son et l'odeur. Cela dure dix minutes et change selon vos directives (aucune concentration n'est nécessaire). Si vous dépassez la courte portée de l'illusion, elle disparaît. Action de créer. (Illusory Duplicate \textendash (150))

\subsection*{Duplicata résilient}\label{subsec:ab_resilient_duplicate}

augmente la santé de tout doublon que vous créez (comme avec Duplicata) par 5. Enabler. (Resilient Duplicate \textendash (176))

\subsection*{Duplication Supérieure}\label{subsec:ab_superior_duplicate}

Lorsque vous utilisez votre capacité Dupliquer, vous pouvez créer une copie supérieure au lieu d'une copie normale. Une copie supérieure est un PNJ de niveau 3 avec 15 points de vie. Facilitateur. (Superior Duplicate \textendash (188))

\subsection*{Dur comme du Bois}\label{subsec:ab_tough_as_nails}

lorsque vous êtes diminué ou handicapé sur le suivi des dégâts, les tâches basées sur la Puissance et les jets de défense que vous tentez sont facilités. Si vous avez également Ignorer la douleur, effectuez un jet de défense de Puissance de difficulté 1 lorsque vous atteignez 0 point dans vos trois Réserves pour regagner immédiatement 1 point de Puissance et éviter de mourir. Chaque fois que vous tentez de vous sauver avec cette capacité avant votre prochain jet de récupération de dix heures, la tâche est entravée. Facilitateur. (Un personnage ne peut pas appliquer d'effort ou d'autres capacités à une tâche accomplie avec Dur comme du Bois.) (Tough As Nails \textendash (192))

\subsection*{Dynamiser un objet}\label{subsec:ab_energize_object}

en concentrant votre capacité Absorber l'énergie cinétique sur un objet (comme une arme), vous lui infusez votre pouvoir. L'objet retient l'énergie jusqu'à ce qu'il soit touché par quelqu'un d'autre que vous, donc le mettre dans votre arme de mêlée ou dans les munitions d'une arme à distance permet à l'arme de déclencher l'énergie au combat. L'énergie inflige 3 points de dégâts à la créature touchée en plus des dégâts que l'arme elle-même pourrait faire. Vous ne pouvez pas avoir plus d'un objet sous tension sur vous à la fois. Action à initier. (Energize Object \textendash (134))

\subsection*{Débat}\label{subsec:ab_debate}

lors de tout rassemblement de deux personnes ou plus essayant d'établir la vérité ou de prendre une décision, vous pouvez influencer le verdict grâce à une rhétorique magistrale. Si vous disposez d'une minute ou plus pour faire valoir votre point de vue, soit la décision vous sera favorable, soit, si quelqu'un d'autre fait valoir un point concurrent, toute tâche de persuasion ou de tromperie associée est facilitée de deux étapes. Action à initier ; une minute pour terminer. (Debate \textendash (126))

\subsection*{Déchiffrer}\label{subsec:ab_decipher}

Si vous passez une minute à examiner un écrit ou un code dans une langue que vous ne comprenez pas, vous pouvez faire un jet d'Intellect de difficulté 3 (ou plus, en fonction de la complexité de la langue ou du code) pour comprendre l'essentiel du message. Action à initier. (Decipher \textendash (126))

\subsection*{Déchirer L'Existence}\label{subsec:ab_shred_existence}

lorsque vous utilisez Toucher perturbateur, Rayer l'Existence ou Détonation de phase, vous infligez 5 points de dégâts supplémentaires qui ignorent l'armure. Facilitateur. (Shred Existence \textendash (183))

\subsection*{Découpe}\label{subsec:ab_slice}

Il s'agit d'une attaque rapide avec une arme blanche ou pointue contre laquelle il est difficile de se défendre. Vous êtes entraîné à cette tâche. Si l'attaque réussit, elle inflige 1 point de dégâts de moins que la normale. Action. (Slice \textendash (183))

\subsection*{Déduire des pensées}\label{subsec:ab_infer_thoughts}

Si vous interagissez avec ou étudiez une cible pendant au moins un round, vous pouvez tenter de lire ses pensées superficielles, même si le sujet ne le souhaite pas. Vous devez être capable de voir la cible. Une fois que vous avez acquis une idée de ce qu'elle pense (à travers son langage corporel, son discours et ce qu'elle fait et ne dit pas), vous pouvez continuer à déduire les pensées superficielles de la cible pendant une minute maximum tant que vous pouvez toujours voir. et entendre la cible. Action à préparer ; action à initier. (Infer Thoughts \textendash (153))

\subsection*{Défendre les innocents}\label{subsec:ab_defend_the_innocent}

pendant les dix prochaines minutes, si quelqu'un que vous avez désigné comme innocent avec la capacité Désignation se tient à côté de vous, cette créature partage tous les avantages défensifs que vous pourriez avoir, autres que l'armure banale. Ces avantages incluent la défense de Célérité de votre bouclier, l'armure offerte par un champ de force, etc. De plus, les jets de défense de Célérité effectués par la créature innocente gagnent un atout. Vous ne pouvez protéger qu'une seule créature innocente à la fois. Action à initier. (Defend the Innocent \textendash (126))

\subsection*{Défendre tous les innocents}\label{subsec:ab_defend_all_the_innocent}

vous protégez tous ceux qui se trouvent à portée immédiate et que vous avez désignés comme innocents grâce à votre capacité de désignation. Les jets de défense de Célérité effectués par de telles créatures gagnent un atout. Facilitateur. (Defend All the Innocent \textendash (126))

\subsection*{Défense avec Arme}\label{subsec:ab_weapon_defense}

tant que l'arme de votre choix est dans votre (vos) main(s), vous êtes entraîné aux jets de défense de Célérité. Facilitateur. (Weapon Defense \textendash (197))

\subsection*{Défense contre les robots}\label{subsec:ab_defense_against_robots}

Vous avez étudié votre ennemi et êtes entraîné à anticiper les actions qu'un robot ou une machine est susceptible d'entreprendre lors d'un combat. Les tâches de défense que vous tentez contre ces ennemis sont facilitées. Facilitateur. (Defense Against Robots \textendash (126))

\subsection*{Défense intérieure}\label{subsec:ab_inner_defense}

Les épreuves de la vie vous ont endurci et vous ont rendu difficile à lire. Vous êtes entraîné à n'importe quelle tâche pour résister à la tentative d'une autre créature de discerner vos véritables sentiments, croyances ou projets. Vous êtes également entraîné à résister à la torture, aux intrusions télépathiques et au contrôle mental. Facilitateur. (Inner Defense \textendash (154))

\subsection*{Défense subconsciente}\label{subsec:ab_subconscious_defense}

votre subconscient exécute constamment des modèles prédictifs pour éviter le danger. Vous gagnez un atout sur vos tâches de défense de Célérité. Facilitateur. (Subconscious Defense \textendash (187))

\subsection*{Défenseur dévoué}\label{subsec:ab_devoted_defender}

choisissez un personnage que vous pouvez voir. Ce personnage devient votre pupille. Vous êtes entraîné à toutes les tâches impliquant la recherche, la guérison, l'interaction avec et la protection de ce personnage. Vous ne pouvez avoir qu'une seule salle à la fois. Action à initier. (Devoted Defender \textendash (128))

\subsection*{Défenseur expérimenté}\label{subsec:ab_experienced_defender}

lorsque vous portez une armure, vous gagnez +1 en armure. Facilitateur. (Experienced Defender \textendash (136))

\subsection*{Défi de combat}\label{subsec:ab_combat_challenge}

toutes les tentatives de tâches qui attirent une attaque contre vous-même (et loin de quelqu'un d'autre) sont facilitées de deux niveaux. Facilitateur. (Combat Challenge \textendash (120))

\subsection*{Défi final}\label{subsec:ab_final_defiance}

alors que vous seriez normalement mort, vous restez conscient et actif pendant un round supplémentaire, plus un round supplémentaire à chaque fois que vous réussissez une tâche de difficulté de Puissance de 5. Durant ces rounds, vous êtes affaibli. Si vous ne recevez pas de soins ou ne gagnez pas de points dans une réserve au cours de votre ou vos derniers tours d'activité, vous êtes soumis aux effets de Pas encore mort. Facilitateur. (Final Defiance \textendash (139))

\subsection*{Définir le bas}\label{subsec:ab_define_down}

La gravité naturelle dans une zone à une courte distance de laquelle vous êtes à portée immédiate change de direction afin qu'elle s'écoule dans la direction que vous déterminez (vers le haut, vers le haut et vers le sud, l'ouest, etc. ) pendant quelques secondes, puis revient. Les cibles affectées pourraient être projetées jusqu'à 20 pieds (6 m) et subir quelques points de dégâts. Action. (Define Down \textendash (127))

\subsection*{Défoncer}\label{subsec:ab_wreck}

Avec vos deux mains, vous maniez une arme ou un outil avec un mouvement puissant. (Si vous combattez à mains nues, cette attaque est effectuée avec les deux poings ou les deux pieds joints.) Lorsque vous l'utilisez comme attaque, vous subissez un malus de -1 au jet d'attaque et vous infligez 3 points de dégâts supplémentaires. Lorsque vous tentez d'endommager un objet ou une barrière, vous êtes entraîné à la tâche. Action. (Wreck \textendash (200))

\subsection*{Déguisement}\label{subsec:ab_disguise}

Vous êtes entraîné au déguisement. Vous pouvez modifier votre posture, votre voix, vos manières et vos cheveux pour ressembler à quelqu'un d'autre aussi longtemps que vous conservez le déguisement. Cependant, il est extrêmement difficile d'adopter l'apparence d'un individu en particulier sans un kit de déguisement à votre disposition. Facilitateur. (Disguise \textendash (129))

\subsection*{Déguisement illusoire}\label{subsec:ab_illusory_disguise}

Vous semblez être quelqu'un ou quelque chose d'autre, à peu près de votre taille et de votre forme, pendant une heure maximum. Une fois créé, le déguisement ne nécessite aucune concentration. Pour chaque point d'Intellect supplémentaire que vous dépensez, vous pouvez déguiser une autre créature. Toutes les créatures déguisées doivent rester à votre vue ou perdre leur déguisement. Action de créer. (Illusory Disguise \textendash (150))

\subsection*{Déguiser un autre}\label{subsec:ab_disguise_other}

Vous appliquez votre capacité de changement de forme à une autre créature de votre taille ou plus petite, en lui donnant une forme que vous êtes capable d'assumer. Cela dure une dizaine de minutes.En plus des options normales d'utilisation de l'Effort, vous pouvez choisir d'utiliser l'Effort pour augmenter la durée ; un niveau d'effort l'augmente à une heure, deux l'augmente à une journée. Une créature peut reprendre sa forme normale par le biais d'une action, mais elle ne peut pas ensuite reprendre sa forme modifiée. Action. Vous ne pouvez probablement pas utiliser Déguisement Other pour déguiser une sorte de créature très différente de vous, comme un humain déguisant un robot, un animal ou un extraterrestre cristallin. (Disguise Other \textendash (129))

\subsection*{Dégâts mortels}\label{subsec:ab_lethal_damage}

choisissez l'une de vos attaques existantes qui inflige des points de dégâts (en fonction de votre type et de votre concentration, cela peut être une arme spécifique, une capacité spéciale telle qu'une explosion de feu ou vos attaques à mains nues). Lorsque vous frappez avec cette attaque, vous infligez 5 points de dégâts supplémentaires. Facilitateur. (Lethal Damage \textendash (158))

\subsection*{Déjouer Intelligemment}\label{subsec:ab_outwit}

Lorsque vous effectuez un jet de défense de Célérité, vous pouvez utiliser votre intelligence à la place de votre Célérité. Facilitateur. (Outwit \textendash (168))

\subsection*{Déjouer le Danger}\label{subsec:ab_foil_danger}

Vous annulez une source de danger potentiel liée à une créature ou un objet dont vous avez conscience à distance immédiate pendant un round. Il peut s'agir d'une arme ou d'un appareil tenu par quelqu'un, d'un piège déclenché par une plaque de pression ou de la capacité naturelle d'une créature (quelque chose de spécial, inné et dangereux, comme le souffle enflammé d'un dragon ou le venin d'un cobra géant). Vous pouvez également essayer de déjouer une action banale d'un ennemi (comme une attaque avec une arme ou une griffe), afin que l'action n'ait pas lieu ce tour-ci. Faites votre jet en fonction du niveau de l'attaque, du danger ou de la créature. Action. (Foil Danger \textendash (142))

\subsection*{Démonstration Impressionnante}\label{subsec:ab_impressive_display}

Vous effectuez un exploit de force, de Célérité ou de combat, impressionnant les personnes à proximité. Pendant la minute suivante, vous gagnez un atout sur toutes les tâches d'interaction avec les personnes qui vous ont vu utiliser cette capacité. Action. (Impressive Display \textendash (151))

\subsection*{Démotiver}\label{subsec:ab_disincentivize}

Vous gênez toutes les actions tentées par un nombre quelconque de cibles à courte portée qui peuvent vous comprendre. Vous choisissez quelles cibles sont affectées. Les actions des cibles affectées sont entravées pendant un round. Facilitateur. (Disincentivize \textendash (129))

\subsection*{Déplacer des montagnes}\label{subsec:ab_move_mountains}

Vous exercez une énorme quantité de force physique à moins de 250 pieds (75 m) de vous. Vous pouvez pousser jusqu'à 10 tonnes (9 t) de matériau jusqu'à 50 pieds (15 m). Cette force peut effondrer des bâtiments, rediriger de petites rivières ou produire d'autres effets dramatiques. Action. (Move Mountains \textendash (164))

\subsection*{Déplacer le métal}\label{subsec:ab_move_metal}

Vous pouvez exercer une force sur des objets métalliques à courte portée pendant un round. Une fois activé, votre pouvoir a une réserve de Puissance effective de 10, un Avantage de Puissance de 1 et un effort de 2 (approximativement égal à la force d'un humain adulte en forme et capable), et vous pouvez l'utiliser pour déplacer des objets métalliques, pousser contre des objets métalliques, etc. Par exemple, au cours de votre ronde, vous pouvez soulever et tirer un objet métallique léger n'importe où à portée de vous ou déplacer un objet lourd (comme un meuble) d'environ 10 pieds (3 m). Ce pouvoir n'a pas le contrôle précis nécessaire pour manier une arme ou déplacer des objets avec beaucoup de Célérité, donc dans la plupart des situations, ce n'est pas un moyen d'attaque. Vous ne pouvez pas utiliser cette capacité sur votre propre corps. Le pouvoir dure une heure ou jusqu'à ce que sa réserve de Puissance soit épuisée, selon la première éventualité. Action. (Move Metal \textendash (164))

\subsection*{Désactiver les mécanismes}\label{subsec:ab_deactivate_mechanisms}

Vous effectuez une attaque au corps à corps qui n'inflige aucun dégât à une machine. Au lieu de cela, si l'attaque réussit, effectuez un deuxième jet basé sur la Célérité. En cas de succès, une machine de niveau 3 ou inférieur est désactivée pendant une minute. Pour chaque niveau d'effort supplémentaire appliqué, vous pouvez affecter un niveau de machine supérieur ou prolonger la durée d'une minute supplémentaire. Si vous possédez la capacité **Brouiller la machine** ou **Désactiver les mécanismes** (ou une capacité fonctionnant de manière similaire), lorsque vous appliquez un niveau d'effort à l'un d'entre eux, vous gagnez un niveau d'effort gratuit supplémentaire. Action. (Deactivate Mechanisms \textendash (125))

\subsection*{Désamorcer la situation}\label{subsec:ab_defuse_situation}

Au cours d'une enquête, vos questions suscitent parfois une réponse colérique, voire violente. Par la dissimulation, la distraction verbale ou une évasion similaire, vous empêchez un ennemi vivant d'attaquer qui que ce soit ou quoi que ce soit pendant un round. Action. (Defuse Situation \textendash (127))

\subsection*{Désignation}\label{subsec:ab_designation}

Vous attribuez une étiquette innocente ou coupable à une créature à portée immédiate, en fonction de votre évaluation d'une situation donnée ou d'un sentiment prédominant. En d'autres termes, une personne qualifiée d'innocente peut l'être dans certaines circonstances, ou elle peut être généralement innocente de crimes terribles (comme un meurtre, un vol majeur, etc.). De même, vous pouvez déclarer qu'une créature est coupable d'un crime particulier ou d'actes terribles en général. L'exactitude de votre évaluation n'est pas importante tant que vous croyez qu'elle est vrai; le MJ peut vous demander de justifier votre décision. Désormais, vos tâches d'interaction sociale avec quelqu'un que vous désignez comme innocent gagnent un atout, et vos attaques contre ceux que vous désignez comme coupables gagnent un atout. Vous pouvez modifier votre évaluation, mais cela nécessite une autre action de désignation. Les avantages de la désignation durent jusqu'à ce que vous la modifiiez ou jusqu'à ce qu'on vous montre la preuve qu'elle est fausse. Action. (Les avantages fournis par la désignation s'appliquent au personnage utilisant cette capacité, à ses alliés et à toute personne qui entend ou est inentraînée de son jugement et croit en son évaluation.) (Designation \textendash (127))

\subsection*{Désignation Améliorée}\label{subsec:ab_improved_designation}

lorsque vous utilisez la Désignation, vous pouvez désigner une créature supplémentaire comme étant innocente ou coupable, ce qui signifie que jusqu'à deux créatures à la fois peuvent être innocentes, ou deux coupables, ou une innocente et une coupable. Facilitateur. (Improved Designation \textendash (151))

\subsection*{Désignation supérieure}\label{subsec:ab_greater_designation}

vous pouvez attribuer une étiquette innocente ou coupable à toutes les créatures à portée immédiate lorsque vous utilisez Désignation. La seule étiquette s'applique à toutes les créatures affectées. Cela dure jusqu'à ce que vous utilisiez à nouveau la Désignation Supérieure. Action. (Greater Designation \textendash (146))

\subsection*{Détecter la vie}\label{subsec:ab_detect_life}

Vous envoyez consciemment une impulsion de votre énergie vitale. Vous détectez toutes les créatures vivantes à courte portée, même si elles sont à couvert, mais pas si elles se trouvent derrière un champ de force. Lorsque vous détectez une créature, vous détectez son emplacement général (à portée immédiate). Si vous appliquez deux niveaux d'effort supplémentaires, vous pouvez augmenter la portée de détection jusqu'à ce qu'elle soit longue. Action. (Detect Life \textendash (128))

\subsection*{Détective}\label{subsec:ab_sleuth}

Trouver les indices est la première étape pour résoudre un mystère. Vous êtes entraîné à la perception. Facilitateur. (Sleuth \textendash (183))

\subsection*{Détermination croissante}\label{subsec:ab_increasing_determination}

si vous échouez dans une tâche physique hors combat (pousser une porte ou escalader une falaise, par exemple) et que vous réessayez ensuite la tâche, la tâche est facilitée. Si vous échouez à nouveau, vous ne bénéficiez d'aucun avantage particulier. Facilitateur. (Increasing Determination \textendash (153))

\subsection*{Détonation biomorphique}\label{subsec:ab_biomorphic_detonation}

Vous rayonnez une impulsion d'énergie biomorphique jusqu'à une courte distance, mais vous la réglez pour perturber la vie dans une zone à une distance immédiate. Tous ceux qui se trouvent dans la détonation subissent 5 points de dégâts qui ignorent l'armure (sauf s'il s'agit d'une armure fournie par un effet de champ de force). Si vous appliquez un Effort supplémentaire pour augmenter les dégâts, vous infligez 2 points de dégâts supplémentaires par niveau d'Effort (au lieu de 3 points) ; les cibles dans la zone subissent 1 point de dégâts même si vous échouez au jet d'attaque. Action. (Biomorphic Detonation \textendash (113))

\subsection*{Détonation de phase}\label{subsec:ab_phase_detonation}

lorsque vous utilisez Sprint de Phase ou Traverser les murs, vous pouvez choisir d'endommager considérablement la matière normale autour de vous avec une explosion d'énergie transdimensionnelle lorsque vous entrez ou sortez de phase pour la première fois (votre choix). Cette détonation inflige 4 points de dégâts qui ignorent l'armure à toutes les créatures et objets à portée immédiate. Si vous appliquez Effort pour augmenter les dégâts plutôt que de faciliter la tâche, vous infligez 2 points de dégâts supplémentaires par niveau d'Effort (au lieu de 3 points) ; les cibles dans la zone subissent 1 point de dégâts même si vous échouez au jet d'attaque. Facilitateur. (Phase Detonation \textendash (169))

\subsection*{Détournement}\label{subsec:ab_misdirect}

lorsqu'un adversaire vous attaque et vous manque, vous pouvez rediriger son attaque vers une autre cible (une créature ou un objet) de votre choix qui se trouve à portée immédiate de vous. Effectuez un jet d'attaque non modifié contre la nouvelle cible (n'utilisez aucun de vos modificateurs ou ceux de l'adversaire au jet d'attaque, mais vous pouvez appliquer l'Effort pour plus de précision). Si l'attaque réussit, la cible subit des dégâts de l'attaque de votre adversaire. Facilitateur. (Misdirect \textendash (163))

\subsection*{Détourner les attaques}\label{subsec:ab_divert_attacks}

pendant une minute, vous déviez ou esquivez automatiquement toutes les attaques de projectiles à distance. Cependant, lors de votre prochain tour après avoir été attaqué avec des projectiles à distance, toutes vos autres actions sont gênées. Action à initier. (Divert Attacks \textendash (130))

\subsection*{Détruire le métal}\label{subsec:ab_destroy_metal}

Vous déchirez, ou faites éclater instantanément un objet métallique qui est à portée de vue, à courte portée et ne dépassant pas la moitié de votre taille. Tentez une tâche Intellect pour détruire l'objet ; la tâche est facilitée par trois niveau par rapport à la rupture avec la force brute. Action. (Destroy Metal \textendash (127))

\subsection*{Dévier les attaques}\label{subsec:ab_deflect_attacks}

En utilisant votre esprit, vous vous protégez des attaques entrantes. Pendant les dix minutes suivantes, vous êtes entraîné aux tâches de défense de Célérité. Action à initier. (Deflect Attacks \textendash (127))

\
%--------------------------
\section*{E}

\subsection*{Echolocation}\label{subsec:ab_echolocation}

Vous êtes particulièrement sensible au son et aux vibrations, à tel point que vous pouvez ressentir votre environnement à courte distance, quelle que soit votre capacité à voir. Facilitateur. (Echolocation \textendash (133))

\subsection*{Eclairs de Puissance}\label{subsec:ab_bolts_of_power}

Vous faites exploser un éventail d'éclairs à courte portée dans un arc d'environ 50 pieds (15 m) de large à l'extrémité. Cette décharge inflige 4 points de dégâts. Si vous appliquez Effort pour augmenter les dégâts plutôt que pour faciliter la tâche, vous infligez 2 points de dégâts supplémentaires par niveau d'Effort (au lieu de 3 points) ; les cibles dans la zone subissent 1 point de dégâts même si vous échouez au jet d'attaque. Action. (Bolts of Power \textendash (115))

\subsection*{Eclairé}\label{subsec:ab_enlightened}

Vous êtes entraîné à toute tâche de perception impliquant la vue. Facilitateur. (Enlightened \textendash (136))

\subsection*{Effacer les souvenirs}\label{subsec:ab_erase_memories}

Vous entrez dans l'esprit d'une créature à portée immédiate et effectuez un jet d'Intellect. En cas de réussite, vous effacez jusqu'aux cinq dernières minutes de sa mémoire. Action. (Erase Memories \textendash (136))

\subsection*{Effets accrus}\label{subsec:ab_increased_effects}

vous traitez les jets de 19 naturel comme les jets de 20 naturel pour les actions de Puissance ou les actions de Célérité (votre choix lorsque vous gagnez cette capacité). Cela vous permet d'obtenir un effet majeur sur un 19 ou un 20 naturel. (Increased Effects \textendash (153))

\subsection*{Efficacité de la machine}\label{subsec:ab_machine_efficiency}

vous pouvez faire tirer un blaster plus loin, obtenir plus de Célérité d'un skycycle, améliorer la clarté d'une caméra, truquer une lumière pour qu'elle soit plus brillante, accélérer une connexion réseau, etc. Vous augmentez le niveau d'un objet de 2 pendant une minute, ou vous traitez l'objet comme un atout qui facilite une tâche associée de deux étapes pendant une minute (votre choix). Action à initier. (Machine Efficiency \textendash (159))

\subsection*{Effort coordonné}\label{subsec:ab_coordinated_effort}

Lorsque vous et la copie de votre capacité **Duplication** attaquez la même créature, vous pouvez choisir de faire un jet d'attaque avec un atout. Si vous touchez, vous infligez des dégâts avec les deux attaques et traitez les attaques comme s'il s'agissait d'une seule attaque dans le but de soustraire l'armure des dégâts. Action. (Coordinated Effort \textendash (122))

\subsection*{Effort incroyable}\label{subsec:ab_amazing_effort}

lorsque vous appliquez au moins un niveau d'effort à une tâche hors combat, vous obtenez un niveau d'effort gratuit sur cette tâche. Lorsque vous choisissez cette capacité, décidez si elle s'applique à l'Effort de Puissance ou à l'Effort de Célérité. Facilitateur. (Amazing Effort \textendash (109))

\subsection*{Elaborer une stratégie}\label{subsec:ab_strategize}

Avoir un plan d'action en place avant de relever un défi améliore les chances de succès, même si ce plan est finalement modifié ou abandonné une fois mis en œuvre. Si vous et vos alliés passez au moins dix minutes à étudier un plan d'action, vous gagnez tous un niveau d'effort gratuit qui peut être appliqué à une tâche que vous tentez pendant l'exécution de ce plan dans les prochaines 24 heures. Le plan d'action doit être quelque chose de concret et exécutable pour obtenir cet avantage. Action à initier, dix minutes à réaliser. (Strategize \textendash (187))

\subsection*{Eloignez-vous}\label{subsec:ab_get_away}

Après votre action pendant votre tour, vous vous déplacez sur une courte distance ou vous placez derrière ou sous un couvert à portée immédiate. Facilitateur. (Get Away \textendash (145))

\subsection*{Embrassez la nuit}\label{subsec:ab_embrace_the_night}

Vous façonnez une façade vraiment horrible d'une créature à partir de rubans tourbillonnants de matière noire et la lancez sur vos ennemis à longue portée. À chaque tour, vous pouvez attaquer une cible à longue portée en utilisant la création comme arme. Lorsque vous attaquez, la créature insère de fines vrilles d'ombre dans les yeux et le cerveau de la cible. La cible subit 3 points de dégâts d'Intellect (ignore l'Armure) et est étourdie pendant un round de sorte qu'elle perd son prochain tour. Alternativement, vous pouvez amener la créature à entreprendre d'autres actions, à condition que vous soyez capable de la voir et de la contrôler mentalement comme votre action. La créature se disperse après environ une minute. Action à initier. (Embrace the Night \textendash (133))

\subsection*{Embrouiller}\label{subsec:ab_fast_talk}

lorsque vous parlez avec une créature intelligente qui peut vous comprendre et qui n'est pas hostile, vous convainquez cette créature d'entreprendre une action raisonnable au tour suivant. Une action raisonnable doit être convenue par le MJ ; cela ne doit pas mettre la créature ou ses alliés en danger évident ou être totalement hors de son caractère. Action. (Fast Talk \textendash (138))

\subsection*{Embrouiller les Mémoires personnelles}\label{subsec:ab_cloud_personal_memories}

Si vous interagissez avec ou étudiez une cible pendant au moins un tour, vous obtenez une idée du fonctionnement de son esprit, que vous pouvez utiliser contre elle de la manière la plus directe possible. Vous pouvez tenter de le confondre et lui faire oublier ce qui vient de se passer. En cas de réussite, vous effacez jusqu'aux cinq dernières minutes de sa mémoire. Action à préparer ; action à initier. (Cloud Personal Memories \textendash (119))

\subsection*{Embuscade}\label{subsec:ab_ambusher}

Lorsque vous attaquez une créature qui n'a pas encore agi lors du premier round de combat, votre attaque est facilitée. Facilitateur. (Ambusher \textendash (109))

\subsection*{En Danger}\label{subsec:ab_in_harm's_way}

Lorsque vous faites passer vos amis avant vous dans votre action, vous facilitez toutes les tâches de défense de tous les personnages que vous choisissez et qui sont adjacents à vous. Cela dure jusqu'à la fin de votre prochain tour. Si l'un de vos amis devait subir des dégâts, vous pouvez choisir de subir jusqu'à la moitié du nombre de points de dégâts qu'il subirait autrement, mais seulement si vous n'êtes pas déjà affaibli ou affaibli. Facilitateur. (In Harm's Way \textendash (152))

\subsection*{En dehors de la réalité}\label{subsec:ab_outside_reality}

Vous existez en dehors de tout jusqu'au début de votre prochain tour. Pour vous, quelques secondes s'écoulent pendant que vous êtes seul dans un vide frais. Pour tout le monde, vous semblez disparaître pendant quelques secondes et réapparaître au même endroit. Dans cet état irréel, vous pouvez utiliser des capacités ou des objets sur vous-même, mais vous ne pouvez pas percevoir, interagir avec ou affecter le reste du monde, et vice versa. Les effets temporels déjà sur vous (comme un poison qui inflige des dégâts à chaque tour) sont suspendus pendant que vous existez en dehors de la réalité, mais lorsque cette capacité prend fin, ils reprennent comme si aucun temps ne s'était écoulé. En plus des options normales d'utilisation de l'Effort, vous pouvez choisir d'utiliser l'Effort pour augmenter la durée ; chaque niveau d'effort utilisé de cette manière ajoute un tour au temps que vous passez en dehors de la réalité. Facilitateur. (Outside Reality \textendash (168))

\subsection*{En un Clin d'oeil}\label{subsec:ab_blink_of_an_eye}

Vous vous déplacez jusqu'à 1 000 pieds (300 m) en un tour. Action. (Blink of an Eye \textendash (115))

\subsection*{Encore et encore}\label{subsec:ab_again_and_again}

Vous pouvez effectuer une action supplémentaire dans un round au cours duquel vous avez déjà agi. Facilitateur. (Again and Again \textendash (109))

\subsection*{Encouragement}\label{subsec:ab_encouragement}

pendant que vous maintenez cette capacité grâce à un discours inspirant continu, vos alliés à courte portée facilitent l'un des types de tâches suivants (votre choix) : tâches de défense, tâches d'attaque ou tâches liées à toute compétence que vous êtes entraînée. ou spécialisé. Action. (Encouragement \textendash (134))

\subsection*{Encouragement de la Nature}\label{subsec:ab_wilderness_encouragement}

Lorsque vous êtes en pleine nature, ou lorsque vous parlez de votre séjour dans la nature, vos mots d'encouragement émouvants accordent à une cible à courte portée qui peut vous comprendre 1d6 points par réserve. Vous ne pouvez plus utiliser cette capacité sur la même créature tant qu'elle n'a pas effectué un jet de récupération. Action. (Wilderness Encouragement \textendash (198))

\subsection*{Endurance}\label{subsec:ab_endurance}

toute durée relative aux actions physiques est soit doublée, soit réduite de moitié, selon ce qui vous convient le mieux. Par exemple, si une personne typique peut retenir sa respiration pendant trente secondes, vous pouvez maintenir la votre pendant une minute. Si une personne typique peut marcher pendant quatre heures sans s'arrêter, vous pouvez le faire pendant huit heures. En termes d'effets nocifs, si un poison paralyse ses victimes pendant une minute, vous êtes paralysé pendant trente secondes. La durée minimale est toujours d'un tour. Facilitateur. (Endurance \textendash (134))

\subsection*{Endurance du joueur}\label{subsec:ab_gamer's_fortitude}

S'asseoir et jouer à un jeu pendant douze heures d'affilée n'est pas quelque chose que la plupart des gens peuvent faire, mais vous y arrivez. Une fois après chaque jet de récupération de dix heures, vous pouvez transférer jusqu'à 5 points entre vos pools dans n'importe quelle combinaison, à raison de 1 point par tour. Par exemple, vous pourriez transférer 3 points de Puissance en Célérité et 2 points d'Intellect en Célérité, ce qui prendrait un total de cinq tours. Action. (Gamer's Fortitude \textendash (144))

\subsection*{Energiser la créature}\label{subsec:ab_energize_creature}

vous étendez votre capacité d'absorption de l'énergie cinétique à une créature à portée immédiate afin qu'elle puisse également absorber l'énergie des attaques physiques et des impacts pendant une heure. Cette créature, cependant, ne peut pas libérer un excès d'énergie sous forme d'explosion. Pour chaque niveau d'Effort que vous appliquez, vous pouvez augmenter d'une le nombre de cibles que vous affectez. Si vous disposez d'Absorption d'énergie pure ou d'Absorption d'énergie cinétique améliorée, ces capacités sont également dupliquées dans votre cible lorsque vous utilisez Energiser la créature. Action à initier. (Energize Creature \textendash (134))

\subsection*{Energiser la foule}\label{subsec:ab_energize_crowd}

vous étendez votre capacité Absorb Kinetic Energy à un maximum de trente créatures à courte portée afin qu'elles puissent également absorber l'énergie des attaques physiques et des impacts pendant une heure. Si vous avez Absorber l'énergie pure ou Absorption d'énergie cinétique améliorée, ces créatures peuvent également utiliser ces capacités. Les créatures, cependant, ne peuvent pas libérer un excès d'énergie sous forme d'explosion. Action à initier. (Energize Crowd \textendash (134))

\subsection*{Enorme}\label{subsec:ab_huge}

lorsque vous utilisez Agrandir, vous pouvez choisir de grandir jusqu'à 16 pieds (5 m) de hauteur. Lorsque vous faites cela, vous ajoutez +1 à l'Armure (un total de +2 à l'Armure) et infligez 2 points de dégâts supplémentaires avec les attaques de mêlée. Facilitateur. (Huge \textendash (149))

\subsection*{Enquêter}\label{subsec:ab_investigate}

vous êtes entraîné à la perception, à la cryptographie, à la tromperie et à l'intrusion dans les ordinateurs. Facilitateur. (Investigate \textendash (155))

\subsection*{Enquêteur}\label{subsec:ab_investigator}

Pour vraiment briller en tant qu'enquêteur, vous devez engager votre esprit et votre corps dans vos déductions. Vous pouvez dépenser des points de votre réserve de Puissance, de votre Réserve de Célérité ou de votre réserve d'intelligence pour appliquer des niveaux d'effort à n'importe quelle tâche basée sur l'intelligence. Facilitateur. (Investigator \textendash (155))

\subsection*{Enseigner une astuce}\label{subsec:ab_teach_trick}

vous passez une heure à expliquer à quelqu'un comment exécuter une capacité de type que vous connaissez. La capacité ne doit pas dépasser le quatrième niveau. Pendant une heure après que vous leur avez enseigné, l'élève peut exécuter cette capacité comme si cela lui était naturel. Ils doivent payer le coût de Puissance, de Célérité ou d'Intellect (le cas échéant) pour utiliser cette capacité. L'étudiant doit être capable de comprendre vos instructions. En plus des options normales d'utilisation de l'Effort, vous pouvez choisir d'utiliser l'Effort pour augmenter la durée pendant laquelle l'élève peut utiliser la capacité ou pour enseigner à d'autres élèves en même temps ; chaque niveau d'Effort utilisé de cette manière augmente la durée d'une heure ou le nombre d'élèves d'un. Une heure pour s'initier. Action; heure pour terminer. (Teach Trick \textendash (189))

\subsection*{Ensemble de détection}\label{subsec:ab_sensing_package}

vous pouvez voir dans la pénombre et dans l'obscurité comme s'il s'agissait d'une lumière vive, et vous pouvez voir jusqu'à une courte distance à travers le brouillard, la fumée et d'autres phénomènes obscurcissants. De plus, si vous appliquez un niveau d'effort à des tâches de perception ou de recherche, vous obtenez un niveau d'effort gratuit sur cette tâche. Facilitateur. (Sensing Package \textendash (181))

\subsection*{Entourage}\label{subsec:ab_entourage}

Vous gagnez un entourage de cinq jeunes de niveau 1 dans la vingtaine qui vous accompagnent partout où vous allez à moins que vous ne le dissolviez volontairement pour une sortie particulière. Vous pouvez leur demander de vous livrer des choses, de vous envoyer des messages, de récupérer votre pressing – à peu près tout ce que vous voulez, dans la limite du raisonnable. Ils peuvent également provoquer des interférences si vous essayez d'éviter quelqu'un, vous aider à vous cacher de l'attention des médias, vous aider à vous frayer un chemin à travers une foule, etc. En revanche, si une situation devient physiquement violente, ils se retirent en lieu sûr. Facilitateur. (Entourage \textendash (136))

\subsection*{Entraînement au bouclier}\label{subsec:ab_shield_training}

si vous utilisez un bouclier, les tâches de défense rapide sont facilitées de deux étapes au lieu d'une. Facilitateur. (Shield Training \textendash (182))

\subsection*{Entraînement aux tâches}\label{subsec:ab_task_training}

choisissez une tâche (autre que les attaques ou la défense) pour laquelle vous n'êtes pas entraîné ou spécialisé. Vous êtes entraîné à cette tâche. Facilitateur. (Task Training \textendash (189))

\subsection*{Entraînement de guilde}\label{subsec:ab_guild_training}

vos capacités de type dont la durée est deux fois plus longue. Vos capacités de type qui ont une courte portée atteignent plutôt une longue portée. Vos capacités de type qui infligent des dégâts infligent 1 point de dégâts supplémentaire. Facilitateur. (Guild Training \textendash (147))

\subsection*{Entraînement magique}\label{subsec:ab_magical_training}

Vous êtes entraîné à tous vos sorts. De ce fait, vous facilitez toute tâche liée à l'utilisation de vos sorts. Facilitateur. (Magical Training \textendash (159))

\subsection*{Entraîné sans armure}\label{subsec:ab_trained_without_armor}

vous êtes entraîné aux tâches de défense rapide lorsque vous ne portez pas d'armure. Facilitateur. (Trained Without Armor \textendash (193))

\subsection*{Entretenir l'amitié}\label{subsec:ab_work_the_friendship}

Vous savez exactement quoi dire pour attirer un petit effort supplémentaire de la part d'un allié. Cela confère à une créature que vous choisissez à courte portée une action supplémentaire immédiate, qu'elle peut effectuer hors de son tour. La créature utilise l'action supplémentaire comme elle le souhaite. Action. (Work the Friendship \textendash (200))

\subsection*{Envoûtement}\label{subsec:ab_enthrall}

pendant que vous parlez, vous captez et maintenez l'attention d'une autre créature, même si celle-ci ne peut pas vous comprendre. Tant que vous ne faites que parler (vous ne pouvez même pas bouger), l'autre créature n'entreprend aucune action autre que se défendre, même sur plusieurs rounds. Si la créature est attaquée, l'effet prend fin. Action. (Enthrall \textendash (136))

\subsection*{Epéiste entraîné}\label{subsec:ab_trained_slayer}

vous êtes entraîné à l'utilisation des épées. Facilitateur. (Trained Slayer \textendash (193))

\subsection*{Equation prédictive}\label{subsec:ab_predictive_equation}

Vous observez ou étudiez une créature, un objet ou un lieu pendant au moins un round. La prochaine fois que vous interagissez avec elle (éventuellement au tour suivant), une tâche associée (comme persuader la créature, l'attaquer ou se défendre contre son attaque) est facilitée. Action. (Predictive Equation \textendash (171))

\subsection*{Equilibre}\label{subsec:ab_balance}

Vous êtes entraîné à l'équilibre. Facilitateur. (Balance \textendash (112))

\subsection*{Eruption d'insectes}\label{subsec:ab_insect_eruption}

Vous appelez un essaim d'insectes dans un endroit où il est possible que des insectes apparaissent. Ils restent pendant une minute et pendant ce temps, ils font ce que vous commandez tant qu'ils sont à longue portée. Ils peuvent envahir et gêner les tâches de n'importe quelle créature, ou vous pouvez concentrer l'essaim et attaquer toutes les cibles à portée immédiate les unes des autres (toutes à longue portée de vous). L'essaim attaquant inflige 2 points de dégâts par round. Vous pouvez également ordonner à l'essaim de déplacer des objets lourds grâce à un effort collectif, de manger à travers des murs en bois et d'effectuer d'autres actions adaptées à un essaim surnaturel. Action à initier. (Insect Eruption \textendash (154))

\subsection*{Esprit Agile}\label{subsec:ab_agile_wit}

lorsque vous tentez une tâche de Célérité, vous pouvez lancer (et dépenser des points) comme s'il s'agissait d'une action d'intelligence. Si vous appliquez un effort à cette tâche, vous pouvez dépenser des points de votre réserve d'intelligence au lieu de votre Réserve de Célérité (auquel cas vous utilisez également votre Avantage d'Intellect au lieu de votre Avantage de Célérité). Facilitateur. (Agile Wit \textendash (109))

\subsection*{Esprit Complice}\label{subsec:ab_spirit_accomplice}

Un esprit de niveau 3 vous accompagne et suit vos instructions. L'esprit doit rester à portée immédiate : s'il s'éloigne, il disparaît à la fin de votre tour suivant et ne peut pas revenir avant un jour. Vous et le MJ devez déterminer les détails de votre complice spirituel, et vous ferez probablement des jets pour lui lorsqu'il entreprendra des actions. L'esprit complice agit pendant votre tour, peut se déplacer sur une courte distance à chaque tour et existe partiellement hors phase (ce qui lui permet de se déplacer à travers les murs, bien qu'il soit un mauvais porteur). L'esprit s'installe dans un objet que vous désignez et se manifeste soit par une présence invisible, soit par une ombre fantomatique. Votre complice spirituel est spécialisé dans une compétence de connaissance déterminée par le MJ. L'esprit est normalement insubstantiel, mais si vous utilisez une action et dépensez 3 points d'Intellect, il accumule suffisamment de substance pour affecter le monde qui l'entoure. En tant que créature de niveau 3 dotée de substance, elle a un nombre cible de 9 et une santé de 9. Elle n'attaque pas les créatures, mais bien que substantielle, elle peut utiliser son action pour servir d'atout pour toute attaque que vous effectuez sur votre tourner. Tant qu'il est corporel, l'esprit ne peut pas se déplacer à travers les objets ni voler. Un esprit reste corporel jusqu'à dix minutes à la fois, mais redevient insubstantiel s'il n'est pas activement engagé. Si votre esprit complice est détruit, il se reforme en 1d6 jours, ou vous pouvez attirer un nouvel esprit en 2d6 jours. Facilitateur. (Une créature insubstantielle ne peut affecter ou être affectée par quoi que ce soit, sauf indication contraire, par exemple lorsqu'une attaque est effectuée avec une arme spéciale. Une créature insubstantielle peut traverser la matière solide sans entrave, mais des barrières d'énergie solides, telles que des champs magiques de force, gardez-le à distance.) (Spirit Accomplice \textendash (185))

\subsection*{Esprit Protecteur}\label{subsec:ab_wraith_cloak}

à votre demande, l'esprit de votre capacité Esprit Complice s'enroule autour de vous pendant dix minutes maximum. L'esprit inflige automatiquement 4 points de dégâts à quiconque tente de vous toucher ou de vous frapper avec une attaque au corps à corps. Tant que la cape spectrale est active, toutes les tâches visant à échapper aux perceptions des autres sont facilitées. Facilitateur. (Wraith Cloak \textendash (200))

\subsection*{Esprit de Puissance}\label{subsec:ab_mind_for_might}

lorsque vous effectuez une tâche qui nécessiterait normalement de dépenser des points de votre réserve d'intelligence, vous pouvez dépenser des points de votre réserve de Puissance à la place, et vice versa. Facilitateur. (Mind for Might \textendash (162))

\subsection*{Esprit de commandement amélioré}\label{subsec:ab_improved_command_spirit}

lorsque vous utilisez votre capacité Esprit de commandement, vous pouvez commander un esprit ou animer une créature mort-vivante jusqu'au niveau 7.Facilitateur. (Improved Command Spirit \textendash (151))

\subsection*{Esprit de leader}\label{subsec:ab_mind_of_a_leader}

Lorsque vous élaborez un plan d'action pour faire face à une situation future, vous pouvez poser au MJ une question très générale sur ce qui est susceptible de se produire si vous exécutez le plan, et vous obtiendrez une réponse simple et brève. Action. (Mind of a Leader \textendash (162))

\subsection*{Esprit fermé}\label{subsec:ab_closed_mind}

vous êtes entraîné aux tâches de défense intellectuelle et disposez de +2 en armure contre les dégâts qui ciblent sélectivement votre réserve d'intelligence (qui ignore normalement l'armure). Facilitateur. (Closed Mind \textendash (119))

\subsection*{Esprit ouvert}\label{subsec:ab_open_mind}

Vous ouvrez votre esprit pour accroître votre conscience. Vous gagnez un atout pour toute tâche impliquant de la perception. Tant que vous possédez cet atout et que vous êtes conscient et capable d'agir, les autres personnages ne gagnent aucun avantage à vous surprendre. L'effet dure une heure. Action. (Open Mind \textendash (167))

\subsection*{Esprit parle moi d'ici}\label{subsec:ab_reading_the_room}

Vous acquérez des connaissances sur une zone en parlant avec des esprits morts ou en lisant les énergies résiduelles du passé. Vous pouvez poser au MJ une seule question concrète sur l'emplacement et obtenir une réponse si vous réussissez le jet d'Intellect. « Qu'est-ce qui a tué le bétail dans cette grange ? est un bon exemple de question simple. « Pourquoi ce bétail a-t-il été tué ? » n'est pas une question appropriée car elle a plus à voir avec l'état d'esprit du tueur que avec la grange. Les questions simples ont généralement une difficulté de 2, mais les questions extrêmement techniques ou celles qui impliquent des faits destinés à rester secrets peuvent avoir une difficulté beaucoup plus élevée. Action. (Reading the Room \textendash (175))

\subsection*{Esprit perspicace}\label{subsec:ab_discerning_mind}

vous disposez d'une armure de +3 contre les attaques et les effets dommageables qui ciblent votre esprit et votre intellect. Les jets de défense que vous effectuez contre les attaques qui tentent de vous confondre, de vous persuader, de vous effrayer ou de vous influencer sont facilités. Facilitateur. (Discerning Mind \textendash (129))

\subsection*{Esprit polyvalent}\label{subsec:ab_versatile_mind}

lorsque vous effectuez un jet de défense de Célérité, vous pouvez utiliser votre intelligence à la place de votre Célérité. Facilitateur. (Versatile Mind \textendash (196))

\subsection*{Esquive}\label{subsec:ab_evasion}

Vous êtes difficile à affecter lorsque vous ne voulez pas être affecté. Vous êtes entraîné à toutes les tâches de défense. Facilitateur. (Evasion \textendash (136))

\subsection*{Esquive et résistance}\label{subsec:ab_dodge_and_resist}

Vous pouvez relancer n'importe lequel de vos jets de défense de Puissance, de Célérité ou d'intelligence et prendre le meilleur des deux résultats. Facilitateur. (Dodge and Resist \textendash (131))

\subsection*{Esquiver et répondre}\label{subsec:ab_dodge_and_respond}

Si une attaque au corps à corps vous manque, vous pouvez immédiatement effectuer une attaque au corps à corps en retour, mais pas plus d'une fois par tour. Facilitateur. (Dodge and Respond \textendash (131))

\subsection*{Essaim mortel}\label{subsec:ab_deadly_swarm}

Si vous vous trouvez dans un endroit où il est possible que votre essaim de créatures provenant de votre capacité Contrôle d'Essaim vienne, vous en appelez un essaim pendant dix minutes. Pendant ce temps, ils font ce que vous commandez par télépathie tant qu'ils sont à longue portée. Ils peuvent envahir et gêner une ou toutes les tâches des adversaires, ou ils peuvent concentrer l'essaim et attaquer tous les adversaires à portée immédiate les uns des autres (tous à longue portée de vous). L'essaim attaquant inflige 4 points de dégâts. Lorsque les créatures se trouvent à longue portée, vous pouvez leur parler par télépathie et percevoir à travers leurs sens. Action à initier. (Les essaims n'ont généralement pas de statistiques de jeu, mais si nécessaire, un essaim typique est de niveau 2. Seules les attaques qui affectent une grande zone affectent l'essaim.) (Deadly Swarm \textendash (125))

\subsection*{Est-ce que tu sais qui je suis?}\label{subsec:ab_do_you_know_who_i_am?}

agissant uniquement comme quelqu'un de célèbre et habitué aux privilèges, vous haranguez verbalement un ennemi vivant qui peut vous entendre et vous comprendre avec une telle force qu'il est incapable d'entreprendre la moindre action, y compris de lancer des attaques, pendant un round. Que vous réussissiez ou échouiez, la prochaine action entreprise par la cible après votre tentative est entravée. Action. (Do You Know Who I Am? \textendash (131))

\subsection*{Estoc}\label{subsec:ab_thrust}

Il s'agit d'un puissant coup de mêlée. Vous effectuez une attaque et infligez 1 point de dégâts supplémentaire si votre arme présente une arête ou une pointe tranchante. Action. (Thrust \textendash (191))

\subsection*{Etude rapide}\label{subsec:ab_quick_study}

vous apprenez des actions répétitives. Vous gagnez un bonus de +1 aux jets pour des tâches similaires après la première fois (comme utiliser le même appareil ou effectuer des attaques contre le même ennemi). Une fois que vous passez à une nouvelle tâche, la familiarité avec l'ancienne tâche s'estompe, à moins que vous ne recommenciez à la refaire. Facilitateur. (Quick Study \textendash (174))

\subsection*{Evanescence}\label{subsec:ab_evanesce}

Vous entrez dans l'ombre ou à l'abri, et tous ceux qui vous observaient perdent complètement votre trace. Bien que vous ne soyez pas invisible, vous ne pouvez pas être vu jusqu'à ce que vous vous révéliez à nouveau en sortant de l'ombre ou en sortant d'un abri (ou en effectuant une attaque). Action. (Evanesce \textendash (136))

\subsection*{Evasion}\label{subsec:ab_escape}

vous glissez vos attaches, vous faufilez à travers les barreaux, brisez l'emprise d'une créature qui vous tient, vous libérez des sables mouvants aspirés ou vous vous détachez de tout ce qui vous maintient en place. Action. (Escape \textendash (136))

\subsection*{Évasion illusoire}\label{subsec:ab_illusory_evasion}

lorsque vous êtes sur le point d'être touché par une attaque, vous vous téléportez à une distance immédiate, laissant derrière vous une copie illusoire de vous-même qui sera frappée par cette attaque à votre place. Cela détruit l'illusion mais vous laisse indemne par l'attaque. Si l'attaque affecte une zone et que la téléportation ne peut pas vous faire sortir de cette zone, l'attaque vous affecte quand même normalement. Facilitateur. (Illusory Evasion \textendash (150))

\subsection*{Evitement}\label{subsec:ab_flight_not_fight}

si vous utilisez votre action uniquement pour vous déplacer, toutes les tâches de défense de Célérité sont facilitées. Facilitateur. (Flight not Fight \textendash (141))

\subsection*{Evitement par microgravité}\label{subsec:ab_microgravity_avoidance}

en profitant des conditions de microgravité, vous obtenez un atout pour accélérer les tâches de défense en apesanteur ouconditions de faible gravité. Facilitateur. (Microgravity Avoidance \textendash (162))

\subsection*{Evolution du robot}\label{subsec:ab_robot_evolution}

votre premier assistant artificiel de la capacité Assistant Robot passe au niveau 5, et chacun de vos robots de niveau 2 de la flotte de robots passe au niveau 3. Au lieu de choisir cette option, vous pouvez choisir une mise à niveau de la capacité Mise à niveau du robot. Facilitateur. (Robot Evolution \textendash (178))

\subsection*{Exil}\label{subsec:ab_exile}

Vous envoyez une cible que vous touchez se précipiter dans une autre dimension ou un autre univers aléatoire, où elle reste pendant dix minutes. Vous n'avez aucune idée de ce qui arrive à la cible pendant son départ, mais au bout de dix minutes, elle revient à l'endroit précis qu'elle a quitté. Action. (Exile \textendash (136))

\subsection*{Explique l'Ineffable}\label{subsec:ab_explains_the_ineffable}

À travers des anecdotes, des récits historiques et en citant des connaissances que peu de gens, sauf vous, ont déjà comprises, vous éclairez vos amis. Après avoir passé 24 heures avec vous, une fois par jour, chacun de vos amis peut alléger une tâche particulière de deux étapes. Cet avantage se poursuit pendant que vous restez en compagnie de vos amis. Cela se termine si vous partez, mais cela reprend si vous revenez chez vos amis dans les 24 heures. Si vous quittez la compagnie de vos amis pour une durée plus longue, vous devez passer encore 24 heures ensemble pour réactiver l'avantage. Facilitateur. (Explains the Ineffable \textendash (137))

\subsection*{Exploiter l'Avantage}\label{subsec:ab_exploit_advantage}

Même si vous pouvez bien faire quelque chose, vous avez appris que vous pouvez toujours le faire encore mieux. Chaque fois que vous disposez d'un atout pour un jet de dé, vous facilitez la tâche d'une étape supplémentaire. Facilitateur. (Exploit Advantage \textendash (137))

\subsection*{Explorateur Confirmé}\label{subsec:ab_superb_explorer}

vous êtes entraîné aux tâches de recherche, d'écoute, d'escalade, d'équilibre et de saut. Facilitateur. (Superb Explorer \textendash (188))

\subsection*{Explorateur de la Nature}\label{subsec:ab_wilderness_explorer}

lorsque vous effectuez une action (y compris un combat) dans la nature, vous ignorez toutes les pénalités dues à des causes naturelles telles que les herbes hautes, les broussailles épaisses, le terrain accidenté, la météo, etc. Facilitateur. (Wilderness Explorer \textendash (199))

\subsection*{Explorateur des ténèbres}\label{subsec:ab_dark_explorer}

vous ignorez les pénalités pour toute action (y compris les combats) dans une lumière extrêmement faible ou dans des espaces exigus. Si vous disposez également de la capacité **Yeux ajustés**, vous pouvez agir sans pénalité même dans l'obscurité totale. Vous êtes entraîné à des tâches furtives dans une lumière faible ou inexistante. Facilitateur. (Dark Explorer \textendash (124))

\subsection*{Explosion Divine}\label{subsec:ab_overawe}

Une explosion de rayonnement divin venant du ciel éclaire une cible que vous sélectionnez à longue portée, la poussant à genoux (ou des appendices similaires, le cas échéant) et la rendant impuissante face à la lumière pendant un maximum de dix minutes. , ou jusqu'à ce qu'il se libère. La cible intimidée ne peut pas se défendre, lancer des attaques ou tenter autre chose que de se libérer de la crainte divine à chaque round. Si la cible est un démon, un esprit ou quelque chose de similaire, elle subit également 1 point de dégâts qui ignore l'armure à chaque round où elle reste affectée. Action à initier. (Overawe \textendash (168))

\subsection*{Explosion commotionnelle}\label{subsec:ab_concussive_blast}

Vous libérez un rayon de force pure qui s'écrase sur une créature à courte portée, lui infligeant 5 points de dégâts et la repoussant à une distance immédiate. Action. (Concussive Blast \textendash (121))

\subsection*{Explosion de Force}\label{subsec:ab_force_blast}

Vous découvrez comment projeter des explosions de force pure à partir des gantelets de l'armure assistée grâce à votre capacité Armure motorisée. Cela vous permet de tirer une explosion de force qui inflige 5 points de dégâts avec une portée de 200 pieds (60 m). Action. (Force Blast \textendash (142))

\subsection*{Explosion de froid}\label{subsec:ab_cold_burst}

Vous émettez une explosion de froid dans toutes les directions, jusqu'à courte portée. Tous ceux qui se trouvent dans la rafale (sauf vous) subissent 5 points de dégâts. Si vous appliquez Effort pour augmenter les dégâts plutôt que pour faciliter la tâche, vous infligez 2 points de dégâts supplémentaires par niveau d'Effort (au lieu de 3 points) ; les cibles dans la zone subissent 1 point de dégâts même si vous échouez au jet d'attaque. Action. (Cold Burst \textendash (119))

\subsection*{Expérience d'exploration}\label{subsec:ab_exploratory_experience}

vous êtes entraîné à deux compétences supplémentaires dans lesquelles vous n'êtes pas déjà entraîné. Choisissez parmi les éléments suivants : navigation, perception, détection du danger, initiative, ouverture pacifique des communications avec des inconnus et suivi. Facilitateur. (Exploratory Experience \textendash (137))

\subsection*{Expérimenté en armure}\label{subsec:ab_experienced_in_armor}

la réduction du coût de votre capacité Pratiqué en armure s'améliore. Vous réduisez désormais le coût en Célérité de 2. Enabler. (Experienced in Armor \textendash (136))

\subsection*{Extraire du hasard}\label{subsec:ab_wrest_from_chance}

Si vous obtenez un 1 naturel sur un d20, vous pouvez relancer le dé. Si vous relancez, vous évitez une intrusion du MJ (à moins que vous n'obteniez un deuxième 1) et vous pourriez réussir votre tâche. Une fois que vous utilisez cette capacité, elle n'est à nouveau disponible qu'après avoir effectué un jet de récupération de dix heures. Facilitateur. (Wrest From Chance \textendash (200))

\
%--------------------------
\section*{F}

\subsection*{Fabricant d'armes}\label{subsec:ab_weapon_crafter}

vous êtes entraîné aux tâches de fabrication associées à l'arme que vous avez choisie. Par exemple, si votre arme est un arc, vous êtes entraîné aux tâches liées à la fabrication d'arcs et à l'empennage de flèches ; si votre arme est une épée, vous êtes entraîné aux tâches consistant à forger des épées et à affûter des lames ; et ainsi de suite. Facilitateur. (Weapon Crafter \textendash (197))

\subsection*{Facilité Inspirante}\label{subsec:ab_inspiring_ease}

à travers des histoires, des chansons, de l'art ou d'autres formes de divertissement, vous inspirez vos amis. Après avoir passé 24 heures avec vous, une fois par jour chacun de vos amis peut vous faciliter une tâche. Cet avantage se poursuit pendant que vous restez en compagnie de l'ami. Cela se termine si vous partez, mais cela reprend si vous revenez en compagnie de l'ami dans les 24 heures. Si vous quittez l'entreprise de votre ami pendant plus de 24 heures, vous devez passer encore 24 heures ensemble pour réactiver l'avantage. Facilitateur. (Inspiring Ease \textendash (154))

\subsection*{Faire Corps avec la Nature}\label{subsec:ab_one_with_the_wild}

Pendant l'heure suivante, les animaux et les plantes naturels à longue portée ne vous feront pas de mal sciemment, ni à ceux que vous désignez. De plus, votre Might Edge, Speed Edge et Intellect Edge augmentent de 1, et si vous effectuez des jets de récupération pendant cette période, vous récupérez deux fois plus de points. Action à initier. (One With the Wild \textendash (167))

\subsection*{Faisceau de Tonnerre}\label{subsec:ab_thunder_beam}

Vous dirigez un faisceau de son concentré sur une cible à longue portée, lui infligeant 2 points de dégâts et induisant une onde destructrice résonnante dans son corps. Chaque round après cette attaque initiale, vous pouvez refaire un jet de vague destructrice pour infliger 1 point de dégâts supplémentaire à la cible. Si vous échouez à ce jet, la vague destructrice prend fin. Contrairement à l'attaque initiale, la vague destructrice ignore l'Armure. Alternativement, vous pouvez créer une résonance destructrice dans une arme physique de mêlée pendant une minute ou jusqu'à ce que vous la lâchez. Toutes les attaques effectuées avec l'arme ciblée infligent 1 point de dégâts supplémentaire. Action à initier. (Thunder Beam \textendash (191))

\subsection*{Faites confiance à la chance}\label{subsec:ab_trust_to_luck}

Parfois, il suffit de lancer les dés et d'espérer que les choses tournent en votre faveur. Lorsque vous utilisez Faites confiance à la chance, lancez un d6. Quel que soit le résultat, la tâche que vous tentez est facilitée par deux étapes. Sur un résultat de 1, la tâche est gênée. Facilitateur. (Trust to Luck \textendash (194))

\subsection*{Faites face aux vicissitudes}\label{subsec:ab_weather_the_vicissitudes}

Aider vos amis, c'est être capable de résister à tout ce que le monde vous réserve. Vous avez +1 en Armure. De plus, vous résistez à la chaleur, au froid et à des extrêmes similaires et disposez d'un +1 supplémentaire à l'armure contre les dégâts ambiants ou d'autres dégâts qui ignoreraient normalement l'armure. Facilitateur. (Weather the Vicissitudes \textendash (197))

\subsection*{Familiarisez-vous}\label{subsec:ab_familiarize}

vous pouvez vous familiariser avec une nouvelle zone si vous passez au moins une heure à étudier une région située sur une longue distance et à laquelle vous pouvez accéder directement et vous déplacer. Une fois que vous vous êtes familiarisé avec une zone, tous vos les tâches liées à la perception, à la navigation, à la récupération et au nettoyage, à la défense et au déplacement dans la zone acquièrent un atout. Chaque fois que vous vous familiarisez avec une nouvelle zone, vous perdez le focus sur une zone précédente à moins que vous ne dépensiez 1 XP pour conserver la familiarité de façon permanente. Action à initier, une heure à réaliser. (Familiarize \textendash (138))

\subsection*{Fantôme}\label{subsec:ab_ghost}

Pendant les dix prochaines minutes, vous gagnez un atout pour vous faufiler. Pendant ce temps, vous pouvez vous déplacer à travers des barrières solides (mais pas des barrières énergétiques) à raison de 30 cm par tour, et vous pouvez percevoir en étant en phase à l'intérieur d'une barrière ou d'un objet, ce qui vous permet de regarder à travers les murs. Action à initier. (Ghost \textendash (145))

\subsection*{Feinte}\label{subsec:ab_feint}

Si vous utilisez une action créant une mauvaise direction ou une diversion, au tour suivant, vous pourrez profiter des défenses réduites de votre adversaire. Effectuez un jet d'attaque au corps à corps contre cet adversaire. Vous gagnez un atout sur cette attaque. Si votre attaque réussit, elle inflige 4 points de dégâts supplémentaires. Action. (Feint \textendash (139))

\subsection*{Feu et Glace}\label{subsec:ab_fire_and_ice}

Vous faites en sorte qu'une cible à courte portée devienne soit très chaude, soit très froide (votre choix). La cible subit 3 points de dégâts ambiants (ignore l'armure) à chaque tour pendant trois tours maximum, bien qu'un nouveau jet soit nécessaire à chaque tour pour continuer à affecter la cible. Action à initier. (Fire and Ice \textendash (140))

\subsection*{Feuillage agrippant}\label{subsec:ab_grasping_foliage}

Les racines, les branches, l'herbe ou tout autre feuillage naturel de la zone accrochent et retiennent un ennemi que vous désignez à courte portée pendant une minute maximum. Un ennemi pris dans le feuillage avide ne peut pas bouger de sa position, et toutes les tâches physiques, attaques et défenses sont entravées, y compris les tentatives de libération. En plus des options normales d'utilisation de l'Effort, vous pouvez choisir d'utiliser l'Effort pour infliger des dégâts lors de l'attaque initiale. Chaque niveau appliqué inflige 2 points de dégâts supplémentaires lorsque Feuillage agrippant accroche et retient pour la première fois votre ennemi.Vous pouvez également utiliser cette capacité pour dégager une zone de végétation enchevêtrée dans le rayon immédiat, comme une zone d'herbes hautes, de broussailles épaisses, de vignes impénétrables, etc. Action. (Grasping Foliage \textendash (146))

\subsection*{Fièvre sanguinolente}\label{subsec:ab_blood_fever}

Lorsque vous n'avez aucun point dans une ou deux Pools, vous gagnez un atout aux jets d'attaque ou de défense (au choix). Facilitateur. (Blood Fever \textendash (115))

\subsection*{Flash}\label{subsec:ab_flash}

Vous créez une explosion d'énergie en un point à courte portée, affectant une zone située à portée immédiate de ce point. Vous devez être en mesure de voir l'endroit où vous comptez centrer l'explosion. L'explosion inflige 2 points de dégâts à toutes les créatures ou objets dans la zone. Si vous appliquez Effort pour augmenter les dégâts, vous infligez 2 points de dégâts supplémentaires par niveau d'Effort (au lieu de 3 points) ; les cibles dans la zone subissent 1 point de dégâts même si vous échouez au jet d'attaque. Action. (Flash \textendash (140))

\subsection*{Floraison réparatrice}\label{subsec:ab_restorative_bloom}

Lorsque Corps en bois ou Grand arbre est actif, vous produisez une fleur, un gland, un fruit ou un objet comestible similaire à base de plante. Une créature qui mange cette nourriture est nourrie pendant une journée complète et restaure sa réserve de Puissance, sa Réserve de Célérité et sa réserve d'intelligence à leurs valeurs maximales, comme si elle était complètement reposée. Manger un deuxième aliment produit par cette capacité dans une journée n'a aucun effet. Si la nourriture n'est pas consommée dans les dix minutes, elle se gâte. Action de produire, action de manger. (Restorative Bloom \textendash (177))

\subsection*{Flotte de robots}\label{subsec:ab_robot_fleet}

vous construisez jusqu'à quatre assistants robots de niveau 2, chacun pas plus grand que vous. (Ils s'ajoutent à l'assistant que vous avez créé au premier niveau avec Assistant Robot, qui a peut-être connu quelques mises à niveau depuis.) Vous et le MJ devez régler les détails de ces robots supplémentaires. Si un robot est détruit, vous pouvez en construire un nouveau (ou réparer l'ancien à partir de ses pièces) après une semaine de travail à mi-temps. Au lieu de cette capacité, vous pouvez sélectionner l'une des capacités suivantes : Disciple expert, Contrôle du robot ou Mise à niveau du robot. Facilitateur. (Robot Fleet \textendash (179))

\subsection*{Fléau des Monstres}\label{subsec:ab_monster_bane}

Vous infligez 1 point de dégâts supplémentaire avec les armes. Lorsque vous infligez des dégâts à des créatures plus de deux fois plus grandes ou massives que vous, vous infligez 3 points de dégâts supplémentaires. Facilitateur. (Monster Bane \textendash (164))

\subsection*{Fléau des Monstres Amélioré}\label{subsec:ab_improved_monster_bane}

lorsque vous infligez des dégâts à des créatures plus de deux fois plus grandes ou massives que vous, vous infligez 3 points de dégâts supplémentaires. Facilitateur. (Improved Monster Bane \textendash (152))

\subsection*{Fléau des Monstres Géants}\label{subsec:ab_heroic_monster_bane}

lorsque vous infligez des dégâts à des créatures plus de deux fois plus grandes ou massives que vous, vous infligez 3 points de dégâts supplémentaires. Facilitateur. (Heroic Monster Bane \textendash (149))

\subsection*{Force avec laquelle il faut compter}\label{subsec:ab_force_to_reckon_with}

Vous pouvez briser les champs de force et les barrières énergétiques comme s'il s'agissait de murs physiques. Facilitateur. (Force to Reckon With \textendash (143))

\subsection*{Force enchevêtrante}\label{subsec:ab_entangling_force}

Une cible à courte portée est soumise à un piège construit de lignes de force semi-tangibles pendant une minute. Le piège de force est une construction de niveau 2. Une cible prise dans le piège de force ne peut pas bouger de sa position, mais elle peut attaquer et se défendre normalement. La cible peut également utiliser son action pour tenter de se libérer. Vous pouvez augmenter le niveau du piège de force de 1 par niveau d'effort appliqué. Action à initier. (Entangling Force \textendash (136))

\subsection*{Force et Précision}\label{subsec:ab_force_and_accuracy}

Vous infligez 3 points de dégâts supplémentaires avec les attaques utilisant les armes que vous lancez. Facilitateur. (Force and Accuracy \textendash (143))

\subsection*{Force à distance}\label{subsec:ab_force_at_distance}

Vous pliez temporairement la loi fondamentale de la gravité autour d'une créature ou d'un objet (jusqu'à deux fois votre masse) à courte portée. La cible est prise dans votre emprise télékinésique et vous pouvez la déplacer sur une courte distance dans n'importe quelle direction à chaque round pendant que vous conservez votre prise. Une créature sous votre emprise peut entreprendre des actions, mais elle ne peut pas se déplacer par ses propres moyens. Chaque tour après l'attaque initiale, vous pouvez tenter de garder votre emprise sur la cible en dépensant 2 points d'Intellect supplémentaires et en réussissant une tâche d'Intellect de difficulté 2. Si votre concentration diminue, la cible retombe au sol. En plus des options normales d'utilisation de l'Effort, vous pouvez choisir d'utiliser l'Effort pour augmenter la quantité de masse que vous pouvez affecter. Chaque niveau vous permet d'affecter une créature ou un objet deux fois plus massif qu'auparavant. Par exemple, appliquer un niveau d'Effort affecterait une créature quatre fois plus massive que vous, deux niveaux affecteraient une créature huit fois plus massive, trois niveaux affecteraient une créature seize fois plus massive, et ainsi de suite. Action à initier. (Force at Distance \textendash (143))

\subsection*{Formation magique}\label{subsec:ab_magic_training}

Vous êtes entraîné aux bases de la magie (y compris le fonctionnement des artefacts magiques et des cyphers) et pouvez tenter de comprendre et d'identifier ses propriétés. Facilitateur. (Magic Training \textendash (159))

\subsection*{Forme animale}\label{subsec:ab_animal_shape}

Vous vous transformez en un animal aussi petit qu'un rat ou jusqu'à votre propre taille (comme un gros chien ou un petit ours) pendant dix minutes. Chaque fois que vous vous transformez, vous pouvez prendre une forme animale différente. Votre équipement devient partie intégrante de la transformation, le rendant inutilisable à moins qu'il n'ait un effet passif, comme une armure. Sous cette forme, vos statistiques restent les mêmes que votre forme normale, mais vous pouvez vous déplacer et attaquer en fonction de la forme de votre animal (les attaques de la plupart des animaux de cette taille sont des armes moyennes, que vous pouvez utiliser sans pénalité). Les tâches nécessitant des mains (comme utiliser des poignées de porte ou appuyer sur des boutons) sont gênées lorsqu'ils sont sous forme animale. Vous ne pouvez pas parler mais pouvez toujours utiliser des capacités qui ne dépendent pas de la parole humaine. Vous gagnez deux capacités mineures associées à la créature que vous devenez (voir le tableau des capacités mineures de la forme animale à la fin de la section Capacités du personnage de ce document). Par exemple, si vous vous transformez en chauve-souris, vous êtes entraîné à la perception et pouvez voler sur une longue distance à chaque tour. Si vous vous transformez en pieuvre, vous êtes entraîné à la furtivité et pouvez respirer sous l'eau. Si vous appliquez un niveau d'effort lorsque vous utilisez cette capacité, vous pouvez soit devenir un animal qui parle, soit prendre une forme hybride. La forme de l'animal qui parle ressemble exactement à celle d'un animal normal, mais vous pouvez toujours parler et utiliser toutes les capacités qui dépendent de la parole humaine. La forme hybride est comme votre forme normale mais avec des caractéristiques animales, même si cet animal est beaucoup plus petit que vous (comme une chauve-souris ou un rat). Dans cette forme hybride, vous pouvez parler, utiliser toutes vos capacités, effectuer des attaques comme un animal et effectuer des tâches en utilisant vos mains sans être gêné. Quiconque vous voit clairement sous cette forme hybride ne vous prendra jamais pour un simple animal. Action de modifier ou de revenir en arrière.« Similaire » est un terme large. Les lions ressemblent aux tigres et aux léopards, les faucons aux corbeaux et aux cygnes, les chiens aux loups et aux renards, etc. Même si la forme de votre animal a plusieurs types d'attaques (comme les griffes et les morsures), vous ne pouvez attaquer qu'une seule fois par tour, à moins que vous ne disposiez d'une autre capacité vous permettant d'effectuer des attaques supplémentaires pendant votre tour. (Animal Shape \textendash (GF, 29)(CTS, 49))

\subsection*{Forme animale plus grande}\label{subsec:ab_bigger_animal_shape}

lorsque vous utilisez Forme animale, la forme de votre animal atteint environ le double de sa taille normale. Etant si grande, votre forme animale bénéficie des bonus supplémentaires suivants: +1 à l'armure, +5 à votre réserve de Puissance, et vous êtes entraîné à utiliser les attaques naturelles de votre forme animale comme des armes lourdes (si ce n'est pas déjà fait). Cependant, vos tâches de défense de Célérité sont gênées. Bien que plus grand, vous gagnez également un atout pour des tâches plus faciles à accomplir pour une créature plus grande, comme grimper, intimider, patauger dans des rivières, etc. Facilitateur. (Bigger Animal Shape \textendash (113))

\subsection*{Forme de bête}\label{subsec:ab_beast_form}

cinq nuits consécutives chaque mois, vous vous transformez en bête monstrueuse pendant une heure maximum chaque nuit. Dans cette nouvelle forme, vous gagnez +8 à votre Réserve de Puissance, +1 à votre Avantage de Puissance, +2 à votre Réserve de Célérité et +1 à votre Avantage de Célérité. Sous forme de bête, vous ne pouvez pas dépenser de points d'Intellect pour une raison autre que d'essayer de reprendre votre forme normale avant la fin de la durée d'une heure (une tâche de difficulté 2). De plus, vous attaquez toute créature vivante à courte portée. Après être revenu à votre forme normale, vous subissez une pénalité de -1 à tous les jets pendant une heure. Si vous n'avez pas tué et mangé au moins une créature substantielle alors que vous étiez sous forme de bête, la pénalité passe à -2 et affecte tous vos jets pendant les prochaines 24 heures. Action de revenir en arrière. (Beast Form \textendash (112))

\subsection*{Forme de bête Supérieure}\label{subsec:ab_greater_beast_form}

lorsque vous utilisez la Forme de bête, votre forme de bête gagne les bonus supplémentaires suivants : +1 à votre Avantage de Puissance, +2 à votre Réserve de Célérité et +1 à votre Avantage de Célérité. Facilitateur. (Greater Beast Form \textendash (146))

\subsection*{Forme de bête améliorée}\label{subsec:ab_enhanced_beast_form}

lorsque vous utilisez Forme de bête, votre forme de bête gagne les bonus supplémentaires suivants : +3 à votre réserve de Puissance, +2 à votre Réserve de Célérité et +2 à l'armure. Facilitateur. (Enhanced Beast Form \textendash (134))

\subsection*{Forme de bête plus grande}\label{subsec:ab_bigger_beast_form}

lorsque vous utilisez Forme de bête, votre forme de bête devient plus grande qu'auparavant, période pendant laquelle vous atteignez une hauteur de 12 pieds (4 m).Etant aussi grande, votre forme de bête bénéficie des bonus supplémentaires suivants: +1 à l'armure, +5 à votre réserve de Puissance, et vous êtes entraîné à utiliser vos poings comme des armes lourdes (si ce n'est pas déjà fait). Cependant, vos tâches de défense de Célérité sont gênées. Bien que plus grand, vous gagnez également un atout pour des tâches plus faciles à accomplir pour une créature plus grande, comme grimper, intimider, patauger dans des rivières, etc. Facilitateur. (Bigger Beast Form \textendash (113))

\subsection*{Forêt Terrifiante}\label{subsec:ab_dreadwood}

Vous manipulez le vent, la brume et les ombres pour incarner la peur primordiale des bois mystérieux. Pendant la minute suivante, vous gagnez un atout sur les tâches d'intimidation. Les créatures à courte portée peuvent devenir effrayées ; effectuez un jet d'attaque d'Intellect séparé pour chaque créature (si vous êtes plus grand que la normale en utilisant le Grand Arbre ou une autre source, ces jets sont facilités). Le succès signifie qu'ils sont figés dans la peur, ne bougent pas et n'agissent pas pendant une minute ou jusqu'à ce qu'ils soient attaqués. Certaines créatures sans esprit pourraient être immunisées contre cette peur. Action. (Dreadwood \textendash (GF, 31))

\subsection*{Frappe Explosive}\label{subsec:ab_power_crash}

vous frappez votre arme enchantée contre le sol (ou une grande surface similaire), créant une explosion d'énergie qui affecte une zone à portée immédiate de ce point. (Si votre arme enchantée est une arme à distance, vous pouvez plutôt cibler un point à courte portée pour être le centre de l'explosion.) L'explosion inflige 2 points de dégâts à toutes les créatures ou objets dans la zone (sauf vous). Puisqu'il s'agit d'une attaque de zone, l'ajout d'Effort pour augmenter vos dégâts fonctionne différemment que pour les attaques à cible unique. Si vous appliquez un niveau d'Effort pour augmenter les dégâts, ajoutez 2 points de dégâts pour chaque cible, et même si vous échouez à votre jet d'attaque, toutes les cibles dans la zone subissent toujours 1 point de dégâts. Action. (Power Crash \textendash (GF, 32)(CTS, 54))

\subsection*{Frappe Renversante}\label{subsec:ab_power_strike}

Si vous réussissez à attaquer une cible, vous la mettez à terre en plus de lui infliger des dégâts. La cible doit être de votre taille ou plus petite. Vous pouvez renverser une cible plus grande que vous si vous appliquez un niveau d'Effort pour ce faire (plutôt que pour faciliter l'attaque). Facilitateur. (Power Strike \textendash (171))

\subsection*{Frappe d'Assassin}\label{subsec:ab_assassin_strike}

Si vous réussissez à attaquer une créature qui ignorait auparavant votre présence, vous infligez 9 points de dégâts supplémentaires. Facilitateur. (Assassin Strike \textendash (110))

\subsection*{Frappe de Force}\label{subsec:ab_force_bash}

Il s'agit d'une attaque de mêlée percutante que vous effectuez avec votre Champ de force Shield. Votre attaque inflige 1 point de dégâts de moins que la normale mais étourdit votre cible pendant un round, durant lequel toutes les tâches qu'elle effectue sont gênées. Facilitateur. (Force Bash \textendash (142))

\subsection*{Frappe de matière noire}\label{subsec:ab_dark_matter_strike}

Lorsque vous attaquez un ennemi à longue portée, la matière noire se condense autour de votre cible et enchevêtre ses membres, la maintenant en place et facilitant votre attaque de deux pas. Cette capacité fonctionne quel que soit le type d'attaque que vous utilisez (mêlée, à distance, énergétique, etc.). Facilitateur. (Dark Matter Strike \textendash (124))

\subsection*{Frappe débilitante}\label{subsec:ab_debilitating_strike}

Vous effectuez une attaque pour délivrer une frappe douloureuse ou débilitante. L'attaque est entravée. Si elle touche, la créature subit 2 points de dégâts supplémentaires à la fin du tour suivant, et ses attaques sont gênées jusqu'à la fin du tour suivant. Action. (Debilitating Strike \textendash (126))

\subsection*{Frappe désarmante}\label{subsec:ab_disarming_strike}

Votre attaque inflige 1 point de dégâts en moins et désarme votre ennemi afin que son arme soit désormais à 10 pieds (3 m) au sol. (Si l'arme choisie est un fouet, vous pouvez à la place déposer l'arme désarmée entre vos mains ; si l'arme choisie est un arc ou une autre arme à distance qui tire des balles physiques, vous pouvez plutôt « clouer » l'arme désarmée sur un objet proche ou Choisir de faire l'une ou l'autre de ces choses entrave votre attaque.) Action. (Disarming Strike \textendash (129))

\subsection*{Frappe mortelle}\label{subsec:ab_deadly_strike}

Si vous frappez un ennemi de niveau 3 ou inférieur avec une arme avec laquelle vous vous entraînez, vous tuez la cible instantanément. Action. (Deadly Strike \textendash (125))

\subsection*{Frappe nocturne}\label{subsec:ab_nightstrike}

lorsque vous attaquez un ennemi dans une faible lumière ou dans l'obscurité, vous obtenez un niveau d'effort gratuit lors de l'attaque. Facilitateur. (Nightstrike \textendash (166))

\subsection*{Frappe rapide}\label{subsec:ab_quick_strike}

Vous effectuez une attaque au corps à corps avec une telle Célérité qu'il est difficile pour votre ennemi de se défendre, et cela le déséquilibre. Votre attaque est facilitée de deux pas et l'ennemi, s'il est touché, subit des dégâts normaux mais est étourdi, de sorte que ses tâches sont entravées pour le tour suivant. Action. (Quick Strike \textendash (174))

\subsection*{Frénésie}\label{subsec:ab_frenzy}

Quand vous le souhaitez, pendant un combat, vous pouvez entrer dans un état de frénésie. Dans cet état, vous ne pouvez pas utiliser de points d'Intellect, mais vous gagnez +1 à votre Avantage de Puissance et à votre Avantage de Célérité. Cet effet dure aussi longtemps que vous le souhaitez, mais il prend fin si aucun combat n'a lieu à portée de vos sens. Facilitateur. (Frenzy \textendash (143))

\subsection*{Frénésie supérieure}\label{subsec:ab_greater_frenzy}

Quand vous le souhaitez, pendant un combat, vous pouvez entrer dans un état de frénésie. Dans cet état, vous ne pouvez pas utiliser de points d'Intellect, mais vous gagnez +2 à votre Avantage de Puissance et à votre Avantage de Vietsse. Cet effet dure aussi longtemps que vous le souhaitez, mais il prend fin si aucun combat n'a lieu à portée de vos sens. Si vous possédez la capacité Frénésie, vous pouvez l'utiliser ou cette capacité, mais vous ne pouvez pas utiliser les deux en même temps. Facilitateur. (Greater Frenzy \textendash (146))

\subsection*{Fréquence de résonance}\label{subsec:ab_resonant_frequency}

Vous pouvez faire entre en résonance un objet jusqu'au niveau 7 que vous pouvez tenir dans une main avec une vibration spéciale générée par votre cœur. L'objet fonctionne alors comme s'il était deux niveaux plus haut pendant une minute. À la fin de cette minute, la fréquence de résonance augmente de façon exponentielle jusqu'à ce que l'objet se brise finalement à cause de l'accumulation d'énergie. Tout ce qui se trouve à portée immédiate de la détonation subit 5 points de dégâts. Action à initier. (Resonant Frequency \textendash (177))

\subsection*{Fuir}\label{subsec:ab_flee}

Tous les non-alliés à courte distance qui peuvent entendre vos paroles terribles et intimidantes fuient vous à toute Célérité pendant une minute. Action. (Flee \textendash (141))

\subsection*{Fureur}\label{subsec:ab_fury}

Pendant la minute suivante, toutes les attaques de mêlée que vous effectuez infligent 2 points de dégâts supplémentaires. Action à initier. (Fury \textendash (144))

\subsection*{Fureur du berger}\label{subsec:ab_shepherd's_fury}

vous infligez 3 points de dégâts supplémentaires lorsque vous engagez un combat directement lié à l'avancement des besoins d'une communauté à laquelle vous êtes associé. (Vous et le MJ pouvez décider si une situation particulière justifie des dégâts supplémentaires.) Facilitateur. (Shepherd's Fury \textendash (182))

\subsection*{Furtif}\label{subsec:ab_sneak}

vous êtes entraîné aux tâches de furtivité et d'initiative. Facilitateur. (Sneak \textendash (183))

\subsection*{Fusion}\label{subsec:ab_fusion}

Vous pouvez fusionner vos chiffres et artefacts manifestes avec votre corps. Ces appareils fusionnés fonctionnent comme s'ils étaient d'un niveau supérieur. Facilitateur. (Fusion \textendash (144))

\subsection*{Fusionne avec les Ténèbres}\label{subsec:ab_embraced_by_darkness}

Durant l'heure suivante, vous adoptez certaines caractéristiques d'une ombre grâce à une adaptation fondamentale de votre chair ou d'un appareil que vous avez gardé secret. Votre apparence est une silhouette sombre. Lorsque vous appliquez un niveau d'effort à des tâches de furtivité, vous obtenez un niveau d'effort gratuit sur la tâche. Pendant ce temps, vous pouvez vous déplacer dans les airs à une vitesse d'une courte distance par round, et vous pouvez traverser des barrières solides (même celles qui sont scellées pour empêcher le passage de la lumière ou de l'ombre), mais pas les barrières énergétiques, à un rythme réduit. taux de 1 pied (30 cm) par tour. Vous pouvez percevoir en passant à travers une barrière ou un objet, ce qui vous permet de regarder à travers les murs. En tant qu'ombre, vous ne pouvez pas affecter ou être affecté par la matière normale. De même, vous ne pouvez pas attaquer, toucher ou affecter quoi que ce soit. Cependant, les attaques et les effets qui dépendent de la lumière peuvent vous affecter, et des éclats de lumière soudains peuvent potentiellement vous faire perdre votre prochain tour. Action à initier. (Embraced by Darkness \textendash (133))

\subsection*{Fustiger}\label{subsec:ab_castigate}

Vous intimidez tellement tout adversaire à longue portée qui comprend la parole (même si ce n'est pas votre langue) qu'il perd sa prochaine action et que toutes ses autres actions sont gênées pendant une minute. Chaque fois que vous tentez cette capacité contre la même cible, vous devez appliquer un niveau d'effort de plus que celui appliqué lors de la tentative précédente. Action. (Castigate \textendash (118))

\
%--------------------------
\section*{G}

\subsection*{Gagner un compagnon inhabituel}\label{subsec:ab_gain_unusual_companion}

vous gagnez un spécimen spécial en tant que compagnon permanent. Il est de niveau 4, probablement de la taille d'un petit chien, et suit vos ordres télépathiques. Vous et le MJ devez définir les détails de votre créature, et vous ferez probablement des jets pour elle en combat ou lorsqu'elle entreprend des actions. Le compagnon agit à votre tour. Si votre compagnon meurt, vous pouvez chasser dans la nature pendant 1d6 jours pour en trouver un nouveau. Facilitateur. (Gain Unusual Companion \textendash (144))

\subsection*{Gantelets d'hiver}\label{subsec:ab_winter_gauntlets}

Lorsque vous utilisez Toucher Glacial, vous infligez 3 points de dégâts supplémentaires si vous touchez une créature, ou 2 points de dégâts supplémentaires si vous infusez une arme. De plus, les cibles endommagées sont figées sur place (si elles se trouvent sur une surface solide) et ne peuvent pas bouger de leur emplacement jusqu'à ce qu'elles utilisent une action pour se libérer. La cible peut toujours attaquer et se défendre. Action pour le toucher ; facilitateur pour arme. (Winter Gauntlets \textendash (199))

\subsection*{Gargantuesque}\label{subsec:ab_gargantuan}

lorsque vous utilisez Agrandir, vous pouvez choisir de grandir jusqu'à 9 m de hauteur et vous ajoutez 3 points temporaires supplémentaires à votre réserve de Puissance (si vous disposez également de la capacité Plus grand les points temporaires de Gargantuesque sont dans en plus des points de Plus grand). Facilitateur. (Gargantuan \textendash (144))

\subsection*{Gestion de bataille}\label{subsec:ab_battle_management}

Tant que vous utilisez votre action à chaque tour pour donner des ordres ou des conseils, les actions d'attaque et de défense entreprises par vos alliés à courte portée sont facilitées. Action. (Battle Management \textendash (112))

\subsection*{Glissant}\label{subsec:ab_slippery}

Vous êtes entraîné pour échapper à tout type de lien ou d'emprise. Facilitateur. (Slippery \textendash (183))

\subsection*{Graine de Tempête}\label{subsec:ab_storm_seed}

Si vous êtes à l'extérieur ou dans un espace clos suffisamment grand, vous pouvez déclencher une tempête naturelle d'un type commun à la région. Cela nécessite au moins une heure de concentration pendant que vous utilisez votre connexion à l'air (que cela soit dû à des nanobots, des esprits élémentaires, de la magie ou une autre source) pour initier les conditions appropriées, bien que cela puisse prendre plus de temps si le MJ estime qu'il y en a. obstacles supplémentaires en jeu. Une fois que la tempête commence, elle dure environ dix minutes. Une fois au cours de cette période, vous pouvez créer un effet plus dramatique et spécifique adapté à ce type de tempête, comme un éclair, une rafale de grêlons géants, le bref atterrissage d'une tornade, une seule rafale de vent de force ouragan, et bientôt. Ces effets doivent se produire à grande distance de votre emplacement. Vous devez passer votre tour à vous concentrer pour créer l'effet, qui se produit un tour plus tard. L'effet inflige 6 points de dégâts, après quoi la tempête commence à se disperser. Action à initier, une heure ou plus à réaliser. (Graine de Tempête appelle généralement des orages, mais dans une zone où le temps est plus étrange, un Graine de Tempête pourrait appeler cela à la place. Par exemple, certains paramètres ont des types particuliers de temps magique.) (Storm Seed \textendash (187))

\subsection*{Graines de fureur}\label{subsec:ab_seeds_of_fury}

vous lancez une poignée de graines dans l'air qui s'enflamment et se précipitent vers une cible à longue portée, grattant l'air avec des traînées de fumée tordues. L'attaque inflige 3 points de dégâts et prend feu à la cible, ce qui inflige 1 point de dégâts supplémentaire par round pendant une minute maximum ou jusqu'à ce que la cible utilise une action pour éteindre les flammes. Action. (Seeds of Fury \textendash (181))

\subsection*{Grand Pas}\label{subsec:ab_far_step}

Vous sautez dans les airs et atterrissez à une certaine distance. Vous pouvez sauter vers le haut, vers le bas ou vers n'importe quel endroit de votre choix à longue portée si vous disposez d'un chemin clair et dégagé vers cet endroit. Vous atterrissez en toute sécurité. Action. (Far Step \textendash (138))

\subsection*{Grand arbre}\label{subsec:ab_great_tree}

lorsque vous utilisez un Corps en bois,, vous pouvez atteindre jusqu'à 4 m de hauteur. Dans cette forme plus grande, vous ajoutez 7 points à votre Réserve de Puissance et +2 à votre Avantage de Puissance. Si vous choisissez de grandir, lorsque Corps en bois se termine, vous soustrayez 7 points de votre Réserve de Puissance (si cela ramène la réserve à 0, soustrayez d'abord le débordement de votre Réserve de Célérité puis, si nécessaire, de votre réserve d'intelligence). Lorsque vous utilisez Corps en bois que vous choisissiez ou non de grandir, au lieu de ressembler à une version en bois de vous-même normal, vous pouvez prendre l'apparence complète d'une créature humanoïde ou d'un arbre réel (y compris la croissance de branches supplémentaires, de feuillage supplémentaire , et ainsi de suite). Cela n'affecte aucune de vos capacités : sous forme d'arbre, vous pouvez utiliser des capacités de type, d'autres capacités de concentration, etc. En forme d'arbre, faire semblant d'être un arbre et se cacher parmi les arbres normaux est facilité par deux étapes. Facilitateur. (Great Tree \textendash (146))

\subsection*{Grande Déception}\label{subsec:ab_grand_deception}

Vous convainquez une créature intelligente qui peut vous comprendre et qui n'est pas hostile à quelque chose qui est extrêmement et manifestement faux. Action. (Grand Deception \textendash (146))

\subsection*{Gravité Coupante}\label{subsec:ab_gravity_cleave}

Vous pouvez blesser une cible à courte portée en augmentant rapidement l'attraction gravitationnelle sur une partie de la cible et en la diminuant sur une autre, infligeant 6 points de dégâts. Action. (Gravity Cleave \textendash (146))

\subsection*{Gravité Coupante Améliorée}\label{subsec:ab_improved_gravity_cleave}

Vous pouvez nuire à un groupe de cibles à longue portée en augmentant rapidement l'attraction gravitationnelle sur une partie de chaque cible et en la diminuant sur une autre, infligeant 6 points de dégâts. Les cibles doivent être à portée immédiate les unes des autres. Action. (Improved Gravity Cleave \textendash (151))

\subsection*{Grondement subsonique}\label{subsec:ab_subsonic_rumble}

Pendant une minute ou jusqu'à ce que vous utilisiez une autre capacité de manipulation sonore, vous émettez un grondement subsonique que la plupart des créatures vivantes ne peuvent pas entendre mais qui a tout de même un effet sur elles. L'effet dure une minute et affecte toutes les créatures que vous sélectionnez à courte portée. Toutes les tâches liées à la résistance à la persuasion, à l'intimidation et à la peur sont entravées par deux étapes pour les cibles affectées. Action à initier. (Subsonic Rumble \textendash (187))

\subsection*{Gros Mensonge}\label{subsec:ab_tall_tale}

vous racontez une courte anecdote à un ennemi qui peut vous comprendre à propos de quelque chose dont vous avez été témoin dans votre vie et qui est tellement exagéré et pourtant si convaincant que, si vous réussissez, l'ennemi est abasourdi pendant un moment. minute, pendant laquelle ses tâches sont entravées. Action. (Tall Tale \textendash (189))

\subsection*{Guerrier Capable}\label{subsec:ab_capable_warrior}

Vos attaques infligent 1 point de dégâts supplémentaire. Facilitateur. (Capable Warrior \textendash (118))

\subsection*{Gueule de Dragon}\label{subsec:ab_dragon's_maw}

Vous façonnez et contrôlez une construction magique fantasmatique « planante » à longue portée qui ressemble à une tête de dragon. La construction dure jusqu'à une heure, jusqu'à ce qu'elle soit détruite ou jusqu'à ce que vous lanciez un autre sort. C'est une construction de niveau 4 qui inflige 6 points de dégâts avec sa morsure lorsqu'elle est dirigée. Tant que la construction persiste, vous pouvez l'utiliser pour manipuler de gros objets, transporter des objets lourds dans sa bouche ou attaquer des ennemis. Si vous l'utilisez pour attaquer des ennemis, vous devez utiliser votre action pour contrôler directement la gueule fantôme à chaque attaque. Action à initier. (Dragon's Maw \textendash (131))

\subsection*{Guide des eaux profondes}\label{subsec:ab_deep_water_guide}

sous l'eau, toute créature que vous choisissez et qui peut vous voir possède un atout pour les tâches de nage. Facilitateur. (Deep Water Guide \textendash (126))

\subsection*{Guérison biomorphique}\label{subsec:ab_biomorphic_healing}

Vous envoyez consciemment une impulsion de votre champ biomorphique (une énergie étrange générée par votre corps) et la concentrez sur une créature vivante à courte portée. La cible bénéficie d'un jet de récupération gratuit et immédiat en une seule action. Vous ne pouvez plus utiliser cette capacité sur cette créature avant son prochain repos de dix heures. Action. (Biomorphic Healing \textendash (113))

\subsection*{Guérison du Golem}\label{subsec:ab_golem_healing}

votre forme de pierre issue de la capacité Corps du Golem est plus difficile à réparer que la chair, ce qui signifie que vous ne pouvez pas utiliser le premier jet de récupération à action unique de la journée auquel les autres PJ ont accès. Ainsi, votre premier jet de récupération un jour donné nécessite dix minutes de repos, le deuxième nécessite une heure de repos et le troisième nécessite dix heures. Facilitateur. (Golem Healing \textendash (145))

\subsection*{Générer un champ de force}\label{subsec:ab_generate_force_field}

Vous créez six plans de force solide (niveau 8), chacun de 30 pieds (9 m) de côté, qui persistent pendant une heure. Les plans doivent être contigus et ils conservent la position que vous choisissez lors du lancement de cette capacité. Par exemple, vous pouvez disposer les plans de manière linéaire, créant un mur de 55 m de long, ou vous pouvez créer un cube fermé. Les plans sont conformes à l'espace disponible. Chaque niveau d'effort supplémentaire que vous appliquez augmente le niveau de la barrière de un (jusqu'à un maximum de niveau 10) ou augmente le nombre d'heures pendant lesquelles elle reste d'un. Action à initier. (Generate Force Field \textendash (145))

\
%--------------------------
\section*{H}

\subsection*{Habiletés motrices}\label{subsec:ab_movement_skills}

Vous êtes entraîné à l'escalade et au saut. Facilitateur. (Movement Skills \textendash (164))

\subsection*{Habiter le cristal}\label{subsec:ab_inhabit_crystal}

Vous transférez votre corps et tout ce que vous transportez dans un cristal au moins de la taille de votre index. Lorsque vous êtes dans le cristal, vous êtes conscient de ce qui se passe autour de lui, voyant et entendant à travers le cristal. Vous pouvez même parler à travers le cristal et poursuivre des conversations. Vous ne pouvez pas entreprendre d'autres actions que quitter le cristal. Vous restez à l'intérieur aussi longtemps que vous le souhaitez, mais vous n'êtes pas en stase et devez sortir pour manger, boire, dormir, etc. normalement (la respiration n'est pas un problème). Si le cristal est détruit ou subit des dégâts importants alors que vous y êtes, vous sortez immédiatement, ne pouvez pas agir pendant trois tours et descendez de deux crans sur la piste des dégâts. Action d'entrer et de sortir. (Un personnage doit préciser où il place le cristal pour la capacité Habiter le cristal avant de l'utiliser, même s'il est juste sur le sol, à ses pieds.) (Inhabit Crystal \textendash (154))

\subsection*{Hacker}\label{subsec:ab_hacker}

Vous obtenez un accès rapide à une information souhaitée dans un ordinateur ou un appareil similaire, ou vous accédez à l'une de ses fonctions principales. Action. (Hacker \textendash (147))

\subsection*{Hackez l'impossible}\label{subsec:ab_hack_the_impossible}

Vous pouvez persuader les robots, les machines et les ordinateurs d'exécuter vos ordres. Vous pouvez découvrir un mot de passe crypté, briser la sécurité d'un site Web, éteindre brièvement une machine telle qu'une caméra de surveillance ou désactiver un robot en un instant. Action. (Hack the Impossible \textendash (147))

\subsection*{Hors de danger}\label{subsec:ab_out_of_harm's_way}

Même si il est prudent, un enquêteur se retrouve parfois dans une bagarre. Savoir comment survivre représente plus de la moitié de la bataille. Vous êtes entraîné aux tâches de défense de Célérité. Facilitateur. (Out of Harm's Way \textendash (167))

\subsection*{Hémorragie}\label{subsec:ab_hemorrhage}

Vous effectuez une frappe puissante et précise qui inflige des dégâts supplémentaires plus tard. Lors de votre prochain tour, la cible de cette attaque subit 3 points de dégâts supplémentaires (ignore l'armure). La cible peut éviter ces dégâts supplémentaires en effectuant un jet de récupération, en utilisant n'importe quelle capacité qui la soigne ou en utilisant son action pour soigner la blessure. En plus des options normales d'utilisation de l'Effort, vous pouvez choisir d'utiliser l'Effort pour augmenter cette durée d'un tour. Action. (Hemorrhage \textendash (149))

\
%--------------------------
\section*{I}

\subsection*{Ignorer l'Affliction}\label{subsec:ab_ignore_affliction}

si vous êtes affecté par une condition ou une affliction non désirée (comme une maladie, une paralysie, un contrôle mental, un membre cassé, etc., mais pas de dommage), vous pouvez l'ignorer et agir comme si c'était le cas. ne vous affecte pas pendant une heure. Si la condition dure normalement moins d'une heure, elle est entièrement annulée. Action. (Ignore Affliction \textendash (150))

\subsection*{Ignorez la Douleur}\label{subsec:ab_ignore_the_pain}

vous ignorez l'état Handicapé et traitez l'état Handicapé comme Diminué. Facilitateur. (Ignore the Pain \textendash (150))

\subsection*{Illusion grandiose}\label{subsec:ab_grandiose_illusion}

Vous créez une scène incroyablement complexe d'images qui s'inscrivent dans un cube de 1,5 km dans lequel vous vous trouvez également. Vous devez pouvoir voir les images lorsque vous les créez. Les images peuvent se déplacer dans le cube et agir selon vos envies. Ils peuvent également agir de manière logique (par exemple réagir de manière appropriée aux tirs ou aux attaques) lorsque vous ne les observez pas directement. L'illusion inclut le son et l'odeur. Par exemple, les armées peuvent s'affronter au combat, avec le soutien aérien de machines ou de créatures volantes, sur et au-dessus du terrain de votre création. L'illusion dure une heure (ou plus si vous vous concentrez dessus après ce délai). Action. (Grandiose Illusion \textendash (146))

\subsection*{Illusion majeure}\label{subsec:ab_major_illusion}

Vous créez une scène complexe d'images à portée immédiate. La scène entière doit s'inscrire dans un cube de 100 pieds (30 m). Les images peuvent bouger, mais elles ne peuvent pas quitter la zone définie par le cube. L'illusion inclut le son et l'odeur. Cela dure dix minutes et change selon vos directives (aucune concentration n'est nécessaire). Si vous vous déplacez au-delà de la portée immédiate du cube, l'illusion disparaît. Action de créer. (Major Illusion \textendash (160))

\subsection*{Illusion mineure}\label{subsec:ab_minor_illusion}

vous créez une image unique d'une créature ou d'un objet à portée immédiate. L'image doit tenir dans un rayon de 10 pieds (3 m)cube. L'image peut bouger (par exemple, vous pouvez donner l'illusion d'une personne marchant ou attaquant), mais elle ne peut pas quitter la zone définie par le cube. L'illusion inclut le son mais pas l'odeur. Cela dure dix minutes, mais si vous souhaitez modifier l'illusion d'origine de manière significative, par exemple en donnant l'impression qu'une créature est blessée, vous devez vous concentrer à nouveau dessus (même si cela ne coûte pas de points d'Intellect supplémentaires). Si vous vous déplacez au-delà de la portée immédiate du cube, l'illusion disparaît. Action de créer ; action à modifier. (Minor Illusion \textendash (162))

\subsection*{Illusion permanente}\label{subsec:ab_permanent_illusion}

une illusion (ou une partie d'une illusion) que vous créez à l'aide d'une illusion mineure ou d'une capacité associée qui s'insère dans un cube de 3 m (10 pieds) devient permanente. Vous pouvez mettre fin définitivement à l'illusion par une action. , mais d'autres doivent déployer une ingéniosité exceptionnelle pour empêcher l'illusion de se régénérer même si elle a apparemment été dispersée. Facilitateur. (Permanent Illusion \textendash (169))

\subsection*{Image Miroir}\label{subsec:ab_illusory_selves}

Lorsque vous êtes sur le point d'être touché par une attaque, vous vous téléportez à une distance immédiate, laissant derrière vous une copie illusoire de vous-même qui sera frappée par cette attaque à votre place. Cela détruit l'illusion mais vous laisse indemne par l'attaque. Si l'attaque affecte une zone et que la téléportation ne peut pas vous faire sortir de cette zone, l'attaque vous affecte quand même normalement. Facilitateur. (Illusory Selves \textendash (150))

\subsection*{Image terrifiante}\label{subsec:ab_terrifying_image}

Vous utilisez un peu de télépathie subtile pour savoir quelles images sembleraient terrifiantes aux créatures que vous choisissez à longue portée. Ces images apparaissent dans cette zone et menacent les créatures appropriées. Faites un jet d'attaque d'Intellect contre chaque créature que vous souhaitez affecter. Le succès signifie que la créature s'enfuit terrorisée pendant une minute, poursuivie par ses cauchemars. L'échec signifie que la créature ignore les images, ce qui ne la gêne en rien. Action. (Terrifying Image \textendash (190))

\subsection*{Immobile}\label{subsec:ab_immovable}

vous gagnez +3 à votre Réserve de Puissance. Vous pouvez tenter une tâche de Puissance pour éviter d'être renversé, repoussé ou déplacé contre votre volonté, même si l'effet tentant de vous déplacer ne le permet pas. Si vous appliquez un effort à cette tâche, vous pouvez appliquer deux niveaux d'effort gratuits. Facilitateur. (Immovable \textendash (150))

\subsection*{Improviser}\label{subsec:ab_improvise}

Lorsque vous effectuez une tâche pour laquelle vous n'êtes pas entraîné, vous pouvez improviser pour gagner un atout sur la tâche. L'atout peut être un outil que vous avez bricolé, une idée soudaine pour surmonter un problème ou un coup de chance. Facilitateur. (L'improvisation peut être utilisée sur une tâche pour laquelle un personnage est incapable, mais au lieu de gagner un atout, le personnage perd simplement la pénalité d'incapacité.) (Improvise \textendash (152))

\subsection*{Inamovible}\label{subsec:ab_unmovable}

Vous évitez d'être renversé, repoussé ou déplacé contre votre volonté tant que vous êtes droit et capable d'agir. Facilitateur. (Unmovable \textendash (195))

\subsection*{Incroyable exploit scientifique}\label{subsec:ab_incredible_feat_of_science}

vous faites quelque chose d'incroyable dans le laboratoire. Cela nécessite des pièces et des matériaux équivalents à trois articles coûteux. Les exploits incroyables possibles incluent :- Réanimez et commandez un cadavre pendant une heure. - Créez un moteur qui fonctionne en mouvement perpétuel. - Créez une porte de téléportation qui reste ouverte pendant une minute. - Transmuer une substance en une autre substance. - Guérissez une personne atteinte d'une maladie ou d'un état incurable. - Créez une arme conçue pour blesser quelque chose qui ne pourrait pas l'être autrement. - Créez une défense conçue pour vous protéger contre quelque chose qui ne pourrait pas être arrêté autrement. - Action à initier ; une journée complète de travail à réaliser. (Incredible Feat of Science \textendash (153))

\subsection*{Incroyable récupération}\label{subsec:ab_incredible_recovery}

vous montez d'un cran sur la piste des dégâts ou vous vous débarrassez de toute condition continue indésirable. Action. (Incredible Recovery \textendash (153))

\subsection*{Infiltrateur}\label{subsec:ab_infiltrator}

Vous êtes entraîné aux interactions impliquant des mensonges ou des supercheries. Facilitateur. (Infiltrator \textendash (153))

\subsection*{Infiltrateur Confirmé}\label{subsec:ab_superb_infiltrator}

vous êtes entraîné au crochetage de serrures et au bricolage d'appareils dans le but de les faire fonctionner, ou du moins de les faire fonctionner pour vous. Facilitateur. (Superb Infiltrator \textendash (188))

\subsection*{Influence d'Essaim}\label{subsec:ab_influence_swarm}

Vous maîtrisez un type de petites créatures (comme des insectes, des rats, des chauves-souris ou même des oiseaux) et elles vous répondent en nombre. Vos créatures à courte portée ne vous feront pas de mal, ni à ceux que vous désignez comme alliés, pendant une heure. Action à initier. (Influence Swarm \textendash (153))

\subsection*{Informateur}\label{subsec:ab_informer}

Vous gagnez un informateur au sein d'une communauté alliée. Ils agissent comme votre informateur secret (ou connu). Si quelque chose d'important se produit dans le lieu où se trouve votre informateur, celui-ci utilisera tous les moyens à sa disposition pour vous en informer. Facilitateur. (Informer \textendash (153))

\subsection*{Information sur une Créature}\label{subsec:ab_creature_insight}

Lorsque vous examinez une créature non humaine, vous pouvez poser une question au MJ pour avoir une idée de son niveau, de ses capacités, de ce qu'elle mange, de ce qui la motive, de ses faiblesses (le cas échéant), de la manière dont elle se nourrit. il peut être réparé, ou toute autre requête similaire. Ceci concerne les créatures difficiles ou étranges au-delà de celles facilement identifiées en utilisant des compétences. Action. (Creature Insight \textendash (123))

\subsection*{Innovateur}\label{subsec:ab_innovator}

vous pouvez modifier n'importe quel artefact pour lui donner des capacités différentes ou meilleures, comme si cet artefact était d'un niveau inférieur à la normale, et la modification prend la moitié du temps normal. Facilitateur. (Innovator \textendash (154))

\subsection*{Insaisissable}\label{subsec:ab_elusive}

Lorsque vous réussissez une action de défense Célérité, vous gagnez immédiatement une action. Vous ne pouvez utiliser cette action que pour vous déplacer. Facilitateur. (Elusive \textendash (133))

\subsection*{Inspiration}\label{subsec:ab_inspiration}

Vous prononcez des mots d'encouragement et d'inspiration. Tous les alliés à courte portée qui peuvent vous entendre bénéficient immédiatement d'un jet de récupération, d'une action gratuite immédiate et disposent d'un atout pour cette action gratuite. Le jet de récupération ne compte pas comme l'un de leurs jets de récupération normaux. Action. (Inspiration \textendash (154))

\subsection*{Inspiration tardive}\label{subsec:ab_late_inspiration}

vous réessayez une tâche que vous avez échouée au cours de la dernière minute, en utilisant la même difficulté et les mêmes modificateurs, sauf que cette fois vous disposez d'un atout sur la tâche. Si cette nouvelle tentative échoue, vous ne pouvez pas utiliser cette fonctionnalité pour réessayer. Facilitateur. (Late Inspiration \textendash (157))

\subsection*{Inspire des actions coordonnées}\label{subsec:ab_inspire_coordinated_actions}

Si vos alliés peuvent vous voir et vous comprendre facilement, vous pouvez demander à chacun d'eux d'effectuer une action spécifique (la même action pour tous). Si l'un d'entre eux choisit d'entreprendre cette action précise, il peut le faire immédiatement comme action supplémentaire. Cela ne les empêche pas d'effectuer leurs actions normales à leur tour. Action. (Inspire Coordinated Actions \textendash (154))

\subsection*{Inspire l'Agression}\label{subsec:ab_inspire_aggression}

Vos mots déforment l'esprit d'un personnage à courte portée qui est capable de vous comprendre, débloquant ainsi ses instincts les plus primitifs. En conséquence, ils gagnent un atout sur leurs jets d'attaque basés sur la Puissance pendant une minute. Action à initier. (Inspire Aggression \textendash (154))

\subsection*{Inspirer l'action}\label{subsec:ab_inspire_action}

Si un allié peut vous voir et vous comprendre facilement, vous pouvez lui demander d'agir. Si l'allié choisit d'effectuer exactement cette action, il peut le faire immédiatement comme une action supplémentaire. Cela n'empêche pas l'allié d'effectuer une action normale à son tour. Action. (Inspire Action \textendash (154))

\subsection*{Inspirez les innocents}\label{subsec:ab_inspire_the_innocent}

Vous prononcez des mots d'encouragement et d'inspiration à tous ceux qui se trouvent à portée immédiate et que vous avez désignés comme innocents grâce à votre capacité de désignation. Ils bénéficient immédiatement d'un jet de récupération gratuit. Une personne que vous choisissez peut bénéficier d'une action gratuite immédiate au lieu d'un jet de récupération gratuit. Si vous possédez également la capacité Inspiration, la cible qui gagne une action gratuite gagne également un atout sur celle-ci. Action. (Inspire the Innocent \textendash (154))

\subsection*{Instinct de Danger}\label{subsec:ab_danger_instinct}

Si vous êtes attaqué par surprise, que ce soit par une créature, un engin ou simplement un danger environnemental (un arbre tombant sur vous), vous pouvez vous déplacer sur une distance immédiate avant que l'attaque ne se produise. Si le mouvement empêche l'attaque, vous êtes en sécurité. Si l'attaque peut encore potentiellement vous affecter (si la créature attaquante peut se déplacer pour suivre le rythme, si l'attaque remplit une zone trop grande pour s'échapper, etc.), la capacité n'offre aucun avantage. Facilitateur. (Danger Instinct \textendash (124))

\subsection*{Instinct de Magie Sauvage}\label{subsec:ab_wild_insight}

Vous obtenez momentanément une compréhension parfaite du flux de magie qui vous entoure à ce moment. Lorsque vous préparez votre magie, choisissez un chiffre subtil spécifique et effectuez un jet de compétence de savoir magique contre le niveau 6. Si vous réussissez, vous gagnez ce chiffre subtil (le niveau du chiffre est 6) ; si vous échouez, vous obtenez un chiffre subtil et aléatoire. Si vous n'êtes pas sûr du chiffre subtil spécifique que vous souhaitez, vous pouvez demander une catégorie large telle que « guérison », « mouvement » ou « compétence » ; cela facilite la tâche de connaissance magique, et si vous réussissez, le MJ choisit un chiffre aléatoire qui correspond à cette catégorie. Vous ne pouvez plus utiliser cette capacité avant d'avoir effectué une action de récupération de dix heures. Facilitateur. (Wild Insight \textendash (198))

\subsection*{Intellect Amélioré Supérieur}\label{subsec:ab_greater_enhanced_intellect}

vous gagnez 6 points dans votre réserve d'Intellect. Facilitateur. (Greater Enhanced Intellect \textendash (146))

\subsection*{Intellect amélioré}\label{subsec:ab_enhanced_intellect}

vous gagnez 3 points dans votre réserve d'Intellect. Facilitateur. (Enhanced Intellect \textendash (135))

\subsection*{Interaction intense}\label{subsec:ab_intense_interaction}

Vous gagnez un atout pour intimider, persuader et influencer les gens pendant dix minutes. Action. (Intense Interaction \textendash (155))

\subsection*{Interface}\label{subsec:ab_interface}

En vous connectant directement à un appareil, vous pouvez l'identifier et apprendre à le faire fonctionner comme si la tâche était d'un niveau inférieur. Facilitateur. (Interface \textendash (155))

\subsection*{Interface Machine}\label{subsec:ab_machine_interface}

Pendant une minute, vous gagnez un atout sur les tâches permettant de discerner le niveau, la fonction et l'activation des appareils technologiques que vous touchez. Facilitateur. (Machine Interface \textendash (159))

\subsection*{Interface distante}\label{subsec:ab_distant_interface}

Vous pouvez activer, désactiver ou contrôler une machine à longue distance comme si vous étiez à côté d'elle, même si normalement vous deviez la toucher ou la faire fonctionner manuellement. Si vous n'avez jamais interagi avec une machine en particulier auparavant, la tâche est entravée par deux étapes. Pour utiliser cette capacité, vous devez comprendre la fonction de la machine, elle doit être de votre taille ou plus petite, et elle ne peut pas être connectée à une autre intelligence (ou être elle-même intelligente). Action. (Distant Interface \textendash (130))

\subsection*{Interface intelligente}\label{subsec:ab_intelligent_interface}

Vous pouvez parler par télépathie avec n'importe quelle machine intelligente à longue portée. De plus, vous êtes entraîné à toutes les interactions avec les machines intelligentes. De telles machines et robots qui ne communiqueraient normalement jamais avec un humain pourraient vous parler. Facilitateur. (Intelligent Interface \textendash (155))

\subsection*{Interlocuteur qualifié}\label{subsec:ab_trained_interlocutor}

Grâce à l'esprit, au charme, à l'humour et à la grâce (ou parfois à l'impolitesse, à une posture menaçante et à l'obscénité), vous êtes mieux à même de convaincre les autres de ce que vous voulez. Vous êtes entraîné à toutes les interactions. Facilitateur. (Trained Interlocutor \textendash (193))

\subsection*{Interrogation de l'âme}\label{subsec:ab_soul_interrogation}

Vous déterminez les faiblesses, les vulnérabilités, les qualités et les manières d'une seule créature à longue portée. Le MJ doit révéler le niveau de la créature, ses capacités de base et ses faiblesses évidentes (le cas échéant). Toutes les actions que vous tentez et qui affectent cette créature (attaque, défense, interaction, etc.) sont ensuite facilitées pendant quelques mois. Action. (Soul Interrogation \textendash (184))

\subsection*{Interrogez les esprits}\label{subsec:ab_question_the_spirits}

Vous pouvez appeler un esprit et lui demander de répondre à quelques questions (généralement pas plus de trois avant que l'esprit ne disparaisse). Tout d'abord, vous devez invoquer un esprit. S'il s'agit d'un esprit des morts, vous devez avoir personnellement connu la créature, posséder un objet qui appartenait à la créature, ou toucher les restes physiques de la créature. Pour les autres esprits, vous devez connaître le nom complet de l'esprit ou posséder une grande partie d'un élément (comme le feu ou la terre) auquel l'esprit est associé. Si l'esprit répond, il peut se manifester sous la forme d'une ombre insubstantielle qui répond d'elle-même, il peut habiter un objet ou tout reste que vous lui fournissez, ou il peut se manifester sous la forme d'une présence invisible au nom de laquelle vous parlez. L'esprit peut ne pas vouloir répondre à vos questions, auquel cas vous devez le persuader de vous aider. Vous pouvez tenter de soumettre psychiquement l'esprit (une tâche intellectuelle), ou vous pouvez essayer la diplomatie, la tromperie ou le chantage (« Répondez-moi, ou je dirai à vos enfants que vous étiez un coureur de jupons » ou « Je détruirai l'esprit ». cette relique qui t'appartenait »). Le MJ détermine ce que l'esprit peut savoir, sur la base des connaissances qu'il possédait dans sa vie. Action à initier. (Question the Spirits \textendash (173))

\subsection*{Interruption}\label{subsec:ab_interruption}

Votre ordre bruyant et retentissant empêche une créature à courte portée d'effectuer une action pendant un round. Il peut se défendre s'il est attaqué, mais lorsqu'il le fait, sa défense est gênée par deux pas. Chaque fois que vous tentez cette capacité contre la même cible, vous devez appliquer un niveau d'effort de plus que celui appliqué lors de la tentative précédente. Action. (Interruption \textendash (155))

\subsection*{Intervention divine}\label{subsec:ab_divine_intervention}

Vous demandez au divin d'intervenir en votre nom, généralement contre une créature à longue portée, modifiant légèrement le cours de sa vie en introduisant sur elle un effet spécial majeur. L'effet spécial majeur est semblable à ce qui se produit lorsque vous obtenez un 20 naturel lors d'une attaque. Si vous souhaitez essayer un effet plus important, et si le MJ le permet, vous pouvez tenter une intervention divine avec un effet plus étendu, qui s'apparente davantage au genre d'intrusion du MJ initiée par le MJ sur ses joueurs. Dans ce cas, l'Intervention divine coûte également 4 XP, l'effet peut ne pas fonctionner exactement comme vous l'espérez et vous ne pouvez pas faire un autre appel à l'intervention divine avant une semaine. Action. (Divine Intervention \textendash (130))

\subsection*{Intouchable}\label{subsec:ab_untouchable}

Vous modifiez votre état de phase pour la minute suivante afin de ne plus pouvoir affecter ou être affecté par la matière ou l'énergie normale. Seules les attaques mentales et les énergies, appareils ou capacités transdimensionnelles spéciales peuvent vous affecter, mais de même, vous ne pouvez pas attaquer, toucher ou affecter quoi que ce soit. Action à initier. (Untouchable \textendash (195))

\subsection*{Intouchable en mouvement}\label{subsec:ab_untouchable_while_moving}

Vous modifiez votre état de phase pour la minute suivante afin que vous ne puissiez pas affecter ou être affecté par la matière ou l'énergie normale, tant que vous vous déplacez au moins sur une distance immédiate à chaque tour pendant que vous êtes en phase. Si vous ne bougez pas pendant votre tour, l'effet prend fin. Pendant que vous êtes en phase, seules les attaques mentales et les énergies, appareils ou capacités transdimensionnelles spéciales peuvent vous affecter, mais de même, vous ne pouvez pas attaquer, toucher ou affecter quoi que ce soit. Action à initier. (Untouchable While Moving \textendash (195))

\subsection*{Inventeur}\label{subsec:ab_inventor}

vous pouvez créer de nouveaux artefacts en deux fois moins de temps, comme s'ils étaient deux niveaux plus bas, en dépensant la moitié de l'XP normale. Facilitateur. (Inventor \textendash (155))

\subsection*{Invisibilité}\label{subsec:ab_invisibility}

Vous devenez invisible pendant dix minutes. Bien qu'invisible, vous êtes spécialisé dans les tâches de furtivité et de défense rapide. Cet effet prend fin si vous faites quelque chose pour révéler votre présence ou votre position : attaquer, utiliser une capacité, déplacer un objet volumineux, etc. Si cela se produit, vous pouvez retrouver l'effet d'invisibilité restant en effectuant une action pour masquer votre position. Si vous disposez d'une autre capacité qui confère également l'invisibilité, utiliser l'une ou l'autre vous permet de rester invisible deux fois plus longtemps que la durée spécifiée. Action à initier ou à relancer. (Invisibility \textendash (155))

\subsection*{Invisibilité Multiple}\label{subsec:ab_multi_vanish}

Vous rendez jusqu'à cinq créatures ou objets à taille humaine invisibles pendant une courte période de temps. Les cibles que vous choisissez doivent se trouver dans une zone immédiate et à courte portée de vous (si vous êtes dans la zone, vous pouvez vous rendre invisible et ne comptez pas dans la limite de cinq cibles invisibles). Tout ce qui est invisible a un atout pour les tâches de furtivité et de défense rapide. Les créatures affectées peuvent se voir de manière limitée et vous pouvez les voir clairement. L'invisibilité prend fin à la fin de votre prochain tour. Si l'une des créatures affectées fait quelque chose pour révéler sa présence ou sa position (attaquer, utiliser une capacité, déplacer un gros objet, etc.), l'invisibilité prend fin prématurément pour cette créature. En plus des options normales d'utilisation de l'Effort, vous pouvez choisir d'utiliser l'Effort pour augmenter la durée ; chaque niveau d'Effort utilisé de cette manière augmente la durée d'un tour (mais les créatures peuvent toujours y mettre fin plus tôt pour elles-mêmes). Action. (Multi-Vanish \textendash (164))

\subsection*{Invisibilité de Phase}\label{subsec:ab_invisible_phasing}

Vous devenez invisible lors de l'utilisation de Sprint de Phase et lors du tour suivant. Bien qu'invisible, la furtivité est facilitée de deux étapes et la défense rapide est facilitée de deux étapes (cela remplace l'atout des tâches de défense rapide fournies par Sprint de Phase). La première attaque que vous effectuez en utilisant les capacités d'attaque Shreds the Walls of the World est également facilitée de deux étapes ; cependant, si vous attaquez une créature, la phase invisible se termine immédiatement au lieu de durer un tour supplémentaire. Si vous possédez la capacité Invisibilité, vous pouvez rester invisible pendant tout le tour, ce qui signifie que si vous utilisez Rayer l'Existence ou Déchirer L'Existence, l'attaque de chaque cible sur votre chemin est facilitée de deux étapes. Facilitateur. (Invisible Phasing \textendash (155))

\subsection*{Invoquer un démon}\label{subsec:ab_summon_demon}

Un démon apparaît à portée immédiate. Si vous avez appliqué un niveau d'Effort dans le cadre de l'invocation, le démon se soumet à vos instructions ; sinon, il agit selon sa nature. Quoi qu'il en soit, le démon persiste jusqu'à une minute avant de disparaître – vous l'espérez. Action à initier. (Summon Demon \textendash (188))

\subsection*{Invoquer une araignée géante}\label{subsec:ab_summon_giant_spider}

Une araignée géante apparaît à portée immédiate. Si vous avez appliqué un niveau d'Effort dans le cadre de l'invocation, l'araignée se soumet à vos instructions ; sinon, il agit selon sa nature. Quoi qu'il en soit, la créature persiste jusqu'à une minute avant de disparaître. Action à initier. (Summon Giant Spider \textendash (188))

\subsection*{Irréprochable}\label{subsec:ab_blameless}

vous êtes entraîné dans l'un des domaines suivants: tromperie, furtivité ou déguisement. Facilitateur. (Blameless \textendash (113))

\
%--------------------------
\section*{J}

\subsection*{Jeux d'esprit}\label{subsec:ab_mind_games}

Vous utilisez des mensonges, des ruses, des moqueries et peut-être même un langage haineux et obscène contre un ennemi qui peut vous comprendre. En cas de succès, l'ennemi est étourdi pendant un round et ne peut pas agir, et il est étourdi le round suivant, période pendant laquelle ses tâches sont entravées. Action. (Mind Games \textendash (162))

\subsection*{Jouer devant la foule}\label{subsec:ab_play_to_the_crowd}

Vous prononcez un discours à la fois entraînant et terrifiant. Ceux qui se trouvent à proximité et qui peuvent vous entendre et comprendre voient leur prochaine action soit facilitée (un atout) soit entravée – vous choisissez, et cela peut être différent pour chaque individu. Action; quelques tours à terminer. (Play to the Crowd \textendash (170))

\subsection*{Joueur}\label{subsec:ab_gamer}

choisissez n'importe quel style de jeu, comme les jeux de stratégie en temps réel, les jeux de hasard dans le style du poker, les jeux de rôle, etc. Vous pouvez appliquer un atout à une tâche liée au jeu de ce style de jeu une fois entre chaque jet de récupération. Facilitateur. (Gamer \textendash (144))

\subsection*{Juggernaut}\label{subsec:ab_juggernaut}

Jusqu'à la fin du tour suivant, vous pouvez vous déplacer à travers des objets solides tels que des portes et des murs. Seuls 2 pieds (60 cm) de bois, 1 pied (30 cm) de pierre ou 6 pouces (15 cm) de métal peuvent arrêter votre mouvement. Facilitateur. (Juggernaut \textendash (156))

\subsection*{Juste un peu fou}\label{subsec:ab_just_a_bit_mad}

vous êtes entraîné aux tâches de défense intellectuelle. Facilitateur. (Just a Bit Mad \textendash (156))

\
%--------------------------
\section*{K}

\subsection*{Knock Out}\label{subsec:ab_knock_out}

Vous effectuez une attaque de mêlée qui n'inflige aucun dégât. Au lieu de cela, si l'attaque réussit, effectuez un deuxième jet basé sur la Puissance. En cas de succès, un ennemi de niveau 3 ou inférieur perd connaissance pendant une minute. Pour chaque niveau d'effort utilisé, vous pouvez affecter un niveau d'ennemi supérieur ou prolonger la durée d'une minute supplémentaire. Action. (Knock Out \textendash (156))

\
%--------------------------
\section*{L}

\subsection*{La chance du voleur}\label{subsec:ab_thief's_luck}

La chance n'est pas l'océan chaotique du hasard que la plupart des gens croient. Si vous échouez dans une tâche (y compris un jet d'attaque ou un jet de défense), vous pouvez changer le résultat du dé en un 20 naturel. Cela pourrait ne pas suffire pour réussir si la difficulté est supérieure à 6. Une fois que vous utilisez cette capacité, il n'est à nouveau disponible qu'après avoir effectué un jet de récupération de dix heures. (La chance du voleur ne fonctionne pas si vous obtenez un 1 naturel pour une tentative de tâche, à moins que vous ne possédiez et n'utilisiez également la capacité Extraire du hasard.) Enabler. (Thief's Luck \textendash (191))

\subsection*{La connaissance, c'est le pouvoir}\label{subsec:ab_knowledge_is_power}

Choisissez deux compétences non liées au combat pour lesquelles vous n'êtes pas entraîné. Vous êtes entraîné à ces compétences. Facilitateur. (Knowledge Is Power \textendash (156))

\subsection*{La mémoire devient une action}\label{subsec:ab_memory_becomes_action}

Vous pouvez dupliquer une capacité de personnage à une action, en l'exécutant comme si cela était naturel pour vous. Vous devez avoir vu la capacité utilisée au cours de la semaine dernière, elle doit être de troisième niveau ou inférieur, et il doit s'agir d'une capacité avec un coût en points. En plus du coût en points de La mémoire devient action, vous devez payer le coût en Puissance, Célérité ou Intellect de la capacité que vous copiez. Par exemple, si vous souhaitez copier l'attaque de fente d'un ami (qui coûte normalement 2 points de Puissance), vous paierez 4 points d'intelligence pour activer la mémoire devient action et 2 points de Puissance pour utiliser la fente. En plus des options normales d'utilisation de l'Effort, vous pouvez choisir d'utiliser l'Effort pour copier une capacité que vous avez vue il y a plus d'une semaine ; chaque niveau d'Effort utilisé de cette manière prolonge la période d'une semaine. Facilitateur. (Memory Becomes Action \textendash (161))

\subsection*{La nature est de votre côté}\label{subsec:ab_the_wild_is_on_your_side}

lorsque vous êtes dans la nature, les ennemis à courte portée trébuchent sur des rochers, s'emmêlent dans les vignes, sont mordus par des insectes et sont distraits ou confus par de petits animaux, ce qui les gênent dans toutes leurs tâches. pour dix minutes. Action à initier. (The Wild is On Your Side \textendash (190))

\subsection*{Lame de Feu}\label{subsec:ab_flameblade}

Lorsque vous le souhaitez, vous étendez votre Manteau de flammes pour couvrir une arme que vous brandissez en flammes pendant une heure. La flamme s'éteint si vous arrêtez de tenir ou de porter l'arme. Tant que dure la flamme, l'arme inflige 2 points de dégâts supplémentaires. Facilitateur. (Flameblade \textendash (140))

\subsection*{Lancement de Cypher}\label{subsec:ab_cypher_casting}

vous pouvez activer n'importe lequel de vos cyphers subtils sur une autre créature au lieu de vous-même. Vous devez toucher la créature pour l'affecter. Facilitateur. (Cypher Casting \textendash (GF, 29))

\subsection*{Lancement de flammes}\label{subsec:ab_hurl_flame}

Pendant que votre Manteau de flammes est actif, vous pouvez atteindre votre halo et lancer une poignée de feu sur une cible. Il s'agit d'une attaque à distance à courte portée qui inflige 4 points de dégâts de feu. Action. (Hurl Flame \textendash (149))

\subsection*{Lancer}\label{subsec:ab_throw}

lorsque vous utilisez Agrandissement et infligez des dégâts à une créature de votre taille ou moins avec une attaque à mains nues, vous pouvez choisir de lancer cette créature jusqu'à 1d20 pieds de vous. La créature atterrit à plat ventre. Facilitateur. (Throw \textendash (191))

\subsection*{Lancer Illusion}\label{subsec:ab_cast_illusion}

Vous pouvez augmenter la portée à laquelle vous créez et maintenez votre illusion à portée immédiate (telles que celles de Illusion mineure) vers n'importe quel endroit à courte portée que vous pouvez percevoir. Facilitateur. (Cast Illusion \textendash (118))

\subsection*{Lancer Objet en Fer}\label{subsec:ab_iron_punch}

Vous ramassez magnétiquement un objet métallique lourd à courte portée et le lancez sur quelqu'un à courte portée, une action d'Intellect qui inflige 6 points de dégâts à la cible et à l'objet lancé. Pour chaque niveau d'effort supplémentaire appliqué, vous pouvez ramasser un objet légèrement plus grand, vous permettant d'affecter une cible supplémentaire à courte portée tant qu'elle se trouve à côté de la cible précédente. Action. (Iron Punch \textendash (155))

\subsection*{Lancer rapide}\label{subsec:ab_quick_throw}

Après avoir utilisé une arme légère de lancer, vous dégainez une autre arme légère et effectuez une autre attaque lancée contre la même cible ou une autre. Action. (Quick Throw \textendash (174))

\subsection*{Lancer un bouclier de force}\label{subsec:ab_throw_force_shield}

vous pouvez lancer votre bouclier de champ de force à courte portée comme une arme légère à distance. Que le bouclier touche ou rate, il se dissipe immédiatement puis se reforme à votre portée. Facilitateur. (Throw Force Shield \textendash (191))

\subsection*{Lancer une arme enchantée}\label{subsec:ab_throw_enchanted_weapon}

vous pouvez lancer votre arme enchantée à courte portée comme une arme à distance légère. Qu'il frappe ou rate, il revient immédiatement dans vos mains et vous pouvez automatiquement l'attraper ou le laisser atterrir à vos pieds. Facilitateur. (Throw Enchanted Weapon \textendash (191))

\subsection*{Lanceur spécialisé}\label{subsec:ab_specialized_throwing}

Vous êtes spécialisé dans les attaques avec toutes les armes que vous lancez. Facilitateur. (Specialized Throwing \textendash (185))

\subsection*{Lavage de cerveau}\label{subsec:ab_brainwashing}

Vous utilisez la ruse, des mensonges bien prononcés et des produits chimiques affectant l'esprit (ou d'autres moyens, comme la magie ou la haute technologie) pour obliger les autres à faire temporairement ce que vous voulez qu'ils fassent. Vous contrôlez les actions d'une autre créature que vous touchez. Cet effet dure une minute. La cible doit être de niveau 3 ou inférieur. Vous pouvez lui permettre d'agir librement ou outrepasser son contrôle au cas par cas, à condition que vous puissiez le voir. En plus des options normales d'utilisation de l'Effort, vous pouvez choisir d'utiliser l'Effort pour augmenter le niveau maximum de la cible ou augmenter la durée d'une minute. Ainsi, pour contrôler l'esprit d'une cible de niveau 6 (trois niveaux au-dessus de la limite normale) ou contrôler une cible pendant quatre minutes (trois minutes au-dessus de la durée normale), vous devez appliquer trois niveaux d'Effort. Lorsque la durée se termine, la créature ne se souvient pas d'avoir été contrôlée ou de quoi que ce soit qu'elle ait fait sous votre influence. Action à initier. (Brainwashing \textendash (116))

\subsection*{Le pied marin}\label{subsec:ab_sea_legs}

Vous vous êtes habitué à une mer agitée et à des vagues inattendues. Vous êtes entraîné à l'équilibre. Toute tâche de mouvement qui serait gênée par un pont de lancement, un déplacement dans un gréement, etc. est une tâche de routine pour vous. Facilitateur. (Sea Legs \textendash (180))

\subsection*{Le rêve devient réalité}\label{subsec:ab_dream_becomes_reality}

Vous créez un objet de rêve de n'importe quelle forme que vous pouvez imaginer, de votre taille ou plus petite, qui prend une substance et un poids apparents. L'objet est brut et ne peut comporter aucune pièce mobile, vous pouvez donc fabriquer une épée, un bouclier, une échelle courte, etc. L'objet du rêve a la masse approximative de l'objet réel, si vous le souhaitez. Les objets de vos rêves sont aussi résistants que le fer, mais si vous ne restez pas à leur portée, ils ne fonctionnent qu'une minute avant de disparaître. Action. (Dream Becomes Reality \textendash (132))

\subsection*{Lecture mentale}\label{subsec:ab_mind_reading}

Vous pouvez lire les pensées superficielles d'une créature à courte portée, même si la cible ne le souhaite pas. Vous devez être capable de voir votre cible. Une fois le contact établi, vous pouvez lire les pensées de la cible pendant une minute maximum. Si vous disposez également de la capacité spéciale Lecture mentale provenant d'une autre source, vous pouvez utiliser cette capacité à longue portée et vous n'avez pas besoin de pouvoir voir la cible (mais vous devez savoir que la cible est à portée). Action à initier. (Mind Reading \textendash (162))

\subsection*{Lentille Cristalline}\label{subsec:ab_crystal_lens}

Vous pouvez concentrer l'énergie inhérente qui vous traverse grâce à la capacité de votre **Corps de Cristal**. Cela permet de tirer une explosion d'énergie qui inflige 5 points de dégâts à une cible à très longue portée. Action. (Crystal Lens \textendash (123))

\subsection*{Leçons de jeu}\label{subsec:ab_game_lessons}

Vous avez joué à tellement de jeux que vous avez acquis de réelles connaissances. Choisissez deux compétences non liées au combat. Vous êtes entraîné à ces compétences. Facilitateur. (Game Lessons \textendash (144))

\subsection*{Leçons de vie}\label{subsec:ab_life_lessons}

choisissez deux compétences non liées au combat. Vous êtes entraîné à ces compétences. Facilitateur. (Life Lessons \textendash (158))

\subsection*{Liaison machine}\label{subsec:ab_machine_bond}

à très longue distance, vous pouvez activer et contrôler un appareil (y compris un robot ou un véhicule) avec lequel vous vous êtes lié. Par exemple, vous pouvez faire exploser un cypher manifeste même s'il est détenu par quelqu'un d'autre, ou faire tirer une tourelle automatisée là où vous le dirigez. La création de liens est un processus qui nécessite 24 heures de méditation en présence de la machine. Action. (Machine Bond \textendash (159))

\subsection*{Libre de se déplacer}\label{subsec:ab_free_to_move}

vous ignorez toutes les pénalités de mouvement et les ajustements dus au terrain ou à d'autres obstacles. Vous pouvez passer dans n'importe quel espace suffisamment grand pour s'adapter à votre tête. Les tâches impliquant de se libérer des liens, de l'emprise d'une créature ou de tout obstacle similaire bénéficient de trois niveaux d'effort gratuits. Facilitateur. (Free to Move \textendash (143))

\subsection*{Libération d'énergie}\label{subsec:ab_release_energy}

vous libérez 1 point d'énergie que vous avez absorbé avec votre capacité Absorber l'énergie cinétique, en l'agrandissant et en la concentrant en une explosion d'énergie qui frappe un seul ennemi à longue portée pour lui infliger 4 points de dégâts. (Si aucune énergie cinétique n'est absorbée, vous pouvez toujours utiliser cette capacité, mais elle nécessite que vous transformiez une fraction de vous-même en explosion, ce qui coûte 1 point de Puissance.) Action. (Release Energy \textendash (175))

\subsection*{Libération explosive}\label{subsec:ab_explosive_release}

Vous pouvez amplifier l'énergie stockée dans votre Réserve de Siphon (à partir de votre capacité de Stocker l'énergie et la libérer dans une explosion massive qui affecte soit une cible à courte portée, soit tout ce qui se trouve à portée immédiate. Si vous choisissez une seule cible, elle subit 2 points de dégâts pour chaque point de votre Réserve de Siphon Si vous choisissez une zone, tout ce qui se trouve dans la zone (sauf vous) subit 1 point de dégâts par point dans votre Réserve de Siphon (ou la moitié si votre attaque échoue contre eux). Cela vide votre Réserve de Siphon à 0 point. Action. (Explosive Release \textendash (138))

\subsection*{Licence poétique}\label{subsec:ab_poetic_license}

Vous êtes entraîné à toutes les interactions sociales, y compris la persuasion, la tromperie et l'intimidation. Vous connaissez également deux langues supplémentaires. Facilitateur. (Poetic License \textendash (170))

\subsection*{Lien d'Objet Amélioré}\label{subsec:ab_improved_object_bond}

lorsque vous manifestez l'allié grâce à votre capacité de créature magique liée, il s'agit désormais d'une créature de niveau 4. De plus, la créature gagne une attaque par impulsion qui rend inutilisables tous les artefacts, machines, chiffres manifestes et dispositifs magiques mineurs à courte portée pendant une minute. Après que la créature ait utilisé cette capacité, elle doit se retirer vers son objet pour se reposer pendant trois heures. Facilitateur. (Improved Object Bond \textendash (152))

\subsection*{Lien d'objet}\label{subsec:ab_object_bond}

Lorsque vous manifestez l'allié magique grâce à votre capacité de Créature magique liée, il peut se déplacer jusqu'à 300 pieds (90 m) de vous avant d'être renvoyé vers son objet lié. En outre, cela peut rester manifeste pendant une période prolongée, jusqu'à la fin de votre prochain jet de récupération de dix heures. Enfin, si vous donnez votre permission, l'allié magique peut sortir et pénétrer dans l'objet lié de sa propre initiative. Facilitateur. (Object Bond \textendash (167))

\subsection*{Lien mental}\label{subsec:ab_mental_link}

Vous ouvrez un chemin vers l'esprit d'une autre créature via un contact léger, ce qui vous permet de vous transmettre des pensées et des images. Le lien mental demeure quelle que soit la distance et dure une heure. En plus des options normales d'utilisation de l'Effort, vous pouvez choisir d'utiliser l'Effort pour prolonger la durée d'une heure pour chaque niveau d'Effort appliqué. Action à initier. (Mental Link \textendash (161))

\subsection*{Lier les sens}\label{subsec:ab_link_senses}

Vous touchez une créature volontaire et reliez ses sens aux vôtres pendant une minute. À tout moment pendant cette durée, vous pouvez vous concentrer pour voir, entendre et sentir ce que vit cette créature, au lieu d'utiliser vos propres sens. Si vous ou la créature sortez d'une longue portée, la connexion est rompue. Action à initier. (Link Senses \textendash (158))

\subsection*{Lire les signes}\label{subsec:ab_read_the_signs}

Vous examinez une zone et apprenez des détails précis et utiles sur le passé (le cas échéant). Vous pouvez poser au MJ jusqu'à quatre questions sur la zone immédiate ; chacun nécessite son propre jet. Action. (Read the Signs \textendash (174))

\subsection*{Lourd}\label{subsec:ab_weighty}

vous augmentez brièvement le poids d'une cible à courte portée suffisamment pour l'arrêter dans son élan, empêchant la cible de bouger et entravant toute tentative de tâche lors de son prochain tour. Action. (Weighty \textendash (197))

\subsection*{Lueur automatique}\label{subsec:ab_automatic_glow}

les objets à lumière dure que vous créez avec votre type et vos capacités de mise au point projettent de la lumière, illuminant tout à portée immédiate. Quand vous le souhaitez, votre corps (en totalité ou en partie) éclaire tout, éclairant tout à courte portée. Facilitateur. (Automatic Glow \textendash (CTS, 49))

\subsection*{Lumière brûlante}\label{subsec:ab_burning_light}

Vous envoyez un faisceau de lumière sur une créature à longue portée, puis resserrez le faisceau jusqu'à ce qu'il brûle, lui infligeant 5 points de dégâts. Action. (Burning Light \textendash (116))

\subsection*{Lumière coupante}\label{subsec:ab_cutting_light}

Vous émettez un mince faisceau de lumière énergique depuis votre main. Cela inflige 5 points de dégâts à un seul ennemi à portée immédiate. Le faisceau est encore plus efficace contre les cibles immobiles et non vivantes, coupant jusqu'à 30 cm de tout matériau de niveau 6 ou inférieur. Le matériau peut atteindre 1 pied d'épaisseur. Action. (Cutting Light \textendash (123))

\subsection*{Lumière du soleil}\label{subsec:ab_sunlight}

Un grain de lumière se déplace de vous vers un endroit que vous choisissez à longue portée. Lorsque la particule atteint cet endroit, elle s'enflamme et projette une lumière vive dans un rayon de 200 pieds (60 m), et l'obscurité à moins de 1 000 pieds (300 m) de la particule devient une faible lumière. La lumière dure une heure ou jusqu'à ce que vous utilisiez une action pour la faire disparaître. Action. (Sunlight \textendash (188))

\subsection*{Lumière plus Forte}\label{subsec:ab_harder_light}

lorsque vous créez un objet avec une lumière forte, l'objet est d'un niveau plus haut que la normale. Facilitateur. (Harder Light \textendash (148))

\subsection*{Lumière sculptée améliorée}\label{subsec:ab_improved_sculpt_light}

Vous créez un objet de lumière solide dans n'importe quelle forme que vous pouvez imaginer dont la taille de base peut tenir dans un cube de 10 pieds (3 m). L'objet apparaît dans une zone adjacente à vous ou flotte librement dans l'espace jusqu'à une longue distance, et l'objet dure quelques jours. L'objet est brut et ne peut comporter aucune pièce mobile. Vous pouvez donc créer un segment de mur, un bloc, une boîte, des escaliers, etc. L'objet sculpté a la masse approximative de l'objet réel et est de niveau 6. Si vous appliquez Effort pour augmenter la taille de l'objet, chaque niveau appliqué augmente la taille d'un cube supplémentaire de 10 pieds (3 m). Action. (Improved Sculpt Light \textendash (152))

\subsection*{Lumière temporaire}\label{subsec:ab_temporary_light}

Vous créez un objet de lumière solide dans n'importe quelle forme que vous pouvez imaginer, de votre taille ou plus petite, et il persiste pendant environ une minute (ou plus, si vous vous concentrez dessus après ce temps). L'objet apparaît dans une zone adjacente à vous, mais vous pouvez ensuite le déplacer sur une courte distance à chaque tour dans le cadre d'une autre action. Il est brut et ne peut comporter aucune pièce mobile, vous pouvez donc fabriquer une épée, un bouclier, une échelle courte, etc. L'objet a la masse approximative de l'objet réel et est de niveau 2. Action. (Temporary Light \textendash (190))

\subsection*{Lumière vivante}\label{subsec:ab_living_light}

Votre corps se dissout dans un nuage de photons qui se déplacent instantanément vers un endroit que vous choisissez puis se reforment. Vous pouvez choisir n'importe quel espace ouvert suffisamment grand pour vous contenir et que vous pouvez voir à très longue portée, ou n'importe quel endroit que vous avez éclairé par Toucher lumineux et qui brille encore. Vous disparaissez et réapparaissez presque instantanément dans l'espace que vous avez choisi. Il faut attendre la fin du tour pour que votre corps devienne complètement solide, donc jusqu'au début du tour suivant, vous subissez un maximum de 1 point de dégâts pour toute attaque ou source de dégâts donnée. Chaque niveau d'Effort que vous appliquez vous permet d'amener une personne supplémentaire en plus de vous, à condition qu'elle soit à portée immédiate lorsque vous partez. Action. (Living Light \textendash (158))

\subsection*{Légèreté}\label{subsec:ab_levity}

Grâce à votre esprit, votre charme, votre humour et votre grâce, vous êtes entraîné à toutes les interactions sociales autres que celles impliquant la coercition ou l'intimidation. Pendant les repos, vous mettez tellement vos amis et camarades à l'aise qu'ils gagnent +1 à leurs jets de récupération. Facilitateur. (Levity \textendash (158))

\
%--------------------------
\section*{M}

\subsection*{Machine aveugle}\label{subsec:ab_blind_machine}

Vous désactivez l'appareil sensoriel d'une machine, la rendant effectivement aveugle jusqu'à ce qu'elle puisse être réparée. Vous devez soit toucher la cible, soit la frapper avec une attaque à distance (ne lui infligeant aucun dégât). Action. (Blind Machine \textendash (114))

\subsection*{Magie Prosaïque}\label{subsec:ab_hedge_magic}

Vous pouvez réaliser de petites astuces : changer temporairement la couleur ou l'apparence de base d'un petit objet, faire flotter de petits objets dans les airs, nettoyer une petite zone, réparer un objet cassé, préparer (mais pas créer ) la nourriture, etc. Vous ne pouvez pas utiliser Magie Prosaïque pour blesser une autre créature ou un autre objet. Action. (Hedge Magic \textendash (149))

\subsection*{Magie Sauvage plus rapide}\label{subsec:ab_faster_wild_magic}

si vous passez dix minutes à préparer votre magie, vous pouvez remplir n'importe lequel de vos emplacements de cypher ouverts avec des cyphers subtils choisis au hasard par le MJ (ce temps peut faire partie d'un cycle de dix minutes, d'une heure ou de dix heures). action de récupération si vous êtes éveillé pendant tout ce temps). Vous ne pouvez plus utiliser cette capacité avant d'avoir effectué une action de récupération de dix heures. Vous pouvez toujours utiliser le Répertoire magique pour remplir vos emplacements de cypher. Action à initier, dix minutes à réaliser. (Faster Wild Magic \textendash (139))

\subsection*{Main Guérisseuse}\label{subsec:ab_healing_touch}

avec un contact, vous restaurez 1d6 points à un pool de statistiques de n'importe quelle créature. Cette capacité est une tâche de difficulté Intellect 2. Chaque fois que vous tentez de soigner la même créature, la tâche est gênée par une étape supplémentaire. La difficulté revient à 2 après que cette créature se repose pendant dix heures. Action. (Healing Touch \textendash (149))

\subsection*{Main Guérisseuse Supérieure}\label{subsec:ab_greater_healing_touch}

vous touchez une créature et restaurez sa Réserve de Puissance, sa Réserve de Célérité et sa Réserve d'Intellect à leurs valeurs maximales, comme si elle était complètement reposée. Une seule créature ne peut bénéficier de cette capacité qu'une fois par jour. Action. (Greater Healing Touch \textendash (147))

\subsection*{Main ardente du destin}\label{subsec:ab_fiery_hand_of_doom}

Pendant que votre Manteau de flammes est actif, vous pouvez atteindre votre halo et produire une main faite de flammes animées qui fait deux fois la taille d'une main humaine. La main agit selon vos directives, flottant dans les airs. Diriger la main est une action. Sans commande, la main ne fait rien. Il peut se déplacer sur une longue distance en un tour, mais il ne s'éloigne jamais plus loin de vous qu'une longue distance. La main peut saisir, déplacer et transporter des objets, mais tout ce qu'elle touche subit 1 point de dégâts par round à cause de la chaleur. La main peut aussi attaquer. C'est une créature de niveau 3 et inflige 1 point de dégâts de feu supplémentaire lorsqu'elle attaque. Une fois créée, la main dure dix minutes. Action de créer ; action à diriger. (Fiery Hand of Doom \textendash (139))

\subsection*{Maintenez le cap}\label{subsec:ab_stay_the_course}

Lorsque vos compagnons faibliront, vous pouvez les inspirer avec un mot ou deux au bon moment. Tout allié (sauf vous) à portée immédiate peut effectuer un jet de récupération qui n'est pas une action et ne compte pas dans sa limite quotidienne. Action. (Stay the Course \textendash (186))

\subsection*{Manteau d'opportunité}\label{subsec:ab_cloak_of_opportunity}

vous faites tourbillonner de petits objets de l'environnement (rochers, objets cassés, mottes de terre, etc.) autour de vous pendant dix minutes maximum, ce qui vous confère +2 d'armure. Action à initier. (Cloak of Opportunity \textendash (119))

\subsection*{Manteau de flammes}\label{subsec:ab_shroud_of_flame}

sur votre commande, votre corps tout entier est enveloppé de flammes qui durent jusqu'à dix minutes. Le feu ne vous brûle pas, mais il inflige automatiquement 2 points de dégâts à quiconque tente de vous toucher ou de vous frapper avec une attaque au corps à corps. Les flammes provenant d'une autre source peuvent toujours vous blesser. Tant que le linceul est actif, vous gagnez +2 en armure contre les dégâts de feu provenant d'une autre source. Facilitateur. (Shroud of Flame \textendash (183))

\subsection*{Manteau de matière noire}\label{subsec:ab_dark_matter_shroud}

des rubans de matière noire se condensent et tourbillonnent autour de vous pendant une minute maximum. Ce manteau facilite vos tâches de défense de Célérité, inflige 2 points de dégâts à quiconque tente de vous toucher ou de vous frapper avec une attaque au corps à corps et vous donne +1 d'armure. Action à initier. (Dark Matter Shroud \textendash (124))

\subsection*{Marche impossible}\label{subsec:ab_impossible_walk}

Vous pouvez marcher (ou ramper ou courir) sur des pentes raides et des surfaces verticales (telles que des murs et des falaises) pendant la minute suivante comme s'il s'agissait d'un terrain plat. Lorsque vous utilisez cette capacité, « vers le bas » correspond pour vous, soit à la surface sur laquelle vous marchez, soit à l'orientation normale de la gravité (votre choix). Si vous appliquez un niveau d'Effort, vous pouvez également marcher au plafond ou sur une surface liquide ou semi-liquide comme de l'eau, de la boue, des sables mouvants ou même de la lave (même si toucher une surface dangereuse comme la lave vous fait toujours du mal). Si vous appliquez deux niveaux d'effort, vous pouvez également marcher dans les airs comme s'il s'agissait d'un sol solide. Facilitateur. (Impossible Walk \textendash (151))

\subsection*{Marin}\label{subsec:ab_sailor}

Vous êtes entraîné aux tâches liées à la voile et entraîné à la géographie des îles et des littoraux. Facilitateur. (Sailor \textendash (179))

\subsection*{Masque}\label{subsec:ab_mask}

Vous transformez votre corps pour devenir quelqu'un d'autre. Vous pouvez modifier toutes les caractéristiques physiques de votre choix, notamment la coloration, la taille, le poids, le sexe et les marques distinctives. Vous pouvez également modifier l'apparence de tout ce que vous portez ou transportez. Vos statistiques, ainsi que celles de vos objets, ne changent pas. Vous restez sous cette forme jusqu'à une journée ou jusqu'à ce que vous utilisiez une action pour reprendre votre apparence normale. Action à initier. (Mask \textendash (160))

\subsection*{Mathématiques Complémentaires}\label{subsec:ab_further_mathematics}

Vous êtes spécialisé en mathématiques supérieures. Si vous êtes déjà spécialisé, choisissez un autre domaine de connaissances dans lequel vous entraîner. (Further Mathematics \textendash (144))

\subsection*{Mathématiques supérieures}\label{subsec:ab_higher_mathematics}

Vous êtes entraîné aux mathématiques standards et supérieures. Facilitateur. (Higher Mathematics \textendash (149))

\subsection*{Maximiser le Cypher}\label{subsec:ab_maximize_cypher}

choisissez un cypher subtil que vous portez. Son niveau devient le niveau maximum possible pour ce cypher. Par exemple, une aide à la méditation a une plage de niveaux de 1d6 + 2, donc maximiser ce cypher change son niveau à 8. Vous ne pouvez avoir qu'un seul cypher subtil maximisé à la fois. Vous ne pouvez plus utiliser cette capacité avant d'avoir effectué une action de récupération de dix heures. Facilitateur. (Maximize Cypher \textendash (161))

\subsection*{Maître artisan}\label{subsec:ab_master_crafter}

Vous êtes entraîné à la fabrication de deux types d'objets, ou vous êtes spécialisé dans deux types d'objets pour lesquels vous êtes déjà entraîné. (Master Crafter \textendash (160))

\subsection*{Maître d'Arme}\label{subsec:ab_weapon_master}

Vous infligez 1 point de dégâts supplémentaire avec l'arme de votre choix. Facilitateur. (Weapon Master \textendash (197))

\subsection*{Maître de la défense}\label{subsec:ab_defense_master}

chaque fois que vous réussissez une tâche de défense rapide, vous pouvez lancer une attaque immédiate contre votre ennemi. (Si vous disposez d'Esquive et de Réponse, vous pouvez échanger cette capacité contre Esquive et Résistance.) Votre attaque doit être du même type (arme de mêlée, arme à distance ou à mains nues) que l'attaque contre laquelle vous vous défendez. Si vous ne disposez pas d'un type d'arme approprié, vous ne pouvez pas utiliser cette capacité. Facilitateur. (Defense Master \textendash (127))

\subsection*{Maître du spectacle}\label{subsec:ab_master_entertainer}

votre capacité Facilité Inspirante fonctionne plus efficacement, facilitant les tâches de vos amis de deux étapes plutôt que d'une seule. Facilitateur. (Master Entertainer \textendash (160))

\subsection*{Maître du style de combat à mains nues}\label{subsec:ab_master_of_unarmed_fighting_style}

Vous êtes spécialisé dans les attaques à mains nues. Si vous êtes déjà spécialisé dans les attaques à mains nues, vous infligez plutôt 2 points de dégâts supplémentaires avec les attaques à mains nues. Facilitateur. (Master of Unarmed Fighting Style \textendash (160))

\subsection*{Maître voleur}\label{subsec:ab_master_thief}

vous êtes entraîné à grimper, à échapper aux liens, à vous faufiler dans des endroits étroits et à d'autres mouvements de contorsion. Facilitateur. (Master Thief \textendash (160))

\subsection*{Maîtrise dans l' Utilisation des Cyphers}\label{subsec:ab_master_cypher_use}

Vous pouvez utiliser cinq cyphers à la fois. Facilitateur. (Master Cypher Use \textendash (160))

\subsection*{Maîtrise de l'Identification des appareils}\label{subsec:ab_master_identifier}

Vous êtes entraîné à identifier la fonction de tout type d'appareil. Facilitateur. (Master Identifier \textendash (160))

\subsection*{Maîtrise de la défense}\label{subsec:ab_mastery_with_defense}

choisissez un type de tâche de défense dans laquelle vous êtes entraîné: Puissance, Célérité ou intelligence. Vous êtes spécialisé dans les tâches de défense de ce type. Vous pouvez sélectionner cette capacité jusqu'à trois fois. Chaque fois que vous le sélectionnez, vous devez choisir un type de tâche de défense différent. Facilitateur. (Mastery with Defense \textendash (161))

\subsection*{Maîtrise des Cyphers}\label{subsec:ab_cyphersmith}

Tous les cyphers manifestes que vous utilisez fonctionnent à un niveau supérieur à la normale. Si vous disposez d'une semaine et des bons outils, produits chimiques et pièces, vous pouvez bricoler l'un de vos cyphers manifestes, le transformant en un autre cypher du même type que celui que vous aviez dans le passé. Le MJ et le joueur doivent collaborer pour s'assurer que la transformation est logique – par exemple, vous ne pourrez probablement pas transformer une pilule en casque. Facilitateur. (Cyphersmith \textendash (124))

\subsection*{Maîtrise des Machines}\label{subsec:ab_master_machine}

Vous pouvez contrôler les fonctions d'une machine avec laquelle vous vous êtes lié en utilisant Liaison machine, intelligente ou autre. De plus, si vous utilisez une action pour vous concentrer sur une machine, vous êtes conscient de ce qui se passe autour d'elle (vous voyez et entendez comme si vous étiez à côté d'elle, quelle que soit la distance dont vous vous trouvez). Vous devez toucher la machine pour créer le lien, mais après, il n'y a aucune limitation de portée. Ce lien dure une semaine. Vous ne pouvez établir de liaison qu'avec une seule machine à la fois. Action à initier. (Master Machine \textendash (160))

\subsection*{Maîtrise des attaques}\label{subsec:ab_mastery_with_attacks}

choisissez un type d'attaque dans lequel vous êtes entraîné : dénigrement léger, lame légère, à distance légère, dénigrement moyen, lame moyenne, portée moyenne, dénigrement lourd, lame lourde ou à distance lourde. Vous êtes spécialisé dans les attaques utilisant ce type d'arme. Facilitateur. (Si vous n'êtes pas entraîné à une attaque, sélectionnez Compétence avec attaques pour vous entraîner à cette attaque.) (Mastery with Attacks \textendash (161))

\subsection*{Maîtrise des liens d'objet}\label{subsec:ab_object_bond_mastery}

lorsque vous manifestez l'allié magique grâce à votre capacité Créature magique liée, il s'agit désormais d'une créature de niveau 7. Il ne peut rester manifeste que trois minutes, après quoi il doit retourner à son objet et se reposer pendant trois jours avant que vous puissiez le manifester à nouveau. L'allié magique peut effectuer ses propres attaques de contact magique (quand il le fait, vous lancez un jet pour lui). S'il utilise son attaque à impulsions de Lien d'objet amélioré, au lieu de désactiver des objets, il peut prendre le contrôle d'un objet à courte portée pendant une minute, le cas échéant. Enfin, l'allié magique peut se transformer en fumée et en flammes lors de son action, lui donnant +10 en Armure mais le rendant incapable d'attaquer les ennemis. Sous cette forme, il peut voler sur une longue distance à chaque tour, et la première fois chaque jour où il revient en chair (sous forme d'action), il regagne 25 points de vie. Facilitateur. (Object Bond Mastery \textendash (167))

\subsection*{Maîtrise des outils}\label{subsec:ab_tool_mastery}

Lorsque vous disposez d'un atout lié à l'utilisation d'un outil, le temps nécessaire à l'exécution de la tâche est réduit de moitié (minimum un tour). Facilitateur. (Tool Mastery \textendash (192))

\subsection*{Maîtrise du Bouclier}\label{subsec:ab_shield_master}

Lorsque vous utilisez un bouclier, en plus de l'atout qu'il vous apporte (facilitant les tâches de défense de Célérité), vous pouvez agir comme si vous étiez entraîné aux tâches de défense de Célérité. Cependant, à chaque tour au cours duquel vous utilisez cet avantage, vos attaques sont gênées. Facilitateur. (Shield Master \textendash (182))

\subsection*{Maîtrise en Armure}\label{subsec:ab_mastery_in_armor}

la réduction du coût de votre capacité Pratique des armures s'améliore. Vous réduisez désormais le coût de l'Effort de Célérité pour porter une armure à 0. (Mastery in Armor \textendash (161))

\subsection*{Maîtrise extrême}\label{subsec:ab_extreme_mastery}

lorsque vous utilisez l'arme de votre choix, vous pouvez relancer n'importe quel jet d'attaque de votre choix et prendre le meilleur des deux résultats. Facilitateur. (Extreme Mastery \textendash (138))

\subsection*{Mener depuis le front}\label{subsec:ab_lead_from_the_front}

vous gagnez 3 nouveaux points à répartir entre vos pools de statistiques comme vous le souhaitez. Facilitateur. (Lead From the Front \textendash (157))

\subsection*{Mentalement résistant}\label{subsec:ab_mentally_tough}

Regarder le tissu nu de l'hyperespace, de l'espace déentraîné ou un effet similaire lié au voyage plus rapide que la lumière est difficile pour l'esprit, mais vous avez développé une résistance. Vous êtes entraîné aux tâches de défense intellectuelle. Facilitateur. (Mentally Tough \textendash (162))

\subsection*{Mené par l'enquête}\label{subsec:ab_lead_by_inquiry}

vous gardez vos alliés en alerte avec des questions occasionnelles, des blagues et même des simulations d'exercices pour ceux qui souhaitent y participer. Après avoir passé 24 heures avec vous, vos alliés sont traités comme s'ils étaient entraînés à des tâches liées à la perception. Cet avantage perdure tant que vous restez en compagnie de vos alliés. Cela se termine si vous partez, mais cela reprend si vous revenez en compagnie des alliés dans les 24 heures. Si vous quittez la compagnie des alliés pendant plus de 24 heures, vous devez passer encore 24 heures ensemble pour réactiver l'avantage. Facilitateur. (Lead by Inquiry \textendash (157))

\subsection*{Mettre le paquet}\label{subsec:ab_all_out_con}

Vous y mettez tout. Vous ajoutez trois niveaux d'effort gratuits à la prochaine tâche que vous tentez. Vous ne pouvez plus utiliser cette capacité avant d'avoir pris une action de récupération de dix heures. Action. (All-Out Con \textendash (109))

\subsection*{Mettre un Ennemi en Phase}\label{subsec:ab_phase_foe}

Vous rassemblez de l'énergie perturbatrice au bout de votre doigt et touchez une créature. Si la cible est un PNJ ou une créature de niveau 3 ou inférieur, elle passe en phase comme si elle avait utilisé la capacité Fantôme. Cependant, à moins qu'il ne parvienne à contrôler son mouvement tout en étant en phase, ce que la plupart des créatures n'ont aucune expérience, il commence à couler à travers la matière solide. S'il ne peut pas se contrôler ou mettre fin à l'effet, il pourrait disparaître pour de bon, car lorsqu'il redeviendra solide après dix minutes, il se trouvera probablement au plus profond de la terre. Pour chaque niveau d'Effort supplémentaire que vous appliquez, vous pouvez tenter d'affecter une cible d'un niveau supérieur. Action. (Phase Foe \textendash (170))

\subsection*{Meurtre Rapide}\label{subsec:ab_fast_kill}

Vous savez tuer rapidement. Lorsque vous frappez avec une attaque au corps à corps ou à distance, vous infligez 4 points de dégâts supplémentaires. Vous ne pouvez pas effectuer cette attaque deux rounds consécutifs. Action. (Fast Kill \textendash (138))

\subsection*{Meurtrier}\label{subsec:ab_murderer}

Avec une attaque rapide et soudaine, vous frappez un ennemi à un endroit vital. Si la cible est de niveau 4 ou inférieur, elle est carrément tuée. Pour chaque niveau d'Effort supplémentaire que vous appliquez, vous pouvez augmenter le niveau de la cible de 1. Action. (Murderer \textendash (165))

\subsection*{Mieux vivre grâce à la chimie}\label{subsec:ab_better_living_through_chemistry}

Vous avez développé des cocktails de médicaments spécialement conçus pour fonctionner avec votre propre biochimie. Selon celui que vous injectez, cela vous rend plus intelligent, plus rapide ou plus résistant, mais quand cela s'estompe, le crash est un désastre, vous ne l'utilisez donc que dans des situations désespérées. Vous gagnez 2 à votre Avantage de Puissance, Speed Edge ou Intellect Edge pendant une minute, après quoi vous ne pouvez plus bénéficier de cet avantage pendant une heure. Pendant cette heure de suivi, chaque fois que vous dépensez des points d'un Pool, augmentez le coût de 1. Action. (Better Living Through Chemistry \textendash (113))

\subsection*{Minuscule}\label{subsec:ab_tiny}

lorsque vous utilisez Rétrécir, vous pouvez choisir de réduire la taille jusqu'à environ un seizième de pouce (0,2 cm). Lorsque vous le faites, vous ajoutez 5 points temporaires supplémentaires à votre Réserve de Célérité (plus ceux des plus petits), et comme vos attaques sont concentrées dans une très petite zone, vous infligez 2 points de dégâts supplémentaires. Pour chaque niveau d'Effort que vous appliquez pour rétrécir encore plus, vous devenez un dixième de votre taille (un centième pour deux niveaux d'Effort, un millième pour trois, et ainsi de suite) et vous ajoutez 1 point supplémentaire à votre Célérité. Piscine. Facilitateur. Dans les campagnes où les personnages peuvent voyager dans des dimensions parallèles, utiliser Minuscule pour réduire à un millième de votre taille normale peut être un moyen d'y parvenir. (Tiny \textendash (192))

\subsection*{Mise à niveau du robot}\label{subsec:ab_robot_upgrade}

vous modifiez votre assistant artificiel à partir de la capacité Assistant Robot avec une nouvelle capacité. Les options standard incluent les éléments suivants. Travaillez avec votre directeur général si vous préférez une capacité différente. - Pod de cypher. Le robot peut transporter un chiffre manifeste supplémentaire pour vous. Facilitateur. - Vol. Le robot peut parcourir une longue distance à chaque tour. Il peut vous transporter, mais seulement pendant une heure maximum entre chacun de vos jets de récupération de dix heures. Facilitateur. - Bouclier de Force. Le robot peut ériger un champ de force opaque de niveau 5 autour de lui et de toute personne à moins de 3 m de lui pendant une minute (ou jusqu'à ce qu'il soit détruit). Il ne peut pas recommencer avant votre prochain jet de récupération. Action. - Configuration laser montée. Le robot peut se reconfigurer et devenir une arme laser immobile sur un support à cardan. Dans cette configuration, le robot est une arme lourde qui inflige 7 points de dégâts. Si le robot agit comme une tourelle autonome, traitez-le comme un niveau inférieur à son niveau normal. Cependant, si le laser est tiré par vous ou par quelqu'un d'autre ayant votre permission, les attaques laser sont atténuées. Action de reconfiguration ; action pour revenir à la configuration normale du robot. (Robot Upgrade \textendash (179))

\subsection*{Modification magistrale de l'armure}\label{subsec:ab_masterful_armor_modification}

choisissez l'une des modifications suivantes à apporter à l'armure motorisée à partir de votre capacité d'armure motorisée. Si vous choisissez d'effectuer une modification différente plus tard, vous pouvez le faire, mais vous devez dépenser 2 XP à chaque fois et remplacer la modification mise à jour par la modification précédente.- Cypher Pod. L'armure assistée fournit une nacelle isolée dans laquelle vous pouvez transporter un cypher manifeste supplémentaire au-delà de ce que votre limite de cypher autorise normalement. Facilitateur. - Drone (3 points d'Intellect). Un drone de niveau 4 ne dépassant pas 30 cm de côté se lance depuis votre armure pendant une heure, volant sur une longue distance à chaque tour. Le drone vous accompagne et suit vos instructions. Il dispose de manipulateurs, lui permettant de tenter d'accomplir des tâches physiques. Vous ferez probablement des jets pour votre drone lorsqu'il entreprendra des actions. Un drone au combat ne réalise généralement pas d'attaques séparées mais vous aide dans les vôtres. Sur votre action, si le drone est à côté de vous, il vous sert d'atout pour une attaque que vous effectuez à votre tour. Si le drone est détruit, vous devez dépenser 2 XP supplémentaires pour le reconstruire ou choisir une autre modification d'armure magistrale. Action à initier. - Renforcement de terrain amélioré. Vous gagnez +1 en Armure lorsque vous portez votre armure assistée. Facilitateur. - Vol assisté par réacteur (3+ points de Puissance). Vous modifiez votre armure assistée pour vous permettre de décoller du sol et de voler une minute à la fois. Pour chaque niveau d'Effort appliqué, vous pouvez augmenter la durée d'une minute supplémentaire. Action. (Masterful Armor Modification \textendash (160))

\subsection*{Modifier l'appareil}\label{subsec:ab_modify_device}

Vous truquez un équipement mécanique ou électrique pour le faire fonctionner au-dessus de ses spécifications nominales pendant une durée très limitée. Pour ce faire, vous devez utiliser des pièces de rechange équivalentes à un article coûteux, disposer d'un kit scientifique de terrain (ou d'un laboratoire permanent, si vous y avez accès) et réussir une tâche basée sur l'intellect de difficulté 3. Une fois terminée, l'utilisation de l'appareil facilite toutes les tâches effectuées conjointement avec l'appareil, jusqu'à ce que l'appareil tombe inévitablement en panne. Par exemple, vous pourriez overclocker un ordinateur pour faciliter les tâches de recherche qui l'utilisent, modifier une machine à expresso pour que chaque tasse de café préparée avec soit meilleure, modifier le moteur d'une voiture pour qu'elle aille plus vite (ou modifier sa direction pour qu'elle se comporte mieux). ), et ainsi de suite. Chaque utilisation de l'appareil modifié nécessite un jet d'épuisement de 1 à 5 sur un d20. Action à initier, une heure à réaliser. (Modify Device \textendash (164))

\subsection*{Modifier la Puissance de l'Artefact}\label{subsec:ab_modify_artifact_power}

Vous ajoutez de manière permanente +1 au niveau d'un artefact jusqu'au niveau 5. La difficulté de cette tâche est égale au niveau supérieur modifié de l'artefact. Si la tâche échoue, l'artefact effectue un jet d'épuisement et n'avance pas en niveau. Une fois modifié, l'artefact ne peut plus être amélioré de la même manière. Action. (Modify Artifact Power \textendash (163))

\subsection*{Modifier les Cyphers}\label{subsec:ab_modify_cyphers}

vous pouvez prendre deux cyphers manifestes et truquer rapidement un nouveau cypher manifeste du même niveau que le cypher de niveau le plus bas. Vous déterminez la fonction du nouveau cypher, mais il doit être celui d'un cypher que vous avez utilisé auparavant (mais pas nécessairement celui que vous avez déjà construit). Le nouveau chiffre est un chiffre capricieux, comme ceux créés avec Toujours bricoler. Les deux chiffres originaux sont consommés au cours de ce processus. Cette capacité ne fonctionne pas si un ou plusieurs des chiffres originaux sont des chiffres capricieux. Action. (Modify Cyphers \textendash (CTS, 54))

\subsection*{Modèle prédictif}\label{subsec:ab_predictive_model}

Si vous avez utilisé l'équation prédictive sur une créature, un objet ou un lieu au cours des derniers jours, vous pouvez apprendre un fait aléatoire sur le sujet qui est pertinent pour un sujet que vous désignez. Si vous possédez également la capacité de saveur magique Prémonition, une utilisation de l'une ou l'autre capacité vous accorde deux faits aléatoires mais liés sur le sujet. De plus, vous pouvez utiliser Modèle Prédictif sur le même sujet plusieurs fois (même si vous avez appris le niveau d'une créature), mais à chaque fois que vous le faites, vous devez appliquer un niveau d'Effort supplémentaire par rapport à votre utilisation précédente. Action. (Predictive Model \textendash (171))

\subsection*{Moment magnifique}\label{subsec:ab_magnificent_moment}

si vous effectuez une attaque ou tentez une tâche avec l'action immédiate que vous obtenez en utilisant Saisissez l'instant, l'attaque ou la tâche est facilitée. Facilitateur. (Magnificent Moment \textendash (159))

\subsection*{Momentum}\label{subsec:ab_momentum}

Si vous utilisez une action pour vous déplacer, votre prochaine attaque effectuée avec une arme de mêlée avant la fin du tour suivant inflige 2 points de dégâts supplémentaires. Facilitateur. (Momentum \textendash (164))

\subsection*{Montrez-leur le chemin}\label{subsec:ab_show_them_the_way}

Votre présence submerge une créature que vous touchez et demandez de vous aider. Essentiellement, si la créature ne parvient pas à se défendre contre votre présence, vous contrôlez ses actions pendant dix minutes maximum. La cible doit être de niveau 3 ou inférieur. Une fois que vous avez établi le contrôle, vous le maintenez grâce à des instructions verbales. Vous pouvez permettre à la cible d'agir librement ou outrepasser le contrôle au cas par cas. En plus des options normales d'utilisation de l'Effort, vous pouvez choisir d'utiliser l'Effort pour augmenter le niveau maximum de la cible. Ainsi, pour affecter une cible de niveau 5 (deux niveaux au-dessus de la limite normale), vous devez appliquer deux niveaux d'Effort. Lorsque l'effet prend fin, la créature se souvient vaguement d'avoir fait votre volonté, mais c'est aussi flou qu'un rêve. Action à initier. (Show Them the Way \textendash (183))

\subsection*{Monture}\label{subsec:ab_mount}

Une créature de niveau 3 vous sert de monture et suit vos instructions. Tant que vous êtes monté dessus, la créature peut se déplacer et vous pouvez attaquer pendant votre tour, ce qui constitue un atout pour votre attaque. Vous et le MJ devez déterminer les détails de la créature, et vous ferez probablement des jets pour elle lorsqu'elle effectue des actions hors combat. La monture agit à votre tour. Si votre monture meurt, vous pouvez chasser dans la nature pendant 3d6 jours pour en trouver une nouvelle. Facilitateur. (Mount \textendash (164))

\subsection*{Moqueries confondantes}\label{subsec:ab_confounding_banter}

Vous crachez un flot d'absurdités pour distraire un ennemi à portée immédiate. En cas de réussite d'un jet d'Intellect, votre jet de défense contre la prochaine attaque de la créature avant la fin du tour suivant est facilité. Action. (Confounding Banter \textendash (121))

\subsection*{Mort rapide}\label{subsec:ab_quick_death}

Vous savez tuer rapidement. Lorsque vous frappez avec une attaque au corps à corps ou à distance, vous infligez 4 points de dégâts supplémentaires. Vous ne pouvez pas effectuer cette attaque en deux tours consécutifs. Action. (Quick Death \textendash (173))

\subsection*{Mot de commandement}\label{subsec:ab_word_of_command}

Vous prononcez un mot si puissant que pour l'investir pleinement, vous sacrifiez un chiffre en votre possession de niveau 6 ou supérieur. Vous envoyez le mot à une créature à longue portée que vous pouvez voir. La cible affectée doit obéir à l'ordre pendant plusieurs heures avant d'être libre d'agir comme elle le souhaite. Les cibles attaquées alors qu'elles sont sous l'effet du commandement peuvent se défendre. Les commandes typiques incluent « retraite », « calme », « viens » et « reste ». Le MJ décide comment la cible agit une fois qu'un ordre est donné. Action. (Word of Command \textendash (199))

\subsection*{Mot de mort}\label{subsec:ab_word_of_death}

Votre attaque est la prononciation d'un mot magique si terrible qu'il éteint la vie d'une cible vivante à courte portée. La cible doit être de niveau 1. En plus des options normales d'utilisation de l'Effort, vous pouvez choisir d'utiliser l'Effort pour augmenter le niveau maximum de la cible. Ainsi, pour tuer une cible de niveau 5 (quatre niveaux au-dessus de la limite normale), vous devez appliquer quatre niveaux d'Effort. Action. (Word of Death \textendash (200))

\subsection*{Mouvement enchanté}\label{subsec:ab_enchanted_movement}

Vous utilisez votre arme enchantée pour vous déplacer vers n'importe quel endroit situé à une longue distance que vous pouvez voir, à condition qu'il n'y ait aucun obstacle ou barrière sur votre chemin. La manière exacte dont cela se produit dépend de votre arme ; vous pouvez lancer votre marteau magique et être entraîné après lui, tirer une flèche avec votre arc qui vous tire vers l'avant comme une ligne de grappin, et ainsi de suite. En plus des options normales d'utilisation de l'Effort, vous pouvez choisir d'utiliser l'Effort pour augmenter la distance parcourue ; chaque niveau d'effort utilisé de cette manière augmente la portée de 100 pieds (30 m) supplémentaires. Si vous disposez d'une autre capacité (comme celle de votre type) qui vous permet de parcourir une longue distance, la portée de cette capacité et de celle-ci augmente jusqu'à très longue. Action. (Enchanted Movement \textendash (GF, 31)(CTS, 52))

\subsection*{Mouvements de culbute}\label{subsec:ab_tumbling_moves}

Lorsque vous utilisez une action pour vous déplacer, les jets de défense de Célérité sont allégés jusqu'à la fin de votre prochain tour. Facilitateur. (Tumbling Moves \textendash (194))

\subsection*{Multiples Proies}\label{subsec:ab_multiple_quarry}

Cette capacité fonctionne comme la capacité Proie sauf que vous pouvez sélectionner jusqu'à trois créatures comme proie. Vous devez être capable de voir les trois créatures lorsque vous lancez cette capacité. Si vous possédez Hunter's Drive, cela s'applique aux trois créatures. Action à initier. (Multiple Quarry \textendash (163))

\subsection*{Multiplicité}\label{subsec:ab_multiplicity}

Cette capacité fonctionne comme Duplicata, sauf que vous pouvez créer deux doublons. Action à initier. (Multiplicity \textendash (165))

\subsection*{Mur avec des dents}\label{subsec:ab_wall_with_teeth}

Vous infligez 2 points de dégâts supplémentaires avec toutes vos attaques lorsque vous utilisez votre capacité Mur vivant. Facilitateur. (Wall With Teeth \textendash (196))

\subsection*{Mur de Force}\label{subsec:ab_force_wall}

Vous pouvez déclencher l'énergie de votre Bouclier de Champ de Force pour qu'elle s'étende vers l'extérieur dans toutes les directions pour créer un plan immobile de force solide jusqu'à 20 pieds sur 20 pieds (6 m sur 6 m) pendant un maximum d'un heure ou jusqu'à ce que vous récupériez votre bouclier. (Le bouclier de force devient le mur de force.) Le plan du mur de force s'adapte à l'espace disponible. Tant que le mur de force reste en place, vous ne pouvez utiliser aucune de vos autres capacités nécessitant un Champ de force Shield. Action à initier. (Force Wall \textendash (143))

\subsection*{Mur de Granit}\label{subsec:ab_granite_wall}

Vous créez un mur de granit de niveau 6 à courte portée. Le mur a une épaisseur de 1 pied (30 cm) et une taille allant jusqu'à 20 pieds sur 20 pieds (6 m sur 6 m). Il semble reposer sur des bases solides et dure une dizaine d'heures. Si vous appliquez trois niveaux d'Effort, le mur est permanent jusqu'à sa destruction naturelle. Action à initier. (Granite Wall \textendash (146))

\subsection*{Mur de foudre}\label{subsec:ab_wall_of_lightning}

Vous créez une barrière d'électricité crépitante mesurant jusqu'à 2 500 pieds carrés (230 m²), façonnée comme vous le souhaitez. Le mur est une barrière de niveau 7. Toute personne se trouvant à proximité immédiate du mur subit automatiquement 10 points de dégâts. Le mur dure une heure. Action de créer. (Wall of Lightning \textendash (196))

\subsection*{Mur de protection}\label{subsec:ab_protective_wall}

lorsque vous engagez un combat directement lié à la défense d'une communauté à laquelle vous êtes associé, vous pouvez attaquer jusqu'à cinq ennemis différents en une seule action, à condition qu'ils soient tous à portée immédiate. Si vous touchez un attaquant, il est repoussé immédiatement. Toutes les attaques doivent être du même type (mêlée ou à distance). Effectuez un jet d'attaque séparé pour chaque ennemi. Vous restez limité par la quantité d'Effort que vous pouvez appliquer sur une action. Tout ce qui modifie votre attaque ou vos dégâts s'applique à toutes ces attaques. En plus des options normales d'utilisation de l'Effort, vous pouvez choisir d'utiliser l'Effort pour augmenter le nombre d'ennemis que vous pouvez attaquer avec cette capacité, un ennemi supplémentaire par niveau d'Effort. Facilitateur. (Protective Wall \textendash (172))

\subsection*{Mur vivant}\label{subsec:ab_living_wall}

vous spécifiez une zone confinée, comme une porte ouverte, un couloir ou un espace entre deux arbres, où vous vous trouvez. Pendant les dix minutes suivantes, si quelqu'un tente d'entrer ou de traverser cette zone et que vous ne le souhaitez pas, vous lancez une attaque automatique contre lui. Si vous frappez, non seulement vous infligez des dégâts, mais ils doivent également arrêter leur mouvement. Facilitateur. (Living Wall \textendash (158))

\subsection*{Muscles de fer}\label{subsec:ab_muscles_of_iron}

Pendant les dix prochaines minutes, toutes les actions basées sur la Puissance autres que les jets d'attaque que vous tentez sont facilitées. Si vous possédez déjà cette capacité provenant d'une autre source, l'effet de cette capacité dure une heure au lieu de dix minutes. Facilitateur. (Muscles of Iron \textendash (165))

\subsection*{Mécanicien de passage}\label{subsec:ab_passing_mechanic}

Vous êtes entraîné aux tâches liées à la réparation et à l'entretien d'un starcraft. Facilitateur. (Passing Mechanic \textendash (168))

\subsection*{Mécanismes de désactivation}\label{subsec:ab_disable_mechanisms}

Avec un œil vif et des mouvements rapides, vous perturbez certaines fonctions d'un robot ou d'une machine et lui infligez l'une des pénalités suivantes :- Toutes ses tâches sont entravées pendant une minute. - Sa Célérité est réduite de moitié. - Il ne peut entreprendre aucune action pendant un round. - Il inflige 2 points de dégâts en moins (minimum 1 point) pendant une minute. Vous devez toucher le robot ou la machine pour le perturber (si vous effectuez une attaque, cela n'inflige aucun dégât). Action. (Disable Mechanisms \textendash (128))

\subsection*{Mémoire de Pouvoir Copié}\label{subsec:ab_power_memory}

lorsque vous utilisez Pouvoir de copie, il vous suffit d'avoir vu la capacité utilisée au cours de la journée précédente (au lieu de l'heure précédente), et l'utilisation de l'effort prolonge la durée pendant laquelle votre copie peut atteindre un jour par niveau d'effort (au lieu de d'une heure par niveau). Facilitateur. (Power Memory \textendash (CTS, 55)))

\
%--------------------------
\section*{N}

\subsection*{Nager}\label{subsec:ab_swim}

Vous pouvez nager comme un poisson dans l'eau et un liquide similaire pendant une heure. Pour chaque niveau d'Effort appliqué, vous pouvez prolonger la durée d'une heure. Vous nagez environ 16 km par heure et vous n'êtes pas affecté par les courants dans l'eau. Action à initier. (Swim \textendash (188))

\subsection*{Nageur agile}\label{subsec:ab_nimble_swimmer}

vous êtes entraîné à toutes les actions de défense sous l'eau. Facilitateur. (Nimble Swimmer \textendash (166))

\subsection*{Nageur entraîné sous l'eau}\label{subsec:ab_trained_swimmer}

vous êtes entraîné aux tâches d'évasion, de perception, de furtivité et de nage, ainsi qu'aux tâches d'identification des créatures aquatiques et de la géographie. Facilitateur. (Trained Swimmer \textendash (193))

\subsection*{Ne jamais échouer}\label{subsec:ab_never_fumble}

si vous obtenez un 1 naturel lorsque vous attaquez avec l'arme de votre choix, vous pouvez ignorer ou annuler l'intrusion du MJ pour ce jet. Vous ne pourrez jamais être désarmé de l'arme que vous avez choisie et vous ne la laisserez jamais tomber accidentellement. Facilitateur. (Never Fumble \textendash (165))

\subsection*{Noblesse privilégiée}\label{subsec:ab_privileged_nobility}

Vous êtes habile à réclamer les récompenses qu'un passé noble peut générer. Une fois reconnu, vous pouvez vous asseoir dans n'importe quel établissement de restauration, aussi complet soit-il, obtenir une chambre dans une auberge même si cela signifie que d'autres sont expulsés, être admis dans n'importe quel tribunal ou autre structure où sont décidées les lois ou les règles de la noblesse, être invité à n'importe quel gala et obtenez une place à une réception privée de quelque nature que ce soit. De plus, vous êtes entraîné à la persuasion. Facilitateur. (Privileged Nobility \textendash (172))

\subsection*{Nuage de brume}\label{subsec:ab_mist_cloud}

Vous créez une zone de brume à une distance immédiate. Le nuage persiste pendant environ une minute, à moins que les conditions (telles que le vent ou les températures glaciales) n'en décident autrement. En plus des options normales d'utilisation de l'Effort, vous pouvez choisir d'utiliser l'Effort pour augmenter la zone (un niveau d'Effort pour remplir une zone courte, deux pour remplir une zone longue ou trois pour remplir une zone très longue). Action. (Mist Cloud \textendash (163))

\subsection*{Nuage de matière}\label{subsec:ab_matter_cloud}

Les cailloux, la terre, le sable et les débris s'élèvent dans l'air autour de vous pour former un nuage tourbillonnant. Le nuage s'étend à portée immédiate, se déplace avec vous et dure une minute. À la fin, tous les matériaux tombent au sol autour de vous. Le nuage rend plus difficile aux autres créatures de vous attaquer, ce qui vous donne un atout lors des jets de défense de Célérité. De plus, tant que le nuage est autour de vous, vous pouvez utiliser une action pour fouetter le matériau afin qu'il abrase tout ce qui se trouve à portée immédiate, infligeant 1 point de dégâts à chaque créature et objet dans la zone. Action à initier. (Matter Cloud \textendash (161))

\subsection*{Nécromancie}\label{subsec:ab_necromancy}

Vous animez le corps d'une créature morte d'environ votre taille ou moins, créant une créature de niveau 1. Il n'a ni l'intelligence, ni la mémoire, ni les capacités spéciales qu'il avait dans la vie. La créature suit vos ordres verbaux pendant une heure, après quoi elle devient un cadavre inerte. À moins que la créature ne soit tuée par des dégâts, vous pouvez la réanimer à nouveau lorsque son temps expire, mais tous les dégâts qu'elle a subis lorsqu'elle est devenue inerte s'appliquent à son état nouvellement réanimé. Si vous avez accès à plusieurs corps, vous pouvez créer une créature mort-vivante supplémentaire pour chaque point d'Intellect supplémentaire que vous dépensez lorsque vous activez la capacité. Action à animer. (Necromancy \textendash (165))

\subsection*{Nécromancie Supérieure}\label{subsec:ab_greater_necromancy}

Cette capacité fonctionne comme la capacité Nécromancie sauf qu'elle crée une créature de niveau 3. Action à animer. (Greater Necromancy \textendash (147))

\subsection*{Négocier}\label{subsec:ab_negotiate}

Dans toute réunion où deux personnes ou plus tentent d'établir la vérité ou de prendre une décision, vous pouvez influencer le verdict grâce à une rhétorique magistrale. Si vous disposez de quelques tours ou plus pour faire valoir votre point de vue, soit la décision passe en votre faveur, soit, si quelqu'un d'autre fait valoir un point concurrent, toute tâche de persuasion ou de tromperie associée est facilitée de deux étapes. Action à initier, un ou plusieurs tours à réaliser. (Negotiate \textendash (165))

\
%--------------------------
\section*{O}

\subsection*{Objectif du tireur d'élite}\label{subsec:ab_sniper's_aim}

à force de vous entraîner presque constamment à jouer à des jeux simulant des attaques à distance, votre coordination œil-main est hors du commun. Vous disposez d'un atout sur toutes les attaques à distance. Facilitateur. (Sniper's Aim \textendash (184))

\subsection*{Oeil perçant}\label{subsec:ab_sharp_eyed}

Parce que vous devez toujours garder un œil ouvert lorsque vous voyagez, vous êtes entraîné à toutes les tâches liées à la perception et à la navigation. Facilitateur. (Sharp-Eyed \textendash (182))

\subsection*{Oeil pour Cibler}\label{subsec:ab_targeting_eye}

vous êtes entraîné à toute attaque physique à distance qui est une capacité de personnage ou provient d'un appareil. Par exemple, vous êtes entraîné à utiliser une explosion de force Assaut Magique car il s'agit d'une attaque physique, mais pas à utiliser une tranche mentale Assaut Magique car il s'agit d'une attaque mentale. Facilitateur. (Targeting Eye \textendash (189))

\subsection*{Oeil pour les détails}\label{subsec:ab_eye_for_detail}

Lorsque vous passez environ cinq minutes à explorer en profondeur une zone ne dépassant pas une courte distance de diamètre, vous pouvez poser une question au MJ sur la zone. Le MJ doit vous répondre honnêtement. Vous ne pouvez pas l'utiliser plus d'une fois par zone et par 24 heures. Action à lancer, cinq minutes à terminer. (Eye for Detail \textendash (138))

\subsection*{Opportuniste}\label{subsec:ab_opportunist}

Vous disposez d'un atout sur tout jet d'attaque que vous effectuez contre une créature qui a déjà été attaquée à un moment donné au cours du tour et qui est à portée immédiate. Facilitateur. (Opportunist \textendash (167))

\subsection*{Orateur pour les morts}\label{subsec:ab_speaker_for_the_dead}

Vous pouvez poser une question à un être mort dont vous touchez le cadavre. Parce que la réponse passe par le filtre de la compréhension et de la personnalité de l'être, il ne peut pas répondre à des questions qu'il n'aurait pas comprises dans la vie, ni fournir des réponses qu'il n'aurait pas connues dans la vie. En fait, l'être n'est pas du tout obligé de répondre, vous devrez donc peut-être interagir avec lui d'une manière qui l'aurait convaincu de répondre de son vivant. Pour chaque point d'Intellect supplémentaire que vous dépensez lorsque vous activez la capacité, vous pouvez poser une question supplémentaire à l'être. Action. (Speaker for the Dead \textendash (184))

\subsection*{Ordonner au Vaisseau}\label{subsec:ab_shipspeak}

vous pouvez effectuer des manœuvres de base à distance planétaire avec un vaisseau spatial avec lequel vous vous êtes lié à l'aide de Liaison machine. Vous pouvez l'envoyer à un endroit désigné, l'appeler, le faire atterrir, autoriser ou refuser l'entrée, etc., même si vous n'êtes pas à bord. La création de liens est un processus qui nécessite une journée de méditation à bord du navire. Action. (Shipspeak \textendash (183))

\
%--------------------------
\section*{P}

\subsection*{Parade}\label{subsec:ab_parry}

Vous pouvez dévier rapidement les attaques entrantes. Lorsque vous activez cette capacité, pendant les dix tours suivants, vous ralentissez tous les jets de défense de Célérité. Facilitateur. (Parry \textendash (168))

\subsection*{Parfait Inconnu}\label{subsec:ab_perfect_stranger}

Vous modpoints de ifiez votre posture et votre façon de parler et apportez une petite mais réelle modification à une tenue (comme mettre ou enlever un chapeau, retourner une cape, etc.). Pendant l'heure suivante (ou aussi longtemps que vous poursuivez l'altération), même les créatures qui vous connaissent bien ne vous reconnaissent pas. Toutes les tâches liées à la dissimulation de votre véritable identité pendant cette période bénéficient d'un niveau d'effort gratuit. Action à initier. (Perfect Stranger \textendash (169))

\subsection*{Partager la défense}\label{subsec:ab_share_defense}

si votre entraînement à une tâche de défense est supérieur à celui d'un allié à courte portée, vos conseils et votre perspicacité leur permettent de remplacer votre entraînement pour cette tâche de défense. Facilitateur. (Share Defense \textendash (181))

\subsection*{Partager les sens}\label{subsec:ab_share_senses}

tant que votre double créé par la capacité Dupliquer existe et se trouve à moins de 1,5 km, vous savez tout ce qu'il vit et pouvez communiquer avec lui par télépathie. Facilitateur. (Share Senses \textendash (182))

\subsection*{Partagez le pouvoir}\label{subsec:ab_share_the_power}

lorsque vous utilisez Drain de Créature ou Drain de Machine pour drainer de l'énergie, vous pouvez la transférer à une autre créature, restaurant ainsi des points dans ses réserves de Puissance ou de Célérité (ou de santé pour un PNJ) au lieu de vous-même. Vous pouvez dépenser les points de votre réserve de siphon (de la capacité Stocker l'énergie) de la même manière. Vous devez toucher la créature que vous souhaitez soigner, à moins que vous n'ayez la capacité Drain à distance, auquel cas elle peut être jusqu'à une courte distance. Facilitateur. (Share the Power \textendash (182))

\subsection*{Pas besoin d'armes}\label{subsec:ab_no_need_for_weapons}

lorsque vous effectuez une attaque à mains nues (comme un coup de poing ou un coup de pied), cela compte comme une arme moyenne au lieu d'une arme légère. Facilitateur. (No Need for Weapons \textendash (166))

\subsection*{Pas encore mort}\label{subsec:ab_not_dead_yet}

alors que vous seriez normalement mort, vous tombez inconscient pendant un round, puis vous vous réveillez. Vous gagnez immédiatement 1d6 + 6 points pour restaurer vos réserves de statistiques, et vous êtes traité comme affaibli jusqu'à ce que vous vous reposiez pendant dix heures. Si vous mourez à nouveau avant d'avoir effectué votre jet de récupération de dix heures, vous êtes vraiment mort. Si vous disposez également de cette capacité provenant d'une autre source, vos soins grâce à cette capacité augmentent à 1d6 + 12. (Not Dead Yet \textendash (166))

\subsection*{Pas subtiles}\label{subsec:ab_subtle_steps}

lorsque vous ne bougez pas sur une courte distance, vous pouvez vous déplacer sans émettre de bruit, quelle que soit la surface sur laquelle vous vous déplacez. Facilitateur. (Subtle Steps \textendash (187))

\subsection*{Passager psychique}\label{subsec:ab_psychic_passenger}

vous placez votre esprit dans le corps d'une créature volontaire que vous choisissez à courte portée et restez dans ce corps pendant une heure maximum. Votre propre corps s'effondre et devient insensible jusqu'à ce que cette capacité prenne fin.Vous voyez, entendez, sentez, touchez et goûtez en utilisant les sens de la créature dont vous habitez le corps. Lorsque vous parlez, les mots viennent de votre corps sans défense, et la créature dans laquelle vous habitez entend ces mots dans son esprit. La créature que vous habitez peut utiliser votre Avantage d'Intellect à la place du sien. De plus, vous et la créature disposez d'un atout pour toute tâche impliquant la perception. Lorsque vous effectuez une action, vous utilisez le corps de la créature pour effectuer cette action si elle le permet. Action à initier. Un personnage qui utilise Passager psychique devrait envisager de cacher son corps réel quelque part à l'abri des regards indiscrets et des bêtes sauvages, sinon il pourrait retourner dans une situation malheureuse. (Psychic Passenger \textendash (172))

\subsection*{Passer l'information au suivant}\label{subsec:ab_pay_it_forward}

vous pouvez transmettre ce que vous avez appris. Lorsque vous donnez à un autre personnage une suggestion impliquant sa prochaine action qui n'est pas une attaque, son action est facilitée pendant une minute. Action. (Pay It Forward \textendash (168))

\subsection*{Passer à travers}\label{subsec:ab_push_on_through}

Vous ignorez les effets du terrain lorsque vous vous déplacez pendant une heure. Facilitateur. (Push on Through \textendash (173))

\subsection*{Penser à l'avance}\label{subsec:ab_thinking_ahead}

Vous produisez un remède qui supprime une condition négative parce que vous avez déjà passé beaucoup de temps à réfléchir à l'avance et à vous préparer à votre situation actuelle. Par exemple, si un autre personnage est empoisonné, vous produisez un antidote, ou s'il est aveuglé, vous produisez un baume qui lui rend la vue (en supposant qu'il n'a pas été aveuglé parce que ses yeux ont été détruits). Le coût en Intellect pour l'utilisation de cette capacité est égal au niveau de l'effet ou de la créature qui a provoqué la condition négative. Action. (Thinking Ahead \textendash (191))

\subsection*{Pensez à votre sortie}\label{subsec:ab_think_your_way_out}

lorsque vous le souhaitez, vous pouvez utiliser les points de votre réserve d'intelligence plutôt que de votre réserve de Puissance ou de votre Réserve de Célérité pour toute action hors combat. Facilitateur. (Think Your Way Out \textendash (191))

\subsection*{Percer les Défenses}\label{subsec:ab_pry_open}

Vous déchirez les défenses d'une créature à longue portée. Toutes les défenses basées sur l'énergie dont il dispose (comme un champ de force ou une capacité de protection) sont annulées pendant 1d6 + 1 rounds. Si la créature n'a aucune défense énergétique, son armure est réduite de 2 pendant une minute. S'il n'a pas de défenses énergétiques ou d'armure, les attaques contre lui sont atténuées pendant une minute. Action. (Pry Open \textendash (172))

\subsection*{Percée scientifique étrange}\label{subsec:ab_weird_science_breakthrough}

vos recherches mènent à une percée et vous confèrez à un objet une propriété vraiment étonnante, bien que vous ne puissiez utiliser l'objet qu'une seule fois. Pour ce faire, vous devez acheter des pièces de rechange équivalentes à un article coûteux, disposer d'un kit scientifique de terrain (ou d'un laboratoire permanent, si vous en avez accès) et réussir un jet basé sur l'Intellect de difficulté 4 pour créer un chiffre manifeste aléatoire. jusqu'au niveau 2. Le MJ décide de la nature du chiffre que vous créez. Tenter de créer un chiffre spécifié entrave la tâche de deux étapes. La création d'un chiffrement ne vous permet pas de dépasser votre limite de chiffrement normale. En plus des options normales d'utilisation d'Effort, vous pouvez choisir d'utiliser Effort pour augmenter le niveau du chiffre que vous créez ; chaque niveau d'effort augmente le niveau du chiffre et la difficulté de la tâche Intellect pour le créer. Action à initier, une heure à réaliser. (Weird Science Breakthrough \textendash (197))

\subsection*{Perdu dans le chaos}\label{subsec:ab_lost_in_the_chaos}

face à plusieurs ennemis à la fois, vous avez développé des tactiques pour utiliser leur nombre contre eux. Lorsque deux ennemis ou plus vous attaquent en même temps au corps à corps, vous affrontez l'un contre l'autre. Les jets de défense rapide ou les jets d'attaque (au choix) contre eux sont facilités. Facilitateur. (Lost in the Chaos \textendash (159))

\subsection*{Performance vindicative}\label{subsec:ab_vindictive_performance}

lorsque vous racontez une blague, interprétez une chanson ou un poème, faites un dessin, racontez une anecdote ou proposez un divertissement de toute autre manière, vous pouvez sélectionner une personne parmi le public qui est capable de vous comprendre. Lors de votre prestation, vous jetez une dérision indirecte mais mordante sur cette cible. Si vous réussissez, la cible ne se rend pas compte qu'elle est devenue la victime de votre performance jusqu'à ce que vous terminiez le divertissement au moment que vous choisissez d'une manière qui vous touche. La cible subit 6 points de dégâts d'Intellect (ignore l'Armure) et perd son prochain tour. Une ou plusieurs actions à initier. (Vindictive Performance \textendash (196))

\subsection*{Personne ne sait mieux}\label{subsec:ab_no_one_knows_better}

Vous êtes entraîné à deux des compétences suivantes : persuasion, tromperie, intimidation, recherche, connaissances dans un domaine ou voir à travers la tromperie. Si vous choisissez une compétence dans laquelle vous êtes déjà entraîné, vous vous spécialisez dans cette compétence. Facilitateur. (No One Knows Better \textendash (166))

\subsection*{Perspicacité}\label{subsec:ab_insight}

Vous êtes entraîné à des tâches permettant de discerner les focus des autres et de déterminer leur nature générale. Vous avez le don de détecter si quelqu'un est vraiment innocent ou non. Facilitateur. (Insight \textendash (154))

\subsection*{Persuasion et Tromperie}\label{subsec:ab_opening_statement}

Vous êtes entraîné aux tâches liées à la persuasion, à la tromperie et à la détection des mensonges des autres. Facilitateur. (Opening Statement \textendash (167))

\subsection*{Petit Compagnon}\label{subsec:ab_critter_companion}

Une créature de niveau 1 vous accompagne et suit vos instructions. Cette créature n'est pas plus grosse qu'un gros chat (environ 20 livres ou 9 kg) et est normalement une sorte d'espèce domestiquée. Vous et le MJ devez définir les détails de votre créature, et vous ferez probablement des jets pour elle en combat ou lorsqu'elle entreprend des actions. La créature compagnon agit à votre tour. En tant que créature de niveau 1, elle a un nombre cible de 3 et 3 points de vie, et elle inflige 1 point de dégâts. Son mouvement est basé sur son type de créature (aviaire, nageur, etc.). Si votre compagnon créature meurt, vous pouvez parcourir un environnement urbain ou sauvage pendant 1d6 jours pour en trouver un nouveau. Facilitateur. (Critter Companion \textendash (123))

\subsection*{Petit vol}\label{subsec:ab_small_flight}

Pendant l'heure suivante, lorsque vous utilisez Rétrécir, vous pouvez voler dans les airs. Vous pouvez accomplir ce vol en faisant pousser des ailes à partir de votre corps, en déployant les ailes de votre combinaison, en appelant une petite créature pour vous porter ou en « surfant » sur les courants aériens. Lorsque vous volez, vous pouvez vous déplacer sur une courte distance dans le cadre d'une autre action ou sur une longue distance si tout ce que vous faites à votre tour est de vous déplacer. Action à initier. (Small Flight \textendash (183))

\subsection*{Phase défensive}\label{subsec:ab_defensive_phasing}

Vous pouvez modifier votre phase pour que certaines attaques vous traversent sans danger. Pendant les dix minutes suivantes, vous gagnez un atout pour vos tâches de défense de Célérité, mais pendant ce temps, vous perdez tout bénéfice de l'armure que vous portez. Action à initier. (Defensive Phasing \textendash (127))

\subsection*{Physique amélioré}\label{subsec:ab_enhanced_physique}

vous gagnez 3 points à répartir entre vos réserves de Puissance et de Célérité comme vous le souhaitez. Facilitateur. (Enhanced Physique \textendash (135))

\subsection*{Physique amélioré supérieur}\label{subsec:ab_greater_enhanced_physique}

vous gagnez 6 points à répartir entre vos réserves de Puissance et de Célérité comme vous le souhaitez. Facilitateur. (Greater Enhanced Physique \textendash (146))

\subsection*{Physiquement doué}\label{subsec:ab_physically_gifted}

chaque fois que vous dépensez des points de votre réserve de Puissance ou de votre Réserve de Célérité pour une action pour quelque raison que ce soit, si vous obtenez un 1 sur le dé associé, vous relancez, en prenant toujours le deuxième résultat (même si c'est un autre 1). Facilitateur. (Physically Gifted \textendash (170))

\subsection*{Pied Léger}\label{subsec:ab_fleet_of_foot}

Vous pouvez vous déplacer sur une courte distance dans le cadre d'une autre action. Vous pouvez vous déplacer sur une longue distance pendant toute votre action pendant un tour. Si vous appliquez un niveau d'Effort à cette capacité, vous pouvez vous déplacer sur une longue distance et effectuer une attaque pendant toute votre action pendant un tour, mais l'attaque est gênée. Facilitateur. (Fleet of Foot \textendash (141))

\subsection*{Pilleur Entraîné}\label{subsec:ab_trained_excavator}

Vous êtes entraîné aux tâches de perception, d'escalade et de récupération. Facilitateur. (Trained Excavator \textendash (193))

\subsection*{Pilleur d'artifact}\label{subsec:ab_artifact_scavenger}

Vous avez développé un sixième sens pour rechercher les objets les plus précieux du désert. Si vous passez le temps nécessaire pour réussir deux tâches de récupération, vous pouvez échanger tous les résultats que vous obtiendriez autrement contre une chance d'obtenir un artefact choisi par le MJ si vous réussissez une tâche d'Intellect de difficulté 6. Vous pouvez utiliser cette capacité au maximum une fois par jour, et jamais dans la même zone générale. Action à initier, plusieurs heures à réaliser. (Artifact Scavenger \textendash (110))

\subsection*{Pilote}\label{subsec:ab_pilot}

Vous êtes entraîné à toutes les tâches liées au pilotage d'un vaisseau spatial. De manière générale, les tâches de pilotage sont des tâches basées sur la Célérité, bien que l'utilisation de capteurs et d'instruments de communication soit des tâches basées sur l'intellect. Facilitateur. (Pilot \textendash (170))

\subsection*{Pilote Expert}\label{subsec:ab_expert_pilot}

Vous êtes spécialisé dans toutes les tâches liées au pilotage d'un vaisseau spatial. Facilitateur. (Expert Pilot \textendash (137))

\subsection*{Pilote Incomparable}\label{subsec:ab_incomparable_pilot}

Lorsque vous êtes sur un vaisseau spatial que vous possédez ou avec lequel vous avez une connexion directe, votre Avantage de Puissance, Avantage de Célérité et Avantage d'Intellect augmentent de 1. Lorsque vous effectuez un jet de récupération sur un vaisseau spatial que vous connaissez, vous récupérez 5 points supplémentaires. Facilitateur. (Incomparable Pilot \textendash (152))

\subsection*{Piste Infernale}\label{subsec:ab_inferno_trail}

Pendant la minute suivante, vous laissez une traînée de flammes dans votre sillage. La piste correspond à votre chemin et dure jusqu'à une minute, créant un mur de flammes d'environ 6 pieds (2 m) de haut qui inflige 5 points de dégâts à toute créature qui la traverse, les prenant potentiellement en feu pour 1 point supplémentaire. de dégâts à chaque tour (s'ils sont inflammables) jusqu'à ce qu'ils passent un tour à éteindre le feu. Action. (Inferno Trail \textendash (153))

\subsection*{Pisteur}\label{subsec:ab_tracker}

Vous êtes entraîné au suivi et à l'identification des traces. Facilitateur. (Tracker \textendash (193))

\subsection*{Piétinement de Golem}\label{subsec:ab_golem_stomp}

Vous piétinez le sol de toutes vos forces, créant une onde de choc qui attaque toutes les créatures à portée immédiate. Les créatures affectées subissent 3 points de dégâts et sont soit poussées hors de portée immédiate, soit tombent (votre choix). Action. (Golem Stomp \textendash (145))

\subsection*{Placard caché}\label{subsec:ab_hidden_closet}

l'allié magique de votre capacité Créature magique liée peut stocker des objets pour vous dans son objet lié, notamment des ensembles supplémentaires de vêtements, d'outils, de nourriture, etc. L'intérieur de l'objet est, en fait, une dimension de poche carrée de 3 mètres (10 pieds) à laquelle seul l'allié magique peut normalement accéder. Facilitateur. (Hidden Closet \textendash (149))

\subsection*{Plan d'évasion}\label{subsec:ab_escape_plan}

lorsque vous tuez un ennemi, vous pouvez tenter une tâche furtive pour vous cacher immédiatement de quiconque se trouve à proximité, en supposant qu'une cachette appropriée se trouve à proximité. Facilitateur. (Escape Plan \textendash (136))

\subsection*{Planificateur méticuleux}\label{subsec:ab_meticulous_planner}

Si vous passez beaucoup de temps à planifier une action, vous gagnez un atout pour la réaliser. Le temps nécessaire pour étudier et planifier l'action est dix fois plus long que celui nécessaire pour réaliser l'action. Par exemple, si vous souhaitez sauter par-dessus un trou dans le sol (une action), vous pouvez étudier la zone pendant dix tours (environ une minute), et lorsque vous tentez de sauter par-dessus le trou, vous disposez d'un atout sur le saut. . Cet avantage ne s'applique qu'à un seul jet : si vous souhaitez effectuer à nouveau la tâche avec le bénéfice d'un atout, vous devez étudier et planifier à nouveau. Facilitateur. (Meticulous Planner \textendash (162))

\subsection*{Plongeur}\label{subsec:ab_diver}

Vous pouvez plonger en toute sécurité dans l'eau depuis des hauteurs allant jusqu'à 100 pieds (30 m) et vous pouvez résister à la pression lorsque vous êtes dans une eau aussi profonde que 100 pieds (30 m). Facilitateur. (Diver \textendash (130))

\subsection*{Plus forts ensemble}\label{subsec:ab_stronger_together}

lorsque vous et votre compagnon de la capacité Une Bête comme Compagnon êtes à distance immédiate l'un de l'autre, vous infligez 2 points de dégâts supplémentaires lorsque vous attaquez et vous gagnez tous les deux un atout pour les actions de défense. Facilitateur. (Stronger Together \textendash (187))

\subsection*{Plus grand}\label{subsec:ab_bigger}

lorsque vous utilisez Agrandir, vous pouvez choisir de grandir jusqu'à 12 pieds (4 m) de hauteur et vous ajoutez 3 points temporaires supplémentaires à votre réserve de Puissance. Facilitateur. (Bigger \textendash (113))

\subsection*{Plus petit}\label{subsec:ab_smaller}

lorsque vous utilisez Rétrécir, vous pouvez choisir de rétrécir jusqu'à environ un demi-pouce (1 cm) de hauteur et vous ajoutez 3 points temporaires supplémentaires à votre Réserve de Célérité. Facilitateur. (Smaller \textendash (183))

\subsection*{Plus rapide que la plupart}\label{subsec:ab_quicker_than_most}

l'expérience a affiné vos temps de réaction, car ceux qui agissent en premier obtiennent l'avantage dans la plupart des situations. Vous êtes entraîné aux tâches liées à l'initiative, à la détection de modèles sous-jacents et à la résolution d'énigmes. Facilitateur. (Quicker Than Most \textendash (174))

\subsection*{Poche de Phase}\label{subsec:ab_phased_pocket}

Vous vous connectez pendant une heure à un petit espace déphasé et qui bouge avec vous. Vous pouvez accéder à cet espace comme s'il s'agissait d'une poche ou d'un sac pratique, mais personne d'autre ne peut percevoir ou accéder à l'espace à moins d'avoir la capacité d'interagir avec les zones transdimensionnelles. L'espace peut contenir jusqu'à 1 pied cube. L'espace fait partie de vous, vous ne pouvez donc pas l'utiliser pour transporter plus de chiffres que votre limite, un chiffre de détonation activé à l'intérieur de l'espace vous nuit, et ainsi de suite. Lorsque la connexion prend fin, tout ce qui se trouve dans l'espace tombe. Pour chaque 2 points d'Intellect supplémentaires dépensés, la poche dure une heure supplémentaire. Facilitateur. (Phased Pocket \textendash (170))

\subsection*{Poids du monde}\label{subsec:ab_weight_of_the_world}

Vous pouvez augmenter considérablement le poids d'une cible. La cible est tirée au sol et ne peut pas bouger physiquement par ses propres moyens pendant une minute. La cible doit être à courte portée. En plus des options normales d'utilisation de l'Effort, vous pouvez choisir d'utiliser l'Effort pour affecter des créatures supplémentaires (une par niveau d'Effort). Action. (Weight of the World \textendash (197))

\subsection*{Poignée élastique}\label{subsec:ab_elastic_grip}

Votre attaque avec vos membres ou votre corps extensibles est facilitée. Si vous touchez, vous pouvez attraper la cible, l'empêchant de bouger lors de son prochain tour. Tant que vous tenez la cible, ses attaques ou ses tentatives de libération sont entravées. Si la cible tente de se libérer au lieu d'attaquer, vous devez réussir une tâche basée sur la Puissance pour maintenir votre emprise. Si la cible ne parvient pas à se libérer, vous pouvez continuer à la maintenir à chaque tour lors de vos actions suivantes, infligeant automatiquement 4 points de dégâts à chaque tour en la serrant. Facilitateur. (Elastic Grip \textendash (133))

\subsection*{Poing de Fer}\label{subsec:ab_iron_fist}

Vos attaques à mains nues infligent 4 points de dégâts. Facilitateur. (Iron Fist \textendash (155))

\subsection*{Poings de fureur}\label{subsec:ab_fists_of_fury}

Vous infligez 2 points de dégâts supplémentaires avec des attaques à mains nues. Facilitateur. (Fists of Fury \textendash (140))

\subsection*{Porte de phase}\label{subsec:ab_phase_door}

Vous pouvez entrer en phase dans la surface d'un objet solide, puis sortir de tout autre objet solide à longue portée du premier, même si les deux objets ne sont pas connectés. Il ne doit y avoir aucune barrière intermédiaire entre les deux objets et vous devez être conscient ou capable de voir l'objet de destination. Action. (Phase Door \textendash (170))

\subsection*{Porter un jugement}\label{subsec:ab_make_judgment}

vous êtes entraîné à discerner la vérité d'une situation, à voir à travers les mensonges ou à surmonter la tromperie. Facilitateur. (Make Judgment \textendash (160))

\subsection*{Portez-la bien}\label{subsec:ab_wear_it_well}

lorsque vous portez une armure de quelque nature que ce soit, vous gagnez +1 supplémentaire en armure. Facilitateur. (Wear It Well \textendash (197))

\subsection*{Poser des Pièges}\label{subsec:ab_trapster}

Vous êtes entraîné à la création de pièges simples pour des cibles de taille humaine ou plus petites, en particulier de nombreuses variétés de pièges et de pièges utilisant des objets naturels de l'environnement. Lorsque vous posez un piège, décidez si vous souhaitez maintenir la victime en place (un piège) ou lui infliger des dégâts (une chute mortelle). Créer un piège est une tâche de difficulté 3, tandis que la difficulté de créer une chute mortelle est égale au nombre de points de dégâts que vous souhaitez qu'il inflige. Par exemple, si vous souhaitez infliger 4 points de dégâts, c'est une tâche de difficulté 4 (l'entraînement qui accompagne cette capacité facilite la tâche). En cas de succès, vous créez votre piège à usage unique en une minute environ, et il est considéré comme de niveau 3 pour éviter d'être détecté avant qu'il ne soit déclenché et pour une victime essayant de se libérer (s'il s'agit d'un piège). Action à initier, une minute ou une heure à réaliser. (Trapster \textendash (193))

\subsection*{Position fortifiée}\label{subsec:ab_fortified_position}

pendant la minute suivante, vous gagnez +1 en armure et un atout pour vos tâches de défense de Puissance, tant que vous ne vous êtes pas déplacé de plus d'une distance immédiate depuis votre dernier tour. Action à initier. (Fortified Position \textendash (143))

\subsection*{Potentiel amélioré}\label{subsec:ab_enhanced_potential}

vous gagnez 3 points à répartir entre vos pools de statistiques comme vous le souhaitez. Facilitateur. (Enhanced Potential \textendash (135))

\subsection*{Potentiel amélioré plus important}\label{subsec:ab_greater_enhanced_potential}

vous gagnez 6 points à répartir entre vos pools de statistiques comme vous le souhaitez. Facilitateur. (Greater Enhanced Potential \textendash (146))

\subsection*{Poussière Retourne à la Poussière}\label{subsec:ab_dust_to_dust}

Vous désintégrez un objet plus petit que vous et dont le niveau est inférieur ou égal à votre niveau. Vous devez toucher l'objet pour l'affecter. Si le MJ le juge approprié selon les circonstances, vous pouvez désintégrer une partie d'un objet (dont le volume total est plus petit que vous) plutôt que la totalité. Action. (Dust to Dust \textendash (133))

\subsection*{Poussée}\label{subsec:ab_push}

Vous poussez par télékinésie une créature ou un objet à une distance immédiate dans n'importe quelle direction de votre choix. Vous devez être capable de voir la cible, qui doit être de votre taille ou plus petite, ne doit être fixée à rien et doit être à courte portée. La poussée est rapide et la force est trop grossière pour être manipulée. Par exemple, vous ne pouvez pas utiliser cette capacité pour tirer un levier ou fermer une porte. Action. (Push \textendash (173))

\subsection*{Pouvoir de copie}\label{subsec:ab_copy_power}

Vous pouvez copier le super pouvoir de quelqu'un d'autre pendant une heure, en l'exécutant comme si cela était naturel pour vous. Au cours de la dernière heure, vous devez avoir touché la créature dont vous souhaitez copier le pouvoir (un jet d'attaque) et devez avoir vu cette capacité utilisée par elle. Choisissez le pouvoir que vous souhaitez copier et le MJ choisit une capacité de bas niveau appropriée qui ressemble le plus à ce pouvoir. Par exemple, si vous combattez un super-vilain capable de créer des explosions de force, si vous copiez cette capacité, vous obtenez une capacité de niveau inférieur qui crée une explosion de force.En plus du coût en points de la Puissance de copie, vous devez payer le coût en Puissance, en Célérité ou en intelligence (le cas échéant) de la capacité équivalente choisie par le MJ. Par exemple, si vous souhaitez copier l'explosion de force d'un super-vilain, le MJ décidera probablement que cela équivaut à **Assaut**, vous paierez donc 2 points d'Intellect pour activer le pouvoir de copie et 1 point d'Intellect pour utiliser Assaut. Vous ne pouvez copier qu'un seul pouvoir à la fois ; en copier un autre met fin à tout autre pouvoir que vous copiez avec cette capacité. Le pouvoir de copie ne copie pas les effets d'un pouvoir qui ajoute de manière permanente des points à vos réserves, comme Corps amélioré. En plus des options normales d'utilisation de l'Effort, vous pouvez choisir d'utiliser l'Effort pour copier une capacité que vous avez vue il y a plus d'une heure ; chaque niveau d'Effort utilisé de cette manière prolonge la période d'une heure. Action. (Copy Power \textendash (122))

\subsection*{Pouvoir inné}\label{subsec:ab_innate_power}

choisissez votre pool de Puissance ou votre pool de Célérité. Lorsque vous dépensez des points pour activer vos capacités de concentration, vous pouvez dépenser des points de cette réserve au lieu de votre réserve d'intelligence (auquel cas vous utilisez votre Avantage de Puissance ou votre Avantage de Célérité au lieu de votre avantage d'intelligence, selon le cas). Facilitateur (Innate Power \textendash (154))

\subsection*{Pouvoirs génériques}\label{subsec:ab_wildcard_powers}

vous avez le don d'utiliser les pouvoirs copiés de manière inhabituelle. Chaque fois que vous essayez une cascade de Puissance et utilisez un niveau d'effort sur le jet spécial pour modifier la capacité, vous obtenez un niveau d'effort gratuit sur ce jet. Facilitateur. (Wildcard Powers \textendash (198))

\subsection*{Pratique de toutes les armes}\label{subsec:ab_practiced_with_all_weapons}

Vous vous entraînez avec des armes légères, moyennes et lourdes et ne subissez aucune pénalité lorsque vous utilisez n'importe quel type d'arme. Facilitateur. (Practiced With All Weapons \textendash (171))

\subsection*{Pratique des armes moyennes}\label{subsec:ab_practiced_with_medium_weapons}

Vous pouvez utiliser des armes légères et moyennes sans pénalité. Si vous utilisez une arme lourde, les attaques avec celle-ci sont entravées. Facilitateur. (Practiced With Medium Weapons \textendash (171))

\subsection*{Pratique des armes à feu}\label{subsec:ab_practiced_with_guns}

Vous êtes entraîné avec des armes à feu et ne subissez aucune pénalité lorsque vous en utilisez une. Facilitateur. (Practiced With Guns \textendash (171))

\subsection*{Pratique des armures}\label{subsec:ab_practiced_in_armor}

Vous pouvez porter une armure pendant de longues périodes sans vous fatiguer et compenser les réactions ralenties liées au port d'une armure. Vous réduisez le coût en Célérité du port d'une armure de 1. Vous démarrez le jeu avec un type d'armure de votre choix. Facilitateur. (Practiced in Armor \textendash (171))

\subsection*{Pratique des épées}\label{subsec:ab_practiced_with_swords}

Vous êtes entraîné avec les épées et pouvez les utiliser sans pénalité. Facilitateur. (Practiced With Swords \textendash (171))

\subsection*{Premier Soins}\label{subsec:ab_wound_tender}

Vous êtes entraîné à la guérison. Facilitateur. (Wound Tender \textendash (200))

\subsection*{Prendre l'avantage}\label{subsec:ab_taking_advantage}

lorsque votre ennemi est affaibli, étourdi, étourdi, déplacé sur la piste des dégâts ou désavantagé d'une autre manière, vos attaques contre cet ennemi sont facilitées au-delà de toute autre modification en raison du désavantage. Facilitateur. (Taking Advantage \textendash (188))

\subsection*{Prendre l'initiative}\label{subsec:ab_seize_the_initiative}

Dans la minute qui suit l'utilisation réussie de votre capacité Tirer la conclusion, vous pouvez effectuer une action supplémentaire immédiate, que vous pouvez effectuer hors de votre tour. Après avoir utilisé cette capacité, vous ne pouvez plus l'utiliser avant votre prochain jet de récupération de dix heures. Facilitateur. (Seize the Initiative \textendash (181))

\subsection*{Prendre le commandement}\label{subsec:ab_take_command}

vous donnez un ordre spécifique à un autre personnage. Si ce personnage choisit d'écouter, toute attaque qu'il tentera lors de son prochain tour sera atténuée et un coup infligera 3 points de dégâts supplémentaires. Si votre commandement consiste à effectuer une tâche autre qu'une attaque, la tâche est allégée comme si elle bénéficiait d'un niveau d'Effort gratuit. Action. (Take Command \textendash (188))

\subsection*{Prestidigitation}\label{subsec:ab_legerdemain}

Vous pouvez réaliser de petits tours apparemment impossibles. Par exemple, vous pouvez faire disparaître un petit objet dans vos mains et le déplacer vers un endroit souhaité à portée de main (comme votre poche). Vous pouvez faire croire à quelqu'un qu'il a en sa possession quelque chose qu'il n'a pas (ou vice versa). Vous pouvez changer d'objets similaires juste devant les yeux de quelqu'un. Action. (Legerdemain \textendash (157))

\subsection*{Prise de Golem}\label{subsec:ab_golem_grip}

Votre attaque avec les poings de pierre de votre capacité Corps de Golem est facilitée. Si vous touchez, vous pouvez attraper la cible, l'empêchant de bouger lors de son prochain tour. Tant que vous tenez la cible, ses attaques ou ses tentatives de libération sont entravées. Si la cible tente de se libérer au lieu d'attaquer, vous devez effectuer un jet basé sur la Puissance pour maintenir votre emprise. Si la cible ne parvient pas à se libérer, vous pouvez continuer à la maintenir à chaque tour lors de vos actions suivantes, infligeant automatiquement 4 points de dégâts à chaque tour en la serrant. Facilitateur. (Golem Grip \textendash (145))

\subsection*{Processeur d'action}\label{subsec:ab_action_processor}

En vous appuyant sur les informations stockées et la capacité de traiter les données entrantes à une rapidité incroyable, vous êtes entraîné à une tâche physique de votre choix pendant dix minutes. Par exemple, vous pouvez choisir de courir, d'escalader, de nager, de défendre votre Célérité ou d'attaquer avec une arme spécifique. Action à initier. (Action Processor \textendash (108))

\subsection*{Programmation informatique}\label{subsec:ab_computer_programming}

Vous êtes entraîné à l'utilisation (et à l'exploitation) de logiciels informatiques, vous connaissez suffisamment bien un ou plusieurs langages informatiques pour écrire des programmes de base et vous maîtrisez le protocole Internet. Facilitateur. (Computer Programming \textendash (121))

\subsection*{Proie}\label{subsec:ab_quarry}

Choisissez une proie (une seule créature individuelle que vous pouvez voir). Vous êtes entraîné à toutes les tâches impliquant de suivre, comprendre, interagir avec ou combattre cette créature. Vous ne pouvez avoir qu'une seule carrière à la fois. Action à initier. (Quarry \textendash (173))

\subsection*{Projection}\label{subsec:ab_projection}

Vous projetez une image de vous-même vers n'importe quel endroit que vous avez vu ou visité précédemment. La distance n'a pas d'importance tant que l'emplacement se trouve dans le même monde que vous. La projection copie votre apparence, vos mouvements et tous les sons que vous émettez pendant les dix minutes suivantes. Toute personne présente sur place peut vous voir et vous entendre comme si vous y étiez. Cependant, vous ne percevez pas à travers votre projection. Action à initier. (Projection \textendash (172))

\subsection*{Projection Psychique}\label{subsec:ab_psychic_burst}

Vous projetez des vagues de force mentale dans l'esprit d'un maximum de trois cibles à courte portée (effectuez un jet d'Intellect contre chaque cible). Cette rafale inflige 3 points de dégâts d'Intellect (ignore l'Armure). Pour chaque tranche de 2 points d'Intellect supplémentaires dépensés, vous pouvez effectuer un jet d'attaque d'Intellect contre une cible supplémentaire. Action. (Psychic Burst \textendash (172))

\subsection*{Projection mentale}\label{subsec:ab_mental_projection}

Votre esprit quitte complètement votre corps et se manifeste partout où vous choisissez à portée immédiate. Votre esprit projeté peut rester séparé de votre corps pendant 24 heures maximum. Cet effet se termine plus tôt si votre réserve d'intelligence est réduite à 0 ou si votre projection touche votre corps au repos. Votre esprit désincarné est une construction psychique qui vous ressemble, même si ses bords effilochés se prolongent dans le néant. Vous contrôlez ce corps comme s'il s'agissait de votre corps normal et pouvez agir et bouger comme vous le feriez normalement, à quelques exceptions près. Vous pouvez vous déplacer à travers des objets solides comme si vous étiez en phase et vous ignorez toute caractéristique du terrain qui gênerait votre mouvement. Vos attaques infligent 3 points de dégâts de moins (jusqu'à un minimum de 1) et vous subissez 3 points de dégâts de moins (jusqu'à un minimum de 1) des attaques physiques, à moins qu'elles ne puissent affecter des êtres transdimensionnels ou phasés, auquel cas vous subissez la totalité des dégâts. . Quelle que soit la source, vous subissez d'abord tous les dégâts sous forme de dégâts d'Intellect. Votre esprit peut parcourir jusqu'à 1,5 km de votre corps. Chaque niveau d'effort supplémentaire appliqué étend la distance que vous pouvez parcourir de 1,5 km. Votre corps physique est impuissant jusqu'à ce que cet effet prenne fin. Vous ne pouvez pas utiliser vos sens physiques pour percevoir quoi que ce soit. Par exemple, votre corps pourrait subir une blessure importante sans que vous le sachiez. Votre corps ne peut pas subir de dégâts d'Intellect, donc si votre corps subit suffisamment de dégâts pour réduire à la fois votre réserve de Puissance et votre Réserve de Célérité à 0, votre esprit revient à votre corps et vous restez étourdi jusqu'à la fin du tour suivant alors que vous essayez de réorientez-vous vers votre situation difficile. Action à initier. Les personnages qui se projettent mentalement peuvent attirer des entités psychiques et des prédateurs auxquels les PJ n'ont normalement pas à faire face, se heurter à des phénomènes psychiques météorologiques qui risquent de rompre leur connexion, et peut-être même se perdre sur un plan métaphysique différent. (Mental Projection \textendash (161))

\subsection*{Protecteur}\label{subsec:ab_protector}

Vous désignez un seul personnage pour être votre responsable. Vous pouvez changer cela librement à chaque tour, mais vous ne pouvez avoir qu'une seule charge à la fois. Tant que cette charge est à portée immédiate, ils gagnent un atout pour les tâches de défense rapide car vous les soutenez. Facilitateur. (Protector \textendash (172))

\subsection*{Protection}\label{subsec:ab_ward}

Vous disposez à tout moment d'un bouclier d'énergie autour de vous qui vous aide à dévier les attaques. Vous gagnez +1 en Armure. Facilitateur. (Ward \textendash (196))

\subsection*{Protection élémentaire}\label{subsec:ab_elemental_protection}

vous et chaque cible que vous désignez à portée immédiate gagnez +5 d'armure contre un type de dégâts élémentaires directs (comme le feu, la foudre, l'ombre ou l'épine) pendant une heure ou jusqu'à ce que vous la lanciez. épeler à nouveau. Chaque niveau d'Effort appliqué augmente la protection élémentaire de +2. Action à initier. (Elemental Protection \textendash (133))

\subsection*{Protection énergétique}\label{subsec:ab_energy_protection}

Choisissez un type d'énergie discret avec lequel vous avez de l'expérience (comme la chaleur, le son, l'électricité, etc.). Vous gagnez +10 en Armure contre les dégâts de ce type d'énergie pendant dix minutes. Alternativement, vous gagnez +1 en Armure contre les dégâts causés par cette énergie pendant 24 heures. Vous devez être familier avec le type d'énergie ; par exemple, si vous n'avez aucune expérience avec un certain type d'énergie extradimensionnelle, vous ne pouvez pas vous en protéger. En plus des options normales d'utilisation de l'Effort, vous pouvez choisir d'utiliser l'Effort pour protéger plus de cibles ; chaque niveau d'Effort utilisé de cette manière affecte jusqu'à deux cibles supplémentaires. Vous devez toucher des cibles supplémentaires pour les protéger. Action à initier. (Energy Protection \textendash (134))

\subsection*{Prouesses au combat}\label{subsec:ab_combat_prowess}

vous ajoutez +1 dégâts à un type d'attaque avec une arme de votre choix: attaques avec une arme de mêlée ou attaques avec une arme à distance. Facilitateur. (Combat Prowess \textendash (120))

\subsection*{Provoquer l'ennemi}\label{subsec:ab_taunt_foe}

vous pouvez lancer une attaque sur un ennemi dans le cadre d'une attaque (ce qui n'est pas quelque chose que vous pouvez faire normalement lorsque vous tentez de lancer une attaque). Dans les cas où un ennemi intelligent ou déterminé n'est pas attiré par vous, vous pouvez tenter une action Intellect dans le cadre de l'attaque. Si cette action Intellect réussit, l'ennemi vous attaque. Vos défenses contre cette attaque sont gênées d'un pas, au lieu d'être gênées de deux pas comme d'habitude lorsque vous lancez une attaque. Facilitateur. (Taunt Foe \textendash (189))

\subsection*{Précision}\label{subsec:ab_precision}

Vous infligez 2 points de dégâts supplémentaires avec les attaques utilisant les armes que vous lancez. Facilitateur. (Precision \textendash (171))

\subsection*{Précognition}\label{subsec:ab_precognition}

Vous ressentez vaguement l'avenir pendant les dix prochaines minutes. Cela a les effets suivants jusqu'à l'expiration de la durée :- Vos tâches de défense gagnent un atout. - Vous pouvez prédire les actions de ceux qui vous entourent. Vous gagnez un atout pour déjouer les tromperies et les tentatives de vous trahir ainsi que pour éviter les pièges et les embuscades. - Vous savez ce que les gens pensent probablement et ce qu'ils diront avant de le dire, ce qui vous donne un avantage. Vous acquérez un atout pour toutes les compétences d'interaction. Facilitateur. (Precognition \textendash (171))

\subsection*{Prémonition}\label{subsec:ab_premonition}

Vous apprenez un fait aléatoire sur une créature ou un lieu pertinent pour un sujet que vous désignez. Alternativement, vous pouvez choisir d'apprendre le niveau d'une créature ; cependant, si vous le faites, vous ne pourrez rien apprendre d'autre plus tard avec cette capacité. Action. (Premonition \textendash (171))

\subsection*{Présence encourageante}\label{subsec:ab_encouraging_presence}

Pendant une minute, les alliés à courte portée gagnent un atout aux jets de défense. Action. (Encouraging Presence \textendash (134))

\subsection*{Présence reposante}\label{subsec:ab_restful_presence}

Les créatures qui effectuent un jet de récupération à courte portée de vous ajoutent +1 à leur jet. Facilitateur. (Restful Presence \textendash (177))

\subsection*{Présence terrifiante}\label{subsec:ab_terrifying_presence}

Vous convainquez une cible intelligente de niveau 3 ou inférieur que vous êtes son pire cauchemar. La cible doit être à courte portée et être capable de vous comprendre. Tant que vous ne faites que parler (vous ne pouvez même pas bouger), la cible est paralysée par la peur, s'enfuit ou prend toute autre mesure appropriée aux circonstances. En plus des options normales d'utilisation de l'Effort, vous pouvez choisir d'utiliser l'Effort pour augmenter le niveau maximum de la cible. Ainsi, pour terroriser une cible de niveau 5 (deux niveaux au-dessus de la limite normale), vous devez appliquer deux niveaux d'Effort. Action. (Terrifying Presence \textendash (190))

\subsection*{Prêter une forme animale}\label{subsec:ab_lend_animal_shape}

vous vous transformez en animal, et une créature volontaire à portée immédiate se transforme également en un animal de ce type (ours, tigre, loup, etc.) pendant dix minutes, comme si elle l'était. en utilisant votre capacité Forme animale. Pour chaque niveau d'Effort appliqué, vous pouvez affecter une créature supplémentaire. Toutes les créatures qui se transforment avec vous doivent être de votre taille ou moins. Une créature peut reprendre sa forme normale par une action, mais elle ne peut pas ensuite reprendre sa forme animale. Une créature (que ce soit vous ou quelqu'un d'autre) qui change de forme n'affecte aucune autre créature affectée par cette capacité. Action. Une créature qui prend une forme animale avec Lend Forme animale compte comme un animal pour l'utilisation de Analyse d'animal. Un personnage peut être capable de prendre la forme d'une créature similaire à un animal commun, comme une licorne au lieu d'un cheval ou un basilic au lieu d'un lézard, mais cela devrait nécessiter l'application d'au moins un niveau d'effort au changer, et le personnage ne gagnerait aucune des capacités magiques de la créature. (Lend Animal Shape \textendash (GF, 32)(CTS, 53))

\subsection*{Psychose}\label{subsec:ab_psychosis}

vos paroles infligent une psychose destructrice dans l'esprit d'une cible à longue portée qui peut vous comprendre, infligeant 6 points de dégâts d'Intellect (ignore l'armure) par tour. La psychose peut être dispersée si une cible utilise une action qui ne fait que se calmer et se recentrer. Action à initier. (Psychosis \textendash (172))

\subsection*{Puissance Améliorée}\label{subsec:ab_enhanced_might}

vous gagnez 3 points dans votre réserve de Puissance. Facilitateur. (Enhanced Might \textendash (135))

\subsection*{Puissance améliorée supérieure}\label{subsec:ab_greater_enhanced_might}

vous gagnez 6 points dans votre réserve de Puissance. Facilitateur. (Greater Enhanced Might \textendash (146))

\subsection*{Puissance d'attraction}\label{subsec:ab_coaxing_power}

Vous augmentez la Puissance ou la fonction d'une machine afin qu'elle fonctionne à un niveau supérieur à la normale pendant une heure. Action à initier. (Coaxing Power \textendash (119))

\subsection*{Pulsation de Guérison}\label{subsec:ab_healing_pulse}

Vous et toutes les cibles que vous choisissez à portée immédiate bénéficiez des avantages immédiats de l'utilisation de l'un de leurs jets de récupération (tant qu'il ne s'agit pas de leur jet de récupération de dix heures) sans avoir à dépenser une action, ou laisser passer dix minutes ou une heure. Les cibles récupèrent immédiatement des points dans leurs Réserves, mais marquent cette utilisation de récupération. Les PJ qui ont déjà utilisé leurs jets de récupération d'une action, de dix minutes et d'une heure pour la journée ne bénéficient d'aucun bénéfice de cette capacité. Les PNJ ciblés par cette capacité récupèrent un nombre de points de vie égal à leur niveau. Action. (Healing Pulse \textendash (148))

\subsection*{Pulvérisation}\label{subsec:ab_spray}

Si une arme a la capacité de tirer des coups rapides sans recharger (généralement appelée arme à tir rapide, comme une arbalète à manivelle ou une mitraillette), vous pouvez pulvériser plusieurs tirs autour de votre cible pour augmenter les chances. de frapper. Cette capacité utilise 1d6 + 1 cartouches de munitions (ou toutes les munitions de l'arme, si elle en contient moins que le nombre obtenu). Vous êtes entraîné à réaliser cette attaque. Si l'attaque réussit, elle inflige 1 point de dégâts de moins que la normale. Vous pouvez également utiliser cette capacité sur plusieurs armes de lancer (pierres, shuriken, poignards, etc.) si vous les portez sur vous ou si elles sont toutes à portée de main. Action. (Spray \textendash (185))

\subsection*{Punir le coupable}\label{subsec:ab_punish_the_guilty}

Pendant les dix minutes suivantes, si vous attaquez quelqu'un que vous avez désigné comme coupable avec votre capacité Désignation, vous infligez 2 points de dégâts supplémentaires. Action à initier. (Punish the Guilty \textendash (173))

\subsection*{Punir tous les coupables}\label{subsec:ab_punish_all_the_guilty}

vous pouvez attaquer jusqu'à cinq ennemis à portée immédiate que vous avez désignés comme coupables avec votre capacité de désignation, le tout dans le cadre de la même action en un seul tour. Effectuez des jets d'attaque séparés pour chaque ennemi, mais toutes les attaques comptent comme une seule action au cours d'un seul round. Vous restez limité par la quantité d'Effort que vous pouvez appliquer sur une action. Tout ce qui modifie votre attaque ou vos dégâts s'applique à toutes les attaques. Si vous disposez également de la capacité Attaque Tournoyante, vous infligez 1 point de dégâts supplémentaire lorsque vous utilisez Punir tous les coupables. Action. (Punish All the Guilty \textendash (173))

\
%--------------------------
\section*{Q}

\subsection*{Quelque chose sur la route}\label{subsec:ab_something_in_the_road}

Lorsque vous utilisez un véhicule comme arme, vous infligez 5 points de dégâts supplémentaires. Facilitateur. (Something in the Road \textendash (184))

\
%--------------------------
\section*{R}

\subsection*{Radiance divine}\label{subsec:ab_divine_radiance}

Votre prière appelle le rayonnement divin du ciel pour punir une cible indigne à longue portée, lui infligeant 4 points de dégâts. Si la cible est un démon, un esprit ou quelque chose de similaire, elle est également involontairement impressionnée par l'énergie divine qui la traverse et est incapable d'agir lors de son prochain tour. Une fois exposée à cette bénédiction, la cible ne peut plus être intimidée par cette attaque avant plusieurs heures. Action. (Divine Radiance \textendash (130))

\subsection*{Ralliez-vous à moi}\label{subsec:ab_rally_to_me}

Vous criez, sonnez dans un cor de combat ou signalez à tout le monde à très longue portée que vous avez besoin d'aide. Toutes les créatures alliées qui répondent en se déplaçant à une distance immédiate de vous au cours des prochains rounds gagnent un atout sur n'importe quelle tâche d'attaque ou de défense dans l'heure suivante que vous suggérez, comme « Tenir la porte », « Charger ce groupe de créatures ». orcs », ou quelque chose de similaire. Action à initier. (Rally to Me \textendash (174))

\subsection*{Ramener}\label{subsec:ab_apportation}

Vous amenez un objet physique à vous par vos pouvoirs magiques ou psychiques. Vous pouvez choisir n'importe quelle pièce d'équipement normal sur la liste d'équipement standard, ou (pas plus d'une fois par jour) vous pouvez permettre au MJ de déterminer l'objet au hasard. Si vous amenez un objet aléatoire, il a 10 pour cent de chances d'être un cypher ou un artefact manifeste, 50 pour cent de chances d'être une pièce d'équipement standard et 40 pour cent de chances d'être un objet sans valeur. Vous ne pouvez pas utiliser cette capacité pour prendre un objet détenu par une autre créature. Action. (Apportation \textendash (110))

\subsection*{Ramener amélioré}\label{subsec:ab_improved_apportation}

Vous appelez une créature jusqu'au niveau 3, qui apparaît à côté de vous. Vous pouvez choisir une créature que vous avez déjà rencontrée ou (pas plus d'une fois par jour) vous pouvez permettre au MJ de déterminer la créature au hasard. Si vous appelez une créature aléatoire, elle a 10 pour cent de chances d'être une créature jusqu'au niveau 5. La créature n'a aucun souvenir de quoi que ce soit avant d'être appelée par vous, bien qu'elle puisse parler et ait les connaissances générales d'une créature de ce type. devrait posséder. La créature est réceptive à la communication et vous aide (sauf indication contraire). Action. (Improved Apportation \textendash (151))

\subsection*{Rapide Tromperie}\label{subsec:ab_pull_a_fast_one}

lorsque vous dirigez une escroquerie, faites les poches, trompez ou trompez un dupe, faufilez quelque chose à un garde, etc., vous gagnez un atout pour la tâche. Facilitateur. (Pull a Fast One \textendash (173))

\subsection*{Rapide à fuir}\label{subsec:ab_quick_to_flee}

vous êtes entraîné aux tâches de furtivité et de mouvement. Facilitateur. (Quick to Flee \textendash (174))

\subsection*{Rat des Allées}\label{subsec:ab_alley_rat}

lorsque vous êtes dans une ville, vous trouvez ou créez un raccourci important, une entrée secrète ou une voie d'évacuation de secours là où il semblait qu'il n'en existait pas. Cela nécessite que vous réussissiez une action Intellect dont la difficulté est fixée par le MJ en fonction de la situation. Vous et le MJdevriez régler les détails. Action. (Alley Rat \textendash (109))

\subsection*{Rayer l'Existence}\label{subsec:ab_scratch_existence}

vous pouvez choisir de passer en phase de manière à « rayer » la matière normale sur une longue séquence pendant que vous exécutez en utilisant Sprint de Phase. Cela vous déchire un peu aussi, comme le reflète le coût de la Puissance. Lorsque vous utilisez Sprint de Phase, vous infligez 2 points de dégâts (ignore l'armure) à une cible que vous sélectionnez lorsque vous passez à portée immédiate, sans déclencher Toucher perturbateur. En plus des options normales d'utilisation de l'Effort, vous pouvez choisir d'utiliser l'Effort pour augmenter le nombre de cibles sur votre chemin que vous pouvez attaquer dans le cadre de la même action. Effectuez un jet d'attaque séparé pour chaque ennemi. Vous restez limité par la quantité d'Effort que vous pouvez appliquer sur une action. Tout ce qui modifie votre attaque ou vos dégâts s'applique à toutes ces attaques. Alternativement, si vous appliquez l'Effort pour augmenter les dégâts plutôt que de faciliter la tâche, vous infligez 2 points de dégâts supplémentaires par niveau d'Effort (au lieu de 3 points) ; la cible subit 1 point de dégâts même si vous échouez au jet d'attaque. Facilitateur. (Scratch Existence \textendash (180))

\subsection*{Rayon de confusion}\label{subsec:ab_ray_of_confusion}

Vous projetez un rayon gris de confusion sur une créature à courte portée, lui infligeant 1 point de dégâts qui ignore l'armure. De plus, jusqu'à la fin du tour suivant, toutes les tâches, attaques et défenses tentées par la cible sont entravées. Action. (Ray of Confusion \textendash (174))

\subsection*{Recharger}\label{subsec:ab_reload}

Lorsque vous utilisez une arme qui nécessite normalement une action pour recharger, comme une arbalète lourde, vous pouvez recharger et tirer (ou tirer et recharger) dans la même action. Facilitateur. (Reload \textendash (176))

\subsection*{Recruter un adjoint}\label{subsec:ab_recruit_deputy}

vous gagnez un suivant de niveau 4. Ils ne sont pas limités sur leurs modifications. Alternativement, vous pouvez choisir de faire progresser un adepte de niveau 3 que vous possédez déjà au niveau 4, puis de gagner un nouveau adepte de niveau 3. Facilitateur. (Recruit Deputy \textendash (175))

\subsection*{Recueillir des renseignements}\label{subsec:ab_gather_intelligence}

Lorsque vous êtes dans un groupe de personnes (une caravane, un palais, un village, une ville, etc.), vous pouvez poser des questions sur n'importe quel sujet de votre choix et repartir avec des informations utiles. Vous pouvez poser une question spécifique ou simplement obtenir des faits généraux. Vous aurez également une bonne idée de la disposition générale du lieu concerné, noterez la présence de tous les sites majeurs et remarquerez peut-être même des détails obscurs. Par exemple, non seulement vous découvrez si quelqu'un dans le palais a vu le garçon disparu, mais vous obtenez également une connaissance pratique de l'agencement du palais lui-même, notez toutes les entrées et celles qui sont utilisées plus souvent que d'autres, et prenez remarquez que tout le monde semble éviter le puits dans la cour est pour une raison quelconque. Action à initier, environ une heure à réaliser. (Gather Intelligence \textendash (144))

\subsection*{Regard terrifiant}\label{subsec:ab_terrifying_gaze}

Vous projetez un regard glacial sur toutes les créatures vivantes à courte portée qui peuvent vous voir. Effectuez un jet d'attaque d'Intellect distinct pour chaque cible. Le succès signifie que la créature est figée dans la peur, sans bouger ni entreprendre d'action pendant une minute ou jusqu'à ce qu'elle soit attaquée. Certaines créatures sans esprit (comme les robots) peuvent être immunisées contre le regard terrifiant. Action. (Terrifying Gaze \textendash (190))

\subsection*{Regarde les en Face}\label{subsec:ab_stare_them_down}

On ne joue pas à des jeux de poule mouillée avec d'autres conducteurs maniaques sans acquérir de force mentale. Vous êtes entraîné aux tâches de défense intellectuelle. Facilitateur. (Stare Them Down \textendash (186))

\subsection*{Relocaliser}\label{subsec:ab_relocate}

Choisissez une créature ou un objet à portée immédiate. Vous le transportez instantanément vers une nouvelle position à longue portée que vous pouvez voir. La nouvelle position peut être n'importe quelle direction par rapport à vous, mais elle ne peut pas être à l'intérieur d'un objet solide. Action. (Relocate \textendash (176))

\subsection*{Remodeler}\label{subsec:ab_reshape}

Vous remodelez la matière à courte portée dans une zone ne dépassant pas un cube de 1,5 m (5 pieds). Si vous n'utilisez qu'une seule action sur cette capacité, les modifications que vous apportez sont au mieux grossières. Si vous passez au moins dix minutes et réussissez une tâche de fabrication appropriée et entravée, vous pouvez apporter des modifications complexes au matériau. On ne peut pas changer la nature du matériau, seulement sa forme. Ainsi, vous pouvez faire un trou dans un mur ou un sol, ou vous pouvez en sceller un. Vous pouvez fabriquer une épée rudimentaire à partir d'un gros morceau de fer. Vous pouvez casser ou réparer une chaîne. Avec de multiples utilisations de cette capacité, vous pourriez provoquer de grands changements, en créant un pont, un mur ou une structure similaire. Action. (Reshape \textendash (176))

\subsection*{Renforcer l'illusion}\label{subsec:ab_bolster_illusion}

vous donnez à l'une de vos illusions visuelles une réalité physique limitée que les spectateurs peuvent sentir, goûter, entendre et ressentir. Cet effet est lié à cette illusion et agit de manière appropriée à l'illusion elle-même. Par exemple, cela peut donner l'illusion d'un mur de briques comme de la brique, l'illusion d'une personne sentant le parfum et capable d'ouvrir une porte, et l'illusion d'une cheminée chaude au toucher.La réalité physique fournie à votre illusion est un effet de niveau 1 avec 3 points de vie. Si l'illusion est utilisée pour réaliser des attaques, elle n'inflige qu'1 point de dégâts (qu'il s'agisse de dégâts normaux comme un coup de poing ou de pied illusoire, ou de dégâts ambiants comme la chute d'un mur de briques ou les flammes d'une cheminée). Vous pouvez augmenter le niveau de l'effet créé en appliquant des niveaux d'Effort à cette capacité, chaque niveau d'Effort augmentant le niveau de l'effet de 1. Vous pouvez activer cette capacité dans le cadre de l'action visant à créer une illusion (en utilisant la capacité que vous utilisez pour créer des illusions, comme Illusion mineure), ou utiliser une action distincte pour l'appliquer à l'une de vos illusions existantes. L'effet prend fin si l'illusion est détruite, si vous laissez l'illusion se dissiper, si la santé de l'effet est réduite à 0 ou si dix minutes s'écoulent. Facilitateur. (Bolster Illusion \textendash (CTS, 51))

\subsection*{Repousser le métal}\label{subsec:ab_repel_metal}

en manipulant le magnétisme, vous êtes entraîné aux tâches de défense rapide contre toute attaque entrante utilisant du métal. Facilitateur. (Repel Metal \textendash (176))

\subsection*{Repérer la faiblesse}\label{subsec:ab_spot_weakness}

Si une créature que vous pouvez voir a une faiblesse particulière, comme une vulnérabilité au feu, une modification négative de la perception, etc., vous savez de quoi il s'agit. (Demandez et le directeur général vous le dira.) Enabler. (Spot Weakness \textendash (185))

\subsection*{Ressources profondes}\label{subsec:ab_deep_resources}

vous gagnez 6 points supplémentaires à votre réserve de Célérité. Facilitateur. (Deep Resources \textendash (126))

\subsection*{Restaurer la vie}\label{subsec:ab_restore_life}

Vous pouvez tenter de redonner la vie à une créature morte jusqu'au niveau 3, à condition que le cadavre n'ait pas plus d'un jour et soit en grande partie intact. Vous pouvez également tenter de redonner vie à un cadavre beaucoup plus ancien mais particulièrement bien conservé. La difficulté de la tâche Intellect est égale au niveau de la créature que vous essayez de redonner vie. Pour chaque niveau d'Effort supplémentaire appliqué, vous pouvez tenter de restaurer la vie d'une créature dont le niveau est supérieur de 1. Lorsqu'elle revient à la vie, une créature est étourdie pendant au moins une journée et toutes les tâches qu'elle entreprend sont entravées. Action; une minute pour démarrer. (Restore Life \textendash (177))

\subsection*{Rester en Alerte}\label{subsec:ab_stand_watch}

Lorsque vous êtes debout (en restant généralement en place pendant une période prolongée), vous restez infailliblement éveillé et alerte pendant jusqu'à huit heures. Pendant ce temps, vous êtes entraîné aux tâches de perception ainsi qu'aux tâches furtives pour vous cacher de ceux qui pourraient s'approcher. Action à initier. (Stand Watch \textendash (186))

\subsection*{Retenir sa respiration}\label{subsec:ab_hold_breath}

Vous pouvez retenir votre respiration pendant cinq minutes maximum. Facilitateur. (Hold Breath \textendash (149))

\subsection*{Retour à l'Obélisque}\label{subsec:ab_return_to_the_obelisk}

Vous transférez votre corps et vos biens personnels dans un cristal de n'importe quelle taille que vous pouvez toucher, et vous sortez d'un autre cristal de n'importe quelle taille, y compris tous les obélisques de cristal dont vous avez connaissance. Vous devez connaître le cristal que vous allez utiliser comme sortie avant d'entrer dans le premier cristal. Vous pouvez emporter une créature supplémentaire avec vous pour chaque niveau d'effort appliqué. Action. (Return to the Obelisk \textendash (177))

\subsection*{Retour à l'expéditeur}\label{subsec:ab_return_to_sender}

Si vous réussissez une tâche de défense rapide contre une attaque de mêlée, vous pouvez lancer une attaque de mêlée immédiate contre votre ennemi. Vous ne pouvez utiliser cette capacité qu'une seule fois par tour. Facilitateur. (Return to Sender \textendash (177))

\subsection*{Rhétorique puissante}\label{subsec:ab_powerful_rhetoric}

Après avoir engagé une conversation avec une créature pendant au moins une minute, vous pouvez tenter d'influencer la façon dont cette créature est perçue, en la promouvant comme une amie, en la traitant d'imbécile ou en la dénonçant comme un ennemi. Vos propos sont si bien choisis que même vous et elle en êtes touchés, car votre conviction et son doute sont primordiaux. L'exactitude de votre évaluation n'est pas importante tant que vous maintenez votre rhétorique. À partir de ce moment-là (ou jusqu'à ce que vous changiez de rhétorique ou que la créature offre une défense convaincante à ceux qui ont entendu votre étiquette), les interactions sociales de l'ami acquièrent un atout, les interactions sociales de l'imbécile sont entravées ou les défenses de l'ennemi sont entravées. Action à lancer, une minute à terminer. (Powerful Rhetoric \textendash (171))

\subsection*{Rien que Défendre}\label{subsec:ab_nothing_but_defend}

Si vous ne faites que défendre pendant votre tour, vous êtes spécialisé dans toutes les tâches de défense pendant un tour. Action. (Nothing but Defend \textendash (166))

\subsection*{Robuste}\label{subsec:ab_sturdy}

Vous êtes entraîné aux tâches de défense de Puissance. Facilitateur. (Sturdy \textendash (187))

\subsection*{Robustesse}\label{subsec:ab_hardiness}

Vous êtes entraîné aux tâches de Défense de Puissance. Facilitateur. (Hardiness \textendash (148))

\subsection*{Rubans de matière noire}\label{subsec:ab_ribbons_of_dark_matter}

pendant la minute suivante, la matière noire se condense dans une zone à longue portée dont le diamètre ne dépasse pas une distance immédiate, se manifestant sous forme de rubans tourbillonnants. Toutes les tâches tentées par les créatures dans la zone sont gênées, et quitter la zone nécessite toute l'action d'une créature pour se déplacer. Vous pouvez rejeter la matière noire dès le début par une action. Action à initier. (Ribbons of Dark Matter \textendash (178))

\subsection*{Réaction}\label{subsec:ab_reaction}

Si une créature que vous avez attaquée lors de votre dernier tour avec une attaque au corps à corps utilise son action pour se déplacer hors de portée immédiate, vous gagnez une action pour attaquer la créature comme un coup final, même si vous avez déjà joué un tour dans le tour. Facilitateur. (Reaction \textendash (174))

\subsection*{Réanimer}\label{subsec:ab_resuscitate}

vous pouvez réanimer un personnage qui se trouve jusqu'à deux niveaux plus bas sur la piste des dégâts lors de votre action. La cible monte d'un cran sur la piste des dégâts. Si un personnage a laissé tomber les trois marches de la piste de dégâts (mort) mais est par ailleurs en un seul morceau et moins d'une minute s'est écoulée depuis qu'il est descendu à la troisième marche, vous pouvez le ressusciter si vous réussissez une tâche de guérison de niveau 6. Si vous utilisez cette capacité sur un PNJ qui n'a aucune santé mais qui est mort depuis moins d'une minute et qui est par ailleurs en un seul morceau, le PNJ est ressuscité avec 1 santé. Action. (Resuscitate \textendash (177))

\subsection*{Récupération Rapide d'un autre}\label{subsec:ab_speedy_recovery}

Vos paroles améliorent la capacité de régénération normale d'un personnage à courte portée qui est capable de vous comprendre. Lorsqu'ils effectuent un jet de récupération, ils ne doivent y consacrer que la moitié du temps normal (minimum une action). Action. (Speedy Recovery \textendash (185))

\subsection*{Récupération améliorée}\label{subsec:ab_improved_recovery}

votre jet de récupération de dix minutes ne nécessite qu'une seule action, de sorte que vos deux premiers jets de récupération représentent une action, le troisième dure une heure et le quatrième dure dix heures. Facilitateur. (Improved Recovery \textendash (152))

\subsection*{Récupération du patient}\label{subsec:ab_patient_recovery}

vous gagnez un jet de récupération supplémentaire de dix minutes chaque jour. Facilitateur. (Patient Recovery \textendash (168))

\subsection*{Récupération et confort}\label{subsec:ab_salvage_and_comfort}

Vous êtes familier avec les espaces ouverts. Si vous passez une heure à utiliser les capteurs de votre vaisseau spatial et que vous effectuez un jet d'Intellect de difficulté 3, vous pouvez trouver des objets de récupération sous la forme d'un vaisseau spatial abandonné, de particules de matière à la dérive qui étaient autrefois habitées, ou d'un endroit où vous cacher des poursuites dans ce que la plupart des gens auraient autrement fait. supposons qu'il s'agit d'un espace vide (comme dans une nébuleuse, un champ d'astéroïdes ou l'ombre d'une lune). La récupération que vous obtenez comprend suffisamment de nourriture et d'eau pour vous et plusieurs autres, ainsi que la possibilité d'armes, de vêtements, d'artefacts technologiques, de survivants ou d'autres objets utilisables. Dans d'autres contextes, cette capacité compte comme un entraînement à des tâches liées à la perception. Action à initier, une heure à réaliser. (Salvage and Comfort \textendash (179))

\subsection*{Récupération rapide}\label{subsec:ab_rapid_recovery}

votre deuxième jet de récupération (nécessitant généralement dix minutes) ne consiste qu'en une seule action. Facilitateur. (Rapid Recovery \textendash (174))

\subsection*{Récupération supplémentaire}\label{subsec:ab_extra_recovery}

vous gagnez une récupération suplémentaire en une seule action chaque jour. Facilitateur. (Extra Recovery \textendash (138))

\subsection*{Récupérer}\label{subsec:ab_fetch}

Vous faites disparaître et réapparaître un objet dans vos mains ou ailleurs à proximité. Choisissez un objet qui peut tenir dans un cube de 2 m et que vous pouvez voir à longue portée. L'objet disparaît et apparaît entre vos mains ou dans un espace ouvert n'importe où à portée immédiate de votre choix. Action. (Fetch \textendash (139))

\subsection*{Récupérer des souvenirs}\label{subsec:ab_retrieve_memories}

Vous touchez les restes d'une créature récemment tuée et effectuez un jet basé sur l'Intellect pour redonner vie à son esprit suffisamment longtemps pour en tirer des informations. Le MJ définit la difficulté en fonction du temps écoulé depuis la mort de la créature. Une créature morte depuis seulement quelques minutes est une tâche de difficulté 2, celle qui est morte depuis une heure est une tâche de difficulté 4 et celle qui est morte depuis quelques jours est une tâche de difficulté 9. Si vous réussissez, vous réveillez le cadavre, ce qui fait que sa tête s'anime et perçoit les choses comme si elle était vivante. Cela permet de communiquer pendant environ une minute, soit le temps qu'il faut à la créature pour se rendre compte qu'elle est morte. La créature est limitée à ce qu'elle a connu dans la vie, même si elle ne peut pas se souvenir de souvenirs mineurs, mais uniquement de grands événements importants pour elle. Lorsque l'effet prend fin, ou si vous échouez au jet, le cerveau de la créature se dissout en bouillie et ne peut plus être réveillé. Action. (Retrieve Memories \textendash (177))

\subsection*{Régénération}\label{subsec:ab_rejeneration}

Vous restaurez des points dans la réserve de Puissance ou de Célérité d'une cible de deux manières : soit la réserve choisie récupère jusqu'à 6 points, soit elle est restaurée à une valeur totale de 12. Vous prenez cette décision lorsque vous initier cette capacité. Les points sont régénérés à raison de 1 point à chaque tour. Vous devez rester à portée immédiate de la cible tout le temps, soit en la touchant, soit en conversant avec elle. En aucun cas cela ne peut faire monter un Pool au-dessus de son maximum. Action. (Rejeneration \textendash (175))

\subsection*{Régénérer}\label{subsec:ab_regenerate}

votre capacité à guérir (que ce soit à partir d'un sort puissant, d'une mutation unique ou d'une greffe cybernétique) continue de fonctionner même si vous mourez de violence, tant que votre corps est en grande partie intact. Une minute après votre mort, cette capacité s'active et vous ramène à la vie ; cependant, vous revenez avec une déduction permanente de 2 points de votre réserve d'intelligence. Facilitateur. (Un personnage pourrait découvrir que Régénérer est à la fois une bénédiction et une malédiction, car trop s'y fier conduit à une sorte de malaise que la vitalité seule ne peut pas résoudre.) (Regenerate \textendash (175))

\subsection*{Régénérer un autre}\label{subsec:ab_regenerate_other}

Vous pouvez conférer votre capacité de régénération à une autre créature que vous touchez et tenter de lui redonner vie, à condition que son corps soit en grande partie intact. (Si vous ne possédez pas la capacité Régénération, vous la gagnez, mais ne pouvez l'utiliser que sur vous-même.) La difficulté de la tâche est égale à 3 plus le nombre de jours pendant lesquels la cible est morte. (Si le corps a été parfaitement préservé en stase ou grâce à un autre mécanisme de préservation non dommageable, aucune limite de temps ne s'applique.) Facilitateur. (Regenerate Other \textendash (175))

\subsection*{Réparer la chair}\label{subsec:ab_repair_flesh}

Lorsque vous touchez un personnage affaibli ou affaibli, vous pouvez le faire monter d'un cran sur la piste des dégâts (par exemple, un PC affaibli devient affaibli et un autre affaibli devient sain). Alternativement, si vous utilisez cette capacité sur un PJ pendant un repos, vous lui accordez un bonus de +2 à son jet de récupération. Action. (Repair Flesh \textendash (176))

\subsection*{Répertoire magique}\label{subsec:ab_magical_repertoire}

Le nombre de chiffres subtils que vous pouvez supporter en même temps augmente de deux. Si vous passez une heure à préparer votre magie, vous pouvez remplir n'importe lequel de vos emplacements de cyphers ouverts avec des cyphers subtils choisis au hasard par le MJ (cette heure peut faire partie d'une action de récupération d'une heure ou de dix heures si vous êtes éveillé pendant toute la durée de la magie). temps). Dans le cadre de ce processus de préparation, vous pouvez supprimer n'importe quel nombre de cyphers subtils que vous portez pour faire place à des chiffres plus subtils. Facilitateur. Si un personnage possède un répertoire magique, le MJ doit donner au PJ des opportunités fréquentes d'acquérir de nouveaux chiffres subtils, que ce soit par préparation ou en les obtenant automatiquement comme expliqué dans le chapitre Cyphers. (Magical Repertoire \textendash (159))

\subsection*{Répertoire étendu}\label{subsec:ab_expanded_repertoire}

le nombre de cyphers subtils que vous pouvez porter en même temps augmente de un. Facilitateur. (Expanded Repertoire \textendash (136))

\subsection*{Réputation de hors-la-loi}\label{subsec:ab_outlaw_reputation}

Les gens connaissent vos exploits notoires, qui ont été racontés et répétés tellement de fois qu'ils ne ressemblent guère à la réalité. Mais les gens craignent votre nom lorsqu'ils vous reconnaissent (ou que vous vous déclarez). Ils deviennent si effrayés que toutes les attaques lancées contre vous par les cibles affectées à portée de voix sont entravées jusqu'à ce qu'une ou plusieurs d'entre elles réussissent à infliger des dégâts à vous ou à l'un de vos alliés, moment auquel leur peur diminue. Facilitateur. (Outlaw Reputation \textendash (168))

\subsection*{Réputation redoutable}\label{subsec:ab_fearsome_reputation}

Vous et ceux avec qui vous voyagez avez acquis une réputation redoutable dans certaines régions. Si vos ennemis ont entendu parler de vous, les cibles affectées à portée de voix deviennent effrayées et toutes les attaques qu'elles lancent contre vous sont entravées jusqu'à ce qu'une ou plusieurs d'entre elles réussissent à infliger des dégâts à vous ou à l'un de vos alliés, moment auquel leur peur diminue. Action. (Fearsome Reputation \textendash (139))

\subsection*{Réseau de capteurs}\label{subsec:ab_sensor_array}

Vous êtes entraîné à l'utilisation des instruments sensoriels d'un vaisseau spatial. Ces instruments permettent aux utilisateurs de répondre à des questions générales sur un lieu, telles que « Combien de personnes y a-t-il dans la colonie minière ? » ou "Où l'autre vaisseau spatial s'est-il écrasé ?" Action. (Sensor Array \textendash (181))

\subsection*{Réseau télépathique}\label{subsec:ab_telepathic_network}

Lorsque vous le souhaitez, vous pouvez contacter jusqu'à dix créatures que vous connaissez, peu importe où elles se trouvent. Toutes les cibles doivent être disposées et capables de communiquer. Vous réussissez automatiquement à établir un réseau télépathique ; aucun rouleau n'est requis. Toutes les créatures du réseau sont liées et peuvent communiquer par télépathie entre elles. Ils peuvent également « entendre » tout ce qui se dit sur le réseau, s'ils le souhaitent. L'activation de cette capacité ne nécessite aucune action et ne coûte pas de points d'Intellect ; pour vous, c'est aussi simple que de parler à voix haute. Le réseau dure jusqu'à ce que vous choisissiez d'y mettre fin. Si vous dépensez 5 points d'Intellect, vous pouvez contacter vingt créatures à la fois, et pour chaque point d'Intellect dépensé au-dessus de ce chiffre, vous pouvez ajouter dix créatures supplémentaires au réseau. Ces réseaux plus larges durent dix minutes. Créer un réseau de vingt créatures ou plus nécessite une action pour établir le contact. Facilitateur. (Telepathic Network \textendash (190))

\subsection*{Réserves Partagées}\label{subsec:ab_deep_reserves}

lorsque les autres sont épuisés, vous pouvez passer au travers. Une fois par jour, vous pouvez transférer jusqu'à 5 points entre vos Réserves dans n'importe quelle combinaison, à raison de 1 point par tour. Par exemple, vous pourriez transférer 3 points de Puissance en Célérité et 2 points d'Intellect en Célérité, ce qui prendrait un total de cinq tours. Action. (Deep Reserves \textendash (126))

\subsection*{Réserves cachées}\label{subsec:ab_hidden_reserves}

lorsque vous utilisez une action pour effectuer un jet de récupération, vous gagnez également +1 à votre Avantage de Puissance et à votre Avantage de Célérité pendant dix minutes par la suite. Facilitateur. (Hidden Reserves \textendash (149))

\subsection*{Résilience}\label{subsec:ab_resilience}

Vous disposez de 1 point d'Armure contre tout type de dégâts physiques, même les dégâts physiques qui ignorent normalement l'Armure. Facilitateur. (Resilience \textendash (176))

\subsection*{Résilience durement gagnée}\label{subsec:ab_hard_won_resilience}

Au cours de vos explorations de lieux sombres, vous avez été exposé à toutes sortes de choses terribles et développez une résistance générale. Vous gagnez +1 en Armure et êtes entraîné aux tâches de Défense de Puissance. Facilitateur. (Hard-Won Resilience \textendash (148))

\subsection*{Résistance au poison}\label{subsec:ab_poison_resistance}

Grâce à une injection d'agents biologiques, une gorgée d'élixir magique, un anneau d'un extraterrestre mourant ou quelque chose d'aussi extrême, vous êtes désormais immunisé contre les poisons, les toxines ou toute sorte de menace particulaire. Vous n'êtes pas à l'abri des virus, des bactéries ou des radiations. Facilitateur. (Poison Resistance \textendash (170))

\subsection*{Résistance énergétique}\label{subsec:ab_energy_resistance}

choisissez un type d'énergie discret avec lequel vous avez de l'expérience (comme la chaleur, le son, l'électricité, etc.). Vous gagnez +5 en armure contre les dégâts de ce type d'énergie. Vous devez être familier avec le type d'énergie ; par exemple, si vous n'avez aucune expérience avec un certain type d'énergie extradimensionnelle, vous ne pouvez pas vous en protéger. Vous pouvez sélectionner cette capacité plusieurs fois. Chaque fois que vous la sélectionnez, vous devez choisir un type d'énergie différent. Facilitateur. (Energy Resistance \textendash (134))

\subsection*{Résister aux dangers sous-marins}\label{subsec:ab_resist_underwater_hazards}

Que vous résistiez aux eaux écrasées tout en explorant les profondeurs ou à la piqûre d'un poisson venimeux, toutes les tâches de défense lorsque vous êtes immergé dans l'eau sont facilitées. Facilitateur. (Resist Underwater Hazards \textendash (176))

\subsection*{Résistez aux éléments}\label{subsec:ab_resist_the_elements}

Vous résistez à la chaleur, au froid et aux extrêmes similaires. Vous disposez d'un +2 spécial à l'armure contre les dégâts ambiants ou d'autres dégâts qui ignoreraient normalement l'armure. Facilitateur. (Resist the Elements \textendash (176))

\subsection*{Résoudre des Enigmes}\label{subsec:ab_resist_tricks}

vous êtes entraîne à résoudre des énigmes et à reconnaître les astuces au fil des années de jeu. Facilitateur. (Resist Tricks \textendash (176))

\subsection*{Rétrécir}\label{subsec:ab_shrink}

vous (et vos vêtements ou votre costume) devenez beaucoup plus petit que votre taille normale. Vous mesurez 15 cm (6 pouces) et restez ainsi pendant environ une minute. Pendant ce temps, vous ajoutez 4 points à votre Speed Pool et ajoutez +2 à votre Speed Edge. Tant que vous êtes plus petit que la normale, vos jets de défense de Célérité sont allégés, votre Célérité de déplacement est d'un dixième de la normale et vos attaques infligent la moitié du montant normal de dégâts (divisez le total des dégâts par moitié après tous les bonus, efforts et autres modificateurs de dégâts. ). Vous pouvez reprendre votre taille normale dans le cadre d'une autre action. Lorsque les effets de Rétrécir prennent fin, votre Speed Edge, votre Célérité de déplacement et vos dégâts reviennent à la normale, et vous soustrayez un nombre de points de votre Speed Pool égal au nombre que vous avez gagné (si cela ramène la Pool à 0, soustrayez d'abord le débordement. depuis votre Réserve de Puissance puis, si nécessaire, depuis votre Réserve d'Intellect). Chaque fois que vous utilisez Rétrécir avant votre prochain jet de récupération de dix heures, vous devez appliquer un niveau d'Effort supplémentaire (un niveau d'Effort pour la deuxième utilisation, deux niveaux d'Effort pour la troisième utilisation, et ainsi de suite).Action à initier . L'augmentation du coût d'effort pour les utilisations répétées de Rétrécir entre des jets de récupération de dix heures s'applique uniquement aux nouvelles activations de Rétrécir, et non aux changements de taille multiples au cours d'une seule utilisation de Rétrécir activée par Changements Rapides. (Shrink \textendash (183))

\subsection*{Rétrécir les autres}\label{subsec:ab_shrink_others}

vous pouvez utiliser Rétrécir sur d'autres créatures consentantes à une distance immédiate. En plus des options normales d'utilisation de l'Effort, vous pouvez choisir d'utiliser l'Effort pour affecter plus de cibles ; chaque niveau d'Effort affecte une cible supplémentaire. À moins que ces créatures n'aient la capacité de changer de taille, elles restent petites jusqu'à ce que la durée d'une minute de Rétrécissement se termine pour elles. Facilitateur. (Shrink Others \textendash (183))

\subsection*{Révèle}\label{subsec:ab_reveal}

vous ajustez la vue d'une créature afin qu'elle puisse voir normalement dans les zones de faible lumière et d'obscurité. Vous pouvez affecter une créature volontaire à portée immédiate pendant une heure. En plus des options normales d'utilisation de l'Effort, vous pouvez choisir d'utiliser l'Effort pour affecter plus de cibles ; chaque niveau d'effort appliqué affecte deux cibles supplémentaires. Vous devez toucher des cibles supplémentaires pour les affecter. Action à initier. (Reveal \textendash (178))

\subsection*{Rêverie}\label{subsec:ab_daydream}

Vous entraînez quelqu'un dans une rêverie, substituant un rêve de votre propre création à la réalité de la cible pendant une minute maximum. Vous pouvez affecter une cible à longue portée que vous pouvez voir, ou une cible à moins de 16 km (10 miles) sur laquelle vous avez des coupures de cheveux ou de peau. Selon toutes les apparences extérieures, une cible affectée reste (ou ment) immobile. Mais à l'intérieur, la réalité substituée (ou le rêve dans un rêve, si la cible dormait) est ce que la cible expérimente. Si la cible est sous contrainte, elle peut tenter un autre jet de défense Intellect à chaque round pour se libérer, même si la cible peut ne pas se rendre compte de son état. Soit le rêve se déroule selon un scénario que vous avez préparé lorsque vous avez utilisé cette capacité, soit si vous utilisez vos propres actions (vous forçant dans un état similaire à celui de la cible), vous pouvez diriger le rêve qui se déroule d'un tour à l'autre. Utiliser cette capacité sur une cible endormie facilite l'attaque initiale. Action à initier ; si vous dirigez le rêve, action à diriger par tour. (Daydream \textendash (124))

\
%--------------------------
\section*{S}

\subsection*{S'appuyer sur les expériences de la vie}\label{subsec:ab_drawing_on_life's_experiences}

Vous avez vu et fait beaucoup de choses, et cette expérience s'avère utile. Posez une question au directeur général et vous recevrez une réponse générale. Le MJ attribue un niveau à la question, donc plus la réponse est obscure, plus la tâche est difficile. Généralement, les connaissances que vous pourriez trouver en cherchant ailleurs que votre emplacement actuel sont de niveau 1, et les connaissances obscures du passé sont de niveau 7. Action. (Drawing on Life's Experiences \textendash (131))

\subsection*{Sachez où chercher}\label{subsec:ab_know_where_to_look}

chaque fois que le MJ obtient un résultat pour vous sur le tableau des éléments utiles, vous obtenez deux résultats au lieu d'un. Si le MJ utilise une autre méthode pour générer des récompenses en trouvant des objets de valeur, vous devriez obtenir le double du résultat que vous obtiendriez autrement. Facilitateur. (Know Where to Look \textendash (156))

\subsection*{Sacrifice volontaire}\label{subsec:ab_willing_sacrifice}

lorsque vous effectuez une attaque destinée à un autre personnage, vous savez comment prendre l'attaque de manière à minimiser son effet. L'attaque vous frappe automatiquement, mais au lieu de subir 1 point de dégâts supplémentaire, vous subissez 1 point de dégâts en moins (jusqu'à un minimum de 1 point). De plus, vous pouvez effectuer plus d'une attaque au cours d'un round donné à condition que toutes les attaques soient initialement destinées à une seule cible. Facilitateur. (Deux personnages tentant de lancer une attaque en même temps s'annulent.) (Willing Sacrifice \textendash (199))

\subsection*{Saisir}\label{subsec:ab_grab}

pendant que vous utilisez la capacité Agrandir, vous pouvez attaquer en essayant d'enrouler vos mains massives autour d'une cible de la taille d'un humain normal ou plus petite. Pendant que vous maintenez votre emprise lors de votre action, vous empêchez la cible de bouger ou d'entreprendre des actions physiques (autres que les tentatives de fuite). La tentative de fuite de la cible est gênée par deux étapes en raison de votre taille. Si vous le souhaitez, vous pouvez automatiquement infliger 3 points de dégâts à chaque round à la cible tant que vous la tenez, mais vous pouvez également la protéger (en subissant toutes les attaques autrement destinées à la cible). Action. (Grab \textendash (146))

\subsection*{Saisissez l'instant}\label{subsec:ab_seize_the_moment}

Si vous réussissez un jet de défense de Célérité pour résister à une attaque, vous gagnez une action. Vous pouvez utiliser l'action immédiatement même si vous avez déjà joué un tour dans le tour. Vous n'effectuez aucune action lors du tour suivant, sauf si vous appliquez un niveau d'Effort lorsque vous utilisez Saisissez l'instant. Facilitateur. (Seize the Moment \textendash (181))

\subsection*{Sans âge}\label{subsec:ab_ageless}

votre corps et votre esprit ne vieillissent pas. À moins que vous ne soyez tué par la violence (ou par une force extérieure telle qu'un poison ou une infection), vous ne mourrez jamais. Facilitateur. (Ageless \textendash (109))

\subsection*{Sans être remarqué}\label{subsec:ab_beneath_notice}

Votre taille réduite rend difficile votre recherche. Pendant que Rétrécir est actif sur vous, toutes les tâches furtives que vous tentez sont facilitées. Facilitateur. (Beneath Notice \textendash (CTS, 49))

\subsection*{Santé incroyable}\label{subsec:ab_incredible_health}

grâce à un plongeon dans une piscine magique, à une injection d'anticorps artificiels et de nanobots de défense immunitaire dans votre sang, à une exposition à des radiations étranges ou à autre chose, vous êtes désormais immunisé contre les maladies, les virus et les mutations de toutes sortes. Facilitateur. (Incredible Health \textendash (153))

\subsection*{Santé miraculeuse}\label{subsec:ab_miraculous_health}

lorsque vous descendez une marche sur la piste des dégâts, vous pouvez tenter une tâche de Puissance pour résister, avec une difficulté égale au niveau de l'ennemi ou de l'effet qui vous a blessé. En cas de réussite, vous ne descendez pas la marche et vous regagnez 1 point dans n'importe quelle Pool dépourvue de points. Vous ne pouvez plus utiliser cette capacité avant votre prochain repos de dix heures. Facilitateur. (Miraculous Health \textendash (163))

\subsection*{Saut de Côté}\label{subsec:ab_spring_away}

Chaque fois que vous réussissez un jet de défense de Célérité, vous pouvez immédiatement vous déplacer sur une courte distance. Vous ne pouvez pas utiliser cette capacité plus d'une fois au cours d'un round donné. Facilitateur. (Spring Away \textendash (185=6))

\subsection*{Saut incroyable}\label{subsec:ab_amazing_leap}

Vous sautez dans les airs et atterrissez en toute sécurité à une certaine distance. Vous pouvez sauter vers le haut, vers le bas ou vers n'importe quel endroit de votre choix à longue portée si vous disposez d'un chemin clair et dégagé vers cet endroit. Si vous disposez de trois changements de Puissance ou plus, votre portée de saut augmente jusqu'à devenir très longue. Si vous avez cinq changements de Puissance ou plus, votre portée de saut augmente à 1 000 pieds (300 m). Action. (Amazing Leap \textendash (CTS, 48))

\subsection*{Sauter en Microgravité}\label{subsec:ab_push_off_and_throw}

vous pouvez effectuer des sauts précis point à point en microgravité, ce qui signifie que vous pouvez vous déplacer sur une longue distance et lancer une attaque au corps à corps ou tenter d'attraper un ennemi de votre taille ou plus petit. . Si vous réussissez à attraper votre ennemi, vous déplacez votre ennemi sur une courte distance de sa position d'origine. Alternativement, pendant que vous vous arrêtez (ou que vous vous éloignez d'une distance immédiate par round dans la direction de votre choix), vous pouvez lancer votre ennemi dans une direction choisie à travers l'espace à raison d'une courte distance par round. Action. (Push Off and Throw \textendash (173))

\subsection*{Sauts de Téléportation}\label{subsec:ab_teleportation_burst}

vous vous téléportez rapidement plusieurs fois dans une zone immédiate, déroutant vos adversaires et vous permettant de lancer une attaque de mêlée supplémentaire ce tour-ci. Vous pouvez utiliser cette capacité une fois par round. Facilitateur. (Teleportation Burst \textendash (190))

\subsection*{Scan}\label{subsec:ab_scan}

Vous scannez une zone de taille égale à un cube de 10 pieds (3 m), y compris tous les objets ou créatures dans cette zone. La zone doit être à courte portée. Scanner une créature ou un objet révèle toujours son niveau. Vous apprenez également tous les faits que le MJ estime pertinents sur la question et l'énergie dans ce domaine. Par exemple, vous apprendrez peut-être que la boîte en bois contient un dispositif en métal et en plastique. Vous apprendrez peut-être que le cylindre en verre est rempli de gaz toxique et que son support métallique est traversé par un champ électrique qui se connecte à un treillis métallique dans le sol. Vous apprendrez peut-être que la créature qui se tient devant vous est un mammifère doté d'un petit cerveau. Cependant, cette capacité ne vous dit pas ce que signifient les informations. Ainsi, dans le premier exemple, vous ne savez pas à quoi sert le dispositif en métal et en plastique. Dans le second cas, on ne sait pas si le fait de marcher sur le sol provoque la libération du gaz par la bouteille. Dans le troisième cas, vous pourriez soupçonner que la créature n'est pas très intelligente, mais les analyses, tout comme les apparences, peuvent être trompeuses. De nombreux matériaux et champs d'énergie empêchent ou résistent au balayage. Action. (Scan \textendash (179))

\subsection*{Science du Sommeil}\label{subsec:ab_oneirochemy}

Vous êtes entraîné aux tâches liées au sommeil et au mélange d'élixirs naturels pour aider les créatures à s'endormir. Facilitateur. (Oneirochemy \textendash (167))

\subsection*{Sculpter la chair}\label{subsec:ab_sculpt_flesh}

Vous faites en sorte que les doigts d'une créature volontaire s'allongent en griffes et que ses dents se transforment en crocs. L'effet dure dix minutes. Les dégâts infligés par les frappes à mains nues de la cible augmentent à 4 points. Action. (Sculpt Flesh \textendash (180))

\subsection*{Sculpter la lumière}\label{subsec:ab_sculpt_light}

Vous créez un objet de lumière solide dans n'importe quelle forme que vous pouvez imaginer, de votre taille ou plus petite, et il persiste pendant environ une heure. L'objet apparaît dans une zone adjacente à vous. Il est brut et ne peut comporter aucune pièce mobile, vous pouvez donc fabriquer une épée, un bouclier, une échelle courte, etc. L'objet a la masse approximative de l'objet réel et est de niveau 4. Action. (Sculpt Light \textendash (180))

\subsection*{Se Fendre}\label{subsec:ab_lunge}

Cette capacité vous oblige à vous fendre en avant pour un coup d'estoc ou de taille. Se fendre atténue le jet d'attaque. Si votre attaque réussit, elle inflige 4 points de dégâts supplémentaires. Action. (Lunge \textendash (159))

\subsection*{Se Glisser dans l'ombre}\label{subsec:ab_slip_into_shadow}

Vous tentez de vous éloigner d'une cible sélectionnée et de vous cacher dans une ombre proche, derrière un arbre ou un meuble, ou dans la pièce voisine, même si vous êtes bien en vue de la cible. Pour chaque niveau d'Effort appliqué, vous pouvez tenter d'affecter une cible supplémentaire, à condition que toutes vos cibles soient côte à côte. Action à initier. (Slip Into Shadow \textendash (183))

\subsection*{Se déplacer comme l'eau}\label{subsec:ab_moving_like_water}

vous tournez sur cous-même et vous vous déplacez de manière à ce que votre défense et vos attaques soient facilitées par votre mouvement fluide. Pendant une minute, toutes vos attaques et tâches de défense rapide gagnent un atout. Facilitateur. (Moving Like Water \textendash (164))

\subsection*{Se fondre dans le décor}\label{subsec:ab_blend_in}

Lorsque vous vous fondez avec le décor, les créatures vous voient toujours, mais elles n'attachent aucune importance à votre présence pendant environ une minute. Tout en vous intégrant, vous êtes spécialisé dans les tâches de furtivité et de défense rapide. Cet effet prend fin si vous faites quelque chose pour révéler votre présence ou votre position : attaquer, utiliser une capacité, déplacer un objet volumineux, etc. Si cela se produit, vous pouvez retrouver la période d'effet restante en prenant des mesures pour vous concentrer sur l'apparence inoffensive et comme si vous en faisiez partie. Action à initier ou à relancer. (Blend In \textendash (113))

\subsection*{Sens aiguisés}\label{subsec:ab_sharp_senses}

Vous êtes entraîné à toutes les tâches impliquant la perception. Facilitateur. (Sharp Senses \textendash (182))

\subsection*{Sens du Danger}\label{subsec:ab_danger_sense}

Votre tâche d'initiative est facilitée. Vous payez le coût à chaque fois que la capacité est utilisée. Facilitateur. (Danger Sense \textendash (124))

\subsection*{Sens et sensibilités animales}\label{subsec:ab_animal_senses_and_sensibilities}

Vous êtes entraîné à écouter et à repérer les objets. De plus, la plupart du temps, le MJ devrait vous alerter si vous êtes sur le point de tomber dans une embuscade ou dans un piège inférieur au niveau 5. Enabler. (Animal Senses and Sensibilities \textendash (109))

\subsection*{Sens surnaturels}\label{subsec:ab_preternatural_senses}

Tant que vous êtes conscient et capable d'utiliser une action, vous ne pouvez pas être surpris. De plus, vous êtes entraîné aux actions d'initiative. Facilitateur. (Preternatural Senses \textendash (171))

\subsection*{Sensibilisation à la nature sauvage}\label{subsec:ab_wilderness_awareness}

Votre connexion avec le monde naturel s'étend à un degré que certains qualifieraient de surnaturel. Dans la nature, vous pouvez étendre vos sens jusqu'à un kilomètre et demi dans n'importe quelle direction et poser au MJ une question générale très simple sur cette zone, comme « Où est le camp des orcs ? ou "Mon ami Deithan est-il toujours en vie?" Si la réponse que vous cherchez ne se trouve pas dans la zone, vous ne recevez aucune information. Action. (Wilderness Awareness \textendash (198))

\subsection*{Sentir les Attitudes}\label{subsec:ab_sense_attitudes}

Vous êtes entraîné à détecter les mensonges et à savoir si une personne est susceptible de croire (ou croit déjà) vos mensonges. Facilitateur. (Sense Attitudes \textendash (181))

\subsection*{Sentir une Embuscade}\label{subsec:ab_sense_ambush}

Vous n'êtes jamais surpris par une attaque. Facilitateur. (Sense Ambush \textendash (181))

\subsection*{Serv-0}\label{subsec:ab_serv_0}

Vous construisez un petit assistant robot. Il est de niveau 1 et ne peut pas entreprendre d'actions indépendantes ni quitter votre zone immédiate. En vérité, c'est plus une extension de vous qu'un être séparé. Il gagne une modification dans l'utilisation des machines et autres dispositifs technologiques. Facilitateur. (Serv-0 \textendash (181))

\subsection*{Serv-0 Bagarreur}\label{subsec:ab_serv_0_brawler}

Votre Serv-0 vous aide dans les combats au corps à corps. Il gagne une modification dans les attaques de mêlée. Facilitateur. (Serv-0 Brawler \textendash (181))

\subsection*{Serv-0 Defenseur}\label{subsec:ab_serv_0_defender}

Votre Serv-0 vous aide au combat en bloquant les attaques. Il gagne une modification en Défense Célérité. Facilitateur. (Serv-0 Defender \textendash (181))

\subsection*{Serv-0 Espion}\label{subsec:ab_serv_0_spy}

Vous pouvez envoyer votre Serv-0 sur une longue distance pendant dix minutes maximum et voir et entendre à travers lui comme si ses sens étaient les vôtres. Vous dirigez son mouvement. Action à initier. (Serv-0 Spy \textendash (181))

\subsection*{Serv-0 Réparateur}\label{subsec:ab_serv_0_repair}

Votre Serv-0 vous aide à réparer d'autres appareils. Il gagne une modification en réparation. Facilitateur. (Serv-0 Repair \textendash (181))

\subsection*{Serv-0 Scanner}\label{subsec:ab_serv_0_scanner}

Votre Serv-0 acquiert la capacité Scan. Facilitateur. (Serv-0 Scanner \textendash (181))

\subsection*{Serv-0 Viseur}\label{subsec:ab_serv_0_aim}

Votre Serv-0 vous aide dans les combats à distance. Il gagne une modification dans les attaques à distance. Facilitateur. (Serv-0 Aim \textendash (181))

\subsection*{Serviteur du Feu}\label{subsec:ab_fire_servant}

Pendant que votre Manteau de flammes est actif, vous piochez votre halo et produisez un automate de feu qui correspond à votre forme et à votre taille générales. Il agit comme vous le dirigez chaque round. Diriger le serviteur est une action, et vous ne pouvez le commander que lorsque vous en êtes à longue portée. Sans ordre, le serviteur continue de suivre votre ordre précédent. Vous pouvez également lui confier une action programmée simple, telle que « Attendez ici et attaquez quiconque s'approche à courte portée jusqu'à ce qu'il soit mort. » Le serviteur dure dix minutes, est une créature de niveau 5 et inflige 1 point de dégâts de feu supplémentaire lorsqu'il attaque. Action de créer ; action à diriger. (Fire Servant \textendash (140))

\subsection*{Silencieux comme l'espace}\label{subsec:ab_silent_as_space}

en profitant des conditions de microgravité, vous obtenez un atout pour les tâches de furtivité et d'initiative en apesanteur ou en apesanteur. Facilitateur. (Silent as Space \textendash (183))

\subsection*{Simple et Direct}\label{subsec:ab_straightforward}

Vous êtes entraîné à l'une des tâches suivantes (au choix) : casser des objets, grimper, sauter ou courir. Facilitateur. (Straightforward \textendash (187))

\subsection*{Siphon solaire}\label{subsec:ab_sun_siphon}

la limite de sécurité de votre réserve de siphon issue de la capacité Stocker l'énergie augmente de 3 points. Si vous passez une heure au soleil (ou une heure en contact avec une source d'énergie puissante appropriée), vous remplissez automatiquement votre Siphon Pool jusqu'à sa limite de sécurité. Vous ne pouvez pas remplir à nouveau votre siphon de cette façon avant votre prochain jet de récupération de dix heures. Facilitateur. (Sun Siphon \textendash (188))

\subsection*{Soif de sang}\label{subsec:ab_bloodlust}

Si vous éliminez un ennemi, vous pouvez vous déplacer sur une courte distance, mais seulement si vous vous dirigez vers un autre ennemi. Vous n'avez pas besoin de dépenser des points tant que vous ne savez pas que le premier ennemi est à terre. Facilitateur. (Bloodlust \textendash (115))

\subsection*{Souhait mineur}\label{subsec:ab_minor_wish}

à votre demande, l'allié magique de votre capacité Créature magique liée peut utiliser son action pour vous lancer un sort mineur. Ensuite, il doit se retirer vers son objet lié pour se reposer pendant une heure. Les effets qu'il peut produire sont les suivants. Action à initier.- Colère dorée. Une lumière dorée touche vos yeux. Durant les minutes suivantes, si vous attaquez une cible, vous lui infligez 2 points de dégâts supplémentaires. - Quartier d'Or. Vous gagnez +1 en armure pendant une heure grâce à un éclat translucide de lumière dorée. - Lumière de la Vérité. Chaque fois que vous essayez de discerner le mensonge au cours de l'heure qui suit, la tâche est facilitée de deux étapes. - Touche de Grâce. Avec le contact de l'allié magique, vous ajoutez 3 points à n'importe quel pool de statistiques. Si vous n'êtes pas endommagé, vous ajoutez les points au maximum de votre pool choisi. Ils restent jusqu'à ce que vous les dépensiez, que vous les perdiez à cause des dégâts ou qu'une heure s'écoule. (Minor Wish \textendash (162))

\subsection*{Souhait modéré}\label{subsec:ab_moderate_wish}

à votre demande, l'allié magique de votre capacité Créature magique liée peut dépenser son action pour vous lancer un sort modéré. Ensuite, il doit se retirer vers son objet lié pour se reposer pendant au moins une heure. Les effets qu'il peut produire sont les suivants. Action à initier. - Armure dorée. Vous gagnez +3 en armure pendant une heure grâce à un éclat translucide de lumière dorée. - Furie dorée. Une lumière dorée brille dans vos yeux. Durant les trois minutes suivantes, si vous attaquez une cible, vous lui infligez 5 points de dégâts supplémentaires. - Touche de grâce améliorée. Avec le contact de l'allié magique, vous ajoutez 6 points à n'importe quel pool de statistiques. Si vous n'êtes pas endommagé, vous ajoutez les points au maximum de votre pool choisi. Ils restent jusqu'à ce que vous les dépensiez, que vous les perdiez à cause des dégâts ou qu'une heure s'écoule. - Invisible. D'un simple contact, l'allié magique courbe la lumière qui tombe sur vous, de sorte que vous semblez disparaître. Vous êtes invisible pour les autres créatures pendant dix minutes. Bien qu'invisible, vous êtes spécialisé dans les tâches de furtivité et de défense rapide. Cet effet prend fin si vous faites quelque chose pour révéler votre présence ou votre position : attaquer, utiliser une capacité, déplacer un objet volumineux, etc. Si cela se produit, vous pouvez retrouver l'effet d'invisibilité restant en prenant une mesure pour vous concentrer sur la dissimulation de votre position. Action à initier. (Moderate Wish \textendash (163))

\subsection*{Soulager}\label{subsec:ab_alleviate}

Vous tentez d'annuler ou de guérir une maladie (comme une maladie ou un poison) chez une créature. Action. (Alleviate \textendash (109))

\subsection*{Source de guérison}\label{subsec:ab_font_of_healing}

Avec votre approbation, d'autres créatures peuvent vous toucher et regagner 1d6 points soit dans leur réserve de Puissance, soit dans leur Réserve de Célérité. Ce soin leur coûte 2 points d'Intellect. Une seule créature ne peut bénéficier de cette capacité qu'une fois par jour. Facilitateur. (Font of Healing \textendash (142))

\subsection*{Sprint de Phase}\label{subsec:ab_phase_sprint}

Vous pouvez courir sur une longue distance tant que vous n'effectuez aucune autre action. Durant votre action et jusqu'au début de votre prochain tour, vous êtes partiellement en phase, et certaines attaques vous traversent sans danger. En phase, vous gagnez un atout pour vos tâches de défense de Célérité, mais vous perdez tout avantage de l'armure que vous portez. Notez que certaines de vos autres capacités spéciales peuvent activer des actions spécifiques que vous pouvez entreprendre lors de l'utilisation de Sprint de Phase. Par exemple, lorsque vous utilisez Toucher perturbateur, vous pouvez effectuer une attaque tactile tout en vous déplaçant (bien que cela mette fin à votre mouvement). Action. (Vous n'êtes pas obligé de courir sur une longue ligne droite lorsque vous utilisez Sprint de Phase, mais vous pouvez à la place zigzaguer, courber ou même revenir à votre point de départ.) (D'autres capacités peuvent être utilisées avec Sprint de Phase pour débloquer des effets supplémentaires. , y compris Toucher perturbateur, Rayer l'Existence, Invisibilité de Phase et Détonation de phase. Ces capacités sont des outils additifs, obligeant l'utilisateur à dépenser des points pour les deux capacités, et parfois à partir de deux pools différents.) (Phase Sprint \textendash (170))

\subsection*{Spécialisation de tâche}\label{subsec:ab_task_specialization}

choisissez une tâche (autre que les attaques ou la défense) pour laquelle vous êtes entraîné. Vous vous spécialisez dans cette tâche. (Vous pouvez plutôt utiliser cette capacité comme entraînement aux tâches pour vous former à une tâche pour laquelle vous n'êtes pas entraîné.) Facilitateur. (Task Specialization \textendash (189))

\subsection*{Stagiaire juridique}\label{subsec:ab_legal_intern}

vous gagnez un adepte de niveau 4 qui est principalement intéressé à vous aider dans vos tâches liées au droit, mais qui pourrait également vous aider dans d'autres domaines. Facilitateur. (Legal Intern \textendash (157))

\subsection*{Stase}\label{subsec:ab_stasis}

Vous entourez un ennemi de votre taille ou plus petit d'une énergie scintillante, l'empêchant de bouger ou d'agir pendant une minute, comme s'il était gelé. Vous devez être capable de voir la cible et elle doit être à courte portée. En stase, la cible est insensible aux dégâts, ne peut pas être déplacée et est immunisée contre tous les effets. Action. (Stasis \textendash (186))

\subsection*{Stimuler}\label{subsec:ab_stimulate}

Vos propos encouragent une cible que vous touchez et qui peut vous comprendre. La prochaine action à entreprendre est facilitée par trois étapes. Action. (Stimulate \textendash (186))

\subsection*{Stimuler l'effort}\label{subsec:ab_spur_effort}

vous sélectionnez un allié à portée immédiate. Si ce personnage applique un effort à une tâche lors de son prochain tour, il peut appliquer un niveau d'effort gratuit sur cette tâche. Facilitateur. (Spur Effort \textendash (186))

\subsection*{Stocker l'énergie}\label{subsec:ab_store_energy}

lorsque vous drainez de l'énergie avec vos capacités de concentration, vous pouvez en stocker une partie pour plus tard dans un siphon. Vous pouvez dépenser les points de votre réserve de siphon comme s'ils provenaient de votre réserve de Puissance ou de Célérité, ou utiliser une action pour les dépenser afin de restaurer un nombre égal de points dans votre réserve de Puissance ou de Célérité. Votre réserve de siphon peut stocker en toute sécurité jusqu'à 3 points ; chaque point au-delà entrave toutes vos tâches. Facilitateur. (Store Energy \textendash (186))

\subsection*{Stratagème mortel}\label{subsec:ab_lethal_ploy}

Une longue expérience vous a révélé que le subterfuge est votre ami dans les situations désespérées. Vous poussez, attaquez ou distrayez la cible d'une manière apparemment sans conséquence qui conduit à la mort de la cible. La cible doit être de niveau 2 ou inférieur. En plus des options normales d'utilisation de l'Effort, vous pouvez choisir d'utiliser l'Effort pour augmenter le niveau maximum de la cible de 1. Ainsi, pour tuer une cible de niveau 5 (trois niveaux au-dessus de la limite normale), vous devez appliquer trois niveaux d'Effort. Effort. Action. (Lethal Ploy \textendash (158))

\subsection*{Structure de matière noire}\label{subsec:ab_dark_matter_structure}

Vous pouvez former de la matière noire en une grande structure composée d'un maximum de dix cubes de 3 m (10 pieds). La structure peut être quelque peu complexe, même si tout a la même couleur noir mat d'où aucune lumière ne brille. Sinon, la structure peut posséder différentes densités, textures et capacités. Cela signifie qu'il peut inclure des fenêtres, des portes avec serrures, des meubles et même de la décoration, à condition qu'ils soient entièrement noirs comme de la poix. Par exemple, vous pourriez façonner la matière noire en une grande structure défendable ; un pont robuste de 100 pieds (30 m); ou quelque chose de similaire. La structure est une création de niveau 6 et dure 24 heures. Vous ne pouvez pas conserver plusieurs de ces structures solides à la fois. Action. (Dark Matter Structure \textendash (124))

\subsection*{Style de combat à mains nues}\label{subsec:ab_unarmed_fighting_style}

Vous êtes entraîné aux attaques à mains nues. Facilitateur. (Unarmed Fighting Style \textendash (194))

\subsection*{Succès amélioré}\label{subsec:ab_improved_success}

lorsque vous obtenez un 17 ou plus lors d'un jet d'attaque qui inflige des dégâts, vous infligez 1 point de dégâts supplémentaire. Par exemple, si vous obtenez un 18 naturel, qui inflige normalement 2 points de dégâts supplémentaires, vous infligez à la place 3 points supplémentaires. Si vous obtenez un 20 naturel et choisissez d'infliger des dégâts au lieu d'obtenir un effet majeur spécial, vous infligez 5 points de dégâts supplémentaires. Facilitateur. (Improved Success \textendash (152))

\subsection*{Succès inspirant}\label{subsec:ab_inspiring_success}

Lorsque vous réussissez un jet pour effectuer une tâche liée à la statistique que vous avez choisie lors de la sélection de cette capacité, et que vous avez appliqué au moins un niveau d'effort, vous pouvez choisir un autre personnage à courte portée. Ce personnage a un atout pour la prochaine tâche qu'il tentera en utilisant cette statistique lors de son prochain tour. Facilitateur. (Inspiring Success \textendash (154))

\subsection*{Suggestion}\label{subsec:ab_suggestion}

Vous suggérez une action à une créature à portée immédiate. Si l'action est quelque chose que la cible pourrait normalement faire de toute façon, elle suit votre suggestion. Si la suggestion est quelque chose qui ne correspond pas à la nature ou au devoir exprès de la cible (comme demander à un garde de laisser passer un intrus), la suggestion échoue. La créature doit être de niveau 2 ou inférieur. L'effet de votre suggestion dure jusqu'à une minute. En plus des options normales d'utilisation de l'Effort, vous pouvez choisir d'utiliser l'Effort pour augmenter de 1 le niveau maximum de la cible que vous pouvez affecter. Ainsi, pour affecter une cible de niveau 5 (trois niveaux au-dessus de la limite normale), vous devez appliquer trois niveaux d'effort. Lorsque les effets de la capacité prennent fin, la créature se souvient avoir suivi la suggestion, mais peut être persuadée de croire qu'elle a choisi de le faire volontairement. Action à initier. (Suggestion \textendash (188))

\subsection*{Suggestion Psychique}\label{subsec:ab_psychic_suggestion}

Vous essayez de faire entreprendre à la cible l'action que vous indiquez lors de son prochain tour. Si l'action que vous souhaitez que la cible entreprenne causerait un préjudice direct à elle-même ou à ses alliés, votre attaque mentale est entravée. Action. (Psychic Suggestion \textendash (172))

\subsection*{Suite de Servants}\label{subsec:ab_retinue}

Quatre adeptes de niveau 2 vous rejoignent (et votre premier adepte, si vous en avez un). L'une de leurs modifications doit concerner les tâches liées au fait de servir d'assistants personnels. En plus des autres tâches qu'ils peuvent accomplir individuellement en votre nom, ils peuvent également travailler ensemble pour interférer si vous essayez d'éviter quelqu'un, vous cacher de l'attention des autres, vous aider à vous frayer un chemin à travers une foule, etc. Si une situation devient physiquement violente, ils constituent un atout pour vos tâches de défense rapide et, si vous le commandez, essayez de retenir l'attention d'un ennemi pendant que vous vous échappez. Facilitateur. (Retinue \textendash (177))

\subsection*{Suivant de base}\label{subsec:ab_basic_follower}

vous gagnez un suivant de niveau 2. L'une de leurs modifications doit être la persuasion. Vous pouvez utiliser cette capacité plusieurs fois, gagnant à chaque fois un autre adepte de niveau 2. Facilitateur. (Lorsque vous utilisez Suivant de base, le MJ peut exiger que vous recherchiez un suivant approprié.) (Basic Follower \textendash (112))

\subsection*{Surcharge d'Appareil}\label{subsec:ab_overcharge_device}

vous infusez 1 point d'énergie gagné en utilisant Absorber l'énergie ou une capacité associée dans un appareil, tel qu'un artefact, augmentant ainsi son niveau effectif de trois lors de sa prochaine utilisation (jusqu'à un maximum de 10). Action. (Overcharge Device \textendash (168))

\subsection*{Surcharge d'énergie}\label{subsec:ab_overcharge_energy}

lorsque vous utilisez Libération d'énergie, elle inflige 2 points de dégâts supplémentaires. Facilitateur. (Overcharge Energy \textendash (168))

\subsection*{Surcharge de Machine}\label{subsec:ab_overload_machine}

grâce à l'assistant robot de votre capacité Serv-0, vous insufflez à un appareil alimenté de niveau 3 ou inférieur plus d'énergie qu'il ne peut en gérer. S'il est affecté, l'appareil est détruit ou désactivé pendant au moins une minute, en fonction de sa taille et de sa complexité. Le MJ peut décider que l'effet invalidant dure jusqu'à ce que l'appareil soit réparé. En plus des options normales d'utilisation de l'Effort, vous pouvez choisir d'utiliser l'Effort pour augmenter le niveau maximum de la cible. Ainsi, pour surcharger un appareil de niveau 5 (deux niveaux au-dessus de la limite normale), vous devez appliquer deux niveaux d'Effort. Action. (Overload Machine \textendash (168))

\subsection*{Surfeur de Matière Noire}\label{subsec:ab_windwracked_traveler}

Vous condensez une large portion de matière noire qui peut vous transporter dans les airs pendant une période pouvant aller jusqu'à une heure. Pour chaque niveau d'Effort appliqué, vous pouvez ajouter une heure à la durée ou transporter une créature supplémentaire de votre taille ou moins. Vous devez toucher les créatures supplémentaires pour qu'elles soient placées sous votre aile. Ils doivent rester relativement immobiles tant que l'aile dure, sinon ils tomberont. En termes de déplacements terrestres, vous volez à environ 32 km par heure et n'êtes pas affecté par le terrain. Action à initier. (Windwracked Traveler \textendash (199))

\subsection*{Surfeur de voiture}\label{subsec:ab_car_surfer}

Vous pouvez vous tenir debout ou vous déplacer sur un véhicule en mouvement (comme sur le capot, sur le toit, dans la porte ouverte, etc.) avec l'espoir raisonnable de ne pas tomber. À moins que le véhicule ne vire brusquement, ne s'arrête brusquement ou ne se livre à des manœuvres extrêmes, rester debout ou se déplacer sur un véhicule en mouvement est une tâche courante pour vous. Si le véhicule s'engage dans des manœuvres extrêmes comme celles décrites, les tâches de maintien à la surface du véhicule sont facilitées. Facilitateur. (Car Surfer \textendash (118))

\subsection*{Surfeur des Vents}\label{subsec:ab_windrider}

Vous invoquez des vents qui vous soulèvent et vous permettent de voler pendant une minute à une Célérité pouvant atteindre une longue distance à chaque tour. Pour chaque niveau d'Effort que vous appliquez, vous pouvez transporter avec vous un allié d'environ votre taille dans les airs ou augmenter la durée de l'effet d'une minute. Action à initier. (Windrider \textendash (199))

\subsection*{Surmontez tous les obstacles}\label{subsec:ab_overcome_all_obstacles}

Ceux qui s'opposent à vous le font à leurs risques et périls et finissent par reculer en votre présence. Lorsque vous vous concentrez sur un ennemi particulier à longue portée, la cible subit 2 points de dégâts d'Intellect (ignore l'armure) à chaque tour pendant une minute ou jusqu'à ce que la cible puisse annuler l'effet. Cette capacité ne peut être active que sur une seule cible à la fois. Vous pouvez appliquer l'Effort pour augmenter les dégâts au cours du premier tour, ainsi que pour tout tour au cours duquel vous appliquez l'Effort et utilisez une autre action. Action à initier. (Overcome All Obstacles \textendash (168))

\subsection*{Sursaut de Cypher}\label{subsec:ab_cypher_surge}

lorsque vous utilisez un sort de cypher subtil, dans le cadre de cette action, vous pouvez dépenser un autre cypher subtil. Au lieu de l'effet normal du deuxième cypĥer, vous ajoutez un niveau d'effort gratuit au premier sort de cypĥer. Facilitateur. (Cypher Surge \textendash (GF, 29))

\subsection*{Sursaut de Célérité}\label{subsec:ab_speed_burst}

vous pouvez effectuer deux actions distinctes au cours de ce tour. Au tour suivant, toutes les actions sont entravées. Vous ne pouvez pas utiliser cette capacité deux rounds de suite. Facilitateur. (Speed Burst \textendash (185))

\subsection*{Sursaut de Célérité Parfait}\label{subsec:ab_perfect_speed_burst}

vous pouvez effectuer deux actions distinctes ce tour-ci. Facilitateur. (Perfect Speed Burst \textendash (169))

\subsection*{Sursaut de confiance}\label{subsec:ab_surging_confidence}

lorsque vous utilisez une action pour effectuer votre premier jet de récupération de la journée, vous gagnez immédiatement une autre action. Facilitateur. (Surging Confidence \textendash (188))

\subsection*{Sursaut mental}\label{subsec:ab_mind_surge}

en plus de vos jets de récupération normaux chaque jour, vous pouvez, à tout moment entre dix heures de repos, récupérer 1d6 + 6 points dans votre réserve d'intelligence. Action. (Mind Surge \textendash (162))

\subsection*{Survivant post-apocalyptique}\label{subsec:ab_post_apocalyptic_survivor}

vous êtes entraîné aux tâches de furtivité et de défense de Puissance. Facilitateur. (Post-Apocalyptic Survivor \textendash (170))

\subsection*{Survivre à l'ennemi}\label{subsec:ab_outlast_the_foe}

si vous avez combattu pendant cinq rounds complets, vous disposez d'un atout pour toutes les tâches pour le reste du combat et vous infligez 1 point de dégâts supplémentaire par attaque. Facilitateur. (Outlast the Foe \textendash (167))

\subsection*{Survol}\label{subsec:ab_hover}

Vous flottez lentement dans les airs. Si vous vous concentrez, vous pouvez contrôler votre mouvement pour rester immobile dans les airs ou flotter sur une courte distance au fur et à mesure de votre action ; sinon, vous dérivez avec le vent ou avec l'élan que vous avez acquis. Cet effet dure jusqu'à dix minutes. Action à initier. (Hover \textendash (149))

\subsection*{Symbole divin}\label{subsec:ab_divine_symbol}

Vous invoquez le pouvoir divin en traçant un symbole lumineux dans l'air avec vos doigts. Des piliers tordus de rayonnement divin lancent jusqu'à cinq cibles à longue portée. Une attaque réussie sur une cible inflige 5 points de dégâts. Si vous appliquez Effort pour augmenter les dégâts, vous infligez 2 points de dégâts supplémentaires par niveau d'Effort (au lieu de 3 points) ; les cibles subissent 1 point de dégâts même si vous échouez au jet d'attaque. Action. (Divine Symbol \textendash (131))

\
%--------------------------
\section*{T}

\subsection*{Tacticien du champ de bataille}\label{subsec:ab_battlefield_tactician}

Vous scrutez votre environnement, apprenant tous les faits que le MJ juge pertinents pour attaquer, défendre, manœuvrer et gérer les dangers environnementaux à courte distance. Par exemple, vous remarquerez peut-être un tas de décombres sur lequel vous pouvez vous appuyer pour avoir un avantage en mêlée, un coin abrité pour vous protéger contre les attaques ennemies, une partie moins glissante d'un lac gelé ou un endroit où le gaz toxique est plus mince que autre part. Si vous (ou quelqu'un à qui vous en informez) vous déplacez vers cet endroit, vous (ou eux) gagnez un atout sur les tâches liées à cette position optimale (telles que les jets d'attaque depuis les hauteurs, les jets de défense de Célérité depuis le coin abrité, les jets d'équilibre sur le terrain). lac gelé, ou jets de défense de Puissance contre le nuage empoisonné). Au lieu d'obtenir un emplacement avantageux, vous pourriez découvrir un emplacement désavantageux que vous pourriez utiliser contre vos ennemis, comme les manœuvrer dans un coin difficile qui gêne leurs attaques de mêlée ou un point faible sur le lac gelé qui se brisera s'ils se tiennent dessus. il. Vous pouvez appliquer l'Effort pour apprendre un emplacement supplémentaire, bon ou mauvais, à portée (un emplacement par niveau d'Effort), augmenter la portée de cette capacité (une autre courte distance par niveau d'Effort), ou les deux. Facilitateur. (Battlefield Tactician \textendash (112))

\subsection*{Tactiques de la Horde}\label{subsec:ab_horde_tactics}

Jusqu'à une heure par jour, vous et au moins trois autres alliés pouvez agir comme une seule créature. Utilisez vos statistiques, mais ajoutez +8 à votre Réserve de Puissance, +1 à votre Avantage de Puissance, +2 à votre Réserve de Célérité, +1 à votre Avantage de Vistesse et +1 à votre Armure. Facilitateur. (Horde Tactics \textendash (149))

\subsection*{Talent Oratoire}\label{subsec:ab_oratory}

lorsque vous parlez avec un groupe de créatures intelligentes qui peuvent vous comprendre et ne sont pas hostiles, vous les convainquez d'entreprendre une action raisonnable au tour suivant. Une action raisonnable ne doit pas mettre les créatures ou leurs alliés en danger évident ou être totalement hors de leur caractère. Action. (Oratory \textendash (167))

\subsection*{Talent de célébrité}\label{subsec:ab_celebrity_talent}

vous êtes entraîné dans deux des domaines suivants: l'écriture, le journalisme, un style artistique particulier, un sport particulier, les échecs, la communication scientifique, le théâtre, la présentation de l'actualité ou toute autre compétence non liée au combat qui a conduit à votre célébrité. Facilitateur. (Celebrity Talent \textendash (119))

\subsection*{Tempête de Glace}\label{subsec:ab_ice_storm}

vous tentez une tâche Intellect supplémentaire dans le cadre de votre attaque Explosion de froid, et en cas de succès, vous aveuglez les ennemis pendant une minute maximum avec une couche de glace gelée. Toutes les tâches des créatures aveuglées sont entravées par deux étapes. Facilitateur. (Ice Storm \textendash (150))

\subsection*{Tenez bon}\label{subsec:ab_tough_it_out}

Travailler pour gagner sa vie vous a endurci au fil du temps. Vous avez +1 à l'Armure contre tout type de dégâts physiques, même les dégâts qui ignorent normalement l'Armure. Facilitateur. (Tough It Out \textendash (193))

\subsection*{Tir Arcanique}\label{subsec:ab_arcane_flare}

Vous améliorez les dégâts d'un autre sort d'attaque avec une charge d'énergie supplémentaire afin qu'il inflige 1 point de dégâts supplémentaire. Alternativement, vous attaquez une cible à longue portée en projetant un tir de magie brute qui inflige 4 points de dégâts. Facilitateur d'amélioration ; action pour une attaque à longue portée. (Arcane Flare \textendash (110))

\subsection*{Tir Double}\label{subsec:ab_trick_shot}

Dans le cadre de la même action, vous effectuez une attaque à distance contre deux cibles se trouvant à portée immédiate l'une de l'autre. Effectuez un jet d'attaque séparé contre chaque cible. Les jets d'attaque sont gênés. Action. (Trick Shot \textendash (194))

\subsection*{Tir Guidé}\label{subsec:ab_guide_bolt}

lorsque vous faites une attaque avec un carreau métallique ou une flèche à pointe métallique sur une cible à courte portée, vous pouvez améliorer la visée et la Célérité de l'attaque, ce qui confère un atout à l'attaque et inflige 2 points de dégâts supplémentaires. Si vous appliquez un niveau d'Effort, vous accordez les mêmes avantages à une attaque à distance effectuée par un allié à portée immédiate. Dans tous les cas, vous ne pouvez utiliser cette capacité qu'une seule fois par tour. Facilitateur. (Guide Bolt \textendash (147))

\subsection*{Tir Perforant}\label{subsec:ab_pierce}

Il s'agit d'une attaque à distance bien ciblée et pénétrante. Vous effectuez une attaque et infligez 1 point de dégâts supplémentaire si votre arme a une pointe acérée. Action. (Pierce \textendash (170))

\subsection*{Tir Précis}\label{subsec:ab_snipe}

Si vous passez une action à viser, au tour suivant, vous pouvez effectuer une attaque à distance précise. Vous avez un atout sur cette attaque. Si votre attaque réussit, elle inflige 4 points de dégâts supplémentaires. Action. (Snipe \textendash (183))

\subsection*{Tir Rapide}\label{subsec:ab_snap_shot}

Vous pouvez effectuer deux attaques au pistolet en une seule action, mais la deuxième attaque est gênée par deux étapes. Facilitateur. (Snap Shot \textendash (183))

\subsection*{Tir Spécial}\label{subsec:ab_special_shot}

Lorsque vous touchez une cible avec une attaque au pistolet, vous pouvez choisir de réduire les dégâts de 1 point mais toucher la cible à un endroit précis. Certains des effets possibles incluent (sans toutefois s'y limiter) les suivants :- Vous pouvez tirer un objet hors de la main de quelqu'un. - Vous pouvez tirer sur la jambe, l'aile ou tout autre membre qu'il utilise pour se déplacer, réduisant ainsi sa Célérité de déplacement maximale à immédiate pendant quelques jours ou jusqu'à ce qu'il reçoive des soins médicaux spécialisés. - Vous pouvez tirer sur une sangle retenant un sac à dos, une armure ou un objet similaire attaché afin qu'il tombe. Facilitateur. (Special Shot \textendash (184))

\subsection*{Tir d'Opportunité}\label{subsec:ab_overwatch}

vous utilisez une arme à distance pour cibler une zone limitée (comme une porte, un couloir ou le côté est de la clairière) et effectuez une attaque contre la prochaine cible viable qui entre dans cette zone. Cela fonctionne comme une action d'attente, mais vous annulez également tout avantage que la cible aurait en matière de couverture, de position, de surprise, de portée, d'éclairage ou de visibilité. De plus, vous infligez 1 point de dégâts supplémentaire avec l'attaque. Vous pouvez rester sous surveillance aussi longtemps que vous le souhaitez, dans des limites raisonnables. Action. (Overwatch \textendash (168))

\subsection*{Tir en apesanteur}\label{subsec:ab_weightless_shot}

Vous avez un sixième sens lorsqu'il s'agit d'aligner des trajectoires et de vous déplacer dans des environnements en apesanteur et en apesanteur, ce qui se traduit également par des attaques à distance. Lorsque vous touchez une cible avec une attaque à distance en microgravité, vous pouvez choisir de réduire les dégâts de 2 points mais de toucher la cible à un endroit précis. Certains des effets possibles incluent (sans toutefois s'y limiter) les suivants : - Vous percez un trou dans la combinaison de la cible, de sorte qu'elle commence à laisser échapper de l'air dans le vide lentement, ou d'un seul coup (votre choix). - Vous touchez la masse de réaction du pack de manœuvre de la cible, ce qui signifie que la cible ne peut plus changer de trajectoire, ou qu'elle part en rotation dans une direction aléatoire (votre choix). - Vous pouvez tirer sur un vaisseau spatial et dégrader le système d'un vaisseau d'une étape (les systèmes incluent les moteurs, les armes et l'atmosphère). Facilitateur. (Weightless Shot \textendash (197))

\subsection*{Tir prudent}\label{subsec:ab_careful_shot}

vous pouvez dépenser des points de votre Réserve de Célérité ou de votre réserve d'intelligence pour appliquer des niveaux d'effort afin d'augmenter les dégâts de votre arme. Chaque niveau d'Effort ajoute 3 points de dégâts à une attaque réussie, et si vous passez un tour à aligner votre tir, chaque niveau d'Effort ajoute à la place 5 points de dégâts à une attaque réussie. Facilitateur. (Careful Shot \textendash (118))

\subsection*{Tirer une conclusion}\label{subsec:ab_draw_conclusion}

Après une observation et une enquête minutieuses (interroger un ou plusieurs PNJ sur un sujet, fouiller une zone ou un dossier, etc.) d'une durée de quelques minutes, vous pouvez apprendre un fait pertinent. Cette capacité est une tâche de difficulté Intellect 3. Chaque fois que vous utilisez cette capacité, la tâche est entravée par une étape supplémentaire. La difficulté revient à 3 après dix heures de repos. Action. (Draw Conclusion \textendash (131))

\subsection*{Tireur}\label{subsec:ab_gunner}

Vous infligez 1 point de dégâts supplémentaire avec les armes à feu. Facilitateur. (Gunner \textendash (147))

\subsection*{Tireur entraîné}\label{subsec:ab_trained_gunner}

vous pouvez choisir entre deux avantages. Soit vous êtes entraîné au maniement des armes à feu, soit vous possédez la capacité Pulvérisation (qui coûte 2 points de Célérité) : Si une arme a la capacité de tirer des coups rapides sans rechargement (généralement appelée arme à tir rapide, comme un pistolet automatique), vous pouvez pulvériser plusieurs tirs autour de votre cible pour augmenter les chances de toucher. Ce mouvement utilise 1d6 + 1 cartouches de munitions (ou toutes les munitions de l'arme, si elle en contient moins que le nombre obtenu). Le jet d'attaque est allégé. Si l'attaque réussit, elle inflige 1 point de dégâts de moins que la normale. Facilitateur (être entraîné à l'utilisation des armes à feu) ou action (Pulvérisation). (Trained Gunner \textendash (193))

\subsection*{Tirs en éventail}\label{subsec:ab_arc_spray}

Si une arme a la capacité de tirer des coups rapides sans recharger (généralement appelée arme à tir rapide, comme une arbalète à manivelle), vous pouvez tirer avec votre arme sur jusqu'à trois cibles (toutes à côté de les uns les autres) en même temps. Effectuez un jet d'attaque séparé contre chaque cible. Chaque attaque est gênée. Action. (Arc Spray \textendash (110))

\subsection*{Torsion du couteau}\label{subsec:ab_twisting_the_knife}

Au cours d'un round après avoir réussi à frapper un ennemi avec une arme de mêlée, vous pouvez choisir d'infliger automatiquement des dégâts standard à l'ennemi avec cette même arme sans aucun modificateur (2 points pour une arme légère, 4 points pour une arme moyenne, ou 6 points pour une arme lourde). Action. (Twisting the Knife \textendash (194))

\subsection*{Totalement Chill}\label{subsec:ab_totally_chill}

Votre jet de récupération de dix minutes ne vous prend qu'un seul round. Facilitateur. (Totally Chill \textendash (192))

\subsection*{Toucher Glacial}\label{subsec:ab_frost_touch}

Vos mains deviennent si froides que la prochaine fois que vous touchez une créature, vous infligez 3 points de dégâts. Alternativement, vous pouvez utiliser cette capacité sur une arme, et pendant dix minutes, elle inflige 1 point de dégâts supplémentaire du froid. Action pour le toucher ; facilitateur pour arme. (Frost Touch \textendash (144))

\subsection*{Toucher Mortel}\label{subsec:ab_death_touch}

Vous rassemblez de l'énergie perturbatrice au bout de votre doigt et touchez une créature. Si la cible est un PNJ ou une créature de niveau 3 ou inférieur, elle meurt. Si la cible est un PJ de n'importe quel niveau, il descend d'un cran sur le suivi des dégâts. Action. (Death Touch \textendash (125))

\subsection*{Toucher de Froid Paralysant}\label{subsec:ab_freezing_touch}

Vos mains deviennent si froides que votre contact gèle une cible vivante de votre taille ou plus petite, la rendant immobile pendant un round. Si vous disposez d'une autre capacité de froid activée par le toucher (comme Toucher Glacial), vous pouvez l'utiliser dans le cadre de l'attaque Toucher de Froid Paralysant. Action. (Freezing Touch \textendash (143))

\subsection*{Toucher lumineux}\label{subsec:ab_illuminating_touch}

Vous touchez un objet, et cet objet émet de la lumière pour tout éclairer à courte portée. La lumière reste jusqu'à ce que vous utilisiez une action pour toucher à nouveau l'objet, ou jusqu'à ce que vous ayez éclairé plus d'objets que vous n'avez de niveaux, auquel cas les objets les plus anciens que vous avez éclairés s'assombrissent en premier. Action. (Illuminating Touch \textendash (150))

\subsection*{Toucher perturbateur}\label{subsec:ab_disrupting_touch}

vous pouvez transformer votre Sprint de Phase en une attaque de mêlée en effleurant délibérément une autre créature pendant que vous courez. Lorsque vous le faites, le contact libère une violente explosion d'énergie qui inflige 2 points de dégâts à la cible (ignore l'Armure). Que vous frappiez ou ratiez, votre mouvement (et votre tour) se termine immédiatement, ce qui vous place à distance immédiate de votre cible. Si vous appliquez Effort pour augmenter les dégâts plutôt que pour faciliter la tâche, vous infligez 2 points de dégâts supplémentaires par niveau d'Effort (au lieu de 3 points) ; la cible subit 1 point de dégâts même si vous échouez au jet d'attaque. Facilitateur. (Disrupting Touch \textendash (129))

\subsection*{Toujours bricoler}\label{subsec:ab_always_tinkering}

si vous disposez d'outils et de matériel et que vous transportez moins de chiffres que votre limite, vous pouvez créer un chiffre manifeste si vous avez une heure à consacrer. Le nouveau chiffre est aléatoire et toujours 2 niveaux inférieurs à la normale (minimum 1). C'est aussi capricieux et fragile. C'est ce qu'on appelle des chiffres capricieux. Si vous en donnez un à quelqu'un d'autre, il s'effondre immédiatement, inutile. Action à initier ; une heure pour terminer. (Always Tinkering \textendash (109))

\subsection*{Toujours comme une statue}\label{subsec:ab_still_as_a_statue}

En vous appuyant sur la Puissance de votre Corps de Golem, vous vous figez sur place, enfouissant votre essence au plus profond de votre noyau de pierre. Pendant ce temps, vous perdez toute mobilité ainsi que la capacité d'entreprendre des actions physiques. Vous ne pouvez pas sentir ce qui se passe autour de vous et aucun temps ne semble s'écouler pour vous. Tant que vous êtes toujours en tant que statue, vous gagnez +10 en armure contre les dégâts de toutes sortes. Dans des circonstances normales, vous retrouvez automatiquement un état d'éveil et une mobilité normale un jour plus tard. Si un allié en qui vous avez confiance vous secoue assez fort (avec un coût minimum de 2 points de Puissance), vous vous réveillez plus tôt. Action à initier. (Still As a Statue \textendash (186))

\subsection*{Tour de Volonté}\label{subsec:ab_tower_of_will}

Vous êtes entraîné aux tâches de défense intellectuelle et gagnez +3 points à votre réserve d'intelligence. Facilitateur. (Tower of Will \textendash (193))

\subsection*{Tour de force}\label{subsec:ab_feat_of_strength}

toute tâche qui dépend de la force brute est facilitée. Les exemples incluent défoncer une porte grillagée, ouvrir un conteneur verrouillé, soulever ou déplacer un objet lourd ou frapper quelqu'un avec une arme de mêlée. Facilitateur. (Feat of Strength \textendash (139))

\subsection*{Tour de l'Intellect}\label{subsec:ab_tower_of_intellect}

Vous êtes entraîné aux tâches de défense d'Intellect. Si vous êtes déjà entraîné, vous êtes plutôt spécialisé dans ces tâches. Facilitateur. (Tower of Intellect \textendash (193))

\subsection*{Tourbillon de lancers}\label{subsec:ab_whirlwind_of_throws}

Avec une grande poignée de petits objets (minuscules couteaux, shuriken, pierres, morceaux de métal déchiquetés, pièces de monnaie ou tout ce que vous avez sous la main), vous attaquez toutes les créatures dans une zone immédiate à courte portée. Vous devez effectuer des jets d'attaque contre chaque cible. Chaque attaque est gênée. Vous infligez 3 points de dégâts aux cibles que vous touchez. Action. (Whirlwind of Throws \textendash (198))

\subsection*{Tout est une arme}\label{subsec:ab_everything_is_a_weapon}

vous pouvez prendre n'importe quel petit objet (une pièce de monnaie, un stylo, une bouteille, une pierre, etc.) et le lancer avec une telle force et précision qu'il inflige des dégâts comme une arme légère. Facilitateur. (Everything Is a Weapon \textendash (136))

\subsection*{Trahison}\label{subsec:ab_betrayal}

Chaque fois que vous convainquez un ennemi que vous n'êtes pas une menace et que vous l'attaquez soudainement (sans provocation), l'attaque inflige 4 points de dégâts supplémentaires. Facilitateur. (Betrayal \textendash (113))

\subsection*{Trahison inattendue}\label{subsec:ab_unexpected_betrayal}

quelques rounds après avoir utilisé avec succès Envoûtement, Embrouiller ou une capacité similaire sur une cible à courte portée, la première attaque que vous effectuez sur cette cible est facilitée de deux étapes. Une fois que vous utilisez Trahison inattendue sur une cible, l'utilisation de vos capacités ou une simple tentative de persuasion sur cette cible est définitivement entravée par deux étapes. Facilitateur. (Unexpected Betrayal \textendash (195))

\subsection*{Traitement rapide}\label{subsec:ab_rapid_processing}

vous ou une cible que vous touchez expérimentez un niveau de temps de réaction mentale et physique plus élevé pendant environ une minute. Pendant cette période, toutes les tâches de Célérité (y compris les jets de défense de Célérité) sont facilitées. De plus, la cible peut effectuer une action supplémentaire à tout moment avant l'expiration de la durée de la capacité. Action. (Rapid Processing \textendash (174))

\subsection*{Transcendez le scénario}\label{subsec:ab_transcend_the_script}

Qu'il s'agisse de lignes que vous avez écrites, jouées, sur lesquelles vous avez rapporté ou incorporées de toute autre manière dans votre talent, vous composez un oratoire à la volée qui est si merveilleux que même vous y croyez. Pour chaque allié qui l'entend (et vous aussi), une tâche tentée dans l'heure qui suit est facilitée de deux étapes. Action. (Transcend the Script \textendash (193))

\subsection*{Transfert de dégâts}\label{subsec:ab_damage_transference}

lorsque vous ou votre double (issu de la capacité Dupliquer) subissez des dégâts, vous pouvez transférer 1 point de dégâts de l'un à l'autre à condition que vous et votre double soyez à moins de 1,5 km l'un de l'autre. Facilitateur. (Damage Transference \textendash (124))

\subsection*{Transmettre la compréhension}\label{subsec:ab_impart_understanding}

votre capacité Apprendre le chemin fonctionne plus efficacement, vous permettant de faciliter une tâche de deux étapes ou de fournir deux atouts à la tâche d'un ami, au lieu de la faciliter normalement. Facilitateur. (Impart Understanding \textendash (151))

\subsection*{Transmettre un idéal}\label{subsec:ab_impart_ideal}

Après avoir interagi pendant au moins une minute avec une créature qui peut vous entendre et vous comprendre, vous pouvez tenter de lui transmettre temporairement un idéal que vous ne pourriez pas autrement la convaincre d'adopter. Un idéal est différent d'une suggestion ou d'un commandement spécifique ; un idéal est une valeur primordiale telle que « Toute vie est sacrée », « Mon parti politique est le meilleur », « Les enfants doivent être vus, pas entendus », etc. Un idéal influence le comportement d'une créature mais ne le contrôle pas. L'idéal transmis dure aussi longtemps que la situation le permet, mais généralement au moins quelques heures. L'idéal est compromis si quelqu'un d'ami envers la créature passe une minute ou plus à la ramener à la raison. Action. (Impart Ideal \textendash (151))

\subsection*{Traqueur}\label{subsec:ab_stalker}

vous gagnez un atout pour tous les types de tâches de mouvement (y compris l'escalade, la natation, le saut et l'équilibre). Facilitateur. (Stalker \textendash (186))

\subsection*{Travail d'équipe}\label{subsec:ab_teamwork}

grâce à l'exemple, aux actes de camaraderie, aux histoires de prouesses martiales ou à d'autres formes d'instruction, vous et vos alliés travaillez mieux ensemble en tant qu'unité cohésive. Lors de tout round au cours duquel vous rallierez votre équipe (en dépensant 2 points d'Intellect dans le cadre d'une autre action), vous et vos alliés infligez 1 point de dégâts supplémentaire en combat. Cet avantage s'applique uniquement aux alliés avec lesquels vous avez passé les dernières 24 heures. Cela se termine si vous partez, mais cela reprend si vous revenez en compagnie de vos alliés dans les 24 heures. Si vous partez plus de 24 heures, vous devez passer encore 24 heures ensemble pour réactiver l'avantage. Facilitateur. (Teamwork \textendash (189))

\subsection*{Travail rapide}\label{subsec:ab_quick_work}

Une utilisation de n'importe quel artefact (ou une minute de sa fonction continue) est augmentée d'un niveau si vous l'utilisez dans la minute suivante. Si vous dépensez 4 points d'Intellect supplémentaires, l'utilisation est augmentée de deux niveaux si vous l'utilisez dans la minute suivante. Action. (Quick Work \textendash (174))

\subsection*{Travaux Manuels}\label{subsec:ab_handy}

Vous travaillez pour gagner votre vie et êtes entraîné aux tâches liées à la menuiserie, à la plomberie et à la réparation électrique. Vos connaissances dans ces domaines vous donnent également un atout pour créer des objets entièrement nouveaux dans vos sphères de connaissances et dans les limites des possibilités du contexte. Facilitateur. (Handy \textendash (148))

\subsection*{Traverser les murs}\label{subsec:ab_walk_through_walls}

Vous pouvez franchir lentement les barrières physiques à raison de 1 pouce (2,5 cm) par tour (minimum d'un tour pour franchir n'importe quelle barrière). Vous ne pouvez pas agir (sauf bouger) ni percevoir quoi que ce soit avant d'avoir entièrement franchi la barrière. Vous ne pouvez pas franchir les barrières énergétiques. Action. (Walk Through Walls \textendash (196))

\subsection*{Traversez les mondes}\label{subsec:ab_traverse_the_worlds}

Vous vous transmettez instantanément à une autre planète, dimension, plan ou niveau de réalité. Il faut savoir que la destination existe ; le MJ décidera si vous disposez de suffisamment d'informations pour confirmer son existence et le niveau de difficulté pour atteindre la destination. En plus des options normales d'utilisation d'Effort, vous pouvez choisir d'utiliser Effort pour amener d'autres personnes avec vous ; chaque niveau d'Effort utilisé de cette manière affecte jusqu'à trois cibles supplémentaires. Vous devez toucher toutes les cibles supplémentaires. Action. (Traverse the Worlds \textendash (194))

\subsection*{Tremblement de résonance}\label{subsec:ab_resonant_quake}

Vous pouvez imprégner le sol sous vous d'une vibration spéciale générée par votre cœur. Cela crée un petit séisme dont vous pouvez sélectionner l'épicentre sur une très longue distance. Toute personne se trouvant à courte portée de l'épicentre est sujette à 8 points de dégâts (en tremblant et en étant frappée par des objets qui tombent, des murs qui s'effondrent, etc.). Cependant, vous restez vous-même hébété pendant un tour, pendant lequel toutes vos tâches sont entravées. Si vous disposez de la capacité Déplacer les montagnes, les deux capacités coûtent 3 points d'Intellect de moins à utiliser. Action. (Resonant Quake \textendash (177))

\subsection*{Tremblement de terre}\label{subsec:ab_earthquake}

Vous dirigez votre résonance destructrice dans le sol et déclenchez un tremblement de terre centré sur un endroit que vous pouvez voir à très longue portée. Le sol à proximité de cet endroit se soulève et tremble pendant cinq minutes, causant des dommages aux structures et au terrain de la zone. Les bâtiments et les caractéristiques du terrain libèrent des débris et des décombres. À chaque tour, les créatures dans la zone subissent soit 3 points de dégâts en raison du tremblement général, soit 6 points de dégâts si elles se trouvent dans ou à proximité d'une structure ou d'un élément de terrain qui laisse tomber des débris. Action à initier. (Earthquake \textendash (133))

\subsection*{Troisième œil}\label{subsec:ab_third_eye}

vous visualisez un endroit à courte portée et projetez votre esprit vers cet endroit, créant un capteur immobile et invisible pendant une minute ou jusqu'à ce que vous choisissiez de mettre fin à cette capacité. Lorsque vous utilisez votre troisième œil, vous voyez à travers votre capteur au lieu de vos yeux en utilisant vos capacités visuelles normales. Vous pouvez percevoir la zone autour de votre corps en utilisant vos autres sens normalement. Action. (Third Eye \textendash (191))

\subsection*{Troisième œil itinérant}\label{subsec:ab_roaming_third_eye}

Lorsque vous utilisez votre capacité Troisième œil, vous pouvez placer le capteur n'importe où à longue portée. De plus, jusqu'à la fin de cette capacité, vous pouvez utiliser une action pour déplacer le capteur n'importe où à courte portée de sa position de départ. Facilitateur. (Roaming Third Eye \textendash (178))

\subsection*{Trou de ver}\label{subsec:ab_wormhole}

Vous créez une porte à travers le temps et l'espace. Le raccourci se manifeste par un trou en réalité suffisamment grand pour accueillir vous et des créatures de votre taille ou plus petites. Un côté de la porte apparaît n'importe où à portée immédiate, et l'autre côté s'ouvre à un endroit que vous choisissez n'importe où à longue portée. Tout personnage ou objet se déplaçant d'un côté sort de l'autre. La porte reste ouverte pendant une minute ou jusqu'à ce que vous utilisiez une action pour la fermer. Action à initier. (Wormhole \textendash (200))

\subsection*{Trouver ce qui est caché}\label{subsec:ab_find_the_hidden}

Vous voyez les traces d'objets lorsqu'ils se déplacent dans l'espace et le temps. Vous pouvez sentir la distance et la direction de tout objet inanimé spécifique que vous avez touché une fois. Cela prend entre une action et des heures de concentration, selon ce que le MJ estime approprié en raison du temps, de la distance ou d'autres circonstances atténuantes. Cependant, vous ne savez pas à l'avance combien de temps cela prendra. Si vous utilisez au moins deux niveaux d'Effort, une fois que vous avez établi la distance et la direction, vous restez en contact avec l'objet pendant une heure par niveau d'Effort utilisé. Ainsi, s'il bouge, vous êtes conscient de sa nouvelle position. Action à initier ; action à chaque tour pour se concentrer. (Find the Hidden \textendash (140))

\subsection*{Trouver le chemin}\label{subsec:ab_find_the_way}

lorsque vous appliquez un effort à une tâche de navigation parce que vous ne connaissez pas le chemin, que vous êtes perdu, que vous essayez de tracer un nouvel itinéraire, que vous devez choisir entre deux ou plusieurs chemins par ailleurs similaires à emprunter, ou quelque chose de très similaire, vous pouvez appliquer un niveau d'Effort gratuit. Facilitateur. (Find the Way \textendash (140))

\subsection*{Trouver le coupable}\label{subsec:ab_find_the_guilty}

si vous avez utilisé la désignation sur une cible, vous êtes entraîné à la suivre, à la repérer lorsqu'elle est cachée ou déguisée, ou à la trouver d'une autre manière. Facilitateur. (Find the Guilty \textendash (139))

\subsection*{Trouver les Pièges}\label{subsec:ab_trapfinder}

vous trouvez des pièges (comme un sol qui céderait sous vous) ou des déclencheurs mécaniques d'un piège ou d'un système de défense qui pourraient constituer une menace. Vous pouvez le faire sans les déclencher et au lieu de lancer un jet pour les trouver. Cette capacité peut trouver des pièges de niveau 4 ou inférieur. En plus des options normales d'utilisation de l'Effort, vous pouvez choisir d'utiliser l'Effort pour augmenter de 2 le niveau des pièges pouvant être trouvés, de sorte qu'en utilisant deux niveaux d'Effort, vous puissiez trouver tous les pièges de niveau 8 ou inférieur. Action. (Trapfinder \textendash (193))

\subsection*{Trouver une ouverture}\label{subsec:ab_find_an_opening}

vous utilisez la ruse pour trouver une ouverture dans les défenses de votre ennemi. Si vous réussissez un jet de Célérité contre une créature à portée immédiate, votre prochaine attaque contre cette créature avant la fin du tour suivant est facilitée. Action. (Find an Opening \textendash (139))

\subsection*{Très long sprint}\label{subsec:ab_very_long_sprinting}

lorsque vous utilisez Sprint de Phase, vous pouvez parcourir une très longue distance au lieu d'une longue distance. Facilitateur. (Very Long Sprinting \textendash (196))

\subsection*{Télécommande}\label{subsec:ab_remote_control}

Vous pouvez utiliser les réseaux de communication et de capteurs d'un vaisseau spatial pour lancer une attaque qui rend brièvement un vaisseau ennemi dans un rayon de 20 miles (32 km) inopérant pendant une minute maximum. Action. (Le contrôle à distance est une tentative magistrale de bloquer ou de pirater un vaisseau spatial ennemi, une tâche nécessitant normalement plusieurs lancers, et vous ne réussissez que si vous obtenez un total de trois succès avant d'obtenir un total de deux échecs. Cependant, toutes ces tâches sont gênées par au moins deux étapes en raison de la sécurité électronique renforcée du vaisseau spatial.) (Remote Control \textendash (176))

\subsection*{Télékinésie}\label{subsec:ab_telekinesis}

Vous pouvez exercer une force sur des objets à courte portée. Une fois activé, votre pouvoir a une réserve de Puissance effective de 10, un Avantage de Puissance de 1 et un effort de 2 (approximativement égal à la force d'un humain adulte en forme et capable), et vous pouvez l'utiliser pour déplacer des objets, pousser contre des objets. objets, etc. Par exemple, vous pouvez soulever et tirer un objet léger n'importe où à portée de vous ou déplacer un objet lourd (comme un meuble) d'environ 10 pieds (3 m). Ce pouvoir n'a pas le contrôle précis nécessaire pour manier une arme ou déplacer des objets avec beaucoup de Célérité, donc dans la plupart des situations, ce n'est pas un moyen d'attaque. Vous ne pouvez pas utiliser cette capacité sur votre propre corps. Le pouvoir dure une heure ou jusqu'à ce que sa réserve de Puissance soit épuisée, selon la première éventualité. Action. (Si vous utilisez la télékinésie pour déplacer un objet à travers la pièce et qu'un humain de forme moyenne pourrait le faire avec ses bras, vous pouvez le faire avec votre psychokinésie. Vous devez utiliser uniquement la réserve de Puissance, l'Avantage de Puissance et l'Effort du la capacité Télékinésie) (Telekinesis \textendash (189))

\subsection*{Télépathie machine}\label{subsec:ab_machine_telepathy}

Vous pouvez lire les pensées superficielles d'une machine à courte portée, même si la machine ne le souhaite pas. Vous devez pouvoir voir la machine. Une fois le contact établi, vous pouvez lire les pensées de la cible pendant une minute maximum. Si vous ou la cible vous déplacez hors de portée, la connexion est rompue. Si vous possédez la capacité de lecture mentale, lorsque vous appliquez l'effort à la télépathie machine, vous gagnez un niveau d'effort gratuit. Action à initier. (Machine Telepathy \textendash (159))

\subsection*{Télépathie mécanique}\label{subsec:ab_mechanical_telepathy}

En touchant une machine pensante, vous accédez à ses « pensées » superficielles. Action. (Mechanical Telepathy \textendash (161))

\subsection*{Télépathique}\label{subsec:ab_telepathic}

Vous pouvez parler par télépathie avec d'autres personnes qui se trouvent à courte portée. La communication est bidirectionnelle, mais l'autre partie doit être disposée et capable de communiquer. Vous n'êtes pas obligé de voir la cible, mais vous devez savoir qu'elle est à portée. Vous pouvez avoir plusieurs contacts actifs à la fois, mais vous devez établir un contact avec chaque cible individuellement. Chaque contact dure jusqu'à dix minutes. Si vous appliquez un niveau d'Effort pour augmenter la durée plutôt que de faciliter la tâche, le contact dure 24 heures. Action pour établir le contact. (Telepathic \textendash (189))

\subsection*{Téléportation}\label{subsec:ab_teleportation}

vous vous transmettez instantanément à n'importe quel endroit que vous avez vu ou visité, quelle que soit la distance, tant qu'il se trouve sur Terre (ou dans le monde dans lequel vous vous trouvez actuellement). En plus des options normales d'utilisation d'Effort, vous pouvez choisir d'utiliser Effort pour amener d'autres personnes avec vous ; chaque niveau d'Effort utilisé de cette manière affecte jusqu'à trois cibles supplémentaires. Vous devez toucher toutes les cibles supplémentaires. Action. (Teleportation \textendash (190))

\subsection*{Téléportation courte}\label{subsec:ab_short_teleportation}

Vous vous téléportez instantanément vers n'importe quel endroit à une courte distance que vous pouvez voir. En plus des options normales d'utilisation de l'Effort, vous pouvez choisir d'utiliser l'Effort pour augmenter votre portée, vous téléporter vers un endroit que vous ne pouvez pas voir ou emmener d'autres personnes avec vous. Chaque courte distance supplémentaire coûte un niveau d'effort. Se téléporter vers une destination que vous ne pouvez pas voir coûte un niveau d'effort. Chaque cible supplémentaire apportée avec vous coûte un niveau d'Effort (vous devez toucher toute cible supplémentaire). Ces niveaux d'effort sont comptés séparément, donc se téléporter sur une courte distance supplémentaire vers un endroit que vous ne pouvez pas voir avec un seul passager coûte un total de trois niveaux d'effort. Action. Si vous disposez déjà de la téléportation courte lorsque vous sélectionnez Téléportation moyenne ou Téléportation, vous pouvez remplacer la téléportation courte par une autre capacité de type niveau 4. (Short Teleportation \textendash (183))

\subsection*{Téléportation en Ligne de Vue}\label{subsec:ab_jaunt}

Vous vous téléportez instantanément vers n'importe quel endroit à longue distance que vous pouvez voir. En plus des options normales d'utilisation de l'Effort, vous pouvez choisir d'utiliser l'Effort pour augmenter la distance que vous pouvez parcourir ; chaque niveau d'effort utilisé de cette manière augmente la portée de 100 pieds (30 m) supplémentaires. Action. (Jaunt \textendash (155))

\subsection*{Téléportation moyenne}\label{subsec:ab_medium_teleportation}

Vous vous téléportez instantanément vers n'importe quel endroit situé à une longue distance que vous pouvez voir. En plus des options normales d'utilisation de l'Effort, vous pouvez choisir d'utiliser l'Effort pour augmenter votre portée, vous téléporter vers un endroit que vous ne pouvez pas voir ou emmener d'autres personnes avec vous. Chaque longue distance supplémentaire coûte un niveau d'effort. Se téléporter vers une destination que vous ne pouvez pas voir coûte un niveau d'effort. Chaque ou deux cibles supplémentaires apportées avec vous coûtent un niveau d'effort (vous devez toucher toutes les cibles supplémentaires). Ces niveaux d'effort sont comptés séparément, donc se téléporter sur une longue distance supplémentaire vers un endroit que vous ne pouvez pas voir avec deux passagers coûte un total de trois niveaux d'effort. Action. (Medium Teleportation \textendash (161))

\
%--------------------------
\section*{U}

\subsection*{Ultra amélioration}\label{subsec:ab_ultra_enhancement}

vous gagnez +1 à l'armure et +5 à chacun de vos trois pools de statistiques. Facilitateur. (Ultra Enhancement \textendash (194))

\subsection*{Un coup du Destin}\label{subsec:ab_twist_of_fate}

L'expérience vous a beaucoup appris, notamment que parfois la chance est quelque chose que vous devez créer vous-même. Lorsque vous obtenez un 1, vous pouvez relancer. Vous devez utiliser le nouveau résultat, même s'il s'agit d'un autre 1. Enabler. (Twist of Fate \textendash (194))

\subsection*{Un sourire et un mot}\label{subsec:ab_a_smile_and_a_word}

lorsque vous utilisez l'Effort sur toute action impliquant des interactions, même celles liées à calmer des animaux ou à communiquer avec quelqu'un ou quelque chose dont vous ne parlez pas la langue, vous gagnez un niveau d'Effort gratuit sur la tâche. Action. (A Smile and a Word \textendash (108))

\subsection*{Une Bête comme Compagnon}\label{subsec:ab_beast_companion}

Une créature de niveau 2 de votre taille ou moins vous accompagne et suit vos instructions. Vous et le MJ devez définir les détails de votre créature, et vous ferez probablement des jets pour elle en combat ou lorsqu'elle entreprend des actions. Le compagnon bête agit à votre tour. En tant que créature de niveau 2, elle a un nombre cible de 6 et 6 points de vie et elle inflige 2 points de dégâts. Son mouvement est basé sur son type de créature (aviaire, nageur, etc.). Si votre compagnon bête meurt, vous pouvez chasser dans la nature pendant 1d6 jours pour en trouver un nouveau. Facilitateur. (Le niveau d'une créature détermine son nombre cible, sa santé et dommages, sauf indication contraire. Ainsi un compagnon bête de niveau 2 a un nombre cible de 6 et une santé de 6, et il inflige 2 points de dégâts. Un compagnon bête de niveau 4 a un nombre cible de 12 et une santé de 12, et il inflige 4 points de dégâts. Et ainsi de suite.) (Beast Companion \textendash (112))

\subsection*{Une information dans le réseau}\label{subsec:ab_network_tap}

Vous pouvez poser une question au MJ et obtenir une réponse très courte si vous réussissez un jet d'Intellect contre une difficulté assignée par le MJ. Plus la réponse est obscure, plus la tâche est difficile. En cas d'échec, un retour d'information ou peut-être une défense du réseau auquel vous accédez vous inflige 4 points de dégâts d'Intellect (ignore l'armure). Action. (En général, les connaissances que vous pourriez trouver en cherchant ailleurs que votre emplacement actuel sont de niveau 1, tandis que les connaissances obscures du passé sont de niveau 7. Acquérir des connaissances sur le futur est impossible.) (Network Tap \textendash (165))

\subsection*{Usurpation de Cypher}\label{subsec:ab_usurp_cypher}

choisissez un cypher que vous portez. Le cypher doit avoir un effet qui n'est pas instantané. Vous détruisez le cypher et obtenez son pouvoir, qui fonctionne pour vous en permanence. Vous pouvez choisir un cypher lorsque vous obtenez cette capacité, ou vous pouvez attendre et faire votre choix plus tard. Cependant, une fois que vous avez usurpé le pouvoir d'un cypher, vous ne pouvez plus passer à un autre cypher : la capacité d'usurpation ne fonctionne qu'une seule fois. Action à initier. (Usurp Cypher \textendash (195))

\subsection*{Usurper l'identité}\label{subsec:ab_impersonate}

pendant une heure, vous modifiez votre voix, votre posture et vos manières, créez un déguisement et obtenez un atout en tentant de vous faire passer pour quelqu'un d'autre, qu'il s'agisse d'un individu spécifique (Bob le flic) ou un rôle général (policier). Action à initier. (Impersonate \textendash (151))

\subsection*{Utilisation adroite des cyphers}\label{subsec:ab_adroit_cypher_use}

Vous pouvez utiliser quatre cyphers à la fois. Facilitateur. (Adroit Cypher Use \textendash (108))

\subsection*{Utilisation de l'environnement}\label{subsec:ab_using_the_environment}

Vous trouvez un moyen d'utiliser l'environnement à votre avantage lors d'un combat. Pendant les dix minutes suivantes, les jets d'attaque et les jets de défense de Célérité sont facilités. Action à initier. (Using the Environment \textendash (195))

\subsection*{Utilisation experte des cyphers}\label{subsec:ab_expert_cypher_use}

Vous pouvez utiliser trois chiffres à la fois. Facilitateur. (Expert Cypher Use \textendash (137))

\subsection*{Utilisation supplémentaire}\label{subsec:ab_extra_use}

Vous tentez d'obtenir une utilisation supplémentaire d'un artefact sans déclencher de jet d'épuisement. La difficulté de la tâche est égale au niveau de l'artefact. Si vous avez fabriqué l'artefact, vous gagnez un atout pour la tâche. En cas d'échec, le jet d'épuisement se déroule normalement. Vous pouvez également essayer d'utiliser un cypher manifeste sans le graver, mais la tâche est entravée. Une tentative infructueuse d'obtenir une utilisation supplémentaire d'un cypher manifeste le détruit avant qu'il ne puisse produire l'effet souhaité. Action. (Extra Use \textendash (138))

\subsection*{Utiliser ce qui est disponible}\label{subsec:ab_using_what's_available}

Si vous avez le temps et la liberté de rechercher des matériaux quotidiens dans votre environnement, vous pouvez créer un Atout temporaire qui vous aidera une fois à accomplir une tâche spécifique. Par exemple, si vous devez escalader un mur, vous pouvez créer une sorte de dispositif d'aide à l'escalade ; si vous avez besoin de sortir d'une cellule, vous pouvez trouver quelque chose à utiliser comme crochet de verrouillage ; si vous avez besoin de créer une petite distraction, vous pouvez assembler quelque chose pour produire un bruit fort et un flash ; et ainsi de suite. L'Atout dure au maximum une minute ou jusqu'à ce qu'il soit utilisé aux fins prévues. Une minute pour assembler les matériaux ; action pour créer un Atout. (Using What's Available \textendash (195))

\subsection*{Utiliser les sens des autres}\label{subsec:ab_use_senses_of_others}

Vous pouvez voir, entendre, sentir, toucher et goûter à travers les sens de toute personne avec laquelle vous avez un contact télépathique en utilisant des capacités télépathiques ou similaires. Vous pouvez tenter d'utiliser cette capacité sur une cible volontaire ou non à longue portée ; une cible réticente peut tenter de résister. Vous n'avez pas besoin de voir la cible, mais vous devez savoir qu'elle est à portée. Vos sens partagés durent dix minutes. Action à établir. (Use Senses of Others \textendash (195))

\
%--------------------------
\section*{V}

\subsection*{Vantardise flamboyante}\label{subsec:ab_flamboyant_boast}

Vous décrivez avec vantardise un acte que vous allez accomplir, puis, dans le cadre de la même action, vous le tentez. Si une personne moyenne trouve l'action difficile (voire impossible) et que vous la réussissez, les créatures qui en ont été témoins et qui ne sont pas vos alliés sont potentiellement étourdies lors de leur prochain tour, et toutes les tâches qu'elles tentent sont entravées. Le MJ vous aidera à déterminer si votre vantardise est quelque chose qui impressionnerait autant les spectateurs. Si vous tentez la tâche dont vous vous vantez mais que vous ne parvenez pas à l'accomplir, toutes vos tentatives pour affecter ou attaquer les spectateurs qui vous ont vu sont entravées pendant environ dix minutes. Facilitateur. (Flamboyant Boast \textendash (140))

\subsection*{Variante de Forme animale}\label{subsec:ab_animal_shape_variant}

si le concept de votre personnage est que vous prenez toujours la même forme animale au lieu de pouvoir choisir parmi plusieurs types d'animaux, doublez la durée de la capacité Forme animale (à vingt minutes par utilisation). Le MJ peut autoriser les personnages soumis à cette restriction à apprendre des formes animales supplémentaires en dépensant 4 XP comme avantage à long terme. Tableau des capacités mineures de la forme animale Utilisez ce qui suit comme exemples ou suggestions de ce qu'un personnage gagne lorsqu'il se présente sous la forme d'un animal. Si une forme animale répertorie deux compétences, le personnage choisit celle qu'il veut à chaque fois qu'il prend cette forme.
|Animal |Compétence entraînée |Autre capacité |
%|--------------------|------------------------|-----------------|
|Blaireau |Escalade |Odeur | |Chat |Escalade ou Furtivité |Petit |
|Chauve-Souris |Perception |Vol | |Cheval |Perception |Rapide |
|Crocodile |Furtivité ou Nager |Constriction  |
|Dauphin |Perception ou Nager |Rapide |
|Deinonychus |Perception |Rapide |
|Grenouille |Sauter ou Nager |Aquatique |
|Léopard |Escalade ou Furtivité |Rapide |
|Lézard |Escalade ou Furtivité |Petit |
|Oiseau |Perception |Vol |
|Ours |Escalade |Odeur |
|Poisson |Furtivité ou Nager |Aquatique |
|Poulpe |Furtivité |Aquatique |
|Requin |Nager |Aquatique |
|Sanglier |Défense de Puissance |Odeur |
|Serpent constricteur|Escalade |Constriction |
|Serpent venimeux |Escalade |Venin |
|Singe |Escalade |Mains |
|Tortue |Défense de Puissance |Armure |
|Loup |Perception |Odeur |
 * **Aquatique:** L'animal respire de l'eau au lieu de l'air ou est capable de respirer de l'eau en plus de respirer de l'air. * **Armure:** L'animal a une peau ou une carapace épaisse, accordant +1 à l'armure. Constriction**: L'animal peut saisir rapidement son adversaire après avoir effectué une attaque au corps à corps (généralement avec une morsure ou une griffe), facilitant ainsi les jets d'attaque contre cet ennemi dans les tours suivants jusqu'à ce qu'il le libère. * **Rapide** : L'animal peut se déplacer sur une longue distance pendant son tour au lieu d'une courte distance. * **Voler**: L'animal peut voler, ce qui (selon le type d'animal) peut parcourir une distance courte ou longue lors de son tour. * **Mains : L'animal a des pattes ou des mains presque aussi agiles que celles d'un humain. Contrairement à la plupart des formes d'animaux, les tâches de l'animal qui nécessitent des mains ne sont pas gênées (bien que le MJ puisse décider que certaines tâches nécessitant de l'agilité humaine, comme jouer de la flûte, sont toujours gênées). * **Odeur**: L'animal a un odorat fort, acquérant un atout pour le pistage et la gestion de l'obscurité ou de la cécité. * **Petit**: L'animal est considérablement plus petit qu'un humain, ce qui facilite ses tâches de défense contre la Célérité mais gêne les tâches de déplacement d'objets lourds. * **Venin** : L'animal est venimeux (généralement par morsure), lui infligeant 1 point de dégâts supplémentaire. (Animal Shape Variant \textendash (GF, 29)(CTS, 49))

\subsection*{Verrouiller une Serrure}\label{subsec:ab_lock}

Une porte, un portail, un coffre, un tiroir, un médaillon ou tout autre objet pouvant être fermé à longue portée se ferme et est verrouillé comme par magie (effet de niveau 3) pendant une heure. Si un objet ou une créature maintient physiquement l'objet cible ouvert, vous devez également réussir une Attaque basée sur l'intelligence. Pour chaque niveau d'Effort que vous appliquez, la qualité du verrou magique augmente d'un niveau. Action à initier. (Lock \textendash (159))

\subsection*{Vibration mortelle}\label{subsec:ab_lethal_vibration}

Vous créez une vibration mortelle dans votre propre corps et la transmettez à une créature que vous touchez avec une attaque réussie. Si la cible est de niveau 2 ou inférieur, elle meurt et explose dans un coup de tonnerre. Si la cible est de niveau 3 ou supérieur, elle subit 6 points de dégâts et est étourdie lors de sa prochaine action. Si la cible est un PJ de n'importe quel niveau, il descend d'un cran sur la piste des dégâts. En plus des options normales d'utilisation de l'Effort, vous pouvez choisir d'utiliser l'Effort pour affecter une cible plus puissante (un niveau d'Effort signifie qu'une cible jusqu'au niveau 3 explose et qu'une cible de niveau 4 ou plus subit des dégâts et est étourdie, et ainsi de suite). Action. (Lethal Vibration \textendash (158))

\subsection*{Vie en pleine nature}\label{subsec:ab_wilderness_life}

vous êtes entraîné dans deux des domaines suivants : l'escalade, la natation, la navigation ou l'identification des plantes et des créatures. Facilitateur. (Wilderness Life \textendash (199))

\subsection*{Vif d'esprit}\label{subsec:ab_quick_wits}

lorsque vous effectuez une tâche qui nécessiterait normalement de dépenser des points de votre réserve d'intelligence, vous pouvez dépenser des points de votre Réserve de Célérité à la place. Facilitateur. (Quick Wits \textendash (174))

\subsection*{Vigilance}\label{subsec:ab_vigilance}

Vous adoptez une approche prudente du combat, en vous concentrant davantage sur votre protection que sur le fait de blesser vos adversaires. Tant que cette capacité est active, vous gagnez un atout aux jets de défense Célérité contre les attaques de mêlée et à distance, et vos attaques de mêlée et à distance sont gênées. Cet effet dure aussi longtemps que vous le souhaitez, mais il prend fin si aucun combat n'a lieu à portée de vos sens. Action à initier. (Vigilance \textendash (196))

\subsection*{Vigilant}\label{subsec:ab_vigilant}

Lorsque vous êtes affecté par une attaque ou un effet qui pourrait vous étourdir ou vous étourdir, vous n'êtes ni étourdi ni étourdi. Facilitateur. (Vigilant \textendash (196))

\subsection*{Visualisation à distance}\label{subsec:ab_remote_viewing}

La distance est une illusion, car tout espace est un seul espace. Avec une grande concentration, vous pouvez voir un autre endroit. Cette capacité peut être utilisée de deux manières : - Distance et direction. Choisissez un endroit à une distance spécifique et dans une direction spécifique. Vous pouvez voir depuis ce point d'observation comme si vous y aviez utilisé la capacité Capteur, mais seulement pendant une minute. - Pensez à un endroit que vous avez déjà vu, soit de manière conventionnelle, soit en utilisant une autre application de ce pouvoir. Vous pouvez voir depuis ce point d'observation comme si vous y aviez utilisé la capacité Capteur, mais seulement pendant un ou deux tours. L'une ou l'autre application prend entre une action et des heures de concentration, selon ce que le MJ estime approprié en raison du temps, de la distance ou d'autres circonstances atténuantes. Cependant, vous ne savez pas à l'avance combien de temps cela prendra. Action à initier ; action à chaque tour pour se concentrer. (Remote Viewing \textendash (176))

\subsection*{Visée mortelle}\label{subsec:ab_deadly_aim}

Pendant la minute suivante, toutes les attaques à distance que vous effectuez infligent 2 points de dégâts supplémentaires. Action à initier. (Deadly Aim \textendash (125))

\subsection*{Visée prudente}\label{subsec:ab_careful_aim}

Vous êtes entraîné aux attaques avec toutes les armes que vous lancez. Facilitateur. (Careful Aim \textendash (118))

\subsection*{Vitalité sauvage}\label{subsec:ab_wild_vitality}

Vous vous connectez à la force vitale d'une créature naturelle (de votre taille ou plus) à longue portée que vous pouvez voir. Il s'agit d'une tâche Intellect de niveau 2. Si vous réussissez, la créature n'est pas blessée, mais grâce à la résonance avec sa vitalité sauvage, vous bénéficiez de plusieurs avantages pendant une minute maximum : un atout pour toutes vos tâches basées sur la Puissance (y compris les attaques et les défenses), +2 pour votre Avantage de Puissance. et Avantage de Célérité, et 2 points de dégâts supplémentaires sur toutes les attaques de mêlée réussies. Action à initier. (Wild Vitality \textendash (198))

\subsection*{Vitesse de course incroyable}\label{subsec:ab_incredible_running_speed}

vous vous déplacez beaucoup plus loin que la normale au cours d'un round. Cela signifie que dans le cadre d'une autre action, vous pouvez vous déplacer sur une longue distance. En tant qu'action, vous pouvez vous déplacer jusqu'à 200 pieds (60 m) ou jusqu'à 500 pieds (150 m) dans le cadre d'une tâche basée sur la Célérité avec une difficulté de 4. Enabler. (Incredible Running Speed \textendash (153))

\subsection*{Vitesse floue}\label{subsec:ab_blurring_speed}

Vous vous déplacez si rapidement que jusqu'à votre prochain tour, vous êtes flou pour le sautres. Pendant que vous êtes flou, si vous appliquez un effort à une tâche d'attaque au corps à corps ou à une tâche de défense rapide, vous obtenez un niveau d'effort gratuit sur cette tâche ; vous pouvez vous déplacer sur une courte distance dans le cadre d'une autre action ou sur une longue distance dans le cadre de l'ensemble de votre action. Facilitateur. (Blurring Speed \textendash (115))

\subsection*{Vivre de la terre}\label{subsec:ab_living_off_the_land}

En une heure environ, vous pouvez toujours trouver de la nourriture comestible et de l'eau potable dans la nature. Vous pouvez même en trouver suffisamment pour un petit groupe de personnes, si besoin est. De plus, comme vous êtes très robuste et que vous avez acquis une résistance au fil du temps, vous êtes entraîné à résister aux effets des poisons naturels (comme ceux des plantes ou des créatures vivantes). Vous êtes également immunisé contre les maladies naturelles. Facilitateur. (Living Off the Land \textendash (158))

\subsection*{Voilà votre problème}\label{subsec:ab_there's_your_problem}

vous êtes entraîné à des tâches liées à la résolution de problèmes avec plusieurs solutions (comme la meilleure façon de préparer un camion, de calmer un client enragé, de donner une injection d'insuline à un chat ou de trouver un itinéraire à travers la ville). pour une Célérité maximale). Facilitateur. (There's Your Problem \textendash (190))

\subsection*{Voir Historique}\label{subsec:ab_see_history}

Vous touchez un objet, lisez les échos subtils de son existence à travers le temps, posez au MJ une question sur le passé de l'objet et obtenez une réponse générale. Les réponses se présentent souvent sous la forme de brèves images ou de sensations plutôt que de réponses spécifiques dans une langue que vous connaissez. Le MJ attribue un niveau à la question, donc plus la réponse est obscure, plus la tâche est difficile. » Généralement, les connaissances que vous pourriez trouver en regardant ailleurs que votre position actuelle sont de niveau 1, et les connaissances obscures du passé sont de niveau 7. Après avoir utilisé cette capacité, vous disposez d'un atout pour identifier l'objet. Action. (« Obscur » est un terme relatif : un sage peut ne pas savoir comment un vampire a acquis un artefact spécifique, mais quelqu'un utilisant Voir l'historique sur cet artefact aurait plus de facilité à ressentir cet événement.) (See History \textendash (180))

\subsection*{Voir dans le Noir}\label{subsec:ab_zero_dark_eyes}

Les yeux de certaines personnes sont dégradés par le fait de jouer constamment à des jeux. Et peut-être que cela vous arrivera, mais pas encore. Vous êtes encore jeune et au lieu de vous dégrader, votre vision est en réalité meilleure grâce à toute votre pratique. Vous pouvez voir dans une lumière très faible comme s'il s'agissait d'une lumière vive. Vous pouvez voir dans l'obscurité totale comme s'il s'agissait d'une lumière très faible. Facilitateur. (Zero Dark Eyes \textendash (200))

\subsection*{Voir l'avenir}\label{subsec:ab_see_the_future}

Sur la base de toutes les variables que vous percevez, vous pouvez prédire les prochaines minutes. Cela a les effets suivants :- Pendant les dix minutes suivantes, vos jets de défense gagnent un atout. - Vous avez une sorte de sens du danger. Pendant les dix minutes suivantes, vous gagnez un atout en détectant les tromperies et les tentatives de vous trahir, ainsi qu'en évitant les pièges et les embuscades. - Vous savez ce que les gens pensent probablement et ce qu'ils diront avant de le dire. Pendant les dix minutes suivantes, vous gagnez un atout pour les tâches impliquant interaction et tromperie. Facilitateur. (See the Future \textendash (180))

\subsection*{Voir l'invisible}\label{subsec:ab_see_the_unseen}

vous pouvez automatiquement percevoir des créatures et des objets qui sont normalement invisibles, déphasés ou seulement partiellement dans cet univers. Lorsque l'on recherche des choses cachées de manière plus conventionnelle, la tâche est facilitée. Facilitateur. (See the Unseen \textendash (180))

\subsection*{Voir à travers la matière}\label{subsec:ab_see_through_matter}

Vous pouvez voir à travers la matière comme si elle était transparente. Vous pouvez voir à travers jusqu'à 6 pouces (15 cm) de matériau pendant un tour. Il s'agit d'une tâche dont la difficulté est égale au niveau du matériau ou de l'objet. En plus des options normales d'utilisation de l'Effort, vous pouvez choisir d'utiliser l'Effort pour voir à travers 6 pouces supplémentaires de matériau pour chaque niveau d'Effort supplémentaire que vous appliquez pour atteindre cet objectif. Action. (See Through Matter \textendash (180))

\subsection*{Voir à travers le temps}\label{subsec:ab_see_through_time}

Le temps est une illusion, car tout le temps est une seule fois. Avec une grande concentration, vous pouvez voir une autre époque. Vous spécifiez une période de temps concernant l'endroit où vous vous trouvez actuellement. Il est intéressant de noter que la période la plus facile à visualiser se situe environ cent ans dans le passé ou dans le futur. Regarder plus loin ou plus loin est une tâche presque impossible. Cela prend entre une action et des heures de concentration, selon ce que le MJ estime approprié en raison du temps, de la distance ou d'autres circonstances atténuantes. Cependant, vous ne savez pas à l'avance combien de temps cela prendra. Action à initier ; action à chaque tour pour se concentrer. (See Through Time \textendash (181))

\subsection*{Vol}\label{subsec:ab_flight}

Vous pouvez flotter et voler dans les airs pendant une heure. Pour chaque niveau d'Effort appliqué, vous pouvez affecter une créature supplémentaire de votre taille ou moins. Vous devez toucher la créature pour lui conférer le pouvoir de voler. Vous dirigez le mouvement de l'autre créature, et pendant qu'elle vole, elle doit rester en vue ou tomber. En termes de mouvement terrestre, une créature volante se déplace à environ 32 km par heure et n'est pas affectée par le terrain. Action à initier. (Flight \textendash (141))

\subsection*{Vol Court}\label{subsec:ab_flight_exertion}

vous pouvez voler sur une courte distance lors de votre mouvement ce tour. Si tout ce que vous faites est de vous déplacer pendant votre tour, vous pouvez voler sur une longue distance. Facilitateur. (Flight Exertion \textendash (141))

\subsection*{Vol électrique}\label{subsec:ab_electrical_flight}

Vous dégagez une aura d'électricité crépitante qui vous permet de voler sur une longue distance à chaque tour pendant dix minutes. Vous ne pouvez pas transporter d'autres créatures avec vous. Action à activer. (Electrical Flight \textendash (133))

\subsection*{Voler le pouvoir}\label{subsec:ab_steal_power}

lorsque vous utilisez Pouvoir de copie pour copier une capacité, la créature à partir de laquelle vous l'avez copiée perd l'accès à cette capacité pendant environ une minute. Tant que vous possédez sa capacité, toute tentative de la créature d'utiliser sa capacité nécessite qu'elle réussisse une tâche (Puissance, Célérité ou Intellect, selon la capacité volée) opposée à votre tâche d'Intellect facilitée. S'ils réussissent, ils retrouvent l'usage de leur capacité et vous la perdez. Facilitateur. Si vous souhaitez rendre plus difficile la récupération du pouvoir volé par quelqu'un, devenez compétent dans la capacité Voler le pouvoir ou modifiez le pouvoir en conséquence. (Steal Power \textendash (CTS, 56))

\subsection*{Voleur de rêves}\label{subsec:ab_dream_thief}

Vous volez un rêve précédent à une créature vivante à courte portée. La créature perd 2 points d'Intellect (ignore l'Armure) et vous apprenez quelque chose que le MJ choisit de révéler sur la créature : sa nature, une partie de ses plans, un souvenir, etc. Action. (Dream Thief \textendash (132))

\subsection*{Volonté d'un leader}\label{subsec:ab_will_of_a_leader}

Vous renforcez le dévouement et les capacités de vos alliés. Chaque allié à portée immédiate gagne +1 Edge pour une statistique de son choix pendant une heure. Vous bénéficiez également de cet avantage pour une statistique de votre choix. Action. (Will of a Leader \textendash (199))

\subsection*{Volonté de Légende}\label{subsec:ab_will_of_legend}

Vous êtes immunisé contre les attaques qui pourraient captiver, hypnotiser, charmer ou autrement influencer votre esprit. Facilitateur. (Will of Legend \textendash (199))

\subsection*{Volonté du Chasseur}\label{subsec:ab_hunters_drive}

Par la force de la volonté, lorsque vous le souhaitez, vous vous accordez de plus grandes prouesses dans la chasse pendant dix minutes. Pendant ce temps, vous gagnez un atout pour toutes les tâches impliquant votre proie, y compris les attaques. Votre proie est la créature que vous avez sélectionnée avec votre capacité Proie. Facilitateur. (Hunter's Drive \textendash (149))

\subsection*{Vous avez une Intuition}\label{subsec:ab_got_a_feeling}

Vous avez une intuition étrange lorsqu'il s'agit de trouver des choses. Pendant l'exploration, vous pouvez étendre vos sens jusqu'à 1,5 km dans n'importe quelle direction et poser au MJ une question générale très simple (généralement une question par oui ou par non) à propos de cette zone, telle que « Y a-t-il un orc ? campement à proximité ? ou "Y a-t-il de la matière noire dans cette carcasse rouillée?" Si la réponse que vous cherchez ne se trouve pas dans la zone, vous ne recevez aucune information. Action. (Got a Feeling \textendash (145))

\subsection*{Vous avez étudié}\label{subsec:ab_you_studied}

Pour pouvoir mettre deux et deux ensemble et parvenir à une déduction, vous devez connaître certaines choses. Vous êtes entraîné dans deux domaines de connaissances de votre choix (à condition qu'ils ne soient pas liés à des actions physiques ou au combat) ou spécialisé dans un domaine. Facilitateur. (You Studied \textendash (0))

\subsection*{Voyage dans le temps}\label{subsec:ab_time_travel}

Vous et jusqu'à trois personnages volontaires que vous choisissez à portée immédiate voyagez jusqu'à un moment que vous spécifiez lorsque vous utilisez cette capacité. Le moment doit se situer dans les dix ans actuels. Pour chaque niveau d'Effort appliqué, vous pouvez voyager dix années supplémentaires ou amener trois créatures supplémentaires avec vous. Lorsque vous apparaissez dans le nouveau moment, vous le faites dans la même position que lorsque vous avez utilisé cette capacité. En arrivant à votre destination temporelle, vous et les autres voyageurs temporels êtes assommés pendant une minute. Afin de revenir à votre époque d'origine, vous devez à nouveau utiliser cette capacité. Action. (Time Travel \textendash (192))

\subsection*{Voyage dans les arbres}\label{subsec:ab_tree_travel}

Vous entrez dans un arbre et émergez instantanément et en toute sécurité d'un autre sur une longue distance. Vous n'avez pas besoin de préciser de quel arbre vous sortez (si vous savez qu'il y a des arbres dans cette direction, vous pouvez décider jusqu'où aller et vous sortirez d'un arbre dans cette zone). Si le tronc de l'arbre de départ n'est pas aussi grand que votre corps, vous devez appliquer un niveau d'effort pour y entrer. Vous pouvez choisir d'utiliser l'Effort pour augmenter la distance parcourue ; un niveau d'effort utilisé de cette manière augmente la portée à très longue, deux niveaux l'élèvent à un mile (1,5 km) et chaque niveau d'effort supplémentaire au-delà de cela l'augmente d'un mile supplémentaire. Action. (Tree Travel \textendash (194))

\subsection*{Voyage rapide}\label{subsec:ab_fast_travel}

Vous déformez le temps et l'espace de sorte que vous et jusqu'à dix autres créatures à proximité immédiate voyagez par voie terrestre à une Célérité dix fois supérieure à la normale pendant huit heures maximum. À cette Célérité, les rencontres les plus dangereuses ou les régions de terrain accidenté sont ignorées, bien que le MJ puisse déclarer des exceptions. Les barrières pures et simples constituent toujours un problème. Action à initier. (Fast Travel \textendash (139))

\subsection*{Vrilles de feu}\label{subsec:ab_fire_tendrils}

Lorsque vous le souhaitez, votre halo (issu de votre capacité Manteau de flammes) fait germer trois vrilles de flammes qui durent jusqu'à dix minutes. Par une action, vous pouvez utiliser les vrilles pour attaquer, en effectuant un jet d'attaque distinct pour chacune. Chaque vrille inflige 4 points de dégâts. Sinon, les attaques fonctionnent comme des attaques standards. Si vous n'utilisez pas les vrilles pour attaquer, elles restent mais ne font rien. Facilitateur. (Fire Tendrils \textendash (140))

\subsection*{Vulnérabilités des machines}\label{subsec:ab_machine_vulnerabilities}

Vous infligez 3 points de dégâts supplémentaires aux robots et animez des machines en tout genre. Facilitateur. (Machine Vulnerabilities \textendash (159))

\subsection*{Véritable Gardien}\label{subsec:ab_true_guardian}

Lorsque vous montez la garde lors de votre action, les alliés à portée immédiate de vous gagnent un atout pour leurs tâches de défense. Cela dure jusqu'à la fin de votre prochain tour. Facilitateur. (True Guardian \textendash (194))

\subsection*{Véritable Nécromancie}\label{subsec:ab_true_necromancy}

Cette capacité fonctionne comme la capacité Nécromancie sauf qu'elle crée une créature de niveau 5. Action à animer. (True Necromancy \textendash (194))

\subsection*{Véritable défenseur}\label{subsec:ab_true_defender}

cette capacité fonctionne comme la capacité de Défenseur dévoué, sauf que l'avantage s'applique à un maximum de trois personnages de votre choix. Si vous choisissez un seul personnage, vous devenez spécialisé dans les tâches décrites sous la capacité Défenseur dévoué. Action à initier. (True Defender \textendash (194))

\subsection*{Véritables sens}\label{subsec:ab_true_senses}

vous pouvez voir dans l'obscurité totale jusqu'à 15 m, comme s'il s'agissait d'une faible lumière. Vous reconnaissez les hologrammes, les déguisements, les illusions d'optique, le mimétisme sonore et autres astuces similaires (pour tous les sens) pour ce qu'ils sont. Facilitateur. (True Senses \textendash (194))

\
%--------------------------
\section*{Y}

\subsection*{Yeux ajustés}\label{subsec:ab_eyes_adjusted}

vous pouvez voir dans une lumière extrêmement faible comme s'il s'agissait d'une lumière vive. Vous pouvez voir dans l'obscurité totale comme s'il s'agissait d'une lumière extrêmement faible. Facilitateur. (Eyes Adjusted \textendash (138))

\subsection*{Yeux de bête}\label{subsec:ab_beast_eyes}

En vous connectant à la créature grâce à votre capacité Compagnon de bête, vous pouvez percevoir par ses sens si elle se trouve à moins de 1,5 km de vous. Cet effet dure jusqu'à dix minutes. Action à établir. (Beast Eyes \textendash (112))

\
%--------------------------



%#######################################################################
%            CHAPTER 10
%#######################################################################
\chapter{\uppercase{équipement}\label{ch:chapter10}}
\fancypagestyle{plain}{ %
\fancyhf{} % remove everything
\renewcommand{\headrulewidth}{1pt} % remove lines as well
\renewcommand{\footrulewidth}{0pt}
\xpretocmd\headrule{\color{BlueViolet}}{}{\PatchFailed}
\fancyhead[RO]{\textcolor{BlueViolet}{\uppercase{é}quipement}}
\fancyhead[LE]{\textcolor{BlueViolet}{CYPHER SYSTEM}}
}

\backmatter
\end{document}