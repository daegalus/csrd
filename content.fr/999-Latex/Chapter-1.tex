\documentclass[a4paper,french,twocolumn,twoside]{report}
\usepackage[T1]{fontenc}    % Encodage T1 (adapté au français)
\usepackage{lmodern}        % Caractères plus lisibles
\usepackage{babel}          % Réglages linguistiques (avec french)
\pagestyle{empty}           % N'affiche pas de numéro de page
\usepackage[explicit]{titlesec}  % pour modifier le style de chapitre
\usepackage{color}        % pour les coleurs de police
\usepackage{graphicx}     % pour inclure des images
\usepackage{wrapfig}
\usepackage[dvipsnames]{xcolor} % pour les couleurs
\usepackage{fancyhdr} % pour les entete

%%  Default Font
\usepackage[sfdefault]{cabin}
\renewcommand*\familydefault{\sfdefault} %% Only if the base font of the document is to be sans serif
\usepackage[T1]{fontenc}

\definecolor{darkblue}{rgb}{0.12,0.47,0.87}
\definecolor{lightpurple1}{rgb}{0.141,0.29,0.117}
 
%%################################################
% https://distrib-coffee.ipsl.jussieu.fr/pub/mirrors/ctan/macros/latex/contrib/titlesec/titlesec.pdf
% https://borntocode.fr/latex-personnaliser-les-titres-chapter/
%% define format for chapter
\titleformat
  {\chapter} % command
  [display] % shape
  {\bfseries\Large\centering} % format
  {
    \textcolor{RedViolet}{
      Chapitre \ \thechapter \\ \huge #1
      }
  } % label
  {0.5ex} % sep
  {
      \centering
  } % before-code
  [
  \centering
  ] % after-code
\titlespacing*{\chapter}
  {0mm} %left
  {-0pt} %before-step
  {0mm} %after-step
  {} %right
\titleclass{\chapter}{top}
%%###############################################
\usepackage[
  a4paper,
  inner=2cm,
  top=2cm,
  bottom=2cm,
  marginparwidth=50cm
  %textwidth=345pt,
]{geometry}

%-------------------------------------
%  définition des en-tête
\setlength{\headheight}{15.2pt}
\pagestyle{fancy}
\fancyhead[LE]{CYPHER SYSTEM}

\fancypagestyle{plain}{ %
  \fancyhf{} % remove everything
  \renewcommand{\headrulewidth}{1pt} % remove lines as well
  \renewcommand{\footrulewidth}{1pt}
  \fancyhead[RO]{DES MONDES D'AVENTURES}
  \fancyhead[LE]{CYPHER SYSTEM}
}
%%################################################
\begin{document}
\thispagestyle{plain}
%%------------------------------------------------
\chapter{DES MONDES D'AVENTURES\label{ch:chapter1}}
Au final, tout ce que nous voulons, c'est précisément de jouer au jeu auquel nous voulons jouer. Toutes les Meneuses et Meneurs ont un cadre de campagne parfait dans un coin de leur tête. Les joueuses et joueurs ont cette idée de personnage qui serait leur meilleur personnage jamais créé, si seulement ils avaient la chance de le créer et de le jouer. Ces rêves de jouer exactement ce que vous souhaitez jouer sont la raison d'être de ce livre.

Il s'agit d'une version révisée du livre de règles original du Cypher System. J'ai réuni le contenu de ce livre à partir des jeux du Cypher System existant à l'époque — Numenera et The Strange. C'était essentiellement une compilation de tout ce matériel de jeu, ainsi que beaucoup de suggestions sur la façon de l'utiliser de la manière que vous souhaitez. Les joueurs et les Meneuses nous ont dit que cela répondait bien à ces besoins.

Nous avons beaucoup appris depuis. Pas tellement sur les règles du système elles-mêmes — qui restent essentiellement inchangées — mais sur la manière dont nous voulons utiliser ce type de livre, et donc sur la manière de présenter l'information. Il est vraiment difficile de créer quelque chose qui soit utilisable par n'importe qui pour n'importe quoi et de le présenter de manière réellement conviviale. Mais je pense que c'est exactement ce que nous avons fait avec ce livre. Les innovations que vous trouverez dans ces pages — la façon dont toutes les capacités ont été cataloguées pour que vous puissiez les utiliser comme bon vous semble, l'accent mis sur les cyphers subtils, l'étendue des genres présentés — rendent ce matériel plus facile à utiliser et plus facile à personnaliser.

Le tout nouveau \marginpar{margin text} contenu, comme le système d'arc narratif, le système d'artisanat, les informations supplémentaires sur les genres, etc., rendra, je l'espère, vos parties plus amusantes et vos histoires plus riches.

Mais permettez-moi de répéter : nous n'avons pas changé la façon dont le jeu fonctionne. Vous pouvez utiliser ce livre en parallèle avec l'ancien livre de règles du Cypher System sans trop de problèmes.

D'une certaine manière, ce livre est un volume complémentaire à un livre que j'ai écrit intitulé Your Best Game Ever. Ce livre est un guide indépendant du système de jeux pour comprendre et apprécier les jeux de rôle. Le présent ouvrage prend les idées et suggestions présentées dans ce dernier et leur donne un ensemble de règles qui les rend possibles. Mon objectif est de vous donner les outils pour avoir votre meilleur jeu jamais joué. Et cela, je crois, implique de pouvoir jouer dans le cadre et avec les personnages que vous avez toujours voulu.

Maintenant, espérons-le, vous pouvez enfin le faire.

Amusez-bien

\end{document}