\documentclass[a4paper,french,twocolumn,twoside]{report}
\usepackage[T1]{fontenc}    % Encodage T1 (adapté au français)
\usepackage{lmodern}        % Caractères plus lisibles
\usepackage{babel}          % Réglages linguistiques (avec french)
\pagestyle{empty}           % N'affiche pas de numéro de page
\usepackage[explicit]{titlesec}  % pour modifier le style de chapitre
\usepackage{color}        % pour les coleurs de police
\usepackage{graphicx}     % pour inclure des images
\usepackage{wrapfig}
\usepackage[dvipsnames]{xcolor} % pour les couleurs
\usepackage{fancyhdr} % pour les entete

%%  Default Font
\usepackage[sfdefault]{cabin}
\renewcommand*\familydefault{\sfdefault} %% Only if the base font of the document is to be sans serif
\usepackage[T1]{fontenc}

\definecolor{darkblue}{rgb}{0.12,0.47,0.87}
\definecolor{lightpurple1}{rgb}{0.141,0.29,0.117}
 
%%################################################
% https://distrib-coffee.ipsl.jussieu.fr/pub/mirrors/ctan/macros/latex/contrib/titlesec/titlesec.pdf
%% define format for chapter
\titleformat
  {\chapter} % command
  [display] % shape
  {\bfseries\Large\centering} % format
  {\textcolor{RedViolet}{Chapitre \ \thechapter \\ \huge #1}} % label
  {0.5ex} % sep
  {
      \centering
  } % before-code
  [
  \centering
  ] % after-code
\titlespacing*{\chapter}
  {0mm} %left
  {-0pt} %before-step
  {0mm} %after-step
  {} %right
\titleclass{\chapter}{top}
%-----------------------------------------------------
%% define format for section
\titleformat
  {\section} % command
  [display] % shape
  {\bfseries\Large} % format
  {\textcolor{RedViolet}{ #1}} % label
  {0.5ex} % sep
  {
    \centering
  } % before-code
  [
  \centering
  ] % after-code

\titlespacing*{\section}
  {0mm} %left
  {-0pt} %before-step
  {0mm} %after-step
  {} %right
\titleclass{\section}{top}
%%###############################################
\usepackage[
  a4paper,
  inner=2cm,
  top=2cm,
  bottom=2cm,
  marginparwidth=50cm
  %textwidth=345pt,
]{geometry}

%-------------------------------------
%  définition des en-tête
\setlength{\headheight}{15.2pt}
\pagestyle{fancy}
\fancyhead[LE]{CYPHER SYSTEM}

\fancypagestyle{plain}{ %
  \fancyhf{} % remove everything
  \renewcommand{\headrulewidth}{1pt} % remove lines as well
  \renewcommand{\footrulewidth}{1pt}
  \fancyhead[RO]{TOUT EST PERMIS}
  \fancyhead[LE]{CYPHER SYSTEM}
}
%%################################################
\begin{document}
\thispagestyle{plain}
%%------------------------------------------------
\chapter{TOUT EST PERMIS \label{ch:chapter2}}\nobreak
%%------------------------------------------------
%%    image de chapitre
\begin{figure}
\hspace{-72pt}\includegraphics[height=8.25cm]{CS-page5-01.png}
\end{figure}
%%------------------------------------------------

\chapter{TOUT EST PERMIS \label{ch:chapter2}}

\section*{CADRES DE CAMPAGNE}\nobreak
Bien que les genres soient des catégories utiles pour organiser vos idées, ce que vous allez réellement créer, c'est un cadre. Des étiquettes comme « science-fiction » ou « space opera » sont pratiques, mais au final, ce qui compte, c'est le cadre spécifique que vous établissez.
Votre cadre—qu'il s'agisse d'une création originale ou d'une adaptation—vous appartient entièrement. Ne vous inquiétez pas de ce que d'autres pourraient considérer comme approprié pour un genre donné. Une fois que vous commencez à assembler votre cadre, vous voudrez peut-être parcourir à nouveau les sections sur la création de personnages dans ce livre. Ce qui est habituellement adapté à un genre fantastique, par exemple, peut ne pas convenir à votre propre univers de fantasy.
Imaginons que, dans votre monde, la magie du feu soit toujours maléfique et uniquement pratiquée par des prêtres possédés par des démons. Dans ce cas, l'axe Porte une auréole de feu ne serait pas approprié pour les personnages des joueurs, même s'il est parfaitement acceptable dans d'autres jeux de fantasy.
Plus vous définissez précisément les détails de votre cadre, plus il sera facile d'en ajuster les éléments. Et plus votre univers s'éloigne des clichés du genre, plus vous devrez adapter les choix possibles. Mais ce n'est pas un problème : les cadres spécifiques et distincts sont souvent les plus amusants, les plus mémorables et les plus engageants pour vos joueurs. Ils valent largement l'effort supplémentaire.

\section*{ADAPTER LES RÈGLES}\nobreak
Parfois, vous devez modifier certaines choses pour qu'elles correspondent à vos besoins et envies. Prenons par exemple la saveur « magie » que vous pouvez attribuer à n'importe quel type présenté dans le chapitre 5. Elle s'appelle « magie » et possède de nombreux éléments associés à ce concept, mais il serait très simple de changer son nom en « psionique », « pouvoirs mutants » ou tout autre terme adapté à votre univers.
En d'autres termes, sélectionner du contenu dans ce livre peut ne pas suffire. Vous pourriez avoir besoin d'ajuster certains éléments ici et là. Heureusement, la plupart du matériel est conçu pour être modifié ou adapté. En fait, grâce à la simplicité des mécaniques de base du Cypher System, effectuer des ajustements est un jeu d'enfant. Ce n'est pas un système où un petit changement risque d'entraîner un effet domino aux conséquences imprévues.
Dans le chapitre 7, vous trouverez des directives pour créer de nouveaux descripteurs. Le chapitre 8 contient une section entière consacrée à la création de nouveaux axes adaptés à votre propre jeu. De plus, les types de personnages du chapitre 5 sont conçus pour être personnalisés et remodelés. 
Lorsque vous apportez des modifications, concentrez-vous moins sur l'équilibrage du jeu et davantage sur la narration des histoires que vous souhaitez raconter, tout en permettant aux joueurs de créer et d'incarner les personnages qu'ils veulent. Si vous parvenez à faire ces deux choses correctement, tout le monde sera satisfait. Et au final, c'est précisément ce qu'est l'équilibrage du jeu. 
Vous pouvez également consulter le chapitre 25 pour approfondir la façon de modifier les mécaniques du jeu. Mais dans l'ensemble, ce chapitre vous rappellera ce que vous venez de lire : c'est *votre* jeu, et vous êtes libre d'en faire ce que vous voulez.


\end{document}