    %#######################################################################
    %            CHAPTER 2
    %#######################################################################
    \chapter{Tout est permis}
\label{ch:chapter2}
    %---------------------------
    % https://texdoc.org/serve/fancyhdr/0
    \fancypagestyle{plain}{ %
        \fancyhf{} % remove everything
        \renewcommand{\headrulewidth}{1pt} % remove lines as well
        \renewcommand{\footrulewidth}{0pt}
        \xpretocmd\headrule{\color{RedViolet}}{}{\PatchFailed}
        \fancyhead[RO]{\textcolor{RedViolet}{textsc{Tout est permis}}}
        \fancyhead[LE]{\textcolor{RedViolet}{CYPHER SYSTEM}}
    }
    \chapterfirstletter{P}{RedViolet}
    our commencer, nous nous adressons directement aux meneuses (ou MJ). Les joueurs comme les MJ utiliseront ce livre, mais il est fort probable que ce soit d'abord le MJ qui le consulte.

    Ce que vous tenez entre vos mains est un guide. Un mode d'emploi. Vous ne pouvez pas simplement vous asseoir et commencer à jouer, car le manuel du Cypher System n'est pas conçu pour être utilisé de cette façon. Il vous faut d'abord y mettre quelque chose de votre propre invention. Il n'y a pas de cadre ni de monde prédéfinis ici. Le système est conçu pour vous aider à dépeindre n'importe quel monde ou cadre que vous pouvez imaginer.

    Considérez ce livre comme un coffre à jouets. Vous pouvez sortir ce que vous voulez et y jouer comme bon vous semble. Vous n'utiliserez pas tout ce qu'il contient, du moins pas d'un seul coup. Vous utiliserez des parties de ce livre pour construire le jeu que vous souhaitez jouer. Sortez quelques éléments et essayez-les. Remettez en place ceux qui ne vous conviennent pas, et essayez d'autres. Utilisez certains maintenant et gardez-en d'autres pour votre prochaine partie. Vous avez toute la liberté possible (en fait, de nombreux mondes).

    A propos de mondes, vous devez décider quel cadre de campagne utiliser, en fonction du genre que vous avez choisi. Cela peut être n'importe quoi. Prenez votre livre ou film préféré, ou concevoir quelque chose à partir de rien.

    Alors, en pratique, ce que vous choisissez ici, c'est l'expérience que vous voulez vivre\textendash et que vous voulez faire vivre aux joueurs. C'est une décision tellement fondamentale que tout le groupe devrait peut-être y participer. Demandez aux autres joueurs quel genre ils aiment et quels types d'expériences ils souhaitent vivre. C'est essentiel, car cela garantit que tout le monde obtient ce qu'il attend du jeu.

    Bien sûr, tout le contenu de ce livre ne convient pas à tous les genres. Vous, en tant que MJ, devrez le lire une fois que vous aurez choisi un genre et sélectionner les types, les axes et ainsi de suite. Ensuite, informez vos joueurs du matériel que vous avez décidé de rendre disponible afin qu'ils puissent créer des personnages adaptés au genre


    \section*{GENRES}
    Jeter un coup d'œil à la 3ième partie Genre qui contient un nombre de chapitres consacrés aux genres. Ce sont des catégories assez larges, et nous les utiliserons dans ce livre comme point de départ. Ces catégories sont : Fantasy, moderne, science-fiction, hoerreur, romance, super-héros, post-apocalyptique, contes de fées, et historique.

    Avec ces descriptions assez génériques, nous pouvons couvrir la plupart des cadres de jeu (mais probablement pas tous)  que vous pouvez jouer avec le Cypher System. Certains de ces genres nécessite du matériel unique, des artifacts, ou des descripteurs. Certains ont besoin de nouvelles règles pour améliorer l'immersion que vous recherchez.

    Nous parlons d'immersion, parce que sous beaucoup d'aspect, c'est ce qu'un genre est. Si vous voulez faire vivre l'expérience d'être terrifié par des zombies qui rodent autour de la maison de votre personnage, vous voulez de l'horreur. Si vous voulez faire vivre l'expérience d'être extrêmement puissant et utiliser ces pouvoirs pour protéger le monde des extra-terrestres, vous voulez des super-héros (avec peut-être une touche de science-fiction).


    \section*{CADRES DE CAMPAGNE}
    Bien que les genres soient des catégories utiles pour organiser vos idées, ce que vous allez réellement créer, c'est un cadre. Des étiquettes comme « science-fiction » ou « space opera » sont pratiques, mais au final, ce qui compte, c'est le cadre spécifique que vous établissez.

    Votre cadre \textendash qu'il s'agisse d'une création originale ou d'une adaptation\textendash vous appartient entièrement. Ne vous inquiétez pas de ce que d'autres pourraient considérer comme approprié pour un genre donné. Une fois que vous commencez à assembler votre cadre, vous voudrez peut-être parcourir à nouveau les sections sur la création de personnages dans ce livre. Ce qui est habituellement adapté à un genre fantastique, par exemple, peut ne pas convenir à votre propre univers de fantasy.

    Imaginons que, dans votre monde, la magie du feu soit toujours maléfique et uniquement pratiquée par des prêtres possédés par des démons. Dans ce cas, l'axe Porte une auréole de feu ne serait pas approprié pour les personnages des joueurs, même s'il est parfaitement acceptable dans d'autres jeux de fantasy.

    Plus vous définissez précisément les détails de votre cadre, plus il sera facile d'en ajuster les éléments. Et plus votre univers s'éloigne des clichés du genre, plus vous devrez adapter les choix possibles. Mais ce n'est pas un problème : les cadres spécifiques et distincts sont souvent les plus amusants, les plus mémorables et les plus engageants pour vos joueurs. Ils valent largement l'effort supplémentaire.


    \section*{ADAPTER LES RÈGLES}
    Parfois, vous devez modifier certaines choses pour qu'elles correspondent à vos besoins et envies. Prenons par exemple la saveur « magie » que vous pouvez attribuer à n'importe quel type présenté dans le chapitre 5. Elle s'appelle « magie » et possède de nombreux éléments associés à ce concept, mais il serait très simple de changer son nom en « psionique », « pouvoirs mutants » ou tout autre terme adapté à votre univers.

    En d'autres termes, sélectionner du contenu dans ce livre peut ne pas suffire. Vous pourriez avoir besoin d'ajuster certains éléments ici et là. Heureusement, la plupart du matériel est conçu pour être modifié ou adapté. En fait, grâce à la simplicité des mécaniques de base du Cypher System, effectuer des ajustements est un jeu d'enfant. Ce n'est pas un système où un petit changement risque d'entraîner un effet domino aux conséquences imprévues.

    Dans le chapitre 7, vous trouverez des directives pour créer de nouveaux descripteurs. Le chapitre 8 contient une section entière consacrée à la création de nouveaux axes adaptés à votre propre jeu. De plus, les types de personnages du chapitre 5 sont conçus pour être personnalisés et remodelés. 

    Lorsque vous apportez des modifications, concentrez-vous moins sur l'équilibrage du jeu et davantage sur la narration des histoires que vous souhaitez raconter, tout en permettant aux joueurs de créer et d'incarner les personnages qu'ils veulent. Si vous parvenez à faire ces deux choses correctement, tout le monde sera satisfait. Et au final, c'est précisément ce qu'est l'équilibrage du jeu. 

    Vous pouvez également consulter le chapitre 25 pour approfondir la façon de modifier les mécaniques du jeu. Mais dans l'ensemble, ce chapitre vous rappellera ce que vous venez de lire : c'est *votre* jeu, et vous êtes libre d'en faire ce que vous voulez.

