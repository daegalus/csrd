%#######################################################################
%            CHAPTER 5
%#######################################################################
\startchapter{Type}{ch:chapter5}{\getcolorpartone}
% \chapter{Type}
% \label{ch:chapter5}
% \fancypagestyle{plain}{ %
% \fancyhf{} % remove everything
% \renewcommand{\headrulewidth}{1pt} % remove lines as well
% \renewcommand{\footrulewidth}{0pt}
% \xpretocmd\headrule{\color{\getcolorpartone}}{}{\PatchFailed}
% \fancyhead[RO]{\textcolor{\getcolorpartone}{Type}}
% \fancyhead[LE]{\textcolor{\getcolorpartone}{CYPHER SYSTEM}}
% }
\chapterfirstletter{L}{\getcolorpartone}e Type de personnage est le coeur de votre personnage. Son type vous aide à déterminer sa place dans le monde et les relations avec les autres dans la campagne. C'est le "nom" dans la phrase "Je suis un/e textit{nom+adjectif} qui textit{proposition}"

Vous pouvez choisir parmi quatre types de personnage: Guerrier, Adepte, Explorateur, et Emissaire. Toutefois, vous pourriez ne pas vouloir utiliser ces termes génériques. Ce chapitre vous propose, pour chaque type, quelques alternatives de noms qui pourraient être plus adapté à un genre particulier. Vous trouverez peut-être que des noms comme "Guerrier" ou "Explorateur" ne sonnent pas juste, en particulier dans des campagnes se déroulant à une époque contemporaine. Comme toujours, vous êtes libre de faire comme vous voulez.

Comme le type est la base sur laquelle votre personnage est bati, il est important de considérer comment le type se tient avec la campagne sélectionnée. Pour se faire, les types sont en pratique des archetypes. Un Guerrier, par exemple, pourrait être n'importe qui du chevalier en armure étincellante au policier dans la rue ou au baroudeur cybernétique vétéran de milliers de guerres futuristes.

Pour faciliter l'usage des quatre types dans les diverses campagnes, différentes méthodes appelées "préférences" sont présentées dans le Chapitre 6: Préférence piur aider à personnaliser les différents types pour de la fantasy, de la science fiction, ou d'autres genres (ou pour s'ajuster à différents concepts de personnage).

Au final, des options plus fondamentales de personnalisation sont fournies à la fin de ce chapitre.

\section*{Guerrier}

\begin{description}
    \item[Fantasy/Contes:] guerrier, combattant, escrimeur, chevalier, barbare, soldat, myrmidon, valkyrie
    \item[Moderne/Horreur/Romance:] officier de police, soldat, gardien, détective, vigile, athlète
    \item[Science fiction:] officier de sécurité, guerrier, homme de troupe, soldat, mercenaire
    \item[Superhéro/Post-Apocalyptique:] héro, brique, cogneur
\end{description}

Vous êtes un bon allié à avoir dans un combat. Vous savez comment utiliser des armes et vous défendre.En fonction du genre et de la campagne, cela pourrait signifier de porter une épée et un bouclier dans une arêne de gladiateurs, un AK-47 et des grenades en bandoulière dans la jungle, ou un fusil blaster et une armure mécanique dans l'exploration d'une planète lointaine.

\begin{description}
    \item[Rôle Individuel:] Les Guerriers sont orienté sur le physique et l'action. ils auront plus l'habitude de surmonter un péril en utilisant la force que d'autre moyen, et ils prennent souvent le chemin le plus court vers leur objectif.
    \item[Rôle dans un Groupe:] Les guerriers subissent et infligent généralement le plus de dégâts dans une situation dangereuse. Il leur incombe souvent de protéger les autres membres du groupe contre les menaces. Cela signifie parfois que les guerriers assument également des rôles de commandement, du moins au combat et dans d'autres moments de danger.
    \item[Rôle en société:] Les Guerriers ne sont pas toujours des soldats ou des mercenaires. N'importe qui est est toujours prêt pour la violence, ou même la violence potentielle, peut être un Guerrier dans un sens général. Cela inclut les gardes, les gardiens, les officiers de police, les marins, ou les personnes dans d'autres rôle ou profession qui savent comment se défendre avec talent.
    \item[Guerier Expérimentés:] Alors que les guerriers progressent, leur compétence en combat—que ce soit en se défendant ou en infligeant des dommages—augmente à un rang impressionant. A un rang supérieur, ils peuvent souvent se prendre un groupe d'aversaires tout seul ou affronter sur son terrain n'importe qui.
\end{description}

\begin{table}[t]
    \begin{tabular}{ | p{70mm} | }
        \hline
        \rowcolor{SkyBlue!50}
        \textcolor{\getcolorpartone}{\Large \uppercase{Intrusion de Joueur}} \\
        \rowcolor{SkyBlue!50}
        Une intrusion de joueur est quand un joueur choisi d'altérer quelque chose dans la campagne, rendant les choses plus facile pour le PJ. Dans l'idée, c'est l'inverse d'un intrusion de MJ qui donne au joueur des points d'expérience (XP) en introduisant une complication inatendue pour le personnage. Dans le cas du joueur, ce dernier dépense un 1 XP et présente une solution à un problème ou une complication. Ce que l'intrusion du joueur peut faire est en général, c'est introduire un changement du monde ou des circonstances, plutôt que de changer directement le personnage. Par exemple, une intrusion qui propose que le cypher qui vient d'être utilisé a une charge supplémentaire, est appropriée, mais une intrusion qui propose que le personnage est soigné ne l'est pas. Si le joueur n'a pas d'XP à dépenser, il ne peut pas utiliser d'intrusion de joueur.

        Quelques exemples d'intrusion de joueur sont proposées pour chaque type. Cela dit, toutes les intrusions de joueur listées ici ne sont pas appropriées à chaque situation. Le MJ peut autoriser les joueurs à proposer d'autres suggestions d'intrusion de joueur, mais la Meneuse a le dernier mot pour savoir si une intrusion est appropriée en fonction du type de personnage et de la situation. Si la Meneuse refuse l'intrusion, le joueur ne dépense pas de XP, et l'intrusion n'a simplement pas lieu.

        Utiliser une intrusion ne requiert pas du personnage d'utiliser une action pour l'activer. Une intrusion de joueur survient tout simplement. \\
        \hline
    \end{tabular}
\end{table}

\section*{Intrusions de Joueur pour un Guerrier}

Vous pouvez dépenser 1 XP pour utiliser une des intrusions de joueur suivantes, à condition que la situation est appropriée et que la Meneuse soit d'accord.
\begin{description}
    \item[Position Parfaite:] Vous combattez au moins trois adversaires et chacun d'eux se trouve exactement à la bonne position pour vous pour faire un mouvement pour lequel vous vous êtes entrainé il y a longtemps, vous permettant de les attaquer tous les trois en une seule action. Faites un jet d'attaque pour chaque adversaire. Vous restez limité par la quantité d'Effort que vous pouvez allouer en une seule action.
    \item[Vieil Ami:] Un ancien companion d'arme se présente de manière spontanée et vous fourni de l'aide dans ce que vous êtes en train de faire. Il doit accomplir sa propre mission et ne peut pas rester plus longtemps que pour vous aider, parler un peu après et peut-être partager un repas rapide.
    \item[Arme cassée:] L'arme de votre adversaire a un point faible. Pendant le combat, l'arme est endommagée et descend de deux rangs sur le suivi des dommages des objets.
\end{description}

\section*{Réserves de Stat pour un Guerrier}
\begin{table}[h]
    \begin{tabular}{ l c }
        \textbf{Stat} & \textbf{Valeur de Réserve au Démarrage} \\
        \rowcolor{SkyBlue!50}
        Puissance & 10 \\
        \rowcolor{SkyBlue!20}
        Célérité & 10 \\
        \rowcolor{SkyBlue!50}
        Intellect & 8 \\
    \end{tabular}
\end{table}

Vous avez 6 points supplémentaires à répartir parmi vos Réserves de stat comme vous le souhaitez.

\section*{Guerrier de Premier Rang}

Les Guerriers de Premier Rang ont les capacités suivantes:
\begin{description}
    \item[Effort:] Votre Effort est de 1.
    \item[Physical Nature:] Vous avez un Avantage de Puissance de 1 et un Avantage de Célérité de 0, ou vous avez un Avantage de Puissance de 0 et un Avantage de Célérité de 1. Dans tous les cas, vous avez un Avantage d'Intellect de 0.
    \item[Cypher Use:] Vous pouvez porter deux cyphers en même temps.
    \item[Armes:] Vous avez la pratique des armes légères, moyennes et lourdes et n'avez aucune pénalité quand vous utilisez une arme quelconque.
    \item[Equipment au départ:] Des vêtements appropriés et deux armes de votre choix, ainsi que un objet cher, deux objets modérement chers, et jusqu'à quatre objets peu chers.
    \item[Capacités Spéciales:] Choisissez quatre capacités listées ci-cessous. Vous ne pouvez pas choisir la même capacité plus d'une fois, à moins que sa description dit le contraire. La description complète de chaque capacité listée se trouve dans le chapitre Capacités, qui dispose aussi des descriptions pour les préférences et les capacités de focus en un seul grand catalogue.
\end{description}

\begin{abnamelist}
    \item Avantage de Stat Amélioré (\pageref{subsec:ab_improved_edge})
    \item Choc (\pageref{subsec:ab_bash})
    \item Claque (\pageref{subsec:ab_swipe})
    \item Compétences physiques (\pageref{subsec:ab_physical_skills})
    \item Contrôler le terrain (\pageref{subsec:ab_control_the_field})
    \item Entraîné sans armure (\pageref{subsec:ab_trained_without_armor})
    \item Lancer rapide (\pageref{subsec:ab_quick_throw})
    \item Pas besoin d'armes (\pageref{subsec:ab_no_need_for_weapons})
    \item Pratique des armures (\pageref{subsec:ab_practiced_in_armor})
    \item Prouesses au combat (\pageref{subsec:ab_combat_prowess})
    \item Tir d'Opportunité \hyperref[subsec:ab_overwatch]{\pageref{subsec:ab_overwatch}}
\end{abnamelist}


\section*{ Guerrier de Second Rang}

Choisissez NNN des capacités ci-dessous (ou du rang inférieur) pour l'ajouter à votre répertoire. Vous pouvez en plus remplacer l'une de vos capacités de rang inférieur par une différente d'un rang inférieur.
\begin{abnamelist}
\item Attaque successive (\pageref{subsec:ab_successive_attack})
\item Compétence avec les attaques (\pageref{subsec:ab_skill_with_attacks})
\item Compétence en défense (\pageref{subsec:ab_skill_with_defense})
\item Coup écrasant (\pageref{subsec:ab_crushing_blow})
\item Hémorragie (\pageref{subsec:ab_hemorrhage})
\item Recharger (\pageref{subsec:ab_reload})
\end{abnamelist}

\section*{ Guerrier de Troisième Rang}

Choisissez NNN des capacités ci-dessous (ou du rang inférieur) pour l'ajouter à votre répertoire. Vous pouvez en plus remplacer l'une de vos capacités de rang inférieur par une différente d'un rang inférieur.
\begin{abnamelist}
\item Découpe (\hyperref[subsec:ab_slice]{\pageref{subsec:ab_slice}})
\item Expérimenté en armure (\pageref{subsec:ab_experienced_in_armor})
\item Fureur (\pageref{subsec:ab_fury})
\item Pulvérisation (\pageref{subsec:ab_spray})
\item Réaction (\pageref{subsec:ab_reaction})
\item Résistance énergétique (\pageref{subsec:ab_energy_resistance})
\item Saisissez l'instant (\pageref{subsec:ab_seize_the_moment})
\item Se Fendre (\pageref{subsec:ab_lunge})
\item Tir Double (\pageref{subsec:ab_trick_shot})
\item Utilisation experte des cyphers (\pageref{subsec:ab_expert_cypher_use})
\item Vigilance (\pageref{subsec:ab_vigilance})
\item Visée mortelle (\pageref{subsec:ab_deadly_aim})
\end{abnamelist}

\begin{table*}[b]
    \hspace*{-40pt}
    \centering
    \begin{tabular}{ c p{18cm} }
        \multicolumn{2}{ l }{\Large \textcolor{\getcolorpartone}{Relation avec l'histoire personnelle d'un Guerrier}} \\
        \multicolumn{2}{ l }{Votre type vous aide à déterminer la relation que vous avez avec le cadre de campagne. Lancez un d20 ou choisissez dans la liste ci-après pour déterminer un fait bien particulier à propos de votre histoire personnel qui fourni une relation avec le reste du monde. Vous pouvez aussi créer votre propre relation.} \\
        \textbf{d20} & \textbf{Histoire personnelle} \\ [0.5ex]
         \rowcolor{SkyBlue!50}
         1 & Vous étiez dans l'armée et vous y avez toujours des amis. Votre ancien commandant se souvient bien de vous. \\
         \rowcolor{SkyBlue!20}
        2 & Vous étiez le garde du corps d'unefemme riche qui vous a accusé de vol. Vous avez quitté son service en disgrâce.  \\
        \rowcolor{SkyBlue!50}
        3 & Vous étiez le videur d'un bar du coin pendant un temps, et les habitués se souviennent de vous. \\
        \rowcolor{SkyBlue!20}
        4 & Vous vous êtes entrainé avec un maître reconnu. Il vous respecte mais il a beaucoup d'ennemis. \\
        \rowcolor{SkyBlue!50}
        5 & Vous vous êtes entrainé dans un monastère isolé. Les moines sont toujours vos frêres mais vous êtes un étranger pourtout les autres. \\
        \rowcolor{SkyBlue!20}
        6 & Vous n'avez pas été vraiment entrainé. Vos compétences viennent naturellement (ou de manière surnatuelle). \\
        \rowcolor{SkyBlue!50}
        7 & Vous avez passé du temps dans les rues et avez été en prison pendant un moment. \\
        \rowcolor{SkyBlue!20}
        8 & Vous avez été réquisitionné dans une armée, mais vous avez déserté rapidement. \\
        \rowcolor{SkyBlue!50}
        9 & Vous avez servi de garde du coprs à un puissant criminel qui vous doit sa vie. \\
        \rowcolor{SkyBlue!20}
        10 & Vous avez travaillé comme officier de police ou comme une sorte de gendarme. Tout le monde vous connait, mais les opinions qu'il ont de vous peuvent varier. \\
        \rowcolor{SkyBlue!50}
        11 & Votre grand frêre ou grande soeur est un personnage tristement célèbre qui a été disgracié. \\
        \rowcolor{SkyBlue!20}
        12 & Vous avez servi comme garde pour quelqu'un qui a beaucoup voyagé. Vous connaissez beaucoup de monde un peu partout. \\
        \rowcolor{SkyBlue!50}
        13 & Votre meilleur ami est un enseignant ou un savant. C'est une bonne source d'information. \\
        \rowcolor{SkyBlue!20}
        14 & Vous et un ami fumez tout les deux la même sorte de tabac rare et cher. Vous vous réunissez au moins une fois par semaine pour parler un peu et fumer. \\
        \rowcolor{SkyBlue!50}
        15 & Votre oncle dirige un théatre en ville. Vous connaissez tous les acteurs et pouvez regarder les spectacles gratuitement. \\
        \rowcolor{SkyBlue!20}
        16 & Votre ami artisan peut quelque fois vous demander de l'aide. Toutefois il vous paie correctement. \\
        \rowcolor{SkyBlue!50}
        17 & Votre maître a écrit un livre sur les arts martiaux. De temps en temps, des personnes vous cherche pour vous demander des éclaircissement sur certains passages un peu étranges. \\
        \rowcolor{SkyBlue!20}
        18 & Une personne avec qui vous avez combattu dans l'armée est maintenant le maire d'une ville voisine. \\
        \rowcolor{SkyBlue!50}
        19 & Vous avez sauvé la vie d'une famille alors que leur maison était en flamme. Elle a une dette envers vous et les voisins voient en vous un héro. \\
        \rowcolor{SkyBlue!20}
        20 & Votre ancien entraineur attend toujours de vous que vous revenniez nettoyer après les cours;quand vous le faites, il partage avec vous de temps en temps des rumeurs interressantes. \\
    \end{tabular}
\end{table*}

\section*{ Guerrier de Quatrième Rang}

Choisissez NNN des capacités ci-dessous (ou du rang inférieur) pour l'ajouter à votre répertoire. Vous pouvez en plus remplacer l'une de vos capacités de rang inférieur par une différente d'un rang inférieur.
\begin{abnamelist}
\item Dur comme du Bois (\pageref{subsec:ab_tough_as_nails})
\item Défenseur expérimenté (\pageref{subsec:ab_experienced_defender})
\item Effets accrus (\pageref{subsec:ab_increased_effects})
\item Effort incroyable (\pageref{subsec:ab_amazing_effort})
\item Feinte (\pageref{subsec:ab_feint})
\item Guerrier Capable (\pageref{subsec:ab_capable_warrior})
\item Momentum (\pageref{subsec:ab_momentum})
\item Percer les Défenses (\pageref{subsec:ab_pry_open})
\item Tir Précis (\pageref{subsec:ab_snipe})
\end{abnamelist}

\section*{ Guerrier de Cinquième Rang}

Choisissez NNN des capacités ci-dessous (ou du rang inférieur) pour l'ajouter à votre répertoire. Vous pouvez en plus remplacer l'une de vos capacités de rang inférieur par une différente d'un rang inférieur.
\begin{abnamelist}
\item Attaque sautée (\pageref{subsec:ab_jump_attack})
\item Maîtrise de la défense (\pageref{subsec:ab_mastery_with_defense})
\item Maîtrise des attaques (\pageref{subsec:ab_mastery_with_attacks})
\item Maîtrise en Armure (\pageref{subsec:ab_mastery_in_armor})
\item Parade (\pageref{subsec:ab_parry})
\item Succès amélioré (\pageref{subsec:ab_improved_success})
\item Tirs en éventail (\pageref{subsec:ab_arc_spray})
\item Utilisation adroite des cyphers (\pageref{subsec:ab_adroit_cypher_use})
\end{abnamelist}

\section*{ Guerrier de Sixième Rang}

Choisissez NNN des capacités ci-dessous (ou du rang inférieur) pour l'ajouter à votre répertoire. Vous pouvez en plus remplacer l'une de vos capacités de rang inférieur par une différente d'un rang inférieur.
\begin{abnamelist}
\item Arme et corps (\pageref{subsec:ab_weapon_and_body})
\item Attaque Tournoyante (\pageref{subsec:ab_spin_attack})
\item Coup final (\pageref{subsec:ab_finishing_blow})
\item Encore et encore (\pageref{subsec:ab_again_and_again})
\item Meurtrier (\pageref{subsec:ab_murderer})
\item Moment magnifique (\pageref{subsec:ab_magnificent_moment})
\end{abnamelist}

\section{ Adepte }
\begin{description}
\item[Fantasy/Contes:]
\item[Moderne/Horreur/Romance:]
\item[Science fiction:]
\item[Superhéro/Post-Apocalyptique:]
\end{description}


\begin{description}
\item[Rôle Individuel:]
\item[Rôle dans un Groupe:]
\item[Rôle en société:]
\item[Adepte Expérimenté:]
\end{description}


\section*{Intrusions de Joueur pour un Adepte }

Vous pouvez dépenser 1 XP pour utiliser un des intrusions de joueur suivantes, à condition que la situation est appropriée et que la Meneuse soit d'accord.
\begin{description}
\item[Intrusion1:]
\item[Intrusion2:]
\item[Intrusion3:]
\end{description}

\section*{Réserves de Stat pour un Adepte }
\begin{table}[h]
    \begin{tabular}{ l c }
        \textbf{stat} & \textbf{Valeurs de Réserve au Démarrage} \\
        \rowcolor{SkyBlue!50}
        Puissance & 7 \\
        \rowcolor{SkyBlue!50}
        Célérité & 9 \\
        \rowcolor{SkyBlue!50}
        Intellect & 12 \\
    \end{tabular}
\end{table}

Vous avez 6 points supplémentaires à répartir parmi vos Réserves de stat comme vous le souhaitez.

\section*{ Adepte de Premier Rang}

Les Adeptes de Premier Rang ont les Capacités suivantes:
\begin{description}
\item[Effort:] Votre Effort est de 1.
\item[Avantage:] Intellect=1
\item[Utilisation de Cypher:] Vous pouvez porter 3 cuphers en même temps.
\item[Equpement au départ:]
\item[Capacités spéciales:]
\end{description}

\begin{abnamelist}
\item Assaut Magique (\pageref{subsec:ab_onslaught})
\item Briser (\pageref{subsec:ab_shatter})
\item Champ de résonance (\pageref{subsec:ab_resonance_field})
\item Coup écrasant (\pageref{subsec:ab_crushing_blow})
\item Distorsion (\pageref{subsec:ab_distortion})
\item Effacer les souvenirs (\pageref{subsec:ab_erase_memories})
\item Formation magique (\pageref{subsec:ab_magic_training})
\item Grand Pas (\pageref{subsec:ab_far_step})
\item Magie Prosaïque (\pageref{subsec:ab_hedge_magic})
\item Poussée (\pageref{subsec:ab_push})
\item Protection (\pageref{subsec:ab_ward})
\end{abnamelist}

\section*{ Adepte de Second Rang}

Choisissez NNN des capacités ci-dessous (ou du rang inférieur) pour l'ajouter à votre répertoire. Vous pouvez en plus remplacer l'une de vos capacités de rang inférieur par une différente d'un rang inférieur.
\begin{abnamelist}
\item Adaptation (\pageref{subsec:ab_adaptation})
\item Lecture mentale (\pageref{subsec:ab_mind_reading})
\item Lumière coupante (\pageref{subsec:ab_cutting_light})
\item Récupérer des souvenirs (\pageref{subsec:ab_retrieve_memories})
\item Révèle (\pageref{subsec:ab_reveal})
\item Stase (\pageref{subsec:ab_stasis})
\item Survol (\pageref{subsec:ab_hover})
\end{abnamelist}

\section*{ Adepte de Troisième Rang}

Choisissez NNN des capacités ci-dessous (ou du rang inférieur) pour l'ajouter à votre répertoire. Vous pouvez en plus remplacer l'une de vos capacités de rang inférieur par une différente d'un rang inférieur.
\begin{abnamelist}
\item Barrière de champ de force (\pageref{subsec:ab_force_field_barrier})
\item Capteur (\pageref{subsec:ab_sensor})
\item Contre-mesures (\pageref{subsec:ab_countermeasures})
\item Feu et Glace (\pageref{subsec:ab_fire_and_ice})
\item Oeil pour Cibler (\pageref{subsec:ab_targeting_eye})
\item Protection énergétique (\pageref{subsec:ab_energy_protection})
\item Utilisation adroite des cyphers (\pageref{subsec:ab_adroit_cypher_use})
\end{abnamelist}

\section*{ Adepte de Quatrième Rang}

Choisissez NNN des capacités ci-dessous (ou du rang inférieur) pour l'ajouter à votre répertoire. Vous pouvez en plus remplacer l'une de vos capacités de rang inférieur par une différente d'un rang inférieur.
\begin{abnamelist}
\item Contrôle mental (\pageref{subsec:ab_mind_control})
\item Exil (\pageref{subsec:ab_exile})
\item Invisibilité (\pageref{subsec:ab_invisibility})
\item Nuage de matière (\pageref{subsec:ab_matter_cloud})
\item Projection (\pageref{subsec:ab_projection})
\item Remodeler (\pageref{subsec:ab_reshape})
\item Régénérer (\pageref{subsec:ab_regenerate})
\item Toucher Mortel (\pageref{subsec:ab_death_touch})
\item Traitement rapide (\pageref{subsec:ab_rapid_processing})
\item Trou de ver (\pageref{subsec:ab_wormhole})
\end{abnamelist}

\section*{ Adepte de Cinquième Rang}

Choisissez NNN des capacités ci-dessous (ou du rang inférieur) pour l'ajouter à votre répertoire. Vous pouvez en plus remplacer l'une de vos capacités de rang inférieur par une différente d'un rang inférieur.
\begin{abnamelist}
\item Absorber l'énergie (\pageref{subsec:ab_absorb_energy})
\item Concussion (\pageref{subsec:ab_concussion})
\item Conjuration (\pageref{subsec:ab_conjuration})
\item Connaître l'inconnu (\pageref{subsec:ab_knowing_the_unknown})
\item Créer (\pageref{subsec:ab_create})
\item Poussière Retourne à la Poussière (\pageref{subsec:ab_dust_to_dust})
\item Téléportation (\pageref{subsec:ab_teleportation})
\item Véritables sens (\pageref{subsec:ab_true_senses})
\end{abnamelist}

\section*{ Adepte de Sixième Rang}

Choisissez NNN des capacités ci-dessous (ou du rang inférieur) pour l'ajouter à votre répertoire. Vous pouvez en plus remplacer l'une de vos capacités de rang inférieur par une différente d'un rang inférieur.
\begin{abnamelist}
\item Contrôle de la météo (\pageref{subsec:ab_control_weather})
\item Déplacer des montagnes (\pageref{subsec:ab_move_mountains})
\item Traversez les mondes (\pageref{subsec:ab_traverse_the_worlds})
\item Tremblement de terre (\pageref{subsec:ab_earthquake})
\item Usurpation de Cypher (\pageref{subsec:ab_usurp_cypher})
\end{abnamelist}

\section{ Explorateur }
\begin{description}
\item[Fantasy/Contes:]
\item[Moderne/Horreur/Romance:]
\item[Science fiction:]
\item[Superhéro/Post-Apocalyptique:]
\end{description}


\begin{description}
\item[Rôle Individuel:]
\item[Rôle dans un Groupe:]
\item[Rôle en société:]
\item[Explorateur Expérimenté:]
\end{description}


\section*{Intrusions de Joueur pour un Explorateur }

Vous pouvez dépenser 1 XP pour utiliser un des intrusions de joueur suivantes, à condition que la situation est appropriée et que la Meneuse soit d'accord.
\begin{description}
\item[Intrusion1:]
\item[Intrusion2:]
\item[Intrusion3:]
\end{description}

\section*{Réserves de Stat pour un Explorateur }
\begin{table}[h]
\begin{tabular}{ l c }
\textbf{stat} & \textbf{Valeurs de Réserve au Démarrage} \\
\rowcolor{SkyBlue!50}
Puissance & 10 \\
\rowcolor{SkyBlue!50}
Célérité & 9 \\
\rowcolor{SkyBlue!50}
Intellect & 9 \\
\end{tabular}
\end{table}

Vous avez 6 points supplémentaires à répartir parmi vos Réserves de stat comme vous le souhaitez.

\section*{ Explorateur de Premier Rang}

Les Explorateurs de Premier Rang ont les Capacités suivantes:
\begin{description}
\item[Effort:] Votre Effort est de 1.
\item[Avantage:] Puissance=1
\item[Utilisation de Cypher:] Vous pouvez porter 2 cuphers en même temps.
\item[Equpement au départ:]
\item[Capacités spéciales:]
\end{description}

\begin{abnamelist}
\item Avantage de Stat Amélioré (\pageref{subsec:ab_improved_edge})
\item Bloquer (\pageref{subsec:ab_block})
\item Compétences en Connaissances (\pageref{subsec:ab_knowledge_skills})
\item Compétences physiques (\pageref{subsec:ab_physical_skills})
\item Déchiffrer (\pageref{subsec:ab_decipher})
\item Endurance (\pageref{subsec:ab_endurance})
\item Entraîné sans armure (\pageref{subsec:ab_trained_without_armor})
\item Muscles de fer (\pageref{subsec:ab_muscles_of_iron})
\item Pas besoin d'armes (\pageref{subsec:ab_no_need_for_weapons})
\item Pied Léger (\pageref{subsec:ab_fleet_of_foot})
\item Pratique de toutes les armes (\pageref{subsec:ab_practiced_with_all_weapons})
\item Pratique des armures (\pageref{subsec:ab_practiced_in_armor})
\item Sens du Danger (\pageref{subsec:ab_danger_sense})
\item Sursaut de confiance (\pageref{subsec:ab_surging_confidence})
\item Trouver le chemin (\pageref{subsec:ab_find_the_way})
\end{abnamelist}

\section*{ Explorateur de Second Rang}

Choisissez NNN des capacités ci-dessous (ou du rang inférieur) pour l'ajouter à votre répertoire. Vous pouvez en plus remplacer l'une de vos capacités de rang inférieur par une différente d'un rang inférieur.
\begin{abnamelist}
\item Activer les autres (\pageref{subsec:ab_enable_others})
\item Augmentation de la portée (\pageref{subsec:ab_range_increase})
\item Compétence en défense (\pageref{subsec:ab_skill_with_defense})
\item Compétences d'enquête (\pageref{subsec:ab_investigative_skills})
\item Compétences de voyage (\pageref{subsec:ab_travel_skills})
\item Coordination Main-Oeil (\pageref{subsec:ab_hand_to_eye})
\item Curieux (\pageref{subsec:ab_curious})
\item Déjouer le Danger (\pageref{subsec:ab_foil_danger})
\item Evasion (\pageref{subsec:ab_escape})
\item Instinct de Danger (\pageref{subsec:ab_danger_instinct})
\item Oeil pour les détails (\pageref{subsec:ab_eye_for_detail})
\item Rester en Alerte (\pageref{subsec:ab_stand_watch})
\item Récupération rapide (\pageref{subsec:ab_rapid_recovery})
\end{abnamelist}

\section*{ Explorateur de Troisième Rang}

Choisissez NNN des capacités ci-dessous (ou du rang inférieur) pour l'ajouter à votre répertoire. Vous pouvez en plus remplacer l'une de vos capacités de rang inférieur par une différente d'un rang inférieur.
\begin{abnamelist}
\item Briseur de Pierre (\pageref{subsec:ab_stone_breaker})
\item Chute contrôlée (\pageref{subsec:ab_controlled_fall})
\item Compétence avec les attaques (\pageref{subsec:ab_skill_with_attacks})
\item Courir et combattre (\pageref{subsec:ab_run_and_fight})
\item Course d'obstacles (\pageref{subsec:ab_obstacle_running})
\item Expérimenté en armure (\pageref{subsec:ab_experienced_in_armor})
\item Extraire du hasard (\pageref{subsec:ab_wrest_from_chance})
\item Ignorez la Douleur (\pageref{subsec:ab_ignore_the_pain})
\item Pensez à votre sortie (\pageref{subsec:ab_think_your_way_out})
\item Résilience (\pageref{subsec:ab_resilience})
\item Saisissez l'instant (\pageref{subsec:ab_seize_the_moment})
\item Trouver les Pièges (\pageref{subsec:ab_trapfinder})
\item Utilisation experte des cyphers (\pageref{subsec:ab_expert_cypher_use})
\end{abnamelist}

\section*{ Explorateur de Quatrième Rang}

Choisissez NNN des capacités ci-dessous (ou du rang inférieur) pour l'ajouter à votre répertoire. Vous pouvez en plus remplacer l'une de vos capacités de rang inférieur par une différente d'un rang inférieur.
\begin{abnamelist}
\item Compétence d'expert (\pageref{subsec:ab_expert_skill})
\item Coureur (\pageref{subsec:ab_runner})
\item Dur comme du Bois (\pageref{subsec:ab_tough_as_nails})
\item Effets accrus (\pageref{subsec:ab_increased_effects})
\item Guerrier Capable (\pageref{subsec:ab_capable_warrior})
\item Lire les signes (\pageref{subsec:ab_read_the_signs})
\item Pas subtiles (\pageref{subsec:ab_subtle_steps})
\end{abnamelist}

\section*{ Explorateur de Cinquième Rang}

Choisissez NNN des capacités ci-dessous (ou du rang inférieur) pour l'ajouter à votre répertoire. Vous pouvez en plus remplacer l'une de vos capacités de rang inférieur par une différente d'un rang inférieur.
\begin{abnamelist}
\item Amitié de groupe (\pageref{subsec:ab_group_friendship})
\item Attaque sautée (\pageref{subsec:ab_jump_attack})
\item Difficile à tuer (\pageref{subsec:ab_hard_to_kill})
\item Libre de se déplacer (\pageref{subsec:ab_free_to_move})
\item Maîtrise de la défense (\pageref{subsec:ab_mastery_with_defense})
\item Parade (\pageref{subsec:ab_parry})
\item Physiquement doué (\pageref{subsec:ab_physically_gifted})
\item Prendre le commandement (\pageref{subsec:ab_take_command})
\item Utilisation adroite des cyphers (\pageref{subsec:ab_adroit_cypher_use})
\item Vigilant (\pageref{subsec:ab_vigilant})
\end{abnamelist}

\section*{ Explorateur de Sixième Rang}

Choisissez NNN des capacités ci-dessous (ou du rang inférieur) pour l'ajouter à votre répertoire. Vous pouvez en plus remplacer l'une de vos capacités de rang inférieur par une différente d'un rang inférieur.
\begin{abnamelist}
\item Annuler le danger (\pageref{subsec:ab_negate_danger})
\item Attaque Tournoyante (\pageref{subsec:ab_spin_attack})
\item Encore et encore (\pageref{subsec:ab_again_and_again})
\item Inspire des actions coordonnées (\pageref{subsec:ab_inspire_coordinated_actions})
\item Maîtrise des attaques (\pageref{subsec:ab_mastery_with_attacks})
\item Maîtrise en Armure (\pageref{subsec:ab_mastery_in_armor})
\item Partager la défense (\pageref{subsec:ab_share_defense})
\item Vitalité sauvage (\pageref{subsec:ab_wild_vitality})
\end{abnamelist}

\section{ Emissaire }
\begin{description}
\item[Fantasy/Contes:]
\item[Moderne/Horreur/Romance:]
\item[Science fiction:]
\item[Superhéro/Post-Apocalyptique:]
\end{description}


\begin{description}
\item[Rôle Individuel:]
\item[Rôle dans un Groupe:]
\item[Rôle en société:]
\item[Emissaire Expérimenté:]
\end{description}


\section*{Intrusions de Joueur pour un Emissaire }

Vous pouvez dépenser 1 XP pour utiliser un des intrusions de joueur suivantes, à condition que la situation est appropriée et que la Meneuse soit d'accord.
\begin{description}
\item[Intrusion1:]
\item[Intrusion2:]
\item[Intrusion3:]
\end{description}

\section*{Réserves de Stat pour un Emissaire }
\begin{table}[h]
\begin{tabular}{ l c }
\textbf{stat} & \textbf{Valeurs de Réserve au Démarrage} \\
\rowcolor{SkyBlue!50}
Puissance & 8 \\
\rowcolor{SkyBlue!50}
Célérité & 9 \\
\rowcolor{SkyBlue!50}
Intellect & 11 \\
\end{tabular}
\end{table}

Vous avez 6 points supplémentaires à répartir parmi vos Réserves de stat comme vous le souhaitez.

\section*{ Emissaire de Premier Rang}

Les Emissaires de Premier Rang ont les Capacités suivantes:
\begin{description}
\item[Effort:] Votre Effort est de 1.
\item[Avantage:] Intellect=1
\item[Utilisation de Cypher:] Vous pouvez porter 2 cuphers en même temps.
\item[Equpement au départ:]
\item[Capacités spéciales:]
\end{description}

\begin{abnamelist}
\item Anecdote (\pageref{subsec:ab_anecdote})
\item Attitude de commandement (\pageref{subsec:ab_demeanor_of_command})
\item Babel (\pageref{subsec:ab_babel})
\item Changement d'Identité (\pageref{subsec:ab_spin_identity})
\item Compréhension (\pageref{subsec:ab_understanding})
\item Compétences d'interaction (\pageref{subsec:ab_interaction_skills})
\item Effacer les souvenirs (\pageref{subsec:ab_erase_memories})
\item Embrouiller (\pageref{subsec:ab_fast_talk})
\item Encouragement (\pageref{subsec:ab_encouragement})
\item Envoûtement (\pageref{subsec:ab_enthrall})
\item Inspire l'Agression (\pageref{subsec:ab_inspire_aggression})
\item Pratique des armes moyennes (\pageref{subsec:ab_practiced_with_medium_weapons})
\item Présence terrifiante (\pageref{subsec:ab_terrifying_presence})
\end{abnamelist}

\section*{ Emissaire de Second Rang}

Choisissez NNN des capacités ci-dessous (ou du rang inférieur) pour l'ajouter à votre répertoire. Vous pouvez en plus remplacer l'une de vos capacités de rang inférieur par une différente d'un rang inférieur.
\begin{abnamelist}
\item Compétence en défense (\pageref{subsec:ab_skill_with_defense})
\item Compétences d'interaction (\pageref{subsec:ab_interaction_skills})
\item Démotiver (\pageref{subsec:ab_disincentivize})
\item Facilité Inspirante (\pageref{subsec:ab_inspiring_ease})
\item Pratique des armures (\pageref{subsec:ab_practiced_in_armor})
\item Recueillir des renseignements (\pageref{subsec:ab_gather_intelligence})
\item Récupération Rapide d'un autre (\pageref{subsec:ab_speedy_recovery})
\item Suivant de base (\pageref{subsec:ab_basic_follower})
\item Trahison inattendue (\pageref{subsec:ab_unexpected_betrayal})
\item Transmettre un idéal (\pageref{subsec:ab_impart_ideal})
\item Calmer un Etranger (\pageref{subsec:ab_calm_stranger})
\end{abnamelist}

\section*{ Emissaire de Troisième Rang}

Choisissez NNN des capacités ci-dessous (ou du rang inférieur) pour l'ajouter à votre répertoire. Vous pouvez en plus remplacer l'une de vos capacités de rang inférieur par une différente d'un rang inférieur.
\begin{abnamelist}
\item Accélérer (\pageref{subsec:ab_up_to_speed})
\item Dire les Choses (\pageref{subsec:ab_telling})
\item Disciple expert (\pageref{subsec:ab_expert_follower})
\item Esprit perspicace (\pageref{subsec:ab_discerning_mind})
\item Grande Déception (\pageref{subsec:ab_grand_deception})
\item Lecture mentale (\pageref{subsec:ab_mind_reading})
\item Mené par l'enquête (\pageref{subsec:ab_lead_by_inquiry})
\item Parfait Inconnu (\pageref{subsec:ab_perfect_stranger})
\item Se fondre dans le décor (\pageref{subsec:ab_blend_in})
\item Talent Oratoire (\pageref{subsec:ab_oratory})
\item Utilisation experte des cyphers (\pageref{subsec:ab_expert_cypher_use})
\item Vif d'esprit (\pageref{subsec:ab_quick_wits})
\end{abnamelist}

\section*{ Emissaire de Quatrième Rang}

Choisissez NNN des capacités ci-dessous (ou du rang inférieur) pour l'ajouter à votre répertoire. Vous pouvez en plus remplacer l'une de vos capacités de rang inférieur par une différente d'un rang inférieur.
\begin{abnamelist}
\item Anticipation de l'attaque (\pageref{subsec:ab_anticipate_attack})
\item Compétences accrues (\pageref{subsec:ab_heightened_skills})
\item Elaborer une stratégie (\pageref{subsec:ab_strategize})
\item Feinte (\pageref{subsec:ab_feint})
\item Lire les signes (\pageref{subsec:ab_read_the_signs})
\item Moqueries confondantes (\pageref{subsec:ab_confounding_banter})
\item Psychose (\pageref{subsec:ab_psychosis})
\item Stimuler l'effort (\pageref{subsec:ab_spur_effort})
\item Suggestion (\pageref{subsec:ab_suggestion})
\end{abnamelist}

\section*{ Emissaire de Cinquième Rang}

Choisissez NNN des capacités ci-dessous (ou du rang inférieur) pour l'ajouter à votre répertoire. Vous pouvez en plus remplacer l'une de vos capacités de rang inférieur par une différente d'un rang inférieur.
\begin{abnamelist}
\item Aura fétide (\pageref{subsec:ab_foul_aura})
\item Compétence avec les attaques (\pageref{subsec:ab_skill_with_attacks})
\item Connaître l'inconnu (\pageref{subsec:ab_knowing_the_unknown})
\item Discipline de vigilance (\pageref{subsec:ab_discipline_of_watchfulness})
\item Expérimenté en armure (\pageref{subsec:ab_experienced_in_armor})
\item Fuir (\pageref{subsec:ab_flee})
\item Régénérer (\pageref{subsec:ab_regenerate})
\item Stimuler (\pageref{subsec:ab_stimulate})
\item Utilisation adroite des cyphers (\pageref{subsec:ab_adroit_cypher_use})
\end{abnamelist}

\section*{ Emissaire de Sixième Rang}

Choisissez NNN des capacités ci-dessous (ou du rang inférieur) pour l'ajouter à votre répertoire. Vous pouvez en plus remplacer l'une de vos capacités de rang inférieur par une différente d'un rang inférieur.
\begin{abnamelist}
\item Assumer le contrôle (\pageref{subsec:ab_assume_control})
\item Brise Esprit (\pageref{subsec:ab_shatter_mind})
\item Contrôle des foules (\pageref{subsec:ab_crowd_control})
\item Gestion de bataille (\pageref{subsec:ab_battle_management})
\item Mot de commandement (\pageref{subsec:ab_word_of_command})
\item Recruter un adjoint (\pageref{subsec:ab_recruit_deputy})
\item Succès inspirant (\pageref{subsec:ab_inspiring_success})
\item Véritables sens (\pageref{subsec:ab_true_senses})
\end{abnamelist}

