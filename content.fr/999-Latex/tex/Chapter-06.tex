%#######################################################################
%            CHAPTER 6
%#######################################################################
\startchapter{Préférence}{ch:chapter6}{\getcolorpartone}
\raggedright
\chapterfirstletter{L}{\getcolorpartone}es préférences sont des groupes de capacités spéciales que la Meneuse et les joueurs peuvent utiliser pour modifier le type de personnage pour mieux l'affiner vers l'image qu'ils en ont ou pour qu'il soit mieux adapté au genre ou au cadre de campagne. Par exemple, si une joueuse veut créer un personnage voleur utilisant la magie, elle pourrait jouer une Adepte avec une préférence pour la furtivité. Dans un cadre de campagne de science fiction, un Guerrier pourrait avoir des connaissances en machinerie, donc le personnage pourrait avoir une préférence pour la technologie.

Lors de l'avancement du personnage vers un nouveau rang, les capacités d'une préférence peuvent être échangées avec les capacités de type (une pour une). Ainsi, pour ajouter la capacité de préférence de furtivité Sens du Danger à un Guerrier, quelque-chose d'autre—peut-être Choc—doit être sacrifiée. Maintenant, le personnage peut choisir Sens du Danger comme csi c'était une capacité de Guerrier de premier rang, mais il ne pourrait plus jamais choisir Choc.

La Meneuse doit toujours être impliquée pour ajouter une préférence à un type. Par exemple, elle pourrait savoir que pour son jeu de science fiction, elle souhaite un type appelé "Glam", qui est un Emissaire avec une préférence pour les capacités de technologie—en particulier celles qui font du personnage un flamboyant pilote de vaisseaux spacial. Ainsi, elle échange les capacités de permier rang Changement d'Identité et Inspire l'Agression pour les capacités de préférence de technologie Datajack et Compétences techniques afin que le personnage puisse se connecter directement au vaisseau et puisse prendre les compétences pilotage et ordinateurs.

Au final, la préférence est un outil pour la Meneuse pour créer facilement des types specifiques pour des campagnes en créant de légères modifications des quatre types de base. Bien que les joueurs puisse souhaiter utiliser les préférences pour avoir le personnage qu'ils souhaitent, souvenez-vous qu'ils peuvent aussi très bien ajuster leurs personnages avec les descripteurs et foci.

La description complète de cheque capacité listée ci-dessous peut être trouvée dans le Chapitre 9: Capacités, qui contient aussi les descriptions pour les capacités de type et de focus.

\section*{Préférence de Combat}

La Préférence de combat rend un personnage plus martial. Un Emissaire avec une préférence de combat dans une campagne de fantasy serait un barde de combat. Un Explorateur avec Préférence de combat dans un jeu historique pourrait être un pirate. Un Adepte avec une préférence de combat dans un décor de science-fiction pourrait être un vétéran de mille guerres psychiques.

Capacités de Combat de rang 1


    Pratique des armes moyennes (171)
    Pratique des armures (171)
    Sens du Danger (124)

Capacités de Combat de rang 2

(Cypher System Rulebook, page 36)

    Entraîné sans armure (193)
    Prouesses au combat (120)
    Soif de sang (115)

Capacités de Combat de rang 3

(Cypher System Rulebook, page 36)

    Attaque successive (187)
    Compétence avec les attaques (183)
    Compétence en défense (183)
    Pratique de toutes les armes (171)

Capacités de Combat de rang 4

(Cypher System Rulebook, page 37)

    Attaque successive (187)
    Compétence avec les attaques (183)
    Compétence en défense (183)
    Pratique de toutes les armes (171)

Capacités de Combat de rang 5

(Cypher System Rulebook, page 37)

    Cible difficile (148)
    Défenseur expérimenté (136)
    Parade (168)

Capacités de Combat de rang 6

(Cypher System Rulebook, page 37)

    Compétence en Attaque Supérieure (147)
    Maîtrise de la défense (161)
    Maîtrise en Armure (161)

\section*{Préférence de Magie}

Vous avez quelques connaissances en magie. Vous pourriez e pas être u sorcier, mais vous en connaissez les bases—comment ça marche, et comment accomplir quelques actes extarordinaires. Bien sûr, dans votre cadre de campagne, la "magie" peut en fait être des pouvoirs psychiques, des capacités de mutant, de la technologie extra-terrestre, ou n'importe quoi d'autre qui produit des effets intéressants et utiles.

Un Explorateur avec une préférence pour la magie peut être un chasseur de mage, et un Emissaire avec une préférence pour la magie pourrait être un barde-sorcier. Bien qu'un Adepte avec une préférence pour la magie est toujours un Adepte, mais vous pouvez trouver qu'échanger certaines des capacités de base du type avec celles données ici peuvent mieux orienter le personnage comme le souhaitez.

\section*{Préférence de Compétences et Connaissances}

Cette préférence est destinée aux personnages occupant des rôles qui nécessitent plus de connaissances et une plus grande application des talents dans le monde réel. C'est moins tape-à-l'œil et dramatique que les pouvoirs surnaturels ou la capacité de diviser plusieurs ennemis, mais parfois l'expertise oule savoir-faire est la véritable solution à un problème.
Un guerrier doté de compétences et de connaissances pourrait être un ingénieur militaire. Un explorateur doté de compétences et de connaissances pourrait être un scientifique de terrain. Un orateur avec cette Préférence pourrait être un enseignant.

Un Guerrier doté de compétences et de connaissances pourrait être un ingénieur militaire. Un Explorateur doté de compétences et de connaissances pourrait être un scientifique de terrain. Un Emissaire avec cette Préférence pourrait être un enseignant.

\section*{Préférence de Furtivité}

Les personnages avec la Préférence de Furtivité sont doués pour la discrétion, l'infiltration dans les endroits protégés, et la tromperie. Ils utilisent ces capacités de façon variée, dont le combat. Un Explorateur avec un préférence en furtivité pourrait être un voleur, tandis qu'un Guerrier avec préférence de furtivité serait un assassin. Un Explorateur avec un préférence en furtivité dans une campagne de super-héros pourrait être un justicier qui rôde dans les rues la nuit.

\section*{Préférence de Technologie}

Les personnages ayant une Préférence de technologie sont généralement issus d'un cadre de campagne de science-fiction ou du moins des temps modernes (même si tout est possible). Ils excellent dans l'utilisation, la manipulation et la construction de machines. Un explorateur doté d'une Préférence technologique pourrait être un pilote de vaisseau spatial, et un orateur doté d'une Préférence technologique pourrait être un techno-prêtre. Certaines des capacités les moins orientées vers l'ordinateur pourraient convenir à un personnage steampunk, tandis qu'un personnage moderne pourrait utiliser certaines des capacités qui n'impliquent pas de vaisseaux spatiaux ou d'ultratechnologie.

