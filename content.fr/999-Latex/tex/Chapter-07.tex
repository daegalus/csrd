%#######################################################################
%            CHAPTER 7
%#######################################################################
\startchapter{Descripteur}{ch:chapter7}{CSCOLORPARTONE}
\raggedright
\chapterfirstletter{v}{CSCOLORPARTONE}otre descripteur définit votre personnage, il lui donne son caractère. Les différences entre un Explorateur Charmeur et un Explorateur Vicieux sont considérables. Le descripteur change la façon de chaque pesonnage d'accomplir chaque action. Votre descripteur place votre personnage dans la situation (la première aventure, qui démarre la campagne) et contribue à le motiver. C'est l'adjectif de la phrase « Je suis un *adjectif nom* qui *verbes* ».

Les descripteurs offrent un ensemble unique de capacités, de compétences ou de modifications supplémentaires à vos Réserves de statistiques. Toutes les propositions d'un descripteur ne sont pas des modifications positives du personnage. Par exemple, certains descripteurs ont des incapacités --- des tâches pour lesquelles un personnage n'est pas doué. Vous pouvez considérer les incapacités comme des compétences négatives : au lieu d'être un peu meilleur dans ce genre de tâche, vous êtes un peu pire. Si vous devenez compétent dans une tâche pour laquelle vous êtes incapable, elles s'annulent. N'oubliez pas que les personnages sont définis autant par ce pour quoi ils ne sont *pas* bons que par ce pour quoi ils *sont* bons.

Les descripteurs proposent également quelques brèves suggestions sur la manière dont votre personnage s'est impliqué avec le reste du groupe lors de sa première aventure. Vous pouvez les utiliser ou non, comme vous le souhaitez.

Cette section détaille cinquante descripteurs. Choisissez-en un pour votre personnage. Vous pouvez choisir le descripteur de votre choix, quel que soit votre type. À la fin de ce chapitre, quelques options sont proposées pour personnaliser les descripteurs, notamment faire de l'espèce d'un personnage leur descripteur.

Votre descripteur compte le plus lorsque vous êtes un personnage débutant. Les avantages (et peut-être les inconvénients) découlant de votre descripteur finiront par être éclipsés par l'importance croissante de votre type et de votre orientation. Cependant, l'influence de votre descripteur restera au moins quelque peu importante tout au long de la vie de votre personnage.

%%%%%%%%%%%%%%%%%%%%%%%%%%%%%%%%%%%%%%%%%%%%%%%%%%%%%%%%%%%%%%%%%%%%%
%      Descriptor details
%--------------------------

%--------------------------
\label{sec:descappealing}\section*{Attirant}

\textcolor{gray}{\emph{Appealing}}

Vous êtes attirant aux yeux des autres, mais peut-être plus important encore, vous êtes sympathique et charismatique. Vous avez ce « quelque chose de spécial » qui attire les autres vers vous. Vous savez souvent ce qu'il faut dire pour faire rire quelqu'un, le mettre à l'aise ou l'inciter à agir. Les gens comme vous veulent vous aider et veulent être votre ami.

\textbf{Charismatique:} +2 à votre Réserve d'Intellect.

\textbf{Compétence:} Vous êtes entraîné dans les relations sociales agréables.

\textbf{Resistant aux Charmes:} Vous êtes conscient de la façon dont les autres peuvent manipuler et charmer les gens, et vous remarquez lorsque ces tactiques sont utilisées contre vous. Grâce à cette prise de conscience, vous êtes entraîné à résister à toute forme de persuasion ou de séduction si vous le souhaitez.

\textbf{Lien initial à la Première Aventure:} Choisissez parmi la liste des options ci-dessous comment vous vous êtes retrouvé impliqué dans la première aventure:

1. Vous avez rencontré un parfait inconnu (l'un des autres PJ) et vous l'avez tellement charmé qu'il vous a invité.

2. Les PJ cherchaient quelqu'un d'autre, mais vous les avez convaincus que vous étiez parfait.

3. Un pur hasard --- parce que vous suivez le flux des choses et que tout se passe généralement bien.

4. Votre charisme a permis à l'un des PJ de se sortir d'une situation difficile il y a longtemps, et il vous demande toujours de le rejoindre dans de nouvelles aventures.


%--------------------------
\label{sec:descsharpeyed}\section*{Au Regard-perçant}

\textcolor{gray}{\emph{Sharp-Eyed}}

Vous êtes perspicace et bien conscient de votre environnement. Vous remarquez les petits détails et vous vous en souvenez. Vous pouvez être difficile à surprendre.

\textbf{Compétence:} Vous êtes entraîné aux actions d'Initiative.

\textbf{Compétence:} Vous êtes entraîné aux actions de perception.

\textbf{Trouve la faille:} Si un adversaire a une faiblesse évidente (subit des dégâts supplémentaires dus au feu, ne peut pas voir de son oeil gauche, etc.), le MJ vous dira de quoi il s'agit.

\textbf{Lien initial à la Première Aventure:} Choisissez parmi la liste des options ci-dessous comment vous vous êtes retrouvé impliqué dans la première aventure:

1. Vous avez entendu parler de ce qui se passait, avez remarqué une faille dans le plan des autres PJ et vous vous êtes joint à eux pour les aider.

2. Vous avez remarqué que les PJ ont un ennemi (ou au moins une queue) dont ils n'avaient pas conscience.

3. Vous avez vu que les autres PJ préparaient quelque chose d'intéressant et vous vous êtes impliqués.

4. Vous avez remarqué des choses étranges, et tout cela semble lié.


%--------------------------
\label{sec:descbeneficent}\section*{Bienfaisant}

\textcolor{gray}{\emph{Beneficent}}

Aider les autres est votre vocation. C'est pour ça que vous êtes là. Les autres se réjouissent de votre nature extravertie et charitable, et vous vous réjouissez de leur bonheur. Vous faites de votre mieux lorsque vous aidez les gens, soit en leur expliquant comment ils peuvent surmonter au mieux un défi, *soit en leur démontrant comment le faire par eux-même (translation to be confirmed)*.

\textbf{Généreux:} Les alliés qui ont passé la dernière journée avec vous ajoute +1 à leur jet de Récupération.

\textbf{Altruiste:} Si vous vous tenez à côté d'une créature qui subit des dégâts, vous pouvez intercéder et subir vous-même 1 point de ces dégâts (réduisant les dégâts infligés à la créature de 1 point). Si vous possédez une Armure, cela ne vous apporte aucun avantage lorsque vous utilisez cette capacité.

\textbf{Compétence:} Vous êtes entraîné à toutes les tâches liées aux intéractions sociales agréables, pour mettre les autres à l'aise et gagner leur confiance.

\textbf{Serviable:} Chaque fois que vous aidez un autre personnage, ce personnage en bénéficie comme si vous aviez été entraîné, même si vous n'êtes pas entraîné ou spécialisé dans la tâche tentée.

\textbf{Inaptitude:} Quand vous êtes seul, toutes les tâches d'Intellect et de Célérité sont entravées.

\textbf{Lien initial à la Première Aventure:} Choisissez parmi la liste des options ci-dessous comment vous vous êtes retrouvé impliqué dans la première aventure:

1. Même si vous ne connaissiez pas auparavant la plupart des autres PJ, vous vous êtes invité à leur quête.

2. Vous avez vu les PJ lutter pour surmonter un problème et vous les avez rejoints de manière désintéressée pour vous aider.

3. Vous êtes presque certain que les PJs échoueront sans vous.

4. Le choix était entre votre vie en lambeaux et aider les autres. Depuis, vous n'avez pas regardé en arrière.


%--------------------------
\label{sec:descstrongwilled}\section*{Borné}

\textcolor{gray}{\emph{Strong-Willed}}

Vous êtes un peu borné, volontaire et indépendant. Personne ne peut vous convaincre ou changer d'avis si vous ne souhaitez pas que cela change. Cette qualité ne vous rend pas nécessairement intelligent, mais elle fait de vous un bastion de volonté et de détermination. Vous vous habillez et agissez probablement avec style unique et tendance, sans vous soucier de ce que pensent les autres.

\textbf{Volontaire:} +4 à votre Réserve d'Intellect.

\textbf{Compétence:} Vous êtes entraîné à résister aux effets mentaux.

\textbf{Compétence:} Vous êtes entraîné dans des tâches nécessitant une concentration ou une concentration incroyable.

\textbf{Inaptitude:} Volontaire ne veut pas dire brillant. Toute tâche qui implique de résoudre des énigmes ou des problèmes, de mémoriser des choses ou d'utiliser des connaissances est entravée.

\textbf{Lien initial à la Première Aventure:} Choisissez parmi la liste des options ci-dessous comment vous vous êtes retrouvé impliqué dans la première aventure:

1. Contre votre bon jugement, vous avez rejoint les autres PJ parce que vous avez vu qu'ils étaient en danger.

2. L'un des autres PJ vous a convaincu que rejoindre le groupe serait dans votre intérêt.

3. Vous avez peur de ce qui pourrait arriver si les autres PC tombaient en panne.

4. Il y a une récompense en jeu et vous avez besoin d'argent.


%--------------------------
\label{sec:desccalm}\section*{Calme}

\textcolor{gray}{\emph{Calm}}

Vous avez passé la majeure partie de votre vie à des activités sédentaires (livres, films, passe-temps, etc.) plutôt qu'à des activités actives. Vous connaissez bien toutes sortes de domaines universitaires ou autres activités intellectuelles, mais rien de physique. Vous n'êtes pas nécessairement faible ou affaibli (bien que ce soit un bon descripteur pour les personnages âgés), mais vous n'avez aucune expérience dans les activités plus physiques.Calme est un excellent descripteur pour les personnages qui n'ont jamais eu l'intention de vivre des aventures mais qui y ont été plongés, un trope qui se produit souvent dans les jeux modernes et en particulier dans les jeux d'horreur.

\textbf{Adore les Livres:} +2 à votre Réserve d'Intellect.

\textbf{Compétences:} Vous êtes entraîné à quatre compétences non-physiques de votre choix.

\textbf{Anecdote:} Vous pouvez trouver un fait aléatoire pertinent à la situation actuelle lorsque vous le souhaitez. Il s'agit toujours d'un fait, et non d'une conjecture ou d'une supposition, et doit être quelque chose que vous auriez pu logiquement lire ou voir dans le passé. Vous pouvez le faire une seule fois, bien que la capacité soit renouvelée à chaque fois que vous effectuez un jet de récupération.

\textbf{Inaptitude:} Non seulement vous n'êtes pas un Guerrier, mais toutes les attaques physiques sont désavantagées.

\textbf{Inaptitude:} Vous n'êtes pas vraiment quelqu'un à l'aise dans la nature. Toutes les tâches pour grimper, courir, sauter et nager sont désavantagées.

\textbf{Lien initial à la Première Aventure:} Choisissez parmi la liste des options ci-dessous comment vous vous êtes retrouvé impliqué dans la première aventure:

1. Vous avez lu quelque part la situation actuelle et avez décidé de la vérifier par vous-même.

2. Vous étiez au bon (mauvais ?) endroit au bon (mauvais ?) moment.

3. Tout en évitant une situation totalement différente, vous êtes entré dans votre situation actuelle.

4. L'un des autres PJ vous a entraîné dans cette aventure.


%--------------------------
\label{sec:descrisktaking}\section*{Casse-cou}

\textcolor{gray}{\emph{Risk-Taking}}

Cela fait partie de votre nature de remettre en question ce que les autres pensent ne pas pouvoir ou ne devrait pas être fait. Vous n'êtes pas fou, bien sûr – vous n'essaieriez pas de franchir un gouffre d'un kilomètre de large simplement parce que vous avez le courage de le faire. Il y a l'impossible et puis il y a le tout juste possible. Vous aimez pousser ces derniers plus loin que les autres, car cela vous procure un élan de satisfaction et de plaisir lorsque vous réussissez. Plus vous réussissez, plus vous vous retrouvez à la recherche du prochain défi risqué contre lequel vous essayer.

\textbf{Agile:} +4 à votre Réserve de Célérité.

\textbf{Compétence:} Vous savez exploiter le risque et vous êtes entraîné à des tâches qui impliquent un élément de hasard, comme jouer à des jeux ou choisir entre deux ou trois options apparemment égales.

\textbf{Tenter la Chance:} Vous pouvez choisir de réussir automatiquement une tâche sans lancer de dés, à condition que la difficulté de la tâche ne dépasse pas 6. Cependant, lorsque vous le faites, vous déclenchez également une intrusion de MJ comme si vous aviez obtenu un 1. L'intrusion n'invalide pas la réussite, mais cela la modifie probablement d'une manière ou d'une autre. Vous pouvez le faire une seule fois, bien que cette capacité se renouvelle à chaque fois que vous effectuez un jet de récupération de dix heures.

\textbf{Inaptitude:} Vous êtes peut-être agile, mais vous n'êtes pas sournois. Les tâches liées à la furtivité et au silence sont entravées.

\textbf{Lien initial à la Première Aventure:} Choisissez parmi la liste des options ci-dessous comment vous vous êtes retrouvé impliqué dans la première aventure:

1. Il semblait y avoir des chances égales que les autres PJ ne réussissent pas, ce qui vous semblait bien.

2. Vous pensez que les tâches qui vous attendent vous présenteront des défis uniques et enrichissants.

3. L'un de vos plus grands risques n'a pas été réalisé et vous avez besoin d'argent pour vous aider à payer cette dette.

4. Vous vous êtes vanté de n'avoir jamais vu de risque que vous n'aimiez pas, c'est ainsi que vous êtes arrivé à votre point actuel.


%--------------------------
\label{sec:desclucky}\section*{Chanceux}

\textcolor{gray}{\emph{Lucky}}

Vous comptez sur le hasard et la chance opportune pour vous sortir de nombreuses situations. Quand les gens disent que quelqu'un est né sous une bonne étoile, ils parlent de vous. Lorsque vous vous essayez à quelque chose de nouveau, aussi peu familier que soit la tâche, vous trouvez le plus souvent une mesure de succès. Même lorsqu'une catastrophe survient, elle est rarement aussi grave qu'elle pourrait l'être. Le plus souvent, les petites choses semblent se dérouler comme vous le souhaitez, vous gagnez des concours et vous êtes souvent au bon endroit au bon moment.

\textbf{Réserve de Chance:} Vous disposez d'une Réserve supplémentaire appelé Chance qui commence par 3 points et a une valeur maximale de 3 points. Lorsque vous dépensez des points d'une autre Réserve, vous pouvez d'abord prendre un, quelques-uns ou tous les points de votre Réserve de chance. Lorsque vous effectuez un jet de récupération pour récupérer des points dans n'importe quelle autre Réserve, votre Réserve de chance est également rafraîchie du même nombre de points. Lorsque votre Réserve de chance est à 0 point, elle ne compte pas dans votre jauge de dégâts. Facilitateur.

\textbf{Avantage:} Quand vous utilisez 1 XP pour relancer un d20 pour n'importe quel jet dont l'effet n'affecte que vous, ajoutez 3 au résultat au nouveau jet de dé.

\textbf{Lien initial à la Première Aventure:} Choisissez parmi la liste des options ci-dessous comment vous vous êtes retrouvé impliqué dans la première aventure:

1. Sachant que les gens chanceux remarquent et profitent activement des opportunités, vous vous êtes impliqué dans votre première aventure par choix.

2. Vous êtes vraiment rentré dans quelqu'un d'autre au cours de cette aventure par pure chance.

3. Vous avez trouvé une mallette au bord de la route. Il était en mauvais état, mais à l'intérieur vous avez trouvé de nombreux documents étranges qui vous ont conduit jusqu'ici.

4. Votre chance vous a sauvé lorsque vous avez évité un véhicule roulant à grande vitesse par une chute fortuite à travers une ouverture dans le sol (un trou d'égout, si dans un cadre moderne). Sous terre, vous avez trouvé quelque chose que vous ne pouviez ignorer.


%--------------------------
\label{sec:descchaotic}\section*{Chaotique}

\textcolor{gray}{\emph{Chaotic}}

Le danger ne signifie pas grand-chose pour vous, principalement parce que vous ne pensez pas beaucoup aux répercussions. En fait, vous aimez créer des surprises, rien que pour voir ce qui va se passer. Plus le résultat est inattendu, plus vous êtes heureux. Parfois, vous êtes particulièrement maniaque et, pour le bien de vos compagnons, vous vous empêchez de prendre des mesures dont vous savez qu'elles mèneront au désastre.

\textbf{Tumultueux:} +4 à votre Réserve de Célérité.

\textbf{Compétence:} Vous êtes entraîné aux actions de Défense d'Intellect.

\textbf{Chaotique:} Une fois après chaque jet de récupération de dix heures, si le premier résultat ne vous plaît pas, vous pouvez relancer un jet de dé de votre choix. Si vous le faites, et quel que soit le résultat, le MJ vous présente une intrusion du MJ.

\textbf{Inaptitude:} Votre corps est un peu usé par les excès occasionnels. Les tâches de défense de Puissance sont désavantagées.

\textbf{Lien initial à la Première Aventure:} Choisissez parmi la liste des options ci-dessous comment vous vous êtes retrouvé impliqué dans la première aventure:

1. Un autre PJ vous a recruté alors que vous aviez un bon comportement, avant de réaliser à quel point vous étiez chaotique.

2. Vous avez des raisons de croire qu'être avec les autres PJ vous aidera à prendre le contrôle de votre comportement erratique.

3. Un autre PJ vous a libéré de captivité, et pour le remercier, vous vous êtes porté volontaire pour l'aider.

4. Vous n'avez aucune idée de la façon dont vous avez rejoint les PJ. Vous continuez simplement avec cela pour le moment jusqu'à ce que les réponses se présentent.


%--------------------------
\label{sec:desccharming}\section*{Charmant}

\textcolor{gray}{\emph{Charming}}

Vous êtes un beau parleur et un charmeur. Que ce soit par des moyens apparemment surnaturels ou simplement par des mots, vous pouvez convaincre les autres de faire ce que vous souhaitez. Très probablement, vous êtes physiquement attirant ou du moins très charismatique, et les autres aiment écouter votre voix. Vous faites probablement attention à votre apparence et restez bien soigné. Vous vous faites facilement des amis. *Vous jouez sur la facette de la personnalité de votre statistique Intellect ; l'intelligence n'est pas votre point fort (translation to be confirmed)*. Vous êtes sympathique, mais pas nécessairement studieux ou volontaire.

\textbf{Personable:} +2 à votre Réserve d'Intellect.

\textbf{Compétence:} Vous êtes entraîné dans toutes les tâches d'intéractions sociales positives ou plaisantes.

\textbf{Compétence:} Vous êtes entraîné quand vous uitliser une capacité spéciale qui influence l'esprit des autres.

\textbf{Contact:} Vous avez un contact important qui occupe une position influente, comme un noble mineur, le capitaine de la garde/police de la ville ou le chef d'une grande bande de voleurs. Vous et le MJ devriez régler les détails ensemble.

\textbf{Inaptitude:} Vous n'avez jamais été doué pour étudier ou retenir les faits. Toute tâche impliquant des connaissances, des connaissances ou une compréhension est désavantagée.

\textbf{Inaptitude:} Votre volonté n'est pas votre point fort. Les actions de défense pour résister à des attaques mentales sont désavantagées.

\textbf{Equipement Supplémentaire:} Vous avez réussi à obtenir des réductions et des bonus décents ces dernières semaines. En conséquence, vous disposez de suffisamment d'argent en poche pour acheter un article à prix modéré.

\textbf{Lien initial à la Première Aventure:} Choisissez parmi la liste des options ci-dessous comment vous vous êtes retrouvé impliqué dans la première aventure:

1. Vous avez convaincu l'un des autres PJ de vous dire ce qu'il faisait.

2. Vous avez tout organisé et convaincu les autres de vous rejoindre.

3. L'un des autres PJ vous a rendu service, et maintenant vous remboursez cette obligation en l'aidant dans la tâche à accomplir.

4. Il y a une récompense en jeu et vous avez besoin d'argent.


%--------------------------
\label{sec:desccruel}\section*{Cruel}

\textcolor{gray}{\emph{Cruel}}

Le malheur et la souffrance ne vous touchent pas. Lorsqu'une autre personne endure des difficultés, vous avez du mal à vous en soucier, et vous pouvez même apprécier la douleur et les difficultés que cette personne éprouve si elle vous a fait du mal dans le passé. Votre côté cruel peut provenir de l'amertume provoquée par vos propres luttes et déceptions. Vous pourriez être un pragmatique acharné, faisant ce que vous estimez devoir faire même si les autres sont pires à cause de cela. Ou vous pourriez être un sadique, se réjouissant de la douleur que vous infligez.Être cruel ne fait pas nécessairement de vous un méchant. Votre cruauté peut être réservée à ceux qui vous contrarient ou à d'autres personnes qui vous sont utiles. Vous êtes peut-être devenu cruel à la suite d'une expérience extrêmement horrible. Les abus et la torture, par exemple, peuvent priver la personne de toute compassion pour les autres êtres vivants.De plus, vous n'avez pas besoin d'être cruel dans toutes les situations. En fait, les autres pourraient vous considérer comme aimable, amical et même serviable. Mais lorsque vous êtes en colère ou frustré, votre double nature se révèle, et ceux qui ont mérité votre mépris risquent d'en souffrir.

\textbf{Calculateur:} +2 à votre Réserve d'Intellect.

\textbf{Cruauté:} Lorsque vous utilisez la force, vous pouvez choisir de mutiler ou de lui infliger des blessures douloureuses pour prolonger la souffrance de votre ennemi. Chaque fois que vous infligez des dégâts, vous pouvez choisir d'infliger 2 points de dégâts de moins pour faciliter votre prochaine attaque contre cet ennemi.

\textbf{Compétence:} Vous êtes entraîné dans les tâches en relation avec in tasks lié à la tromperie, à l'intimidation et à la persuasion lorsque vous interagissez avec des personnages éprouvant une douleur physique ou émotionnelle.

\textbf{Inaptitude:} Vous avez du mal à vous connecter avec les autres, à comprendre leurs Focus ou à partager leurs sentiments. Toute tâche visant à déterminer les Focus, les sentiments ou les dispositions d'un autre personnage est désavantagée.

\textbf{Equipement Supplémentaire:} Vous possédez un souvenir précieux de la dernière personne que vous avez détruite. Le prix du souvenir est modéré et vous pouvez le vendre ou l'échanger contre un objet de valeur égale ou moindre.

\textbf{Lien initial à la Première Aventure:} Choisissez parmi la liste des options ci-dessous comment vous vous êtes retrouvé impliqué dans la première aventure:

1. Vous pensez que vous pourriez obtenir un avantage à long terme en aidant les autres PJ et que vous pourrez peut-être utiliser cet avantage contre vos ennemis.

2. En rejoignant les PJ, vous voyez une opportunité d'accroître votre pouvoir et votre statut personnels aux dépens des autres.

3. Vous espérez rendre la vie d'un autre PJ plus difficile en rejoignant le groupe.

4. Rejoindre les PJ vous donne l'opportunité d'échapper à la justice pour un crime que vous avez commis.


%--------------------------
\label{sec:desccreative}\section*{Créatif}

\textcolor{gray}{\emph{Creative}}

Peut-être avez-vous un cahier dans lequel vous notez vos idées afin de pouvoir les développer plus tard. Peut-être vous envoyez-vous par courrier électronique des idées qui vous frappent à l'improviste afin de pouvoir les trier dans un document électronique. Ou peut-être que vous vous asseyez simplement, regardez votre écran et, par une force de volonté incroyable, produisez quelque chose à partir de rien. Quelle que soit la manière dont votre don fonctionne, vous êtes créatif : vous codez, écrivez, composez, sculptez, concevez, dirigez ou créez de toute autre manière des récits qui captivent les autres avec votre vision.

\textbf{Inventive:} +2 à votre Réserve d'Intellect.

\textbf{Original:} Vous proposez toujours quelque chose de nouveau. Vous êtes entraîné dans toute tâche liée à la création d'un récit (comme une histoire, une pièce de théâtre ou un scénario). Cela inclut la tromperie, si la tromperie implique un récit que vous êtes capable de raconter.

\textbf{Compétence:} Vous êtes naturellement inventif. Vous êtes entraîné à une compétence créative spécifique de votre choix : écriture, codage informatique, composition musicale, peinture, dessin, etc.

\textbf{Compétence:} Vous aimez résoudre des énigmes, etc. Vous êtes entraîné aux tâches de résolution d'énigmes.

\textbf{Compétence:} Pour être créatif, il faut toujours apprendre. Vous êtes entraîné à toute tâche qui implique la découverte de quelque chose de nouveau, par exemple lorsque vous fouillez dans une bibliothèque, une banque de données, des archives d'actualités ou une collection de connaissances similaire.

\textbf{Inaptitude:} Vous êtes inventif mais pas charmant. Toutes les tâches liées à une interaction sociale agréable sont désavantagées.

\textbf{Lien initial à la Première Aventure:} Choisissez parmi la liste des options ci-dessous comment vous vous êtes retrouvé impliqué dans la première aventure:

1. Vous faisiez des recherches pour un projet et avez convaincu les PJ de vous accompagner.

2. Vous recherchez de nouveaux marchés pour les résultats de votre production créative.

3. Vous êtes tombé sur la mauvaise clientèle, mais elle a grandi en vous.

4. Une vie créative est souvent confrontée à des obstacles financiers. Vous avez rejoint les PJ parce que vous espériez que cela serait rentable.


%--------------------------
\label{sec:descinquisitive}\section*{Curieux}

\textcolor{gray}{\emph{Inquisitive}}

Le monde est vaste et mystérieux, avec des merveilles et des secrets qui vous surprendront pendant plusieurs vies. Vous ressentez un tiraillement dans votre cœur, un appel à explorer les ruines des civilisations passées, à découvrir de nouveaux peuples, de nouveaux lieux et toutes les merveilles bizarres que vous pourriez trouver en cours de route. Cependant, même si vous ressentez le besoin de parcourir le monde, vous savez qu'il existe de nombreux dangers et vous prenez des précautions pour vous assurer que vous êtes prêt à toute éventualité. La recherche, la préparation et la préparation vous aideront à vivre assez longtemps pour voir tout ce que vous voulez voir et faire tout ce que vous voulez faire.Vous avez probablement à tout moment sur vous une douzaine de livres et de récits de voyage sur le monde. Lorsque vous ne prenez pas la route et ne regardez pas autour de vous, vous passez votre temps le nez dans un livre, apprenant tout ce que vous pouvez sur l'endroit où vous allez afin de savoir à quoi vous attendre une fois sur place.

\textbf{Intelligent:} +4 à votre Réserve d'Intellect.

\textbf{Compétence:} Vous avez envie d'apprendre. Vous êtes entraîné à toute tâche qui implique d'apprendre quelque chose de nouveau, que vous parliez à un local pour obtenir des informations ou que vous fouilliez dans de vieux livres pour découvrir des traditions.

\textbf{Compétence:} Vous avez fait une étude du monde. Vous êtes entraîné à toute tâche impliquant la géographie ou l'histoire.

\textbf{Inaptitude:} Vous avez tendance à vous concentrer sur les détails, ce qui vous rend quelque peu inconscient de ce qui se passe autour de vous. Toute tâche visant à entendre ou à remarquer les dangers autour de vous est entravée.

\textbf{Inaptitude:} Lorsque vous voyez quelque chose d'intéressant, vous hésitez en prenant en compte tous les détails. Les actions d'initiative (pour déterminer qui commence le combat en premier) sont désavantagées.

\textbf{Equipement Supplémentaire:} Vous disposez de trois livres sur les sujets que vous choisissez.

\textbf{Lien initial à la Première Aventure:} Choisissez parmi la liste des options ci-dessous comment vous vous êtes retrouvé impliqué dans la première aventure:

1. L'un des PJ vous a approché pour obtenir des informations relatives à la mission, après avoir entendu dire que vous étiez un expert.

2. Vous avez toujours voulu voir l'endroit où vont les autres PJ.

3. Vous étiez intéressé par ce que faisaient les autres PJ et avez décidé de les suivre.

4. L'un des PJ vous fascine, peut-être en raison d'une capacité spéciale ou étrange dont il dispose.


%--------------------------
\label{sec:descmechanical}\section*{Doué pour la mécanique}

\textcolor{gray}{\emph{Mechanical}}

Vous avez un talent particulier avec les machines de toutes sortes, et vous savez les comprendre et, le cas échéant, les réparer. Peut-être êtes-vous un peu un inventeur, créant de nouvelles machines de temps en temps. Vous êtes appelé "technophile", "tech", "mech", "gear-head", "motor-head" ou l'un des nombreux autres surnoms. Les mécaniciens portent généralement des vêtements de travail pratiques et transportent de nombreux outils.

\textbf{Intelligent:} +2 à votre Réserve d'Intellect.

\textbf{Compétence:} Vous êtes entraîné dans toutes les actions impliquant l'identification ou la compréhension des machines.

\textbf{Compétence:} Vous êtes entraîné dans toutes les actions impliquant l'utilisation, la réparation ou la fabrication de machines.

\textbf{Equipement Supplémentaire:} Vous commencez avec une variété de machines-outils.

\textbf{Lien initial à la Première Aventure:} Choisissez parmi la liste des options ci-dessous comment vous vous êtes retrouvé impliqué dans la première aventure:

1. Alors que vous répariez une machine à proximité, vous avez entendu les autres PJ parler.

2. Vous avez besoin d'argent pour acheter des outils et des pièces.

3. Il était clair que la mission ne pourrait pas réussir sans vos compétences et connaissances.

4. Un autre PJ vous a demandé de les rejoindre.


%--------------------------
\label{sec:desctough}\section*{Dur-à-Cuire}

\textcolor{gray}{\emph{Tough}}

Vous êtes costaud et vous pouvez subir pas mal de chocs physiques. Vous avez des épaules larges et une machoire carrée. Les durs-à-cuir ont souvent des cicatrices visibles.

\textbf{Resilient:} +1 à l'Armure.

\textbf{Bonne santé:} Ajoutre 1 à vos jets de récupération.

\textbf{Compétence:} Vous êtes entraîné dans les actions de défense de Puissance.

\textbf{Equipement Supplémentaire:} Vous avez une arme légère supplémentaire.

\textbf{Lien initial à la Première Aventure:} Choisissez parmi la liste des options ci-dessous comment vous vous êtes retrouvé impliqué dans la première aventure:

1. Vous agissez comme garde du corps pour l'un des autres PJ.

2. L'un des PJ est votre frère et sœur, et vous êtes venu pour veiller sur eux.

3. Vous avez besoin d'argent parce que votre famille est endettée.

4. Vous êtes intervenu pour défendre l'un des PJ lorsque ce personnage a été menacé. En discutant avec eux par la suite, vous avez entendu parler de la tâche du groupe.


%--------------------------
\label{sec:descempathic}\section*{Empathique}

\textcolor{gray}{\emph{Empathic}}

Les autres sont des livres ouverts pour vous. Vous avez peut-être le don de lire les récits d'une personne, ces mouvements subtils qui traduisent l'humeur et la disposition d'un individu. Ou alors, vous pouvez recevoir des informations de manière plus directe, en ressentant les émotions d'une personne comme s'il s'agissait de choses tangibles, des sensations qui effleurent légèrement votre esprit. Votre don pour l'empathie vous aide à naviguer dans les situations sociales et à les contrôler pour éviter les malentendus et empêcher que des conflits inutiles n'éclatent.Le bombardement constant d'émotions de la part de votre entourage a probablement des conséquences néfastes. Vous pourriez évoluer selon l'humeur du moment, passant d'un bonheur vertigineux à un chagrin amer sans aucun avertissement. Ou vous pourriez vous fermer et rester impénétrable aux yeux des autres par sentiment d'auto-préservation ou par peur inconsciente que tout le monde puisse apprendre ce que vous ressentez vraiment.

\textbf{Overt d'esprit:} +4 à votre Réserve d'Intellect.

\textbf{Compétence:} Vous êtes entraîné dans les tâches impliquant de ressentir d'autres émotions, de discerner des dispositions ou d'avoir une idée des gens qui vous entourent.

\textbf{Compétence:} Vous êtes entraîné dans toutes les tâches impliquant une interaction sociale, agréable ou non.

\textbf{Inaptitude:} Être si réceptif aux pensées et aux humeurs des autres vous rend vulnérable à tout ce qui attaque votre esprit. Les jets de défense intellectuelle sont désavantagés.

\textbf{Lien initial à la Première Aventure:} Choisissez parmi la liste des options ci-dessous comment vous vous êtes retrouvé impliqué dans la première aventure:

1. Vous avez senti l'engagement des autres PJ dans la tâche et vous vous êtes senti poussé à les aider.

2. Vous avez établi un lien étroit avec un autre PJ et vous ne supportez pas de vous en séparer.

3. Vous avez senti quelque chose d'étrange chez l'un des PJ et avez décidé de rejoindre le groupe pour voir si vous pouvez le ressentir à nouveau et découvrir la vérité.

4. Vous avez rejoint les PJ pour échapper à une relation désagréable ou à un environnement négatif.


%--------------------------
\label{sec:deschardy}\section*{Endurant}

\textcolor{gray}{\emph{Hardy}}

Votre corps a été construit pour supporter les abus. Que vous buviez des boissons fortes tout en tenant le bar dans votre troquet préféré ou que vous échangez des coups avec un voyou dans une ruelle, vous continuez, ignorant les blessures et les blessures qui pourraient ralentir ou neutraliser une personne inférieure. Ni la faim ni la soif, ni la chair coupée ni les os brisés ne peuvent vous arrêter. Vous continuez simplement à surmonter la douleur et à continuer.Aussi en forme et en bonne santé que vous soyez, les signes d'usure se manifestent dans la myriade de cicatrices qui sillonnent votre corps, votre nez trois fois cassé, vos oreilles en chou-fleur et de nombreuses autres défigurations que vous portez avec fierté.

\textbf{Puissant:} +4 à votre Réserve de Puissance.

\textbf{Guérit rapidement:} Vous divisez par deux le temps nécessaire pour effectuer un jet de récupération (minimum une action).

\textbf{Quasiment Inarrêtable:} Tant que vous êtes diminué sur le suivi des dégâts, vous fonctionnez comme si vous étiez en bonne santé. Pendant que vous êtes handicapé, vous fonctionnez comme si vous étiez diminué. En d'autres termes, vous ne subissez pas les effets d'une déficience jusqu'à ce que vous deveniez handicapé, et vous ne subissez jamais les effets d'une déficience. Vous mourrez quand même si tous vos Réserves de statistiques sont à 0.

\textbf{Compétence:} Vous êtes entraîné auc actions de défense de Puissance.

\textbf{Inaptitude:} Vous êtes grand, fort, et lent à réagir. Toute tâche impliquant de l'initiative est désavantagée.

\textbf{Lourd:} Lorsque vous appliquez un Effort lors d'un jet de Célérité, vous devez dépenser 1 point supplémentaire de votre réserve de Célérité.

\textbf{Lien initial à la Première Aventure:} Choisissez parmi la liste des options ci-dessous comment vous vous êtes retrouvé impliqué dans la première aventure:

1. Les PJ vous ont recruté après avoir pris connaissance de votre réputation de survivant.

2. Vous avez rejoint les PJ parce que vous voulez ou avez besoin d'argent.

3. Les PJ vous ont proposé un défi égal à votre puissance physique.

4. Vous pensez que la seule façon pour les PJ de réussir est que vous soyez là pour les protéger.


%--------------------------
\label{sec:desclearned}\section*{Erudit}

\textcolor{gray}{\emph{Learned}}

Vous avez étudié seul ou avec un moniteur. Vous connaissez beaucoup de choses et êtes expert sur quelques sujets, comme l'histoire, la biologie, la géographie, la mythologie, la nature ou tout autre domaine d'étude. Les personnages érudits transportent généralement quelques livres avec eux et passent leur temps libre à lire.

\textbf{Intelligent:} +2 à votre Réserve d'Intellect.

\textbf{Compétence:} Vous êtes entraîné dans trois domaines de connaissances de votre choix.

\textbf{Inaptitude:} Vous avez peu d'aptitudes sociales. Toute tâche impliquant du charme, de la persuasion ou de l'étiquette est désavantagée.

\textbf{Equipement Supplémentaire:} Vous disposez de deux livres supplémentaires sur des sujets de votre choix.

\textbf{Lien initial à la Première Aventure:} Choisissez parmi la liste des options ci-dessous comment vous vous êtes retrouvé impliqué dans la première aventure:

1. Un des autres PJ vous a demandé de venir grâce à vos connaissances.

2. Vous avez besoin d'argent pour financer vos études.

3. Vous pensiez que cette tâche pourrait mener à des découvertes importantes et intéressantes.

4. Un collègue vous a demandé de participer à la mission en guise de faveur.


%--------------------------
\label{sec:descweird}\section*{Etrange}

\textcolor{gray}{\emph{Weird}}

Vous n'êtes pas comme les autres, et ça vous va. Les gens ne semblent pas vous comprendre – ils semblent même découragés par vous – mais peu importe ? Vous comprenez le monde mieux qu'eux parce que vous êtes bizarre, tout comme le monde dans lequel vous vivez. Le concept de « bizarre » vous est bien connu. Des appareils étranges, des lieux anciens, des créatures bizarres, des tempêtes qui peuvent vous transformer, des champs d'énergie vivants, des conspirations, des extraterrestres et des choses que la plupart des gens ne peuvent même pas nommer peuplent le monde, et vous prospérez grâce à eux. Vous avez un attachement particulier à tout cela, et plus vous en découvrez sur l'étrangeté du monde, plus vous pourriez en découvrir sur vous-même.Les personnages étranges peuvent être des mutants ou des personnes nées avec des qualités étranges, mais parfois ils ont commencé « normaux » et ont adopté l'étrange par choix.

\textbf{Lumière Intérieure:} +2 à votre Réserve d'Intellect.

\textbf{Bizarrerie Physique:} Vous avez un aspect physique unique qui est bizarre. Selon le paramètre, cela peut varier considérablement. Vous pourriez avoir des cheveux violets ou des pointes métalliques sur la tête. Peut-être que vos mains ne se connectent pas à vos bras, même si elles bougent comme si c'était le cas. Peut-être qu'un troisième œil regarde du côté de votre tête, ou que des vrilles superflues poussent dans votre dos. Quoi qu'il en soit, votre bizarrerie peut être une mutation, un trait surnaturel (une bénédiction ou une malédiction), une caractéristique sans explication, ou simplement un tatouage vraiment sauvage qui attire beaucoup d'attention.

\textbf{Un sens pour l'étrange:} Parfois, à la discrétion du MJ, des choses étranges liées au surnaturel ou à ses effets sur le monde semblent vous interpeller. Vous pouvez les sentir de loin, et si vous vous approchez à distance d'une telle chose, vous pouvez sentir si elle est ouvertement dangereuse ou non.

\textbf{Compétence:} Vous êtes entraîné dans les connaissances surnaturelles.

\textbf{Inaptitude:} Les gens vous trouvent énervant. Toutes les tâches liées à une interaction sociale agréable sont entravées.

\textbf{Lien initial à la Première Aventure:} Choisissez parmi la liste des options ci-dessous comment vous vous êtes retrouvé impliqué dans la première aventure:

1. Cela semblait bizarre, alors pourquoi pas ?

2. Que les autres PJ s'en rendent compte ou non, leur mission est liée à quelque chose d'étrange que vous connaissez, alors vous vous êtes impliqué.

3. En tant qu'expert en matière d'étrangeté, vous avez été spécifiquement recruté par les autres PJ.

4. Vous vous êtes senti attiré par l'idée de rejoindre les autres PJ, mais vous ne savez pas pourquoi.


%--------------------------
\label{sec:descexiled}\section*{Exilé}

\textcolor{gray}{\emph{Exiled}}

Vous avez parcouru un chemin long et solitaire, laissant votre maison et votre vie derrière vous. Vous avez peut-être commis un crime odieux, quelque chose de si horrible que votre peuple vous a forcé à partir, et si vous osez revenir, vous risquez la mort. Vous avez peut-être été accusé d'un crime que vous n'avez pas commis et vous devez maintenant payer le prix du mauvais acte de quelqu'un d'autre. Votre exil peut être le résultat d'une gaffe sociale : peut-être avez-vous fait honte à votre famille ou à un ami, ou vous êtes-vous embarrassé devant vos pairs, une autorité ou quelqu'un que vous respectez. Quelle que soit la raison, vous avez laissé votre ancienne vie derrière vous et vous efforcez maintenant d'en refaire une nouvelle.

\textbf{Autonome:} +2 à votre Réserve de Puissance.

\textbf{Solitaire:} Vous ne gagnez aucun avantage lorsque vous recevez de l'aide pour une tâche d'un autre personnage entraîné ou spécialisé dans cette tâche.

\textbf{Compétence:} Vous êtes entraîné dans toutes les tâches impliquant de se faufiler.

\textbf{Compétence:} Vous êtes entraîné dans toutes les tâches impliquant la recherche de nourriture, la chasse et la recherche d'endroits sûrs pour se reposer ou se cacher.

\textbf{Inaptitude:} Vivre seul aussi longtemps que vous avez pu le faire vous rend lent à faire confiance aux autres et vous rend maladroit dans les situations sociales. Toute tâche impliquant une interaction sociale est entravée.

\textbf{Equipement Supplémentaire:} Vous avez un souvenir de votre passé : une vieille photo, un médaillon avec quelques mèches de cheveux à l'intérieur ou un briquet que vous a offert une personne importante. Vous gardez l'objet à portée de main et vous le retirez pour vous aider à vous souvenir de meilleurs moments.

\textbf{Lien initial à la Première Aventure:} Choisissez parmi la liste des options ci-dessous comment vous vous êtes retrouvé impliqué dans la première aventure:

1. Les autres PJ ont gagné votre confiance en vous aidant lorsque vous en aviez besoin. Vous les accompagnez pour les rembourser.

2. En explorant par vous-même, vous avez découvert quelque chose d'étrange. Lorsque vous vous êtes rendu dans un lieu particulier, les PJ étaient les seuls à vous croire, et ils vous ont accompagné pour vous aider à résoudre le problème.

3. L'un des autres PJ vous rappelle quelqu'un que vous avez connu.

4. Vous êtes fatigué de votre isolement. Rejoindre les autres PJ vous donne une chance d'appartenir.


%--------------------------
\label{sec:descstrong}\section*{Fort}

\textcolor{gray}{\emph{Strong}}

Vous êtes extrêmement fort et physiquement puissant, et vous utilisez bien ces qualités, que ce soit par la violence ou par des prouesses. Vous avez probablement une carrure musclée et des muscles impressionnants.

\textbf{Très Puissant:} +4 à votre Réserve de Puissance.

\textbf{Compétence:} Vous êtes entraîné dans toutes les actions impliquant la destruction d'objets inanimés.

\textbf{Compétence:} Vous êtes entraîné dans toutes les actions de saut.

\textbf{Equipement Supplémentaire:} Vous disposez d'une arme moyenne ou lourde supplémentaire.

\textbf{Lien initial à la Première Aventure:} Choisissez parmi la liste des options ci-dessous comment vous vous êtes retrouvé impliqué dans la première aventure:

1. Contre votre bon jugement, vous avez rejoint les autres PJ parce que vous avez vu qu'ils étaient en danger.

2. L'un des autres PJ vous a convaincu que rejoindre le groupe serait dans votre intérêt.

3. Vous avez peur de ce qui pourrait arriver si les autres PC tombaient en panne.

4. Il y a une récompense en jeu et vous avez besoin d'argent.


%--------------------------
\label{sec:descmad}\section*{Fou}

\textcolor{gray}{\emph{Mad}}

Vous avez approfondi des sujets que les gens n'étaient pas censés connaître. Vous possédez des connaissances dans des domaines qui dépassent la portée de la plupart des gens, mais ces connaissances ont un prix terrible. Vous êtes probablement dans une forme physique douteuse et souffrez occasionnellement de tics nerveux. Vous marmonnez parfois sans vous en rendre compte.

\textbf{Bien informé:} +4 à votre Réserve d'Intellect.

\textbf{Eclairs de Génie:} Chaque fois qu'une telle connaissance est appropriée, le MJ vous fournit des informations bien qu'il n'y ait aucune explication claire sur la manière dont vous pourriez savoir une telle chose. Ceci est à la discrétion du MJ, mais cela devrait se produire aussi souvent qu'une fois par session.

\textbf{Comportement ératique:} Vous avez tendance à agir de manière erratique ou irrationnelle. Lorsque vous êtes en présence d'une découverte majeure ou soumis à un stress important (comme une menace physique sérieuse), le MJ peut introduire une intrusion du MJ qui oriente votre prochaine action sans vous attribuer d'XP. Vous pouvez toujours payer 1 XP pour refuser l'intrusion. L'influence du MJ est la manifestation de votre folie et c'est donc toujours quelque chose que vous ne feriez probablement pas autrement, mais elle ne vous est pas directement et évidemment nuisible, sauf circonstances atténuantes. (Par exemple, si un ennemi surgit soudainement de l'obscurité, vous pourriez passer le premier tour à babiller de manière incohérente ou à crier le nom de votre premier véritable amour.)

\textbf{Compétence:} Vous êtes entraîné à un domaine de connaissance (probablement quelque chose de bizarre ou d'ésotérique).

\textbf{Inaptitude:} Votre esprit est assez fragile. Les tâches visant à résister aux attaques mentales sont désavantagées.

\textbf{Lien initial à la Première Aventure:} Choisissez parmi la liste des options ci-dessous comment vous vous êtes retrouvé impliqué dans la première aventure:

1. Des voix dans votre tête vous ont dit de partir.

2. Vous avez tout déclenché et convaincu les autres de vous rejoindre.

3. L'un des autres PJ a obtenu un livre de connaissances pour vous, et maintenant vous lui rendez cette faveur en l'aidant dans la tâche à accomplir.

4. Vous vous sentez contraint par une intuition inexplicable.


%--------------------------
\label{sec:descstealthy}\section*{Furtif}

\textcolor{gray}{\emph{Stealthy}}

Vous êtes sournois, glissant et rapide. Ces talents vous aident à vous cacher, à vous déplacer tranquillement et à réaliser des tours qui nécessitent un tour de passe-passe. Très probablement, vous êtes nerveux et petit. Cependant, vous n'êtes pas vraiment un sprinter --- vous êtes plus adroit que rapide.

\textbf{Rapide:} +2 à votre Réserve de Célérité.

\textbf{Compétence:} Vous êtes entraîné dans toutes les tâches furtives.

\textbf{Compétence:} Vous êtes entraîné dans toutes les interactions impliquant des mensonges ou des supercheries.

\textbf{Compétence:} Vous êtes entraîné dans toutes les capacités spéciales impliquant des illusions ou des supercheries.

\textbf{Inaptitude:} Vous êtes sournois mais pas rapide. Toutes les tâches liées au mouvement sont gênées.

\textbf{Lien initial à la Première Aventure:} Choisissez parmi la liste des options ci-dessous comment vous vous êtes retrouvé impliqué dans la première aventure:

1. Vous avez tenté de voler l'un des autres PC. Ce personnage vous a attrapé et vous a forcé à les accompagner.

2. Vous suiviez l'un des autres PJ pour des raisons qui vous sont propres, ce qui vous a amené à l'action.

3. Un employeur de PNJ vous a secrètement payé pour vous impliquer.

4. Vous avez entendu les autres PJ parler d'un sujet qui vous intéressait, vous avez donc décidé d'approcher le groupe.


%--------------------------
\label{sec:desckind}\section*{Gentil}

\textcolor{gray}{\emph{Kind}}

Il a toujours été facile pour vous de voir les choses du point de vue des autres. Cette capacité vous a rendu sensible à ce qu'ils veulent ou ont réellement besoin. De votre point de vue, vous appliquez simplement le vieux proverbe selon lequel « il est plus facile d'attraper des mouches avec du miel qu'avec du vinaigre », mais d'autres voient simplement votre comportement comme de la gentillesse. Bien sûr, être gentil prend du temps, et le vôtre est limité. Vous avez appris qu'une petite fraction des gens ne mérite pas votre temps ou votre gentillesse : les vrais sadiques, narcissiques et autres personnes similaires ne feront que gaspiller votre énergie. Vous les traitez donc rapidement, réservant votre gentillesse à ceux qui la méritent et peuvent bénéficier de votre attention.

\textbf{Emotionellement Intuitif:} +2 à votre Réserve d'Intellect.

\textbf{Compétence:} Vous savez ce que c'est de se mettre à la place de quelqu'un d'autre. Vous êtes entraîné dans toutes les tâches liées à une interaction sociale agréable et au discernement des dispositions des autres.

\textbf{Karma:} Parfois, des inconnus vous aident simplement. Pour obtenir l'aide d'un étranger, vous devez dépenser un jet de récupération d'une action, de dix minutes ou d'une heure (sans bénéficier de son bénéfice de guérison), en échange de quoi le MJ détermine la nature de l'aide que vous obtenez. Habituellement, un acte de gentillesse ne suffit pas à renverser complètement une mauvaise situation, mais il peut atténuer une mauvaise situation et ouvrir de nouvelles opportunités. Par exemple, si vous êtes capturé, un garde desserre légèrement vos liens, vous apporte de l'eau ou vous délivre un message.

\textbf{Inaptitude:} Être gentil comporte quelques risques. Toutes les tâches liées à la détection des mensonges sont désavantagées.

\textbf{Lien initial à la Première Aventure:} Choisissez parmi la liste des options ci-dessous comment vous vous êtes retrouvé impliqué dans la première aventure:

1. Un PJ avait besoin de votre aide et vous avez accepté de l'accompagner par gentillesse.

2. Vous avez donné à la mauvaise personne accès à votre argent et vous devez maintenant en récupérer une partie.

3. Vous êtes prêt à mettre votre bienveillance en route et à aider plus de personnes que vous ne le pourriez si vous ne rejoigniez pas les PJ.

4. Votre travail, qui semblait personnellement gratifiant, en est l'exact opposé. Vous rejoignez les PJ pour échapper à la corvée.


%--------------------------
\label{sec:descgraceful}\section*{Grâcieux}

\textcolor{gray}{\emph{Graceful}}

Vous avez un parfait sens de l'équilibre, vous bougez et parlez avec grâce et beauté. Vous êtes rapide, souple, flexible et adroit. Votre corps est parfaitement adapté à la danse et vous utilisez cet avantage au combat pour esquiver les coups. Vous pouvez porter des vêtements qui améliorent votre agilité de mouvement et votre sens du style.

\textbf{Agile:} +2 à votre Réserve de Célérité.

\textbf{Compétence:} Vous êtes entraîné dans toutes les tâches impliquant un équilibre et des mouvements prudents.

\textbf{Compétence:} Vous êtes entraîné dans toutes les tâches impliquant les arts du spectacle physique.

\textbf{Compétence:} Vous êtes entraîné dans toutes les tâches de défense de Célérité.

\textbf{Lien initial à la Première Aventure:} Choisissez parmi la liste des options ci-dessous comment vous vous êtes retrouvé impliqué dans la première aventure:

1. Contre votre bon sens, vous avez rejoint les autres PJ parce que vous avez vu qu'ils étaient en danger.

2. L'un des autres PJ vous a convaincu que rejoindre le groupe serait dans votre intérêt.

3. Vous avez peur de ce qui pourrait arriver si les autres PC tombaient en panne.

4. Il y a une récompense en jeu et vous avez besoin d'argent.


%--------------------------
\label{sec:deschideous}\section*{Hideux}

\textcolor{gray}{\emph{Hideous}}

Vous êtes physiquement répugnant selon presque toutes les normes humaines. Vous avez peut-être eu un accident grave, une mutation nuisible ou simplement une mauvaise chance génétique, mais vous êtes incontestablement laid.Cependant, vous avez plus que compensé votre apparence par d'autres moyens. Parce que vous devez cacher votre apparence, vous excellez à vous faufiler inaperçu ou à vous déguiser. Mais peut-être le plus important, étant ostracisé pendant que les autres socialisaient, vous avez pris le temps de grandir pour vous développer comme bon vous semble --- vous êtes devenu fort ou rapide, ou vous avez affiné votre esprit.

\textbf{Versatile:} Vous obtenez 4 points supplémentaires à répartir entre vos Réserves de statistiques.

\textbf{Compétence:} Vous êtes entraîné à l'intimidation et toute autre interaction basée sur la peur, si vous montrez votre vrai visage.

\textbf{Compétence:} Vous êtes entraîné aux tâches de déguisement et de furtivité.

\textbf{Inaptitude:} Toutes les tâches liées à une interaction sociale agréable sont désavantagées.

\textbf{Lien initial à la Première Aventure:} Choisissez parmi la liste des options ci-dessous comment vous vous êtes retrouvé impliqué dans la première aventure:

1. L'un des autres PJ s'est approché de vous alors que vous étiez déguisé, vous recrutant en croyant que vous étiez quelqu'un d'autre.

2. En rôdant aux alentours, vous avez entendu les plans des autres PJs et réalisé que vous souhaitiez participer.

3. Un des autres PJ vous a invité, mais vous vous demandez si c'était par pitié.

4. Vous avez fait preuve d'intimidation et de fanfaronnades pour vous frayer un chemin.


%--------------------------
\label{sec:deschonorable}\section*{Honorable}

\textcolor{gray}{\emph{Honorable}}

Vous êtes digne de confiance, juste et franc. Vous essayez de faire ce qui est juste, d'aider les autres et de bien les traiter. Mentir et tricher ne sont pas des moyens d'avancer : ces choses sont réservées aux faibles, aux paresseux ou aux méprisables. Vous passez probablement beaucoup de temps à réfléchir à votre honneur personnel, à la meilleure façon de le préserver et de le défendre en cas de défi. Au combat, vous êtes franc et offrez du quart à n'importe quel ennemi.Ce sentiment d'honneur vous a probablement été inculqué par un parent ou un mentor. Parfois, la distinction entre ce qui est honorable et ce qui ne l'est pas varie selon les écoles de pensée, mais dans l'ensemble, les personnes honorables peuvent s'entendre sur la plupart des aspects de ce que signifie l'honneur.

\textbf{Vigoureux:} +2 à votre Réserve de Puissance.

\textbf{Compétence:} Vous êtes entraîné aux intéractions plaisantes.

\textbf{Compétence:} Vous êtes entraîné à discerner les véritables Focus des gens ou voir à travers les mensonges.

\textbf{Lien initial à la Première Aventure:} Choisissez parmi la liste des options ci-dessous comment vous vous êtes retrouvé impliqué dans la première aventure:

1. Les objectifs des PJ semblent honorables et louables.

2. Vous voyez que ce que les autres PJ sont sur le point de faire est dangereux et vous aimeriez aider à les protéger.

3. Un des autres PJ vous a invité, ayant entendu parler de votre fiabilité.

4. Vous avez demandé poliment si vous pouviez rejoindre les autres PJ dans leur mission.


%--------------------------
\label{sec:descimpulsive}\section*{Impulsif}

\textcolor{gray}{\emph{Impulsive}}

Vous avez du mal à contenir votre enthousiasme. Pourquoi attendre quand vous pouvez simplement le faire (quoi que ce soit) et le faire ? Vous traitez les problèmes lorsqu'ils surviennent plutôt que de planifier à l'avance. Éteindre les petits incendies évite désormais qu'ils ne se transforment en un grand incendie plus tard. Vous êtes le premier à prendre des risques, à vous lancer et à donner un coup de main, à vous engager dans des passages sombres et à trouver le danger.Votre impulsivité vous cause probablement des ennuis. Alors que d'autres peuvent prendre le temps d'étudier les objets qu'ils découvrent, vous utilisez ces objets sans hésitation. Après tout, la meilleure façon d'apprendre ce qu'une chose peut faire est de l'utiliser. Lorsqu'un explorateur prudent peut regarder autour de lui et vérifier s'il y a un danger à proximité, vous devez vous empêcher physiquement d'avancer. Pourquoi tourner auour du pot alors que des événements passionnants sont à notre porté ?Les personnages impulsifs ont des ennuis. C'est leur truc, et c'est très bien. Mais si vous entraînez constamment vos collègues PJ dans des ennuis (ou pire, si vous les faites gravement blesser ou tuer), ce sera pour le moins ennuyeux. Une bonne règle de base est que l'impulsivité ne signifie pas toujours une prédilection à faire la mauvaise chose. Parfois, c'est l'envie de faire la bonne chose.

\textbf{Casse-cou:} +2 à votre Réserve de Célérité.

\textbf{Compétence:} Vous êtes entraîné dans les actions d'initiative (pour déterminer qui commence le combat en premier).

\textbf{Compétence:} Vous êtes entraîné dans les actions de défense de Célérité.

\textbf{Inaptitude:} Vous essayez n'importe quoi une fois, mais vous vous ennuyez rapidement par la suite. Toute tâche qui implique de la patience, de la volonté ou de la discipline est désavantagée.

\textbf{Lien initial à la Première Aventure:} Choisissez parmi la liste des options ci-dessous comment vous vous êtes retrouvé impliqué dans la première aventure:

1. Vous avez entendu ce que faisaient les autres PJ et avez soudainement décidé de les rejoindre.

2. Vous avez rassemblé tout le monde après avoir entendu des rumeurs concernant quelque chose d'intéressant que vous vouliez voir ou faire.

3. Vous avez dépensé tout votre argent et vous vous retrouvez maintenant à court d'argent.

4. Vous avez des ennuis pour avoir agi de manière imprudente. Vous rejoignez les autres PC car ils offrent une issue à votre problème.


%--------------------------
\label{sec:descbrash}\section*{Impétieux}

\textcolor{gray}{\emph{Brash}}

Vous êtes du genre affirmé, confiant en vos capacités, énergique et peut-être un peu irrévérencieux envers les idées avec lesquelles vous n'êtes pas d'accord. Certaines personnes vous qualifient d'audacieux et de courageux, mais ceux que vous avez mis à leur place pourraient vous qualifier d'enflé et d'arrogant. Peu importe. Ce n'est pas dans votre nature de vous soucier de ce que les autres pensent de vous, à moins que ces personnes ne soient vos amis ou votre famille. Même quelqu'un d'aussi impétueux que vous sait que les amis doivent parfois passer en premier.

\textbf{Energetique:} +2 à votre Réserve de Célérité.

\textbf{Compétence:} Vous êtes entraîné à l'initiative.

\textbf{Audacieux:} Vous êtes entraîné à toutes les actions qui impliquent de dépasser ou d'ignorer les effets de la peur ou de l'intimidation.

\textbf{Lien initial à la Première Aventure:} Choisissez parmi la liste des options ci-dessous comment vous vous êtes retrouvé impliqué dans la première aventure:

1. Vous avez remarqué quelque chose de bizarre et, sans trop réfléchir, vous avez sauté à pieds joints.

2. Vous êtes arrivé là où vous êtes à cause d'un défi parce que, hé, vous ne reculez pas devant les défis.

3. Quelqu'un vous a appelé, mais au lieu de vous lancer dans une bagarre, vous vous êtes retrouvé dans votre situation actuelle.

4. Vous avez dit à votre ami que rien ne pouvait vous effrayer et que rien de ce que vous verriez ne vous ferait changer d'avis. Ils vous ont amené à votre point actuel.


%--------------------------
\label{sec:descintelligent}\section*{Intelligent}

\textcolor{gray}{\emph{Intelligent}}

Vous êtes plutôt intelligent. Votre mémoire est vive et vous comprenez facilement les concepts avec lesquels d'autres pourraient avoir du mal. Cette aptitude ne signifie pas nécessairement que vous avez suivi des années d'éducation formelle, mais que vous avez beaucoup appris dans votre vie, principalement parce que vous maîtrisez rapidement les choses et que vous retenez beaucoup de choses.

\textbf{Intelligent:} +2 à votre Réserve d'Intellect.

\textbf{Compétence:} Vous êtes entraîné dans un domaine de connaissance de votre choix.

\textbf{Compétence:} Vous êtes entraîné dans toutes les actions qui impliquent de se souvenir ou de mémoriser des choses que vous vivez directement. Par exemple, au lieu de bien vous souvenir des détails géographiques que vous avez lus dans un livre, vous pouvez vous souvenir d'un chemin à travers un ensemble de tunnels que vous avez déjà explorés.

\textbf{Lien initial à la Première Aventure:} Choisissez parmi la liste des options ci-dessous comment vous vous êtes retrouvé impliqué dans la première aventure:

1. L'un des autres PJ vous a demandé votre avis sur la mission, sachant que si vous pensiez que c'était une bonne idée, c'était probablement le cas.

2. Vous avez vu la valeur de ce que faisaient les autres PJ.

3. Vous pensiez que cette tâche pourrait mener à des découvertes importantes et intéressantes.

4. Un collègue vous a demandé de participer à la mission en guise de faveur.


%--------------------------
\label{sec:descintuitive}\section*{Intuitif}

\textcolor{gray}{\emph{Intuitive}}

Vous êtes souvent chatouillé par le sentiment de savoir ce que quelqu'un va dire, comment il va réagir ou comment les événements pourraient se dérouler. Peut-être avez-vous un sens mutant, peut-être pouvez-vous voir quelques instants à venir dans le temps, ou peut-être êtes-vous simplement doué pour lire les gens et extrapoler une situation. Quoi qu'il en soit, beaucoup de ceux qui vous regardent dans les yeux détournent immédiatement le regard, comme s'ils avaient peur de ce que vous pourriez voir dans leur expression.

\textbf{Inné:} +2 à votre Réserve d'Intellect.

\textbf{Compétence:} Vous êtes entraîné aux tâches de perception.

\textbf{Sais quoi faire:} Vous pouvez agir immédiatement, même si ce n'est pas votre tour. Ensuite, lors de votre prochain tour normal, toute action que vous entreprenez est désavantagée. Vous pouvez le faire une seule fois, bien que la capacité soit renouvelée à chaque fois que vous effectuez un jet de récupération.

\textbf{Lien initial à la Première Aventure:} Choisissez parmi la liste des options ci-dessous comment vous vous êtes retrouvé impliqué dans la première aventure:

1. Vous saviez simplement que vous deviez venir.

2. Vous avez convaincu l'un des autres PJ que votre intuition est inestimable.

3. Vous pensiez que quelque chose de terrible allait se produire si vous n'y alliez pas.

4. Vous êtes convaincu que la raison pour laquelle vous en êtes arrivé à ce point deviendra bientôt claire.


%--------------------------
\label{sec:descjovial}\section*{Jovial}

\textcolor{gray}{\emph{Jovial}}

Vous êtes joyeux, sympathique et extraverti. Vous mettez les autres à l'aise avec un grand sourire et une blague, éventuellement à vos frais, même si taquiner légèrement vos compagnons qui peuvent le supporter est aussi l'un de vos passe-temps favoris. Parfois, les gens disent qu'on ne prend jamais rien au sérieux. Ce n'est pas vrai, bien sûr, mais vous avez appris que s'attarder trop longtemps sur le mal prive rapidement le monde de sa joie. Vous avez toujours une nouvelle blague dans votre poche parce que vous les collectionnez comme certaines personnes collectionnent les bouteilles de vin.

\textbf{Spirituel:} +2 à votre Réserve d'Intellect.

\textbf{Compétence:} Vous êtes convivial et mettez la plupart des gens à l'aise avec votre attitude. Vous êtes entraîné à toutes les tâches liées à une interaction sociale agréable.

\textbf{Compétence:} Vous avez un avantage pour comprendre les punchlines des blagues que vous n'avez jamais entendues auparavant. Vous êtes entraîné à toutes les tâches liées à la résolution d'énigmes et d'énigmes.

\textbf{Lien initial à la Première Aventure:} Choisissez parmi la liste des options ci-dessous comment vous vous êtes retrouvé impliqué dans la première aventure:

1. Vous avez résolu une énigme avant de réaliser que répondre vous lancerait dans l'aventure.

2. Les autres PJ pensaient que vous apporteriez une légèreté bien nécessaire à l'équipe.

3. Vous avez décidé que se divertir sans travailler n'était pas le meilleur moyen de vivre la vie, alors vous avez rejoint les PJ.

4. Il s'agissait soit d'aller avec les PJ, soit d'affronter une situation qui n'était tout sauf joviale.


%--------------------------
\label{sec:desccraven}\section*{Lâche}

\textcolor{gray}{\emph{Craven}}

Le courage vous fait défaut à chaque instant. Vous manquez de volonté et de détermination pour tenir bon face au danger. La peur vous ronge le cœur, ronge votre esprit, vous conduit à la distraction jusqu'à ce que vous ne puissiez plus la supporter. La plupart du temps, vous reculez devant les confrontations. Vous fuyez les menaces et hésitez face à des décisions difficiles.Pourtant, malgré toute la peur qui vous hante et peut-être vous fait honte, votre nature lâche se révèle être un allié utile de temps en temps. Écouter vos peurs vous a aidé à échapper au danger et à éviter de prendre des risques inutiles. D'autres ont peut-être souffert à votre place, et vous êtes peut-être le premier à l'admettre, mais secrètement, vous ressentez un intense soulagement d'avoir évité un destin impensable et terrible.Des descripteurs comme Lâche, Cruel et Déshonorant pourraient ne pas convenir à tous les groupes. Ce sont des traits de caractère attribués plutôt pour les opposants des PJs et certaines personnes souhaitent que leurs PJ soient entièrement héroïques. Mais d'autres ne voient pas d'inconvénient à ce qu'un peu de grisaille morale vienne s'ajouter à ce mélange. D'autres encore voient des choses comme Lâche et Cruel comme des traits à surmonter à mesure que leurs personnages se développent (ce qui leur vaut probablement des descripteurs différents).

\textbf{Furtive:} +2 à votre Réserve de Célérité.

\textbf{Compétence:} Vous êtes entraîné dans les tâches de furtivité.

\textbf{Compétence:} Vous êtes entraîné dans les actions pour courir.

\textbf{Compétence:} Vous êtes entraîné pour toutes les actions prises pour échapper au danger, fuir devant une situation dangeureuse ou se sortir du pétrin.

\textbf{Inaptitude:} Vous n'entrez pas volontairement dans des situations dangereuses. Toute action d'initiative (pour déterminer qui commence le combat en premier) est désavantagée.

\textbf{Inaptitude:} Vous vous décomposez lorsque vous devez entreprendre seul une tâche potentiellement dangereuse. Toute tâche de ce type (comme attaquer une créature par vous-même) est désavantagée.

\textbf{Equipement Supplémentaire:} Vous disposez d'un porte-bonheur ou d'un dispositif de protection pour vous protéger du danger.

\textbf{Lien initial à la Première Aventure:} Choisissez parmi la liste des options ci-dessous comment vous vous êtes retrouvé impliqué dans la première aventure:

1. Vous pensez être pourchassé et vous avez engagé l'un des autres PJ comme votre protecteur.

2. Vous cherchez à échapper à votre honte et à vous rapprocher de personnes compétentes dans l'espoir de réparer votre réputation.

3. L'un des autres PJ vous a intimidé pour que vous veniez.

4. Le groupe a répondu à vos appels à l'aide lorsque vous étiez en difficulté.


%--------------------------
\label{sec:descclumsy}\section*{Maladroit}

\textcolor{gray}{\emph{Clumsy}}

Sans grâce et maladroit, on vous a dit que vous en sortiriez, mais vous ne l'avez jamais fait. Vous faites souvent tomber des objets, trébuchez sur vos propres pieds ou renversez des objets (ou des personnes). Certaines personnes sont frustrées par cette qualité, mais la plupart la trouvent drôle et même un peu charmante.Certains joueurs ne veulent peut-être pas être définis par une qualité « négative » comme Maladroit, mais en vérité, même ce type de descripteur présente suffisamment d'avantages pour en faire des personnages capables et talentueux. Ce que font réellement les descripteurs négatifs, c'est créer des personnages plus intéressants et complexes qui sont souvent très amusants à jouer.

\textbf{Empôté:} -2 à votre Réserve de Célérité.

\textbf{Musclé:} +2 à votre Réserve de Puissance.

\textbf{Inélegant:} Vous avez un certain charme adorable. Vous êtes entraîné à toutes les interactions sociales agréables lorsque vous vous exprimez une manière légère et avec autodérision.

\textbf{Chance Bête:} Le MJ peut introduire une intrusion du MJ sur vous, en fonction de votre maladresse, sans vous attribuer d'XP (comme si vous aviez obtenu un 1 sur un jet de d20). Cependant, si cela se produit, 50 % du temps, votre maladresse joue à votre avantage. Plutôt que de vous faire (beaucoup) du mal, cela vous aide, ou cela fait du mal à vos ennemis. Vous glissez, mais c'est juste à temps pour esquiver une attaque. Vous tombez, mais vous faites trébucher vos ennemis lorsque vous vous écrasez dans leurs jambes. Vous vous retournez trop rapidement, mais vous finissez par faire tomber l'arme des mains de votre ennemi. Vous et le MJ devez travailler ensemble pour déterminer les détails. Si le MJ le souhaite, il peut normalement utiliser les intrusions du MJ en fonction de votre maladresse (en attribuant de l'XP).

\textbf{Compétence:} Vous êtes un gars costaud. Vous êtes entraîné aux tâches impliquant de casser des objets.

\textbf{Inaptitude:} Toute tâche qui implique l'équilibre, la grâce ou la coordination main-oeil est désavantagée.

\textbf{Lien initial à la Première Aventure:} Choisissez parmi la liste des options ci-dessous comment vous vous êtes retrouvé impliqué dans la première aventure:

1. Vous étiez au bon endroit au bon moment.

2. Vous disposiez d'une information dont les autres PJ avaient besoin pour élaborer leurs plans.

3. Un frère ou une sœur vous a recommandé aux autres PC.

4. Vous êtes tombé sur les PJ alors qu'ils discutaient de leur mission, et ils se sont pris d'affection pour vous.


%--------------------------
\label{sec:descclever}\section*{Malin}

\textcolor{gray}{\emph{Clever}}

Vous avez l'esprit vif et vous réfléchissez bien. Vous comprenez les gens, vous pouvez donc les tromper, mais vous êtes rarement dupe. Parce que vous voyez facilement les choses telles qu'elles sont, vous obtenez rapidement un aperçu du terrain, évaluez les menaces et les alliés et évaluez les situations avec précision. Peut-être êtes-vous physiquement attirant ou utilisez-vous votre esprit pour surmonter vos imperfections physiques ou mentales.

\textbf{Malin:} +2 à votre Réserve d'Intellect.

\textbf{Compétence:} Vous êtes entraîné dans toutes les interactions impliquant des mensonges et de la tricherie.

\textbf{Compétence:} Vous êtes entraîné dans les jets de défense pour résister aux attaques mentales.

\textbf{Compétence:} Vous êtes entraîné dans toutes les tâches pour identifier ou évaluer un danger, des mensonges, ou la qualité, l'importance ou la fonction de quelque chose.

\textbf{Inaptitude:} YVous n'avez jamais été doué pour étudier ou retenir des connaissances triviales. Toute tâche impliquant des connaissances, des connaissances ou une compréhension est désavantagée.

\textbf{Equipement Supplémentaire:} Vous voyez à travers les plans des autres et les convainquez parfois de vous croire, même si, peut-être, ils ne devraient pas le faire. Grâce à votre comportement intelligent, vous disposez d'un article coûteux supplémentaire.

\textbf{Lien initial à la Première Aventure:} Choisissez parmi la liste des options ci-dessous comment vous vous êtes retrouvé impliqué dans la première aventure:

1. Vous avez convaincu l'un des autres PJ de vous dire ce qu'il faisait.

2. De loin, tu as observé qu'il se passait quelque chose d'intéressant.

3. Vous avez parlé de cette situation parce que vous pensiez que cela pourrait rapporter de l'argent.

4. Vous soupçonnez que les autres PJ ne réussiront pas sans vous.


%--------------------------
\label{sec:descdoomed}\section*{Maudit}

\textcolor{gray}{\emph{Doomed}}

Vous êtes bien certain que votre destin vous mène, inextricablement, vers une fin terrible. Ce destin pourrait être le vôtre seul, ou vous pourriez entraîner les personnes les plus proches de vous.

\textbf{Jumpy:} +2 à votre Réserve de Célérité.

\textbf{Compétence:} Toujours à l'affût du danger, vous êtes entraîné aux tâches liées à la perception.

\textbf{Compétence:} Vous avez l'esprit défensif, vous êtes donc entraîné aux tâches de défense de Célérité.

\textbf{Compétence:} Vous êtes cynique et vous attendez au pire. Ainsi, vous résistez aux chocs mentaux. Vous êtes entraîné à des tâches de défense d'Intellect liées à la perte de votre santé mentale ou de votre équanimité.

\textbf{Maudit:} Une fois sur deux, le MJ utilise l'intrusion du MJ sur votre personnage, vous ne pouvez pas le refuser et ne recevez pas d'XP pour cela (vous obtenez toujours un XP à attribuer à un autre joueur). C'est parce que vous êtes condamné. L'univers est un endroit froid et indifférent, et vos efforts sont pour le moins vains.

\textbf{Lien initial à la Première Aventure:} Choisissez parmi la liste des options ci-dessous comment vous vous êtes retrouvé impliqué dans la première aventure:

1. Vous avez tenté de l'éviter, mais les événements semblent conspirer pour vous attirer là où vous êtes.

2. Pourquoi pas ? Cela n'a pas d'importance. Vous êtes maudit, quoi que vous fassiez.

3. L'un des autres PJ vous a sauvé la vie, et maintenant vous remboursez cette obligation en l'aidant dans la tâche à accomplir.

4. Vous pensez que votre seul espoir d'éviter votre sort réside peut-être sur cette voie.


%--------------------------
\label{sec:desctonguetied}\section*{Mutique}

\textcolor{gray}{\emph{Tongue-Tied}}

Vous n'avez jamais été très bavard. Lorsque vous êtes obligé d'interagir avec les autres, vous ne pensez jamais à la bonne chose à dire : les mots vous manquent complètement, ou ils sont complètement faux. Vous finissez souvent par dire précisément la mauvaise chose et par insulter quelqu'un sans le vouloir. La plupart du temps, vous restez silencieux. Cela fait de vous un auditeur --- un observateur attentif. Cela signifie également que vous êtes plus doué pour faire les choses que pour en parler. Vous êtes prompt à agir.

\textbf{Des actions, pas des mots:} +2 à votre Réserve de Puissance, et +2 à votre Réserve de Célérité.

\textbf{Compétence:} Vous êtes entraîné à la compétence perception.

\textbf{Compétence:} Vous êtes entraîné à l'initiative( sauf s'il s'agit d'une situation sociale).

\textbf{Inaptitude:} Toutes les tâches liées aux interactions sociales sont désavantagées.

\textbf{Inaptitude:} Toutes les tâches impliquant une communication verbale ou un relais d'information sont désavantagées.

\textbf{Lien initial à la Première Aventure:} Choisissez parmi la liste des options ci-dessous comment vous vous êtes retrouvé impliqué dans la première aventure:

1. Vous venez de suivre et personne ne vous a dit de partir.

2. Vous avez vu quelque chose d'important que les autres PJ n'ont pas vu et (avec quelques efforts) avez réussi à le leur faire comprendre.

3. Vous êtes intervenu pour sauver l'un des autres PJ alors qu'il était en danger.

4. L'un des autres PJ vous a recruté pour vos talents.


%--------------------------
\label{sec:descmystical}\section*{Mystique}

\textcolor{gray}{\emph{Mystical}}

Vous vous considérez comme mystique, en harmonie avec le mystérieux et le paranormal. Vos vrais talents résident dans le surnaturel. Vous avez probablement de l'expérience avec les traditions anciennes et vous pouvez ressentir et manier le surnaturel --- même si cela signifie « magie », « phénomènes psychiques » ou autre chose, cela dépend de vous (et probablement aussi de ceux qui vous entourent). Les personnages mystiques portent souvent des bijoux, comme une bague ou une amulette, ou ont des tatouages ou d'autres marques qui montrent leurs intérêts.

\textbf{Intelligent:} +2 à votre Réserve d'Intellect.

\textbf{Compétence:} Vous êtes entraîné dans toutes les actions impliquant l'identification ou la compréhension du surnaturel.

\textbf{Sentir la Magie:} Vous pouvez sentir si le surnaturel est actif dans des situations où sa présence n'est pas évidente. Vous devez étudier attentivement un objet ou un lieu pendant une minute pour savoir si une touche mystique est à l'œuvre.

\textbf{Sort:} Vous pouvez exécuter la capacité Magie Prosaïque comme un sort lorsque vous avez une main libre et que vous pouvez payer le coût en points d'Intellect.

\textbf{Inaptitude:} Vous avez des manières ou une aura que les autres trouvent un peu déconcertantes. Toute tâche impliquant le charme, la persuasion ou la tromperie est désavantagée.

\textbf{Lien initial à la Première Aventure:} Choisissez parmi la liste des options ci-dessous comment vous vous êtes retrouvé impliqué dans la première aventure:

1. Un rêve vous a guidé jusqu'à ce point.

2. Vous avez besoin d'argent pour financer vos études.

3. Vous pensiez que la mission serait un excellent moyen d'en apprendre davantage sur le surnaturel.

4. Divers signes et présages vous ont conduit ici.


%--------------------------
\label{sec:descmysterious}\section*{Mystérieux}

\textcolor{gray}{\emph{Mysterious}}

La silhouette sombre qui se cache silencieusement dans un coin ? C'est vous. Personne ne sait vraiment d'où vous venez ni quelles sont vos Focus : vous cachez bien votre jeu. A plupart des gens sont perplexes, mais cela ne fait pas de vous un mauvais ami ou un mauvais allié. Vous êtes simplement doué pour garder les choses pour vous, vous déplacer sans être vu et dissimuler votre présence et votre identité.

\textbf{Compétence:} Vous êtes entraîné aux tâches de furtivité.

\textbf{Compétence:} Vous êtes entraîné à résister à un interrogatoire ou à des astuces pour vous faire parler.

\textbf{Touche-à-tout:} Vous tirez des talents et des capacités apparemment de nulle part. Vous pouvez tenter une tâche pour laquelle vous n'avez aucune formation comme si vous étiez entraîné, tenter une tâche pour laquelle vous êtes entraîné comme si vous étiez spécialisé ou gagner un niveau d'effort gratuit avec une tâche pour laquelle vous êtes spécialisé. Cette capacité est actualisée à chaque fois. vous faites un jet de récupération, mais les utilisations ne s'accumulent jamais.

\textbf{Inaptitude:} Les gens ne savent jamais où ils en sont avec vous. Toute tâche consistant à amener les gens à vous croire ou à vous faire confiance est désavantagée.

\textbf{Lien initial à la Première Aventure:} Choisissez parmi la liste des options ci-dessous comment vous vous êtes retrouvé impliqué dans la première aventure:

1. Vous vous présentez soudainement un jour.

2. Vous avez convaincu l'un des autres PJ que vous aviez des compétences inestimables.

3. Un personnage tout aussi mystérieux vous a indiqué où être et quand (mais pas pourquoi) rejoindre le groupe.

4. Quelque chose --- un sentiment, un rêve --- vous a dit où être et quand rejoindre le groupe.


%--------------------------
\label{sec:descnaive}\section*{Naif}

\textcolor{gray}{\emph{Naive}}

Vous avez vécu une vie protégée. Votre enfance a été sûre et sécurisée, vous n'avez donc pas eu l'occasion d'en apprendre beaucoup sur le monde --- et encore moins d'en faire l'expérience. Que vous vous entraîniez pour quelque chose, que vous ayez le nez plongé dans un livre ou que vous soyez simplement séquestré dans un endroit isolé, vous n'avez pas fait grand-chose, rencontré beaucoup de gens ou vu beaucoup de choses intéressantes jusqu'à présent. Cela va probablement changer bientôt, mais à mesure que vous avancez dans un monde plus vaste, vous le faites sans une partie de la compréhension que les autres possèdent sur la façon dont tout cela fonctionne.

\textbf{Frais et Dispo:} Vous ajoutez +1 à vos vos jets de Récupération.

\textbf{Incorruptible:} Vous êtes entraîné à aux tâches de défense d'Intellect et toutes tâches impliquant de résister à la tentation.

\textbf{Compétence:} You're wide-eyed. Vous êtes entraîné aux tâches de perception.

\textbf{Inaptitude:} Toute tâche qui implique de démystifier des tromperies ou de déterminer les Focus secrètes de quelqu'un est désavantagée.

\textbf{Lien initial à la Première Aventure:} Choisissez parmi la liste des options ci-dessous comment vous vous êtes retrouvé impliqué dans la première aventure:

1. Quelqu'un vous a dit que vous devriez vous impliquer.

2. Vous aviez besoin d'argent, et cela vous semblait être un bon moyen d'en gagner.

3. Vous pensiez que vous pourriez apprendre beaucoup en rejoignant les autres PJ.

4. Cela avait l'air amusant.


%--------------------------
\label{sec:descfoolish}\section*{Pas très brillant}

\textcolor{gray}{\emph{Foolish}}

Tout le monde ne peut pas être brillant. Oh, vous ne vous considérez pas comme stupide, et vous ne l'êtes pas. C'est juste que d'autres pourraient avoir un peu plus de...sagesse. Vous préférez avancer tête première dans la vie et laisser les autres s'inquiéter des choses. S'inquiéter ne vous a jamais aidé, alors pourquoi s'embêter ? Vous prenez les choses au pied de la lettre et ne vous inquiétez pas de ce que demain pourrait vous apporter.Les gens vous traitent d'« idiot » ou de « crétin », mais cela ne vous dérange pas beaucoup.Cela peut être libérateur et vraiment amusant de jouer un personnage un peu idiot. D'une certaine manière, la pression de toujours faire ce qui est juste et intelligent a disparu. D'un autre côté, si vous incarnez un personnage comme un crétin maladroit dans toutes les situations, cela peut devenir ennuyeux pour tout le monde autour de la table. Comme pour tout, la modération est la clé.

\textbf{Imprudent:} –4 à votre Réserve d'Intellect.

\textbf{Insouciant:} Vous réussissez plus par chance qu'autre chose. Chaque fois que vous lancez un dé pour une tâche, lancez deux fois et obtenez le résultat le plus élevé.

\textbf{Faiblesse intellectuelle :} Chaque fois que vous dépensez des points de votre réserve d'Intellect, cela vous coûte 1 point de plus que d'habitude.

\textbf{Inaptitude:} Toute tâche de défense intellectuelle est désavantagée.

\textbf{Inaptitude:} Toute tâche qui implique de voir à travers une tromperie, une illusion ou un piège est désavantagée.

\textbf{Lien initial à la Première Aventure:} Choisissez parmi la liste des options ci-dessous comment vous vous êtes retrouvé impliqué dans la première aventure:

1. Qui sait ? Sur le moment cela semblait être une bonne idée.

2. Quelqu'un vous a demandé de rejoindre les autres PJ. Ils vous ont dit de ne pas poser trop de questions, et cela vous a semblé bien.

3. Votre parent (ou une figure parentale/mentor) vous a impliqué pour vous donner quelque chose à faire et peut-être « vous apprendre un peu de bon sens ».

4. Les autres PJ avaient besoin de muscles pour ne pas trop réfléchir.


%--------------------------
\label{sec:descperceptive}\section*{Perspicace}

\textcolor{gray}{\emph{Perceptive}}

Peu vous échappe. Vous repérez les petits détails du monde qui vous entoure et êtes habile à faire des déductions à partir des informations que vous trouvez. Vos talents font de vous un détective exceptionnel, un scientifique redoutable ou un éclaireur talentueux.Même si vous êtes habile à trouver des indices, vous n'avez aucune compétence pour détecter les signaux sociaux. Vous négligez une infraction causée par vos déductions ou à quel point votre examen minutieux peut rendre les gens autour de vous inconfortables. Vous avez tendance à considérer les autres comme des nains intellectuels par rapport à vous, ce qui ne vous sert pas à grand-chose lorsque vous avez besoin d'une faveur.

\textbf{Malin:} +2 à votre Réserve d'Intellect.

\textbf{Compétence:} Vous avez le sens du détail. Vous êtes entraîné à toute tâche qui implique de trouver ou de remarquer de petits détails.

\textbf{Compétence:} Vous savez un peu tout. Vous êtes entraîné à toute tâche qui implique d'identifier des objets ou de rappeler un détail mineur ou une anecdote.

\textbf{Compétence:} Votre capacité à faire des déductions peut être imposante. Vous êtes entraîné à toute tâche qui implique d'intimider une autre créature.

\textbf{Inaptitude:} Votre confiance en vous apparaît comme de l'arrogance aux yeux des personnes qui ne vous connaissent pas. Toute tâche impliquant des interactions sociales positives est désavantagée.

\textbf{Equipement Supplémentaire:} Vous disposez d'un sac d'outils légers.

\textbf{Lien initial à la Première Aventure:} Choisissez parmi la liste des options ci-dessous comment vous vous êtes retrouvé impliqué dans la première aventure:

1. Vous avez entendu les autres PJ discuter de leur mission et avez proposé vos services.

2. L'un des PJ vous a demandé de venir, pensant que vos talents seraient inestimables pour la mission.

3. Vous pensez que la mission des PJ est liée d'une manière ou d'une autre à l'une de vos enquêtes.

4. Un tiers vous a recruté pour suivre les PJ et voir ce qu'ils faisaient.


%--------------------------
\label{sec:descfast}\section*{Prompt}

\textcolor{gray}{\emph{Fast}}

Vous vous déplacez rapidement, êtes capable de sprinter par courtes rafales et de travailler avec vos mains avec dextérité. Vous êtes doué pour franchir des distances rapidement, mais pas toujours en douceur. Vous êtes probablement mince et musclé.

\textbf{Rapide:} +4 à votre Réserve de Célérité.

\textbf{Compétence:} Vous êtes entraîné dans les actions d'initiative (pour déterminer qui commence le combat en premier).

\textbf{Compétence:} Vous êtes entraîné dans les actions en course.

\textbf{Inaptitude:} Vous êtes rapide mais pas nécessairement gracieux. Toute tâche impliquant l'équilibre est désavantagé.

\textbf{Lien initial à la Première Aventure:} Choisissez parmi la liste des options ci-dessous comment vous vous êtes retrouvé impliqué dans la première aventure:

1. Contre votre bon jugement, vous avez rejoint les autres PJ parce que vous avez vu qu'ils étaient en danger.

2. L'un des autres PJ vous a convaincu que rejoindre le groupe serait dans votre intérêt.

3. Vous avez peur de ce qui pourrait arriver si les autres PC tombaient en panne.

4. Il y a une récompense en jeu et vous avez besoin d'argent.


%--------------------------
\label{sec:descswift}\section*{Rapide}

\textcolor{gray}{\emph{Swift}}

Vous êtes vif. Parce que vous êtes rapide, vous pouvez accomplir des tâches plus rapidement que les autres. Cependant, vous n'êtes pas seulement rapide avec vos pieds : vous êtes rapide avec vos mains, et vous réfléchissez et réagissez rapidement. Vous parlez même vite.

\textbf{Energetique:} +2 à votre Réserve de Célérité.

\textbf{Compétence:} Vous êtes entraîné à courir.

\textbf{Rapide:} Vous pouvez vous déplacer sur une courte distance tout en effectuant une autre action au cours du même tour, ou vous pouvez vous déplacer sur une longue distance au cours de votre action sans avoir besoin d'effectuer un quelconque jet de dés.

\textbf{Inaptitude:} Vous êtes un sprinter, pas un coureur de fond. Vous n'avez pas beaucoup d'endurance. Les jets de défense pourraient être désavantagés.

\textbf{Lien initial à la Première Aventure:} Choisissez parmi la liste des options ci-dessous comment vous vous êtes retrouvé impliqué dans la première aventure:

1. Vous êtes intervenu pour sauver l'un des autres PJ qui en avait cruellement besoin.

2. L'un des autres PJ vous a recruté pour vos talents uniques.

3. Vous êtes impulsif et cela semblait être une bonne idée à l'époque.

4. Cette mission est liée à un objectif personnel qui vous est propre.


%--------------------------
\label{sec:descrugged}\section*{Rugueux}

\textcolor{gray}{\emph{Rugged}}

Vous êtes un amoureux de la nature, habitué à vivre dans la dure et à affronter les éléments. Très probablement, vous êtes un chasseur, un cueilleur ou un naturaliste expérimenté. Des années de vie dans la nature ont laissé des traces avec un visage usé, des cheveux sauvages ou des cicatrices. Vos vêtements sont probablement beaucoup moins raffinés que ceux portés par les citadins.

\textbf{Compétence:} Vous êtes entraîné dans toutes les tâches impliquant l'escalade, le saut, la course et la natation.

\textbf{Compétence:} Vous êtes entraîné dans toutes les tâches impliquant le dressage, l'équitation ou l'apaisement d'animaux naturels.

\textbf{Compétence:} Vous êtes entraîné dans toutes les tâches impliquant l'identification ou l'utilisation de plantes naturelles.

\textbf{Inaptitude:} Vous n'avez aucune grâce sociale et préférez les animaux aux humains. Toute tâche impliquant le charme, la persuasion, l'étiquette ou la tromperie est entravée.

\textbf{Equipement Supplémentaire:} Vous transportez un sac d'explorateur avec une corde, des rations pour deux jours, un sac de couchage et d'autres outils nécessaires à la survie en plein air.

\textbf{Lien initial à la Première Aventure:} Choisissez parmi la liste des options ci-dessous comment vous vous êtes retrouvé impliqué dans la première aventure:

1. Contre votre bon jugement, vous avez rejoint les autres PJ parce que vous avez vu qu'ils étaient en danger.

2. L'un des autres PJ vous a convaincu que rejoindre le groupe serait dans votre intérêt.

3. Vous avez peur de ce qui pourrait arriver si les autres PC tombaient en panne.

4. Il y a une récompense en jeu et vous avez besoin d'argent.


%--------------------------
\label{sec:descresilient}\section*{Résistant}

\textcolor{gray}{\emph{Resilient}}

Vous pouvez subir de nombreuses épreuves, tant physiques que mentales, tout en en redemandant. Il en faut beaucoup pour vous rabaisser. Ni les chocs ni les dommages physiques ou mentaux n'ont d'effet durable. Vous êtes difficile à dissuader. Imperturbable. Inarrêtable.

\textbf{Résistant:} +2 à votre Réserve de Puissance, et +2 à votre Réserve d'Intellect.

\textbf{Récupération:} Vous pouvez effectuer un jet de récupération supplémentaire chaque jour. Ce jet n'est qu'une action. Vous pouvez donc faire deux jets de récupération qui demandent chacun une action, un jet qui prend dix minutes, un quatrième lancer qui prend une heure et un cinquième lancer qui nécessite dix heures de repos.

\textbf{Compétence:} Vous êtes entraîné aux tâches de défense de Puissance.

\textbf{Compétence:} Vous êtes entraîné aux tâches de défense d'Intellect.

\textbf{Inaptitude:} Vous êtes robuste mais pas nécessairement fort. Toute tâche impliquant de déplacer, plier ou casser des objets est désavantagée.

\textbf{Inaptitude:} Vous avez beaucoup de volonté et de force mentale, mais vous n'êtes pas nécessairement intelligent. Toute tâche impliquant des connaissances ou la résolution de problèmes ou d'énigmes est désavantagée.

\textbf{Lien initial à la Première Aventure:} Choisissez parmi la liste des options ci-dessous comment vous vous êtes retrouvé impliqué dans la première aventure:

1. Vous avez vu que les PJ ont clairement besoin de quelqu'un comme vous pour les aider.

2. Quelqu'un vous a demandé de surveiller un des PJ en particulier, et vous avez accepté.

3. Vous vous ennuyez et avez désespérément besoin de relever un défi.

4. Vous avez perdu un pari – injustement, pensez-vous – et avez dû prendre la place de quelqu'un dans cette mission.


%--------------------------
\label{sec:descdishonorable}\section*{Sans Honneur}

\textcolor{gray}{\emph{Dishonorable}}

Il n'y a pas d'honneur parmi les voleurs, ni les traîtres, les traîtres, les menteurs ou les tricheurs. Vous êtes toutes ces choses, et soit vous n'en perdez pas le sommeil, soit vous niez la vérité aux autres ou à vous-même. Quoi qu'il en soit, vous êtes prêt à faire tout ce qu'il faut pour parvenir à vos fins. L'honneur, l'éthique et les principes ne sont que des mots. À votre avis, ils n'ont pas leur place dans le monde réel.

\textbf{Sneaky:} +4 à votre Réserve de Célérité.

\textbf{Bien mérité:} Lorsque le MJ donne à un autre joueur un point d'expérience à attribuer à quelqu'un pour une intrusion du MJ, ce joueur ne peut pas vous le donner.

\textbf{Compétence:} Vous êtes entraîné à la tromperie.

\textbf{Compétence:} Vous êtes entraîné à la furtivité.

\textbf{Compétence:} Vous êtes entraîné à l'intimidation.

\textbf{Inaptitude:} Les gens ne vous aiment pas ou ne vous font pas confiance. Les interactions sociales agréables sont désavantagées.

\textbf{Lien initial à la Première Aventure:} Choisissez parmi la liste des options ci-dessous comment vous vous êtes retrouvé impliqué dans la première aventure:

1. Vous êtes intéressé par ce que font les PJ, alors vous leur avez menti pour entrer dans leur groupe.

2. En vous rôdant, vous avez entendu les plans des PJ et réalisé que vous souhaitiez entrer.

3. L'un des autres PJ vous a invité, n'ayant aucune idée de ce que vous êtes vraiment.

4. Vous avez fait preuve d'intimidation et de fanfaronnades pour vous frayer un chemin.


%--------------------------
\label{sec:descskeptical}\section*{Sceptique}

\textcolor{gray}{\emph{Skeptical}}

Vous possédez une attitude interrogative face à des affirmations qui sont souvent tenues pour acquises par les autres. Vous n'êtes pas nécessairement un « Thomas qui doute » (un sceptique qui refuse de croire quoi que ce soit sans expérience personnelle directe), mais vous avez souvent tiré profit de la remise en question des déclarations, des opinions et des connaissances reçues qui vous ont été présentées par d'autres.

\textbf{Insightful:} +2 à votre Réserve d'Intellect.

\textbf{Compétence:} Vous êtes entraîné dans l'identification.

\textbf{Compétence:} Vous êtes entraîné dans toutes les actions qui consistent à décrypter une ruse, une illusion, une ruse rhétorique destinée à éluder le problème, ou un mensonge. Par exemple, vous êtes plus doué pour garder un oeil sur la tasse contenant la balle cachée, ressentir une illusion ou réaliser si quelqu'un vous ment (mais seulement si vous vous concentrez spécifiquement et utilisez cette compétence).

\textbf{Lien initial à la Première Aventure:} Choisissez parmi la liste des options ci-dessous comment vous vous êtes retrouvé impliqué dans la première aventure:

1. Vous avez entendu d'autres PJ s'exprimer sur un sujet avec une opinion sur laquelle vous étiez plutôt sceptique, vous avez donc décidé d'approcher le groupe et de demander des preuves.

2. Vous suiviez l'un des autres PJ parce que vous vous méfiiez de lui, ce qui vous a amené à l'action.

3. Votre théorie sur la non-existence du surnaturel ne peut être invalidée que par vos propres sens, alors vous êtes arrivés.

4. Vous avez besoin d'argent pour financer vos recherches.


%--------------------------
\label{sec:descguarded}\section*{Suspicieux}

\textcolor{gray}{\emph{Guarded}}

Vous cachez votre vraie nature derrière un masque et répugnez à laisser quiconque voir qui vous êtes vraiment. Vous protéger, physiquement et émotionnellement, est ce qui vous tient le plus à coeur et vous préférez garder tout le monde à une distance de sécurité. Vous vous méfiez peut-être de tous ceux que vous rencontrez, vous attendant au pire de la part des gens pour ne pas être surpris lorsqu'ils vous donnent raison. Ou vous pourriez simplement être un peu réservé, en faisant attention à ne pas laisser les gens montrer votre extérieur bourru à la personne que vous êtes vraiment.Personne ne peut être aussi réservé que vous et se faire de nombreux amis. Très probablement, vous avez une personnalité abrasive et avez tendance à être pessimiste dans vos perspectives. Vous soignez probablement une vieille blessure et découvrez que la seule façon de la gérer est de la garder, ainsi que votre personnalité, verrouillées.

\textbf{Suspicieux:} +2 à votre Réserve d'Intellect.

\textbf{Compétence:} Vous êtes entraîné aux tâches de défense d'Intellect.

\textbf{Compétence:} Vous êtes entraîné à toutes les tâches impliquant de discerner la vérité, de percer les déguisements et de reconnaître les mensonges et autres tromperies.

\textbf{Inaptitude:} Votre nature méfiante vous rend désagréable. Toute tâche impliquant la tromperie ou la persuasion est désavantagée.

\textbf{Lien initial à la Première Aventure:} Choisissez parmi la liste des options ci-dessous comment vous vous êtes retrouvé impliqué dans la première aventure:

1. L'un des PJ a réussi à surmonter vos défenses et à se lier d'amitié avec vous.

2. Vous voulez voir ce que font les PJ, alors vous les accompagnez pour les surprendre en train de commettre des actes répréhensibles.

3. Vous vous êtes fait quelques ennemis et vous vous associez aux PJ pour vous protéger.

4. Les PJ sont les seuls à vous supporter.


%--------------------------
\label{sec:descvirtuous}\section*{Vertueux}

\textcolor{gray}{\emph{Virtuous}}

Faire ce qu'il faut est un mode de vie. Vous vivez selon un code, et ce code est quelque chose auquel vous êtes attentif chaque jour. A chaque fois que vous glissez, vous vous reprochez votre faiblesse et vous vous remettez aussitôt sur les rails. Votre code inclut probablement la modération, le respect des autres, la propreté et d'autres caractéristiques que la plupart des gens considéreraient comme des vertus, tandis que vous évitez leurs contraires : la paresse, l'avidité, la gourmandise, etc.

\textbf{Intrépide:} +2 à votre Réserve de Puissance.

\textbf{Compétence:} Vous êtes entraîné à discerner les véritables Focus des gens ou voir à travers les mensonges.

\textbf{Compétence:} Votre adhésion à un code moral strict a endurci votre esprit contre la peur, le doute et les influences extérieures. Vous êtes entraîné à des tâches de défense intellectuelle.

\textbf{Lien initial à la Première Aventure:} Choisissez parmi la liste des options ci-dessous comment vous vous êtes retrouvé impliqué dans la première aventure:

1. Les PJ font quelque chose de vertueux, et vous êtes tout à fait dans ce sens.

2. Les PJ sont sur le chemin de la perdition, et vous considérez que votre tâche est de les mettre sur la bonne voie morale.

3. Un des autres PJ vous a invité, ayant entendu parler de vos voies vertueuses.

4. Vous faites passer la vertu avant le sens et défendez l'honneur de quelqu'un face à une organisation ou un pouvoir bien plus grand que vous. Vous avez rejoint les PJ parce qu'ils vous offraient aide et amitié alors que, par peur des représailles, personne d'autre ne le faisait.


%--------------------------
\label{sec:descvicious}\section*{Vicieux}

\textcolor{gray}{\emph{Vicious}}

Vous essayez de cacher ce qu'il y a à l'intérieur, de le replier sur vous-même quand tout en vous crie de lâcher prise, de les faire payer, de les faire souffrir et de les faire saigner. Parfois, vous réussissez pour vos amis : souriez comme ils sourient, riez quand ils rient et parfois même ressentez vos propres émotions. Mais il est toujours là, ce sentiment de joie frénétique mêlé de haine qui jaillit parfois de vous lorsque vous affrontez un ennemi. La violence que vos amis peuvent tolérer, mais vous craignez parfois qu'ils apprennent également que vous êtes cruel.

\textbf{Compétence:} Vous êtes entraîné à traquer les créatures. Si une créature vous a fait du tort, la tâche de suivi est facilitée.

\textbf{Sanguinaire:} Une fois que vous commencez à vous battre, vous ne voyez que du rouge. Vous infligez 2 points de dégâts supplémentaires à chaque attaque.

\textbf{Berserk:} Une fois que vous commencez à vous battre, il est difficile pour vous de vous arrêter. En fait, c'est une tâche de difficulté 2 en Intellect de le faire, même si votre ennemi se rend ou si vous n'avez plus d'ennemis. Si ce dernier cas se produit et que vous ne parvenez pas à vous arrêter, vous attaquez l'allié le plus proche à courte portée.

\textbf{Equipement Supplémentaire:} Vous disposez d'un carnet que vous utilisez pour répertorier ceux qui vous ont fait du tort.

\textbf{Lien initial à la Première Aventure:} Choisissez parmi la liste des options ci-dessous comment vous vous êtes retrouvé impliqué dans la première aventure:

1. Un autre PJ vous a vu abattre un méchant ivrogne dans une taverne, sans se rendre compte que c'était vous qui aviez commencé le combat.

2. Vous vouliez vous éloigner d'une mauvaise situation, alors vous êtes allé avec les autres PJ.

3. Vous voulez changer et vous espérez qu'être avec les autres PJ vous aidera à vous calmer.

4. L'un des autres PJ vous a demandé de venir, pensant que votre méchanceté pourrait être exploitée au profit de la mission.


