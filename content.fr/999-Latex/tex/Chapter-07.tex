%#######################################################################
%            CHAPTER 7
%#######################################################################
\chapter{Descripteur}
\label{ch:chapter7}
\fancypagestyle{plain}{ %
\fancyhf{} % remove everything
\renewcommand{\headrulewidth}{1pt} % remove lines as well
\renewcommand{\footrulewidth}{0pt}
\xpretocmd\headrule{\color{BlueViolet}}{}{\PatchFailed}
\fancyhead[RO]{\textcolor{BlueViolet}{Descripteur}}
\fancyhead[LE]{\textcolor{BlueViolet}{CYPHER SYSTEM}}
}
Votre descripteur définit votre personnage, il lui donne son caractère. Les différences entre un Explorateur Charmeur et un Explorateur Vicieux sont considérables. Le descripteur change la façon de chaque pesonnage d'accomplir chaque action. Votre descripteur place votre personnage dans la situation (la première aventure, qui démarre la campagne) et contribue à le motiver. C'est l'adjectif de la phrase « Je suis un *adjectif nom* qui *verbes* ».

Les descripteurs offrent un ensemble unique de capacités, de compétences ou de modifications supplémentaires à vos Réserves de statistiques. Toutes les propositions d'un descripteur ne sont pas des modifications positives du personnage. Par exemple, certains descripteurs ont des incapacités --- des tâches pour lesquelles un personnage n'est pas doué. Vous pouvez considérer les incapacités comme des compétences négatives : au lieu d'être un peu meilleur dans ce genre de tâche, vous êtes un peu pire. Si vous devenez compétent dans une tâche pour laquelle vous êtes incapable, elles s'annulent. N'oubliez pas que les personnages sont définis autant par ce pour quoi ils ne sont *pas* bons que par ce pour quoi ils *sont* bons.

Les descripteurs proposent également quelques brèves suggestions sur la manière dont votre personnage s'est impliqué avec le reste du groupe lors de sa première aventure. Vous pouvez les utiliser ou non, comme vous le souhaitez.

Cette section détaille cinquante descripteurs. Choisissez-en un pour votre personnage. Vous pouvez choisir le descripteur de votre choix, quel que soit votre type. À la fin de ce chapitre, quelques options sont proposées pour personnaliser les descripteurs, notamment faire de l'espèce d'un personnage leur descripteur.


Votre descripteur compte le plus lorsque vous êtes un personnage débutant. Les avantages (et peut-être les inconvénients) découlant de votre descripteur finiront par être éclipsés par l'importance croissante de votre type et de votre orientation. Cependant, l'influence de votre descripteur restera au moins quelque peu importante tout au long de la vie de votre personnage.
