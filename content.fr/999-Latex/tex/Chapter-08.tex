%#######################################################################
%            CHAPTER 8
%#######################################################################
\startchapter{Focus}{ch:chapter8}{CSCOLORPARTONE}
\raggedright
Le Focus est ce qui rend un personnage unique. Il n'y a pas deux PJ dans un groupe qui devraient avoir le même focus. Un Focus donne des avantages à un personnage lorsqu'il crée son personnage et à chaque fois qu'il passe au rang suivant. C'est le groupe verbal de la phrase "Je suis un *adjectif nom* qui *groupe verbal*".

Ce chapitre contient près d'une centaine d'exemples de Focus, tels que Se Revêt d'un Halo de Feu, Serait plutôt en train de lire et Pilote un Vaisseau Spatial. Ces Focus peuvent être choisies et utilisées telles que présentées par un joueur, ou par le MJ qui les ajoute à une liste de Focus disponibles pour ses joueurs lors de leur prochaine campagne.

De plus, la seconde moitié de ce chapitre fournit des outils permettant au MJ ou à un joueur entreprenant de créer ses propres Focus personnalisés qui correspondent parfaitement aux besoins d'un jeu ou d'une campagne donnée, comme présenté dans Création de nouveaux Focus.

\section*{Choisir un Focus}
Tous les Foci ne conviennent pas à tous les genres. Le chapitre Genre fournit des conseils, mais cette section propose quelques grandes généralisations. Évidemment, le MJ peut inclure toutes les Focus disponibles dans son environnement. Les Focus finissent par être une distinction importante dans ce cas, car Dispose de Pouvoirs Mentaux, par exemple, indique clairement que des capacités psychiques existent dans la campagne, tout comme Hurle à la Lune implique l'existence de lycanthropes comme les loups-garous, et Pilote des Vaisseaux Spaciaux, nécessite bien sûr des vaisseaux spatiaux disponibles pour piloter.

Lorsqu'un Focuss est choisi pour un personnage, celui-ci obtient une connexion spéciale avec un ou plusieurs de ses camarades PJ, une capacité de premier rang et peut-être un équipement de départ supplémentaire : une ou deux pièces d'équipement qui pourraient être nécessaires pour que le personnage puisse utiliser leur capacité, ou cela pourrait bien se marier avec le Focus. Par exemple, un personnage capable de construire des choses a besoin d'un ensemble d'outils. Un personnage constamment en feu a besoin d'un ensemble de vêtements insensibles aux flammes. Un personnage qui dessine des runes pour lancer des sorts a besoin d'outils d'écriture. Un personnage qui tue des monstres avec une épée a besoin d'une épée. Et ainsi de suite. Cela dit, de nombreuses Focus ne nécessitent pas d'équipement supplémentaire.

Chaque focus propose également une ou plusieurs suggestions --- intrusions GM --- pour les effets ou conséquences possibles de très bons ou de très mauvais lancers de dés.

Quelques Focus présentés dans ce chapitre fournissent une « Option d'échange de type » qui permet à un joueur d'échanger une capacité qui serait autrement acquise grâce à son type contre la capacité indiquée. Un joueur n'est pas obligé de procéder à l'échange; ils ont simplement le choix. Par exemple, le Focus Aime le Aime le Vide offre la possibilité d'acquérir la capacité Ayez une Combinaison Spatiale, Vous Voyagerez au lieu d'une capacité de type.

Au fur et à mesure qu'un personnage progresse vers un nouveau rang, un Focus confère plus de capacités. L'avantage de chaque rang est généralement appelé Action ou Enabler. Si une capacité est étiquetée Action, un personnage doit effectuer une action pour l'utiliser. Si une capacité est étiquetée Facilitateur, elle améliore d'autres actions ou donne un autre avantage, mais ce n'est pas une action. Une capacité qui permet à un personnage de faire exploser ses ennemis avec des lasers est une action. Une capacité qui accorde des dégâts supplémentaires lorsqu'une attaque est effectuée est un Facilitateur. Un Facilitateur est utilisé au même tour qu'une autre action, et souvent dans le cadre d'une autre action.

Les avantages de chaque rang sont indépendants et cumulatifs avec les avantages des autres rangs (sauf indication contraire). Ainsi, si une capacité de premier rang confère +1 à l'armure et qu'une capacité de quatrième rang confère également +1 à l'armure, lorsque le personnage atteint le quatrième rang, un total de +2 à l'armure est accordé.

Aux rangs 3 et 6, le personnage est invité à choisir une capacité parmi les deux options proposées.

Enfin, vous pouvez choisir si vous souhaitez développer l'histoire derrière le Focus (bien que ce ne soit pas obligatoire).


\section*{Connexions de Focus}
Choisissez une connexion qui va bien avec le Focus. Si vous êtes un MJ en train de choisir (ou de créer) un ou plusieurs Focus pour vos joueurs, choisissez jusqu'à quatre des connexions suivantes.
\begin{enumerate}
\item Choisissez un autre PJ. Pour des raisons que vous ne connaissez pas, ce personnage est totalement immunisé contre vos capacités de Focus, que vous les utilisiez pour vous aider ou pour vous nuire.
\item Choisissez un autre PJ. Vous connaissiez ce personnage il y a des années, mais vous ne pensez pas qu'il vous connaissait.
\item Choisissez un autre PJ. Vous essayez toujours de l'impressionner, mais vous ne savez pas pourquoi.
\item Choisissez un autre PJ. Ce personnage a une habitude qui vous ennuie, mais vous êtes par ailleurs assez impressionné par ses capacités.
\item Choisissez un autre PJ. Ce personnage montre du potentiel pour apprécier votre paradigme particulier, votre style de combat ou tout autre attribut fourni par le Focus. Vous aimeriez le former, mais vous n'êtes pas forcément qualifié pour enseigner (c'est à vous de décider), et il pourrait ne pas être intéressé (c'est à lui de décider).
\item Choisissez un autre PJ. S'il est à portée immédiate lorsque vous êtes en combat, il constitue parfois un atout, et parfois il gêne accidentellement vos jets d'attaque (50\% de chances dans tous les cas, déterminés par combat).
\item Choisissez un autre PJ. Vous lui avez déjà sauvé la vie et il se sente clairement redevable envers vous. Vous souhaiteriez qu'il ne le fasse pas; cela fait juste partie du travail.
\item Choisissez un autre PJ. Ce personnage s'est récemment moqué de vous d'une manière qui vous a vraiment blessé. La façon dont vous gérez cela (le cas échéant) dépend de vous.
\item Choisissez un autre PJ. Ce personnage sait que vous avez souffert aux mains d'entités robotiques dans le passé. C'est à vous de décider si vous détestez les robots maintenant, ce qui peut affecter votre relation avec le personnage s'il est amical avec les robots ou s'il possède des pièces robotiques.
\item Choisissez un autre PJ. Ce personnage vient du même endroit que vous et vous vous connaissez étant enfants.
\item Choisissez un autre PJ. Dans le passé, il vous enseignait quelques astuces à utiliser lors d'un combat.
\item Choisissez un autre PJ. Ce personnage ne semble pas approuver vos méthodes.
\item Choisissez un autre PJ. Il y a bien longtemps, vous étiez tous les deux dans des camps opposés dans un combat. Vous avez gagné, même si vous avez « triché » à ses yeux (mais de votre point de vue, tout est juste dans un combat). Il est peut-être prêt pour une revanche, mais cela dépend de lui.
\item Choisissez un autre PJ. Vous essayez toujours d'impressionner ce personnage avec vos compétences, votre esprit, votre apparence ou votre bravade. Peut-être qu'il est un rival, peut-être que vous avez besoin de son respect, ou peut-être que vous êtes intéressé de manière romantique par lui.
\item Choisissez un autre PJ. Vous craignez que ce personnage soit jaloux de vos capacités et craignez que cela puisse entraîner des problèmes.
\item Choisissez un autre PJ. Vous l'avez accidentellement attrapé dans un piège que vous aviez tendu et il a dû se libérer tout seul.
\item Choisissez un autre PJ. Vous avez déjà été embauché pour retrouver quelqu'un qui était proche de ce personnage.
\item Choisissez deux PJ (de préférence ceux qui sont susceptibles de gêner vos attaques). Lorsque vous ratez une attaque et que les règles de la MJ vous imposent de frapper quelqu'un d'autre que votre cible, vous touchez l'un de ces deux personnages.
\item Choisissez un autre PJ. Vous ne savez pas comment ni d'où, mais ce personnage a un plan pour les bouteilles d'alcool rare et peut vous les procurer à moitié prix.
\item Choisissez un autre PJ. Vous avez récemment perdu un bien et vous êtes convaincu qu'il l'a pris. Qu'il l'ait fait ou non, cela dépend de lui.
\item Choisissez un autre PJ. Il semble toujours savoir où vous êtes, ou du moins dans quelle direction vous vous situez par rapport à lui.
\item Choisissez un autre PJ. Vous voir utiliser vos capacités de concentration semble déclencher un souvenir désagréable chez ce personnage. Cette mémoire appartient à l'autre PJ, même s'il ne peut pas être en mesure de s'en souvenir consciemment.
\item Choisissez un autre PJ. Quelque chose chez lui interfère avec vos capacités. Lorsqu'il se trouve à côté de vous, vos capacités de concentration coûtent 1 point supplémentaire.
\item Choisissez un autre PJ. Quelque chose en eux complète vos capacités. Lorsqu'il se tient à côté de vous, la première capacité de concentration que vous utilisez sur une période de 24 heures coûte 2 points de moins.
\item Choisissez un autre PJ. Vous connaissez ce personnage depuis un certain temps et il vous a aidé à prendre le contrôle de vos capacités de concentration.
\item Choisissez un autre PJ. Dans le passé de ce personnage, il a vécu une expérience dévastatrice en tentant quelque chose que vous faites naturellement grâce à votre concentration. C'est à lui de décider s'il choisit de vous en parler.
\item Choisissez un autre PJ. Sa maladresse occasionnelle et son comportement bruyant vous irritent.
\item Choisissez un autre PJ. Dans un passé récent, alors que vous vous entraîniez, vous l'avez accidentellement frappé lors d'une attaque, le blessant grièvement. C'est à lui de décider s'il vous en veu ou s'il vous pardonne.
\item Choisissez un autre PJ. Il vous doit une somme d'argent importante.
\item Choisissez un autre PJ. Dans un passé récent, alors que vous échappiez à une menace, vous avez accidentellement laissé ce personnage se débrouiller tout seul. Il  a survécu, mais de justesse. C'est au joueur de ce personnage de décider s'il vous en veut ou s'il a décidé de vous pardonner.
\item Choisissez un autre PJ. Récemment, il vous a mis accidentellement (ou peut-être intentionnellement) dans une position de danger. Vous allez bien maintenant, mais vous vous méfiez de lui.
\item Choisissez un autre PJ. De votre point de vue, il semble nerveux face à une idée, une personne ou une situation spécifique. Vous aimeriez lui apprendre à être plus à l'aise avec ses peurs (s'il vous le permette).
\item Choisissez un autre PJ. Il vous a traité de lâche une fois.
\item Choisissez un autre PJ. Ce personnage vous reconnaît toujours, vous ou ce que vous produisez, que vous soyez déguisé ou disparu depuis longtemps lorsqu'il arrive sur les lieux.
\item Choisissez un autre PJ. Vous avez provoqué par inadvertance un accident qui l'a plongés dans un sommeil si profond qu'il ne se sont pas réveillé pendant trois jours. Qu'il vous pardonne ou non, c'est à lui de décider.
\item Choisissez un autre PJ. Vous êtes presque sûr d'avoir un lien de parenté d'une manière ou d'une autre.
\item Choisissez un autre PJ. Vous avez accidentellement appris quelque chose qu'il essayait de garder secret.
\item Choisissez un autre PJ. Il est particulièrement sensible à l'utilisation de vos capacités de Focus plus flashy et deviennent parfois éblouis pendant quelques rounds, ce qui gêne leurs actions.
\item Choisissez un autre PJ. Il semble posséder un objet précieux qui vous appartenait autrefois, mais que vous avez perdu à un jeu de hasard il y a des années.
\item Choisissez un autre PJ. Sans vous, ce personnage aurait échoué à un test de réussite mentale.
\item Choisissez un autre PJ. Sur la base de quelques commentaires que vous avez entendus, vous soupçonnez qu'il n'accorde pas la plus haute estime à votre domaine de formation ou à votre passe-temps favori.
\item Choisissez un autre PJ dont l'intérêt se confond avec le vôtre. Cette étrange connexion l'affecte d'une manière ou d'une autre. Par exemple, si le personnage utilise une arme, votre capacité de concentration améliore parfois son attaque d'une manière ou d'une autre.
\item Choisissez un autre PJ. Il souffre d'un vertige terrible. Vous aimeriez lui apprendre à être plus à l'aise en hauteur. Il doit décider d'accepter ou non votre offre.
\item Choisissez un autre PJ. Il est sceptique quant à vos affirmations sur quelque chose d'important qui s'est produit dans votre passé. Il pourrait même tenter de vous discréditer ou de découvrir le « secret » de votre histoire, mais cela dépend de lui.
\item Choisissez un autre PJ. Il a le don de reconnaître les points faibles de vos plans ou de vos projets.
\item Choisissez un autre PJ. Le visage de ce personnage vous intrigue tellement d'une manière que vous ne comprenez pas que vous vous retrouvez parfois à dessiner son image dans la terre ou sur un autre support auquel vous avez accès.
\item Choisissez un autre PJ. Ce personnage possède un élément supplémentaire de l'équipement régulier que vous lui avez donné, soit quelque chose que vous avez fabriqué, soit un objet que vous vouliez simplement lui donner. (Il choisit l'article.)
\item Choisissez un autre PJ. Il vous a chargé de faire un travail pour lui. Vous avez déjà été payé mais n'avez pas encore terminé le travail.
\item Choisissez un autre PJ. Vous avez travaillé ensemble dans le passé et le travail s'est mal terminé.
\item Choisissez un autre PJ. Pendant qu'il se tient à côté de vous et utilise leur action pour se concentrer sur votre aide, la portée de l'une de vos capacités de concentration est doublée.
\end{enumerate}

\section*{L'Histoire derrière les Focus}
Les Focus de ce livre ont été volontairement réduites à l'essentiel afin d'avoir l'application la plus large possible dans plusieurs genres. Une ou deux phrases descriptives résument chacune d'entre elles. Après avoir choisi un Focus, vous avez la possibilité d'élargir sa présentation en ajoutant plus d'histoire et de description pertinentes pour le monde ou pour le personnage.

Par exemple, si vous choisissez Opère sous Couverture, la description récapitulative est "Sous l'apparence de quelqu'un d'autre, vous cherchez à trouver des réponses que les puissants ne veulent pas divulguer". Si vous choisissez Poursuit des Sciences Etranges, le résumé est le suivant: "Votre perspicacité et vos capacités surnaturelles font de vous un scientifique capable de prouesses incroyables." Ces descriptions fournissent ce que vous devez savoir pour utiliser le focus.

Cependant, si vous le souhaitez (et *uniquement* si vous le souhaitez ; rien n'est obligatoire), vous pouvez ajouter davantage à ces descriptions d'une manière pertinente pour votre jeu. Par exemple, si vous choisissez Opère sous Couverture et Poursuit des Sciences Etranges pour une utilisation dans un genre moderne tel que l'horreur, la fantasy urbaine, l'espionnage ou quelque chose de similaire, vous pouvez développer les descriptions comme indiqué dans les exemples suivants.

**Opère sous Couverture:** L'espionnage n'est pas quelque chose dont vous savez rien. Du moins, c'est ce que vous voulez faire croire à tout le monde, car en vérité, vous avez été formé comme espion ou agent secret. Vous pourriez travailler pour un gouvernement ou pour vous-même. Vous pourriez être un détective de police ou un criminel. Vous pourriez même devenir journaliste d'investigation.

Quoi qu'il en soit, vous apprenez des informations que d'autres tentent de garder secrètes. Vous collectez des rumeurs et des chuchotements, des histoires et des preuves durement acquises, et vous utilisez ces connaissances pour vous aider dans vos propres efforts et, le cas échéant, pour fournir à vos employeurs les informations qu'ils souhaitent. Alternativement, vous pouvez vendre ce que vous avez appris à ceux qui sont prêts à payer plus cher.

Vous portez probablement des couleurs sombres (noir, gris anthracite ou bleu nuit) pour vous fondre dans l'ombre, à moins que la couverture que vous avez choisie ne vous oblige à ressembler à quelqu'un d'autre.

**Poursuit des Sciences Etranges:** Vous pourriez être un scientifique respecté, ayant publié dans plusieurs revues à comité de lecture. Ou bien vous pourriez être considéré comme un excentrique par vos contemporains, poursuivant des théories marginales sur ce que d'autres considèrent comme peu de preuves. La vérité est que vous avez un don particulier pour passer au crible ce qui est possible. Vous pouvez trouver de nouvelles informations et débloquer des phénomènes étranges grâce à vos expériences. Là où d'autres voient une corne d'abondance farfelue, vous passez au crible les théories du complot à la recherche de révélations. Que vous meniez vos enquêtes en tant qu'entrepreneur gouvernemental, chercheur universitaire, scientifique d'entreprise ou curieux dans votre propre laboratoire en suivant votre muse, vous repoussez les limites de ce qui est possible.

Vous vous souciez probablement plus de votre travail que de trivialités telles que votre apparence, votre comportement poli ou approprié, ou les normes sociales, mais là encore, un excentrique comme vous pourrait également renverser la situation sur ce stéréotype.

Si vous souhaitez aller encore plus loin, vous pouvez déterminer d'où viennent les capacités de concentration d'un personnage. Selon le genre, ils pourraient tirer ces capacités d'un entraînement avancé et persistant, via des runes magiques, des éléments cybernétiques, de leur héritage génétique ou de leur accès à une technologie avancée. Par exemple, un personnage pourrait être capable de faire exploser des cibles avec des éclairs parce qu'elles ont été zappées par un rayonnement étrange ou parce qu'elles ont ramassé un pistolet éclair. D'un autre côté, c'est peut-être parce que leur entraînement intense leur a permis d'apprendre la magie de la foudre. Les possibilités sont presque infinies et c'est à vous de les inclure ou de les renoncer. Parce que quelle que soit la manière dont les capacités d'un focus ont été acquises, il suffit également qu'elles fonctionnent.

\section*{Focus}
La description complète de chaque capacité de concentration répertoriée dans cette section se trouve dans le chapitre Capacités, qui contient des descriptions du type, de la saveur et des capacités de concentration dans un seul et vaste catalogue.
%____________________________________________________________________---------------
%%%%%%%%%%%%%%%%%%%%%%%%%%%%%%%%%%%%%%%%%%%%%%%%%%%%%%%%%%%%%%%%%%%%%%%
%_____________________________________________________%
\phantomsection\label{sec:focusplaystoomanygames}\section*{ A Joué à Trop de Jeux}

\textcolor{gray}{\emph{ Plays Too Many Games}}

Les leçons, les réflexes et les stratégies que vous avez appris en jouant à trop de jeux ont des applications dans le monde réel, où les gens qui ne jouent pas assez travaillent dur et vivent leur vie morne.

\begin{abnamelist}
\item Rang 1: Joueur (\pageref{subsec:ab_gamer}) Leçons de jeu (\pageref{subsec:ab_game_lessons})
\item Rang 2:
Résoudre des Enigmes (\pageref{subsec:ab_resist_tricks}) Voir dans le Noir (\pageref{subsec:ab_zero_dark_eyes})
\item Rang 3:
Choisissez en une: Avantage de Célérité Amélioré (\pageref{subsec:ab_enhanced_speed_edge}) ou Objectif du tireur d'élite (\pageref{subsec:ab_snipers_aim})
\item Rang 4:
Intellect amélioré (\pageref{subsec:ab_enhanced_intellect}) Jeux d'esprit (\pageref{subsec:ab_mind_games})
\item Rang 5:
Endurance du joueur (\pageref{subsec:ab_gamers_fortitude})
\item Rang 6:
Choisissez en une: Dieu du jeu (\pageref{subsec:ab_gaming_god}) ou Sursaut mental (\pageref{subsec:ab_mind_surge})
\end{abnamelist}
\textbf{Intrusion de la Meneuse:}
%_____________________________________________________%
\phantomsection\label{sec:focussailedbeneaththejollyroger}\section*{ A Navigué sous Pavillon Pirate}

\textcolor{gray}{\emph{ Sailed Beneath the Jolly Roger}}

Vous avez navigué avec un équipage de redoutables pirates, mais vous avez décidé de mettre fin à vos jours de pirate et de rejoindre une autre cause. La question est : votre passé vous laissera-t-il partir si facilement ?

\begin{abnamelist}
\item Rang 1: Ignorez la Douleur (\pageref{subsec:ab_ignore_the_pain}) Marin (\pageref{subsec:ab_sailor})
\item Rang 2:
Prendre l'avantage (\pageref{subsec:ab_taking_advantage}) Réputation redoutable (\pageref{subsec:ab_fearsome_reputation})
\item Rang 3:
Choisissez en une: Compétence avec les attaques (\pageref{subsec:ab_skill_with_attacks}) ou Compétence en Défense Supérieure (\pageref{subsec:ab_greater_skill_with_defense})
\item Rang 4:
Habiletés motrices (\pageref{subsec:ab_movement_skills}) Le pied marin (\pageref{subsec:ab_sea_legs})
\item Rang 5:
Perdu dans le chaos (\pageref{subsec:ab_lost_in_the_chaos})
\item Rang 6:
Choisissez en une: Attaque successive (\pageref{subsec:ab_successive_attack}) ou Duel à mort (\pageref{subsec:ab_duel_to_the_death})
\end{abnamelist}
\textbf{Intrusion de la Meneuse:}
%_____________________________________________________%
\phantomsection\label{sec:focusdescendsfromnobility}\section*{ A des Ascendants Nobles}

\textcolor{gray}{\emph{ Descends From Nobility}}

Descendant de la richesse et du pouvoir, vous portez un titre noble et les capacités conférées par une éducation privilégiée. \newline\textbf{Option d'échange de type:} Suite de Servants

\begin{abnamelist}
\item Rang 1: Noblesse privilégiée (\pageref{subsec:ab_privileged_nobility})
\item Rang 2:
Interlocuteur qualifié (\pageref{subsec:ab_trained_interlocutor})
\item Rang 3:
Choisissez en une: Commandement avancé (\pageref{subsec:ab_advanced_command}) ou Courage du noble (\pageref{subsec:ab_nobles_courage})
\item Rang 4:
Disciple expert (\pageref{subsec:ab_expert_follower})
\item Rang 5:
Affirmer votre privilège (\pageref{subsec:ab_asserting_your_privilege})
\item Rang 6:
Choisissez en une: Assistance compétente (\pageref{subsec:ab_able_assistance}) ou Esprit de leader (\pageref{subsec:ab_mind_of_a_leader})
\end{abnamelist}
\textbf{Intrusion de la Meneuse:}
%_____________________________________________________%
\phantomsection\label{sec:focusislicensedtocarry}\section*{ A le Droit de Porter une Arme à Feu}

\textcolor{gray}{\emph{ Is Licensed to Carry}}

Vous portez une arme à feu et vous savez comment l'utiliser lors d'un combat.  Bien que Is Licensed to Carry soit conçu pour les armes à feu modernes, il pourrait s'appliquer aux armes à silex, aux blasters laser futuristes ou à d'autres armes à distance. 

\begin{abnamelist}
\item Rang 1: Pratique des armes à feu (\pageref{subsec:ab_practiced_with_guns}) Tireur (\pageref{subsec:ab_gunner})
\item Rang 2:
Tir prudent (\pageref{subsec:ab_careful_shot})
\item Rang 3:
Choisissez en une: Augmente les dommages (\pageref{subsec:ab_damage_dealer}) ou Tireur entraîné (\pageref{subsec:ab_trained_gunner})
\item Rang 4:
Tir Rapide (\pageref{subsec:ab_snap_shot})
\item Rang 5:
Tirs en éventail (\pageref{subsec:ab_arc_spray})
\item Rang 6:
Choisissez en une: Dégâts mortels (\pageref{subsec:ab_lethal_damage}) ou Tir Spécial (\pageref{subsec:ab_special_shot})
\end{abnamelist}
\textbf{Intrusion de la Meneuse:}
%_____________________________________________________%
\phantomsection\label{sec:focushasathousandfaces}\section*{ A un Millier de Visages}

\textcolor{gray}{\emph{ Has A Thousand Faces}}

Vous pouvez changer votre apparence pour ressembler à n’importe qui d’autre.

\begin{abnamelist}
\item Rang 1: Changement de Visage (\pageref{subsec:ab_face_morph}) Compétences d'interaction (\pageref{subsec:ab_interaction_skills})
\item Rang 2:
Altération corporelle (\pageref{subsec:ab_body_morph}) Chair de guerre (\pageref{subsec:ab_war_flesh})
\item Rang 3:
Choisissez en une: Déguiser un autre (\pageref{subsec:ab_disguise_other}) ou Résilience (\pageref{subsec:ab_resilience})
\item Rang 4:
Pensez à votre sortie (\pageref{subsec:ab_think_your_way_out}) Sans âge (\pageref{subsec:ab_ageless})
\item Rang 5:
La mémoire devient une action (\pageref{subsec:ab_memory_becomes_action})
\item Rang 6:
Choisissez en une: Divisez votre esprit (\pageref{subsec:ab_divide_your_mind}) ou Déduire des pensées (\pageref{subsec:ab_infer_thoughts})
\end{abnamelist}
\textbf{Intrusion de la Meneuse:}
%_____________________________________________________%
\phantomsection\label{sec:focuswasforetold}\section*{ A été Choisi(e)}

\textcolor{gray}{\emph{ Was Foretold}}

Vous êtes « l’élu » et la prophétie, la prédiction, le pronostic ou toute autre méthode de détermination attend de vous de grandes choses un jour.

\begin{abnamelist}
\item Rang 1: Compétences d'interaction (\pageref{subsec:ab_interaction_skills}) Connaître (\pageref{subsec:ab_knowing})
\item Rang 2:
Destiné à la grandeur (\pageref{subsec:ab_destined_for_greatness})
\item Rang 3:
Choisissez en une: Résilience durement gagnée (\pageref{subsec:ab_hard_won_resilience}) ou Surmontez tous les obstacles (\pageref{subsec:ab_overcome_all_obstacles})
\item Rang 4:
Centre d'attention (\pageref{subsec:ab_center_of_attention})
\item Rang 5:
Montrez-leur le chemin (\pageref{subsec:ab_show_them_the_way})
\item Rang 6:
Choisissez en une: Comme le prédit la prophétie (\pageref{subsec:ab_as_foretold_in_prophecy}) ou Potentiel amélioré plus important (\pageref{subsec:ab_greater_enhanced_potential})
\end{abnamelist}
\textbf{Intrusion de la Meneuse:}
%_____________________________________________________%
\phantomsection\label{sec:focusabsorbsenergy}\section*{ Absorbe l'Energie}

\textcolor{gray}{\emph{ Absorbs Energy}}

Vous pouvez exploiter l’énergie cinétique et la transformer en d’autres types d’énergie.

\begin{abnamelist}
\item Rang 1: Absorber l'énergie cinétique (\pageref{subsec:ab_absorb_kinetic_energy}) Libération d'énergie (\pageref{subsec:ab_release_energy})
\item Rang 2:
Dynamiser un objet (\pageref{subsec:ab_energize_object})
\item Rang 3:
Choisissez en une: Absorber l'énergie pure (\pageref{subsec:ab_absorb_pure_energy}) ou Absorption d'énergie cinétique améliorée (\pageref{subsec:ab_improved_absorb_kinetic_energy})
\item Rang 4:
Surcharge d'énergie (\pageref{subsec:ab_overcharge_energy})
\item Rang 5:
Energiser la créature (\pageref{subsec:ab_energize_creature})
\item Rang 6:
Choisissez en une: Energiser la foule (\pageref{subsec:ab_energize_crowd}) ou Surcharge d'Appareil (\pageref{subsec:ab_overcharge_device})
\end{abnamelist}
\textbf{Intrusion de la Meneuse:}
%_____________________________________________________%
\phantomsection\label{sec:focusperformsfeatsofstrength}\section*{ Accompli des Prouesses de Force}

\textcolor{gray}{\emph{ Performs Feats of Strength}}

Prodige en musculature, vous pouvez transporter un poids incroyable, projeter votre corps dans les airs et percer des portes.

\begin{abnamelist}
\item Rang 1: Athlète (\pageref{subsec:ab_athlete}) Avantage de Puissance Amélioré (\pageref{subsec:ab_enhanced_might_edge})
\item Rang 2:
Tour de force (\pageref{subsec:ab_feat_of_strength})
\item Rang 3:
Choisissez en une: Lancer (\pageref{subsec:ab_throw}) ou Poing de Fer (\pageref{subsec:ab_iron_fist})
\item Rang 4:
Puissance améliorée supérieure (\pageref{subsec:ab_greater_enhanced_might})
\item Rang 5:
Coup brutal (\pageref{subsec:ab_brute_strike})
\item Rang 6:
Choisissez en une: Attaque sautée (\pageref{subsec:ab_jump_attack}) ou Puissance améliorée supérieure (\pageref{subsec:ab_greater_enhanced_might})
\end{abnamelist}
\textbf{Intrusion de la Meneuse:}
%_____________________________________________________%
\phantomsection\label{sec:focushelpstheirfriends}\section*{ Aide ses Amis}

\textcolor{gray}{\emph{ Helps Their Friends}}

Vous aimez vos amis et les aidez à sortir de toute difficulté, quoi qu'il arrive. \newline\textbf{Option d'échange de type:} Conseils d'un ami

\begin{abnamelist}
\item Rang 1: Aide amicale (\pageref{subsec:ab_friendly_help}) Courageux (\pageref{subsec:ab_courageous})
\item Rang 2:
Faites face aux vicissitudes (\pageref{subsec:ab_weather_the_vicissitudes})
\item Rang 3:
Choisissez en une: Compétence avec les attaques (\pageref{subsec:ab_skill_with_attacks}) ou Copain (\pageref{subsec:ab_buddy_system})
\item Rang 4:
En Danger (\pageref{subsec:ab_in_harms_way}) Physique amélioré (\pageref{subsec:ab_enhanced_physique})
\item Rang 5:
Inspirer l'action (\pageref{subsec:ab_inspire_action})
\item Rang 6:
Choisissez en une: Compétence en défense (\pageref{subsec:ab_skill_with_defense}) ou Considération approfondie (\pageref{subsec:ab_deep_consideration})
\end{abnamelist}
\textbf{Intrusion de la Meneuse:}
%_____________________________________________________%
\phantomsection\label{sec:focuslovesthevoid}\section*{ Aime le Vide}

\textcolor{gray}{\emph{ Loves the Void}}

Lorsqu'il n'y a que vous, votre combinaison spatiale et le panorama d'étoiles qui défilent pour toujours et à jamais, vous êtes en paix. \newline\textbf{Option d'échange de type:} Ayez une Combinaison Spatiale, Vous Voyagerez

\begin{abnamelist}
\item Rang 1: Adepte de la microgravité (\pageref{subsec:ab_microgravity_adept}) Compétences sous Vide Spatial (\pageref{subsec:ab_vacuum_skilled})
\item Rang 2:
Avantage de Célérité Amélioré (\pageref{subsec:ab_enhanced_speed_edge}) Physique amélioré (\pageref{subsec:ab_enhanced_physique})
\item Rang 3:
Choisissez en une: Armure Corporelle (\pageref{subsec:ab_fusion_armor}) ou Combat spatial (\pageref{subsec:ab_space_fighting})
\item Rang 4:
Sauter en Microgravité (\pageref{subsec:ab_push_off_and_throw}) Silencieux comme l'espace (\pageref{subsec:ab_silent_as_space})
\item Rang 5:
Evitement par microgravité (\pageref{subsec:ab_microgravity_avoidance})
\item Rang 6:
Choisissez en une: Champ réactif (\pageref{subsec:ab_reactive_field}) ou Tir en apesanteur (\pageref{subsec:ab_weightless_shot})
\end{abnamelist}
\textbf{Intrusion de la Meneuse:}
%_____________________________________________________%
\phantomsection\label{sec:focuslearnsquickly}\section*{ Apprend Rapidement}

\textcolor{gray}{\emph{ Learns Quickly}}

Vous faites face aux mauvaises situations au fur et à mesure qu’elles surviennent, apprenant à chaque fois de nouvelles leçons.

\begin{abnamelist}
\item Rang 1: Intellect amélioré (\pageref{subsec:ab_enhanced_intellect}) Voilà votre problème (\pageref{subsec:ab_theres_your_problem})
\item Rang 2:
Etude rapide (\pageref{subsec:ab_quick_study})
\item Rang 3:
Choisissez en une: Difficile à distraire (\pageref{subsec:ab_hard_to_distract}) Avantage d'Intellect Amélioré (\pageref{subsec:ab_enhanced_intellect_edge}) ou Compétences en Gage (\pageref{subsec:ab_flex_skill})
\item Rang 4:
Passer l'information au suivant (\pageref{subsec:ab_pay_it_forward})
\item Rang 5:
Appris des trucs (\pageref{subsec:ab_learned_a_few_things}) Intellect amélioré (\pageref{subsec:ab_enhanced_intellect})
\item Rang 6:
Choisissez en une: Compétence en Défense Supérieure (\pageref{subsec:ab_greater_skill_with_defense}) ou Deux choses à la fois (\pageref{subsec:ab_two_things_at_once})
\end{abnamelist}
\textbf{Intrusion de la Meneuse:}
%_____________________________________________________%
\phantomsection\label{sec:focusmurders}\section*{ Assassine}

\textcolor{gray}{\emph{ Murders}}

Vous êtes un assassin, que ce soit par métier, par inclination, ou parce que vous vouliez vous faire tuer. (Quelqu'un qui Assassine peut disposer d'un équipement supplémentaire, notamment trois doses d'un poison de lame de niveau 2 qui inflige 5 points de dégâts.)

\begin{abnamelist}
\item Rang 1: Attaque surprise (\pageref{subsec:ab_surprise_attack}) Compétences d'assassin (\pageref{subsec:ab_assassin_skills})
\item Rang 2:
Infiltrateur (\pageref{subsec:ab_infiltrator}) Mort rapide (\pageref{subsec:ab_quick_death})
\item Rang 3:
Choisissez en une: Artisan de Poisons (\pageref{subsec:ab_poison_crafter}) ou Conscience (\pageref{subsec:ab_awareness})
\item Rang 4:
Attaque Surprise Améliorée (\pageref{subsec:ab_better_surprise_attack})
\item Rang 5:
Augmente les dommages (\pageref{subsec:ab_damage_dealer})
\item Rang 6:
Choisissez en une: Meurtrier (\pageref{subsec:ab_murderer}) ou Plan d'évasion (\pageref{subsec:ab_escape_plan})
\end{abnamelist}
\textbf{Intrusion de la Meneuse:}
%_____________________________________________________%
\phantomsection\label{sec:focusmoveslikeacat}\section*{ Bouge comme un Chat}

\textcolor{gray}{\emph{ Moves Like a Cat}}

Souple, flexible et gracieux, vous vous déplacez rapidement et en douceur, et ne semblez jamais être là où se trouve le danger.

\begin{abnamelist}
\item Rang 1: Célérité améliorée supérieure (\pageref{subsec:ab_greater_enhanced_speed}) Equilibre (\pageref{subsec:ab_balance})
\item Rang 2:
Chute en toute sécurité (\pageref{subsec:ab_safe_fall}) Habiletés motrices (\pageref{subsec:ab_movement_skills})
\item Rang 3:
Choisissez en une: Difficile à toucher (\pageref{subsec:ab_hard_to_hit}) Avantage de Célérité Amélioré (\pageref{subsec:ab_enhanced_speed_edge}) ou Célérité améliorée supérieure (\pageref{subsec:ab_greater_enhanced_speed})
\item Rang 4:
Frappe rapide (\pageref{subsec:ab_quick_strike})
\item Rang 5:
Glissant (\pageref{subsec:ab_slippery})
\item Rang 6:
Choisissez en une: Célérité améliorée supérieure (\pageref{subsec:ab_greater_enhanced_speed}) ou Sursaut de Célérité Parfait (\pageref{subsec:ab_perfect_speed_burst})
\end{abnamelist}
\textbf{Intrusion de la Meneuse:}
%_____________________________________________________%
\phantomsection\label{sec:focusbrandishesanexoticshield}\section*{ Brandit un Bouclier Exotique}

\textcolor{gray}{\emph{ Brandishes an Exotic Shield}}

Vous déployez un incroyable bouclier de force pure qui offre une protection et des options offensives.

\begin{abnamelist}
\item Rang 1: Bouclier de Champ de Force (\pageref{subsec:ab_force_field_shield}) Frappe de Force (\pageref{subsec:ab_force_bash})
\item Rang 2:
Bouclier enveloppant (\pageref{subsec:ab_enveloping_shield})
\item Rang 3:
Choisissez en une: Lancer un bouclier de force (\pageref{subsec:ab_throw_force_shield}) ou Pulsation de Guérison (\pageref{subsec:ab_healing_pulse})
\item Rang 4:
Bouclier énergisé (\pageref{subsec:ab_energized_shield})
\item Rang 5:
Mur de Force (\pageref{subsec:ab_force_wall})
\item Rang 6:
Choisissez en une: Bouclier Explosif (\pageref{subsec:ab_shield_burst}) ou Bouclier rebondissant (\pageref{subsec:ab_bouncing_shield})
\end{abnamelist}
\textbf{Intrusion de la Meneuse:}
%_____________________________________________________%
\phantomsection\label{sec:focuscalculatestheincalculable}\section*{ Calcule l'Incalculable}

\textcolor{gray}{\emph{ Calculates the Incalculable}}

Des capacités mathématiques impressionnantes vous permettent de modéliser le monde en temps réel, vous donnant ainsi un avantage sur tout le monde.

\begin{abnamelist}
\item Rang 1: Equation prédictive (\pageref{subsec:ab_predictive_equation}) Mathématiques supérieures (\pageref{subsec:ab_higher_mathematics})
\item Rang 2:
Modèle prédictif (\pageref{subsec:ab_predictive_model})
\item Rang 3:
Choisissez en une: Défense subconsciente (\pageref{subsec:ab_subconscious_defense}) ou Intellect amélioré (\pageref{subsec:ab_enhanced_intellect})
\item Rang 4:
Diagramme de combat (\pageref{subsec:ab_cognizant_offense})
\item Rang 5:
Intellect Amélioré Supérieur (\pageref{subsec:ab_greater_enhanced_intellect}) Mathématiques Complémentaires (\pageref{subsec:ab_further_mathematics})
\item Rang 6:
Choisissez en une: Connaître l'inconnu (\pageref{subsec:ab_knowing_the_unknown}) ou Intellect Amélioré Supérieur (\pageref{subsec:ab_greater_enhanced_intellect})
\end{abnamelist}
\textbf{Intrusion de la Meneuse:}
%_____________________________________________________%
\phantomsection\label{sec:focuschannelsdivineblessings}\section*{ Canalise les Bénédictions Divines}

\textcolor{gray}{\emph{ Channels Divine Blessings}}

Fervent disciple d’un être divin, vous canalisez une partie du pouvoir de votre divinité pour réaliser des merveilles.

\begin{abnamelist}
\item Rang 1: Bénédiction des Dieux (\pageref{subsec:ab_blessing_of_the_gods})
\item Rang 2:
Intellect amélioré (\pageref{subsec:ab_enhanced_intellect})
\item Rang 3:
Choisissez en une: Cube de flammes (\pageref{subsec:ab_fire_bloom}) ou Radiance divine (\pageref{subsec:ab_divine_radiance})
\item Rang 4:
Explosion Divine (\pageref{subsec:ab_overawe})
\item Rang 5:
Intervention divine (\pageref{subsec:ab_divine_intervention})
\item Rang 6:
Choisissez en une: Invoquer un démon (\pageref{subsec:ab_summon_demon}) ou Symbole divin (\pageref{subsec:ab_divine_symbol})
\end{abnamelist}
\textbf{Intrusion de la Meneuse:}
%_____________________________________________________%
\phantomsection\label{sec:focus}\section*{ Chasse}

\textcolor{gray}{\emph{ }}

Vous êtes un chasseur traquant qui excelle à abattre la proie que vous avez choisie.

\begin{abnamelist}
\item Rang 1: Attaque avec style (\pageref{subsec:ab_attack_flourish}) Pisteur (\pageref{subsec:ab_tracker})
\item Rang 2:
Furtif (\pageref{subsec:ab_sneak}) Proie (\pageref{subsec:ab_quarry})
\item Rang 3:
Choisissez en une: Combats de Horde (\pageref{subsec:ab_horde_fighting}) ou Court et Attrape (\pageref{subsec:ab_sprint_and_grab})
\item Rang 4:
Attaque surprise (\pageref{subsec:ab_surprise_attack})
\item Rang 5:
Volonté du Chasseur (\pageref{subsec:ab_hunters_drive})
\item Rang 6:
Choisissez en une: Compétence en Attaque Supérieure (\pageref{subsec:ab_greater_skill_with_attacks}) ou Multiples Proies (\pageref{subsec:ab_multiple_quarry})
\end{abnamelist}
\textbf{Intrusion de la Meneuse:}
%_____________________________________________________%
\phantomsection\label{sec:focuslooksfortrouble}\section*{ Cherche les Ennuis}

\textcolor{gray}{\emph{ Looks for Trouble}}

Vous êtes un bagarreur et aimez les bons combats.

\begin{abnamelist}
\item Rang 1: Poings de fureur (\pageref{subsec:ab_fists_of_fury}) Premier Soins (\pageref{subsec:ab_wound_tender})
\item Rang 2:
Protecteur (\pageref{subsec:ab_protector}) Simple et Direct (\pageref{subsec:ab_straightforward})
\item Rang 3:
Choisissez en une: Compétence avec les attaques (\pageref{subsec:ab_skill_with_attacks}) ou Potentiel amélioré plus important (\pageref{subsec:ab_greater_enhanced_potential})
\item Rang 4:
Knock Out (\pageref{subsec:ab_knock_out})
\item Rang 5:
Maîtrise des attaques (\pageref{subsec:ab_mastery_with_attacks})
\item Rang 6:
Choisissez en une: Dégâts mortels (\pageref{subsec:ab_lethal_damage}) ou Puissance améliorée supérieure (\pageref{subsec:ab_greater_enhanced_might})
\end{abnamelist}
\textbf{Intrusion de la Meneuse:}
%_____________________________________________________%
\phantomsection\label{sec:focusfightswithpanache}\section*{ Combat avec Panache}

\textcolor{gray}{\emph{ Fights With Panache}}

Vous êtes un casse-cou audacieux qui se bat avec un style flamboyant et amusant à regarder.

\begin{abnamelist}
\item Rang 1: Attaque avec style (\pageref{subsec:ab_attack_flourish})
\item Rang 2:
Blocage rapide (\pageref{subsec:ab_quick_block})
\item Rang 3:
Choisissez en une: Attaque acrobatique (\pageref{subsec:ab_acrobatic_attack}) ou Vantardise flamboyante (\pageref{subsec:ab_flamboyant_boast})
\item Rang 4:
Bloquer pour un autre (\pageref{subsec:ab_block_for_another}) Meurtre Rapide (\pageref{subsec:ab_fast_kill})
\item Rang 5:
Utilisation de l'environnement (\pageref{subsec:ab_using_the_environment})
\item Rang 6:
Choisissez en une: Esprit Agile (\pageref{subsec:ab_agile_wit}) ou Retour à l'expéditeur (\pageref{subsec:ab_return_to_sender})
\end{abnamelist}
\textbf{Intrusion de la Meneuse:}
%_____________________________________________________%
\phantomsection\label{sec:focusbattlesrobots}\section*{ Combat les Robots}

\textcolor{gray}{\emph{ Battles Robots}}

Vous excellez dans la lutte contre les robots, les automates et les entités machines.

\begin{abnamelist}
\item Rang 1: Compétences techniques (\pageref{subsec:ab_tech_skills}) Vulnérabilités des machines (\pageref{subsec:ab_machine_vulnerabilities})
\item Rang 2:
Chasse aux machines (\pageref{subsec:ab_machine_hunting}) Défense contre les robots (\pageref{subsec:ab_defense_against_robots})
\item Rang 3:
Choisissez en une: Attaque surprise (\pageref{subsec:ab_surprise_attack}) ou Mécanismes de désactivation (\pageref{subsec:ab_disable_mechanisms})
\item Rang 4:
Combattant de Robot (\pageref{subsec:ab_robot_fighter})
\item Rang 5:
Drain de Puissance (\pageref{subsec:ab_drain_power/})
\item Rang 6:
Choisissez en une: Dégâts mortels (\pageref{subsec:ab_lethal_damage}) ou Désactiver les mécanismes (\pageref{subsec:ab_deactivate_mechanisms})
\end{abnamelist}
\textbf{Intrusion de la Meneuse:}
%_____________________________________________________%
\phantomsection\label{sec:focuscommandsmentalpowers}\section*{ Commande aux pouvoirs Mentaux}

\textcolor{gray}{\emph{ Commands Mental Powers}}

Vous avez perfectionné le pouvoir de votre esprit pour accomplir des actes psychiques incroyables.

\begin{abnamelist}
\item Rang 1: Télépathique (\pageref{subsec:ab_telepathic})
\item Rang 2:
Lecture mentale (\pageref{subsec:ab_mind_reading})
\item Rang 3:
Choisissez en une: Projection Psychique (\pageref{subsec:ab_psychic_burst}) ou Suggestion Psychique (\pageref{subsec:ab_psychic_suggestion})
\item Rang 4:
Utiliser les sens des autres (\pageref{subsec:ab_use_senses_of_others})
\item Rang 5:
Précognition (\pageref{subsec:ab_precognition})
\item Rang 6:
Choisissez en une: Contrôle mental (\pageref{subsec:ab_mind_control}) ou Réseau télépathique (\pageref{subsec:ab_telepathic_network})
\end{abnamelist}
\textbf{Intrusion de la Meneuse:}
%_____________________________________________________%
\phantomsection\label{sec:focusfocusesmindovermatter}\section*{ Concentre l'Esprit sur la Matière}

\textcolor{gray}{\emph{ Focuses Mind Over Matter}}

Vous pouvez déplacer des objets par télékinésie avec votre esprit sans les toucher physiquement.

\begin{abnamelist}
\item Rang 1: Détourner les attaques (\pageref{subsec:ab_divert_attacks})
\item Rang 2:
Télékinésie (\pageref{subsec:ab_telekinesis})
\item Rang 3:
Choisissez en une: Améliorer la force (\pageref{subsec:ab_enhance_strength}) ou Manteau d'opportunité (\pageref{subsec:ab_cloak_of_opportunity})
\item Rang 4:
Ramener (\pageref{subsec:ab_apportation})
\item Rang 5:
Attaque psychokinétique (\pageref{subsec:ab_psychokinetic_attack})
\item Rang 6:
Choisissez en une: Ramener amélioré (\pageref{subsec:ab_improved_apportation}) ou Remodeler (\pageref{subsec:ab_reshape})
\end{abnamelist}
\textbf{Intrusion de la Meneuse:}
%_____________________________________________________%
\phantomsection\label{sec:focusdriveslikeamaniac}\section*{ Conduit comme un Dingue}

\textcolor{gray}{\emph{ Drives Like A Maniac}}

Qu'il s'agisse d'être en équilibre sur deux roues, de sauter un autre véhicule ou de conduire de front vers une voiture ennemie venant en sens inverse, vous ne pensez pas aux risques lorsque vous êtes au volant. (Quelqu'un qui Conduit comme un Dingue doit avoir accès à un véhicule.)

\begin{abnamelist}
\item Rang 1: Chauffeur (\pageref{subsec:ab_driver}) Conduite Dangereuse (\pageref{subsec:ab_driving_on_the_edge})
\item Rang 2:
Regarde les en Face (\pageref{subsec:ab_stare_them_down}) Surfeur de voiture (\pageref{subsec:ab_car_surfer})
\item Rang 3:
Choisissez en une: Avantage de Célérité Amélioré (\pageref{subsec:ab_enhanced_speed_edge}) ou Chauffeur expert (\pageref{subsec:ab_expert_driver})
\item Rang 4:
Célérité améliorée (\pageref{subsec:ab_enhanced_speed}) Oeil perçant (\pageref{subsec:ab_sharp_eyed})
\item Rang 5:
Quelque chose sur la route (\pageref{subsec:ab_something_in_the_road})
\item Rang 6:
Choisissez en une: As du Volant (\pageref{subsec:ab_trick_driver}) ou Dégâts mortels (\pageref{subsec:ab_lethal_damage})
\end{abnamelist}
\textbf{Intrusion de la Meneuse:}
%_____________________________________________________%
\phantomsection\label{sec:focusbuildsrobots}\section*{ Construit des Robots}

\textcolor{gray}{\emph{ Builds Robots}}

Vos créations robotiques font ce qu'on leur commande.  Le mot « robot » est utilisé dans ce Focus, bien que le robot que vous créez puisse être très différent de celui créé par quelqu'un d'autre, selon le genre. Les robots Steampunk, les robots organiques ou même les golems magiques sont tous des « robots » réalisables. 

\begin{abnamelist}
\item Rang 1: Assistant Robot (\pageref{subsec:ab_robot_assistant}) Constructeur de robots (\pageref{subsec:ab_robot_builder})
\item Rang 2:
Contrôle du robot (\pageref{subsec:ab_robot_control})
\item Rang 3:
Choisissez en une: Compétence en Défense Supérieure (\pageref{subsec:ab_greater_skill_with_defense}) ou Disciple expert (\pageref{subsec:ab_expert_follower})
\item Rang 4:
Mise à niveau du robot (\pageref{subsec:ab_robot_upgrade})
\item Rang 5:
Flotte de robots (\pageref{subsec:ab_robot_fleet})
\item Rang 6:
Choisissez en une: Evolution du robot (\pageref{subsec:ab_robot_evolution}) ou Mise à niveau du robot (\pageref{subsec:ab_robot_upgrade})
\end{abnamelist}
\textbf{Intrusion de la Meneuse:}
%_____________________________________________________%
\phantomsection\label{sec:focusworksforaliving}\section*{ Construit et Répare}

\textcolor{gray}{\emph{ Works for a Living}}

Vous tirez une grande satisfaction d'un travail bien fait, qu'il s'agisse de coder, de construire des maisons ou d'exploiter des astéroïdes.

\begin{abnamelist}
\item Rang 1: Travaux Manuels (\pageref{subsec:ab_handy})
\item Rang 2:
Muscles de fer (\pageref{subsec:ab_muscles_of_iron})
\item Rang 3:
Choisissez en une: Improviser (\pageref{subsec:ab_improvise}) ou Oeil pour les détails (\pageref{subsec:ab_eye_for_detail})
\item Rang 4:
Puissance Améliorée (\pageref{subsec:ab_enhanced_might}) Tenez bon (\pageref{subsec:ab_tough_it_out})
\item Rang 5:
Compétence d'expert (\pageref{subsec:ab_expert_skill})
\item Rang 6:
Choisissez en une: Potentiel amélioré plus important (\pageref{subsec:ab_greater_enhanced_potential}) ou Résilience durement gagnée (\pageref{subsec:ab_hard_won_resilience})
\end{abnamelist}
\textbf{Intrusion de la Meneuse:}
%_____________________________________________________%
\phantomsection\label{sec:focusworksthesystem}\section*{ Contourne le Système}

\textcolor{gray}{\emph{ Works the System}}

Vous pouvez exploiter les failles des systèmes artificiels, y compris, mais sans s'y limiter, le code informatique.

\begin{abnamelist}
\item Rang 1: Hackez l'impossible (\pageref{subsec:ab_hack_the_impossible}) Programmation informatique (\pageref{subsec:ab_computer_programming})
\item Rang 2:
Connecté (\pageref{subsec:ab_connected})
\item Rang 3:
Choisissez en une: Artiste de la confiance (\pageref{subsec:ab_confidence_artist}) ou Compétence avec les attaques (\pageref{subsec:ab_skill_with_attacks})
\item Rang 4:
Confondre l'ennemi (\pageref{subsec:ab_confuse_enemy})
\item Rang 5:
Entretenir l'amitié (\pageref{subsec:ab_work_the_friendship})
\item Rang 6:
Choisissez en une: Demande d'une faveur (\pageref{subsec:ab_call_in_favor}) ou Potentiel amélioré plus important (\pageref{subsec:ab_greater_enhanced_potential})
\end{abnamelist}
\textbf{Intrusion de la Meneuse:}
%_____________________________________________________%
\phantomsection\label{sec:focuscontrolsgravity}\section*{ Contrôle la Gravité}

\textcolor{gray}{\emph{ Controls Gravity}}

Vous pouvez influencer l’attraction de la gravité elle-même. \newline\textbf{Option d'échange de type:} Lourd

\begin{abnamelist}
\item Rang 1: Survol (\pageref{subsec:ab_hover})
\item Rang 2:
Avantage de Célérité Amélioré (\pageref{subsec:ab_enhanced_speed_edge})
\item Rang 3:
Choisissez en une: Définir le bas (\pageref{subsec:ab_define_down}) ou Gravité Coupante (\pageref{subsec:ab_gravity_cleave})
\item Rang 4:
Champ de gravité (\pageref{subsec:ab_field_of_gravity})
\item Rang 5:
Vol (\pageref{subsec:ab_flight})
\item Rang 6:
Choisissez en une: Gravité Coupante Améliorée (\pageref{subsec:ab_improved_gravity_cleave}) ou Poids du monde (\pageref{subsec:ab_weight_of_the_world})
\end{abnamelist}
\textbf{Intrusion de la Meneuse:}
%_____________________________________________________%
\phantomsection\label{sec:focusemploysmagnetism}\section*{ Contrôle le Magnétisme}

\textcolor{gray}{\emph{ Employs Magnetism}}

Vous maîtrisez le métal et le pouvoir du magnétisme.

\begin{abnamelist}
\item Rang 1: Déplacer le métal (\pageref{subsec:ab_move_metal})
\item Rang 2:
Repousser le métal (\pageref{subsec:ab_repel_metal})
\item Rang 3:
Choisissez en une: Détruire le métal (\pageref{subsec:ab_destroy_metal}) ou Tir Guidé (\pageref{subsec:ab_guide_bolt})
\item Rang 4:
Champ magnétique (\pageref{subsec:ab_magnetic_field})
\item Rang 5:
Commander le Métal (\pageref{subsec:ab_command_metal})
\item Rang 6:
Choisissez en une: Diamagnétisme (\pageref{subsec:ab_diamagnetism}) ou Lancer Objet en Fer (\pageref{subsec:ab_iron_punch})
\end{abnamelist}
\textbf{Intrusion de la Meneuse:}
%_____________________________________________________%
\phantomsection\label{sec:focuscontrolsbeasts}\section*{ Contrôle les Bêtes Sauvages}

\textcolor{gray}{\emph{ Controls Beasts}}

Votre capacité à communiquer et à diriger des bêtes est surnaturelle.

\begin{abnamelist}
\item Rang 1: Une Bête comme Compagnon (\pageref{subsec:ab_beast_companion})
\item Rang 2:
Apaiser la Bête Sauvage (\pageref{subsec:ab_soothe_the_savage}) Communication (\pageref{subsec:ab_communication})
\item Rang 3:
Choisissez en une: Monture (\pageref{subsec:ab_mount}) ou Plus forts ensemble (\pageref{subsec:ab_stronger_together})
\item Rang 4:
Compagnon amélioré (\pageref{subsec:ab_improved_companion}) Yeux de bête (\pageref{subsec:ab_beast_eyes})
\item Rang 5:
Appel de Bête (\pageref{subsec:ab_beast_call})
\item Rang 6:
Choisissez en une: Comme s'il n'y a qu'une seule créature (\pageref{subsec:ab_as_if_one_creature}) ou Contrôler le sauvage (\pageref{subsec:ab_control_the_savage})
\end{abnamelist}
\textbf{Intrusion de la Meneuse:}
%_____________________________________________________%
\phantomsection\label{sec:focuswieldsinvisibleforce}\section*{ Contrôle une Force Invisible}

\textcolor{gray}{\emph{ Wields Invisible Force}}

Vous pliez la lumière et manipulez des faisceaux de force pour l’attaque et la défense.

\begin{abnamelist}
\item Rang 1: Disparaître (\pageref{subsec:ab_vanish})
\item Rang 2:
Force enchevêtrante (\pageref{subsec:ab_entangling_force}) Sens aiguisés (\pageref{subsec:ab_sharp_senses})
\item Rang 3:
Choisissez en une: Barrière de champ de force (\pageref{subsec:ab_force_field_barrier}) ou Invisibilité Multiple (\pageref{subsec:ab_multi_vanish})
\item Rang 4:
Invisibilité (\pageref{subsec:ab_invisibility})
\item Rang 5:
Champ défensif (\pageref{subsec:ab_defensive_field})
\item Rang 6:
Choisissez en une: Concussion (\pageref{subsec:ab_concussion}) ou Générer un champ de force (\pageref{subsec:ab_generate_force_field})
\end{abnamelist}
\textbf{Intrusion de la Meneuse:}
%_____________________________________________________%
\phantomsection\label{sec:focuscopiessuperpowers}\section*{ Copie des Superpouvoirs}

\textcolor{gray}{\emph{ Copies Superpowers}}

Vous pouvez copier les compétences, les capacités et les super pouvoirs des autres.

\begin{abnamelist}
\item Rang 1: Compétences en Gage (\pageref{subsec:ab_flex_skill})
\item Rang 2:
Pouvoir de copie (\pageref{subsec:ab_copy_power})
\item Rang 3:
Choisissez en une: Pouvoirs génériques (\pageref{subsec:ab_wildcard_powers}) ou Voler le pouvoir (\pageref{subsec:ab_steal_power})
\item Rang 4:
Copie Améliorée (\pageref{subsec:ab_improved_copying})
\item Rang 5:
Mémoire de Pouvoir Copié (\pageref{subsec:ab_power_memory})
\item Rang 6:
Choisissez en une: Copie multiple (\pageref{subsec:ab_multiple_copying}) ou Copie étonnante (\pageref{subsec:ab_amazing_copying})
\end{abnamelist}
\textbf{Intrusion de la Meneuse:}
%_____________________________________________________%
\phantomsection\label{sec:focusabidesinstone}\section*{ Demeure dans la pierre}

\textcolor{gray}{\emph{ Abides in Stone}}

Votre chair est constituée de minéraux durs, ce qui fait de vous un humanoïde imposant et difficile à blesser.

\begin{abnamelist}
\item Rang 1: Corps de Golem (\pageref{subsec:ab_golem_body}) Guérison du Golem (\pageref{subsec:ab_golem_healing})
\item Rang 2:
Prise de Golem (\pageref{subsec:ab_golem_grip})
\item Rang 3:
Choisissez en une: Cogneur Entraîné (\pageref{subsec:ab_trained_basher}) Armement (\pageref{subsec:ab_weaponization}) ou Piétinement de Golem (\pageref{subsec:ab_golem_stomp})
\item Rang 4:
Réserves Partagées (\pageref{subsec:ab_deep_reserves})
\item Rang 5:
Cogneur Spécialisé (\pageref{subsec:ab_specialized_basher}) Toujours comme une statue (\pageref{subsec:ab_still_as_a_statue})
\item Rang 6:
Choisissez en une: Sursaut mental (\pageref{subsec:ab_mind_surge}) ou Ultra amélioration (\pageref{subsec:ab_ultra_enhancement})
\end{abnamelist}
\textbf{Intrusion de la Meneuse:}
%_____________________________________________________%
\phantomsection\label{sec:focusleads}\section*{ Dirige}

\textcolor{gray}{\emph{ Leads}}

Votre capacité naturelle de leadership vous permet de commander aux autres, y compris à un groupe de fidèles.

\begin{abnamelist}
\item Rang 1: Bon conseil (\pageref{subsec:ab_good_advice}) Charisme naturel (\pageref{subsec:ab_natural_charisma})
\item Rang 2:
Potentiel amélioré (\pageref{subsec:ab_enhanced_potential}) Suivant de base (\pageref{subsec:ab_basic_follower})
\item Rang 3:
Choisissez en une: Commandement avancé (\pageref{subsec:ab_advanced_command}) ou Disciple expert (\pageref{subsec:ab_expert_follower})
\item Rang 4:
Captiver ou Inspirer (\pageref{subsec:ab_captivate_or_inspire})
\item Rang 5:
Potentiel amélioré plus important (\pageref{subsec:ab_greater_enhanced_potential})
\item Rang 6:
Choisissez en une: Bande de suivants (\pageref{subsec:ab_band_of_followers}) ou Esprit de leader (\pageref{subsec:ab_mind_of_a_leader})
\end{abnamelist}
\textbf{Intrusion de la Meneuse:}
%_____________________________________________________%
\phantomsection\label{sec:focuskeepsamagically}\section*{ Dispose d'un Allié Magique}

\textcolor{gray}{\emph{ Keeps a Magic Ally}}

Une créature magique alliée liée à un objet (comme un djinn mineur dans une lampe, ou un fantôme dans une pipe) est votre ami, votre Protecteur et votre arme.

\begin{abnamelist}
\item Rang 1: Créature magique liée (\pageref{subsec:ab_bound_magic_creature})
\item Rang 2:
Lien d'objet (\pageref{subsec:ab_object_bond}) Placard caché (\pageref{subsec:ab_hidden_closet})
\item Rang 3:
Choisissez en une: Monture (\pageref{subsec:ab_mount}) ou Souhait mineur (\pageref{subsec:ab_minor_wish})
\item Rang 4:
Lien d'Objet Amélioré (\pageref{subsec:ab_improved_object_bond})
\item Rang 5:
Souhait modéré (\pageref{subsec:ab_moderate_wish})
\item Rang 6:
Choisissez en une: Faites confiance à la chance (\pageref{subsec:ab_trust_to_luck}) ou Maîtrise des liens d'objet (\pageref{subsec:ab_object_bond_mastery})
\end{abnamelist}
\textbf{Intrusion de la Meneuse:}
%_____________________________________________________%
\phantomsection\label{sec:focusentertains}\section*{ Divertit}

\textcolor{gray}{\emph{ Entertains}}

Vous vous donnez en spectacle principalement pour divertir les autres.

\begin{abnamelist}
\item Rang 1: Légèreté (\pageref{subsec:ab_levity})
\item Rang 2:
Facilité Inspirante (\pageref{subsec:ab_inspiring_ease})
\item Rang 3:
Choisissez en une: Compétences en Connaissances (\pageref{subsec:ab_knowledge_skills}) ou Potentiel amélioré plus important (\pageref{subsec:ab_greater_enhanced_potential})
\item Rang 4:
Calme (\pageref{subsec:ab_calm})
\item Rang 5:
Assistance compétente (\pageref{subsec:ab_able_assistance})
\item Rang 6:
Choisissez en une: Maître du spectacle (\pageref{subsec:ab_master_entertainer}) ou Performance vindicative (\pageref{subsec:ab_vindictive_performance})
\end{abnamelist}
\textbf{Intrusion de la Meneuse:}
%_____________________________________________________%
\phantomsection\label{sec:focusshredsthewallsoftheworld}\section*{ Déchire les Murs du Monde}

\textcolor{gray}{\emph{ Shreds the Walls of the World}}

La vitesse et le phasage vous donnent une capacité unique à échapper au danger et à infliger simultanément des dégâts.

\begin{abnamelist}
\item Rang 1: Sprint de Phase (\pageref{subsec:ab_phase_sprint}) Toucher perturbateur (\pageref{subsec:ab_disrupting_touch})
\item Rang 2:
Rayer l'Existence (\pageref{subsec:ab_scratch_existence})
\item Rang 3:
Choisissez en une: Invisibilité de Phase (\pageref{subsec:ab_invisible_phasing}) ou Traverser les murs (\pageref{subsec:ab_walk_through_walls})
\item Rang 4:
Détonation de phase (\pageref{subsec:ab_phase_detonation})
\item Rang 5:
Très long sprint (\pageref{subsec:ab_very_long_sprinting})
\item Rang 6:
Choisissez en une: Déchirer L'Existence (\pageref{subsec:ab_shred_existence}) ou Intouchable en mouvement (\pageref{subsec:ab_untouchable_while_moving})
\end{abnamelist}
\textbf{Intrusion de la Meneuse:}
%_____________________________________________________%
\phantomsection\label{sec:focusdefendstheweak}\section*{ Défend les Faibles}

\textcolor{gray}{\emph{ Defends the Weak}}

Vous défendez les impuissants, les faibles et ceux qui ne sont pas protégés.

\begin{abnamelist}
\item Rang 1: Bouclier de protection (\pageref{subsec:ab_warding_shield}) Courageux (\pageref{subsec:ab_courageous})
\item Rang 2:
Perspicacité (\pageref{subsec:ab_insight}) Véritable défenseur (\pageref{subsec:ab_true_defender})
\item Rang 3:
Choisissez en une: Double protection (\pageref{subsec:ab_dual_wards}) ou Véritable Gardien (\pageref{subsec:ab_true_guardian})
\item Rang 4:
Défi de combat (\pageref{subsec:ab_combat_challenge})
\item Rang 5:
Sacrifice volontaire (\pageref{subsec:ab_willing_sacrifice})
\item Rang 6:
Choisissez en une: Réanimer (\pageref{subsec:ab_resuscitate}) ou Véritable défenseur (\pageref{subsec:ab_true_defender})
\end{abnamelist}
\textbf{Intrusion de la Meneuse:}
%_____________________________________________________%
\phantomsection\label{sec:focusisidolizedbymillions}\section*{ Est Idolatré par Millions}

\textcolor{gray}{\emph{ Is Idolized by Millions}}

Vous êtes une célébrité et la plupart des gens vous adorent.

\begin{abnamelist}
\item Rang 1: Entourage (\pageref{subsec:ab_entourage}) Talent de célébrité (\pageref{subsec:ab_celebrity_talent})
\item Rang 2:
Avantages de la célébrité (\pageref{subsec:ab_perks_of_stardom})
\item Rang 3:
Choisissez en une: Compétence avec les attaques (\pageref{subsec:ab_skill_with_attacks}) ou Santé incroyable (\pageref{subsec:ab_incredible_health})
\item Rang 4:
Captiver avec Eclat (\pageref{subsec:ab_captivate_with_starshine}) Disciple expert (\pageref{subsec:ab_expert_follower})
\item Rang 5:
Est-ce que tu sais qui je suis? (\pageref{subsec:ab_do_you_know_who_i_am?})
\item Rang 6:
Choisissez en une: Compagnon amélioré (\pageref{subsec:ab_improved_companion}) ou Transcendez le scénario (\pageref{subsec:ab_transcend_the_script})
\end{abnamelist}
\textbf{Intrusion de la Meneuse:}
%_____________________________________________________%
\phantomsection\label{sec:focusiswantedbythelaw}\section*{ Est Recherché par la Loi}

\textcolor{gray}{\emph{ Is Wanted by the Law}}

Des affiches "WANTED, DEAD or ALIVE" (ou leur équivalent) sont apparues avec votre visage. C'est à vous de décider si c'est une erreur qui est devenue incontrôlable ou si vous tueriez quelqu'un juste pour un regard.

\begin{abnamelist}
\item Rang 1: Célérité améliorée (\pageref{subsec:ab_enhanced_speed}) Sens du Danger (\pageref{subsec:ab_danger_sense})
\item Rang 2:
Attaque surprise (\pageref{subsec:ab_surprise_attack})
\item Rang 3:
Choisissez en une: Attaque successive (\pageref{subsec:ab_successive_attack}) ou Réputation de hors-la-loi (\pageref{subsec:ab_outlaw_reputation})
\item Rang 4:
Meurtre Rapide (\pageref{subsec:ab_fast_kill})
\item Rang 5:
Bande de Desperados (\pageref{subsec:ab_band_of_desperados})
\item Rang 6:
Choisissez en une: Dégâts mortels (\pageref{subsec:ab_lethal_damage}) ou Pas encore mort (\pageref{subsec:ab_not_dead_yet})
\end{abnamelist}
\textbf{Intrusion de la Meneuse:}
%_____________________________________________________%
\phantomsection\label{sec:focusemergedfromtheobelisk}\section*{ Est sorti de l'Obélisque}

\textcolor{gray}{\emph{ Emerged From the Obelisk}}

Votre corps, dur comme du cristal, vous confère une suite de capacités uniques, acquises après une interaction avec un obélisque cristallin flottant.

\begin{abnamelist}
\item Rang 1: Corps de Cristal (\pageref{subsec:ab_crystalline_body})
\item Rang 2:
Survol (\pageref{subsec:ab_hover})
\item Rang 3:
Choisissez en une: Habiter le cristal (\pageref{subsec:ab_inhabit_crystal}) ou Immobile (\pageref{subsec:ab_immovable})
\item Rang 4:
Lentille Cristalline (\pageref{subsec:ab_crystal_lens})
\item Rang 5:
Fréquence de résonance (\pageref{subsec:ab_resonant_frequency})
\item Rang 6:
Choisissez en une: Retour à l'Obélisque (\pageref{subsec:ab_return_to_the_obelisk}) ou Tremblement de résonance (\pageref{subsec:ab_resonant_quake})
\end{abnamelist}
\textbf{Intrusion de la Meneuse:}
%_____________________________________________________%
\phantomsection\label{sec:focusscavenges}\section*{ Est un Récupérateur}

\textcolor{gray}{\emph{ Scavenges}}

Lorsque vous ne courez pas et ne vous cachez pas, vous fouillez les ruines de la civilisation à la recherche de vestiges utiles pour assurer votre survie.

\begin{abnamelist}
\item Rang 1: Connaissance des ruines (\pageref{subsec:ab_ruin_lore}) Survivant post-apocalyptique (\pageref{subsec:ab_post_apocalyptic_survivor})
\item Rang 2:
De Déchet à Objet (\pageref{subsec:ab_junkmonger})
\item Rang 3:
Choisissez en une: Prendre l'avantage (\pageref{subsec:ab_taking_advantage}) ou Santé incroyable (\pageref{subsec:ab_incredible_health})
\item Rang 4:
Sachez où chercher (\pageref{subsec:ab_know_where_to_look})
\item Rang 5:
Cyphers recyclés (\pageref{subsec:ab_recycled_cyphers})
\item Rang 6:
Choisissez en une: Champ réactif (\pageref{subsec:ab_reactive_field}) ou Pilleur d'artifact (\pageref{subsec:ab_artifact_scavenger})
\end{abnamelist}
\textbf{Intrusion de la Meneuse:}
%_____________________________________________________%
\phantomsection\label{sec:focusexistspartiallyoutofphase}\section*{ Existe Partiellement Hors de Phase}

\textcolor{gray}{\emph{ Exists Partially Out of Phase}}

Un peu translucide, vous êtes légèrement déphasé et pouvez vous déplacer à travers des objets solides.

\begin{abnamelist}
\item Rang 1: Traverser les murs (\pageref{subsec:ab_walk_through_walls})
\item Rang 2:
Phase défensive (\pageref{subsec:ab_defensive_phasing})
\item Rang 3:
Choisissez en une: Attaque de Phase (\pageref{subsec:ab_phased_attack}) ou Porte de phase (\pageref{subsec:ab_phase_door})
\item Rang 4:
Fantôme (\pageref{subsec:ab_ghost})
\item Rang 5:
Intouchable (\pageref{subsec:ab_untouchable})
\item Rang 6:
Choisissez en une: Attaque de Phase Améliorée (\pageref{subsec:ab_enhanced_phased_attack}) ou Mettre un Ennemi en Phase (\pageref{subsec:ab_phase_foe})
\end{abnamelist}
\textbf{Intrusion de la Meneuse:}
%_____________________________________________________%
\phantomsection\label{sec:focusexistsintwoplacesatonce}\section*{ Existe en Deux Endroits en Même Temps}

\textcolor{gray}{\emph{ Exists in Two Places at Once}}

Vous existez en deux endroits en même Ttemps.

\begin{abnamelist}
\item Rang 1: Duplicata (\pageref{subsec:ab_duplicate})
\item Rang 2:
Partager les sens (\pageref{subsec:ab_share_senses})
\item Rang 3:
Choisissez en une: Duplicata résilient (\pageref{subsec:ab_resilient_duplicate}) ou Duplication Supérieure (\pageref{subsec:ab_superior_duplicate})
\item Rang 4:
Transfert de dégâts (\pageref{subsec:ab_damage_transference})
\item Rang 5:
Effort coordonné (\pageref{subsec:ab_coordinated_effort})
\item Rang 6:
Choisissez en une: Duplicata résilient (\pageref{subsec:ab_resilient_duplicate}) ou Multiplicité (\pageref{subsec:ab_multiplicity})
\end{abnamelist}
\textbf{Intrusion de la Meneuse:}
%_____________________________________________________%
\phantomsection\label{sec:focusexploresdarkplaces}\section*{ Explore des Endroits Sombres}

\textcolor{gray}{\emph{ Explores Dark Places}}

Vous êtes l'archétype du chasseur de trésors, du charognard et du chercheur d'objets perdus.

\begin{abnamelist}
\item Rang 1: Explorateur Confirmé (\pageref{subsec:ab_superb_explorer})
\item Rang 2:
Infiltrateur Confirmé (\pageref{subsec:ab_superb_infiltrator}) Yeux ajustés (\pageref{subsec:ab_eyes_adjusted})
\item Rang 3:
Choisissez en une: Client Fuyant (\pageref{subsec:ab_slippery_customer}) ou Frappe nocturne (\pageref{subsec:ab_nightstrike})
\item Rang 4:
Résilience durement gagnée (\pageref{subsec:ab_hard_won_resilience})
\item Rang 5:
Explorateur des ténèbres (\pageref{subsec:ab_dark_explorer})
\item Rang 6:
Choisissez en une: Attaque aveuglante (\pageref{subsec:ab_blinding_attack}) ou Fusionne avec les Ténèbres (\pageref{subsec:ab_embraced_by_darkness})
\end{abnamelist}
\textbf{Intrusion de la Meneuse:}
%_____________________________________________________%
\phantomsection\label{sec:focusthunders}\section*{ Fait Résonner le Tonnerre}

\textcolor{gray}{\emph{ Thunders}}

Vous émettez un son destructeur et manipulez les paysages sonores.

\begin{abnamelist}
\item Rang 1: Faisceau de Tonnerre (\pageref{subsec:ab_thunder_beam})
\item Rang 2:
Barrière de conversion sonore (\pageref{subsec:ab_sound_conversion_barrier})
\item Rang 3:
Choisissez en une: Annuler le son (\pageref{subsec:ab_nullify_sound}) ou Echolocation (\pageref{subsec:ab_echolocation})
\item Rang 4:
Cri fracassant (\pageref{subsec:ab_shattering_shout})
\item Rang 5:
Amplifier les sons (\pageref{subsec:ab_amplify_sounds}) Grondement subsonique (\pageref{subsec:ab_subsonic_rumble})
\item Rang 6:
Choisissez en une: Tremblement de terre (\pageref{subsec:ab_earthquake}) ou Vibration mortelle (\pageref{subsec:ab_lethal_vibration})
\end{abnamelist}
\textbf{Intrusion de la Meneuse:}
%_____________________________________________________%
\phantomsection\label{sec:focusworksmiracles}\section*{ Fait des Miracles}

\textcolor{gray}{\emph{ Works Miracles}}

Vous pouvez guérir les autres d’un simple toucher, modifier le temps pour aider les autres et êtes généralement aimé de tous.

\begin{abnamelist}
\item Rang 1: Main Guérisseuse (\pageref{subsec:ab_healing_touch})
\item Rang 2:
Soulager (\pageref{subsec:ab_alleviate})
\item Rang 3:
Choisissez en une: Santé miraculeuse (\pageref{subsec:ab_miraculous_health}) ou Source de guérison (\pageref{subsec:ab_font_of_healing})
\item Rang 4:
Inspirer l'action (\pageref{subsec:ab_inspire_action})
\item Rang 5:
Annuler (\pageref{subsec:ab_undo})
\item Rang 6:
Choisissez en une: Main Guérisseuse Supérieure (\pageref{subsec:ab_greater_healing_touch}) ou Restaurer la vie (\pageref{subsec:ab_restore_life})
\end{abnamelist}
\textbf{Intrusion de la Meneuse:}
%_____________________________________________________%
\phantomsection\label{sec:focuscraftsillusions}\section*{ Façonne des Illusions}

\textcolor{gray}{\emph{ Crafts Illusions}}

Vous façonnez des images à partir de la lumière qui sont si parfaites qu’elles semblent réelles.

\begin{abnamelist}
\item Rang 1: Illusion mineure (\pageref{subsec:ab_minor_illusion})
\item Rang 2:
Déguisement illusoire (\pageref{subsec:ab_illusory_disguise})
\item Rang 3:
Choisissez en une: Illusion majeure (\pageref{subsec:ab_major_illusion}) ou Lancer Illusion (\pageref{subsec:ab_cast_illusion})
\item Rang 4:
Image Miroir (\pageref{subsec:ab_illusory_selves})
\item Rang 5:
Image terrifiante (\pageref{subsec:ab_terrifying_image})
\item Rang 6:
Choisissez en une: Illusion grandiose (\pageref{subsec:ab_grandiose_illusion}) ou Illusion permanente (\pageref{subsec:ab_permanent_illusion})
\end{abnamelist}
\textbf{Intrusion de la Meneuse:}
%_____________________________________________________%
\phantomsection\label{sec:focuscraftsuniqueobjects}\section*{ Façonne des Objets Uniques}

\textcolor{gray}{\emph{ Crafts Unique Objects}}

Vous êtes un inventeur d'objets étranges et utiles.

\begin{abnamelist}
\item Rang 1: Artisan (\pageref{subsec:ab_crafter}) Maîtrise de l'Identification des appareils (\pageref{subsec:ab_master_identifier})
\item Rang 2:
Bricoleur d'artefacts (\pageref{subsec:ab_artifact_tinkerer}) Travail rapide (\pageref{subsec:ab_quick_work})
\item Rang 3:
Choisissez en une: Armes intégrées (\pageref{subsec:ab_built_in_weaponry}) ou Maître artisan (\pageref{subsec:ab_master_crafter})
\item Rang 4:
Maîtrise des Cyphers (\pageref{subsec:ab_cyphersmith})
\item Rang 5:
Innovateur (\pageref{subsec:ab_innovator})
\item Rang 6:
Choisissez en une: Armure Corporelle (\pageref{subsec:ab_fusion_armor}) ou Inventeur (\pageref{subsec:ab_inventor})
\end{abnamelist}
\textbf{Intrusion de la Meneuse:}
%_____________________________________________________%
\phantomsection\label{sec:focusridesthelightning}\section*{ Façonne la Foudre}

\textcolor{gray}{\emph{ Rides the Lightning}}

Vous créez et déchargez de l’énergie électrique.

\begin{abnamelist}
\item Rang 1: Charge (\pageref{subsec:ab_charge}) Choc Electrique (\pageref{subsec:ab_shock})
\item Rang 2:
Comme l'éclair (\pageref{subsec:ab_bolt_rider})
\item Rang 3:
Choisissez en une: Armure électrique (\pageref{subsec:ab_electric_armor}) ou Charge Drainante (\pageref{subsec:ab_drain_charge})
\item Rang 4:
Eclairs de Puissance (\pageref{subsec:ab_bolts_of_power})
\item Rang 5:
Vol électrique (\pageref{subsec:ab_electrical_flight})
\item Rang 6:
Choisissez en une: Aussi Rapide que l'Eclair (\pageref{subsec:ab_flash_across_the_miles}) ou Mur de foudre (\pageref{subsec:ab_wall_of_lightning})
\end{abnamelist}
\textbf{Intrusion de la Meneuse:}
%_____________________________________________________%
\phantomsection\label{sec:focusfusesfleshandsteel}\section*{ Fusionne la Chair et l'Acier}

\textcolor{gray}{\emph{ Fuses Flesh and Steel}}

Votre corps est en partie une machine.

\begin{abnamelist}
\item Rang 1: Corps amélioré (\pageref{subsec:ab_enhanced_body})
\item Rang 2:
Interface (\pageref{subsec:ab_interface})
\item Rang 3:
Choisissez en une: Armement (\pageref{subsec:ab_weaponization}) ou Ensemble de détection (\pageref{subsec:ab_sensing_package})
\item Rang 4:
Fusion (\pageref{subsec:ab_fusion})
\item Rang 5:
Réserves Partagées (\pageref{subsec:ab_deep_reserves})
\item Rang 6:
Choisissez en une: Sursaut mental (\pageref{subsec:ab_mind_surge}) ou Ultra amélioration (\pageref{subsec:ab_ultra_enhancement})
\end{abnamelist}
\textbf{Intrusion de la Meneuse:}
%_____________________________________________________%
\phantomsection\label{sec:focusfusesmindandmachine}\section*{ Fussionne l'Esprit et la Machine}

\textcolor{gray}{\emph{ Fuses Mind and Machine}}

Les aides électroniques implantées dans votre cerveau font de vous une surpuissance cérébrale.

\begin{abnamelist}
\item Rang 1: Compétences en Connaissances (\pageref{subsec:ab_knowledge_skills}) Intellect amélioré (\pageref{subsec:ab_enhanced_intellect})
\item Rang 2:
Une information dans le réseau (\pageref{subsec:ab_network_tap})
\item Rang 3:
Choisissez en une: Processeur d'action (\pageref{subsec:ab_action_processor}) ou Télépathie machine (\pageref{subsec:ab_machine_telepathy})
\item Rang 4:
Compétences en Connaissances (\pageref{subsec:ab_knowledge_skills}) Intellect Amélioré Supérieur (\pageref{subsec:ab_greater_enhanced_intellect})
\item Rang 5:
Voir l'avenir (\pageref{subsec:ab_see_the_future})
\item Rang 6:
Choisissez en une: Amélioration de la machine (\pageref{subsec:ab_machine_enhancement}) ou Sursaut mental (\pageref{subsec:ab_mind_surge})
\end{abnamelist}
\textbf{Intrusion de la Meneuse:}
%_____________________________________________________%
\phantomsection\label{sec:focusdefendsthegate}\section*{ Garde le Passage}

\textcolor{gray}{\emph{ Defends the Gate}}

Tout le monde veut que vous soyez à ses côtés lorsqu’il s’agit d’un combat, car rien ne vous échappe.

\begin{abnamelist}
\item Rang 1: Position fortifiée (\pageref{subsec:ab_fortified_position}) Ralliez-vous à moi (\pageref{subsec:ab_rally_to_me})
\item Rang 2:
Esprit de Puissance (\pageref{subsec:ab_mind_for_might})
\item Rang 3:
Choisissez en une: Constructeur de fortifications (\pageref{subsec:ab_fortification_builder}) ou Détourner les attaques (\pageref{subsec:ab_divert_attacks})
\item Rang 4:
Puissance améliorée supérieure (\pageref{subsec:ab_greater_enhanced_might})
\item Rang 5:
Champ de renforcement (\pageref{subsec:ab_reinforcing_field})
\item Rang 6:
Choisissez en une: Attaque Etourdissante (\pageref{subsec:ab_stun_attack}) ou Générer un champ de force (\pageref{subsec:ab_generate_force_field})
\end{abnamelist}
\textbf{Intrusion de la Meneuse:}
%_____________________________________________________%
\phantomsection\label{sec:focusgrowstotoweringheights}\section*{ Grandit Jusqu'au Ciel}

\textcolor{gray}{\emph{ Grows to Towering Heights}}

Pendant de brèves périodes, vous pouvez grandir et, avec suffisamment d'expérience, atteindre des hauteurs imposantes.

\begin{abnamelist}
\item Rang 1: Agrandir (\pageref{subsec:ab_enlarge}) Agrandissement Efrayant (\pageref{subsec:ab_freakishly_large})
\item Rang 2:
Avantages d'être grand (\pageref{subsec:ab_advantages_of_being_big}) Plus grand (\pageref{subsec:ab_bigger})
\item Rang 3:
Choisissez en une: Enorme (\pageref{subsec:ab_huge}) ou Lancer (\pageref{subsec:ab_throw})
\item Rang 4:
Saisir (\pageref{subsec:ab_grab})
\item Rang 5:
Gargantuesque (\pageref{subsec:ab_gargantuan})
\item Rang 6:
Choisissez en une: Colossal (\pageref{subsec:ab_colossal}) ou Dégâts mortels (\pageref{subsec:ab_lethal_damage})
\end{abnamelist}
\textbf{Intrusion de la Meneuse:}
%_____________________________________________________%
\phantomsection\label{sec:focusshepherdsthecommunity}\section*{ Guide la Communauté}

\textcolor{gray}{\emph{ Shepherds the Community}}

Vous gardez le lieu où vous vivez à l'abri de tout danger.

\begin{abnamelist}
\item Rang 1: Activiste communautaire (\pageref{subsec:ab_community_activist}) Connaissance de la communauté (\pageref{subsec:ab_community_knowledge})
\item Rang 2:
Compétence avec les attaques (\pageref{subsec:ab_skill_with_attacks})
\item Rang 3:
Choisissez en une: Compétence en Défense Supérieure (\pageref{subsec:ab_greater_skill_with_defense}) ou Fureur du berger (\pageref{subsec:ab_shepherds_fury})
\item Rang 4:
Potentiel amélioré plus important (\pageref{subsec:ab_greater_enhanced_potential})
\item Rang 5:
Evasion (\pageref{subsec:ab_escape})
\item Rang 6:
Choisissez en une: Compétence en Attaque Supérieure (\pageref{subsec:ab_greater_skill_with_attacks}) ou Mur de protection (\pageref{subsec:ab_protective_wall})
\end{abnamelist}
\textbf{Intrusion de la Meneuse:}
%_____________________________________________________%
\phantomsection\label{sec:focusshepherdsspirits}\section*{ Guide les Esprits}

\textcolor{gray}{\emph{ Shepherds Spirits}}

Les âmes errantes, les esprits de la nature et les êtres élémentaires vous aident et vous soutiennent. (Dans certains contextes, le Focus Guide les Esprits s’applique à un seul type d’esprit, comme les esprits des défunts, les esprits de la nature, etc.)

\begin{abnamelist}
\item Rang 1: Interrogez les esprits (\pageref{subsec:ab_question_the_spirits})
\item Rang 2:
Esprit Complice (\pageref{subsec:ab_spirit_accomplice})
\item Rang 3:
Choisissez en une: Commander un Esprit (\pageref{subsec:ab_command_spirit}) ou Sens surnaturels (\pageref{subsec:ab_preternatural_senses})
\item Rang 4:
Esprit Protecteur (\pageref{subsec:ab_wraith_cloak})
\item Rang 5:
Appeler l'esprit d'un mort (\pageref{subsec:ab_call_dead_spirit})
\item Rang 6:
Choisissez en une: Absorber l'Esprit (\pageref{subsec:ab_infuse_spirit}) ou Appeler un esprit d'un autre monde (\pageref{subsec:ab_call_otherworldly_spirit})
\end{abnamelist}
\textbf{Intrusion de la Meneuse:}
%_____________________________________________________%
\phantomsection\label{sec:focushowlsatthemoon}\section*{ Hurle à la Lune}

\textcolor{gray}{\emph{ Howls at the Moon}}

Pendant de brèves périodes, vous devenez une créature redoutable et puissante avec des problèmes de contrôle.

\begin{abnamelist}
\item Rang 1: Forme animale (\pageref{subsec:ab_animal_shape})
\item Rang 2:
Contrôle du Changement de Forme (\pageref{subsec:ab_controlled_change})
\item Rang 3:
Choisissez en une: Forme de bête Supérieure (\pageref{subsec:ab_greater_beast_form}) ou Forme de bête plus grande (\pageref{subsec:ab_bigger_beast_form})
\item Rang 4:
Changement contrôlé Supérieur (\pageref{subsec:ab_greater_controlled_change})
\item Rang 5:
Forme de bête améliorée (\pageref{subsec:ab_enhanced_beast_form})
\item Rang 6:
Choisissez en une: Contrôle parfait (\pageref{subsec:ab_perfect_control}) ou Dégâts mortels (\pageref{subsec:ab_lethal_damage})
\end{abnamelist}
\textbf{Intrusion de la Meneuse:}
%_____________________________________________________%
\phantomsection\label{sec:focusignoresphysicaldistance}\section*{ Ignore les Distances Physiques}

\textcolor{gray}{\emph{ Ignores Physical Distance}}

Vous pouvez vous téléporter d'un endroit à un autre en traversant brièvement une dimension parallèle.

\begin{abnamelist}
\item Rang 1: Compression dimensionnelle (\pageref{subsec:ab_dimensional_squeeze})
\item Rang 2:
Opportuniste (\pageref{subsec:ab_opportunist})
\item Rang 3:
Choisissez en une: Clignotement défensif (\pageref{subsec:ab_defensive_blinking}) ou Sauts de Téléportation (\pageref{subsec:ab_teleportation_burst})
\item Rang 4:
Téléportation courte (\pageref{subsec:ab_short_teleportation})
\item Rang 5:
Téléportation moyenne (\pageref{subsec:ab_medium_teleportation})
\item Rang 6:
Choisissez en une: Blessure de téléportation (\pageref{subsec:ab_teleportive_wound}) ou Téléportation (\pageref{subsec:ab_teleportation})
\end{abnamelist}
\textbf{Intrusion de la Meneuse:}
%_____________________________________________________%
\phantomsection\label{sec:focusblazeswithradiance}\section*{ Illumine avec Eclat}

\textcolor{gray}{\emph{ Blazes With Radiance}}

Vous pouvez créer de la lumière, la sculpter, la détourner de vous ou la rassembler pour l'utiliser comme une arme.

\begin{abnamelist}
\item Rang 1: Eclairé (\pageref{subsec:ab_enlightened}) Toucher lumineux (\pageref{subsec:ab_illuminating_touch})
\item Rang 2:
Couleurs Eblouissantes (\pageref{subsec:ab_dazzling_sunburst})
\item Rang 3:
Choisissez en une: Compétence en Défense Supérieure (\pageref{subsec:ab_greater_skill_with_defense}) ou Lumière brûlante (\pageref{subsec:ab_burning_light})
\item Rang 4:
Lumière du soleil (\pageref{subsec:ab_sunlight})
\item Rang 5:
Disparaître (\pageref{subsec:ab_vanish})
\item Rang 6:
Choisissez en une: Champ défensif (\pageref{subsec:ab_defensive_field}) ou Lumière vivante (\pageref{subsec:ab_living_light})
\end{abnamelist}
\textbf{Intrusion de la Meneuse:}
%_____________________________________________________%
\phantomsection\label{sec:focusinterpretsthelaw}\section*{ Interprète la Loi}

\textcolor{gray}{\emph{ Interprets the Law}}

Vous excellez à convaincre les autres de partager vos opinions.

\begin{abnamelist}
\item Rang 1: Connaissance de la loi (\pageref{subsec:ab_knowledge_of_the_law}) Persuasion et Tromperie (\pageref{subsec:ab_opening_statement})
\item Rang 2:
Débat (\pageref{subsec:ab_debate})
\item Rang 3:
Choisissez en une: Assistance compétente (\pageref{subsec:ab_able_assistance}) ou Avantage d'Intellect Amélioré (\pageref{subsec:ab_enhanced_intellect_edge})
\item Rang 4:
Fustiger (\pageref{subsec:ab_castigate})
\item Rang 5:
Personne ne sait mieux (\pageref{subsec:ab_no_one_knows_better})
\item Rang 6:
Choisissez en une: Potentiel amélioré plus important (\pageref{subsec:ab_greater_enhanced_potential}) ou Stagiaire juridique (\pageref{subsec:ab_legal_intern})
\end{abnamelist}
\textbf{Intrusion de la Meneuse:}
%_____________________________________________________%
\phantomsection\label{sec:focustouchesthesky}\section*{ Invoque la Tempète}

\textcolor{gray}{\emph{ Touches The Sky}}

Vous pouvez invoquer des tempêtes ou les briser.

\begin{abnamelist}
\item Rang 1: Survol (\pageref{subsec:ab_hover})
\item Rang 2:
Armure de vent (\pageref{subsec:ab_wind_armor})
\item Rang 3:
Choisissez en une: Eclairs de Puissance (\pageref{subsec:ab_bolts_of_power}) ou Graine de Tempête (\pageref{subsec:ab_storm_seed})
\item Rang 4:
Surfeur des Vents (\pageref{subsec:ab_windrider})
\item Rang 5:
Explosion de froid (\pageref{subsec:ab_cold_burst})
\item Rang 6:
Choisissez en une: Chariot à vent (\pageref{subsec:ab_wind_chariot}) ou Contrôle de la météo (\pageref{subsec:ab_control_weather})
\end{abnamelist}
\textbf{Intrusion de la Meneuse:}
%_____________________________________________________%
\phantomsection\label{sec:focusthrowswithdeadlyaccuracy}\section*{ Lance avec une Précision Mortelle}

\textcolor{gray}{\emph{ Throws With Deadly Accuracy}}

Tout ce qui quitte votre main va exactement là où vous souhaitez qu'il aille et à la portée et à la vitesse nécessaires pour produire l'impact parfait.

\begin{abnamelist}
\item Rang 1: Précision (\pageref{subsec:ab_precision})
\item Rang 2:
Visée prudente (\pageref{subsec:ab_careful_aim})
\item Rang 3:
Choisissez en une: Compétence en Défense Supérieure (\pageref{subsec:ab_greater_skill_with_defense}) ou Lancer rapide (\pageref{subsec:ab_quick_throw})
\item Rang 4:
Lanceur spécialisé (\pageref{subsec:ab_specialized_throwing}) Tout est une arme (\pageref{subsec:ab_everything_is_a_weapon})
\item Rang 5:
Tourbillon de lancers (\pageref{subsec:ab_whirlwind_of_throws})
\item Rang 6:
Choisissez en une: Dégâts mortels (\pageref{subsec:ab_lethal_damage}) ou Maîtrise de la défense (\pageref{subsec:ab_mastery_with_defense})
\end{abnamelist}
\textbf{Intrusion de la Meneuse:}
%_____________________________________________________%
\phantomsection\label{sec:focusdanceswithdarkmatter}\section*{ Manipule la Matière Noire}

\textcolor{gray}{\emph{ Dances With Dark Matter}}

Vous pouvez manipuler l'ombre et la matière « noire ».

\begin{abnamelist}
\item Rang 1: Rubans de matière noire (\pageref{subsec:ab_ribbons_of_dark_matter})
\item Rang 2:
Ailes du Vide (\pageref{subsec:ab_void_wings})
\item Rang 3:
Choisissez en une: Frappe de matière noire (\pageref{subsec:ab_dark_matter_strike}) ou Manteau de matière noire (\pageref{subsec:ab_dark_matter_shroud})
\item Rang 4:
Coquille de matière noire (\pageref{subsec:ab_dark_matter_shell})
\item Rang 5:
Surfeur de Matière Noire (\pageref{subsec:ab_windwracked_traveler})
\item Rang 6:
Choisissez en une: Embrassez la nuit (\pageref{subsec:ab_embrace_the_night}) ou Structure de matière noire (\pageref{subsec:ab_dark_matter_structure})
\end{abnamelist}
\textbf{Intrusion de la Meneuse:}
%_____________________________________________________%
\phantomsection\label{sec:focuswalksthewildwoods}\section*{ Marche dans Les Forêts Primaires}

\textcolor{gray}{\emph{ Walks The Wild Woods}}

Un adepte de la magie de la nature qui s'appuie sur le pouvoir et la force des arbres.

\begin{abnamelist}
\item Rang 1: Récupération du patient (\pageref{subsec:ab_patient_recovery}) Vie en pleine nature (\pageref{subsec:ab_wilderness_life})
\item Rang 2:
Corps en bois (\pageref{subsec:ab_wooden_body})
\item Rang 3:
Choisissez en une: Compagnon Arbre (\pageref{subsec:ab_tree_companion}) ou Sensibilisation à la nature sauvage (\pageref{subsec:ab_wilderness_awareness})
\item Rang 4:
Voyage dans les arbres (\pageref{subsec:ab_tree_travel})
\item Rang 5:
Grand arbre (\pageref{subsec:ab_great_tree})
\item Rang 6:
Choisissez en une: Floraison réparatrice (\pageref{subsec:ab_restorative_bloom}) ou Forêt Terrifiante (\pageref{subsec:ab_dreadwood})
\end{abnamelist}
\textbf{Intrusion de la Meneuse:}
%_____________________________________________________%
\phantomsection\label{sec:focusmastersweaponry}\section*{ Maîtrise l'Armement}

\textcolor{gray}{\emph{ Masters Weaponry}}

Vous êtes un maître d'arme d'un type particulier d'arme, qu'il s'agisse d'une épée, d'un fouet, d'un poignard, d'un pistolet ou autre. (Quelqu’un qui Maîtrise l'Armement peut disposer d’un équipement supplémentaire, notamment une arme de haute qualité.)

\begin{abnamelist}
\item Rang 1: Fabricant d'armes (\pageref{subsec:ab_weapon_crafter}) Maître d'Arme (\pageref{subsec:ab_weapon_master})
\item Rang 2:
Défense avec Arme (\pageref{subsec:ab_weapon_defense})
\item Rang 3:
Choisissez en une: Attaque rapide (\pageref{subsec:ab_rapid_attack}) ou Frappe désarmante (\pageref{subsec:ab_disarming_strike})
\item Rang 4:
Ne jamais échouer (\pageref{subsec:ab_never_fumble})
\item Rang 5:
Maîtrise extrême (\pageref{subsec:ab_extreme_mastery})
\item Rang 6:
Choisissez en une: Frappe mortelle (\pageref{subsec:ab_deadly_strike}) ou Meurtrier (\pageref{subsec:ab_murderer})
\end{abnamelist}
\textbf{Intrusion de la Meneuse:}
%_____________________________________________________%
\phantomsection\label{sec:focusmasterstheswarm}\section*{ Maîtrise l'Essaim}

\textcolor{gray}{\emph{ Masters the Swarm}}

Insectes. Rats. Chauves-souris. Même les oiseaux. Vous maîtrisez un type de petite créature qui vous obéit.

\begin{abnamelist}
\item Rang 1: Influence d'Essaim (\pageref{subsec:ab_influence_swarm})
\item Rang 2:
Contrôle de l'Essaim (\pageref{subsec:ab_control_swarm})
\item Rang 3:
Choisissez en une: Armure vivante (\pageref{subsec:ab_living_armor}) ou Compétence avec les attaques (\pageref{subsec:ab_skill_with_attacks})
\item Rang 4:
Appeller un essaim (\pageref{subsec:ab_call_swarm})
\item Rang 5:
Gagner un compagnon inhabituel (\pageref{subsec:ab_gain_unusual_companion})
\item Rang 6:
Choisissez en une: Compétence en Défense Supérieure (\pageref{subsec:ab_greater_skill_with_defense}) ou Essaim mortel (\pageref{subsec:ab_deadly_swarm})
\end{abnamelist}
\textbf{Intrusion de la Meneuse:}
%_____________________________________________________%
\phantomsection\label{sec:focusmastersdefense}\section*{ Maîtrise la Défense}

\textcolor{gray}{\emph{ Masters Defense}}

Vous utilisez un équipement de protection et des techniques pratiquées pour éviter de vous blesser lors d'un combat.

\begin{abnamelist}
\item Rang 1: Maîtrise du Bouclier (\pageref{subsec:ab_shield_master})
\item Rang 2:
Pratique des armures (\pageref{subsec:ab_practiced_in_armor}) Robuste (\pageref{subsec:ab_sturdy})
\item Rang 3:
Choisissez en une: Esquive et résistance (\pageref{subsec:ab_dodge_and_resist}) ou Esquiver et répondre (\pageref{subsec:ab_dodge_and_respond})
\item Rang 4:
Expérimenté en armure (\pageref{subsec:ab_experienced_in_armor}) Tour de Volonté (\pageref{subsec:ab_tower_of_will})
\item Rang 5:
Rien que Défendre (\pageref{subsec:ab_nothing_but_defend})
\item Rang 6:
Choisissez en une: Maître de la défense (\pageref{subsec:ab_defense_master}) ou Portez-la bien (\pageref{subsec:ab_wear_it_well})
\end{abnamelist}
\textbf{Intrusion de la Meneuse:}
%_____________________________________________________%
\phantomsection\label{sec:focusmastersspells}\section*{ Maîtrise les Sortilèges}

\textcolor{gray}{\emph{ Masters Spells}}

En vous spécialisant dans le lancement de sortilèges et en tenant un livre de sorts, vous pouvez rapidement lancer des sorts d'arc de foudre, de feu roulant, d'ombre rampante et d'invocation.

\begin{abnamelist}
\item Rang 1: Tir Arcanique (\pageref{subsec:ab_arcane_flare})
\item Rang 2:
Rayon de confusion (\pageref{subsec:ab_ray_of_confusion})
\item Rang 3:
Choisissez en une: Cube de flammes (\pageref{subsec:ab_fire_bloom}) ou Invoquer une araignée géante (\pageref{subsec:ab_summon_giant_spider})
\item Rang 4:
Interrogation de l'âme (\pageref{subsec:ab_soul_interrogation})
\item Rang 5:
Mur de Granit (\pageref{subsec:ab_granite_wall})
\item Rang 6:
Choisissez en une: Invoquer un démon (\pageref{subsec:ab_summon_demon}) ou Mot de mort (\pageref{subsec:ab_word_of_death})
\end{abnamelist}
\textbf{Intrusion de la Meneuse:}
%_____________________________________________________%
\phantomsection\label{sec:focusneedsnoweapon}\section*{ N'a pas Besoin d'Arme}

\textcolor{gray}{\emph{ Needs No Weapon}}

Des coups de poing, de pied, de coude, des genoux et des mouvements complets du corps sont toutes les armes dont vous avez besoin.

\begin{abnamelist}
\item Rang 1: Chair de Pierre (\pageref{subsec:ab_flesh_of_stone}) Poings de fureur (\pageref{subsec:ab_fists_of_fury})
\item Rang 2:
Avantage par Désavantage (\pageref{subsec:ab_advantage_to_disadvantage}) Style de combat à mains nues (\pageref{subsec:ab_unarmed_fighting_style})
\item Rang 3:
Choisissez en une: Potentiel amélioré plus important (\pageref{subsec:ab_greater_enhanced_potential}) ou Se déplacer comme l'eau (\pageref{subsec:ab_moving_like_water})
\item Rang 4:
Détourner les attaques (\pageref{subsec:ab_divert_attacks})
\item Rang 5:
Attaque Etourdissante (\pageref{subsec:ab_stun_attack})
\item Rang 6:
Choisissez en une: Dégâts mortels (\pageref{subsec:ab_lethal_damage}) ou Maître du style de combat à mains nues (\pageref{subsec:ab_master_of_unarmed_fighting_style})
\end{abnamelist}
\textbf{Intrusion de la Meneuse:}
%_____________________________________________________%
\phantomsection\label{sec:focusdoesn'tdomuch}\section*{ Ne Fait pas Grand Chose}

\textcolor{gray}{\emph{ Doesn't Do Much}}

Vous êtes un fainéant, mais vous en savez un peu sur beaucoup de choses.

\begin{abnamelist}
\item Rang 1: Leçons de vie (\pageref{subsec:ab_life_lessons})
\item Rang 2:
Totalement Chill (\pageref{subsec:ab_totally_chill})
\item Rang 3:
Choisissez en une: Compétence avec les attaques (\pageref{subsec:ab_skill_with_attacks}) ou Improviser (\pageref{subsec:ab_improvise})
\item Rang 4:
Compétence en Défense Supérieure (\pageref{subsec:ab_greater_skill_with_defense}) Leçons de vie (\pageref{subsec:ab_life_lessons})
\item Rang 5:
Potentiel amélioré plus important (\pageref{subsec:ab_greater_enhanced_potential})
\item Rang 6:
Choisissez en une: S'appuyer sur les expériences de la vie (\pageref{subsec:ab_drawing_on_lifes_experiences}) ou Vif d'esprit (\pageref{subsec:ab_quick_wits})
\end{abnamelist}
\textbf{Intrusion de la Meneuse:}
%_____________________________________________________%
\phantomsection\label{sec:focusneversaysdie}\section*{ Ne S'Avoue Jamais Vaincu}

\textcolor{gray}{\emph{ Never Says Die}}

Vous n’abandonnez jamais, vous pouvez ignorer les coups et revenir toujours pour vous battre.

\begin{abnamelist}
\item Rang 1: Récupération améliorée (\pageref{subsec:ab_improved_recovery}) Passer à travers (\pageref{subsec:ab_push_on_through})
\item Rang 2:
Ignorez la Douleur (\pageref{subsec:ab_ignore_the_pain})
\item Rang 3:
Choisissez en une: Fièvre sanguinolente (\pageref{subsec:ab_blood_fever}) ou Réserves cachées (\pageref{subsec:ab_hidden_reserves})
\item Rang 4:
Choisissez en une: Détermination croissante (\pageref{subsec:ab_increasing_determination}) ou Survivre à l'ennemi (\pageref{subsec:ab_outlast_the_foe})
\item Rang 5:
Pas encore mort (\pageref{subsec:ab_not_dead_yet})
\item Rang 6:
Choisissez en une: Défi final (\pageref{subsec:ab_final_defiance}) ou Ignorer l'Affliction (\pageref{subsec:ab_ignore_affliction})
\end{abnamelist}
\textbf{Intrusion de la Meneuse:}
%_____________________________________________________%
\phantomsection\label{sec:focusoperatesundercover}\section*{ Opère sous Couverture}

\textcolor{gray}{\emph{ Operates Undercover}}

Sous l’apparence de quelqu’un d’autre, vous cherchez à trouver des réponses que les puissants ne veulent pas divulguer. (Quelqu'un qui Opère sous Couverture pourrait avoir un équipement supplémentaire comprenant un kit de déguisement.)

\begin{abnamelist}
\item Rang 1: Enquêter (\pageref{subsec:ab_investigate})
\item Rang 2:
Déguisement (\pageref{subsec:ab_disguise})
\item Rang 3:
Choisissez en une: Agent Provocateur (\pageref{subsec:ab_agent_provocateur}) ou Courir et combattre (\pageref{subsec:ab_run_and_fight})
\item Rang 4:
Rapide Tromperie (\pageref{subsec:ab_pull_a_fast_one})
\item Rang 5:
Utiliser ce qui est disponible (\pageref{subsec:ab_using_whats_available})
\item Rang 6:
Choisissez en une: Faites confiance à la chance (\pageref{subsec:ab_trust_to_luck}) ou Frappe mortelle (\pageref{subsec:ab_deadly_strike})
\end{abnamelist}
\textbf{Intrusion de la Meneuse:}
%_____________________________________________________%
\phantomsection\label{sec:focusspeaksfortheland}\section*{ Parle au Nom de la Terre}

\textcolor{gray}{\emph{ Speaks for the Land}}

Votre connexion spirituelle avec la nature et l’environnement vous confère des capacités mystiques.

\begin{abnamelist}
\item Rang 1: Connaissances en milieu sauvage (\pageref{subsec:ab_wilderness_lore}) Graines de fureur (\pageref{subsec:ab_seeds_of_fury})
\item Rang 2:
Feuillage agrippant (\pageref{subsec:ab_grasping_foliage})
\item Rang 3:
Choisissez en une: Apaiser la Bête Sauvage (\pageref{subsec:ab_soothe_the_savage}) ou Communication (\pageref{subsec:ab_communication})
\item Rang 4:
Carnivore sous la Lune (\pageref{subsec:ab_moon_shape})
\item Rang 5:
Eruption d'insectes (\pageref{subsec:ab_insect_eruption})
\item Rang 6:
Choisissez en une: Appeler la tempête (\pageref{subsec:ab_call_the_storm}) ou Tremblement de terre (\pageref{subsec:ab_earthquake})
\end{abnamelist}
\textbf{Intrusion de la Meneuse:}
%_____________________________________________________%
\phantomsection\label{sec:focustalkstomachines}\section*{ Parle aux Machines}

\textcolor{gray}{\emph{ Talks to Machines}}

Vous utilisez votre cerveau organique comme un ordinateur, en interface « sans fil » avec n’importe quel appareil électronique. Vous pouvez les contrôler et les influencer d’une manière que d’autres ne peuvent pas.

\begin{abnamelist}
\item Rang 1: Affinité machine (\pageref{subsec:ab_machine_affinity}) Interface distante (\pageref{subsec:ab_distant_interface})
\item Rang 2:
Charmer une Machine (\pageref{subsec:ab_charm_machine}) Puissance d'attraction (\pageref{subsec:ab_coaxing_power})
\item Rang 3:
Choisissez en une: Commander une Machine (\pageref{subsec:ab_command_machine}) ou Interface intelligente (\pageref{subsec:ab_intelligent_interface})
\item Rang 4:
Combattant de Robot (\pageref{subsec:ab_robot_fighter}) Compagnon machine (\pageref{subsec:ab_machine_companion})
\item Rang 5:
Collecte d'informations (\pageref{subsec:ab_information_gathering})
\item Rang 6:
Choisissez en une: Compagnon Machine Amélioré (\pageref{subsec:ab_improved_machine_companion}) ou Contrôle de Machine (\pageref{subsec:ab_control_machine})
\end{abnamelist}
\textbf{Intrusion de la Meneuse:}
%_____________________________________________________%
\phantomsection\label{sec:focusseparatesmindfrombody}\section*{ Peut Séparer son Esprit de son Corps}

\textcolor{gray}{\emph{ Separates Mind From Body}}

Vous pouvez projeter votre esprit hors de votre corps pour voir des endroits lointains et découvrir des secrets qui autrement resteraient cachés.

\begin{abnamelist}
\item Rang 1: Troisième œil (\pageref{subsec:ab_third_eye})
\item Rang 2:
Esprit ouvert (\pageref{subsec:ab_open_mind}) Sens aiguisés (\pageref{subsec:ab_sharp_senses})
\item Rang 3:
Choisissez en une: Troisième œil itinérant (\pageref{subsec:ab_roaming_third_eye}) ou Trouver ce qui est caché (\pageref{subsec:ab_find_the_hidden})
\item Rang 4:
Capteur (\pageref{subsec:ab_sensor})
\item Rang 5:
Passager psychique (\pageref{subsec:ab_psychic_passenger})
\item Rang 6:
Choisissez en une: Capteur amélioré (\pageref{subsec:ab_improved_sensor}) ou Projection mentale (\pageref{subsec:ab_mental_projection})
\end{abnamelist}
\textbf{Intrusion de la Meneuse:}
%_____________________________________________________%
\phantomsection\label{sec:focuspilotsstarcraft}\section*{ Pilote un Vaisseau Spatial}

\textcolor{gray}{\emph{ Pilots Starcraft}}

Vous êtes un excellent pilote de vaisseau.

\begin{abnamelist}
\item Rang 1: Connaissances en Prêt (\pageref{subsec:ab_flex_lore}) Pilote (\pageref{subsec:ab_pilot})
\item Rang 2:
Mentalement résistant (\pageref{subsec:ab_mentally_tough}) Récupération et confort (\pageref{subsec:ab_salvage_and_comfort})
\item Rang 3:
Choisissez en une: Pilote Expert (\pageref{subsec:ab_expert_pilot}) A l'Aise à Bord (\pageref{subsec:ab_ship_footing}) ou Compagnon machine (\pageref{subsec:ab_machine_companion})
\item Rang 4:
Célérité améliorée (\pageref{subsec:ab_enhanced_speed}) Réseau de capteurs (\pageref{subsec:ab_sensor_array})
\item Rang 5:
Comme le dos de votre main (\pageref{subsec:ab_like_the_back_of_your_hand})
\item Rang 6:
Choisissez en une: Pilote Incomparable (\pageref{subsec:ab_incomparable_pilot}) Compétence avec les attaques (\pageref{subsec:ab_skill_with_attacks}) ou Télécommande (\pageref{subsec:ab_remote_control})
\end{abnamelist}
\textbf{Intrusion de la Meneuse:}
%_____________________________________________________%
\phantomsection\label{sec:focuswearsasheenofice}\section*{ Porte un Eclat de Glace}

\textcolor{gray}{\emph{ Wears a Sheen of Ice}}

Vous maîtrisez la puissance hivernale du froid et de la glace.

\begin{abnamelist}
\item Rang 1: Armure de Glace (\pageref{subsec:ab_ice_armor})
\item Rang 2:
Toucher Glacial (\pageref{subsec:ab_frost_touch})
\item Rang 3:
Choisissez en une: Création de Glace (\pageref{subsec:ab_ice_creation}) ou Toucher de Froid Paralysant (\pageref{subsec:ab_freezing_touch})
\item Rang 4:
Armure de glace résiliente (\pageref{subsec:ab_resilient_ice_armor})
\item Rang 5:
Explosion de froid (\pageref{subsec:ab_cold_burst})
\item Rang 6:
Choisissez en une: Gantelets d'hiver (\pageref{subsec:ab_winter_gauntlets}) ou Tempête de Glace (\pageref{subsec:ab_ice_storm})
\end{abnamelist}
\textbf{Intrusion de la Meneuse:}
%_____________________________________________________%
\phantomsection\label{sec:focuswieldsanenchantedweapon}\section*{ Porte une Arme Enchantée}

\textcolor{gray}{\emph{ Wields An Enchanted Weapon}}

Vous possédez une arme aux capacités étranges, et votre connaissance de ses pouvoirs vous a permis de créer avec elle un style de combat unique.

\begin{abnamelist}
\item Rang 1: Arme Electrique (\pageref{subsec:ab_charge_weapon}) Arme enchantée (\pageref{subsec:ab_enchanted_weapon}) Pouvoir inné (\pageref{subsec:ab_innate_power})
\item Rang 2:
Frappe Explosive (\pageref{subsec:ab_power_crash})
\item Rang 3:
Choisissez en une: Attaque rapide (\pageref{subsec:ab_rapid_attack}) ou Lancer une arme enchantée (\pageref{subsec:ab_throw_enchanted_weapon})
\item Rang 4:
Arme de défense (\pageref{subsec:ab_defending_weapon})
\item Rang 5:
Mouvement enchanté (\pageref{subsec:ab_enchanted_movement})
\item Rang 6:
Choisissez en une: Attaque Tournoyante (\pageref{subsec:ab_spin_attack}) ou Frappe mortelle (\pageref{subsec:ab_deadly_strike})
\end{abnamelist}
\textbf{Intrusion de la Meneuse:}
%_____________________________________________________%
\phantomsection\label{sec:focuswearspowerarmor}\section*{ Porte une Armure Mécanique}

\textcolor{gray}{\emph{ Wears Power Armor}}

Vous portez une armure fantastique.

\begin{abnamelist}
\item Rang 1: Armure motorisée (\pageref{subsec:ab_powered_armor}) Puissance Améliorée (\pageref{subsec:ab_enhanced_might})
\item Rang 2:
Affichage Tête Haute (\pageref{subsec:ab_heads_up_display})
\item Rang 3:
Choisissez en une: Armure Corporelle (\pageref{subsec:ab_fusion_armor}) ou Santé incroyable (\pageref{subsec:ab_incredible_health})
\item Rang 4:
Explosion de Force (\pageref{subsec:ab_force_blast})
\item Rang 5:
Armure Renforcée par Champs de Force (\pageref{subsec:ab_field_reinforced_armor})
\item Rang 6:

\end{abnamelist}
\textbf{Intrusion de la Meneuse:}
%_____________________________________________________%
\phantomsection\label{sec:focusconductsweirdscience}\section*{ Poursuit des Sciences Etranges}

\textcolor{gray}{\emph{ Conducts Weird Science}}

Votre perspicacité et vos capacités surnaturelles font de vous un scientifique capable de prouesses incroyables.

\begin{abnamelist}
\item Rang 1: Analyse en laboratoire (\pageref{subsec:ab_lab_analysis}) Compétences en Connaissances (\pageref{subsec:ab_knowledge_skills})
\item Rang 2:
Modifier l'appareil (\pageref{subsec:ab_modify_device})
\item Rang 3:
Choisissez en une: Mieux vivre grâce à la chimie (\pageref{subsec:ab_better_living_through_chemistry}) ou Santé incroyable (\pageref{subsec:ab_incredible_health})
\item Rang 4:
Compétences en Connaissances (\pageref{subsec:ab_knowledge_skills}) Juste un peu fou (\pageref{subsec:ab_just_a_bit_mad})
\item Rang 5:
Percée scientifique étrange (\pageref{subsec:ab_weird_science_breakthrough})
\item Rang 6:
Choisissez en une: Incroyable exploit scientifique (\pageref{subsec:ab_incredible_feat_of_science}) Champ défensif (\pageref{subsec:ab_defensive_field}) ou Inventeur (\pageref{subsec:ab_inventor})
\end{abnamelist}
\textbf{Intrusion de la Meneuse:}
%_____________________________________________________%
\phantomsection\label{sec:focustakesanimalshape}\section*{ Prend une Forme Animale}

\textcolor{gray}{\emph{ Takes Animal Shape}}

Vous pouvez vous transformer en animal.

\begin{abnamelist}
\item Rang 1: Forme animale (\pageref{subsec:ab_animal_shape})
\item Rang 2:
Apaiser la Bête Sauvage (\pageref{subsec:ab_soothe_the_savage}) Communication (\pageref{subsec:ab_communication})
\item Rang 3:
Choisissez en une: Forme animale plus grande (\pageref{subsec:ab_bigger_animal_shape}) ou Forme de bête Supérieure (\pageref{subsec:ab_greater_beast_form})
\item Rang 4:
Analyse d'animal (\pageref{subsec:ab_animal_scrying})
\item Rang 5:
Difficile à tuer (\pageref{subsec:ab_hard_to_kill})
\item Rang 6:
Choisissez en une: Prêter une forme animale (\pageref{subsec:ab_lend_animal_shape}) ou Vitesse floue (\pageref{subsec:ab_blurring_speed})
\end{abnamelist}
\textbf{Intrusion de la Meneuse:}
%_____________________________________________________%
\phantomsection\label{sec:focuswouldratherbereading}\section*{ Préfèrerait Lire}

\textcolor{gray}{\emph{ Would Rather Be Reading}}

Les livres sont vos amis. Qu'y a-t-il de plus important que la connaissance ? Rien.

\begin{abnamelist}
\item Rang 1: La connaissance, c'est le pouvoir (\pageref{subsec:ab_knowledge_is_power})
\item Rang 2:
Intellect Amélioré Supérieur (\pageref{subsec:ab_greater_enhanced_intellect})
\item Rang 3:
Choisissez en une: Appliquer vos connaissances (\pageref{subsec:ab_applying_your_knowledge}) ou Compétences en Gage (\pageref{subsec:ab_flex_skill})
\item Rang 4:
Connaître l'inconnu (\pageref{subsec:ab_knowing_the_unknown}) La connaissance, c'est le pouvoir (\pageref{subsec:ab_knowledge_is_power})
\item Rang 5:
Intellect Amélioré Supérieur (\pageref{subsec:ab_greater_enhanced_intellect})
\item Rang 6:
Choisissez en une: La connaissance, c'est le pouvoir (\pageref{subsec:ab_knowledge_is_power}) Lire les signes (\pageref{subsec:ab_read_the_signs}) ou Tour de l'Intellect (\pageref{subsec:ab_tower_of_intellect})
\end{abnamelist}
\textbf{Intrusion de la Meneuse:}
%_____________________________________________________%
\phantomsection\label{sec:focusmetesoutjustice}\section*{ Rend la Justice}

\textcolor{gray}{\emph{ Metes Out Justice}}

Vous redressez les torts, protégez les innocents et punissez les coupables.

\begin{abnamelist}
\item Rang 1: Désignation (\pageref{subsec:ab_designation}) Porter un jugement (\pageref{subsec:ab_make_judgment})
\item Rang 2:
Défendre les innocents (\pageref{subsec:ab_defend_the_innocent}) Désignation Améliorée (\pageref{subsec:ab_improved_designation})
\item Rang 3:
Choisissez en une: Défendre tous les innocents (\pageref{subsec:ab_defend_all_the_innocent}) ou Punir le coupable (\pageref{subsec:ab_punish_the_guilty})
\item Rang 4:
Désignation supérieure (\pageref{subsec:ab_greater_designation}) Trouver le coupable (\pageref{subsec:ab_find_the_guilty})
\item Rang 5:
Punir tous les coupables (\pageref{subsec:ab_punish_all_the_guilty})
\item Rang 6:
Choisissez en une: Au diable les coupables (\pageref{subsec:ab_damn_the_guilty}) ou Inspirez les innocents (\pageref{subsec:ab_inspire_the_innocent})
\end{abnamelist}
\textbf{Intrusion de la Meneuse:}
%_____________________________________________________%
\phantomsection\label{sec:focusstandslikeabastion}\section*{ Résiste Comme une Citadelle}

\textcolor{gray}{\emph{ Stands Like a Bastion}}

Votre armure, ainsi que votre taille, votre force, votre entraînement incroyable ou l'amélioration de votre machine, vous rendent difficile à déplacer ou à blesser.  Certains personnages qui Résiste Comme une Citadelle sont peut-être déjà des experts en armure. Ils peuvent choisir une capacité de Rang 1 différente au lieu de Pratique des armures. 

\begin{abnamelist}
\item Rang 1: Défenseur expérimenté (\pageref{subsec:ab_experienced_defender}) Pratique des armures (\pageref{subsec:ab_practiced_in_armor})
\item Rang 2:
Résistez aux éléments (\pageref{subsec:ab_resist_the_elements})
\item Rang 3:
Choisissez en une: Inamovible (\pageref{subsec:ab_unmovable}) Pratique de toutes les armes (\pageref{subsec:ab_practiced_with_all_weapons}) ou Puissance améliorée supérieure (\pageref{subsec:ab_greater_enhanced_might})
\item Rang 4:
Mur vivant (\pageref{subsec:ab_living_wall})
\item Rang 5:
Maîtrise en Armure (\pageref{subsec:ab_mastery_in_armor}) Robustesse (\pageref{subsec:ab_hardiness})
\item Rang 6:
Choisissez en une: Dégâts mortels (\pageref{subsec:ab_lethal_damage}) ou Entraînement au bouclier (\pageref{subsec:ab_shield_training})
\end{abnamelist}
\textbf{Intrusion de la Meneuse:}
%_____________________________________________________%
\phantomsection\label{sec:focussolvesmysteries}\section*{ Résout des Mystères}

\textcolor{gray}{\emph{ Solves Mysteries}}

Vous maîtrisez la déduction et utilisez des faits et des indices pour trouver la réponse.

\begin{abnamelist}
\item Rang 1: Détective (\pageref{subsec:ab_sleuth}) Enquêteur (\pageref{subsec:ab_investigator})
\item Rang 2:
Hors de danger (\pageref{subsec:ab_out_of_harms_way})
\item Rang 3:
Choisissez en une: Compétence avec les attaques (\pageref{subsec:ab_skill_with_attacks}) ou Vous avez étudié (\pageref{subsec:ab_you_studied})
\item Rang 4:
Tirer une conclusion (\pageref{subsec:ab_draw_conclusion})
\item Rang 5:
Désamorcer la situation (\pageref{subsec:ab_defuse_situation})
\item Rang 6:
Choisissez en une: Compétence en Défense Supérieure (\pageref{subsec:ab_greater_skill_with_defense}) ou Prendre l'initiative (\pageref{subsec:ab_seize_the_initiative})
\end{abnamelist}
\textbf{Intrusion de la Meneuse:}
%_____________________________________________________%
\phantomsection\label{sec:focusawakensdreams}\section*{ Réveille les rêves}

\textcolor{gray}{\emph{ Awakens Dreams}}

Vous pouvez extraire des images de rêves et leur donner vie dans le monde éveillé.

\begin{abnamelist}
\item Rang 1: Artisanat des Rêves (\pageref{subsec:ab_dreamcraft}) Science du Sommeil (\pageref{subsec:ab_oneirochemy})
\item Rang 2:
Voleur de rêves (\pageref{subsec:ab_dream_thief})
\item Rang 3:
Choisissez en une: Intellect amélioré (\pageref{subsec:ab_enhanced_intellect}) ou Le rêve devient réalité (\pageref{subsec:ab_dream_becomes_reality})
\item Rang 4:
Rêverie (\pageref{subsec:ab_daydream})
\item Rang 5:
Cauchemar (\pageref{subsec:ab_nightmare})
\item Rang 6:
Choisissez en une: Chambre des rêves (\pageref{subsec:ab_chamber_of_dreams}) ou Champ réactif (\pageref{subsec:ab_reactive_field})
\end{abnamelist}
\textbf{Intrusion de la Meneuse:}
%_____________________________________________________%
\phantomsection\label{sec:focusworksthebackalleys}\section*{ Rôde dans les Bas Quartiers}

\textcolor{gray}{\emph{ Works the Back Alleys}}

Vous avancez sans être vu, volant les riches pour parvenir à vos fins.

\begin{abnamelist}
\item Rang 1: Compétences furtives (\pageref{subsec:ab_stealth_skills})
\item Rang 2:
Contacts avec la pègre (\pageref{subsec:ab_underworld_contacts})
\item Rang 3:
Choisissez en une: Entraînement de guilde (\pageref{subsec:ab_guild_training}) ou Rapide Tromperie (\pageref{subsec:ab_pull_a_fast_one})
\item Rang 4:
Maître voleur (\pageref{subsec:ab_master_thief})
\item Rang 5:
Combattant Hors-la-loi (\pageref{subsec:ab_dirty_fighter})
\item Rang 6:
Choisissez en une: Mettre le paquet (\pageref{subsec:ab_all_out_con}) ou Rat des Allées (\pageref{subsec:ab_alley_rat})
\end{abnamelist}
\textbf{Intrusion de la Meneuse:}
%_____________________________________________________%
\phantomsection\label{sec:focusconsortswiththedead}\section*{ S'Associe avec les Morts}

\textcolor{gray}{\emph{ Consorts With the Dead}}

Les morts répondent à vos questions, et leurs cadavres réanimés vous servent.

\begin{abnamelist}
\item Rang 1: Orateur pour les morts (\pageref{subsec:ab_speaker_for_the_dead})
\item Rang 2:
Nécromancie (\pageref{subsec:ab_necromancy})
\item Rang 3:
Choisissez en une: Esprit parle moi d'ici (\pageref{subsec:ab_reading_the_room}) ou Réparer la chair (\pageref{subsec:ab_repair_flesh})
\item Rang 4:
Réparer la chair (\pageref{subsec:ab_repair_flesh})
\item Rang 5:
Regard terrifiant (\pageref{subsec:ab_terrifying_gaze})
\item Rang 6:
Choisissez en une: Mot de mort (\pageref{subsec:ab_word_of_death}) ou Véritable Nécromancie (\pageref{subsec:ab_true_necromancy})
\end{abnamelist}
\textbf{Intrusion de la Meneuse:}
%_____________________________________________________%
\phantomsection\label{sec:focussoarsonamazingwings}\section*{ S'Envole Grâce à ses Ailes}

\textcolor{gray}{\emph{ Soars On Amazing Wings}}

De nombreux super-héros peuvent voler et certains ont même des ailes. Vous pouvez utiliser vos ailes pour vous déplacer, attaquer et vous défendre.

\begin{abnamelist}
\item Rang 1: Survol (\pageref{subsec:ab_hover}) Vol Court (\pageref{subsec:ab_flight_exertion})
\item Rang 2:
Ailes comme Arme (\pageref{subsec:ab_wing_weapons})
\item Rang 3:
Choisissez en une: Attaque acrobatique (\pageref{subsec:ab_acrobatic_attack}) ou Compagnon volant (\pageref{subsec:ab_flying_companion})
\item Rang 4:
Difficile à toucher (\pageref{subsec:ab_hard_to_hit})
\item Rang 5:
Accélérer (\pageref{subsec:ab_up_to_speed})
\item Rang 6:
Choisissez en une: Cible difficile (\pageref{subsec:ab_hard_target}) ou Maître de la défense (\pageref{subsec:ab_defense_master})
\end{abnamelist}
\textbf{Intrusion de la Meneuse:}
%_____________________________________________________%
\phantomsection\label{sec:focusstretches}\section*{ S'Etire}

\textcolor{gray}{\emph{ Stretches}}

Votre corps est élastique et caoutchouteux, capable de s’étirer sur de grandes longueurs et de se comprimer lorsqu’il est frappé.

\begin{abnamelist}
\item Rang 1: Contorsionniste (\pageref{subsec:ab_contortionist}) Grand Pas (\pageref{subsec:ab_far_step})
\item Rang 2:
Chute en toute sécurité (\pageref{subsec:ab_safe_fall}) Poignée élastique (\pageref{subsec:ab_elastic_grip})
\item Rang 3:
Choisissez en une: Contourner la barrière (\pageref{subsec:ab_bypass_barrier}) ou Détournement (\pageref{subsec:ab_misdirect})
\item Rang 4:
Résilience (\pageref{subsec:ab_resilience})
\item Rang 5:
Libre de se déplacer (\pageref{subsec:ab_free_to_move})
\item Rang 6:
Choisissez en une: Briser les rangs (\pageref{subsec:ab_break_the_ranks}) ou Pas encore mort (\pageref{subsec:ab_not_dead_yet})
\end{abnamelist}
\textbf{Intrusion de la Meneuse:}
%_____________________________________________________%
\phantomsection\label{sec:focusrunsaway}\section*{ S'enfuit}

\textcolor{gray}{\emph{ Runs Away}}

Votre premier réflexe est de fuir le danger, et vous y êtes devenu très fort.

\begin{abnamelist}
\item Rang 1: Devenez défensif (\pageref{subsec:ab_go_defensive})
\item Rang 2:
Célérité améliorée (\pageref{subsec:ab_enhanced_speed}) Rapide à fuir (\pageref{subsec:ab_quick_to_flee})
\item Rang 3:
Choisissez en une: Célérité améliorée supérieure (\pageref{subsec:ab_greater_enhanced_speed}) ou Vitesse de course incroyable (\pageref{subsec:ab_incredible_running_speed})
\item Rang 4:
Détermination croissante (\pageref{subsec:ab_increasing_determination}) Vif d'esprit (\pageref{subsec:ab_quick_wits})
\item Rang 5:
Aller au sol (\pageref{subsec:ab_go_to_ground})
\item Rang 6:
Choisissez en une: Bouquet d'évasion (\pageref{subsec:ab_burst_of_escape}) ou Compétence en Défense Supérieure (\pageref{subsec:ab_greater_skill_with_defense})
\end{abnamelist}
\textbf{Intrusion de la Meneuse:}
%_____________________________________________________%
\phantomsection\label{sec:focussculptshardlight}\section*{ Sculpte la Lumière Solide}

\textcolor{gray}{\emph{ Sculpts Hard Light}}

Vous créez des objets physiques à partir d’une lumière solide que vous pouvez utiliser à des fins offensives et défensives.

\begin{abnamelist}
\item Rang 1: Lueur automatique (\pageref{subsec:ab_automatic_glow}) Lumière temporaire (\pageref{subsec:ab_temporary_light})
\item Rang 2:
Force enchevêtrante (\pageref{subsec:ab_entangling_force})
\item Rang 3:
Choisissez en une: Lumière plus Forte (\pageref{subsec:ab_harder_light}) ou Sculpter la lumière (\pageref{subsec:ab_sculpt_light})
\item Rang 4:
Intellect Amélioré Supérieur (\pageref{subsec:ab_greater_enhanced_intellect})
\item Rang 5:
Lumière sculptée améliorée (\pageref{subsec:ab_improved_sculpt_light})
\item Rang 6:
Choisissez en une: Champ défensif (\pageref{subsec:ab_defensive_field}) ou Vol (\pageref{subsec:ab_flight})
\end{abnamelist}
\textbf{Intrusion de la Meneuse:}
%_____________________________________________________%
\phantomsection\label{sec:focusfightsdirty}\section*{ Se Bat Sans Respecter de Règle}

\textcolor{gray}{\emph{ Fights Dirty}}

Vous ferez n'importe quoi pour gagner un combat : mordre, gratter, donner un coup de pied, tromper et pire encore.

\begin{abnamelist}
\item Rang 1: Pisteur (\pageref{subsec:ab_tracker}) Traqueur (\pageref{subsec:ab_stalker})
\item Rang 2:
Furtif (\pageref{subsec:ab_sneak}) Proie (\pageref{subsec:ab_quarry})
\item Rang 3:
Choisissez en une: Attaque surprise (\pageref{subsec:ab_surprise_attack}) ou Trahison (\pageref{subsec:ab_betrayal})
\item Rang 4:
Guerrier Capable (\pageref{subsec:ab_capable_warrior}) Jeux d'esprit (\pageref{subsec:ab_mind_games})
\item Rang 5:
Utilisation de l'environnement (\pageref{subsec:ab_using_the_environment})
\item Rang 6:
Choisissez en une: Meurtrier (\pageref{subsec:ab_murderer}) ou Torsion du couteau (\pageref{subsec:ab_twisting_the_knife})
\end{abnamelist}
\textbf{Intrusion de la Meneuse:}
%_____________________________________________________%
\phantomsection\label{sec:focuswieldstwoweaponsatonce}\section*{ Se Bat avec Deux Armes à la fois}

\textcolor{gray}{\emph{ Wields Two Weapons at Once}}

Vous portez de l'acier dans chaque main, prêt à affronter n'importe quel ennemi.

\begin{abnamelist}
\item Rang 1: Deux Armes Légères (\pageref{subsec:ab_dual_light_wield})
\item Rang 2:
Double frappe (\pageref{subsec:ab_double_strike}) Infiltrateur (\pageref{subsec:ab_infiltrator})
\item Rang 3:
Choisissez en une: Coupe Précise (\pageref{subsec:ab_precise_cut}) ou Deux Armes Moyennes (\pageref{subsec:ab_dual_medium_wield})
\item Rang 4:
Double défense (\pageref{subsec:ab_dual_defense})
\item Rang 5:
Double Distraction (\pageref{subsec:ab_dual_distraction})
\item Rang 6:
Choisissez en une: Attaque Tournoyante (\pageref{subsec:ab_spin_attack}) ou Attaque de désarmement (\pageref{subsec:ab_disarming_attack})
\end{abnamelist}
\textbf{Intrusion de la Meneuse:}
%_____________________________________________________%
\phantomsection\label{sec:focusinfiltrates}\section*{ Se Cache dans les Ombres}

\textcolor{gray}{\emph{ Infiltrates}}

La subtilité, la ruse et la furtivité vous permettent d'accéder là où les autres ne peuvent pas aller.

\begin{abnamelist}
\item Rang 1: Compétences furtives (\pageref{subsec:ab_stealth_skills}) Sentir les Attitudes (\pageref{subsec:ab_sense_attitudes})
\item Rang 2:
Evitement (\pageref{subsec:ab_flight_not_fight}) Usurper l'identité (\pageref{subsec:ab_impersonate})
\item Rang 3:
Choisissez en une: Compétence avec les attaques (\pageref{subsec:ab_skill_with_attacks}) ou Conscience (\pageref{subsec:ab_awareness})
\item Rang 4:
Invisibilité (\pageref{subsec:ab_invisibility})
\item Rang 5:
Esquive (\pageref{subsec:ab_evasion})
\item Rang 6:
Choisissez en une: Lavage de cerveau (\pageref{subsec:ab_brainwashing}) ou Saut de Côté (\pageref{subsec:ab_spring_away})
\end{abnamelist}
\textbf{Intrusion de la Meneuse:}
%_____________________________________________________%
\phantomsection\label{sec:focusrages}\section*{ Se Met en Rage}

\textcolor{gray}{\emph{ Rages}}

Quand vous devenez fou, tout le monde vous craint.

\begin{abnamelist}
\item Rang 1: Frénésie (\pageref{subsec:ab_frenzy})
\item Rang 2:
Habiletés motrices (\pageref{subsec:ab_movement_skills}) Puissance améliorée supérieure (\pageref{subsec:ab_greater_enhanced_might})
\item Rang 3:
Choisissez en une: Combattant sans armure (\pageref{subsec:ab_unarmored_fighter}) ou Frappe Renversante (\pageref{subsec:ab_power_strike})
\item Rang 4:
Frénésie supérieure (\pageref{subsec:ab_greater_frenzy})
\item Rang 5:
Attaquez et attaquez encore (\pageref{subsec:ab_attack_and_attack_again})
\item Rang 6:
Choisissez en une: Dégâts mortels (\pageref{subsec:ab_lethal_damage}) ou Potentiel amélioré plus important (\pageref{subsec:ab_greater_enhanced_potential})
\end{abnamelist}
\textbf{Intrusion de la Meneuse:}
%_____________________________________________________%
\phantomsection\label{sec:focusbearsahalooffire}\section*{ Se Revêt d'un Halo de Feu}

\textcolor{gray}{\emph{ Bears a Halo of Fire}}

Vous pouvez envelopper votre corps de flammes, ce qui vous protège et nuit à vos ennemis.

\begin{abnamelist}
\item Rang 1: Manteau de flammes (\pageref{subsec:ab_shroud_of_flame})
\item Rang 2:
Lancement de flammes (\pageref{subsec:ab_hurl_flame})
\item Rang 3:
Choisissez en une: Ailes de Feu (\pageref{subsec:ab_wings_of_fire}) ou Main ardente du destin (\pageref{subsec:ab_fiery_hand_of_doom})
\item Rang 4:
Lame de Feu (\pageref{subsec:ab_flameblade})
\item Rang 5:
Vrilles de feu (\pageref{subsec:ab_fire_tendrils})
\item Rang 6:
Choisissez en une: Piste Infernale (\pageref{subsec:ab_inferno_trail}) ou Serviteur du Feu (\pageref{subsec:ab_fire_servant})
\end{abnamelist}
\textbf{Intrusion de la Meneuse:}
%_____________________________________________________%
\phantomsection\label{sec:focusshrinkstominutesize}\section*{ Se Réduit à une Taille Infime}

\textcolor{gray}{\emph{ Shrinks To Minute Size}}

Vous pouvez réduire à la taille d'un bug et, avec suffisamment d'expérience, encore plus petit.

\begin{abnamelist}
\item Rang 1: Rétrécir (\pageref{subsec:ab_shrink}) Sans être remarqué (\pageref{subsec:ab_beneath_notice})
\item Rang 2:
Avantages d'être petit (\pageref{subsec:ab_advantages_of_being_small}) Plus petit (\pageref{subsec:ab_smaller})
\item Rang 3:
Choisissez en une: Agrandir (\pageref{subsec:ab_enlarge}) ou Changements Rapides (\pageref{subsec:ab_quick_switch})
\item Rang 4:
Petit vol (\pageref{subsec:ab_small_flight})
\item Rang 5:
Rétrécir les autres (\pageref{subsec:ab_shrink_others})
\item Rang 6:
Choisissez en une: Minuscule (\pageref{subsec:ab_tiny}) ou Plus grand (\pageref{subsec:ab_bigger})
\end{abnamelist}
\textbf{Intrusion de la Meneuse:}
%_____________________________________________________%
\phantomsection\label{sec:focussiphonspower}\section*{ Siphonne les Pouvoirs}

\textcolor{gray}{\emph{ Siphons Power}}

Vous aspirez le pouvoir des machines et des créatures afin de vous donner du pouvoir.  Les robots et autres machines vivantes doivent être traités comme des créatures, et non comme des machines, dans le but d’en siphonner l’énergie. 

\begin{abnamelist}
\item Rang 1: Drain de Machine (\pageref{subsec:ab_drain_machine})
\item Rang 2:
Drain de Créature (\pageref{subsec:ab_drain_creature})
\item Rang 3:
Choisissez en une: Drain à distance (\pageref{subsec:ab_drain_at_a_distance}) ou Draîner la vie (\pageref{subsec:ab_unraveling_consumption})
\item Rang 4:
Stocker l'énergie (\pageref{subsec:ab_store_energy})
\item Rang 5:
Partagez le pouvoir (\pageref{subsec:ab_share_the_power})
\item Rang 6:
Choisissez en une: Libération explosive (\pageref{subsec:ab_explosive_release}) ou Siphon solaire (\pageref{subsec:ab_sun_siphon})
\end{abnamelist}
\textbf{Intrusion de la Meneuse:}
%_____________________________________________________%
\phantomsection\label{sec:focusslaysmonsters}\section*{ Tue les Monstres}

\textcolor{gray}{\emph{ Slays Monsters}}

Vous tuez des monstres.  Bien que manier une épée dans un environnement où les gens ne portent généralement pas de telles armes soit acceptable, vous pouvez modifier les capacités liées à l'épée de Tue les Monstres pour utiliser une arme différente, comme un pistolet à balles d'argent. 

\begin{abnamelist}
\item Rang 1: Connaissance des monstres (\pageref{subsec:ab_monster_lore}) Fléau des Monstres (\pageref{subsec:ab_monster_bane}) Pratique des épées (\pageref{subsec:ab_practiced_with_swords})
\item Rang 2:
Volonté de Légende (\pageref{subsec:ab_will_of_legend})
\item Rang 3:
Choisissez en une: Epéiste entraîné (\pageref{subsec:ab_trained_slayer}) Détournement (\pageref{subsec:ab_misdirect}) ou Fléau des Monstres Amélioré (\pageref{subsec:ab_improved_monster_bane})
\item Rang 4:
Continuez le combat (\pageref{subsec:ab_fight_on})
\item Rang 5:
Compétence en Attaque Supérieure (\pageref{subsec:ab_greater_skill_with_attacks})
\item Rang 6:
Choisissez en une: Fléau des Monstres Géants (\pageref{subsec:ab_heroic_monster_bane}) ou Meurtrier (\pageref{subsec:ab_murderer})
\end{abnamelist}
\textbf{Intrusion de la Meneuse:}
%_____________________________________________________%
\phantomsection\label{sec:focususeswildmagic}\section*{ Utilise la Magie Sauvage}

\textcolor{gray}{\emph{ Uses Wild Magic}}

Lanceur de sorts qui apprend une variété de sorts au lieu de se concentrer sur un seul type de magie.

\begin{abnamelist}
\item Rang 1: Lancement de Cypher (\pageref{subsec:ab_cypher_casting}) Répertoire magique (\pageref{subsec:ab_magical_repertoire})
\item Rang 2:
Répertoire étendu (\pageref{subsec:ab_expanded_repertoire})
\item Rang 3:
Choisissez en une: Magie Sauvage plus rapide (\pageref{subsec:ab_faster_wild_magic}) ou Sursaut de Cypher (\pageref{subsec:ab_cypher_surge})
\item Rang 4:
Répertoire étendu (\pageref{subsec:ab_expanded_repertoire})
\item Rang 5:
Entraînement magique (\pageref{subsec:ab_magical_training})
\item Rang 6:
Choisissez en une: Instinct de Magie Sauvage (\pageref{subsec:ab_wild_insight}) ou Maximiser le Cypher (\pageref{subsec:ab_maximize_cypher})
\end{abnamelist}
\textbf{Intrusion de la Meneuse:}
%_____________________________________________________%
\phantomsection\label{sec:focususeswildmagic}\section*{ Utilise la Magie Sauvage}

\textcolor{gray}{\emph{ Uses Wild Magic}}

Lanceur de sorts qui apprend une variété de sorts au lieu de se concentrer sur un seul type de magie.

\begin{abnamelist}
\item Rang 1: Lancement de Cypher (\pageref{subsec:ab_cypher_casting}) Répertoire magique (\pageref{subsec:ab_magical_repertoire})
\item Rang 2:
Répertoire étendu (\pageref{subsec:ab_expanded_repertoire})
\item Rang 3:
Choisissez en une: Magie Sauvage plus rapide (\pageref{subsec:ab_faster_wild_magic}) ou Sursaut de Cypher (\pageref{subsec:ab_cypher_surge})
\item Rang 4:
Répertoire étendu (\pageref{subsec:ab_expanded_repertoire})
\item Rang 5:
Entraînement magique (\pageref{subsec:ab_magical_training})
\item Rang 6:
Choisissez en une: Instinct de Magie Sauvage (\pageref{subsec:ab_wild_insight}) ou Maximiser le Cypher (\pageref{subsec:ab_maximize_cypher})
\end{abnamelist}
\textbf{Intrusion de la Meneuse:}
%_____________________________________________________%
\phantomsection\label{sec:focusmoveslikethewind}\section*{ Va Comme le Vent}

\textcolor{gray}{\emph{ Moves Like the Wind}}

Vous pouvez vous déplacer si vite que vous devenez flou.

\begin{abnamelist}
\item Rang 1: Célérité améliorée supérieure (\pageref{subsec:ab_greater_enhanced_speed}) Pied Léger (\pageref{subsec:ab_fleet_of_foot})
\item Rang 2:
Difficile à toucher (\pageref{subsec:ab_hard_to_hit})
\item Rang 3:
Choisissez en une: Célérité améliorée supérieure (\pageref{subsec:ab_greater_enhanced_speed}) ou Sursaut de Célérité (\pageref{subsec:ab_speed_burst})
\item Rang 4:
En un Clin d'oeil (\pageref{subsec:ab_blink_of_an_eye})
\item Rang 5:
Difficile à voir (\pageref{subsec:ab_hard_to_see})
\item Rang 6:
Choisissez en une: Sursaut de Célérité Parfait (\pageref{subsec:ab_perfect_speed_burst}) ou Vitesse de course incroyable (\pageref{subsec:ab_incredible_running_speed})
\end{abnamelist}
\textbf{Intrusion de la Meneuse:}
%_____________________________________________________%
\phantomsection\label{sec:focuslivesinthewilderness}\section*{ Vit dans la Nature Sauvage}

\textcolor{gray}{\emph{ Lives in the Wilderness}}

Vous pouvez survivre dans des étendues sauvages où d'autres périssent.

\begin{abnamelist}
\item Rang 1: Puissance Améliorée (\pageref{subsec:ab_enhanced_might}) Vie en pleine nature (\pageref{subsec:ab_wilderness_life})
\item Rang 2:
Explorateur de la Nature (\pageref{subsec:ab_wilderness_explorer}) Vivre de la terre (\pageref{subsec:ab_living_off_the_land})
\item Rang 3:
Choisissez en une: Encouragement de la Nature (\pageref{subsec:ab_wilderness_encouragement}) ou Sens et sensibilités animales (\pageref{subsec:ab_animal_senses_and_sensibilities})
\item Rang 4:
Sensibilisation à la nature sauvage (\pageref{subsec:ab_wilderness_awareness})
\item Rang 5:
La nature est de votre côté (\pageref{subsec:ab_the_wild_is_on_your_side})
\item Rang 6:
Choisissez en une: Camouflage sauvage (\pageref{subsec:ab_wild_camouflage}) ou Faire Corps avec la Nature (\pageref{subsec:ab_one_with_the_wild})
\end{abnamelist}
\textbf{Intrusion de la Meneuse:}
%_____________________________________________________%
\phantomsection\label{sec:focusseesbeyond}\section*{ Voit Au-Delà}

\textcolor{gray}{\emph{ Sees Beyond}}

Vous avez un sens psychique qui vous permet de voir ce que les autres ne peuvent pas voir.

\begin{abnamelist}
\item Rang 1: Voir l'invisible (\pageref{subsec:ab_see_the_unseen})
\item Rang 2:
Voir à travers la matière (\pageref{subsec:ab_see_through_matter})
\item Rang 3:
Choisissez en une: Capteur (\pageref{subsec:ab_sensor}) ou Trouver ce qui est caché (\pageref{subsec:ab_find_the_hidden})
\item Rang 4:
Visualisation à distance (\pageref{subsec:ab_remote_viewing})
\item Rang 5:
Voir à travers le temps (\pageref{subsec:ab_see_through_time})
\item Rang 6:
Choisissez en une: Conscience totale (\pageref{subsec:ab_total_awareness}) ou Projection mentale (\pageref{subsec:ab_mental_projection})
\end{abnamelist}
\textbf{Intrusion de la Meneuse:}
%_____________________________________________________%
\phantomsection\label{sec:focusfliesfasterthanabullet}\section*{ Vole Plus Vite qu'une Balle}

\textcolor{gray}{\emph{ Flies Faster Than a Bullet}}

Vous pouvez voler et vous êtes super fort, difficile à blesser et rapide aussi. Y a-t-il quelque chose que vous ne pouvez pas faire ?

\begin{abnamelist}
\item Rang 1: Survol (\pageref{subsec:ab_hover})
\item Rang 2:
Potentiel amélioré plus important (\pageref{subsec:ab_greater_enhanced_potential})
\item Rang 3:
Choisissez en une: Réserves cachées (\pageref{subsec:ab_hidden_reserves}) ou Voir à travers la matière (\pageref{subsec:ab_see_through_matter})
\item Rang 4:
Accélérer (\pageref{subsec:ab_up_to_speed}) En un Clin d'oeil (\pageref{subsec:ab_blink_of_an_eye})
\item Rang 5:
Pas encore mort (\pageref{subsec:ab_not_dead_yet})
\item Rang 6:
Choisissez en une: Ignorer l'Affliction (\pageref{subsec:ab_ignore_affliction}) ou Lumière brûlante (\pageref{subsec:ab_burning_light})
\end{abnamelist}
\textbf{Intrusion de la Meneuse:}
%_____________________________________________________%
\phantomsection\label{sec:focustravelsthroughtime}\section*{ Voyage à Travers le Temps}

\textcolor{gray}{\emph{ Travels Through Time}}

Vous pouvez voir à travers le temps, essayer de le traverser et éventuellement même le parcourir.  Bien que tous les choix de personnages soient soumis à l'approbation du MJ, Voyage à Travers le Temps est un sujet sur lequel le MJ et le joueur devraient probablement avoir une longue conversation à l'avance, afin que le joueur connaisse les règles du voyage dans le temps (le cas échéant) qui existent dans le réglage du MJ. Un personnage avec cette concentration peut modifier radicalement un décor, si les règles du voyage dans le temps le permettent. 

\begin{abnamelist}
\item Rang 1: Anticipation (\pageref{subsec:ab_anticipation})
\item Rang 2:
Voir Historique (\pageref{subsec:ab_see_history})
\item Rang 3:
Choisissez en une: Accélération temporelle (\pageref{subsec:ab_temporal_acceleration}) ou Boucle temporelle (\pageref{subsec:ab_time_loop})
\item Rang 4:
Dislocation temporelle (\pageref{subsec:ab_temporal_dislocation})
\item Rang 5:
Doppelganger Temporel (\pageref{subsec:ab_time_doppelganger})
\item Rang 6:
Choisissez en une: Appel à travers le temps (\pageref{subsec:ab_call_through_time}) ou Voyage dans le temps (\pageref{subsec:ab_time_travel})
\end{abnamelist}
\textbf{Intrusion de la Meneuse:}
%%%%%%%%%%%%%%%%%%%%%%%%%%%%%%%%%%%%%%%%%%%%%%%%%%%%%%%%%%%%%%%%%%%%%%%
\section*{Créer un nouveau Focus}
Cette section fournit tout ce dont vous avez besoin pour créer vos propres Focus.

Chaque Focus a un thème principal, qu'il s'agisse d'exploration, de manipulation d'énergie ou simplement d'infliger beaucoup de dégâts au combat. Ces grandes classifications sont appelées catégories de Focus.

Chaque catégorie de Focus a un thème principal, suivi de directives de sélection qui décrivent comment choisir les capacités pour chaque rang du chapitre Capacités, du rang 1 au rang 6.

Le Focus nouvellement créée doit être nommée sous la forme d'un verbe, comme Contrôle les Bêtes Sauvages ou Demeure dans la pierre. Par exemple, un Focus utilisant le feu créée en suivant les directives de la catégorie des Focus de manipulation d'énergie pourrait être appelé Se Revêt d' un Halo de Feu (l'un des exemples de Focus de ce chapitre). Alternativement, un nouveau Focus utilisant le feu devrait recevoir un tout nouveau nom comme Attiser les flammes de l'Apocalypse ou Allumer les feux avec une pensée.

\section*{Catégories de Focus}
\begin{itemize}
    \item Basic (\pageref{subsec:basic})
    \item Combat défensif  (\pageref{subsec:combat_defensif})
    \item Combat offensif  (\pageref{subsec:combat_defensif})
    \item Expertise des mouvements  (\pageref{subsec:expertise_des_mouvements})
    \item Exploration (\pageref{subsec:exploration})
    \item Influence  (\pageref{subsec:influence})
    \item Irrégulier  (\pageref{subsec:irregulier})
    \item Manipulation d'énergie (\pageref{subsec:manipulation_denergie})
    \item Manipulation de l'environement (\pageref{subsec:manipulation_de_lenvironement})
    \item Soutien (\pageref{subsec:soutien})
    \item Utilisation d'Alliés (\pageref{subsec:utilisation_dallies})
\end{itemize}

\section*{Choisir une Capacité en fonction de la Puissance Relative}

Les indications de sélection des capacités vous invitent à choisir une capacité parmi l'une des trois gammes : Rang bas, Rang moyen et Rang élevé. Ces plages correspondent aux «grades» de puissance donnés pour chaque capacité. Ces capacités sont ensuite classées en catégories de capacités en fonction du type de choses qu'ils font : les capacités qui améliorent les attaques physiques sont dans la catégorie des compétences d'attaque, les capacités qui aident les alliés sont dans la catégorie de soutien, et ainsi de suite. Recherchez les notes et les catégories dans la section Catégories de capacités et puissance relative du chapitre Capacités.

Les capacités de rang bas sont mieux adaptées aux options de Focus aux rangs 1 et 2. Les capacités de rang intermédiaire sont mieux adaptées aux options de Focus aux rangs 3 et 4. Les capacités de rang élevé sont mieux adaptées aux options de Focus aux rangs 5 et 6.

Cela dit, vous trouverez parfois approprié d'attribuer une capacité de bas rang au rang 3 ou 4, ou peut-être une capacité de milieu de gamme au rang 1 ou 2. Faites-le avec parcimonie, mais ne l'excluez pas. C'est peut-être le seul moyen d'obtenir toutes les capacités souhaitées pour le Focus que vous développez. Les capacités de rang supérieur coûtent généralement plus de points de réserve à utiliser. Ainsi, si une capacité de milieu de gamme est disponible au rang 1 ou 2, ou qu'une capacité de rang supérieur est disponible au rang 3 ou 4, le coût plus élevé sera un facteur d'équilibrage.

\section*{Equilibrer les Capacités}
Les indications au sein de chaque catégorie contribuent grandement à garantir que le Focus que vous développez sera équilibrée. Parfois, il peut être approprié d'accorder une capacité de faible puissance avec une capacité normale à un rang donné, en fonction des besoins du focus. Une "capacité de faible puissance" est délibérément ouverte à l'interprétation du MJ, mais d'une manière générale, elle ne devrait pas être plus puissante qu'une capacité de bas rang (c'est-à-dire une capacité qui est normalement disponible au rang 1 ou 2).

Par exemple, quelqu'un qui utilise le froid pourrait être capable de créer de petites sculptures de neige en plus d'émettre un rayon froid. Quelqu'un qui utilise de l'électricité pourrait être en mesure de recharger un artefact épuisé ou disposer d'un atout pour gérer les systèmes électriques. Et ainsi de suite.

Souvent, les indications de Focus mentionnent cela comme une possibilité. Cependant, vous disposez d'une grande latitude pour décider si un Focus nécessite une capacité supplémentaire, même si les indications de ce rang n'en indiquent pas. Si vous ajoutez une capacité, ou s'il existe une capacité de puissance plus élevée dans un rang qui ne devrait normalement pas l'avoir, cela peut signifier que le choix donné au rang suivant, ou au rang précédent, n'est pas aussi bon. Équilibrer un Focus est un peu un art. Résistez à l'envie de donner trop de puissance à le Focus, mais ne la sous-estimez pas non plus.

\section*{Les Indications de Capacité ne sont pas Prescriptives}
Chaque catégorie de Focus fournit une ligne directrice sur le type de capacité que vous devez sélectionner à chaque rang. Mais ne considérez pas les indications comme quelque chose que vous ne pouvez pas modifier. Elles ne sont pas prescriptives; elles ne sont qu'un point de départ. Vous souhaiterez peut-être varier le type de capacité d'un rang particulier qui n'est pas indiqué dans les indications. Tant que la capacité choisie se situe dans la courbe de puissance attendue pour ce Rang, tout va bien. L'indication' n'est pas censée être une camisole de force.

Par exemple, si vous construisez un Focus d'utilisation du froid pour un jeu se déroulant dans un genre fantastique, vous pouvez décider qu'une capacité qui invoque un démon est un meilleur choix à un rang particulier qu'une capacité qui inflige des dégâts dans une zone, ce que demande l'indication Rang 5 pour la manipulation de l'énergie. Faire le changement est probablement particulièrement valable si vous appelez votre nouveau focus quelque chose comme Cannalise le Neuvième Cercle.

\section*{Echange de Capacité}
Si vous créez un Focus et que vous pensez qu'elle devrait fournir une suite de capacités au premier rang qui la surchargeraient mécaniquement, vous avez la possibilité d'en ajouter une en tant que capacité « d'échange ». Pour ce faire, il suffit de permettre à un personnage d'échanger l'une de ses capacités de type contre une capacité de Focus de moindre rang. La capacité est acquise à la place d'une des capacités normalement accordées par le type du personnage.

\section*{Concept et Categorie}
Choisir de créer un Focus qui utilise un concept particulier --- par exemple, créer des illusions --- ne vous oblige pas à créer un Focus dans une catégorie particulière --- dans ce cas, la manipulation de l'environnement. Un Focus peut être construite de différentes manières en utilisant une énergie, un outil ou un concept particulier, chacun conduisant finalement à un Focus qui fournit des résultats différents. Tout dépend de vos objectifs. Dans ce cas, la création d'illusions pourrait être utilisée pour influencer les autres, ce qui plaide en faveur d'un Focus basée sur les indications de la catégorie d'influence.

De la même manière, si un Focus accorde à un personnage la possibilité d'invoquer une sorte de force ou d'énergie, cela ne signifie pas que le Focus doit automatiquement être construite en utilisant les indications de la catégorie de manipulation d'énergie (même si bien sûr cela serait possible si attaquer et vous protéger avec cette énergie est le but). Mais un Focus pourrait être construite pour accorder des capacités de création d'énergie ou de force principalement axées sur la résistance, cela suggère une orientation vers le combat défensif (quelqu'un qui peut encaisser beaucoup de dégats dans un combat); ou des capacités se concentrant sur tirer avec le souci principal de maximiser les dégâts, suggérant ainsi un Focus de catégorie combat offensif ; ou alors vous créez un suivant composé de cette énergie ou force, suggérant ainsi un Focus de catégorie Utilisation d'Alliés (c'est-à-dire quelqu'un qui utilise des créatures aidantes, des PNJ, ou même qui duplique des versions d'eux-mêmes pour vous donner un coup de pouce).

Voici un autre exemple : la Motivaton Contrôle la Gravité pourrait éventuellement être un Focus de catégorie manipulation de l'environnement ou de catégorie manipulation d'énergie. Cela dépend si l'accent est davantage mis sur l'écrasement et le maintien des objets en place (manipulation de l'environnement) ou sur le fait de faire exploser des objets et de se protéger grâce à la gravité (manipulation d'énergie).

La même souplesse du concept s'applique à d'autres domaines. Par exemple, si quelqu'un est capable d'invoquer et de modeler de la terre brute, il peut l'utiliser pour se transformer en un être de pierre (combat défensif), pour battre des ennemis (combat offensif), ou pour créer des murs, des barricades et des boucliers pour protéger leurs alliés (soutien).

Si vous recherchez une capacité et que vous n'arrivez pas à trouver celle qui vous convient dans le vaste catalogue du chapitre Capacités, envisagez d'en modifier une afin d'en créer une nouvelle (et pour accomplir ce dont vous avez besoin). Cette modification consiste à utiliser les mécanismes sous-jacents d'une capacité tels qu'ils sont écrits, mais que vous en modifiez les effets visibles d'une manière ou d'une autre. Par exemple, vous êtes peut-être en train de créer un nouvel objectif de déplacement de terre, mais vous ne parvenez pas à trouver suffisamment de capacités liées à la terre pour répondre à vos besoins. Il est assez facile de modifier d'autres capacités pour qu'ils utilisent la terre au lieu du feu, du froid ou du magnétisme. Par exemple, Ailes de Feu pourrait devenir Ailes de Terre, Armure de Glace pourrait devenir Armure de Terre, et ainsi de suite. Ces altérations ne changent rien sauf le type de dégâts et les éventuelles répercussions (par exemple, Ailes de Terre pourrait générer des nuages de poussière dans leur sillage).

\section*{Des Capacités qui font référence à d'autres Capacités}
Certaines capacités du chapitre Capacités font référence à d'autres capacités. Si,pour votre Focus ou votre type, vous sélectionnez une capacité qui fait référence ou modifie une capacité de rang inférieur, incluez également cette capacité de rang inférieur dans votre type ou focus en tant que sélection qu'un PJ peut faire à un rang inférieur.

\section*{Création d'une toute nouvelle Capacité}
Vous pouvez aller plus loin que la modification superficielle et créer une ou plusieurs nouvelles capacités. Ce faisant, essayez de trouver quelque chose d'aussi proche que possible de l'effet souhaité, puis utilisez-le comme modèle. Dans tous les cas, décider du coût d'une capacité en ce qui concerne la réserve d'un personnage est l'un des aspects les plus importants pour obtenir une bonne capacité.

Vous avez pu remarqué que les capacités de haut-rang sont les plus coûteuses. C'est en parti parcequ'elles permettent plus de choses, mais aussi parceque les personnages de haut-rang ont plus d'Avantage que les personnages de rang inférieur, ce qui signifie qu'ils dépensent moins de points dans leurs Réserves. Un personnage de troisème rang avec un Avantage de 3 dans la Réserve appropriée ne dépensera aucun point pour une capacité qui coûte 3 points ou moins. C'est parfait pour les capacités de rang inférieur, mais vous devez plutôt vouloir qu'un personnage réfléchisse un petit peu avant d'utiliserleurs capacités les plus puissantes. Cela veut dire que les capacités devraient coûter au moins 1 point de plus que l'Avantage que le personnage est supposé avoir à ce rang. (Souvent un personnage aura un Avantage dans les Réserve principale égal à son rang.)

Une bonne règle approximative est qu'une capacité typique devrait coûter autant de point que son rang.

\section*{Choisir des intrusions de MJ}
Pensez aux genres de choses qui pourraient surprendre, alarmer, ou tourner à la catastrophe pour quelqu'un avec le Focus qui vient d'être créée, et assignez les en tant qu'intrusion de MJ pour ce Focus. En général c'est souvent fait de manière improvisée en cours de partie. Mais en leur accordant un peu de réflexion pendant l'élaboration du focus, quand les idées sont toutes fraîches dans votre tête, cela a de bonnes chances de fournir des options diaboliques.
%%%%%%%%%%%%%%%%%%%%%%%%%%%%%%%%%%%%%%%%%%%%%%%%%%%%%%%%%%%%%%%%%%%
\section*{Catégories de Focus}
%%%%%%%%%%%%%%%%%%%%%%%%%%%%%%%%%%%%%%%%%%%%%%%%%%
\section*{Basic}
\label{subsec:basic}
Les Focus qui reposent principalement sur la fourniture d'un entrainement à des compétences, d'atouts pour les tâches et d'améliorations des Réserves de statistiques et des Avantages afin d'améliorer un personnage entrent dans la catégorie Basic. Un thème général est également inclus, comme pour la plupart des autres catégories, qui donne un sens aux différentes capacités de base fournies.

De plus, comme les avantages apportés par de tels Focus sont pour la plupart simples (généralement à quelques exceptions près), la plupart des Focus Basic seraient également appropriés pour des campagnes non fantastiques où la magie, la superscience ou les capacités psychiques n'entrent normalement pas en jeu. Cela dit, ce n'est pas parce que les capacités accordées par les Focus Basic sont simples qu'elles ne sont pas puissantes lorsqu'elles sont combinées avec les capacités accordées par le type, le descripteur, les cyphers et d'autres aspects du personnage.

\textbf{Connexion} Choisissez quatre connexions pertinentes dans la liste des Connections de Focus.
\begin{description}
    \item[Equipement Supplémentaire] N'importe quel objet nécessaire pour accomplir le thème général du focus. Par exemple, un Focus appelée Préfèrerait Lire devrait fournir quelques livres au personnage. Un Focus appélée Construit et Répar" devrait fournir un ensemble d'outils.

    \item[Suggestions d'Effet Mineur] La prochaine action est facilité.

    \item[Suggestions d'Effet Majeur] Faites un jet de Récupération gratuit, qui ne prend pas une action et qui ne compte pas dans le total des jets de récupération de la journée.
\end{description}

La liste ci-après ne sont que des exemples et n'est pas une liste complète de toutes les Focus possibles pour cette catégorie.

* Ne Fait pas Grand Chose
* Interprète la Loi
* Apprend Rapidement
* Construit et Répare
* Préfèrerait Lire

\textbf{Indications pour la Sélection de Capacités}
\begin{description}
\item[Rang1] Choisissez une Capacité qui donne un entrainement ou un atout aux compétences associées au thème du focus, ou qui donne 5 ou 6 points à une Réserve particulière.
Autrement, choisissez une Capacité qui donne seulement 2 ou 3 points à une Réserve spécifique et une Capacité qui donne un entrainement ou un atout dans seulement une tâche.

\item[Rang 2] Choisissez n'importe quelle sorte de Capacité qui n'a pas été sélectionnée au rang 1.

\item[Rang 3] Choisissez deux Capacités de rang intermédiare. Donnez-les toutes deux en tant qu'options pour le Focus; un PJ choisira l'une ou l'autre.
Une option pourrait être une Capacité non-fantastique qui améliore les capacités du personnage dans le thème du focus. Par exemple, si le thème implique de faire attention d'une certaine manière, une capacité de récupération d'information serait appropriée.

L'autre option peut être de, soit améliorer l'Avantage du personnage dans une statistique, soit fournir au personnage une forme de défence.

\item[Rang 4] Choisissez une autre Capacité qui donne un entrainement supplémentaire ou un atout à des compétences associées avec le thème du focus, ou qui donne 5 ou 6 points à la Réserve la plus appropriée pour le Focus, ou choisissez deux Capacités qui fournissent seulement 2 ou 3 points en plus d'une autre capacité de rang 4 qui améliore une seule tâche ou compétence.
Une alternative peut être de fournir une Capacité qui dévie un peu du thème du focus, comme suggéré au rang 5 ci-après.

Au final, si la motivation n'a pas encore fournie une forme de protection, une Capacité défensive peut être inscrite au rang 4.

\item[Rang 5] Choisissez une Capacité qui permet au pesonnage de dévier légèrement---peut-être comme la Capacité Compétence d'expert qui lui donne un succès automatique ans une tâche pour laquelle il est entrainé.

Ou alors, si un bénéfice non-standard a été fourni au rang4, accordez les bénéfices suggérés au rang 4 ici.

\item[Rang 6] choisissez deux Capacités de rang supérieur. Donnez les en tant qu'options pour le Focus; un PJ choisira l'une ou l'autre.
Une option peut être une Capacité qui donné à nouveau 5 ou 6 points  à la Réserve la plus appropriée pour le Focus, ou bien que le joueur peut répartir comme il le souhaite. Ou alors, un entraînement dans une compétence offensive, ou bien défensive, conviendrait.

L'autre option du rang 6 peut être de donner au personnage une Capacité toute nouvelle dans le thème, mais par contre, en-dehors du champs du fantastique. Par exemple, une Capacité qiu permet au personnage de faire deux actions au lieu d'une seule est raisonable. Donner un nouvel entrainement , un nouvel atout ou un Avantage fera l'affaire.
\end{description}
%%%%%%%%%%%%%%%%%%%%%%%%%%%%%%%%%%%%%%%%%%%%%%%%%%
\section*{Exploration}
\label{subsec:exploration}
Les Focus qui permettent à un personnage de récupérer des informations, de survivre dans des environnements non-familiers, et de trouver leur chemin vers de nouveaux emplacements ou de traquer des créatures particulières ou des adversaires, sont des Focus d'exploration. Survivre dans des environnements non-familiers requiert une sélection raisonnable d'options défensives; toutefois, les Capacités qui permettent à un personnage de trouver et d'apprendre sont à donenr en priorité.

Les Focus d'exploration sont basées sur une variété de méthodes, bien que les piliers sont l'entrainement et l'expertise. Certaines méthodes requierent des outils spécifiques (comme unvéhicule) pour accorder les bénéfices fournis, tandis que d'autres pourraient se baser sur le supernaturel ou la superscience  pour apprendre de nouvelles choses et explorer de nouveaux endroits étranges et lointains.

\textbf{Connexion:} Choisissez quatre connexions pertinentes dans la liste des Connections de Focus.
\begin{description}
    \item[Equipement Supplémentaire] Tout objet nécessaire à l'exploration. Par exemple, des cartes et/ou un compas pourraient faire partie du matériel de base, tandis qu'un personnage qui utilise des pouvoirs psychiques pourrait avoir besoin d'un miroir ou d'une sphère de cristal pour regarder dedans. L'équipement pourrait aussi inclure l'accès à un véhicule nécessaire pour l'exploration.

    \item[Suggestions d'Effet Mineur] Vous avez un Atout pour n'importe quelle action qui implique vos sens, pour percevoir ou pour attaquer, jusqu'à la fin du prochain round.

    \item[Suggestions d'Effet Majeur] Votre Avantage d'Intellect augmente de 1 jusqu'à la fun du prochain round.
\end{description}

La liste ci-après ne sont que des exemples et n'est pas une liste complète de toutes les Focus possibles pour cette catégorie.

* Explore des Endroits Sombres
* Infiltrates
* Opère sous Couverture
* Peut Séparer son Esprit de son Corps
* Pilote un Vaisseau Spatial
* Voit Au-Delà

\textbf{Indications pour la Sélection de Capacités}
\begin{description}
\item[Rang 1] Choisissez une Capacité de rang inférieur qui permet au personnage des moyens pour de l'exploration de base, de la survie ou pour de la récupération d'information, dans le thème du focus.
Quelque fois, en fonction du focus, une Capacité de rang inférieur supplémenaire est appropriée. Souvent, c'est une Capacité qui donne un entrainement dans une compétence dans un type de connaissance associée ou une compétence associée (bien que cela puisse être couvert par la Capacité principale). D'une autre manière, elle pourrait donner un simple bonus de 2 ou 3 points à la Réserve de Puissance.

\item[Rang 2] Choisissez une autre Capacité de rang inférieur qui donne un moyen supplémentaire lié à l'exploration, la survie ou la récupértion d'information.
Par exemple, un Focus  dédiée à survivre dans des conditions sauvages pourrait donner une Capacité (ou deux) qui rend plus facile à éviter les catastrophes naturelles, les poisons, les terrains difficiles, et ainsi de suite. Un Focus dédiée à l'exploration d'un endroit en particulier pourrait donner des moyens pour avoir accès à cet endroit, ou un moyen que les autres n'ont pas habituellement (comme un moyen de voir dans le noir).

\item[Rang 3] Choisissez deux capacités de rang intermédiaire. Donnez-les tous les deux comme options pour le Focus; le PJ choisira l'un ou l'autre.
Une des options devrait améliorer un peu plus le moyen d'exploration de base déjà acquis, ou alors qui devrait donner un nouveau moyen d'exploration, de survie ou de récupération d'information.
L'autre option devrait être quelque chose qui bénéficie au personnage, soit de manière offensive, soit de manière défensive (en particulier si le Focus ne l'a pas déjà accordée) ou quelque chose qui étend un peu plus le moyen pour le personnage pour explorer dans le thème du focus.

\item[Rang 4] Choisissez une capacité de rang intermédiaire offensive ou défensive (quoique ce soit qui n'ait pas été proposé au rang 3) qui bénéficie au personnage. Ou alors, si les moyens offensifs et défensif sont déjà bien representésn choisissez une Capacité de rang intermédiaire différente qui étend la capacité du personnagé à explorer, survivre ou récupérer des informations.

\item[Rang 5] Choisissez une capacité de rang supérieur qui atténue certaines pénalités pour l'exploration, la survie ou la récupération d'information dans un endroit normalement inhospitaliers.

\item[Rang 6] Choisissez deux capacités de rang supérieur. Donnez-les tous les deux comme options pour le Focus; le PJ choisira l'un ou l'autre.
Une des options devrait améliorer encore plus le moyen d'exploration de base déjà accordé, ou elle devrait donner un tout nouveau moyen d'exploration, de survie ou de récupération d'information.
L'autre option devrait être quelque chose qui bénéficie au personnage, soit de manière offensive, soit de manière défensive, ou encore un tout moyen qui étend encore plus sa possibilité d'explorer dans le thème du focus.
\end{description}
%%%%%%%%%%%%%%%%%%%%%%%%%%%%%%%%%%%%%%%%%%%%%%%%%%
\section*{Influence}
\label{subsec:influence}
Un Focus qui donne la priorité sur l'autorité et l'influence (que ce soit pour commander des personnes ou des machines), pour aider les autres, ou pour atteindre un autre position prestigieuse et importante, est un Focus d'Influence.

Ces Focus donne de l'influence au travers de l'entrainement et la persuasion, par une manipulation mentale directe, par l'utilisation de la célébrité pour attirer l'attention des personnes et influencer leurs actions, ou simplement en connaissant et en apprennant des choses pour peut affecter les décisions utltérieures. Ainsi le concept d'influence est assez large.

\textbf{Connexion:} Choisissez quatre connexions pertinentes dans la liste des Connections de Focus.

\begin{description}
    \item[Equipement Supplémentaire] Toute objet nécessaire pour permettre l'influence suggérée devrait être accordé en tant qu'équipement suppllémentaire. Certaines Focus d'influence ne nécessite rien de spécial pour obtenir ou conserver leurs bénéfices.

\item[Suggestions d'Effet Mineur] La portée ou la durée de la Capacité d'influence est doublée.

\item[Suggestions d'Effet Majeur] Un allié ou une cible indirecte peut effectuer une action supplémentaire.

La liste ci-après ne sont que des exemples et n'est pas une liste complète de toutes les Focus possibles pour cette catégorie.

* Commande aux pouvoirs Mentaux
* Contourne le Système
* Est Idolatré par Millions
* Fussionne l'Esprit et la Machine
* Parle aux Machines
* Poursuit des Sciences Etranges
* Résout des Mystères

\textbf{Indications pour la Sélection de Capacités}

\item[Rang 1] Choississez une Capacité de rang inférieur qui permet au personnage d'apprendre quelque chose assez significatif pour qu'il puisse choisir une meilleure suite d'actions (ou d'utiliser cette connaissance pour persuader ou intimider). Comment le personnage apprend l'information varie selon les spécificités du focus. Un personnage pourrait avoir besoin de faire des expériences pour obtenir des réponses, un autre pourrait ouvrir un lien télépathique avec d'autres pour échanger de l'information secrètement et rapidement, tandis qu'un autre pourrait simplement avoir été formé dans les tâches d'intéractions.
Quelque fois une Capacité supplémentaire de rang inférieur peut être appropriée en fonction du focus. Souvent c'est une Capacité qui donne un entrainement dans une compétence dans un champs de connaissance.

\item[Rang 2] Choississez une Capacité de rang inférieur qui améliore le moyen par lequel le personnage peut exercer son influence. Cela pourrait ouvrir de nouvelles possibilités pour le thème du focus, ou bien simplement augmenter le moyen de base déjà fourni. Par exemple, cette Capacité de rang 2 peut faciliter un peu plus les tâches liées à l'influence, comme en autorisant un télépathe à lire l'esprit de personnes qui ont des secrets qu'ils ne dévoileraient pas autrement, ou en donnant de l'influence sur des objets physiques (soit en les améliorant, soit en en apprennant plussur eux).

\item[Rang 3] Choisissez deux capacités de rang intermédiaire. Donnez-les tous les deux comme options pour le Focus; le PJ choisira l'un ou l'autre.
Une des options devrait fournir un moyen offensif ou défensif dans le cadre spécifique deu type d'influence du focus. Par exemple, un inventeur peut créer un sérum qui lui donne une capacité améliorée (qui pourrait être utilisée pour l'ataque ou la défense), un télépathe peut avoir une méthode pour blesser des adversaires ave de l'énergié mentale, et une personne avec seulement les compétences de base dans les débats et l'influence au travers de la célébrité pourrait avoir besoin d'entrainement aux armes ou bien de son entourage.
L'autre option de rang intermédiaire devrait fournir une Capacité supplémentaire pour influencer, toujours dans le thème du focus, ou améliorer un peu plus la Capacité de base d'influence déjà choisie. Cette option n'est pas directement offensive ou défensice, mais fournit soit une toute nouvelle Capacité liée à la Capacité de base, ou améliore la force, la portée ou débloque une extension de la Capacité déjà retenue. Par exemple, un Télépaths pourrait avoir une Capacité de suggestion psychique.

\item[Rang 4] Choisissez une capacité de rang intermédiaire qui est une utilisation, soit offensive soit défensive, de la Capacité d'influence, en tout cas une qui n'ait pas été séletionnée au rang précédent.
Ou alors, cette Capacité peut permettre un nouveau moyen lié au genre d'influence donné par le Focus.

\item[Rang 5] Choisissez une avant-dernière Capacité de rang supérieur qui utilise la Capacité d'influence spécifique donnée aux rangs inférieurs.
Ou alors, choisissez une Capacité qui n'a pas été sélectionnée à un rang précédent, qui ouvre une nouvelle possibilité de moyen d'influence. Par exemple, si l'influence sur laquelle se base le Focus est télépathique, la Capacité de rang 5 pourrait autoriser le personnage à voir dans le future pour obtenir des atouts pour s'occuper des adverversaires (et des alliés).

\item[Rang 6] Choisissez deux capacités de rang supérieur. Donnez-les tous les deux comme options pour le Focus; le PJ choisira l'un ou l'autre.
Une des options devrait fournir un moyen soit offensif, soit défensif, à l'opposé du moyen fourni au rang 4 (bien que ce soit de rang supérieur au lieu de rang intermédiaire).
L'autre option devrait être quelquechose qui explore un peut plus l'usage de l'influence de base permise par le Focus. Si le choix du rang 5 était offensif ou défensif, cela pourrait être une Capacité encore meilleure liée au genre d'influence exercé, ou une manière différente d'utiliser cette Capacité pour débloquer une facette encore incore inconnue de la Capacité.
%%%%%%%%%%%%%%%%%%%%%%%%%%%%%%%%%%%%%%%%%%%%%%%%%%
\section*{Manipulation d'énergie}
\label{subsec:manipulation_denergie}
Un Focus de Manipulation d'énergie offre des Capacités pour invoquer le feu, l'électricité, la force, le magnétisme ou des formes d'énergies non-standard comme le froid, la pierre, ou quelque chose d'étrange comme le vide ou l'ombre. Ces Capacités donne dh'abitude au personnage une maière d'atteindre quelque chose comme un équilibre entre l'attaque et se donner à eux-mêmes ou à ses alliés une protection supplémentaire. Le Focus donne aussi habituellement des Capacités qui permettent d'autres façon d'utiliser une énergie spécifique pour des choses comme le transport, créer une grande concentration d'énergie qui affecte plusieurs cibles, ou créer un objet temporairement ou une barrière d'énergie.

\textbf{Connexion:} Choisissez quatre connexions pertinentes dans la liste des Connections de Focus.

\begin{description}
    \item[Equipement Supplémentaire] Une ou plusieurs pièces d'équipement immunisées contre l'énergie manipulée, cela peut être un ensemble de vêtements. Ou alors, quelque chose qui est lié à l'énergie générée. Certaines Focus dans cette catégorie ne requiert pas d'équipement supplémentaire.
    \item[Capacités énergétiques] Si le Type de pesonnage fournit des Capacités spéciales qui normalement utilisent une autre forme d'énergie, ces capacités produisent alors la sorte d'énergie du focus. Par exemple, si un personnage utilise cette Focus pour manipuler l'électricité, les éclairs de force deviennent des éclairs d'électricité. Ces altérations ne changent rien à part le type de dommage et tout effet secondaire (par exemple, l'électricité peut provoquer des courts-circuits dans les systèmes électroniques).
    \item[Suggestions d'Effet Mineur] La cible ou quelque chose près de la cible est désavantagée à cause de l'énergie résiduelle.
    \item[Suggestions d'Effet Majeur] Un objet important porté par la cible est détruit.
\end{description}
La liste ci-après ne sont que des exemples et n'est pas une liste complète de toutes les Focus possibles pour cette catégorie.

* Absorbe l'Energie
* Façonne la Foudre
* Fait Résonner le Tonnerre
* Manipule la Matière Noire
* Porte un Eclat de Glace
* Se Revêt d'un Halo de Feu

\textbf{Indications pour la Sélection de Capacités}

    \item[Rang 1] Choisissez une Capacité de rang inférieur qui soit inflige des dommages, soit fournit une protection, par l'utilisation de l'énergie d'une manière ou d'une autre.
    Quelque fois, une capacité supplémentaire de moindre puissance est appropriée en fonction du type d'énergie. Par exemple, un Focus qui contrôle le froid peut accorder une Capacité pour créer des sculptures de neige. Un Focus qui contrôle l'électricité pour fournir une Capacité pour charger un artifact épuisé, ou avoir un atout pour manipuler des systèmes électriques. Un Focus qui absorbe l'énergie peut donner une Capacité pour la libérer en tant qu'attaque de base. Et ainsi de suite.
    \item[Rang 2] Choisissez n'importe quel sorte de Capacité qui n'a pa sété choisie au rang 1.

    \item[Rang 3] Choisissez desu Capacités de rang intermédiaire. Proposez les en tant qu'options pour le Focus. Le PJ choisira l'une ou l'autre.
    Une option peut être une Capacité qui inflige des dommages par l'utilisation du type d'énergie sélectionnée (avec peut-être un effet secondaire).

    L'autre option peut accorder, par l'utilisation du type d'énergie, un mouvement amélioré, une protection supplémentaire, ou quelque chose de complètement nouveau, comme de drainer l'énergie d'une machine (si elle utilise de l'électricité), d'enfermer une victime dans des couches de glace (si elle utilise le froid), de créer un silence parfait (si elle utilise le son), de créer un flash de lumière éblouissant (si elle utilise de la lumière), etc.

    \item[Rang 4] Choisissez n'impote quelle genre de Capacité qui n'a pas été choisie au rang 3.

    \item[Rang 5] Choisissez une Capacité de rang supérieur (avec si possible un effet secondaire) qui affecte plus d'une cible en utilisant l'énergie sélectionnée, ou une Capacité qui utilise l'énergie d'une façon qui n'a pas été précédement décrite, comme indiqué aux rangs 3 et 6.

    \item[Rang 6] Choisissez deux Capacités de rang supérieur. Proposez les en tant qu'options pour le Focus; Le PJ choisira l'une ou l'autre.
    Une de ces Capacités de rang supérieur devrait utiliser l'énergie choisie pour infliger beaucoup de dommages à une ou plusieurs cibles.

    L'autre option devrait utiliser l'énergie sélectionnée pour accomplir une tâche qui n'est pas proposée par une capacité de rang inférieur. Par exemple, façonner un suivant enflammé (si c'est le feu), se téléporter sur une grande distance en tant qu'éclair (si c'est l'électricité), créer un objet solide à partir de lénergie, etc.
\end{description}
%%%%%%%%%%%%%%%%%%%%%%%%%%%%%%%%%%%%%%%%%%%%%%%%%%
\section*{Manipulation de l'Environnement}
\label{subsec:manipulation_de_lenvironement}
Les Focus qui permettent à un personnage de déplacer des objets, d'affecter la gravité, de créer des objets (ou des illusions d'objets), et ainsi de suite, sont des Focus de Manipulation de l'environnement. Ceci dit, dans beaucoup de cas, comme de l'énergie est utilisée pour toutes ces actions, les catégories de l'énergie et de l'environnement se recoupent par endroit. Un Focus de Manipulation de l'environnement donne la priorité aux Capacités qui affectent indirectement les adversaires et les alliés au travers d'objets, de forces et de modification de l'environnement; un Focus de Manipulation de l'énergie  donne la priorité à provoquer des dommages directement sur les cibles avec l'énergie ou la force choisie.

Par exemple, plutôt que de foudroyer un adversaire avec une pulsation de gravité qui fait des dommages, un personnag, utilisant un Focus de Manipulation de l'environnement basée sur la gravité, a plus de chance d'avoir des Capacités qui maintiennent la cible sur place, ou qui utilisent la gravité pour lancer des objets lourds pour attaquer, ou qui diminuent la gravité dans une zone définie ou sur un objet particulier.

\textbf{Connexion:} Choisissez quatre connexions pertinentes dans la liste des Connections de Focus.

\begin{description}
    \item[Equipement Supplémentaire] Tout objet nécessaire pour manipuler l'environnement autour du personnage. Par exemple, une personne avec un Focus qui donne une Capacité de fabrication d'objets pourrait nécessiter des outils. Certaines Focus dans cette catégorie ne nécessitent rien de particulier pour acquérir ou conserver leurs bénéfices.
    \item[Capacités de Manipulation de l'environnement] Les thèmes des Focus qui implique des énergies visibles ou non peuvent avoir un impact sur l'apparence des Capacités de votre type. De telles changements, s'il y en a, ne font rien d'autre que changer l'apparence des effets. Si la gravité est manipulée, peut-être qu'une pâle lueur bleutée apparait à chaque utilisation des Capacités, même les Capacités du Type. Si une illusion est générée, peut-être que des effets visuels et auditifs impressionants accompagnent ce genre de Capacités, comme l'apparence d'un tentacule bestial qui enserre la cible quand la Capacité Stase est utilisée.
    \item[Suggestions d'Effet Mineur] La cible pivote et sa prochaine attaque est atténuée.
    \item[Suggestions d'Effet Majeur] Le personnage est régénéré et récupère 4 points dans une Réserve.
\end{description}
La liste ci-après ne sont que des exemples et n'est pas une liste complète de toutes les Focus possibles pour cette catégorie.

* Calcule l'Incalculable
* Concentre l'Esprit sur la Matière
* Contrôle la Gravité
* Contrôle le Magnétisme
* Façonne des Illusions
* Façonne des Objets Uniques
* Illumine avec Eclat
* Réveille les rêves

\textbf{Indications pour la Sélection de Capacités}
\begin{description}
\item[Rang 1] Choisissez une Capacité de rang inférieur qui confère une utilisation de base d'une Capacité qui modifie l'environnement en utilisant le thème du focus. Par exemple, un Focus qui affecte la gravité peut fournir une Capacité qui peut rendre une cible plus légère ou plus lourde. Un Focus de création d'illusion peut fournir une Capacité qui permet la création d'une image. Un Focus de fabrication d'objet peut fournir une compétence de base pour la création d'un type particulier d'objet. Un Focus de prédiction, peut calculer des probabilités de résultats et fournir au personnage des bénéfices liés à ces informations.

Quelque fois, une Capacité supplémentaire de faible puissance, dépendant du focus. Souvent, c'est une Capacité qui donne un entrainement dans une compétence dans une catégorie de connaisance.

\item[Rang 2] Choisissez une Capacité de rang inférieur qui confère un moyen offensif ou défensif lié au thème du focus.
Ou alors, cette Capacité peut fournir un moyen supplémentaire ou tout nouveau pour manipuler l'environnement lié au thème du focus.

\item[Rang 3] Choisissez deux capacités de rang intermédiaire. Donnez-les tous les deux comme options pour le Focus; le PJ choisira l'un ou l'autre.
Une des options devrait être une Capacité de rang intermédiaire lié au thème du focus qui fournit une Capacité de manipulation de l'environnement supplémentaire, ou qui améliore une Capacité de manipulation de l'environnement qui a été choisie précédemment. Cette Capacité n'est pas directement offensive ou défensive, mais fournit soit une toute nouvelle Capacité liée à la Capacité de base, soit une Capacité qui augmente la puissance, la portée, ou une autre extension d'une Capacité précédemment choisie.

L'autre option de rang intermédiaire devrait fournir une Capacité offensive ou défensive liée à la forme spécifique du mouvement que le Focus permet.

\item[Rang 4] Choisissez une Capacité de rang intermédiaire dont l'usage est soit offensive soit défensive, dans tous les cas il ne faut pas que ce soit une option déjà proposée à un rang précédent.

\item[Rang 5] Choisissez une Capacité de rang supérieur de manipulation de l'environnement. Par exemple, si le Focus de manipulation est sur l'illusion, cette Capacité pourrait hanter une cible avec des images terrifiantes. Si le Focus est basée sur la gravité, elle pourrait débloquer le vol. Si c'est magnétique, la Capacité peut permettre au personnage de changer la forme du métal. Si ce sont des pouvoirs télékinétiques, la Capacité pourrait autoriser le personnage à balancer des objets volumineux sur des cibles.

\item[Rang 6] Choisissez deux capacités de rang supérieur. Donnez-les tous les deux comme options pour le Focus; le PJ choisira l'un ou l'autre.
Une des Capacités devrait fournir un moyen soit offensif, soit défensif, à l'opposé de la Capacité proposée au rang 4 (bien que que de rang supérieur).

L'autre option devrait être quelqu chose qui explorer un peu plus loin l'usage de la manipulation de l'environnement. Cette Capacité devrait être plus puissante que celle de rang 5, ou une autre façon d'utiliser cette Capacité pour débloquer une facette non encore explorée.
\end{description}
%%%%%%%%%%%%%%%%%%%%%%%%%%%%%%%%%%%%%%%%%%%%%%%%%%
\section*{Utilisation d'Alliés}
\label{subsec:utilisation_dallies}
Les Focus qui donne la priorité à fournir des suivants PNJ à un personnage sont des Focus d'Utilisation d'Alliés. Les suivants fournissent une aide au PJ de façon très variée, mais surtout ils donnent un Atout pour les actions du personnage.

Il y a plusieurs thèmes potentiels dans la catégorie Utilisation d'Alliés, depuis les capacités qui permettent au pesonnage d'invoquer ou de fabriquer des alliés, à celles qui leur donne la possibilité d'attirer des suivants par la célébrité, la magie, l'autorité ou le charisme.

\textbf{Connexion:} Choisissez quatre connexions pertinentes dans la liste des Connections de Focus.

\begin{description}
    \item[Equipement Supplémentaire] N'importe quel objet qui peut être nécessaire au personnage pour conserver un allié. Par exemple, quelqu'un avec un Focus qui utilise de la super-science pour créer des alliés robots pourrait avoir besoin d'outils pour construire et réparer ces alliés. Certaines Focus dans cette catégorie n'ont besoin de rien pour acquérir oou conserver leurs bénéfices.

    \item[Suggestions d'Effet Mineurs] Les tâches de l'allié PNJ sont facilitées à son prohain tour..

    \item[Suggestions d'Effets Majeurs] L'allié PNJ gagne une action supplémentaire immédiatement.
\end{description}

La liste ci-après ne sont que des exemples et n'est pas une liste complète de toutes les Focus possibles pour cette catégorie.

* Construit des Robots
* S'Associe avec les Morts
* Contrôle les Bêtes Sauvages
* Existe en Deux Endroits en Même Temps
* Dirige
* Maîtrise l'Essaim
* Guide les Esprits

\textbf{Indications pour la Sélection de Capacités}
\begin{description}
\item[Rang 1] Choisissez une capacité de rang inférieur qui donne un PNJ de niveau 2 au personnage, ou qui donne un bénéfice similaire fournit par un PNJ. Vous pouvez également fournir les bases pour gagner de tels alliés PNJ à des rangs plus élevés en choisissant une capacité qui donne au personnage une influence sur les autres.
Parfois, une capacité supplémentaire de rang inférieur est appropriée, en fonction du focus. Il s'agit souvent d'une capacité qui confère un entrainement dans une compétence dans un domaine de connaissances ou une compétence connexe. Par exemple, un entrainement dans une compétence liée au type d'allié PNJ que le personnage gagne serait appropriée.

\item[Rang 2] Choisissez une capacité de rang inférieur qui confère une influence sur des types de PNJ similaires à ceux obtenus par l'allié au rang précédent. Si aucun allié n'a été gagné au Rang précédent, cette capacité devrait offrir cet avantage maintenant.
Parfois, une capacité secondaire peut être appropriée en plus de la capacité fournie ci-dessus, par exemple une capacité de faible puissance qui accorde 2 ou 3 points à une Réserve.

\item[Rang 3] Choisissez deux capacités de rang intermédiaire. Donnez-les tous les deux comme options pour le Focus; le PJ choisira l'un ou l'autre.
Une option devrait être une capacité de rang intermédiaire qui améliore l'allié PNJ précédemment fourni (généralement du niveau 2 au niveau 3) ou accorde un allié supplémentaire.

L'autre option devrait être quelque chose qui profite au personnage --- peut-être une capacité offensive ou défensive, ou quelque chose qui élargit son influence sur ses alliés (ou alliés potentiels).

\item[Rang 4] Choisissez une capacité de rang intermédiaire qui donne au personnage une capacité offensive ou défensive s'il n'en a pas acquis auparavant, de préférence dans le thème du focus. Par exemple, si le personnage gagne des alliés en raison de son charisme, cette capacité peut lui permettre de commander ses ennemis pendant de brèves périodes. Si le personnage gagne des alliés en les construisant ou en les appelant, cette capacité peut lui permettre d'affecter des entités du même type qui ne sont pas déjà ses alliés.
Ou alors, cette capacité pourrait améliorer davantage un allié précédemment gagné du niveau 3 au niveau 4, ou accorder un allié supplémentaire.

\item[Rang 5] Choisissez une capacité qui améliore le personnage en lui fournissant une défense, une Réserve améliorée de statistiques ou un autre type de protection. Alternativement, cette capacité pourrait ouvrir une nouvelle possibilité pour influencer et appeler des alliés PNJ liés au thème du focus. Par exemple, quelqu'un qui garde des bêtes alliées pourrait acquérir la capacité d'invoquer une horde de bêtes inférieures. Quelqu'un qui construit des robots pourrait acquérir la capacité de construire plusieurs robots assistants de moindre importance. Et ainsi de suite.

Enfin, cette capacité pourrait améliorer un allié précédemment gagné au niveau 5.

\item[Rang 6] Choisissez deux capacités de rang supérieur. Donnez-les tous les deux comme options pour le Focus; le PJ choisira l'un ou l'autre.
L'une des capacités devrait améliorer un allié précédemment acquis au niveau 5, si cela n'était pas déjà fourni au rang 5. Si tel est le cas, cette capacité pourrait être fournie *en plus* de deux autres capacités associées.

L'autre option de rang supérieur devrait fournir au personnage une poignée d'alliés de niveau 3.

La dernière capacité de rang supérieur pourrait être une nouvelle possibilité pour influencer et appeler des alliés PNJ liés au thème du focus. Par exemple, quelqu'un qui gagne des alliés grâce à un charisme et un entrainement pourrait acquérir la capacité d'apprendre des informations autrement impossibles à glaner.
\end{description}
%%%%%%%%%%%%%%%%%%%%%%%%%%%%%%%%%%%%%%%%%%%%%%%%%%
\section*{Irrégulier}
\label{subsec:irregulier}
La plupart des foci ont un thème principal, un "histoire pour le personnage" qui implique logiquement une série de capacités associées. Toutefois, certains thèmes de foci sont si vagues qu'ils ne correspondent à aucune catégorie, à part une catégorie irrégulière spécifique.

Les foci irréguliers offrent un ensemble de capacités disparates. Cela est généralement dû au fait que le thème global exige de la variabilité et l'accès à plusieurs types de capacités différents. Souvent, ces foci se retrouvent dans des genres qui suggèrent des ajustements de règles supplémentaires pour exploiter encore davantage leur utilisation, comme les changements de puissance dans le genre des super-héros ou la magie dans le genre de la fantasy. Cependant, d'autres foci irréguliers sont possibles.

\textbf{Connexion:} Choisissez quatre connexions pertinentes dans la liste des Connections de Focus.

\begin{description}
    \item[Equipement Supplémentaire] Tout objet nécessaire au thème du focus. Par exemple, un focus sur un thème de super-héro peut fournir un costume de super-héro.

    \item[Suggestions d'Effet Mineur] La cible est aussi étourdie pendant un round, durant lequel toutes ses tâches sont entravées.

    \item[Suggestions d'Effet Majeur] La cible est sonnée et perd son prochain tour.
\end{description}

La liste ci-après ne sont que des exemples et n'est pas une liste complète de toutes les Focus possibles pour cette catégorie.

* Canalise les Bénédictions Divines
* A des Ascendants Nobles
* Est sorti de l'Obélisque
* Vole Plus Vite qu'une Balle
* Maîtrise les Sortilèges
* Parle au Nom de la Terre

\textbf{Indications pour la Sélection de Capacités}
\begin{description}
\item[Rang 1] Choisissez une capacité de rang inférieur qui donne un des bénéfices promis par le thème du focus et qu'un personnage de rang 1 devrait avoir.
Quelque fois, pour certain focus, une capacité supplémentaire de faible puissance est appropriée. Souvent c'est une capacité qui confère un entrainement dans une compétence dans un domaine de connaissance associé au focus, ou une autre compétence qui peut être associée au focus. D'une autre manière, le focus peut accorder un simple bonus de 2 ou 3 points à une Réserve.

\item[Rang 2] Choisissez une capacité de rang inférieur qui donne un des bénéfices promis par le thème du focus, et qui n'est pas liée directement à la capacité fournie au rang 1. Ceci dit, si une défense n'a pas été donnée au rang 1, le rang 2 est un bon endroit pour la placer.

\item[Rang 3] Choisissez deux capacités de rang intermédiaire. Donnez-les tous les deux comme options pour le Focus; le PJ choisira l'un ou l'autre.
Une option devrait fournir un des bénéfices promis par le thème du focus, une qui ne serait pas liée à ceux fournit aux rangs précédents.
L'autre option devrait inclure une méthode d'attaque si aucune n'a été déjà accordée. D'une autre manière, si les capacités de rang inférieur ne fournit pas exactement ce qu'il faut au personnage à ce stade, cette option peut augmenter un peu plus une capacité débloquée à un rang inférieur.

\item[Rang 4] Choisissez une capacité de rang intermédiaire qui fournit un des bénéfices promis par le thème du focus, et qui n'est pas liée aux capacités fournies précédement.

\item[Rang 5] Choisissez une capacité de rang supérieur qui fournit un des bénéfices promis par le thème du focus, et qui n'est pas liée aux capacités fournies aux rangs précédents.

\item[Rang 6] Choisissez deux capacités de rang supérieur. Donnez-les tous les deux comme options pour le Focus; le PJ choisira l'un ou l'autre.
Une option devrait fournir un des bénéfices promis par le thème du focus, une qui ne serait pas liée à ceux fournit aux rangs précédents. Toutefois, cette capacité pourrait aussi fournir un version supérieure d'une capacité de rang inférieur si une option de rang intermédiaire ou de rang inférieur n'était pas suffisante.

L'autre option devrait offrir une méthode alternative pour compléter le personnage d'une manière qui ne reproduit pas la première option de Rang 6. Par exemple, si la première option fournissait une sorte d'attaque, celle-ci pourrait être une interaction, une collecte d'informations ou une capacité de guérison, en fonction du thème général du focus.

\end{description}
%%%%%%%%%%%%%%%%%%%%%%%%%%%%%%%%%%%%%%%%%%%%%%%%%%
\section*{Expertise des Mouvement}
\label{expertise_des_mouvements}
Les foci qui privilégient des formes de mouvement novatrices — pour exceller en combat, échapper à des situations quand la plupart ne le peuvent pas, se déplacer furtivement dans un but de vol ou d'évasion, ou accéder à des lieux normalement inaccessibles — appartiennent à la catégorie de l'expertise en mouvement. Ces foci offrent généralement des moyens d'accorder soit de l'offensive, soit de la défense par le mouvement, bien qu'ils puissent parfois permettre les deux.

Le focus classique d'expertise en mouvement repose sur la vitesse pour effectuer plus d'attaques et éviter d'être touché, bien que l'agilité générale puisse également offrir le même avantage. D'autres focus de cette catégorie peuvent s'inscrire dans ce thème en permettant à un personnage de devenir immatériel, de changer de forme pour devenir quelque chose comme de l'eau ou de l'air, ou de se déplacer instantanément par téléportation.

\textbf{Connexion:} Choisissez quatre connexions pertinentes dans la liste des Connections de Focus.

\begin{description}
    \item[Equipement Supplémentaire] Tout objet nécessaire pour atteindre de grandes vitesses, changer d'état ou obtenir autrement les avantages du focus devrait être fourni en tant qu'équipement supplémentaire. Certains foci dans cette catégorie ne nécessitent rien pour obtenir ou conserver leurs avantages.

\item[Suggestions d'Effet Mineur] La cible est aussi étourdie pendant un round, durant lequel toutes ses tâches sont entravées.

\item[Suggestions d'Effet Majeur] La cible est sonnée et perd sa prochaine action.
\end{description}

La liste ci-après ne sont que des exemples et n'est pas une liste complète de toutes les Focus possibles pour cette catégorie.

* Existe Partiellement Hors de Phase
* Bouge comme un Chat
* Va Comme le Vent
* S'enfuit
* Déchire les Murs du Monde
* Voyage à Travers le Temps
* Rôde dans les Bas Quartiers

\textbf{Indications pour la Sélection de Capacités}
\begin{description}
\item[Rang 1] Choisissez une capacité de rang inférieur qui accorde le bénéfice de base du style de mouvement défini pour le focus, que ce soit une vitesse accrue, de l'agilité, de l'immatérialité, etc. Parfois, une capacité de faible puissance supplémentaire est appropriée, selon le focus. Si le bénéfice de base du mouvement exige une certaine compréhension ou un entraînement supplémentaire, cette capacité pourrait fournir ce genre de compétence. Alternativement, si le mouvement fourni semble également devoir débloquer un avantage offensif ou défensif de base (reposant sur l'utilisation de la capacité de base initiale), ajoutez-le également.

\item[Rang 2] Choisissez une capacité de rang inférieur qui procure une faculté offensive ou défensive liée au thème du focus. Alternativement, cette capacité peut fournir une faculté supplémentaire liée au type de mouvement et qui donne des informations utiles au personnage qui ne seraient pas accessible par ailleurs sans le focus.

\item[Rang 3] Choisissez deux capacités de rang intermédiaire. Donnez-les tous les deux comme options pour le Focus; le PJ choisira l'un ou l'autre.
Une option devrait fournir une faculté de mouvement supplémentaire ou qui améliore un peu plus la capacité de mouvement de base, en relation avec les thème du focus. Ce n'est pas directement offensif ou défensif, mais donne au personnage un nouveau niveau de capacity ou un capacité complètement nouvelle liée à la capacité de base du mouvement choisi.

L'autre option devrait fournir une faculté soit offensive, soit défensive, liée à la forme spécifique du mouvement fournie par le focus.

\item[Rang 4] Choisissez une capacité de rang intermédiaire qui améliore les avantages donnés par le paradigme du focus d'amélioration du mouvement. Cela peut fournir une forme de défense nouvelle ou meilleure (diretement ou indirectement si le déplacement se fait dans un endroit ou une temporalité où le danger ne menace plus), ou une forme offensive nouvelle ou améliorée.

\item[Rang 5] Choisissez une avant-dernière capacité de rang supérieur liée au mouvement. Par exemple, si le mouvement fourni par le focus est temporel, cette capacité peut autoriser un saut temporel. Si le focus améliore la rapidité, cette capacité peut autoriser le personnage à se déplacer sur une très longue distance en une action. Et ainsi de suite.
alternativement, débloquez un capacité qui n'est pas encore découverte et qui découle du pouvoir de mouvement fournit par le focus.

\item[Rang 6] Choisissez deux capacités de rang supérieur. Donnez-les tous les deux comme options pour le Focus; le PJ choisira l'un ou l'autre.
Une des options devrait fournir un moyen soit offensif, soit défensif, à l'opposé de la capacité du rang 4 (bien qu'elle soit de rang supérieur au lieu de rang intermédiaire).

L'autre option devrait être quelque chose qui étend un peu plus l'utilisation de la capacité de mouvement. Si le choix du rang 5 était l'avant-dernière capacité, cela peut être une dernière capacité encore meilleure liée au mouvement.
\end{description}
%%%%%%%%%%%%%%%%%%%%%%%%%%%%%%%%%%%%%%%%%%%%%%%%%%
\section*{Combat Offensif}
\label{subsec:combat_offensif}
Les foci de combat offensif donnent le plus d'importance à infliger des dommages lors d'une bataille. Les foci dans cette catégorie donne quelques capacités défensives, mais la priorité est donnée à augmenter les dommages au maximum.

Pour y arriver, un focus de combat offensif accorde la maitrise d'un style particulier d'art martial, comme par exemple un entrainement avec une arme en particulier ou sans arme, ou alors l'utilisation d'un outil spécial (ou même une forme d'énergie). Un style peut être quelque chose d'aussi singulier que d'être le meilleur à combattre un type spécifique d'ennemi, ou quelque chose de plus générique, comme d'adopter un style de combat particulièrement viceux ou qui ne respecte pas les règles. Un combattant offensif peut utiliser le feu, une force, ou du magnétisme comme méthode préférée pour infliger des dommages.

\textbf{Connexion:} Choisissez quatre connexions pertinentes dans la liste des Connections de Focus.

\begin{description}
    \item[Equipement Supplémentaire] L'arme, l'outil, l'objet spécial ou substance (s'il y en a) nécessaire pour pratiquer le style de combat particulier. Par exemple, une dose de poison de niveau 5 pour Se Bat Sans Respecter de Règle ou Assassine, un trophée d'un ennemi abbatu pour Combat les Robots, ou des vêtements remarquables pour Combat avec Panache.

\item[Suggestions d'Effet Mineur] La cible est tellement impressionnée par votre style qu'elle est étourdie pour un round, entravant ses actions pendant ce temps.

\item[Suggestions d'Effet Majeur] Faites une attaque supplémentaire immediatement en utilisant la méthode d'attaque du focus pendant votre tour.
\end{description}

La liste ci-après ne sont que des exemples et n'est pas une liste complète de toutes les Focus possibles pour cette catégorie.

* Combat les Robots
* Se Bat Sans Respecter de Règle
* Combat avec Panache
* Chasse
* A le Droit de Porter une Arme à Feu
* Cherche les Ennuis
* Maîtrise l'Armement
* Assassine
* N'a pas Besoin d'Arme
* Accompli des Prouesses de Force
* Se Met en Rage
* Tue les Monstres
* Lance avec une Précision Mortelle
* Se Bat avec Deux Armes à la fois

\textbf{Indications pour la Sélection de Capacités}
\begin{description}
\item[Rang 1] Choisissez une capacité de rang inférieur qui inflige un dommage supplémentaire quand le pesonnage attaque en utilisant le style, l'énergie ou l'attitude spécifique du focus, ou quand elle est dirigée contre l'ennemi choisi pour le focus.
Quelque fois, une capacité de rang inférieur supplémentaire est appropriée, en fonction du focus. Par exemple, un focus qui donne une abilité avec une arme spéciale peut accorder un entrainement pour les tâches d'artisanat associées avec cette arme. Un focus qui permet d'augmenter les domages contre une sorte particulière d'ennemi peut accorder un entrainement dans des copétences pour reconnaitre, localiser, ou juste avoir des connaissances générales à propos de cet ennenmi. Un style de combat qui implique de combattre de manière vicieuse ou sale peut accorder un entrainement en intimidation. Et ainsi de suite.

Si le focus consiste à combattre un ennemi particulier, des pouvoirs secondaires supplémentaires (plus que ce qui pourrait être proposés) peuvent être appropriés. Ceux-ci peuvent soit améliorer l'efficacité contre l'ennemi choisi, soit offir des capacités plus génériques, mais liées au focus, qui permettent au personnage de les utiliser même quand ils ne combattent pas cet ennemi particulier.

\item[Rang 2] Choisissez une capacité de rang inférieur qui accorde une forme de défense en utilisant cette arme, ce style d'arme, ou l'énergie sélectionnée. Si le style d'arme est particulièrement efficace pour combattre un certain type d'adversaire, cette capacité devrait être une déense contre ce type d'ennemi. D'une autre manière, le focus peut accorder une autre méthode pour augmenter les dommages ans le contexte du focus.
Quelque fois une capacité de rang inférieur supplémentaire est aappropriée au rang 2. Dans ce cas, choisissez une capacité de rang inférieur qui n'a pas été acquise au rang 1.

\item[Rang 3] Choisissez deux capacités de rang intermédiaire. Donnez-les tous les deux comme options pour le Focus; le PJ choisira l'un ou l'autre.
Une option devrait être d'infliger des dommages supplémentaires quand on utilise le style de combat, l'énergie, ou l'attitude, ou l'ennemi sélectionné du focus. Cela pourrait être aussi simple qu'une capacité qui accorde une attaque supplémentaire contre l'ennemi sélectionné.

L'autre option devrait fournir une méthode pour neutraliser temporairement un adversaire en le désarmant, l'étourdissant ou l'assomant, le ralentissant, le restreignant, ou le déconcertant en utilisant le style de combat, l'énergie, ou l'attitude, ou l'ennemi sélectionné du focus.

\item[Rang 4] Choisissez une capacité de ang intermédiaire qui améliore un peu plus les avantages accordés par le paradigme du focus. Souvent cela inclut un entrainement dans une attaque particulière. D'une autre manière, la capacité peut augmenter les avantages acquis en gagnant un certain statut lors du combat, comme de gagner la surprise.

\item[Rang 5] Choisissez une capacité de rang supérieur qui inflige des dommages. D'une autre manière, si le focus est de se concentrer sur un type particulier d'ennemi, cette capacité peut accorder au personnage une chance de complètement neutraliser, détruire, aveugler ou tuer une cible spécifique d'au plus de niveau 3 (ou plus élevé, si le focus se concentre sur un seul ennemi).

\item[Rang 6] Choisissez deux capacités de rang supérieur. Donnez-les tous les deux comme options pour le Focus; le PJ choisira l'un ou l'autre.
Une des options devrait utiliser le paradigme du focus pour infliger des dommages exceptionnels.

L'autre option peut être une différente manière d'infliger des dommages, soit en utilisant le paradigme du focus, soit juste beaucoup de dommages en général (et en s'appuyant sur les capacités des rangs précédents pour améliorer les possibilités de toucher). Cela peut être contre des cibles multiples si la première option est pour une seule cible, ou pour tuer directement ou neutraliser une cible (à partir du niveau 4, mais avec des indications pour utiliser l'Effort pour augmenter le niveau de la cible), ou pour sélectionner un autre adversaire, faire une autre attaque ou se retirer pour combattre une autre fois.

\end{description}
%%%%%%%%%%%%%%%%%%%%%%%%%%%%%%%%%%%%%%%%%%%%%%%%%%
\section*{Soutien}
\label{subsec:soutien}
Les foci qui permettent au pesonnage d'aider les autres, de les défendre, de les soigner, et ainsi de suite, sont des foci de support. Bien sur, la plupat des capacités des foci sont souvent utilisés pour aider les autres, mais les foci de support (comme Siphonne les Pouvoirs)

Foci that allow a character to help others succeed, defend others, heal others who are hurt, and so on are support foci. Of course, most foci abilities are often used in aid of others, but support foci (such as Siphonne les Pouvoirs) prioritize aiding, healing, and improving the character who takes the focus.

Support foci rely on a variety of methods to provide their help, including martial training used in defense, supernatural ou sci-fi means of providing healing, ou simply easing the cares of others through entertainment.

\textbf{Connexion:} Choisissez quatre connexions pertinentes dans la liste des Connections de Focus.

\begin{description}
    \item[Equipement Supplémentaire] Any object necessary to provide support. For instance, someone with a focus that uses entertainment to help others would require an instrument ou similar object in aid of their craft. Some foci in this category don't require anything to gain ou retain their benefits.

\item[Suggestions d'Effet Mineur] You can draw an attack without having to use an action at any point before the end of the next round.

\item[Suggestions d'Effet Majeur] You can take an extra action in aid of an ally.
\end{description}

La liste ci-après ne sont que des exemples et n'est pas une liste complète de toutes les Focus possibles pour cette catégorie.

* Défend les Faibles
* Divertit
* Aide ses Amis
* Rend la Justice
* Guide la Communauté
* Siphonne les Pouvoirs
* Fait des Miracles

\textbf{Indications pour la Sélection de Capacités}
\begin{description}
\item[Rang 1] Choose a low-Rang ability that provides some form of defense, aid ou entertainment, benefit to recovery ou healing, ou protection. That defense ou protection could be to the PJ and not to an ally, as one cannot protect another without first being able to protect themselves (and sometimes protecting themselves is the entire point).
Sometimes an additional low-power ability is appropriate, depending on the focus. Often, this is an ability that grants skill training in a related area of knowledge ou a related skill, but it might be something that works with the initial ability that, by itself, wouldn't do much.

\item[Rang 2] Choose a low-Rang ability that follows up on the support style opened in the previous Rang. If the previous Rang's ability provided a means of protection only for the focus taker, this Rang 2 ability should specifically provide aid to another. If the previous Rang specifically provided aid to another, this Rang 2 ability could defend the focus taker ou provide an offensive capability grounded, if possible, in the focus's theme.

\item[Rang 3] Choisissez deux capacités de rang intermédiaire. Donnez-les tous les deux comme options pour le Focus; le PJ choisira l'un ou l'autre.
One option should work within the focus's theme to aid, heal, protect, ou otherwise help another.

The other option should be something that benefits the character, either an offensive ou defensive ability, ou something that broadens their expertise in some fashion. Alternatively, it could be another, different method of helping someone else.

\item[Rang 4] Choose a mid-Rang ability that gives an ally a direct boon ou provides the character with a way to help another. It could also be an ability that harms ou nullifies a foe, as removing foes certainly helps allies.

\item[Rang 5] Choose a high-Rang ability that provides an offensive ou defensive option for the character, if none have been provided yet. If this need has been previously addressed ou is deemed unnecessary, choose a high-Rang ability that provides some form of defense, aid ou entertainment, benefit to recovery ou healing, ou protection to another. For example, a Rang 5 ability might grant an ally an additional free action ou allow them to repeat a failed action.

\item[Rang 6] Choisissez deux capacités de rang supérieur. Donnez-les tous les deux comme options pour le Focus; le PJ choisira l'un ou l'autre.
One of the options should provide an ultimate method of helping another in the theme of the focus.

The other option could provide an alternative ultimate method of helping another; many foci in this category do. However, an option that provides high-Rang offense ou defense is also completely reasonable.
\end{description}
%%%%%%%%%%%%%%%%%%%%%%%%%%%%%%%%%%%%%%%%%%%%%%%%%%
\section*{Combat défensif}
\label{subsec:combat_defensif}
Les Motivations qui donnent la priorité pour être capable de subir des assauts et d'absorber les dommages d'adversaires font parti de la catégorie Combat défensif. Ces Motivations fournissent aussi des Capacités offensives liées à la méthode particulière par laquelle la protection améliorée est donnée, mais les Capacités défensives sont plus accentueés.

Certaines Motivations de Combat défensif impliquent une transformation physique qui donne une protection supplémentaire, et d'autres se basent sur de l'entrainement spécialisé, utilisent des outils comme des boucliers ou une armure lourde, ou fournissent le moyen de se soigner très vite. Les sortes de tranformation physique qu'une Motivation de Combat défensif peuvent fournir varient beaucoup. Une Motivation peut transformer la peau d'un personnage en pierre, renforcer son corps avec du métal, le transformer en créature monstrueuse, les faire grandir tellement que cela devient difficile de les blesser, et ainsi de suite.

\textbf{Connexion:} Choisissez quatre connexions pertinentes dans la liste des Connections de Focus.

\begin{description}
    \item[Equipement Supplémentaire] Tout objet nécessaire pour maintenir une transformation physique (comme un outil pour des réparation pour une transformation partiellement robotique, un bouclier ou un autre outil défensif pour lequel le personnage est entrainé, ou un genre d'amulette ou de sérum). Certaines Motivations de combat défensif n'ont besoin de rien pour obtenir ou maintenir leurs bénéfices.

\item[Suggestions d'Effet Mineur] +2 à l'Armure pour quelques rounds.

\item[Suggestions d'Effet Majeur] Regagner 2 points à la Réserve de Puissance.
\end{description}

La liste ci-après ne sont que des exemples et n'est pas une liste complète de toutes les Focus possibles pour cette catégorie.

* Brandit un Bouclier Exotique
* Demeure dans la pierre
* Fusionne la Chair et l'Acier
* Garde le Passage
* Grandit Jusqu'au Ciel
* Hurle à la Lune
* Maîtrise la Défense
* Ne S'Avoue Jamais Vaincu
* Résiste Comme une Citadelle
* Vit dans la Nature Sauvage

\textbf{Indications pour la Sélection de Capacités}
\begin{description}
\item[Rang 1] Choisissez une Capacité de rang inférieur qui permet une défense dans le thème de la Motivation. Si le thème est simplement de l'entrainement intensif ou l'utilisation d'un outil défensif, la Capacité peut être aussi simple qu'un bonus à l'Armure. Si la protection vient d'une transformation physique cette Capacité fournit des effets de la forme de base, des nénéfices et dans certains cas des inconvénients pour la réalisation de la transformation. Une Capacité de soin de rang inférieur pourrait être appropriée au premier rang.
Quelque fois une Capacité supplémentaire de faible puissance convient en fonction de la Motivation. Si le personnage se transforme, cette Capacité peut fournir un effet secondaire, bien que dans le cas de certaines transformations, cela pourrait être la description de comment quelqu'un avec une physionomie anormale peut complètement guérir. Pour d'autres Capacités, le pouvoir secondaire peut être simplement un entrainement dans une compétence liée, ou cela peut débloquer la possibilité d'utiliser une armure particulière ou un bouclier sans pénalité.

\item[Rang 2] Si le thème de la Motivation n'est pas une transformation physique, choisissez une Capacité de rang inférieur qui offre une méthode supplémentaire pour défendre, soigner des dommages ou éviter des attaques.
Si le thème de la Motivation est une transformation physique, coisissez une Capacité de rang inférieur qui débloque une nouvelle possibilité liée à la forme que prend le personnage.Cela peut être de gagner un meilleur contrôle de la transformation, débloquer une interface robotique, ou bien débloquer encore plus cette forme. Cette Capacité n'est pas nécessaire défensive, bien qu'elle pourrait l'être.

\item[Rang 3] Choisissez deux capacités de rang intermédiaire. Donnez-les tous les deux comme options pour la Motivation; le PJ choisira l'un ou l'autre.
Une des options devrait fournir une forme supplémentaire de protection dans le thème de la Motivation, telle que plus de possibilités défensives débloquées de la transformation (qui peut aussi venir avec de nouvelles posisbilités offensives) ou un simple entrainement physique si la défense est acquise par les compétences ou les soins.
L'autre option devrait fournir une possibilité offensive , particulièrement si vous créez une Motivation qui n'est pas une transformation et qui n'a pas encore de bénéfices offensifs. Cette possibilité pourrait être une attaque améliorée ou qui permet un autre néfice au combat, comme s'échapper rapidement ou (de l'autre côté du spectre) devenir inébranlable.

\item[Rang 4] Choisissez une capacité de rang intermédiaire qui améliore les avantages donnés par le paradigme de focus "j'inflige des dommages". Cela inclut souvent un entrainement dans un type particulier de défense. Ou alors, cela peut augmenter les avantages fournis par des capacités de défense précédentes, que ce soit par un meilleur contrôle sur une transformation, gagner des chances supplémentaires pour éviter des dommages ou des tâches pour ré-essayer des tâches associées avec une détermination renforcée, et ainsi de suite. Si le focus manque d'options offensives, c'est une bonne place pour en ajouter.

\item[Rang 5] Choisisssez une capacité de rang supérieur qui accorde une protection, sous la forme par exemple de se débarrasser d'un état affaibli (incluant la mort). Si le focus accorde une transformation physique, cette capacité peut débloquer ou améliorer un peu plus une capacité déjà acquise, que ce soit de manière offensive, défensive, ou quelque chose lié à de l'exploration ou des intéractions (comme de voler si la forme est ailée, ou d'intimider si la forme est effrayante, et ainsi de suite).

\item[Rang 6] Choisissez deux capacités de rang supérieur. Donnez-les tous les deux comme options pour le Focus; le PJ choisira l'un ou l'autre.
Une des options devrait utiliser le paradigme du focus pour améliorer la défense, la protection ou une capacité pour se débarraser des dommages.

L'autre option peut être une manière différente de défendre. Dans certains cas, la meilleure des défenses étant une bonne attaque, cette option peut fournir une capacité offensive de rang supérieur  pour rester dans le thème du focus, que ce soit une augmentation des dommages pendant l'attaque, ou un meilleur contrôle d'une transformation physique instable.
\end{description}
%%%%%%%%%%%%%%%%%%%%%%%%%%%%%%%%%%%%%%%%%%%%%%%%%%
\section*{Foci de Superhéro}
%%%%%%%%%%%%%%%%%%%%%%%%%%%%%%%%%%%%%%%%%%%%%%%%%%
\section*{Personnaliser des Focus}
De temps en temps, tout le contenu d'un Focus n'est pas adapté à l'idée d'un personnage, ou peut-être que le MJ a besoin d'indications supplémentaires pour créer un nouveau Focus. Dans tous les cas, la solution réside dans les Capacités des Focus à leurs niveaux de base.

Pour chaque rang, un joueur peut sélectionner une des Capacités ci-dessous à la place de la Capacité fournie par le rang. Plusieurs de ces Capacités de remplacement, particulièrement aux rangs les plus élevés, peuvent impliquer une modification corporelle, de l'intégration de systèmes high-tech, d'apprendre des sorts puissants, de découvrir des secrets interdits, ou quelque chose similaire approprié au genre.

\textbf{Rang 1}

* Potentiel amélioré
* Prouesses au combat

\textbf{Rang 2}

* Capacité de rang inférieur: choisissez une Capacité de remplacement de rang 1 ci-dessus.
* Compétence avec les attaques
* Compétence en Défense Supérieure
* Pratique de toutes les armes

\textbf{Rang 3}

* Capacité de rang inférieur: choisissez n'importe quelle Capacité de remplacement de rang 1 ou 2 ci-dessus.
* Armure Corporelle
* Santé incroyable

\textbf{Rang 4}

* Capacité de rang inférieur: choisissez n'importe quelle Capacité de remplacement de Rang 1, 2, ou 3, ci-dessus.
* Armes intégrées
* Résistance au poison

\textbf{Rang 5}

* Capacité de rang inférieur: choisissez n'importe quelle Capacité de remplacement de Rang 1, 2, 3, ou 4 ci-dessus.
* Adaptation
* Champ défensif

\textbf{Rang 6}

* Capacité de rang inférieur: choisissez n'importe quelle Capacité de remplacement de Rang 1, 2, 3, 4, ou 5 ci-dessus.
* Abilities
* Champ réactif
