%#######################################################################
%            CHAPTER 11
%#######################################################################
\startchapter{Les Règles du Jeu}{ch:chapter11}{CSCOLORPARTTWO}

Les parties de jeux de rôle avec le Cypher System sont joués par les imaginations conjointes de tous les joueurs, y compris la Meneuse. La Meneuse pose la scène, les joueurs décident ce que leurs personnages vont faire, et la Meneuse décrit ce qui se passe ensuite. Les règles et les dés peuvent aider à guider le jeu en douceur, mais ce sont les joueurs, et non les règles ou les dés, qui dirigent l'action, déterminent l'histoire, et qui s'amusent. Si une règle se met en travers de l'histoire ou distrait le jeu, les joueurs et la Meneuse trouveront un moyen de la changer.

\textbf{Voici comment jouer au Cypher System:}
\begin{enumerate}
    \item Le joueur dit à la Meneuse ce qu'il veut faire. C'est l'action du personnage.
    \item La Meneuse détermine si l'action est de la routine (pas besoin de lancer le dé) ou si il y a une chance d'échouer.
    \item Si il y a une chance d'échouer, la Meneuse définit quelle stat la tâche va utiliser (Puissance, Célérité, Intelligence) et la difficulté de la tâche---à quel point c'est difficile entre 1 (très facile) et 10 (quasiment impossible).
    \item Le joueur et la Meneuse détermine un aspect du personnage---tel que l'entrainement, l'équipement, des capacités spéciales---peut modifier la difficulté, en mieux ou en pire, de quelques crans. Si ces modifications réduisent la difficulté à moins que un, l'action est de la routine (et réussit sans avoir besoin de lancer un dé).
    \item Si l'action n'est toujours pas de la routine, la Meneuse utilise la difficulté pour déterminer le nombre seuil---quel est le minimum à avoir avec un jet de d20 pour réussir l'action (voir la table de \hyperref[table:taskdifficulty]{table de Difficulté des Tâches}. La Meneuse n'est pas obligée de dire au joueur quel est le nombre seuil, mais elle peut donner un indice au joueur, tout particulièrement si le personnages pourrait raisonablement savoir si l'action est facile, moyenne, difficile ou impossible.
    \item Le joueur lance un d20. Si le résultat est supérieur ou égal au nombre seuil, le personnage réussit.
\end{enumerate}

C'est tout. Voilà comment faire absolument tout, que ce soit identifier un appareil inconnu, calmer un poivrot excité, grimper une falaise, ou combattere un demi-dieu. Même si vous ignorez toutes les autres règles, vous pourriez quand même jouer au Cypher System avec seulement ces informations. Les principes clés sont: personnage, actions, déterminer la difficulté de la tâche, et déterminer les modifications.


\begin{table*}[t!]
    \hspace*{-10mm}
    \centering
        \begin{tabular}{ p{17.5cm} } 
\begin{tcolorbox}[coltitle=CSCOLORPARTTWO,colback=CSCOLORPARTTWO!20!white,colframe=CSCOLORPARTTWO,title=Les Concepts Clés,sharp corners=downhill]
    \begin{multicols}{2}
        \small
        \textbf{ACTION:} Tout ce qu'un personnage fait qui soit significatif---frapper un adversaire, sauter par-dessus un torrent, activer un appareil, utiliser un pouvoir spécial, et ainsi de suite. Chaque personnage peut faire une action en un round.
        \newline\newline
        \textbf{PERSONNAGE:} Toute créature qui peut être interprétée, que ce soit un personnage-jour (PJ) interprété par un joueur, ou un personnage-non-joueur (PNJ) interprété par la Meneuse. Dans le Cypher System, les créature bizarres, les machines conscientes, et l'énergie vivante, peuvent toutes être des personnage.
        \newline\newline
        \textbf{\uppercase{Difficulté}:} Une mesure d'à quel point il est facile d'accomplir une tâche. La difficulté est évaluée sur une échelle de 1 (le plus facile) à 10 (le plus difficile). Modifier la difficulté pour rendre une tâche plus difficile est appelé "désavantager". La rendre plus facile est appelé "faciliter". Un changement de la difficulté est mesuré en crans. La difficulté est souvent égale au niveau, donc ouvrir une porte fermée de niveau 3 aura probablement une difficulté de 3.
        \newline\newline
        \textbf{FACILITER:} Une diminution de la difficulté d'une tâche, d'habitude d'un cran. Si il n'est pas indiqué de combien de cran une tâche est facilitée, alors la difficulté est réduite de un cran.
        \newline\newline
        \textbf{EFFORT:} Dépenser des points d'une Réserve de stat pour réduire la difficulté d'une tâche. Un PJ peut décider si, oui ou non, il applique de l'effort pendant son tour avant que le jet de dé soit effecté. Les PNJs n'applique jamais d'Effort.
        \newline\newline
        \textbf{ENTRAVER:} Une augmentation de la difficulté de la tâche, d'habitude d'un cran. Si il n'est pas indiqué de combien de cran une tâche est désavantagée, alors la difficulté est augmentée de un cran.
        \newline\newline
        \textbf{INAPTITUDE:} L'opposé d'être entraîné---vous êtes désavantagé à achque fois que vous tentez d'accomplir une tâche pour laquelle vous êtes inapte. Si vous devenez entraîné à cette tâche, l'entrainement et l'inaptitude se compensent l'un l'autre et vous avez maintenant de la pratique dans la tâche.
        \newline\newline
        \textbf{NIVEAU:} Une façon de mesurer la force, la difficulté, la puissance, le défi de quelque chose dans le jeu. Tout, dans le jeu, a un niveau. Les PNJs, et les objets ont un niveau pour déterminer la difficulté de toute tâche les concernant. Par exemple, le niveau d'un adversaire définit à quel point il est difficile à toucher ou à éviter dans un combat. Le niveau d'une porte indique à quel point il est difficile de la briser. Le niveau d'une serrure définit la difficulté de la crocheter. Les niveaux sont définis sur une échelle de 1 (le plus facile) à 10 (le plus difficile). Les rangs des personnages sont un petit peu comme les niveaux mais ne cont que de 1 à 6 et mécaniquement fonctionne de maière très différente des niveaux---par exemple, le rang d'un personnage ne détermine pas la difficulté d'une tâche.
        \newline\newline
        \textbf{PRATIQUE:} La capacité normale, non-altérée, d'utiliser une compétence---non-entraîné, non-spécialisé, sans inaptitude. Votre type détermine quelles sont vos compétences en arme pour lesquelles vous avez de la pratique; si vous n'avez pas de pratique dans un type d'arme, vous être inapte avec lui.
        \newline\newline
        \textbf{\uppercase{Jet de dé}:} Un jet de d20 fait par un PJ pour déterminer si une action est un succès. Bien que le jeu utilise occasionellement d'autres dés, quand le texte fait simplement fait référence à "un jet de dé" c'est toujours un jet de d20.
        \newline\newline
        \textbf{ROUND:} Une durée entre 5 et 10 secondes de longueur. Il y a environ 10 rounds dans une minute. Quand il est très important de suivre le temps de manière précise, utilisez le round. En pratique, c'est la durée que prend une action dans le jeu, mais comme tout le monde agit plus ou moins de manière simultanée, tous les personnages peuvent effectuer une action chaque round.
        \newline\newline
        \textbf{\uppercase{Spécialisé}:} Avoir un niveau exceptionnel dans la compétence pour une tâche. Etre spécialisé facilite la tâche de deux crans. Donc, si vous êtes spécialisé dans l'escalade, toutes les tâches d'escalade sont facilités de deux crans.
        \newline\newline
        \textbf{STAT:} Une des trois caractéristiques définies pour les PJs: Puissance, Célérité, ou Intelligence. Chaque stat a deux valeurs: Réserve et Avantage. Votre Réserve représente votre faculté pure et innée, tandis que votre Avantage représente votre habilité à utiliser votre faculté. Chaque Réserve de stat pour augmenter ou diminuer au cours du jeu---par exemple, vous pouvez perdre des points de votre Réserve de Puissance quand vous êtes frappé par un adversaire, ou vous pouvez dépensez des points de votre Réserve d'Intelligence pour activer une activité spéciale, ou vous reposez pour récupérer des points de Réserve de Célérité après un long jour de marche. Tout ce qui diminue, restaure, améliore  ou pénalise une stat afecte la Réserve de la stat.
        \newline\newline
        \textbf{\uppercase{Tâche}:} Toute action qu'un PJ essaie d'accomplir. La Meneuse détermine la difficulté d'une tâche. En général, une tâche est quelque chose que vous faîtes, et une action est votre accomplissement de cette tâche, mais dans la plupart des cas ils signifient la même chose.
        \newline\newline
        \textbf{\uppercase{Entraîné}:} Avoir un niveau raisonable dans la compétence pour une tâche. Être entraîné facilite la tâche. Par exemple, si vous êtes entraîné en escalade, toutes les tâches d'escalade sont facilités pour vous. Si vous devenez très entraîné à cette tâche, vous devenez spécialisé à la place d'entraîné. Vous n'avez pas besoin d'être entrapiné pour tenter d'accompir une tâche.
        \newline\newline
        \textbf{TOUR:} La partie du round pendant laquelle le personnage ou la créature réalise ses actions. Par exemple, si un Guerrier et un Adepte combattent un orc, le Guerrier fait une action pendant son tour, l'Adepte fait une action pendant son tour, et l'orc fait son action pendant son tour. Certaines capacités ou effets durent seulement pendant un tour, ou se termine quand le prochain tour démarre.
        \vspace{10mm}
\end{multicols}
\end{tcolorbox}
\end{tabular}
\end{table*}

\section*{Réaliser une Action}

Chaque personnage a un tour à chaque round. Pendant le tour d'un personnage, il peut réaliser une chose---une action. Toutes les actions sont réparties en trois catégories: Puissance, Célérité, ou Intelligence (exactement pour les trois stats). Beaucoup d'actions nécessitent des jets de dé---jeter un d20.

Chaque action réalise une tâche, et chaque tâche a une difficulté qui détermine quel nombre le PJ doit atteindre ou dépasser avec un jet de dé pour réussir.

La plupart des tâches ont une difficulté de 0, ce qui signifie que le personnage réussit automatiquement. Par exemple, marcher dans une pièce, ouvrir une porte, lancer une pierre dans un baquet sur le côté, sont toutes des actions, mais aucune ne nécessite un jet de dé. Les actions qui sont difficiles habituellement, ou qui le deviennent à cause de la situation (comme de tirer sur une cible dans le brouillard) ont une plus grande difficulté. Ces actions nécessitent un jet de dé.

Certaines actions ont besoin de dépenser un minimum de points de Puissance, Célérité ou Intelligence. Si un personnage ne peut pas dépenser ce minimum de points requis pour accomplir l'action, il échoue automatiquement.

\subsection*{Déterminer la Stat pour une Tâche}

Chaque tâche est associée à une des trois stats du personnage: Puissance, Célérité ou Intelligence. Les activités physiques qui demandent de la force, de la puissance ou de l'endurance sont associées à la Puissance. Les activités qui demandent de l'agilité, de la flexibilité ou des bons reflexes sont associées à la Célérité. Les activités mentales qui demandent de la volonté, de la mémoire ou un pouvoir mentaux sont associées à l'Intelligence. Cela signifie que vous pouvez généraliser les tâches en trois catégories: les tâches de Puissance, les tâches de Célérité, et les tâches d'Intelligence. Vous pouvez aussi généraliser les jets de dé en trois catégories: jets de Puissance, jets de Célérité, et jets d'Intelligence.

La catégorie de la tâche ou du jet de dé détermine quel genre d'Effort pour pouvez appliquer au jet de dé et peut déterminer comment les autres capacités du personnage peuvent affecter le jet de dé. Par exemple, un Adepte peut avoir une capacité qui le rend meilleur aux jets d'Intelligence, et un Guerrier peut avoir une capacité qui le rend meilleur aux jets de Célérité.

\subsection*{Déterminer la Difficulté de la Tâche}

La chose la plus fréquente que la Meneuse fait pendant le jeu---et probablement la chose la plus importante---est de définir la difficulté d'une tâche. Pour vous rendre la vie plus facile, utilisez la \hyperref[table:taskdifficulty]{table de Difficulté des Tâches}, qui associe un niveau de difficulté à une description, un nombre cible et un guide général à propos de la difficulté.

Chaque difficulté de 1 à 10 a un nombre seuil cible qui lui est associé. Le nombre seuil est facile à se souvenir: c'est toujours trois fois le niveau de difficulté. Le nombre seuil est le nombre minimum que le joueur a besoin d'obtenir en lançant un d20 pour réussir la tâche. Se déplacer vers le haut ou le bas de la table est appelé désavantager ou faciliter, et est mesuré en crans.

Par exemple, réduire la difficulté d'une tâche du niveau 5 au niveau 4 est appelé "faciliter la difficulté d'un cran" ou simplement "faciliter la difficulté" ou "faciliter la tâche". La plupart des modificateurs affectent la difficulté plutôt que le jet de dé du joueur. Cela a deux conséquences:

Les nombres seuils peu élevés comme 3 ou 6, qui pourraient être ennuyeux dans la plupart des jeux utilisant un d20, ne sont pas ennuyeux dans le Cypher System. Par exmple, si vous avez besoin de faire un 6 ou plus au jet de dé, vous avez toujours 25% de chance d'échouer.

Les plus hauts niveaux de difficultés (7, 8, 9, and 10) ne devraient pas être possible car les nombres seuils sont supérieurs ou égaux à 21, que vous ne pouvez pas sortir avec un d20. Toutefois, il est commun pour les PJs d'avoir des capacités ou de l'équipement qui facilitent un tâche et ainsi diminuent le nombre seuil à quelque chose que le joueurs peuvent sortir avec un d20.

Un rang de personnage ne détermine pas le niveau de la tâche. Les choses ne deviennent pas plus difficiles si le rang d'un personnage augmente---le monde n'est pas soudainement devenu un endroit plus difficile. Les personnages de quatrième rang ne rencontrent pas seulement des créatures de niveau 4 ou des tâches de difficulté 4 (bien que le personnage de rang 4 aura plus de chance de réussir qu'un personnage de rang 1). Ce n'est pas parce que quelque chose est de niveau 4 que c'est forcément et uniquement pour des personnages de niveau 4. De manière similaire, en fonction de la situation, un personnage de rang 5 pourrait trouver un tâche de difficulté 2 tout aussi compliquée que pour un personnage de rang 2.

Ainsi, quand elle définit la difficulté d'une tâche, la Meneuse devrait évaluer la tâche en tant que telle, et non en fonction de la puissance des personnages.

\section*{Modifier La Difficulté}

Après que la Meneuse a défini la difficulté d'une tâche, le joueur peut essayer de la modifier pour son personnage. N'importe laquelle de ces modifications ne s'applique que pour cette tentative bien particulière pour cette tâche. En d'autres mots, essayer un court-circuit sur une serrure électronique devrait être de difficulté 6, mais comme le personnage qui s'y attelle est compétent dans ce type de tâche, qu'il a les bons outils, et qu'un autre personnage l'assiste, la difficulté dans ce cas particulier peut être plus basse. C'est pourquoi il est important pour la Meneuse d'évaluer la difficulté d'une tâche sans tenir compte du personnage. Le personnage n'intervient qu'après cette évaluation.

Par l'utilisation de ses compétences et atouts, en travaillant ensemble, et---peut-être le plus important---en appliquant de l'Effort, un personnage peut faciliter une tâche de plusieurs crans pour la rendre plus facile. Plutôt que d'ajouter des bonus au jet de dé du joueur, réduire la difficulté permet de diminuer le nombre seuil. Si un PJ peut réduire la difficulté de la tâche à 0, aucun jet de dé n'est nécessaire; la réussite est automatique. (Il y a une exceptio si la Meneuse décide d'utiliser un Intrusion de MJ sur la tâche, dans ce cas, le joueur devra faire un jet de dé contre la difficulté originale)

Il y a trois manières de base avec lesquelles le personnage peut faciliter une tâche: les compétences, les atouts et l'Effort. Chaque méthode facilite la tâche d'au moins un cran---jamais par plus petit que 1.

\marginpar{Par l'utilisation des compétences, atouts et Effort, un personnage peut faciliter une tâche d'un maximum de 10 crans: un ou deux crans par les compétences, on ou deux par les atouts, et entre un et six par l'Effort.}

\subsection*{Skills}

Les pesonnages peuvent être compétents pour accomplir une tâche spécifique. Une compétence peut varier en fonction du personnage. Par exemple, un personnage peut être compétent pour mentir, un autre peut être compétent pour la tricherie, et un troisième peut être compétent dans toutes les intéractions interpersonnelles. Le premier niveau de compétence est dêtre entraîné, et il facilite la tâche d'un cran. Plus rarement, un personnage peut être incroyablement doué pour accomplir une tâche. Ce cas est appelé être spéciallisé, et cela facilite la tâche de deux crans au lieu d'un. Les compétences nepeuvent jamais diminuer un tâche de plus de deux crans---tout de qui donne plus de deux crans par des compétences n'est pas pris en compte.

\subsection*{Atouts}

Un atout est tout ce qui peut aider un personnage dans une tâche, tel qu'avoir un bon pied-de-biche quand on veut forcer une porte, ou être sous une pluie torrentielle quand on veut éteindre un feu. Les atouts appropriés varient suivant la tâche. Un ciseau bien aiguisé peut aider pour travailler le bois, mais il ne pourra pas améliorer une chorégraphie de danse. Habituellement, un atout facilite une tâche d'un cran. Les atouts ne peuvent pas faciliter une tâche de plus de deux crans---tout de qui donne plus de deux crans par des atouts n'est pas pris en compte.

(La chose importante à se rappeler est qu'une compétence ne peut réduire la difficulté d'une tâche que de deux crans maximum, et que les atouts ne peuvent pas réduire la difficulté d'une tâche de plus de deux crans, quelle que soit la situation. Ainsi, aucune difficulté pour une tâche ne peut être réduite de plus de quatre crans sans l'utilisation de l'Effort.)

\subsection*{Effort}

Un joueur peut appliquer de l'Effortpour faciliter une tâche. Pour ce faire, un joueur dépense des points de la Réserve de stat qui est la plus appropriée pour la tâche. Par exemple, appliquer de l'effort pour pousser un gros rocher par-dessus un rebord nécessite qu'un joueur doive dépenser des points de sa Réserve de Puissance; applique de l'Effort pour activer l'interface d'une machine étrange nécessite de dépenser des points de la Réserve d'Intellect. Pour tout niveau d'Effort dépensé pour une tâche, cette dernière est facilitée. Cela coûte 3 points de votre Réserve de stat pour appliquer un niveau d'Effort, et cela coûte 2 points supplémentaires pour chaque niveau d'Effort appliqué en plus (donc cela coûte 5 points pour deux niveaux d'Effort, 7 points pour trois niveaux d'Effort, et ainsi de suite). Un personnage doit dépenser des points de la même Réserve que celle associée à la tâche ou jet de dé---des points de Réserve de Puissance pour un jet de Puissance, des points de Réserve de Célérité pour un jet de Célérité, ou de points de Réserve d'Intellect pour un jet d'Intellect.

Chaque personnage a un niveau maximum d'Effort qu'il peut appliquer à une seule tâche. L'Effort ne peut pas pas faciliter une tâche de plus de six crans---tout de qui donne plus de six crans par l'Effort n'est pas pris en compte.

\textbf{Niveaux Gratuit d'Effort:} Il y a quelques capacités qui peuvent vous donner un niveau d'Effort gratuit (en général elles vous demandent d'appliquer au moins un niveau d'Effort pour la tâche). En pratique, vous gagnez un niveau d'Effort supplémentaire en plus de ce que pourquoi vous avez payé. Ce niveau d'Effort gratuit peut dépasser la limite de l'Effort pour votre personnage, mais pas la lmite des six crans d'Effort maximum pour faciliter une tâche.

\subsection*{Lancer le dé}

Pour déterminer si une action réussit ou échoue, un joueur lance un dé (toujours un d20). Si le résultat du dé est supérieur ou égal au nombre seuil, c'est une réussite. La plupart du temps, c'est tout, il n'y a rien d'autre à faire. Dans certains cas assez rares, le personnage peut appliquer un petit modificateur au résultat du dé. Si il a un bonus de +2 quand il accomplit des actions spécifiques, il ajoute alors 2 au résultat du dé. Toutefois, c'est le résultat du dé qui importe dans le cas d'un jet de dé spécial.

Si un pesonnage applique un modificateur au résulat du jet de dé, il est possible d'obtenir un résultat de 21 ou plus, ce qui implique que le personnages peuvent essayer d'accomplir un tâche dont le nombre seuil est supérieur à 20. Mais si il n'y a aucune possibilité de réussite---si même avec un 20 naturel (c'est à dire que le dé indique ce nombre) ce n'est pas suffisant pour réussir la tâche---alors aucun jet de dé n'est réalisé. Autrement les personnages auraient toujours une chance de réussir n'importe quoi, même les tâches impossibles ou ridicules comme de grimper sur de la lumière, lancer des éléphants, ou toucher une cible de l'autre côté de la montagne avec une flèche.

Si la somme des modificateurs du personnage est égale à +3, considérez les comme un atout à la place. En d'autres mots, au lieu d'ajouter un bonus de +3 a udé, réduisez la diffculté d'un cran. Par exemple, si un Guerrier a un bonus de +1 à un jet d'attaque grâce à un effet mineur, un bonus de +1 au jet d'attaque provenant de la qualité élevée de son arme, et un bonus de +1 venant d'une capacité spéciale, il n'ajoutera pas 3 à son jet de dé pour l'attaque---à la place, il réduira la difficulté de l'attaque d'un cran. Ainsi, si il attaque un adversaire de niveau 3, il aurait dû lancer un dé contre une difficulté 3 et essayer d'atteindre le nombre seuil de 9, mais grâce à ses avantages, il lancera un dé contre une difficulté de 2 en essayant d'atteindre le nombre seuil de 6.

La distinction est importante quand on empile les compétences et les atouts pour diminuer la difficulté d'une action, particulirement pour réduire la difficulté à 0 ou moins car aucun jet de dé n'est nécessaire.

\section*{C'est toujours le joueur qui lance le dé}

Dans le Cypher system, ce sont les joueurs qui dirigent l'action. Cela signifie qu'ils font tous les jets de dé. Si un PJ saute d'une véhicule en mouvement, le joueur lance un dé pour savoir si il réussit. Si un PJ cherche un panneau caché, le joueur lance un dé pour déterminer si il trouve. Si une pierre tombe sur un PJ, le joueur lance un dé pour essayer de l'éivter. Si un PJ et un PNJ luttent ensembles, le joueur lance le dé, et le niveau du PNJ détermine le nombre seiul. Si le PJ attaque un adversaire, le joueur lance le dé pour savoir si il touche. Si un adversaire attaque un PJ, le joueur lance le dé pour savoir si il évite le coup.

Comme vu dans les deux derniers exemples ci-dessus, le PJ lance le dé que ce soit pour l'attaque ou la défense. Ainsi, quelque chose qui améliore la défense peut faciliter ou désavantager le jet de dé. Par exemple, si un PJ utilise un muret pour se protéger de l'attaque, le mur facilite le jet de défense du joueur. Si l'adversaire utilise le muret pour se protéger de l'attaque du pJ, cela désavantage le jet d'attaque.

\section*{Jets de dé spéciaux}

Si le résulat du jet de dé du joueur est un 1, 17, 18, 19 ou 20 (le d20 indique un de ces nombres), alors des règles spéciales s'appliquent. Elles ont détaillées dans les sections suivantes.

\textbf{1: Intrusion de la Meneuse.} La Meneuse a une intrusion gratuite (voir plus bas) et ne dnne pas de points d'expérience (XP) en échange.

\textbf{17: Bonus aux dommages.} Si le jet était pour une attaque faisant des dommages, cela augmente les dommages de 1 point supplémentaire.

\textbf{18: Bonus aux dommages.} Si le jet était pour une attaque faisant des dommages, cela augmente les dommages de 2 points supplémentaires.

\textbf{19: Effet Mineur.} Bonus aux dommages. Si le jet était pour une attaque faisant des dommages, cela augmente les dommages de 3 points supplémentaires ou alors le PJ dispose d'un effet mineur en plus des dommages normaux. Si le jet de dé était pour autre chose qu'une attaque, le PJ dispose d'un effet mineur en plus du résultat normal de la tâche.

\textbf{20: Effet Majeur. Bonus aux dommages.} Si le jet était pour une attaque faisant des dommages, cela augmente les dommages de 4 points supplémentaires ou alors le PJ dispose d'un effet majeur ou d'un effet mineur en plus des dommages normaux. Si le jet de dé était pour autre chose qu'une attaque, le PJ dispose d'un effet majeur en plus du résultat normal de la tâche. Si le PJ a dépensé des points de sa Réserve de stat pour l'action, le coût de ces points devient nul, ce qui signifie que le pesonnage regagne ces points comme si il ne les avait pas dépensé.

\subsection*{Intrusion de la Meneuse}

Une intrusion de la Meneuse est expliquée en détail dans le chapitre [Diriger une partie avec le Cypher System](../05-game-mastering/04-running-the-cypher-system.md), mais essentiellement, cela veut dire que quelque chose survient pour compliquer la vie du personnage. Le personnage n'a pas nécesairement fait un échec critique ou fait quelque chose de mal. Cela peut simplement être que la tâche en cours présente une difficulté imprévue ou quelque chose d'autre modifie la situation actuelle.

Pour une intrusion de la Meneuse sur une jet de dé de défense, un 1 signifie que le personnage perd 2 points de dommages supplémentaires de l'attaque, car l'adversaire a eu un coup chanceux.

(Pour des détails plus complets sur l'intrusion de la Meneuse et comment l'utiliser pour produire les meilleurs effets dans le jeu, consultez le chapitre [Diriger une partie avec le Cypher System](../05-game-mastering/04-running-the-cypher-system.md).)


\subsection*{Effet Mineur}

Un effet mineur survient quand un joueur obtient un 19 naturel. La plupart du temps, un effet mineur est légèrement bénéfique pour le PJ, mais pas de façon démesurée.

Un grimpeur peut franchir une pente plus rapidement. Une machine réparée fonctionne un peu mieux. Un personnage sautant dans un puit atterri sur ses pieds. La meneuse ou le joueur peuvent inventer un effet mineur possible qui correspond à la situation, les tous les doivent être d'accord.

Ne passez pas trop de temps à penser à un effet mineur si rien d'approprié ne survient. Quelque fois, si seul le succès ou l'échec importe, il est parfaitement acceptable de ne pas avoir d'effet mineur. Gardez le mouvement dans le jeu à un rythme excitant.

En combat, l'effet mineur le plus simple est d'influger 3 points de dommages supplémentaires à l'attaque. Voici ci-dessous d'autres effets mineurs possibles en combat: combat:

\begin{description}
    \item[Objet endommagé] Au lieu de frapper la cible, l'attaque frappe ce que tient la cible. Si l'attaque touche, le pesonnage peut faire un jet de Puissance contre une difficulté égale au niveau de l'objet. Sur un succès, l'objet descend d'un ou deux crans sur le suivi des dommages aux objets.
    \item[Cible Distraite:] Pendant un round, toutes les tâches de la cible sont désavantagées.
    \item[Cible Repoussée:] La cible est repoussée de un ou deux mètres. La plupart du temps cela n'a pas beaucoup d'importance, mais si le combat a lieu près d'un rebord ou à côté d'un puit de lave l'effet peut importer.
    \item[Dépasser la cible:] Le personnage peut bouger sur une courte distance à la fin de l'attaque. Cet effet est utile pour dépasser un adversaire gardant une porte par exemple.
    \item[Frapper une partie du corps spécifique] L'attaquant frappe un endroit spécifique du corps de l'adversaire. La Meneuse définit la nature de l'effet spécial, si il y en a, qui en découle. Par exemple, frapper le tentacule d'une créature qui est enroulé autour d'un allié peut rendre sa fuite plus facile. Frapper un adversaire dans les yeux peut l'aveugler pendant un round. Frapper un créture à l'un de ses points faibles peut ignorer l'Armure.
\end{description}

D'habitude, la Meneuse juge que seule l'effet mineur se produit. Par exemple, obtenir un 19 contre un adversaire relativement faible signifie qu'il tombre de la falaise. L'effet rend le round plus interressant, mais la défaite d'une créature mineure n'a pas impact significatif sur l'histoire. Dans d'autres cas, la Meneuse peut décider qu'un jet de dé supplémentaire est nécessaire pour réussir l'effet---le jet de dé spécial  ne fait que donner au personnage *l'opportunité* d'un effet mineur. Cela arrive souvent quand l'effet souhaité est très improbable, comme de pousser un robot de combat de 50 tonnes par-dessus le rebord de la falaise. Si le joueur veut simplement 3 points de dommages supplémentaires comme effet mineur, pas besoin de jet de dé en plus.

\subsection*{Effet Majeur}

\begin{tcolorbox}[floatplacement=b,float,colback=CSCOLORPARTTWO!20!white,colframe=CSCOLORPARTTWO,title=\uppercase{Règle optionelle: choisir un effet de combat à l'avance},coltitle=CSCOLORPARTTWO,colbacktitle=CSCOLORPARTTWO!20!white,sharp corners=downhill]
Alors qu'en temps normal vous obtenez un effet mineur ou majeur en fonction de votre jet de dé, de temps en temps le personnage en situation de combat peut essayer d'obtenir un effet mineur ou majeur par pure stratégie, comme désarmer un adversaire qu'il ne veut pas blesser ou tirer dans l'oeil d'un grosse bête pour l'aveugler afin de s'enfuir. Un personnage peut choisir un effet de combat mineur ou majeur à l'avance, dans avoir à obtenir un 19 ou 20 naturel, mais l'attaque est modifiée comme suit:

\begin{itemize}
    \item Pour un effet mineur, vous enlevez 4 à vos dommages, et l'attaque est entravée.
    \item Pour un effet majeur, vous enlevez 8 à vos dommages, et l'attaque est entravée de deux crans.
\end{itemize}

Dans les deux cas, si votre attaque inflige 0 point de dégât ou moins, il n'y a aucun dommage ou même d'effet.

\end{tcolorbox}


Un effet majeur survient quand un joueur obtient un 20 naturel au dé. La plupart du temps, un effet majeur est un bénéfice important pour le personnage. Un grimpeur atteint le sommet en moitié moins de temps. Un sauteur atterri avec un tel panache que les spectateurs sont impressionnés et sans doute intimidés. Un défenseur peut faire une attaque gratuite sur un adversaire.

La meneuse ou le joueur peuvent inventer un effet majeur possible qui correspond à la situation, les tous les doivent être d'accord. Comme pour les effets mineurs, ne passez pas trop de temps à vous triturer le cerveau sur les détails de l'effet majeur. Dans les cas où seuls le succès ou l'échec est important, un effet majeur peut simplement offir au personnage un atout à usage unique (une modification d'un cran) à utiliser la prochaine fois qu'il tente la même action. Si rien d'autre ne semble approprié, la Meneuse peut aussi simplement donner au PJ une action supplémentaire dans son tour pour ce round.

En combat, l'effet majeur le plus simple et direct est d'occasioner 4 points de dommages supplémentaires pendant l'attaque. La liste ci-dessous donnent des effets majeurs communs dans un combat.

\begin{description}
    \item[Désarmer]L'adversaire laisse tomber un objet qu'il tenait
    \item[Blesser] Pour le reste du combat, toutes les tâches de l'adversaires sont atténuées.
    \item[Renverser] L'adversaire est mis à terre. Il peut se relever à son prochain tour.
    \item[Assommer] L'adversaire perd sa prochaine action.
\end{description}

Comme pour les effets mineurs, la Meneuse juge, en général, que seul l'effet majeur se produit, mais dans certains cas, la Meneuse peut décider qu'un jet de dé supplémentaire est nécessaire si l'effet majeur est inhabituel ou improbable.

\section*{Retenter une tâche après un échec}

Si un personnage échoue à accomplir une tâche (que ce soit de grimper un mur, crocheter une serrure, identifier un appareil mystérieux, ou autre), il peut retenter sa chance, mais il doit appliquer au moins un niveau d'effort pour cette nouvelle tentative. Un nouvel essai est une nouvelle action, différente de l'action qui a échoué, et qui prendra le même temps à accomplir que pour la première fois.

Dans certains cas, la Meneuse peut décider qu'un nouvel essai est impossible. Peut-être que le personnage n'a qu'une seule chance de convaincre le chef d'un groupe de bandits de ne pas attaquer, et après ça, aucune discussion ne les arrêtera.

Cette rêgle ne s'applique pas au combat cas un combat est toujours changeant. Chaque situation de chaque round est nouvelle, ce n'est pas la répétition de la situation précédente, et donc une attaque ratée ne peut pas être retentée.

\section*{Coût initial}

La Meneuse peut assigner un nombre de points pour essayer d'accomplir une tâche. Ce nombre de points est appelé coût initial et c'est simplement une indication qu'une tâche est particulièrement difficile. Par exemple, disons qu'un personnage veut tenter une action de Puissance pour ouvrir une lourde porte de cellule qui en partie scellée par la rouille. La Meneuse répond que forcer la porte est une tâche de difficulté 5, et qu'il y a un coût initial de 3 points de Puissance pour simplement essayer. Ce coût initial vient en plus de tout point que e personnage choisi de dépenser pour le jet de dé (comme pour appliquer de l'Effort), et le coût initial n'affecte pas la difficulté de la tâche. En d'autres mots, le personnage doit dépenser 3 points de Puissance pour ne faire que tenter d'accomplir la tâche, mais cela ne l'aide aucunement à ouvrir la porte. Si il veut appliquer de l'Effort pour faciliter la tâche, il devra dépenser plus de points de sa Réserve de Puissance.

L'Avantage permet de diminuer le coût initial d'une tâche, comme il le fait pour toute dépense de points de Réserve de stat. Dans l'exemple précédent, si le personnage a un Avantage de Puissance de 2, il pourrait n'avoir à dépenser que seulement 1 point (3 points moins 2 de l'Avantage de Puissance) du coût initial pour tenter d'accomplir la tâche. si il applique aussi un niveau d'effort pour ouvrir la porte, il ne pourrait pas utiliser à nouveau son Avantage---l'Avantage ne s'applique qu'une seule fois par action---et ainsi l'Effort couterait l'inégralité des 3 points de Réserve. Ainsi, il dépenserait un total de 4 points (1 pour le coût initial plus 3 pour l'Effort) de sa Réserve de Puissance.

La raison principale pour le coût initial est que même dans le Cypher System, quand les choses telles que l'Effort peut aider un personnage à réussir une action, la logique suggère néanmoins que certaines actions sont particulièrement difficiles et exigeantes, en particulier pour certains PJs.

\section*{Distance ou Portée}

Les distances (ou portées) sont simplifiées en quatre catégories de base: immédiate, courte, longue et très longue.

Une distance immédiate à partir d'un personnage est à portée de main ou à quelques pas; si le personnage se tient dans une petite pièce, tout ce qui est dans la pièce est à distance immédiate. Au plus, une distance immédiate est de 3 m (10 pieds). 

(les mots "immédiate" et "proche" peuvent être utilisés de manière interchangeable pour parler de distance. Si une créature ou un objet est à portée de main d'un personnage, ils peuvent être considérés comme à portée immédiate ou proche.)

Une courte distance est supérieure à une distance immédiate mais inférieure à 15 m (50 pieds) environ.

Une longue distance est supérieure à une courte distance mais inférieure à 30 m (100 pieds) environ.

Une très longue distance est supérieure à une longue distance mais inférieure à 150 m (500 pieds) environ.

Au-delà de cette portée, les distances sont toujours spécifiées---300 m (1000 pieds), 1,5 km (1 mile), et ainsi de suite.

Toutes les armes et capacités spéciales utilisent ces termes pour les portées ou distances. Par exemple, toutes les armes de mélées ont une portée immédiate---ce sont des armes de combat rapproché, et vous pouvez les utiliser contre quiconque se trouve à portée immédiate. Un couteau de lancer (et la plupart des armes de lancer) ont une courte portée. Un petit pistoler a aussi une courte portée. Un fisul a une longue portée.

Un personnage peut se déplacer sur une distance immédiate pendant le déroulement d'une action. En d'autres mots, il peut faire quelques pas jusqu'à l'interrupteur pour le basculer. Il peut aussi se fendre au travers d'une petit pièce pour attaquer un adversaire. Il peut ouvrir une porte et la traverser.

Un personnage peut se déplacer sur une courte distance en tant qu'action unique pendant son tour. Il peut aussi essayer de se déplacer sur une longe distance en tant qu'action, mais le joueur devrait faire une jet de dé pour voir si le personnage glisse ou trébuche en déplaçant aussi vite.

Les Meneuses et les joueurs n'ont pas besoin de déterminer les distances exactes. Par exemple, si les PJs se battent contre un groupede gardes, n'importe quel personnage peut attaquer n'importe quel adversaire dans une mélée générale---ils sont tous dans une portée immédiate. Toutefois, si un soldat se retire pour faire feu avec son blaster, un personnage peut avoir à utiliser toute son action pour se déplacer sur une courte distance pour pouvoir l'attaquer. Ce n'est pas vraiment important de savoir si le soldat est à 6 m ou 12 m--- il faut simplement considérer une courte distance. Ce sera important si le soldat est à plus de 15 m de distance car dans ce cas cela nécessitera un déplacement sur une longue distance.

\begin{tcolorbox}[colback=CSCOLORPARTTWO!20!white,colframe=CSCOLORPARTTWO,title=Autres Distances,coltitle=CSCOLORPARTTWO,colbacktitle=CSCOLORPARTTWO!20!white,sharp corners=downhill]
Dans de rares cas quand les distances au-delà de "très longue" sont nécessaire, les distances du monde réel sont celles qui fonctionnent le mieux (10 km, 100 k, ...). Toutefois, les noms de distances si-dessous peuvent être utiles dans certains cadre de campagne:
\begin{description}
    \item[Planetaire] Sur la même planète.
    \item[Interplanetaire] Dans le même système solaire.
    \item[Interstellaire] Dans la même galaxie.
    \item[Intergalactique] N'importe où dans le même univers.
    \item[Interdimensionelle] N'importe où.
\end{description}
\end{tcolorbox}

\section*{Compter le Temps}

En général, vous pouvez compter le temps comme vous le ferez normalement, en utilsant des minutes, heures, jours et semaines. Ainsi, si les personnages marchent sur 24 km, alors environ 8 heures sont passées, même si le voyage peut être décrit en seulement quelques secondes autour de la table. Compter le temps avec précision est rarement important. La plupart du temps, dire quelque chose comme "Cela prennd environ une heure" suffit largement.

C'est particulièrement vrai quand une capacité spéciale a une durée spécifique. Dans une rencontre, une durée de "une minute" est généralement la même cose que de dire "le reste de la rencontre". Vous n'avez pas besoin de décompter chaque round qui passe si vous ne voulez pas. De la même manière, une capacité qui dure pendant 10 minutes peut sans problème être considérée comme ayant la durée d'une longue conversation, le temps qu'il faut pour explorer rapidement une petite zone, ou le temps qu'il faut pour se reposer après une activité fatiguante.

\subsection*{Le Temps (table)}

\begin{table*}[t!]
    \hspace*{-40pt}
    \centering
        \begin{tabular}{ l l } 
        \multicolumn{2}{ l }{\Large \textcolor{CSCOLORPARTTWO}{Le Temps }} \\
        \textbf{Action} & \textbf{Temps Nécessaire Habituellement} \\ [0.5ex]
        \rowcolor{CSCOLORPARTTWO!50}
        Marcher sur 1,5 km sur un terrain facile & Environ 15 minutes \\
        \rowcolor{CSCOLORPARTZERO!20}
        Marcher sur 1,5 km sur un terrain accidenté (forêt, neige, collines)  & Environ 30 minutes \\
        \rowcolor{CSCOLORPARTTWO!50}
        Marcher sur 1,5 km sur un terrain difficile (maontagne, jungle dense) & Environ 45 minutes \\
        \rowcolor{CSCOLORPARTZERO!20}
        Se déplacer depuis un endroit important d'une cité dans un autre      & Environ 15 minutes \\
        \rowcolor{CSCOLORPARTTWO!50}
        S'introduire dans un endroit gardé                       & Environ 15 minutes \\
        \rowcolor{CSCOLORPARTZERO!20}
        Surveiller un nouvel endroit pour reveler des informations importantes & Environ 15 minutes \\
        \rowcolor{CSCOLORPARTTWO!50}
        Avoir une discussion en profondeur                       & Environ 10 minutes \\
        \rowcolor{CSCOLORPARTZERO!20}
        Se reposer après un combat ou une activité fatiguante  & Environ 10 minutes \\
        \rowcolor{CSCOLORPARTTWO!50}
        Se reposer et prendre un repas rapide              & Environ 30 minutes \\
        \rowcolor{CSCOLORPARTZERO!20}
        Monter ou démonter le camp                         & Environ 30 minutes \\
        \rowcolor{CSCOLORPARTTWO!50}
        Acheter du ravitaillement sur un marché ou dans une boutique    & Environ une heure \\
        \rowcolor{CSCOLORPARTZERO!20}
        Rencontrer un contact important                                 & Environ 30 minutes \\
        \rowcolor{CSCOLORPARTTWO!50}
        Rechercher un livre ou un site web                              & Environ 30 minutes \\
        \rowcolor{CSCOLORPARTZERO!20}
        Chercher des éléments cahcés dans une pièce                     & Au moins 30 minutes, peut-être une heure \\
        \rowcolor{CSCOLORPARTTWO!50}
        Charcher des Cyphers ou des objets de valeurs                   & Environ une heure \\
        \rowcolor{CSCOLORPARTZERO!20}
        Identifier et comprendre un cypher                              & Entre 15 et 30 minutes \\
        \rowcolor{CSCOLORPARTTWO!50}
        Identifier et Comprendre un artifact                            & Au moins 15 minutes, peut-être 3 heures \\
        \rowcolor{CSCOLORPARTZERO!20}
        Réparer un appareil (avec des pièces détachées et des outils disponibles)    & Au moins une heure, peut-être une journée \\
        \rowcolor{CSCOLORPARTTWO!50}
        Construire un appareil (avec des pièces détachées et des outils disponibles) & Au moins un jour, peut-être une semaine \\
    \end{tabular}
\label{table:timekeeping}
\end{table*}

\section*{Rencontres, Rounds et Initiative}

De temps en temps dans le jeu, la Meneuse ou les joueurs vont faire référnce à une "rencontre." Les rencontres ne sont pas tant des mesures du temps qu'elles sont des évènements ou des moments qui indiquent que quelqhe chose survient, comme une scène dans un film ou un chapitre dans un livre. Une rencontre peut être un combat avec un adversaire, le franchissement d'une rivière en crue, ou une négociation stressante avec un notable important. Il est pratique d'utiliser ce mot pour faire référnce à une scène spécifique, comme dans "Ma Réserve de Puissance est faible après cette rencontre hier avec le sorcier des âmes."

Un round s'étend entre 5 et 10 secondes. La longueur de cette durée est variable car dans certains cas, un round peut être d'une durée plus longue qu'un autre round. Vous n'avez pas besoin de mesurer le temps plus précisément que ça. Vous pouvez estimer qu'en moyenne, il y a environ 10 rounds dans une minute. En un round, tout le monde---chaque personnage et PNJ---peut accomplir une action.

Pour déterminer qui va en premier, en second et ainsi de suite dans un round, chaque joueur fait un jet de Célérité appellé Jet d'initiative. LA plupart du temps, ce n'est important que pour savoir quels personnages agit avant les PNJs et lesquels agissent après. Sur un jet d'initiative, un personnage qui a un résultat supérieur au nombre cible du PNJ peuvent faire leur action avant celle du PNJ. Comme pour tous le snombres cibles, celui d'un PNJ pour un jet d'initiative est trois fois le niveau du PNJ. La plupart du temps, la meneuse fera en sorte que tous les PNJs réalisent leurs actions en même temps, en utilisant le plus grand nombre cible parmi tous les PNJs. Avec cette méthode, tout personnage qui fait un jet qui dépasse que ce nombre cible agira d'abord, puis ce serton aux PNJs d'agir, en enfin ce sera la tour de tout autre personnage qui aura un résultat inférieur au nombre cible.

\marginpar(\textcolor{CSCOLORPARTTWO}{Un jet d'initiative est réalisé avec un d20. Comme votre initiative dépend de votre rapidité, si vous dépenser de l'Effort sur le jet de dén les points dépensés proviendront de votre Réserve de Célérité.}

L'ordre dans lequel les personnages agissent n'est pas très important habituellement. Si les joueurs souhaitent agir dans un ordre précis, ils peuvent prendre l'ordre décroissant dans le jet d'initiative (du plus grand au plus petit), ou dans l'ordre autour de la table, ou du plus âgé au plus jeune, etc \dots

Par exemple, les personnages de Charles, Tammie et Shanna sont dans un combat avec deux gardes de sécurité de niveau 2. La Meneuse demande aux joueurs de faire leur jet de Célérité pour déterminer l'initiative. Charles fait un 8, Shanna un 15 et Tammie un 4. Le nombre cible pour une créature de niveau 2 est de 6, et par conséquent, chaque round, Charles et Shanna agiront avant ls gardes, puis ce sera aux gardes d'agir, et enfin ce sera la tour de Tammie. Cela n'a pas forcément d'importance de savoir si Charles agit avant ou après Shanna, du moment qu'ils pensent que c'est juste.

Quand tout le monde---tous les PJS et PNJs---qui sont dans un combat ont terminé leur tour, le round se termine et un nouveau commence. Dans tous les rounds après le premier, chacun agit dans le même ordre que celui du premier round. Les personnages prennent ainsi leur tour de manièr cyclique jusqu'à la fin logiquede la rencontre (la fin du combat ou celui de l'évènemement) ou jusqu'à ce que la Meneuse leur demande de faire un nouveau jet d'initiative. La Meneuse peut demander un jet d'initiative au début de chaque nouveau round si le sconditions changent de manière dramatique. Par exemple, si les PNJs reçoivent des renforts, si l'environnement change (peut-être que les lumières s'éteignent), si le terrain change (peut-être que le balcon s'écroule sous les PJs), ou quelque chose de similaire se produit, la Meneuse peut demander de nouveaux jets d'initiative.

Comme l'action se déroule de manière cyclique, tout ce qui ne dure qu'un round se termine quand cela a commencé dans le cycle. Si Umberto utilise une capacité sur un adversaire qui lui entrave sa défense pour un round, l'effet dure jusqu'à ce que Umberto commence son prochain tour.

\textbf{Initiative plus Rapide (Règle Optionelle):} Pour rendre la rencontre encore plus rapide, si au moins un personnage fait un jet de dé plusgrand que le nombre cible du PNJ, tous les personnages prennent leur tour avant le PNJ. Si personne ne fait un jet de dé plus grand que le nombre cible du PNJ, tous les personnages agissent après le PNJ. Sur le tour des personnages, passer le tour dans le sens des aiguilles d'une montre. Si vous jouez avec un chat vidéo ou une table virtuelle, commencer par le personnage le plus à gauche et passer à droite; en répétant ensuite ce processus.

\begin{tcolorbox}[title=\uppercase{Regardons de plus près les situations qui n'impliquent pas de PJ},colbacktitle=CSCOLORPARTTWO!20!white,colframe=CSCOLORPARTTWO,colbacktitle=CSCOLORPARTTWO!20!white,coltitle=CSCOLORPARTTWO]
Dans tous les cas, c'est la Meneuse qui est l'arbitre dans les conflits qui n'impliquent pas les PJs. Ils devraient être résolus de la manière la plus interressante, logique et centrée sur la narration qu'il soit possible. En cas de doute, alignez le niveau des PNJs (personnages ou créatures), ou leurs effets respectifs, pour déterminer les résultats. Ainsi, si un PNJ de niveua 4 combat un PNJ de niveau 3, les PNJ de niveau 4 gagnera toujours, mais si il fait face à un PNJ de niveau 7, il perdra. De la même manière, une créature de niveua 4 resistera à des poisons ou des appareils de niveau3 ou moins, mais pas à ceux de niveau 5 ou plus.

Le principe est le suivant: avec le Cypher System, cela n'a pas d'importance si quelque chose est une créature, un poison, ou un rayon annulant la gravité. Si c'est d'un niveau plus élevé, ce quelque chose va gagner; si c'est de niveau inférieur, ce quelque chose va perdre. Si deux choses de niveau égal s'opposent l'un l'autre, il peut y avoir une longue et pénible bataille qui que l'un ou l'autre peut gagner.
\end{tcolorbox}

\section*{Actions}

Tout ce que fait votre personnage dans un round est une action. Il est plus simple de s'imaginer une action comme quelque chose de simple que vous pouvez faire en 5 ou dix secondes. Par example, si vous utilisez votre lance-flechettes pour tirer sur un orbe flottant, c'est une action. Idem pour courrir se cacher derrière des tonneaux, forcer une porte coincée, utiliser une corde pour tirer votre ami d'un puits, ou activer un cypher (même si il est dans votre sac).

Ouvrir une porte et attaquer un garde de l'autre côté sont deux actions. C'est plus un problème d'attention que de temps. Dégainer votre épée et attaquer un adversaire ne forment qu'une seule action. Mettre de côté votre arc et pousser une lourde bibliothèque pour bloquer une porte sont deux actions qui ont chacune besoin d'une attention propre.

Si l'action que vous souhaitez accomplir n'est pas à vore portée, vous pouvez bouger un petit peu. En pratique, vous pouvez vous déplacer sur une courte distance pour accomplir votre action. Par exemple, vous pouvez vous déplacer sur une distance immédiate et attaquer un adversaire, ouvrir une porte et vous déplacer sur une distance immédiate dans le couloir qui est derrière, ou attraper votre ami blessé qui est par terre et le traîner sur quelques pas. Ce mouvement peut être effectué avant ou après votre action, vous pouvez donc aller à une porte et l'ouvrir, ou vous pouvez ouvrir une porte et passer.

Les actions les plus commmunes sont:
\begin{itemize}
    \item Attaquer
    \item Activer une capacité spéciale (une qui n'est pas une attaque)
    \item Se déplacer
    \item Attendre
    \item Défendre
    \item Faire autre chose
\end{itemize}

\section*{Action: Attaquer}

une attaque est tout ce que vous faites à quelqu'un qui ne veut pas que vous le fassiez. Frapper un adversaire avec une dague est une attaque, tirer sur un ennemi avec un artifact produisant des éclairs est une attaque, capturer un adversaire dans des cables contrôlés magnétiquement est une attauqe, et contrôler l'esprit de quelqu'un est une attaque. Un attaque va certainement nécessiter un jet de dé pour voir si vous touchez ou si vous affectez votre cible.

Ave l'attaque la plus simple, comme un PJ qui poignarde un bandit avec un couteau, le joueur lance le dé et compare le résultat avec le nombre cible de son adversaire. Si le jet de dé est supérieur ou égal au nombrecible, l'attaque touche. Comme avec n'importe quel type de tâche, la Meneuse peut modifier la difficulté en fonction de la situation, et le joueur peut avoir des bonus au jet de dé ou peut essayer de faciliter la tâche en utilisant des compétences, des atouts ou de l'Effort.

Une attaque moins simple peut être une capacité spéciale qui assome un adversaire avec une attaque mentale. Toutefois, c'est le même principe qui s'applique: le joueur fait le jet de dé contre le nombre cible de la cible. De manière similaire, une tentative pour taquler un adversaire et le mettre à terre est simplement un jet de dé contre le nombre cible de la cible.

Les attaques sont quelque fois catégorisées comme des attaques "de mélée", ceci signifiant que vous frapper ou affecter quelqu'un ou quelque chose dans votre portée immédiate. Les attaques peuvent aussi être "à distance", ce qui veut dire que vous toucher ou affecter quelqu'un ou quelque chose d'éloigné.

Les attaques de mélée peuvent être des actions de Puissance ou de Célérité---au joueur de choisir. Les attaques physiques à distance (comme les arcs, les armes de jet, et les explosions de flammes) sont certainement des actions de Célérité, mais celles qui proviennent de capacités spéciales seraient plutôt des actions d'Intellect.

Les capacités spéciales qui nécessitent de toucher la cible nécessite une attaque de mélée. Si l'attaque échoue, le pouvoir n'est pas perdu, et vous pouvez essayer à nouveau chaque round comme action jusqu'à ce que vous touchiez la cible, que vous utilisiez une autre capacité, ou que vous choisissiez de faire une autre action qui nécessite que vous utilisiez vos mains. Ces tentatives sur les rounds qui suivent comptent comme des actions différentes, vous n'avez donc pas à garder une trace de combien d'Effort vous avez utilisé quand vous avez activé la capacité ou de comment vous avez utilisé l'Avantage. Par exemple, disons que pour le premier round de combat vous activez une activité spéciale qui nécessite que vous touchiez votre adversaire et vous décidez d'utiliser de l'Effort poru faciliter l'attaque, mais vous faite un mauvais jet de dé et vous ratez votre la cible. Au second round de combat, vous pouvez essayer d'attaquer à nouveau et utiliser de l'Effort pour faciliter le jet d'attaque.

La Meneuse et les joueurs sont encouragés à décrire chaque attaque avec des détails et de la créativité. Un jet d'attaque peut être un coup dans le bras de l'ennemi. Un échec peut être que l'épée du PJ heurte le mur. Les combattants se fendent, bloquent, plongent, sautent, et font toute sorte de mouvements qui devraient rendre le combat visuellement interressant et engageant. Le chapitre Mener le Jeu avec le Cypher System contient bien d'autres conseils à ce propos.

Les éléments classiques qui modifie la difficulté dans un combat sont la couverture, la portée, et les ténèbres. Les règles pour cela et d'autres modificateurs sont expliquées dans la section Modificateurs d'Attaque et Situations Spéciales dans ce chapitre.

\subsection*{Dommages}

Quand une attaque réussit à toucher un personnage, cela signifie en général que le personange subit es dommages.

Une attaque contre un PJ soustrait des points de l'une des Réserves de stat du personnage---d'habitude la Réserve de Puissance. a chaque fois qu'une attaque spécifie simplement "dommages" sans en spécifier le type, cela signifie que ec sont des dommages de Puissance, qui est de loin le type le plus fréquent. Les dommages d'Intellect, qui vient en général d'une attaque mentale, est toujours consignée comme des dommage d'Intellect. Les dommages de Célérité sont souvent associé à une attaque physique, mais les attaques qui infligent des dommages de Célérité sont assez rares.

Les PNJs n'ont pas de Réserves de stat. A la place, ils ont une caractéristique appelée santé. Quand un PNJ subit des dommages de n'importe quel type, le montant est retiré à sa santé. A moins que ce soit précisé autrement, la santé d'un PNJ est toujours égale à son nombre cible. Certains PNJs peuvent avoir une réaction spéciale ou un défense contre des attaques qui infligent normalement des dommages d'Intellect ou de Célérité, mais à moins que la description du PNJ ne le spécifie, tenez que tous les dommages sont soustraits de la santé du PNJ.

Les objets n'ont pas de Réserve de stat ou de santé. Ils ont un suivi des dommages, de la même forme que celui des PJs. Attaquer des objets peut les faire descendre dans leur suivi des dommages.

Les dommages sont toujours un nombre déterminé par l'attaque. Par exemple, un coup de taille avec une épée large ou un tir d'un lance-épieu peuvent infliger 4 points de dommages. L'Assaut Magique d'un Adepte inflige 4 points de dommages. Souvent il y a des moyens pour un attaquant pour augmenter les dommages. Par exemple, un PJ peut appliquer de l'Effort pour infliger 3 points de dommages supplémentaires, et obtenir un 17 naturel au dé dans une attaque permet d'infliger un point de dommage supplémentaire.

\subsection*{Armure}

Des pièces d'équipement et des capacités spéciales peuvent protéger un personnage des dommages en fournissant de l'Armure. A chaque fois qu'un personnage subit des dommages, soustrayez leur Arumure des dommages avant de réduire leur Réserve de stat ou la santé. Par exemple, si un guerrier avec une Armure de 2 est touché par un tir de pistolet qiu inflige 4 poitns de dommages, il ne subit que 2 points de dommages (4 moins 2 de l'Armure). Si l'Armure réduit les dommages à 0 ou moins, le personnage ne subit aucun dommage de l'attaque. Par exemple, l'Armure de 2 du guerrier le protège de toute attaque physique qui inflige 1 ou 2 points de dommages.

La manière la plus simple d'avoir de l'Armure est de porter une armure physique, comme une veste de cuir, un gilet pare-balle, un haubert de mailles, une carapace bioconçue, ou quelque chose d'autre en fonction du cadre de campagne. Les armures physiques sont classées en trois catégories: légères, moyennes ou lourdes. Une armure légère donne au porteur 1 point d'Armure, une moyenne donne 2 points d'Armure, et une lourde donne 3 points d'Armure.

Quand vous voyez le mot "Armure" avec une majuscule ans les règles du jeu (autre que le nom d'une capacité spéciale), cela fait réfèrence à votre caractéristique d'Armure---le nombre que vous soustrayez des dommages infligés. Quan vous voyez le mot "armure" avec un lettre miniscule, cela fait réfèrence à n'importe quelle armure physique que vosu portez.

D'autres effets peuvent ajouter de l'Armure au personnage. Si un personnage porte une côte de maille (+2 à l'Armure) et a une capacité qui le recouvre d'un champs de protection lui conférant +1 à 'Armure, son total d'Armure est de 3. Si il utilise aussi un cypher qui renforce sa peau temporairement pour un +1 à l'Armure, son total est de 4 à l'Armure.

Certains types de dommages ignorent l'armure physique. Les attaques qui infligent spécifiquement des dommages d'Intellect ou de Célérité ignorent l'Armure; la créature subit le nombre de dommages indiqué sans appliquer de réduction avec l'Armure. Les dommages ambients (voir ci-dessous) ignorent en général aussi l'Armure.

Une créature peut avoir un bonus spécial à l'Armure contre certains types d'attaques. Par exemple, une tenue de protection faite d'un robuste matériaux résistant au feu peut normalement donner au porteur un +1 à l'Armure mais compte comme +3 à l'Armure contre les attaques de feu. Un artifact porté comme un casque peut fournir un +2 à l'Armure mais seulement contre les attaques mentales.

\subsection*{Dommages Ambient}

Certains types de dommages ne sont pas des attaques directes contre une créature, mais affectent indirectement tout se qui trouve dans la zone d'effet. La plupart de ces dommages sont es efets de l'environement comme un froid hivernal, des hautes températures, ou des radiations. Les dommages de ce genre de source sont appelés dommages ambients. L'armure physique ne protège pas d'habitude contre les dommages ambients, bien qu'une tenue adaptée peut protéger contre le froid. 

\subsection*{Dommages causés par des Accidents}

Les attaques ne sont pas seule manière d'infliger des dommages à un personnage. Des accidents comme de tomber d'une grande hauteur, être brûlé par un feu, et passer du temps dans des conditions météos extrèmes infligent aussi des dommages. Bien qu'aucune liste de dangers et accidents potentiels ne soit exhaustive, la table Dommages causés par des Accidents propose des exemples classiques.

\begin{table*}[b!]
    \centering
        \begin{tabular}{ l l l } 
        \multicolumn{3}{ l }{\Large \textcolor{CSCOLORPARTTWO}{Dommages causés par des Accidents}} \\
        \textbf{Source} & \textbf{Dommages} & \textbf{Notes} \\ [0.5ex]
        \rowcolor{CSCOLORPARTTWO!50}
        Tomber       & 1 point par 3m de chuten (dommages ambients) & --- \\
        \rowcolor{CSCOLORPARTTWO!20}
        Feu mineur    & 3 points par round (dommages ambients)  & Torche \\
        \rowcolor{CSCOLORPARTTWO!50}
        Feu Majeur  & 6 points par round (dommages ambients)  & Englouti dans des flammes; lave \\
        \rowcolor{CSCOLORPARTTWO!20}
        Eclaboussement d'acide   & 2 points par round (dommages ambients)  & --- \\
        \rowcolor{CSCOLORPARTTWO!50}
        Bain d'acide     & 6 points par round (dommages ambients)  & Immergé dans l'acide \\
        \rowcolor{CSCOLORPARTTWO!20}
        Froid          & 1 point par round (dommages ambients)   & En-dessous du gel \\
        \rowcolor{CSCOLORPARTTWO!50}
        Froid extrème   & 3 points par round (dommages ambients)  & Nitrogen liquide \\
        \rowcolor{CSCOLORPARTTWO!20}
        Choc         & 1 point par round (dommages ambients)   & Implique souvent de perdre la prochaine action \\
        \rowcolor{CSCOLORPARTTWO!50}
        Electrocution & 6 points par round (dommages ambients)  & Implique souvent de perdre la prochaine action \\
        \rowcolor{CSCOLORPARTTWO!20}
        Ecrasement         & 3 points     & Un objet ou une créature tombe sur le personnage \\
        \rowcolor{CSCOLORPARTTWO!50}
        Ecrasement important    & 6 points & Toit qui s'écroule; effondrement \\
        \rowcolor{CSCOLORPARTTWO!20}
        Collision     & 6 points  & Objet grand et rapide frappe le personnage \\
    \end{tabular}
\label{table:hazarddamage}
\end{table*}

\subsection*{Dangers dans l'Espace}

Quelques dangers spécifiques que vous pouvez inclure dans une rencontre impliquant un vaisseau spatial sont décrits ci-dessous. Ces dangers sont plus spécifiques au site que les menaces générales présentées dans le Chapitre 5 : Conflits du Futur.

\subsection*{Puits Gravitationnel}

Tous les corps dans l'espace produisent un champ gravitationnel, bien que généralement seuls les objets de la taille d'une petite lune ou plus grands représentent un danger pour un vaisseau spatial non préparé (et parfois même pour un vaisseau préparé). Plus le corps est grand, plus le champ gravitationnel associé est « profond » et étendu. Chaque fois qu'un vaisseau spatial décolle d'une lune ou d'une planète, il doit échapper au puits gravitationnel. Pour les besoins du jeu de rôle, cela représente une tâche de routine ou de faible difficulté (à supposer qu'aucune complication ne soit présente).

Les puits gravitationnels deviennent un danger lorsqu'un vaisseau les rencontre de manière inattendue --- généralement à cause d'une erreur de navigation ou de capteurs, mais parfois à cause de la présence imprévue d'une lune ou d'une source gravitationnelle extrême.

\textbf{Trajectoire en Fronde :} Une rencontre inattendue avec un puits gravitationnel peut projeter un vaisseau sur une trajectoire nouvelle et non désirée en cas d'échec à une tâche de pilotage, la difficulté étant déterminée par la situation.

\textbf{Capturé :} Une rencontre inattendue avec un puits gravitationnel peut également capturer un vaisseau dans l'orbite du puits, forçant le vaisseau à dépenser de l'énergie supplémentaire pour s'en libérer (énergie qu'il peut ne pas posséder).

\subsection*{Trou Noir}

Les trous noirs ne sont que des puits gravitationnels extrêmes. Tous les dangers associés à un puits gravitationnel s'appliquent aussi aux trous noirs. Quelques dangers supplémentaires s'y ajoutent, notamment la destruction par effet de marée (« tidal disruption »), la dilatation temporelle, et le fait d'être avalé.

\textbf{Destruction par Effet de Marée :} Mécaniquement, lorsqu'un vaisseau spatial subit les effets des forces de marée en passant trop près de l'horizon des événements d'un trou noir, toutes les tâches à bord du vaisseau sont entravées, les Règles du Vide sont en vigueur, et si une intrusion de la Meneuse est déclenchée, le vaisseau subit des dégâts importants et risque de se disloquer. Pendant ce temps, les PJ à bord (sauf si une technologie fantastique de type annulateur de gravité est utilisée) subissent 1 point de dégât ambiant à chaque round.

Un vaisseau proche d'un très grand trou noir (comme Sagittarius A*, le trou noir supermassif au centre de la Voie Lactée) peut éviter les effets de marée car le gradient gravitationnel est beaucoup plus large, mais ressentira tout de même une dilatation temporelle relativiste.

\textbf{Dilatation Temporelle Relativiste :} D'un point de vue mécanique, les vaisseaux qui survivent à une rencontre rapprochée avec un trou noir et reviennent à l'espace normal découvrent que plus de temps s'est écoulé que prévu --- cela peut aller de minutes ou heures peu importantes à des jours, mois, années, voire des siècles ou davantage.

\textbf{Au-Delà de l'Horizon des Événements :} L'horizon des événements est le point de non-retour, où même la lumière ne peut échapper à l'emprise de la gravité. Si un vaisseau tombe dans un trou noir, en supposant qu'il ne soit pas spaghettifié par les forces de marée, il est tout de même perdu pour l'univers d'origine. À moins qu'une IA post-singularité ou une ultra ancienne d'une technologie fantastique n'intervienne, il est considéré comme perdu.

\subsection*{Ceinture de Radiation / Éruption Solaire}

Les ceintures de radiation composées de particules fortement chargées piégées par les champs magnétiques autour de certaines planètes et lunes peuvent entrer en surtension, causant une exposition à la radiation. Une éruption solaire imprévue, ou la traînée d'un vaisseau massif, peut causer la même exposition inattendue.

\textbf{Dégâts au Vaisseau :} Le vaisseau subit des dégâts mineurs ou majeurs, nécessitant des réparations voire le remplacement de certaines pièces. Ces dégâts peuvent être aussi graves que nécessaire pour servir l'histoire.

\textbf{Maladie Radiative :} Lorsque les PJ sont exposés à une radiation intense, ils subissent 3 points de dégât ambiant par minute pour chaque échec à une tâche de défense de \textbf{Puissance} de difficulté 3. Si le personnage échoue trois fois à cette tâche au cours d'une même période d'exposition, il souffre d'une maladie radiative aiguë, une maladie de niveau 8 qui le fait descendre d'un cran sur le \textit{suivi des dommages} chaque jour où il échoue à un jet de défense de \textbf{Puissance}, jusqu'à sa mort.

\subsection*{Champ d'Astéroïdes / Débris}

Les films dépeignent souvent les ceintures d'astéroïdes comme des champs densément peuplés de roches en mouvement constant qu'un vaisseau doit éviter sans cesse pour ne pas entrer en collision. Ce type de lieu est rare dans le système solaire, mais peut exister dans des cadres fantastiques ou d'autres systèmes solaires.

\textbf{Évitement des Astéroïdes :} À chaque round où un vaisseau traverse un champ d'astéroïdes ou de débris dense, la pilote (ou l'intelligence du vaisseau) doit réussir une tâche de pilotage, dont la difficulté est déterminée par la situation. En cas d'échec, une collision se produit. Chaque collision endommage le vaisseau (et peut-être son équipage) selon les règles de suivi. Les collisions impliquent des rochers majeurs ou des fragments qui pénètrent les défenses du vaisseau.

\textbf{Trouver un Abri :} Le meilleur moyen de trouver un abri pour effectuer des réparations ou se cacher est de repérer un astéroïde ou un débris assez grand pour que le vaisseau puisse s'y poser ou glisser dans une crevasse. Atterrir sur un astéroïde ou un gros débris est une tâche de pilotage difficile (difficulté 5) pour correspondre à la rotation de l'objet, puis se glisser dans un espace étroit.

\subsection*{Les Effets des Dégâts Subis}

Quand un PNJ atteint 0 point de santé, il est soit mort, soit (si l'attaquante le souhaite) mis hors d'état de nuire, c'est-à-dire inconscient ou contraint à la reddition.

Comme mentionné précédemment, les dégâts issus de la plupart des sources sont appliqués à la \textbf{Réserve de Puissance} d'un personnage. Sinon, les dégâts à une stat réduisent toujours la \textbf{Réserve de stat} correspondante.

Si les dégâts réduisent la \textbf{Réserve de stat} d'un personnage à 0, tout dégât supplémentaire à cette stat (y compris les dégâts excédentaires de l'attaque ayant réduit la stat à 0) est appliqué à une autre \textbf{Réserve de stat}. Les dégâts sont appliqués dans l'ordre suivant :

\begin{enumerate}
    \item Puissance (à moins que la Réserve soit de 0)
    \item Célérité (à moins que la Réserve soit de 0)
    \item Intellect
\end{enumerate}

Même si les dommages sont appliqués à une autre Réserve de stat, ils comptent toujours comme leur type d'origine pour l'Armure et les capacités spéciales affectant les dommages. Par exemple, si un personnage avec 2 d'Armure tombe à 0 en Réserve de Puissance, puis est touché par une griffe de créature infligeant 3 points de dommages, cela compte toujours comme des dommages de Puissance. Son Armure réduit donc les dommages à 1 point, qui est alors appliqué à sa Réserve de Célérité. En d'autres termes, même si les dommages sont prélevés sur la Réserve de Célérité, cela n'ignore pas l'Armure comme le feraient normalement les dommages de Célérité.

En plus de subir des dommages à leur Réserve de Puissance, de Célérité ou d'Intellect, les PJ possèdent également un suivi des dommages. Le suivi des dommages a quatre états (du meilleur au pire) : en pleine forme, diminué, affaibli, et mort. Lorsqu'une des Réserves de stat d'un PJ atteint 0, il descend d'un cran dans le suivi des dommages. Ainsi, s'il est en pleine forme, il devient diminué. S'il est déjà diminué, il devient affaibli. S'il est déjà affaibli, il meurt.

Certains effets peuvent immédiatement faire descendre un PJ de un ou plusieurs crans dans le suivi des dommages. Cela inclut des poisons rares, des attaques de perturbation cellulaire, et des traumatismes massifs (telles que des chutes de grandes hauteurs, être renversé par un véhicule en mouvement, etc., comme déterminé par la Meneuse).

Certaines attaques, comme la morsure venimeuse d'un serpent ou l'Envoûtement d'un Emissaire, ont d'autres effets que les dommages à une Réserve de stat ou une descente dans le suivi des dommages. Ces attaques peuvent causer une perte de conscience, une paralysie, etc.

(Lorsque les PNJ (qui n'ont que des points de santé) subissent des dommages de Célérité ou d'Intellect, cela est généralement traité comme des dommages de Puissance. Cependant, la Meneuse ou le joueur peut suggérer un effet alternatif approprié---le PNJ subit une pénalité, se déplace plus lentement, est étourdi, etc.)

\subsection*{Le Suivi des Dommages}

Comme indiqué ci-dessus, le suivi des dommages comporte quatre états : en pleine forme, diminué, affaibli, et mort.

\begin{description}
    \item[En pleine forme] est l'état normal d'un personnage : les trois Réserves de stat sont à 1 ou plus, et le PJ n'a aucune pénalité liée à des conditions néfastes. Lorsqu'un PJ en pleine forme subit assez de dommages pour qu'une de ses Réserves de stat tombe à 0, il devient diminué. Notez qu'un personnage avec des Réserves de stat très basses peut encore être en pleine forme.
    \item[Diminué] est un état de blessure ou de choc. Lorsqu'un personnage diminué utilise de l'Effort, cela coûte 1 point supplémentaire par niveau appliqué. Par exemple, appliquer un niveau d'Effort coûte 4 points au lieu de 3, et deux niveaux coûtent 7 points au lieu de 5.
    Un personnage diminué ignore les effets mineurs et majeurs sur ses jets, et inflige moins de dommages supplémentaires au combat en cas de jet spécial. En combat, un jet de 17 ou plus inflige seulement 1 point de dommage supplémentaire. Lorsqu'un PJ diminué subit assez de dommages pour qu'une de ses Réserves de stat tombe à 0, il devient affaibli.
    \item[Affaibli] est un état de blessure critique. Un personnage affaibli ne peut effectuer aucune action autre que se déplacer (probablement ramper) sur une distance immédiate au maximum. Si sa Réserve de Célérité est à 0, il ne peut pas se déplacer du tout. Lorsqu'un PJ affaibli subit assez de dommages pour qu'une Réserve de stat tombe à 0, il meurt.
    \item[Mort] est mort.
\end{description}

\marginpar{ Le suivi des dommages permet de savoir à quelle distance vous êtes de la mort. Si vous êtes en pleine forme, vous êtes à trois pas de la mort. Si vous êtes diminué, à deux pas. Si vous êtes affaibli, vous êtes à un petit pas du seuil de la mort.}

\subsection*{Récupération de Points dans une Réserve}

Après avoir perdu ou dépensé des points d'une Réserve, vous les récupérez en vous reposant. Vous ne pouvez pas augmenter une Réserve au-delà de son maximum par le repos---juste revenir à son niveau normal. Les points supplémentaires sont perdus. La quantité de points récupérés et le temps de repos dépendent du nombre de fois où vous vous êtes déjà reposé ce jour-là.

Lorsque vous vous reposez, effectuez un jet de récupération. Lancez 1d6 et ajoutez votre rang. Vous récupérez ce nombre de points et vous pouvez les répartir entre vos Réserves de stat comme vous le souhaitez. Par exemple, si votre jet est 4 et que vous avez perdu 4 points de Puissance et 2 de Célérité, vous pouvez récupérer 4 points de Puissance, ou 2 de chaque, ou toute autre combinaison totalisant 4 points.

La première fois que vous vous reposez chaque jour, cela prend seulement quelques secondes. Si vous le faites en plein combat, cela prend une action pendant votre tour.

La deuxième fois, vous devez vous reposer dix minutes. La troisième fois, une heure. La quatrième fois, dix heures (généralement en fin de journée).

Après ce repos prolongé, c'est un nouveau jour. Le prochain repos prend quelques secondes, puis dix minutes, puis une heure, etc.

Si vous ne vous êtes pas encore reposé de la journée et que vous subissez beaucoup de dommages, vous pouvez d'abord vous reposer quelques secondes (1d6 + 1 point par rang), puis immédiatement dix minutes (encore 1d6 + 1 point par rang). Ainsi, en un jour entier à ne faire que se reposer, vous pouvez récupérer 4d6 points + 4 points par rang.

Chaque personnage choisit quand faire ses jets de récupération. Si un groupe de cinq PJ se repose dix minutes parce que deux veulent faire un jet de récupération, les autres ne sont pas obligés de le faire en même temps. Plus tard, les trois restants peuvent se reposer dix minutes et faire leur propre jet.

\begin{table*}[h]
    \begin{tabular}{ l l }
        \multicolumn{2}{ l }{\Large \textcolor{CSCOLORPARTTWO}{Jet de Récupération}} \\
        \textbf{Jet de Récupération} & \textbf{Temps de Récupération Nécessaire} \\ [0.5ex]
        \rowcolor{CSCOLORPARTTWO!50}
        Premier jet  & Une action \\
        \rowcolor{CSCOLORPARTTWO!20}
        Deuxième jet & Dix minutes \\
        \rowcolor{CSCOLORPARTTWO!50}
        Troisième jet  & Une heure \\
        \rowcolor{CSCOLORPARTTWO!20}
        Quatrième jet & Dix heures \\
    \end{tabular}
    \label{table:recoveryroll}
\end{table*}

\subsection*{Rétablir le Suivi des Dommages}

Utiliser des points d'un jet de récupération pour faire passer une Réserve de stat de 0 à 1 ou plus fait également remonter le personnage d'un cran dans le suivi des dommages.

Si toutes les Réserves de stat d'un PJ sont au-dessus de 0 mais que le personnage est descendu dans le suivi des dommages à cause de dommages spéciaux, il peut utiliser un jet de récupération pour remonter d'un cran dans le suivi au lieu de récupérer des points. Par exemple, un personnage affaibli par une attaque biotech de disruption cellulaire peut se reposer et passer à diminué plutôt que de récupérer des points.
%%%%%%%%%%%%%%%%%%%%%%%%%%%%%%%%%%%%%%%%%%%%%%%%%%%%%
\subsection*{Dégâts Spéciaux}

Au cours du jeu, les personnages font face à toutes sortes de menaces et de dangers qui peuvent les blesser de diverses manières, dont certaines ne sont pas facilement représentées par des points de dégâts.

\textbf{Étourdissement et Assommement :} Les personnages peuvent être étourdis lorsqu'ils sont frappés violemment à la tête, exposés à des sons extrêmement forts ou affectés par une attaque mentale. Lorsque cela se produit, pendant la durée de l'effet d'étourdissement (généralement un tour), toutes les tâches du personnage sont entravées. Des attaques similaires mais plus graves peuvent assommer les personnages. Les personnages assommés perdent leur tour (mais peuvent toujours se défendre normalement contre les attaques).

\textbf{Poison et Maladie :} Lorsque les personnages rencontrent du poison --- qu'il s'agisse du venin d'un serpent, de mort-aux-rats glissé dans un burrito, de cyanure dissous dans du vin ou d'une surdose d'acétaminophène --- ils effectuent un jet de défense de Puissance pour y résister. L'échec de la résistance peut entraîner des points de dégâts, un déplacement sur le suivi des dommages, ou un effet spécifique tel que la paralysie, l'inconscience, l'incapacité ou quelque chose de plus étrange. Par exemple, certains poisons affectent le cerveau, rendant impossible de dire certains mots, d'effectuer certaines actions, de résister à certains effets ou de récupérer des points dans une Réserve de stat.

Les maladies fonctionnent comme les poisons, mais leur effet se produit chaque jour, donc la victime doit effectuer un jet de défense de Puissance chaque jour ou subir les effets. Les effets des maladies sont aussi variés que ceux des poisons : points de dégâts, déplacement sur le suivi des dommages, incapacité, etc. De nombreuses maladies infligent des dégâts qui ne peuvent pas être restaurés par des moyens conventionnels.

\textbf{Paralysie :} Les effets paralysants font tomber un personnage au sol, incapable de bouger. Sauf indication contraire, le personnage peut toujours effectuer des actions qui ne nécessitent aucun mouvement physique.

\textbf{Autres Effets :} D'autres effets spéciaux peuvent rendre un personnage aveugle ou sourd, incapable de se tenir debout sans tomber, ou incapable de respirer. Des effets plus étranges peuvent annuler la gravité pour le personnage (ou l'augmenter d'un facteur cent), le transporter ailleurs, le rendre hors phase, muter sa forme physique, implanter de faux souvenirs ou sens, altérer la façon dont son cerveau traite les informations, ou enflammer ses nerfs de sorte qu'il soit dans une douleur constante et atroce. Chaque effet spécial doit être traité au cas par cas. La Meneuse décide comment le personnage est affecté et comment la condition peut être soulagée (si possible).

\subsection*{PNJ et Dégâts Spéciaux}

La Meneuse a toujours le dernier mot sur les dégâts spéciaux qui affecteront un PNJ. Les PNJ humains réagissent généralement comme les personnages, mais les créatures non humaines peuvent réagir très différemment. Par exemple, une petite quantité de venin est peu susceptible de blesser un dragon gigantesque, et elle n'affectera pas du tout un androïde ou un démon.

Si un PNJ est sensible à une attaque qui déplacerait un personnage sur le suivi des dommages, utiliser cette attaque sur le PNJ le rend généralement inconscient ou mort. Alternativement, la Meneuse pourrait appliquer la condition affaiblie au PNJ, avec le même effet que sur un PJ.

\subsection*{Modificateurs d'Attaque et Situations Spéciales}

Dans les situations de combat, de nombreux modificateurs peuvent entrer en jeu. Bien que la Meneuse soit libre d'évaluer les modificateurs qu'elle juge appropriés à la situation (c'est son rôle dans le jeu), les suggestions et directives suivantes peuvent faciliter cela. Souvent, le modificateur est appliqué comme un palier de difficulté. Ainsi, si une situation entrave les attaques, cela signifie que si un PJ attaque un PNJ, la difficulté du jet d'attaque est augmentée d'un palier, et si un PNJ attaque un PJ, la difficulté du jet de défense est diminuée d'un palier. Cela est dû au fait que les joueurs effectuent tous les jets, qu'ils attaquent ou se défendent --- les PNJ n'effectuent jamais de jets d'attaque ou de défense.

En cas de doute, si cela semble plus difficile d'attaquer dans une situation, entravez les jets d'attaque. Si cela semble que les attaques devraient obtenir un avantage ou être plus faciles d'une certaine manière, entravez les jets de défense.

(Les portées précises ne sont pas importantes dans le système Cypher. Les portées définies de manière large "immédiate", "courte", "longue" et "très longue" permettent à la Meneuse de prendre rapidement une décision et de faire avancer les choses. En gros, l'idée est : votre cible est juste là, votre cible est proche, votre cible est assez loin, ou votre cible est extrêmement loin.)

\subsubsection*{Couverture}

Si un personnage est derrière une couverture de sorte qu'une partie significative de son corps est derrière quelque chose de solide, les attaques contre le personnage sont entravées.

Si un personnage est entièrement derrière une couverture (tout son corps est derrière quelque chose de solide), il ne peut pas être attaqué à moins que l'attaque puisse traverser la couverture. Par exemple, si un personnage se cache derrière un écran en bois mince et que son adversaire tire sur l'écran avec un fusil capable de pénétrer le bois, le personnage peut être attaqué. Cependant, comme l'attaquant ne voit pas clairement le personnage, cela compte toujours comme une couverture (les attaques contre le personnage sont entravées).

\subsubsection*{Position}

Parfois, l'endroit où se tient un personnage lui donne un avantage ou un désavantage.

\textbf{Cible à Terre :} En mêlée, une cible à terre est plus facile à toucher (les attaques contre elle sont facilitées). En combat à distance, une cible à terre est plus difficile à toucher (les attaques contre elle sont entravées).

\textbf{Terrain Élevé :} En combat à distance ou en mêlée, les attaques d'un adversaire en terrain élevé sont facilitées.

\subsubsection*{Surprise}

Lorsqu'une cible n'est pas consciente d'une attaque imminente, l'attaquant a un avantage. Un tireur d'élite à distance dans une position cachée, un assaillant invisible ou la première salve d'une embuscade réussie sont tous facilités de deux paliers. Cependant, pour que l'attaquant obtienne cet avantage, le défenseur ne doit vraiment avoir aucune idée que l'attaque arrive.

Si le défenseur n'est pas sûr de l'emplacement de l'attaquant mais reste sur ses gardes, les attaques sont facilitées d'un seul palier.

\subsubsection*{Portée}

En mêlée, vous pouvez attaquer un ennemi qui est adjacent à vous (à côté de vous) ou à portée (portée immédiate). Si vous entrez en mêlée avec un ou plusieurs ennemis, généralement vous pouvez attaquer la plupart ou tous les combattants, ce qui signifie qu'ils sont à côté de vous, à portée ou à portée si vous bougez légèrement ou si vous avez une arme longue qui étend votre portée.

La majorité des attaques à distance n'ont que deux portées : courte portée et longue portée (quelques-unes ont une portée très longue). La courte portée est généralement inférieure à 50 pieds (15 m) environ. La longue portée est généralement de 50 pieds (15 m) à environ 100 pieds (30 m). La portée très longue est généralement de 100 pieds (30 m) à 500 pieds (150 m). Une précision supérieure à cela n'est pas importante dans le système Cypher. Si quelque chose est plus long que la portée très longue, la portée exacte est généralement précisée, comme avec un objet qui peut tirer un faisceau à 1 000 pieds (300 m) ou vous téléporter jusqu'à 1 mile (1,5 km).

Ainsi, le jeu a quatre mesures de distance : immédiate, courte, longue et très longue. Celles-ci s'appliquent également au mouvement. Quelques cas particuliers --- portée à bout portant et portée extrême --- modifient la chance de réussite d'une attaque.

\textbf{Portée à Bout Portant :} Si un personnage utilise une arme à distance contre une cible à portée immédiate, l'attaque est facilitée.

\textbf{Portée Extrême :} Les cibles juste à la limite de la portée d'une arme sont à portée extrême. Les attaques contre de telles cibles sont entravées.

\begin{tcolorbox}
La Meneuse peut permettre à un personnage avec une arme à distance d'attaquer au-delà de la portée extrême, mais l'attaque serait entravée de deux paliers pour chaque catégorie de portée au-delà de la limite normale. Les attaques avec des limites strictes, comme le rayon d'explosion d'une bombe, ne peuvent pas être modifiées.
\end{tcolorbox}

\begin{tcolorbox}
Dans certaines situations, comme un PJ en haut d'un bâtiment regardant à travers un champ ouvert, la Meneuse devrait permettre aux attaques à distance de dépasser leur portée maximale. Par exemple, dans des conditions parfaites, un bon archer peut toucher une grande cible avec un arc et une flèche à 500 pieds (150 m), bien plus loin que la portée longue typique d'un arc.
\end{tcolorbox}

\subsubsection*{Éclairage}

Ce que les personnages peuvent voir (et à quel point ils peuvent bien voir) joue un rôle énorme dans le combat.

\textbf{Lumière Tamisée :} La lumière tamisée est approximativement la quantité de lumière par une nuit de pleine lune brillante ou l'éclairage fourni par une torche, une lampe de poche ou une lampe de bureau. La lumière tamisée permet de voir jusqu'à courte portée. Les cibles en lumière tamisée sont plus difficiles à toucher. Les attaques contre de telles cibles sont entravées. Les attaquants formés à la détection en faible luminosité annulent ce modificateur.

\textbf{Lumière Très Tamisée :} La lumière très tamisée est approximativement la quantité de lumière par une nuit étoilée sans lune visible, ou la lueur fournie par une bougie ou un panneau de contrôle éclairé. La lumière très tamisée permet de voir clairement seulement à portée immédiate et de percevoir des formes vagues à courte portée. Les cibles en lumière très tamisée sont plus difficiles à toucher. Les attaques contre des cibles à portée immédiate sont entravées, et les attaques contre celles à courte portée sont entravées de deux paliers. Les attaquants formés à la détection en faible luminosité modifient ces difficultés d'un palier en leur faveur. Les attaquants spécialisés en détection en faible luminosité modifient ces difficultés de deux paliers en leur faveur.

\textbf{Obscurité :} L'obscurité est une zone sans éclairage du tout, comme une nuit sans lune avec couverture nuageuse ou une pièce sans lumières. Les cibles dans l'obscurité totale sont presque impossibles à toucher. Si un attaquant peut utiliser d'autres sens (comme l'ouïe) pour avoir une idée de l'endroit où l'adversaire pourrait se trouver, les attaques contre de telles cibles sont entravées de quatre paliers. Sinon, les attaques dans l'obscurité totale échouent sans avoir besoin d'un jet à moins que le joueur ne dépense 1 PX pour "faire un tir chanceux" ou que la Meneuse n'utilise une intrusion de Meneuse. Les attaquants formés à la détection en faible luminosité facilitent la tâche. Les attaquants spécialisés en détection en faible luminosité facilitent la tâche de deux paliers.

\subsubsection*{Visibilité}

Similaire à l'éclairage, les facteurs qui obscurcissent la vision affectent le combat.

\textbf{Brouillard :} Une cible dans le brouillard est similaire à une cible en lumière tamisée. Les attaques à distance contre de telles cibles sont entravées. Un brouillard particulièrement dense rend les attaques à distance presque impossibles (traiter comme l'obscurité), et même les attaques en mêlée sont entravées.

\textbf{Cible Cachée :} Une cible dans une végétation dense, derrière un écran ou rampant parmi les décombres dans une ruine est difficile à toucher parce qu'elle est difficile à voir. Les attaques à distance contre de telles cibles sont entravées.

\textbf{Cible Invisible :} Si un attaquant peut utiliser d'autres sens (comme l'ouïe) pour avoir une idée de l'endroit où l'adversaire pourrait se trouver, les attaques contre de telles cibles sont entravées de quatre paliers. Sinon, les attaques contre une créature invisible échouent sans avoir besoin d'un jet à moins que le joueur ne dépense 1 PX pour "faire un tir chanceux" ou que la Meneuse n'utilise une intrusion de Meneuse.

\subsubsection*{Eau}

Être dans l'eau peu profonde peut rendre difficile le déplacement, mais cela n'affecte pas le combat. Être dans l'eau profonde peut rendre les choses difficiles, et être complètement sous l'eau peut sembler aussi différent que d'être sur une autre planète.

\textbf{Eau Profonde :} Être dans l'eau jusqu'à la poitrine (ou l'équivalent) entrave vos attaques. Les créatures aquatiques ignorent ce modificateur.

\textbf{Combat en Mêlée Sous l'Eau :} Pour les créatures non aquatiques, être complètement sous l'eau rend l'attaque très difficile. Les attaques avec des armes perforantes sont entravées, et les attaques en mêlée avec des armes tranchantes ou contondantes sont entravées de deux paliers. Les créatures aquatiques ignorent ces pénalités.

\textbf{Combat à Distance Sous l'Eau :} Comme pour le combat en mêlée, les créatures non aquatiques ont des problèmes pour se battre sous l'eau. Certaines attaques à distance sont impossibles sous l'eau --- vous ne pouvez pas lancer des objets, tirer à l'arc ou à l'arbalète, ou utiliser une sarbacane. De nombreuses armes à feu ne fonctionnent également pas sous l'eau. Les attaques avec des armes qui fonctionnent sous l'eau sont entravées. Les portées sous l'eau sont réduites d'une catégorie ; les armes à très longue portée fonctionnent seulement à longue portée, les armes à longue portée fonctionnent seulement à courte portée, et les armes à courte portée fonctionnent seulement à portée immédiate.

%%%%%%%%%%%%%%%%%%%%%%%%%%%%%%%%%%%%%%%%%%%%%%%%%%%%%%
\subsubsection*{Cibles en Mouvement}

Les cibles en mouvement sont plus difficiles à toucher, et les attaquants en mouvement ont également du mal.

\textbf{La Cible est en Mouvement :} Les attaquants essayant de toucher un ennemi qui se déplace très rapidement sont entravés. (Un ennemi se déplaçant très rapidement est quelqu'un qui ne fait que courir, est monté sur une créature en mouvement, est à bord d'un véhicule ou d'un moyen de transport en mouvement, etc.)

\textbf{L'Attaquant est en Mouvement :} Un attaquant essayant de faire une attaque tout en se déplaçant par ses propres moyens (marcher, courir, nager, etc.) ne subit aucune pénalité. Les attaques depuis une monture en mouvement ou un véhicule en mouvement sont entravées ; un attaquant formé à la monte ou à la conduite ignore cette pénalité.

\textbf{L'Attaquant est Bousculé :} Être bousculé, comme se tenir sur un navire qui tangue ou une plateforme vibrante, rend l'attaque difficile. De telles attaques sont entravées. Les personnages formés à l'équilibre ou à la navigation ignoreraient les pénalités pour être sur un navire.

\subsection*{SITUATION SPÉCIALE : COMBAT ENTRE PNJ}

Lorsqu'un PNJ allié des PJ attaque un autre PNJ, la Meneuse peut désigner un joueur pour faire le jet et le gérer comme si un PJ attaquait. Souvent, le choix est évident. Par exemple, un personnage qui a un animal d'attaque dressé doit faire le jet lorsque son animal attaque des ennemis. Si un allié PNJ accompagnant le groupe se lance dans la mêlée, l'allié choisit son PJ préféré pour faire le jet à sa place. Les PNJ ne peuvent pas appliquer d'Effort. Bien sûr, il est tout à fait approprié (et plus facile) que l'allié PNJ utilise les règles d'action coopérative pour aider un PJ au lieu de faire des attaques directes, ou de comparer les niveaux des deux PNJ (le plus élevé gagne).

\subsection*{SITUATION SPÉCIALE : COMBAT ENTRE PJ}

Lorsqu'un PJ attaque un autre PJ, le personnage attaquant fait un jet d'attaque, et l'autre personnage fait un jet de défense, en ajoutant les modificateurs appropriés. Si le PJ attaquant a une compétence, une capacité, un atout ou un autre effet qui faciliterait l'attaque si elle était faite contre un PNJ, le personnage ajoute 3 au jet pour chaque réduction de palier (+3 pour un palier, +6 pour deux paliers, etc.). Si le résultat final de l'attaquant est plus élevé, l'attaque touche. Si le résultat du défenseur est plus élevé, l'attaque rate. Les dégâts sont résolus normalement. La Meneuse médiatise tous les effets spéciaux.

\subsection*{SITUATION SPÉCIALE : ATTAQUES DE ZONE}

Parfois, une attaque ou un effet affecte une zone plutôt qu'une seule cible. Par exemple, une grenade ou un glissement de terrain peut potentiellement blesser ou affecter tout le monde dans la zone.

Dans une attaque de zone, tous les PJ dans la zone font des jets de défense appropriés contre l'attaque pour déterminer son effet sur eux. S'il y a des PNJ dans la zone, l'attaquant fait un seul jet d'attaque contre eux tous (un seul jet, pas un jet par PNJ) et le compare au nombre cible de chaque PNJ. Si le jet est égal ou supérieur au nombre cible d'un PNJ particulier, l'attaque touche ce PNJ.

Certaines attaques de zone infligent toujours au moins un minimum de dégâts, même si les attaques ratent ou si un PJ réussit son jet de défense.

Par exemple, considérons un personnage qui utilise Briser pour attaquer six sectateurs (niveau 2 ; nombre cible 6) et leur chef (niveau 4 ; nombre cible 12). Le PJ applique de l'Effort pour augmenter les dégâts et obtient un 11 pour le jet d'attaque. Cela touche les six sectateurs, mais pas le chef, donc la capacité inflige 3 points de dégâts à chacun des sectateurs. La description de Briser dit qu'appliquer de l'Effort pour augmenter les dégâts signifie également que les cibles subissent 1 point de dégâts si le PJ rate le jet d'attaque, donc le chef subit 1 point de dégâts. En termes de ce qui se passe dans l'histoire, les sectateurs sont pris au dépourvu par la soudaine détonation de l'un de leurs couteaux, mais le chef esquive et est protégé de l'explosion. Malgré les mouvements rapides du chef, l'explosion est si intense que quelques morceaux de métal le coupent.

\subsection*{SITUATION SPÉCIALE : ATTAQUER DES OBJETS}

Attaquer un objet n'est presque jamais une question de le toucher. Bien sûr, vous pouvez toucher le côté large d'une grange, mais pouvez-vous l'endommager ? Attaquer des objets inanimés avec une arme de mêlée est une action de Puissance. Les objets ont des niveaux et donc des nombres cibles. Les objets ont une piste de dégâts qui fonctionne comme la piste de dégâts pour les PJ.

\textbf{Intact} est l'état par défaut d'un objet.

\textbf{Dégâts mineurs} est un état légèrement endommagé. Un objet avec des dégâts mineurs réduit son niveau de 1.

\textbf{Dégâts majeurs} est un état critique endommagé. Un objet avec des dégâts majeurs est cassé et ne fonctionne plus.

\textbf{Détruit} est détruit. L'objet est ruiné, ne fonctionne plus et ne peut pas être réparé.

Si l'action de Puissance pour endommager un objet est un succès, l'objet descend d'un cran sur la piste de dégâts de l'objet. Si le jet de Puissance dépasse la difficulté de 2 niveaux, l'objet descend de deux crans sur la piste de dégâts de l'objet. Si le jet de Puissance dépasse la difficulté de 4 niveaux, l'objet descend de trois crans sur la piste de dégâts de l'objet. Les objets avec des dégâts mineurs ou majeurs peuvent être réparés, les faisant remonter d'un ou plusieurs crans sur la piste de dégâts de l'objet.

Les objets fragiles ou cassants, comme le papier ou le verre, diminuent le niveau effectif de l'objet pour déterminer s'il est endommagé. Les objets durs, comme ceux en bois ou en pierre, ajoutent 1 au niveau effectif. Les objets très durs, comme ceux en métal, ajoutent 2. (La Meneuse peut décider que certains matériaux exotiques ajoutent 3.)

L'outil ou l'arme utilisée pour attaquer l'objet doit être au moins aussi dur que l'objet lui-même. De plus, si la quantité de dégâts que l'attaque pourrait infliger --- non modifiée par un jet de dé spécial --- n'égale pas ou ne dépasse pas le niveau effectif de l'objet, l'attaque ne peut pas endommager l'objet peu importe le jet.

\section*{Action : Activer une Capacité Spéciale}

Les capacités spéciales sont accordées par des focalisations, des types et des saveurs, ou fournies par des cyphers ou d'autres dispositifs. Si une capacité spéciale affecte un autre personnage de quelque manière indésirable, elle est gérée comme une attaque. Cela est vrai même si la capacité n'est normalement pas considérée comme une attaque. Par exemple, si un personnage a un toucher guérisseur, et que son ami ne veut pas être guéri pour une raison quelconque, une tentative de guérir son ami réticent est gérée comme une attaque.

De nombreuses capacités spéciales n'affectent pas un autre personnage de manière indésirable. Par exemple, un PJ pourrait utiliser Flottement sur lui-même pour flotter dans les airs. Un personnage avec un dispositif de réorganisation de la matière pourrait transformer un mur de pierre en verre. Un personnage qui active un cypher de changement de phase pourrait traverser un mur. Aucune de ces actions ne nécessite de jet d'attaque (bien que lors de la transformation d'un mur de pierre en verre, le personnage doive toujours faire un jet pour affecter le mur avec succès).

Si le personnage dépense des points pour appliquer de l'Effort à la tentative, il pourrait vouloir faire un jet de toute façon pour voir s'il obtient un effet majeur, ce qui réduirait le coût de son action.

\section*{Action : Se Déplacer}

Dans le cadre d'une autre action, un personnage peut ajuster sa position --- reculer de quelques pieds tout en utilisant une capacité, glisser en combat pour affronter un autre adversaire afin d'aider un ami, pousser à travers une porte qu'il vient d'ouvrir, etc. Cela est considéré comme une distance immédiate, et un personnage peut se déplacer de cette distance dans le cadre d'une autre action.

Dans une situation de combat, si un personnage est dans une grande mêlée, il est généralement considéré comme étant à côté de la plupart des autres combattants, sauf si la Meneuse décide qu'il est plus loin parce que la mêlée est particulièrement grande ou que la situation le dicte.

S'ils ne sont pas en mêlée mais toujours à proximité, ils sont considérés comme étant à une courte distance --- généralement moins de 50 pieds (15 m). S'ils sont plus loin que cela mais toujours impliqués dans le combat, ils sont considérés comme étant à une longue distance, généralement de 50 à 100 pieds (15 à 30 m), ou éventuellement même à une très longue distance, généralement plus de 100 pieds à 500 pieds (30 à 150 m).

Dans un tour, comme action, un personnage peut faire un court déplacement. Dans ce cas, il ne fait rien d'autre que se déplacer jusqu'à environ 50 pieds (15 m). Certains terrains ou situations changeront la distance qu'un personnage peut parcourir, mais généralement, faire un court déplacement est considéré comme une action de difficulté 0. Aucun jet n'est nécessaire ; ils arrivent simplement à destination comme leur action.

Un personnage peut essayer de faire un long déplacement --- jusqu'à 100 pieds (30 m) ou plus --- en un tour. C'est une tâche de Célérité avec une difficulté de 4. Comme pour toute action, ils peuvent utiliser des compétences, des atouts ou de l'Effort pour faciliter la tâche. Le terrain, les obstacles ou d'autres circonstances peuvent entraver la tâche. Un jet réussi signifie que le personnage a parcouru la distance en toute sécurité. Un échec signifie qu'à un moment donné pendant le déplacement, ils s'arrêtent ou trébuchent (la Meneuse détermine où cela se produit).

Un personnage peut également essayer de faire un court déplacement et de prendre une autre action physique (relativement simple), comme faire une attaque. Comme pour la tentative de faire un long déplacement, c'est une tâche de Célérité avec une difficulté de 4, et un échec signifie que le personnage s'arrête à un moment donné, glisse ou trébuche ou est autrement retardé.

\subsection*{Déplacement à Long Terme}

Lorsqu'on parle de déplacement en termes de voyage plutôt que d'action tour par tour, les personnages typiques peuvent parcourir sur une route environ 20 miles (32 km) par jour, en moyenne environ 3 miles (5 km) par heure, y compris quelques arrêts. Lorsqu'ils voyagent en terrain découvert, ils peuvent parcourir environ 12 miles (19 km) par jour, en moyenne 2 miles (3 km) par heure, à nouveau avec quelques arrêts. Les personnages montés, comme ceux à cheval, peuvent aller deux fois plus loin. D'autres modes de déplacement (voitures, avions, aéroglisseurs, voiliers, etc.) ont leurs propres taux de déplacement.

\subsection*{Modificateurs de Déplacement}

Différents environnements affectent le déplacement de différentes manières.

\textbf{Terrain Accidenté :} Une surface considérée comme un terrain accidenté est couverte de pierres ou d'autres matériaux en vrac, inégale ou avec un appui incertain, instable, ou une surface qui nécessite un déplacement à travers un espace étroit, comme un couloir exigu ou une corniche étroite. Les escaliers sont également considérés comme un terrain accidenté. Le terrain accidenté ne ralentit pas le déplacement normal tour par tour, mais entrave les jets de déplacement. Le terrain accidenté réduit de moitié les taux de déplacement à long terme.

\textbf{Terrain Difficile :} Un terrain difficile est une zone remplie d'obstacles délicats --- de l'eau jusqu'à la taille, une pente très raide, une corniche particulièrement étroite, de la glace glissante, un pied ou plus de neige, un espace si petit qu'il faut ramper à travers, etc. Le terrain difficile entrave les jets de déplacement et réduit de moitié le déplacement tour par tour. Cela signifie qu'un court déplacement est d'environ 25 pieds (8 m), et un long déplacement est d'environ 50 pieds (15 m). Le terrain difficile réduit le déplacement à long terme à un tiers de son taux normal.

\textbf{Eau :} L'eau profonde, dans laquelle un personnage est principalement ou entièrement submergé, entrave les jets de déplacement et réduit le déplacement tour par tour et à long terme à un quart de son taux normal. Cela signifie qu'un court déplacement est d'environ 12 pieds (4 m), et un long déplacement est d'environ 25 pieds (7,5 m). Les personnages formés à la natation réduisent de moitié leur déplacement uniquement lorsqu'ils sont dans l'eau profonde.

\textbf{Faible Gravité :} Le déplacement en faible gravité est plus facile mais pas beaucoup plus rapide. Tous les jets de déplacement sont facilités.

\textbf{Gravité Élevée :} Dans un environnement à gravité élevée, traitez tous les personnages en déplacement comme s'ils étaient sur un terrain difficile. Les personnages formés aux manœuvres en gravité élevée annulent cette pénalité. La gravité élevée réduit le déplacement à long terme à un tiers de son taux normal.

\textbf{Gravité Zéro :} Dans un environnement sans gravité, les personnages ne peuvent pas se déplacer normalement. Au lieu de cela, ils doivent se propulser depuis une surface et réussir un jet de Puissance pour se déplacer (la difficulté est égale à un quart de la distance parcourue en pieds). Sans surface pour se propulser, un personnage ne peut pas se déplacer. À moins que le déplacement du personnage ne l'amène à un objet stable qu'il peut attraper ou contre lequel il peut atterrir, il continue à dériver dans cette direction à chaque tour, parcourant la moitié de la distance de la poussée initiale.

\subsection*{SITUATION SPÉCIALE : UNE POURSUITE}

Lorsqu'un PJ poursuit un PNJ ou vice versa, le joueur doit tenter une action de Célérité, avec la difficulté basée sur le niveau du PNJ. Si le PJ réussit le jet, il attrape le PNJ (s'il poursuit), ou il s'échappe (s'il est poursuivi). En termes d'histoire, cette mécanique à un seul jet peut être le résultat d'une longue poursuite sur plusieurs tours.

Alternativement, si la Meneuse souhaite jouer une longue poursuite, le personnage peut faire plusieurs jets (peut-être un par niveau du PNJ) pour terminer la poursuite avec succès. Pour chaque échec, le PJ doit faire un autre succès, et s'il a plus d'échecs que de succès, le PJ échoue à attraper le PNJ (s'il poursuit) ou est attrapé (s'il est poursuivi). Comme pour le combat, la Meneuse est encouragée à décrire les résultats de ces jets avec des détails. Un succès pourrait signifier que le PJ a tourné à un coin et gagné un peu de distance. Un échec pourrait signifier qu'un panier de fruits se renverse devant lui, le ralentissant. Les poursuites en véhicule sont gérées de manière similaire.

\section*{Action : Attendre}

Vous pouvez attendre de réagir à l'action d'un autre personnage.

Vous décidez quelle action déclenchera votre action, et si l'action déclencheuse se produit, vous pouvez prendre votre action en premier (sauf si le fait de passer en premier n'aurait pas de sens, comme attaquer un ennemi avant qu'il n'entre dans votre champ de vision). Par exemple, si un orc vous menace avec une hallebarde, à votre tour, vous pouvez décider d'attendre, en disant "S'il me poignarde, je vais le trancher avec mon épée." Au tour de l'orc, il poignarde, alors vous faites votre attaque à l'épée avant que cela n'arrive.

Attendre est également un bon moyen de gérer un attaquant à distance qui se lève de derrière une couverture, tire une attaque, puis se baisse à nouveau. Vous pourriez dire "J'attends de le voir surgir de derrière la couverture, puis je lui tire dessus."

(Attendre est également un outil utile pour les actions coopératives (voir ci-dessous).)

\section*{Action : Défendre}

Défendre est une action spéciale que seuls les PJ peuvent faire, et seulement en réponse à une attaque. En d'autres termes, un PNJ utilise son action pour attaquer, ce qui force un PJ à faire un jet de défense. Cela est géré comme tout autre type d'action, avec les circonstances, la compétence, les atouts et l'Effort pouvant tous potentiellement entrer en jeu. Défendre est un type spécial d'action en ce sens qu'il ne se produit pas au tour du PJ. Ce n'est jamais une action qu'un joueur décide de prendre ; c'est toujours une réaction à une attaque. Un PJ peut prendre une action de défense lorsqu'il est attaqué (au tour du PNJ attaquant) et toujours prendre une autre action à son propre tour.

Le type de jet de défense dépend du type d'attaque. Si un ennemi attaque un personnage avec une hache, ils peuvent utiliser la Célérité pour esquiver ou bloquer avec ce qu'ils tiennent. S'ils sont frappés par un dard empoisonné, ils peuvent utiliser une action de Puissance pour résister à ses effets. Si un ver psi tente de contrôler leur esprit, ils peuvent utiliser l'Intellect pour repousser l'intrusion.

Parfois, une attaque provoque deux actions de défense. Par exemple, un reptile venimeux essaie de mordre un PJ. Ils essaient d'esquiver la morsure avec une action de Célérité. S'ils échouent, ils subissent des dégâts de la morsure, et ils doivent également tenter une action de Puissance pour résister aux effets du poison.

Si un personnage ne sait pas qu'une attaque arrive, généralement, il peut toujours faire un jet de défense, mais il ne peut pas ajouter de modificateurs (y compris le modificateur d'un bouclier), et il ne peut pas utiliser de compétence ou d'Effort pour faciliter la tâche. Si les circonstances le justifient --- comme si l'attaquant est juste à côté du personnage --- la Meneuse peut décider que l'attaque surprise touche simplement.

Un personnage peut toujours choisir de renoncer à une action de défense, auquel cas l'attaque touche automatiquement.

Certaines capacités (comme la capacité spéciale Contremesures) peuvent vous permettre de faire quelque chose de spécial comme action de défense.

%%%%%%%%%%%%%%%%%%%%%%%%%%%%%%%%%%%%%%%%%%%%%%%%%%%
\section*{Action : Faire Autre Chose}

Les joueurs peuvent essayer tout ce qu'ils peuvent imaginer, bien que cela ne signifie pas que tout est possible. La Meneuse fixe la difficulté --- c'est son rôle principal dans le jeu. Cependant, guidés par les limites de la logique, les joueurs et la Meneuse trouveront toutes sortes d'actions et d'options qui ne sont pas couvertes par une règle. C'est une bonne chose.

Les joueurs ne doivent pas se sentir contraints par les mécaniques de jeu lorsqu'ils entreprennent des actions. Les compétences ne sont pas nécessaires pour tenter une action. Quelqu'un qui n'a jamais crocheté une serrure peut toujours essayer. La Meneuse peut rendre la tâche plus difficile, mais le personnage peut toujours tenter l'action.

Ainsi, les joueurs et la Meneuse peuvent revenir au début de ce chapitre et regarder l'expression la plus basique des règles. Un joueur veut entreprendre une action. La Meneuse décide, sur une échelle de 1 à 10, à quel point cette tâche est difficile et quelle stat utiliser. Le joueur détermine s'il a quelque chose qui pourrait modifier la difficulté et envisage d'appliquer de l'Effort. Une fois la détermination finale faite, il lance les dés pour voir si son personnage réussit. C'est aussi simple que cela.

Pour plus de conseils, voici quelques-unes des actions les plus courantes qu'un joueur pourrait entreprendre.

(Les joueurs sont encouragés à trouver leurs propres idées pour les actions de leurs personnages plutôt que de consulter une liste d'actions possibles. C'est pourquoi il y a une action "faire autre chose". Les PJ ne sont pas des pions sur un plateau de jeu --- ce sont des personnes dans une histoire. Et comme des personnes réelles, ils peuvent essayer tout ce qu'ils peuvent imaginer. (Réussir est une autre affaire.) Le système de difficulté des tâches fournit à la Meneuse les outils nécessaires pour arbitrer tout ce que les joueurs peuvent imaginer.)

\subsection*{Escalade}

Lorsqu'un personnage escalade, la Meneuse fixe une difficulté basée sur la surface escaladée. L'escalade est comme se déplacer sur un terrain difficile : le jet de déplacement est entravé et le mouvement est à demi-vitesse. Des circonstances inhabituelles, comme escalader sous le feu, imposent des pénalités supplémentaires.

\begin{table*}[h]
    %\hspace*{-40pt}
    %\centering
        \begin{tabular}{ l l }
        \multicolumn{2}{ l }{\Large \textcolor{CSCOLORPARTTWO}{Difficulté d'escalade}} \\
        \textbf{Difficulté} & \textbf{Surface} \\ [0.5ex]
        \rowcolor{CSCOLORPARTTWO!50}
    2          & Surface avec beaucoup de prises \\
    \rowcolor{CSCOLORPARTTWO!20}
    3          & Mur de pierre ou surface similaire (quelques prises) \\
    \rowcolor{CSCOLORPARTTWO!50}
    4          & Surface qui s'effrite ou glissante \\
    \rowcolor{CSCOLORPARTTWO!20}
    5          & Mur de pierre lisse ou surface similaire \\
    \rowcolor{CSCOLORPARTTWO!50}
    6          & Mur métallique ou surface similaire \\
    \rowcolor{CSCOLORPARTTWO!20}
    8          & Surface lisse, horizontale (grimpeur à l'envers) \\
    \rowcolor{CSCOLORPARTTWO!50}
    10         & Mur de verre ou surface similaire \\
\end{tabular}
\label{table:climbingdifficulty}
\end{table*}

\subsection*{Actions Coopératives}

Il existe de nombreuses façons pour plusieurs personnages de travailler ensemble. Aucune de ces options, cependant, ne peut être utilisée en même temps par les mêmes personnages.

\textbf{Aider :} Si vous utilisez votre action pour aider quelqu'un avec une tâche, vous facilitez la tâche. Si vous avez une incapacité dans une tâche, votre aide n'a aucun effet. Si vous utilisez votre action pour aider quelqu'un avec une tâche pour laquelle vous êtes formé ou spécialisé, la tâche est facilitée de deux niveaux. L'aide est considérée comme un atout, et quelqu'un recevant de l'aide ne peut généralement pas obtenir plus de deux atouts sur une seule tâche si cette aide est fournie par un autre personnage.

Par exemple, si Scott essaie d'escalader une pente raide et que Sarah (qui est formée à l'escalade) passe son tour à l'aider, la tâche de Scott est facilitée de deux niveaux.

Parfois, vous pouvez aider en effectuant une tâche qui complète ce qu'un autre essaie de faire. Si votre action complémentaire réussit, vous facilitez la tâche de l'autre personne. Par exemple, si Scott essaie de persuader un capitaine de navire de le laisser monter à bord, Sarah pourrait essayer de compléter les paroles de Scott avec un mensonge flatteur sur le capitaine (une action de tromperie), une démonstration de connaissances sur la région où le navire se dirige (une action de géographie), ou une menace directe au capitaine (une action d'intimidation). Si le jet de Sarah est un succès, la tâche de persuasion de Scott est facilitée.

\textbf{Distraction :} Lorsqu'un personnage utilise son tour pour distraire un ennemi, les attaques de cet ennemi sont entravées pendant un tour. Plusieurs personnages distrayant un ennemi n'ont pas plus d'effet qu'un seul personnage le faisant --- un ennemi est soit distrait, soit ne l'est pas. Une distraction pourrait être de crier un défi, de tirer un coup de semonce, ou une activité similaire qui ne blesse pas l'ennemi.

\textbf{Attirer l'Attaque :} Lorsqu'un PNJ attaque un personnage, un autre PJ peut se présenter de manière visible, crier des insultes, et se déplacer pour essayer de faire attaquer l'ennemi à sa place. Dans la plupart des cas, cette action réussit sans jet --- l'opposant attaque le PJ visible au lieu de ses compagnons. Dans d'autres cas, comme avec des ennemis intelligents ou déterminés, le personnage visible doit réussir une action d'Intellect pour attirer l'attaque. Si cette action d'Intellect réussit, l'ennemi attaque le personnage visible, dont les défenses sont entravées de deux niveaux. Deux personnages essayant d'attirer une attaque en même temps s'annulent mutuellement.

\begin{tcolorbox}
Deux personnages essayant d'attirer une attaque en même temps s'annulent mutuellement.
\end{tcolorbox}

\textbf{Prendre l'Attaque :} Un personnage peut utiliser son action pour se jeter devant une attaque réussie d'un ennemi pour sauver un camarade proche. L'attaque réussit automatiquement contre le personnage sacrificiel, et inflige 1 point de dégât supplémentaire. Un personnage ne peut pas volontairement prendre plus d'une attaque de cette manière par tour.

\subsection*{Artisanat, Construction et Réparation}

L'artisanat est un sujet délicat dans le système Cypher car les mêmes règles qui régissent la fabrication d'une lance couvrent également la réparation d'une machine pouvant vous emmener dans l'hyperespace. Normalement, le niveau de l'objet détermine la difficulté de sa création ou de sa réparation ainsi que le temps requis. Pour les cyphers, artefacts, autres objets nécessitant des connaissances spécialisées, ou objets uniques à un monde ou une espèce autre que la vôtre (comme un marcheur trépied martien), ajoutez 5 au niveau de l'objet pour déterminer la difficulté de sa fabrication ou de sa réparation.

Parfois, si l'objet est de nature artistique, la Meneuse augmentera la difficulté et le temps requis. Par exemple, un tabouret en bois brut pourrait être assemblé en une heure. Une pièce finie magnifique pourrait prendre une semaine ou plus et nécessiterait plus de compétence de la part de l'artisan.

La Meneuse est libre de refuser certaines tentatives de création, de construction ou de réparation, exigeant que le personnage ait un certain niveau de compétence, des outils et des matériaux appropriés, etc.

Un objet de niveau 0 ne nécessite aucune compétence pour être fabriqué et se trouve facilement dans la plupart des endroits. Les pierres de fronde et le bois de chauffage sont des objets de niveau 0 --- les produire est routinier. Fabriquer une torche à partir de bois de récupération et de tissu imbibé d'huile est simple, c'est donc un objet de niveau 1. Fabriquer une flèche ou une lance est assez standard mais pas simple, c'est donc un objet de niveau 2.

En règle générale, un dispositif à fabriquer nécessite des matériaux égaux à son niveau et à tous les niveaux inférieurs. Ainsi, un dispositif de niveau 5 nécessite un matériau de niveau 5, un matériau de niveau 4, un matériau de niveau 3, un matériau de niveau 2 et un matériau de niveau 1 (et, techniquement, un matériau de niveau 0).

La Meneuse et les joueurs peuvent passer rapidement sur de nombreux détails de fabrication, si souhaité. Rassembler tous les matériaux pour fabriquer un objet banal pourrait ne pas valoir la peine d'être joué --- mais cela pourrait l'être. Par exemple, fabriquer une lance en bois dans une forêt n'est pas très intéressant, mais que faire si les personnages doivent fabriquer une lance dans un désert sans arbres ? Trouver l'épave de quelque chose fait de bois ou forcer un PJ à fabriquer une lance à partir des os d'une grande bête pourrait être des situations intéressantes.

Le temps requis pour créer un objet dépend de la Meneuse, mais les directives du tableau d'artisanat sont un bon point de départ. En général, réparer un objet prend entre la moitié et la totalité du temps de création, selon l'objet, l'aspect nécessitant réparation et les circonstances. Par exemple, si la création d'un objet prend une heure, sa réparation prend de trente minutes à une heure.

Parfois, la Meneuse peut permettre un travail rapide si les circonstances le justifient. C'est différent de l'utilisation de compétences pour réduire le temps requis. Dans ce cas, la qualité de l'objet est affectée. Par exemple, un personnage doit créer un outil qui peut couper l'acier solide avec un laser (un objet de niveau 7), mais il doit le faire en un jour. La Meneuse pourrait l'autoriser, mais l'appareil pourrait être extrêmement volatil, infligeant des dégâts à l'utilisateur, ou ne fonctionner qu'une seule fois. L'appareil est toujours considéré comme un objet de niveau 7 à créer à tous autres égards. Parfois, la Meneuse peut décider qu'il est impossible de réduire le temps. Par exemple, un humain seul ne peut pas fabriquer un gilet en cotte de mailles en une heure sans une machine pour l'aider.

Les compétences d'artisanat possibles incluent :



* Fabrication d'armures
* Fabrication d'arcs/flèches
* Chimie
* Informatique
<--->
* Électronique
* Moteurs
* Génie génétique
* Soufflage de verre
* Armurerie
<--->
* Travail du cuir
* Travail des métaux
* Ingénierie neuronale
* Forge d'armes
* Travail du bois


Les personnages pourraient essayer de faire faire à un cypher, un artefact ou un vaisseau stellaire psionique alien autre chose que sa fonction prévue. Parfois, la Meneuse déclarera simplement la tâche impossible. Vous ne pouvez pas transformer une fiole d'élixir de guérison en communicateur bidirectionnel. Mais la plupart du temps, il y a une chance de succès.

Cela dit, bricoler des objets étranges n'est pas facile. Évidemment, la difficulté varie d'une situation à l'autre, mais des difficultés commençant à 7 ne sont pas déraisonnables. Le temps, les outils et la formation requis seraient similaires au temps, aux outils et à la formation nécessaires pour réparer un dispositif. Si le bricolage résulte en un avantage à long terme pour le personnage --- comme créer un artefact qu'il peut utiliser --- la Meneuse devrait exiger qu'il dépense des PX pour le faire.

\begin{tcolorbox}
Les circonstances comptent vraiment. Par exemple, coudre une robe à la main pourrait prendre cinq fois plus de temps (ou plus) qu'en utilisant une machine à coudre.
\end{tcolorbox}

\begin{tcolorbox}
La Meneuse est libre de refuser certaines tentatives de création, de construction ou de réparation, exigeant que le personnage ait un certain niveau de compétence, des outils et des matériaux appropriés, etc.
\end{tcolorbox}

\begin{tcolorbox}
Évidemment, ce qui est considéré comme "objets étranges" variera d'un cadre à l'autre, et parfois le concept pourrait ne pas s'appliquer du tout. Mais souvent, il y aura quelque chose dans le cadre qui est trop étrange, trop alien, trop puissant ou trop dangereux pour que les PJ puissent y toucher (ou au moins y toucher facilement). Einstein était peut-être extraordinaire, mais cela ne signifie pas qu'il aurait pu rétroconcevoir un téléporteur fabriqué dans une autre dimension.
\end{tcolorbox}

%%%%%%%%%%%%%%%%%%%%%%%%%%%%%%%%%%%%%%%%%%%%%%%%%%%%%%%%%%%%%%%%%%%%%%%%%%%%%%%%%%%%%%%%%
\begin{table*}[h]
    %\hspace*{-40pt}
    %\centering
        \begin{tabular}{ l l l }
        \multicolumn{2}{ l }{\Large \textcolor{CSCOLORPARTTWO}{Difficulté et Temps de Fabrication}} \\
        \textbf{Difficulté} & \textbf{Artisanat} & \textbf{Temps moyen de fabrication} \\ [0.5ex]
        \rowcolor{CSCOLORPARTTWO!50}
    0          & Quelque chose d'extrêmement simple comme attacher une corde ou trouver une pierre de taille appropriée       & Quelques minutes au plus \\
    \rowcolor{CSCOLORPARTTWO!20}
    1          & Torche                                                                                     & Cinq minutes \\
    \rowcolor{CSCOLORPARTTWO!50}
    2          & Lance, abri simple, meuble                                                                 & Une heure \\
    \rowcolor{CSCOLORPARTTWO!20}
    3          & Arc, porte, article de vêtement basique                                                      & Un jour \\
    \rowcolor{CSCOLORPARTTWO!50}
    4          & Épée, gilet en cotte de mailles                                                                     & Un à deux jours \\
    \rowcolor{CSCOLORPARTTWO!20}
    5          & Objet technologique courant (lumière électrique), beau bijou ou objet d'art                           & Une semaine \\
    \rowcolor{CSCOLORPARTTWO!50}
    6          & Objet technologique (montre, émetteur), très beau bijou ou objet d'art, travail d'artisanat élégant & Un mois \\
    \rowcolor{CSCOLORPARTTWO!20}
    7          & Objet technologique (ordinateur), œuvre d'art majeure                                          & Un an \\
    \rowcolor{CSCOLORPARTTWO!50}
    8          & Objet technologique (quelque chose d'extra-terrestre)                                          & Plusieurs années \\
    \rowcolor{CSCOLORPARTTWO!20}
    9          & Objet technologique (quelque chose d'extra-terrestre)                                          & Plusieurs années \\
    \rowcolor{CSCOLORPARTTWO!50}
    10         & Objet technologique (quelque chose d'extra-terrestre)                                          & Plusieurs années \\
        \end{tabular}
\label{craftingdifficulty}
    \end{table*}

\subsection*{Garder}

Dans une situation de combat, un personnage peut monter la garde comme action. Il ne fait pas d'attaques, mais toutes ses tâches de défense sont facilitées. De plus, si un PNJ essaie de passer ou de faire une action qu'il garde, le personnage peut tenter une action de Célérité facilitée basée sur le niveau du PNJ. Le succès signifie que le PNJ est empêché de faire l'action ; l'action du PNJ ce tour est gaspillée. Cela est utile pour bloquer une porte, protéger un ami, etc.

Si un PNJ monte la garde, utilisez la même procédure, mais pour passer le garde, le PJ tente une action de Célérité entravée contre le PNJ. Par exemple, Diana est un PNJ humain avec un garde du corps de niveau 3. Le garde du corps utilise son action pour protéger Diana. Si un PJ veut attaquer Diana, le PJ doit d'abord réussir une tâche de Célérité de difficulté 4 pour passer le garde. Si le PJ réussit, il peut faire son attaque normalement.

\subsection*{Soins}

Vous pouvez administrer des soins par des bandages et autres secours, en essayant de soigner chaque patient une fois par jour. Ces soins restaurent des points à une Réserve de stat de votre choix. Décidez combien de points vous voulez soigner, puis faites une action d'Intellect avec une difficulté égale à ce nombre. Par exemple, si vous voulez soigner quelqu'un de 3 points, c'est une tâche de difficulté 3 avec un nombre cible de 9.

\subsection*{Interaction avec les Créatures}

Le niveau de la créature détermine le nombre cible, comme pour le combat. Ainsi, soudoyer un garde fonctionne beaucoup comme le frapper ou l'affecter avec une capacité. Cela est vrai pour persuader quelqu'un, intimider quelqu'un, calmer une bête sauvage, ou toute autre interaction de ce type. L'interaction est une tâche d'Intellect. L'interaction nécessite généralement une langue commune ou un autre moyen de communiquer. Apprendre de nouvelles langues est comme apprendre une nouvelle compétence.

\subsection*{Sauter}

Décidez quelle distance vous voulez sauter, et cela fixe la difficulté de votre jet de Puissance. Pour un saut sur place, soustrayez 4 de la distance en pieds pour déterminer la difficulté du saut. Par exemple, sauter 10 pieds (3 m) a une difficulté de 6.

Si vous courez une distance immédiate avant de sauter, cela compte comme un atout, facilitant le saut.

Si vous courez une courte distance avant de sauter, divisez la distance du saut (en pieds) par 2 puis soustrayez 4 pour déterminer la difficulté du saut. Parce que vous courez une distance immédiate (et ensuite un peu plus), vous comptez également votre course comme un atout. Par exemple, sauter une distance de 20 pieds (6 m) avec un court élan a une difficulté de 5 (20 pieds divisé par 2 donne 10, moins 4 donne 6, moins 1 pour courir une distance immédiate).

Pour un saut vertical, la distance que vous franchissez (en pieds) est égale à la difficulté de la tâche de saut. Si vous courez une distance immédiate, cela compte comme un atout, facilitant le saut.

(Il n'y a rien de mal à ce que la Meneuse fixe simplement un niveau de difficulté pour un saut sans se soucier de la distance précise. Les règles ici sont juste pour que tout le monde ait quelques directives.)

\subsection*{Regarder ou Écouter}

En général, la Meneuse décrira tout ce qui est visible ou audible et qui n'est pas délibérément difficile à détecter. Mais si vous voulez chercher un ennemi caché, une porte secrète, ou écouter quelqu'un qui s'approche furtivement, faites un jet d'Intellect. Si c'est une créature, son niveau détermine la difficulté du jet. Si c'est autre chose, la Meneuse détermine la difficulté du jet.

\subsection*{Déplacer un Objet Lourd}

Vous pouvez pousser ou tirer quelque chose de très lourd et le déplacer d'une distance immédiate comme action.

Le poids de l'objet détermine la difficulté du jet de Puissance pour le déplacer ; chaque tranche de 50 livres (23 kg) entrave la tâche d'un niveau. Ainsi, déplacer quelque chose qui pèse 150 livres (68 kg) est de difficulté 3, et déplacer quelque chose qui pèse 400 livres (180 kg) est de difficulté 8. Si vous pouvez faciliter la tâche à 0, vous pouvez déplacer un objet lourd jusqu'à une courte distance comme action.

\subsection*{Utiliser ou Désactiver un Dispositif, ou Crocheter une Serrure}

Comme pour comprendre un dispositif, le niveau du dispositif détermine généralement la difficulté du jet d'Intellect. Sauf si un dispositif est très complexe, la Meneuse décidera souvent qu'une fois que vous l'avez compris, aucun jet n'est nécessaire pour l'utiliser sauf dans des circonstances spéciales. Donc, si les PJ comprennent comment utiliser un aéroglisseur, ils peuvent le piloter. S'ils sont attaqués, ils pourraient avoir besoin de faire un jet pour s'assurer qu'ils ne font pas s'écraser le véhicule contre un mur en essayant d'éviter d'être touchés.

Contrairement à l'utilisation d'un dispositif, désactiver un dispositif ou crocheter une serrure nécessite généralement des jets. Ces actions impliquent souvent des outils spéciaux et supposent que le personnage n'essaie pas de détruire le dispositif ou la serrure. (Un PJ qui essaie de le détruire devrait probablement faire un jet de Puissance pour le briser plutôt qu'un jet de Célérité ou d'Intellect nécessitant patience et savoir-faire.)

\subsection*{Monter ou Piloter}

Si vous montez un animal dressé comme monture, ou conduisez ou pilotez un véhicule, vous n'avez pas besoin de faire un jet pour faire quelque chose de routinier comme aller d'un point A à un point B (de la même manière que vous n'auriez pas besoin de faire un jet pour y marcher). Cependant, rester monté pendant un combat ou faire quelque chose de délicat avec un véhicule nécessite un jet de Célérité pour réussir. Une selle ou un autre équipement approprié est un atout et facilite la tâche.

\subsection*{Difficulté de Montage ou de Pilotage}

\begin{table*}[h]
    %\hspace*{-40pt}
    %\centering
        \begin{tabular}{ l l }
        \multicolumn{2}{ l }{\Large \textcolor{CSCOLORPARTTWO}{Difficulté de pilotage}} \\
        \textbf{Difficulté} & \textbf{Manœuvre} \\ [0.5ex]
        \rowcolor{CSCOLORPARTTWO!50}
    0          & Monter \\
    \rowcolor{CSCOLORPARTTWO!20}
    1          & Rester sur la monture (y compris une moto ou un véhicule similaire) dans un combat ou une autre situation difficile \\
    \rowcolor{CSCOLORPARTTWO!50}
    3          & Rester sur une monture (y compris une moto ou un véhicule similaire) lorsque vous subissez des dégâts \\
    \rowcolor{CSCOLORPARTTWO!20}
    4          & Monter un destrier en mouvement \\
    \rowcolor{CSCOLORPARTTWO!50}
    4          & Faire un virage brusque avec un véhicule en mouvement rapide \\
    \rowcolor{CSCOLORPARTTWO!20}
    4          & Faire avancer un véhicule deux fois plus vite que la normale pendant un tour \\
    \rowcolor{CSCOLORPARTTWO!50}
    5          & Pousser une monture à se déplacer ou sauter deux fois plus vite ou plus loin que la normale pendant un tour \\
    \rowcolor{CSCOLORPARTTWO!20}
    5          & Faire un long saut avec un véhicule non conçu pour voler (comme une voiture) et rester en contrôle \\
\end{tabular}
\label{table:ridingdifficulty}
\end{table*}

\subsection*{Se Faufiler}

La difficulté de se faufiler devant une créature est déterminée par son niveau. Se faufiler est un jet de Célérité. Se déplacer à demi-vitesse facilite la tâche de se faufiler. Un camouflage ou un autre équipement approprié peut compter comme un atout et faciliter la tâche, tout comme des conditions de faible éclairage et avoir beaucoup de choses derrière lesquelles se cacher.

\subsection*{Nager}

Si vous nagez simplement d'un endroit à un autre, comme traverser une rivière ou un lac calme, utilisez les règles de mouvement standard, en notant que votre personnage est en eau profonde. Cependant, parfois, des circonstances spéciales nécessitent un jet de Puissance pour progresser en nageant, comme essayer d'éviter un courant ou d'être entraîné dans un tourbillon.

\subsection*{Comprendre, Identifier ou Se Souvenir}

Lorsque les personnages essaient d'identifier ou de comprendre comment utiliser un dispositif, le niveau du dispositif détermine la difficulté. Pour un élément de connaissance, la Meneuse détermine la difficulté.

\begin{table*}[h]
    %\hspace*{-40pt}
    %\centering
        \begin{tabular}{ l l }
        \multicolumn{2}{ l }{\Large \textcolor{CSCOLORPARTTWO}{Difficulté de Connaissance}} \\
        \textbf{Difficulté} & \textbf{Connaissance} \\ [0.5ex]
        \rowcolor{CSCOLORPARTTWO!50}
    0          & Connaissance commune \\
    \rowcolor{CSCOLORPARTTWO!20}
    1          & Connaissance simple \\
    \rowcolor{CSCOLORPARTTWO!50}
    3          & Quelque chose qu'un érudit connaît probablement \\
    \rowcolor{CSCOLORPARTTWO!20}
    5          & Quelque chose qu'un érudit pourrait ne pas connaître \\
    \rowcolor{CSCOLORPARTTWO!50}
    7          & Connaissance que très peu de personnes possèdent \\
    \rowcolor{CSCOLORPARTTWO!20}
    10         & Connaissance complètement perdue \\
\end{tabular}
\label{table:understanding}
\end{table*}

\subsection*{Mouvement Véhiculaire}

Les véhicules se déplacent comme les créatures. Chacun a une vitesse de déplacement, qui indique jusqu'où il peut se déplacer en un tour. La plupart des véhicules nécessitent un conducteur, et en mouvement, ils nécessitent généralement que le conducteur dépense chaque action pour contrôler le mouvement. C'est une tâche routinière qui nécessite rarement un jet. Tout tour non passé à conduire le véhicule entrave la tâche au tour suivant et empêche tout changement de vitesse ou de direction. En d'autres termes, conduire sur la route normalement est de difficulté 0. Dépensez une action pour récupérer un sac à dos sur le siège arrière signifie qu'au tour suivant, le conducteur doit tenter une tâche de difficulté 1. S'il utilise son action pour sortir un pistolet du sac à dos, au tour suivant, la difficulté pour conduire sera de 2, et ainsi de suite. Les échecs résultent de la situation mais pourraient impliquer une collision ou quelque chose de similaire.

Dans une poursuite en véhicule, les conducteurs tentent des actions de Célérité comme dans une poursuite normale, mais la tâche peut être basée soit sur le niveau du conducteur (modifié par le niveau et la vitesse du véhicule), soit sur le niveau du véhicule (modifié par le niveau du conducteur). Donc, si un PJ conduisant une voiture typique poursuit un PNJ de niveau 3 conduisant une voiture de sport de niveau 5, le PJ ferait trois jets de poursuite avec une difficulté de 5. Si la voiture du PJ est un véhicule personnalisé amélioré, cela pourrait accorder au PJ un atout dans la poursuite. Si le PJ n'est pas du tout en voiture, mais à vélo, cela pourrait entraver les jets de poursuite de deux ou trois niveaux, ou la Meneuse pourrait simplement décider que c'est impossible.

\subsection*{Combat Véhiculaire}

La plupart du temps, un combat entre ennemis en voitures, bateaux ou autres véhicules est comme n'importe quelle autre situation de combat. Les combattants ont probablement une couverture et se déplacent rapidement. Les attaques pour désactiver un véhicule ou une partie de celui-ci sont basées sur le niveau du véhicule. Si le véhicule est une voiture blindée ou un char, toutes les attaques sont probablement dirigées contre le véhicule, qui a un niveau et probablement une cote d'Armure appropriée, un peu comme une créature.

La seule fois où ce n'est pas vrai, c'est avec des batailles où seuls des véhicules et non des personnages sont impliqués. Ainsi, si les PJ sont dans une fusillade avec des braqueurs de banque et les deux groupes sont en voiture, utilisez les règles standard. Cependant, les batailles entre vaisseaux spatiaux de différents types --- des énormes vaisseaux capitaux aux chasseurs monoplaces --- sont fréquentes dans les cadres de science-fiction futuriste. Un combat sous-marin entre deux engins des grandes profondeurs pourrait être très excitant. Les personnages dans un jeu contemporain pourraient se retrouver dans un combat de chars. Si les PJ sont impliqués dans un combat où ils sont entièrement enfermés dans des véhicules (de sorte que ce ne sont pas vraiment les personnages qui combattent, mais les véhicules), utilisez les directives suivantes, simples et faciles.

À cette échelle, le combat entre véhicules n'est pas comme le combat traditionnel. Ne vous souciez pas de la santé, de l'Armure, ou quoi que ce soit de ce genre. Comparez plutôt les niveaux des véhicules impliqués. Si le véhicule des PJ a un niveau plus élevé, la différence de niveaux est le nombre de niveaux dont les jets d'attaque et de défense des PJ sont facilités. Si le véhicule des PJ a un niveau inférieur, leurs jets sont entravés. Si les niveaux sont les mêmes, il n'y a pas de modification.

Ces jets d'attaque et de défense sont modifiés par la compétence et l'Effort, comme d'habitude. Certains véhicules ont également des armes supérieures, ce qui facilite l'attaque (puisqu'il n'y a pas de quantité de "dégâts" à prendre en compte), mais cette circonstance est probablement rare dans ce système abstrait et ne devrait pas affecter la difficulté de plus d'un ou peut-être deux niveaux. De plus, si deux véhicules coordonnent leur attaque contre un seul véhicule, l'attaque est facilitée. Si trois véhicules ou plus coordonnent, l'attaque est facilitée de deux niveaux.

L'attaquant doit essayer de cibler un système spécifique ou une partie d'un véhicule ennemi. Cela entrave l'attaque en fonction du système ou de la partie ciblée.

Cela fait beaucoup de modifications. Mais ce n'est pas vraiment si difficile. Prenons un exemple de bataille spatiale. Un PJ dans un petit chasseur de niveau 2 attaque une frégate de niveau 4. Puisque la frégate est de niveau 4, la difficulté de l'attaque commence à 4. Mais le vaisseau attaquant est plus faible que le défenseur, donc l'attaque est entravée d'un nombre égal à la différence de leurs niveaux (2). Le pilote du chasseur doit faire une attaque de difficulté 6 sur la frégate. Cependant, le chasseur essaie de plonger et d'endommager la propulsion de la frégate, ce qui entrave l'attaque de trois niveaux supplémentaires, pour une difficulté totale de 9. Si le pilote du chasseur est formé au combat spatial, il réduit la difficulté à 8, mais c'est toujours impossible sans aide. Donc, disons que deux autres PJ --- également dans des chasseurs de niveau 2 --- se joignent et coordonnent leur attaque. Trois vaisseaux coordonnant une attaque sur une seule cible facilitent la tâche de deux niveaux, résultant en une difficulté finale de 6. Le PJ attaquant serait sage d'utiliser de l'Effort.

Ensuite, la frégate riposte, et le PJ doit faire un jet de défense. La différence de niveau entre les vaisseaux (2) signifie que la défense du PJ est entravée de deux niveaux, donc la difficulté du jet de défense du PJ commence à 6. Mais la frégate essaie de détruire les armes du chasseur, entraver leur attaque (facilitant la défense du PJ) de deux niveaux. Ainsi, le PJ doit réussir une tâche de difficulté 4 ou perdre ses systèmes d'armes principaux.

Il est important de se rappeler qu'une attaque ratée ne signifie pas toujours un échec. Le vaisseau cible pourrait vaciller sous l'impact, mais la majeure partie des dégâts aurait été absorbée par les boucliers, donc il n'y aurait pas de dégâts significatifs.

Ce système simplifié devrait permettre à la Meneuse et aux joueurs de créer des rencontres excitantes impliquant tout le groupe. Par exemple, peut-être qu'un PJ pilote un vaisseau, un autre manie les canons, et un autre répare frénétiquement les dégâts aux propulseurs de manœuvre avant qu'ils ne s'écrasent dans la station spatiale qu'ils essaient de défendre.

\begin{tcolorbox}
Pendant une bataille de véhicules, en particulier une bataille spatiale, il y a beaucoup de discussions sur les boucliers qui tombent en panne, l'intégrité de la coque, le fait d'être dépassé, d'arriver trop vite, et autres. Ces détails sont géniaux, mais ils sont tous de la saveur, donc ils sont représentés dans les règles de manière générale, plutôt que spécifique.
\end{tcolorbox}

\begin{tcolorbox}
Pour plus de détails sur les véhicules, consultez le chapitre sur les genres.
\end{tcolorbox}

\begin{tcolorbox}
Être formé à la conduite permet au personnage d'avoir la pratique d'utilisation d'un véhicule comme une arme. Si le véhicule est utilisé pour écraser une victime ou percuter un véhicule ennemi, traitez une moto comme une arme moyenne et une voiture ou un camion comme une arme lourde.
\end{tcolorbox}

\begin{table*}[h]
    %\hspace*{-40pt}
    %\centering
        \begin{tabular}{ l l l }
    \textbf{Objectif du Ciblage}      & \textbf{Attaque Entravée} & \textbf{Effet} \\
    \rowcolor{CSCOLORPARTTWO!50}
    Désactiver les armes             & Deux niveaux       & Une ou plusieurs des armes du véhicule ne fonctionnent plus \\
    \rowcolor{CSCOLORPARTTWO!20}
    Désactiver les défenses (si applicable) & Deux niveaux       & Les attaques contre le véhicule sont facilitées \\
    \rowcolor{CSCOLORPARTTWO!50}
    Désactiver le moteur/propulsion & Trois niveaux     & Le véhicule ne peut plus se déplacer, ou son mouvement est entravé \\
    \rowcolor{CSCOLORPARTTWO!20}
    Désactiver la maniabilité          & Deux niveaux       & Le véhicule ne peut plus changer de cap \\
    \rowcolor{CSCOLORPARTTWO!50}
    Frapper le cœur de puissance ou un point vital  & Cinq niveaux      & Le véhicule est complètement détruit \\
\end{tabular}
\label{table:understanding}
\end{table*}

\section*{suivants}

Les personnages-joueurs ont la possibilité de gagner des suivants à mesure qu'ils avancent en rang, comme le permettent les capacités spéciales de type ou de focalisation. Les suivants n'ont pas besoin d'être payés, nourris ou logés, bien qu'un personnage qui gagne des suivants puisse bien sûr faire de tels arrangements s'il le souhaite. Un suivant est quelqu'un que le personnage a inspiré (ou demandé) à venir travailler avec lui pendant un certain temps, l'aidant dans diverses entreprises. Un suivant place les intérêts du PJ avant les siens, ou au moins au même niveau.

Le PJ fait généralement les jets pour son suivant lorsque le suivant entreprend des actions, bien que généralement les modifications d'un suivant fournissent un atout à une action spécifique entreprise par le PJ qu'il suit.

\begin{tcolorbox}
Si un suivant meurt, le personnage obtient un nouveau après au moins deux semaines et un recrutement approprié.
\end{tcolorbox}

\textbf{Modifications :} Un suivant peut aider un PJ dans une ou plusieurs tâches, accordant au PJ un atout pour cette tâche. Le niveau du suivant indique le nombre de tâches différentes qu'il peut aider. Les tâches que le suivant est capable d'aider sont prédéterminées, généralement choisies par le PJ lorsqu'il gagne le suivant. Un suivant de niveau 2 qui, selon le joueur, est un espion pourrait accorder au PJ un atout sur deux tâches différentes, comme la discrétion et la tromperie. Les suivants ne peuvent pas aider avec des tâches pour lesquelles ils n'ont pas de modifications ; pour les besoins de l'aide, traitez le suivant comme s'il avait des incapacités dans toutes les tâches non modifiées.

Lorsque le suivant agit de manière autonome plutôt que d'aider le PJ, il agit comme un PNJ normal qui a des modifications. Ainsi, la modification augmente son niveau effectif pour la tâche associée d'un niveau. Par exemple, le suivant espion de niveau 2 avec des modifications pour la discrétion et la tromperie tente les tâches de discrétion et de tromperie comme s'il était de niveau 3 et toutes les autres tâches comme s'il était de niveau 2.

\textbf{Atouts des suivants pour le Combat et la Défense :} Un suivant ne peut pas accorder d'atout aux attaques ou à la défense d'un personnage tant que le suivant n'est pas de niveau 3 ou plus. Même alors, le suivant ne peut aider avec les attaques et la défense que s'il a une modification pour ce type de tâche.

Certaines capacités peuvent accorder une exception spéciale à cette règle. Par exemple, la capacité Serv-0 Défenseur donne à votre suivant Serv-0 de niveau 1 (un compagnon machine) une modification pour la défense en Célérité.

\textbf{Progression de Niveau des suivants :} Un suivant augmente de niveau de 1 chaque fois qu'un PJ avance de deux rangs après avoir obtenu ce suivant. Lorsque le suivant gagne un niveau, le PJ choisit également la tâche pour laquelle le suivant gagne une modification.

\textbf{suivant Exceptionnel :} Lorsqu'un personnage gagne un suivant, il y a une petite chance que le suivant soit exceptionnel d'une manière ou d'une autre, un cran au-dessus des autres suivants de son genre. La Meneuse détermine quand un suivant exceptionnel est trouvé, peut-être comme récompense supplémentaire pour un jeu de rôle intelligent ou engageant où les PJ impressionnent ou interagissent positivement avec un ou plusieurs PNJ, dont certains pourraient devenir plus tard l'un de leurs suivants. Un suivant exceptionnel a les mêmes qualités qu'un suivant normal mais est d'un niveau supérieur.

\textbf{Animal de Compagnie :} Tout PJ peut potentiellement gagner un animal de compagnie, bien qu'un animal de compagnie n'accorde généralement pas de modifications. Si un personnage veut un animal de compagnie qui peut le faire, il doit obtenir l'animal par le biais d'une capacité ou d'une focalisation qui accorde des suivants. D'un autre côté, un animal de compagnie bien soigné accorde un atout aux tâches du PJ liées à l'obtention de la paix de l'esprit, au confort et à la résistance à la solitude.

\section*{Donner Vie aux suivants}

Les modifications fournies par les suivants pourraient sembler assez sèches et mécaniques. Pour éviter cela, vous pourriez présenter chaque suivant de manière à le rendre plus captivant et intéressant. Voici quelques exemples de descriptions de suivants, selon leur combinaison de modifications.

\begin{itemize}
    \item Un diplomate enflammé par son discours capable de convaincre une horde ennemie de battre en retraite.
    \item Un commandant vétéran dont la présence renforce toute la puissance militaire de la communauté.
    \item Un médecin de génie qui impressionne tout le monde avec ses techniques de guérison.
    \item Un architecte imaginatif dont les œuvres embellissent et défendent la ville.
    \item Un espion rusé dont les renseignements sur les mouvements ennemis sont inestimables.
\end{itemize}
